\chapter{Formatting Code}

In this chapter, we will show how to use the \texttt{listings} package to format source code in LaTeX. This package allows you to highlight the syntax of various programming languages and customize the style to fit your document.

\section{Inserting Code}

After defining the style, you can use it to format code within the document. To do this, use the \texttt{\textbackslash lstset} command to set the style and the \texttt{\textbackslash lstinputlisting} or \texttt{\textbackslash begin{lstlisting}...\textbackslash end{lstlisting}} command to include the code. Here is an example:

\subsection{Example with External File} You can also include an external file containing the code:

\lstinputlisting[language=Python, style=hkn]{res/py/2d_example.py}

\section{Formatting the Code}

The \texttt{listings} package allows you to display source code within a LaTeX document while preserving formatting and syntax highlighting. In this section, we will explore how to use \texttt{listings} to include and format code in our document.

\subsection{Predefined Styles}

\texttt{listings} provides some predefined styles for syntax highlighting of various programming languages. These styles allow you to easily include code blocks with correct syntax highlighting.

For example, to include Python code in the document, you can use the following command:

\begin{lstlisting}
    \begin{lstlisting}[language=Python]
        def hello_world():
        print("Hello, World!")
  \end {lstlisting}
\end{lstlisting}

Result:
\begin{lstlisting}[language=Python]
    def hello_world():
    print("Hello, World!")
\end{lstlisting}

In this example, the \texttt{listings} package automatically recognizes Python syntax and highlights it correctly.

\subsection{Supported Programming Languages}

\texttt{listings} supports many programming languages, so you can use automatic syntax highlighting for a wide range of languages. Just specify the language using the \texttt{language} option within the \texttt{lstlisting} environment. Some supported languages include:

\begin{itemize}
    \item Python: \texttt{language=Python}
    \item C/C++: \texttt{language=C}
    \item Java: \texttt{language=Java}
    \item SQL: \texttt{language=SQL}
    \item HTML: \texttt{language=HTML}
    \item CSS: \texttt{language=CSS}
    \item JavaScript: \texttt{language=JavaScript}
    \item XML: \texttt{language=XML}
\end{itemize}

Example of SQL code:

\begin{lstlisting}[language=SQL]
    SELECT * FROM users WHERE age > 18;
\end{lstlisting}

\subsection{Customizing Styles}

Although the package offers predefined styles, it is possible to customize the appearance of the code. If there is a need to create new styles, contact the responsible person. Some customizable parameters, listed here to highlight the potential of the styles, include:

\begin{itemize}
    \item \texttt{language}: Defines the language of the code.
    \item \texttt{basicstyle}: Sets the base font for the code.
    \item \texttt{keywordstyle}: Modifies the style of keywords.
    \item \texttt{stringstyle}: Modifies the style of strings.
    \item \texttt{commentstyle}: Defines the style for comments.
    \item \texttt{numbers}: Displays line numbers.
    \item \texttt{frame}: Adds a border around the code.
    \item \texttt{backgroundcolor}: Sets the background color of the code.
\end{itemize}

This is the before and after applying the SQL style defined in HKNtools.tex:

\begin{lstlisting}[language=SQL]
    -- Query to select user data
    SELECT name, age
    FROM users
    WHERE age > 18
    ORDER BY age DESC;
\end{lstlisting}

\begin{lstlisting}[language=SQL, style=hkn]
    -- Query to select user data
    SELECT name, age
    FROM users
    WHERE age > 18
    ORDER BY age DESC;
\end{lstlisting}

\subsection{Global Configuration}

You can also configure \texttt{listings} globally to apply the same style settings throughout the document using the \texttt{\\lstset} command. For example:

\begin{lstlisting}
\lstset{
    language=Python,
    numbers=left,
    frame=single,
    backgroundcolor=\color{lightgray},
    stepnumber=1
}
\end{lstlisting}

This will set the options for all Python code blocks in the document.
Here are the effects before and after using the instruction:

\begin{lstlisting}[language=Python]
    def hello_world():
    print("Hello, World!")
\end{lstlisting}

\lstset{
    language=Python,
    numbers=left,
    frame=single,
    backgroundcolor=\color{lightgray},
    stepnumber=1
}

\begin{lstlisting}[language=Python]
    def hello_world():
    print("Hello, World!")
\end{lstlisting}

\section{Currently Defined Custom Styles}

Additional styles will be implemented as needed by collaborators. For now, the following styles are currently defined:

\begin{itemize}
    \item SQL: sqlstyle
\end{itemize}

\section{Complete Example}

Below is an example of a C program highlighted with the \texttt{listings} package. This program prints a greeting on the screen. As you can see, I used a \texttt{label} to easily reference the code later.

\begin{lstlisting}[language=C, style=hkn, caption={Demo}, label=code:demo]
#include <stdio.h>

int x = 10;
uint32_t y = 10;

// Example program
int main() {
  printf("Hello, World!\n");
  return 0;
}
\end{lstlisting}

This example is labeled with \texttt{label=code:demo}, so we can refer to this code anywhere in the document using the command \texttt{\ref{code:demo}}. For example, we can reference the code \ref{code:demo} in the text.
