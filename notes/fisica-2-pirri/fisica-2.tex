% !TEX root = fisica-2.tex

% Possible arguments for the HKNdocument class:
% Language:
%   italian (default): Sets the document language to Italian.
%   english: Sets the document language to English.
%
% Table of Contents (ToC) depth:
%   toc=chapters: Shows only chapters in the ToC (tocdepth=0).
%   toc=sections: Shows chapters and sections in the ToC (tocdepth=1).
%   toc=subsections: Shows up to subsections in the ToC (default, tocdepth=2).
%   toc=subsubsections: Shows up to sub-subsections in the ToC (tocdepth=3).
%
% Font size:
%   10pt: Sets the base font size to 10 points.
%   11pt (default): Sets the base font size to 11 points.
%   12pt: Sets the base font size to 12 points.
%
% Draft mode:
%   draft: Compiles the document in draft mode (useful for proofreading, dummy images and imported files).
\documentclass[italian,12pt,toc=sections]{HKNdocument}
% Packages
\usepackage{listings}                    % Code highlighting
\usepackage{xcolor}                      % Custom colors
\usepackage{longtable}                   % Breakable tables
\usepackage{ulem}                        % Underline
\usepackage{contour}                     % Border around text
\usepackage{tcolorbox}                   % Custom boxes

% Primary (Accent) Colors
% Primary (Accent) Colors
\definecolor{accentYellow}{RGB}{254, 196, 41}  % #FEC421
\definecolor{accentRed}{RGB}{236, 45, 36}      % #EC2D24

% Secondary Colors
\definecolor{supportOrange}{RGB}{242, 183, 5}  % #F2B705
\definecolor{supportDarkBlue}{RGB}{55, 81, 113} % #375171

% Background Colors
\definecolor{backgroundLight}{RGB}{242, 242, 242} % #F2F2F2

% Text & Border Colors
\definecolor{textGrayBlue}{RGB}{100, 117, 140}   % #64758C
\definecolor{textGrayMedium}{RGB}{146, 154, 166}  % #929AA6
\definecolor{textGrayLight}{RGB}{184, 187, 191}   % #B8BBBF



% Listings style
\lstdefinestyle{hkn}{
  basicstyle=\ttfamily\small\color{textGrayBlue},                         % Base style (size and font)
  keywordstyle=\bfseries\color{accentRed},           % Keywords in red (important, eye-catching)
  identifierstyle=\color{supportDarkBlue},               % Identifiers in blue (clear distinction)
  commentstyle=\color{textGrayMedium},                 % Comments in gray-blue (less prominent)
  stringstyle=\color{supportOrange},                 % Strings in orange (warm and readable)
  numberstyle=\ttfamily\scriptsize\color{textGrayMedium}, % Line numbers in gray (non-intrusive)
  backgroundcolor=\color{backgroundLight},           % Light background for contrast
  rulecolor=\color{textGrayLight},                   % Soft gray border for structure
  frame=single,                          % Border around code (single, double, shadowbox, none)
  framerule=0.8pt,                       % Border thickness
  frameround=tttt,                       % Round all corners
  framesep=5pt,                          % Distance between border and code
  rulesep=2pt,                           % Distance between border and code line
  numbers=left,                          % Line number position (left, right, none)
  stepnumber=1,                          % Line number interval
  numbersep=10pt,                        % Distance between line numbers and code
  xleftmargin=30pt,                      % Left margin
  xrightmargin=30pt,                     % Right margin
  resetmargins=true,                     % Reset margins
  numberblanklines=false,                % Number blank lines
  firstnumber=auto,                      % Initial line number
  columns=fixed,                         % Fixed column width
  showstringspaces=false,                % Show spaces in strings
  tabsize=2,                             % Tab size
  breaklines=true,                       % Automatic line break for long lines
  breakatwhitespace=true,                % Line break at whitespace
  breakautoindent=true,                  % Automatic indentation after line break
  escapeinside={(*@}{@*)}                % LaTeX commands in code
}

% Underline settings
\renewcommand{\ULdepth}{1.8pt} % Underline depth
\contourlength{0.8pt}

% Custom underline command
\newcommand{\myuline}[1]{%
\uline{\phantom{#1}}%
\llap{\contour{white}{#1}}%
}

% tcolorbox color settings
\definecolor{tcolorboxLeftColor}{RGB}{2, 65, 191}
\definecolor{tcolorboxBackTitleColor}{RGB}{119, 152, 255}
\definecolor{tcolorboxBackColor}{RGB}{210, 226, 255}

% Custom boxes
\newtcolorbox[auto counter, number within=chapter]{definition}[1]{
  title={\iflanguage{italian}{Definizione}{Definition}\par~\arabic{\tcbcounter}.~#1},
  boxrule=0mm,                       % Bordo principale (disabilitato)
  leftrule=1mm,                    % Bordo sinistro principale
  arc=2mm,
  colframe=accentRed,       % Colore bordo
  colbacktitle=textGrayMedium,
  colback=backgroundLight,        % Colore sfondo
  fonttitle=\bfseries,
  rounded corners=all,               % Bordi arrotondati
  }

\newtcolorbox[auto counter, number within=chapter]{theorem}[1]{
  title={\iflanguage{italian}{Teorema}{Theorem}~\arabic{\tcbcounter}.~#1},
  boxrule=0mm,                       % Bordo principale (disabilitato)
  leftrule=1mm,                    % Bordo sinistro principale
  arc=2mm,
  colframe=accentYellow,       % Colore bordo
  colbacktitle=textGrayMedium,
  colback=backgroundLight,        % Colore sfondo
  fonttitle=\bfseries,
  rounded corners=all,               % Bordi arrotondati
}

\newtcolorbox[auto counter, number within=chapter]{corollary}[1]{
  title={\iflanguage{italian}{Corollario}{Corollary}~\arabic{\tcbcounter}.~#1},
  boxrule=0mm,                       % Bordo principale (disabilitato)
  leftrule=1mm,                    % Bordo sinistro principale
  arc=2mm,
  colframe=supportOrange,       % Colore bordo
  colbacktitle=textGrayMedium,
  colback=backgroundLight,        % Colore sfondo
  fonttitle=\bfseries,
  rounded corners=all,               % Bordi arrotondati
}

\newtcolorbox[auto counter, number within=chapter]{exercise}[1]{
  title={\iflanguage{italian}{Esercizio}{Exercise}~\arabic{\tcbcounter}.~#1},
  boxrule=0mm,                       % Bordo principale (disabilitato)
  leftrule=1mm,                    % Bordo sinistro principale
  arc=2mm,
  colframe=supportDarkBlue,       % Colore bordo
  colbacktitle=textGrayMedium,
  colback=backgroundLight,        % Colore sfondo
  fonttitle=\bfseries,
  rounded corners=all,               % Bordi arrotondati
}

\newtcolorbox[auto counter, number within=chapter]{observation}[1]{
  title={\iflanguage{italian}{Osservazione}{Observation}~\arabic{\tcbcounter}.~#1},
  boxrule=0mm,                       % Bordo principale (disabilitato)
  leftrule=1mm,                    % Bordo sinistro principale
  arc=2mm,
  colframe=textGrayBlue,       % Colore bordo
  colbacktitle=textGrayMedium,
  colback=backgroundLight,        % Colore sfondo
  fonttitle=\bfseries,
  rounded corners=all,               % Bordi arrotondati
  }


\usepackage{parskip}

% Math
\usepackage{cancel}  % \cancel
\usepackage{mleftright}
\mleftright  % remove extra space before and after \left and \right delimiters
\usepackage[version=4]{mhchem}  % \ce
\usepackage{siunitx}
\sisetup{
    exponent-product = \cdot,
    input-digits = 0123456789\pi\pi^2,
    per-mode = single-symbol,
    range-open-phrase = {\text{da} },
}
\DeclareSIUnit\angstrom{\text{\AA}}

% Hyperlinks
\usepackage[capitalise]{cleveref}
\crefname{equation}{eq.\!}{eq.\!}
\Crefname{equation}{L'equazione}{Le equazioni}
\crefname{pluralequation}{eqs.\!}{eqs.\!}
\Crefname{pluralequation}{Le equazioni}{Le equazioni}
\newcommand{\shorturl}[1]{\href{https://#1}{\texttt{#1}}}

% Images
\usepackage{float}  % [H] option for figure and table placing
\newcommand{\addfigure}[3][]{
    \begin{figure}[H]  % !h or H
        \centering
        \includegraphics[width=#3\textwidth]{res/#2}
        \caption{#1}
        \label{fig:#2}
    \end{figure}
}
\newcommand{\addsvg}[3][]{
    \begin{figure}[H]  % !h or H
        \centering
        \includesvg[width=#3\textwidth]{res/svg/#2.svg}
        \caption{#1}
        \label{fig:#2}
    \end{figure}
}
\newcommand{\adddrawio}[3][]{
    \begin{figure}[H]  % !h or H
        \centering
        \includesvg[width=#3\textwidth,inkscapelatex=true]{res/drawio/#2.drawio.svg}
        \caption{#1}
        \label{fig:#2}
    \end{figure}
}
\usepackage{caption}
\captionsetup{margin=1cm}

%%% Command definitions
\usepackage{xparse}  % \NewDocumentCommand

% Classes
\newcommand{\vt}[1]{\mathbf{#1}}  % vectors
\newcommand{\mat}[1]{\mathrm{#1}}  % matrices
\newcommand{\uvt}[1]{\hat{\vt{#1}}}  % unit vectors
\newcommand{\ver}[1]{\uvt{u}_{#1}}  % unit vectors
\newcommand{\func}[1]{\operatorname{#1}}  % functions

% Functions and operations
\newcommand{\pts}[1]{\left( #1 \right)}  % parentheses
% \renewcommand{\exp}[1]{e^{#1}}  % exponential
\renewcommand{\exp}[1]{\func{exp}\pts{#1}}  % exponential
\newcommand{\norm}[1]{\left\lVert #1 \right\rVert}  % norm
\newcommand{\abs}[1]{\left\lvert #1 \right\rvert}  % absolute value
\newcommand{\avg}[1]{\left\langle #1 \right\rangle}  % average
\newcommand{\inner}[2]{\left\langle #1, #2 \right\rangle}  % inner product
\newcommand{\set}[1]{\left\{ #1 \right\}}  % set
\newcommand{\evalat}[2]{\left. #1 \right\rvert_{#2}}  % evaluation at
\newcommand{\dimension}{\func{dim}}  % physical dimension
\newcommand{\sgn}[1]{\func{sgn}(#1)}  % sign function
\newcommand{\flux}[2]{\Phi_{#1}\pts{#2}}  % flux

% Complex numbers
\newcommand{\im}{i}  % imaginary unit
\newcommand{\real}[1]{\func{Re}\pts{#1}}  % real part
\newcommand{\imag}[1]{\func{Im}\pts{#1}}  % imaginary part
\newcommand{\conj}[1]{#1^*}  % complex conjugate
\newcommand{\herm}[1]{#1^\dagger}  % conjugate transpose

% General entities
\newcommand{\p}{\vt{r}}  % generic point
\newcommand{\N}{\mathbb{N}}  % set of natural numbers
\newcommand{\Z}{\mathbb{Z}}  % set of integer numbers
\newcommand{\R}{\mathbb{R}}  % set of real numbers
\newcommand{\C}{\mathbb{C}}  % set of complex numbers
\newcommand{\ux}{\ver{x}}  % x unit vector
\newcommand{\uy}{\ver{y}}  % y unit vector
\newcommand{\uz}{\ver{z}}  % z unit vector
% \newcommand{\ux}{\hat{\text{\bfseries\i}}}
% \newcommand{\uy}{\hat{\text{\bfseries\j}}}
% \newcommand{\uz}{\hat{\vt{k}}}

% Calculus
\newcommand{\de}{\mathrm{d}}  % symbol of differential
\newcommand{\lde}{\cdot \de \vt{l}}  % line integral differential
\newcommand{\sde}{\cdot \de \vt{S}}  % flux differential
\NewDocumentCommand{\der}{O{} O{} m}{  % total derivative
    \mathchoice
        {\frac{\de^{#2} #1}{\de #3^{#2}}}  % displaystyle
        {\frac{\de^{#2}}{\de #3^{#2}} #1}  % textstyle
        {\frac{\de^{#2}}{\de #3^{#2}} #1}  % scriptstyle
        {\frac{\de^{#2}}{\de #3^{#2}} #1}  % scriptscriptstyle
}
\NewDocumentCommand{\parder}{O{} O{} m}{  % partial derivative
    \mathchoice
        {\frac{\partial^{#2} #1}{\partial #3^{#2}}}  % displaystyle
        {\partial^{#2}_{#3} #1}  % textstyle
        {\partial^{#2}_{#3} #1}  % scriptstyle
        {\partial^{#2}_{#3} #1}  % scriptscriptstyle
}
\newcommand{\grad}{\nabla}  % gradient
\newcommand{\diver}{\nabla \cdot}  % divergence
\newcommand{\curl}{\nabla \times}  % curl
\newcommand{\lapl}{\nabla^2}  % Laplacian
\newcommand{\vlapl}{\boldsymbol{\nabla}^2}  % vector Laplacian

% Physical entities
\newcommand{\eps}{\varepsilon}  % permittivity
\newcommand{\E}{\vt{E}}  % electric field
\newcommand{\B}{\vt{B}}  % magnetic flux density
\newcommand{\D}{\vt{D}}  % displacement field
\renewcommand{\H}{\vt{H}}  % magnetizing field
\newcommand{\dcurr}{\vt{J}}  % current density
\newcommand{\spot}{V}  % scalar potential
\newcommand{\vpot}{\vt{A}}  % vector potential
\newcommand{\force}{\vt{F}}  % force
\newcommand{\fem}{\mathcal{E}}  % electromotive force
\newcommand{\poy}{\vt{S}}  % Poynting vector
\newcommand{\energy}{{E_n}}  % energy
% Subscripts
\newcommand{\free}{_f}  % free
\newcommand{\bound}{_b}  % bound
\newcommand{\eff}{_\mathrm{eff}}  % effective quantity
\newcommand{\EM}{_\mathrm{EM}}  % 'electromagnetic' subscript

% Bra-ket notation
\newcommand{\ket}[1]{\left\lvert #1 \right\rangle}  % ket
\newcommand{\bra}[1]{\left\langle #1 \right\rvert}  % bra
\makeatletter
\newcommand{\braket}[3][]{  % braket
    \ifx\relax#1\relax
        \left\langle #2 \middle\vert #3 \right\rangle
    \else
        \left\langle #2 \middle\vert #1 \middle\vert #3 \right\rangle
    \fi
}
\makeatother

% Formatting
\newcommand{\redtext}[1]{{\color{red} #1}}
\newcommand{\important}[1]{\textbf{#1}}

\begin{document}
% Document metadata
\title{Appunti di Fisica II per Ingegneria Fisica}
\shorttitle{Appunti di Fisica II}
% You can include multiple \autor commands to list all autors in the frontpage.
\author{Giulio Cosentino}


\docdate{Primo semestre 2024/2025}
\docversion{1.1}

\frontmatter
% Create the title page
\maketitle
\cclicense
\tableofcontents
\clearpage

\mainmatter
\chapter{Elettrostatica}

\section{Principi dell'elettrostatica}

L'elettrostatica nasce dalla constatazione che esiste una proprietà della materia diversa dalla massa che si manifesta attraverso forze attrattive e repulsive.
Questa proprietà è detta \important{carica elettrica}, posseduta da un corpo materiale come è posseduta la massa, e ha un segno.

\subsection{Legge di Coulomb}

La forza elettrica è evidentemente più forte di quella gravitazionale e per quantificarla devo raggiungere il limite della carica puntiforme: sferette cariche poste a grande distanza (come fossero masse puntiformi).

Così facendo, si scopre la \important{legge di Coulomb}: due cariche $q_1$ e $q_2$ poste a distanza $r$ esercitano l'una sull'altra una forza pari a
\begin{equation}
    \force = k_e \frac{q_1 q_2}{r^2} \ver{r}
\end{equation}
dove $k_e$ è una costante di proporzionalità e di conversione tra unità di misura e $\ver{r}$ è il vettore unitario che punta dalla carica che esercita la forza a quella che la sente.
Se la forza è attrattiva, allora le cariche sono di segno opposto (cioè $q_1 q_2 < 0$).

La legge di Coulomb permette di definire la grandezza fisica ``carica elettrica'' e la sua unità di misura.
Questo è possibile poiché nell'equazione figurano solo grandezze fisiche già definite, come la forza e la distanza.

Si fissa il valore numerico di $k_e$ a $\{k_e\} = 10^{-7} (c / (\qty{1}{\metre\per\second}))^2$ (dove $c$ è la velocità della luce nel vuoto) e si ottiene l'unità di carica \qty{1}{\coulomb}.

La costante ha un valore numerico ``brutto'' perché in realtà il coulomb è stato definito dopo l'ampere (vedi \autoref{sec:def_ampere}).

Fatto questo, l'unità di misura di $k_e$ risulta essere $[k_e] = \unit{\newton\metre\squared\per\coulomb\squared}$.

Si definisce anche $\eps_0$, la \important{costante dielettrica del vuoto}, in modo che valga
\begin{equation}
    k_e = \frac{1}{4\pi \eps_0}
\end{equation}

Fatto ciò, la legge di Coulomb, da legge empirica, diventa un principio.

(Le leggi empiriche sono quelle che non si dimostrano, ma si osservano direttamente.)

\subsection{Conservazione della carica}

Si osserva anche un'altra proprietà: \important{la carica elettrica si conserva}.
Cioè, considerando un volume $V \subset \R^3$ delimitato da una superficie $S$, le cariche entrano o escono attraverso $S$, ma non nascono né spariscono lì dentro.

Di preciso, a conservarsi è la carica netta (cioè, considerata con segno).

\subsection{Principio di sovrapposizione degli effetti}

Un terzo principio è il \important{principio di sovrapposizione degli effetti}: se in presenza di una carica $q_0$ in posizione $\p_0$ ci sono altre $n$ cariche $q_i$ in posizioni $\p_i$, per $i = 1, \ldots, n$, la forza totale su $q_0$ è la somma vettoriale delle forze che le altre cariche esprimono singolarmente su $q_0$.
\begin{equation}
    \force_0 = \sum_{i=1}^n \frac{q_0 q_i}{4 \pi \eps_0 \norm{\p_0 - \p_i}^2} \ver{r_i}
\end{equation}
dove $\ver{r_i}$ è il versore che va da $\p_i$ a $\p_0$:
\begin{equation}
    \ver{r_i} = \frac{\p_0 - \p_i}{\norm{\p_0 - \p_i}}
\end{equation}

Attenzione: il principio di sovrapposizione non è qualcosa di implicito nella legge di Coulomb, ma va verificato sperimentalmente.
Infatti, in generale, ci sono interazioni che non lo rispettano.

Grazie al principio di sivrapposizione, di può definire il \important{campo elettrico}: la forza che agisce per unità di carica in un punto dello spazio, con unità di misura $\unit{\newton\per\coulomb}$.

Il campo elettrico generato dalle $n$ cariche di prima valutato in $\p_0$ è
\begin{equation}
    \E(\p_0) = \frac{\force_0}{q_0} = \sum_{i=1}^n \frac{q_i}{4 \pi \eps_0 \norm{\p_0 - \p_i}^2} \ver{r_i}
\end{equation}

È una proprietà definita in ogni punto dello spazio, e si misura tramite cariche campione.

Il campo elettrico venne introdotto da Faraday come semplice strumento matematico.
Tuttavia, sembra che sia molto più fondamentale di così dal punto di vista fisico, così come altri campi che incontreremo in seguito.
Sembra infatti che sia una proprietà non solo nello spazio, ma \textit{dello} spazio.

In sintesi:

L'elettrostatica è lo studio delle proprietà elettriche della materia ferma nello spazio.
Si basa su tre principi:
\begin{itemize}
    \item legge di Coulomb
    \item conservazione della carica
    \item principio di sovrapposizione degli effetti
\end{itemize}

\section{Conservatività del campo elettrico}

Se la carica $q$ che genera il campo è nell'origine, il campo elettrico in un punto generico $P$ a distanza $r$ dall'origine è
\begin{equation}
    \E(P) = \frac{q}{4\pi\eps_0 r^2} \ver{r}
\end{equation}
con $\ver{r}$ versore radiale verso $P$.

Se vi è una carica $q_0$ in $P$, la forza che agisce su essa è $q_0 \E(P)$.

\adddrawio[Carica $q_0$ in moto lungo una curva $\gamma$ in presenza del campo elettrico generato dalla carica $q$.][0.4]{efield_work}

Se la carica viene spostata lungo un arco $\gamma$ dal punto $A$ al punto $B$ a distanze $r_A$ e $r_B$ dall'origine, la forza elettrica svolge un lavoro e si dimostra che questo lavoro dipende solo dalla componente radiale dello spostamento:
\begin{equation}
\label{eq:potenziale_elettrostatico}
    \begin{gathered}
        L_{AB, \gamma} = \int_\gamma \force \lde = \frac{q_0 q}{4\pi \eps_0} \int_{r_A}^{r_B} \frac{1}{r^2} \de r = - \frac{q_0 q}{4\pi \eps_0} \pts{\frac{1}{r_B} - \frac{1}{r_A}} = \\
        = -q_0 \pts{\frac{q}{4\pi \eps_0 r_B} - \frac{q}{4\pi \eps_0 r_A}}
    \end{gathered}
\end{equation}

Poiché $L_{AB,\gamma}$ non dipende da $\gamma$, \important{il campo elettrico è conservativo} (cioè genera forze conservative).

I due termini nell'\cref{eq:potenziale_elettrostatico}, se moltiplicati per $q_0$, sono energie che determinano una variazione di energia potenziale.

Trascurando la carica campione, si ottiene il potenziale elettrico $\spot(P)$: l'energia potenziale per unità di carica, con unità di misura volt: $\unit{\volt} = \unit{\joule\per\coulomb}$.

Si ha quindi
\begin{equation}
\label{eq:spot_integrale}
    \int_A^B \E \lde = -\int_A^B \de \spot
\end{equation}
% \begin{equation}
%     \int_A^B \E \lde = \frac{1}{q_0} \int_A^B \force \lde = \frac{1}{q_0} (- q_0) \pts{\frac{q}{4\pi \eps_0 r_B} - \frac{q}{4\pi \eps_0 r_A}} = - (\spot_B - \spot_A) = -\int_A^B \de \spot
% \end{equation}
In termini di differenziali,
\begin{equation}
\label{eq:spot_differenziali}
    \E \lde = - \de \spot
\end{equation}
Questa è una legge di conservazione dell'energia (energia cinetica a sinistra, ``meno energia potenziale'' a destra).

Quindi, $[\E] = \unit{\volt\per\metre}$, preferito a \unit{\newton\per\coulomb}, che non è usato.

Risolvendo l'\cref{eq:spot_integrale} per $\spot$ si ottiene
\begin{equation}
    \spot = \frac{q}{4\pi \eps_0 r} + \text{costante}
\end{equation}

Dal punto di vista ficico, ha senso porre a zero l'energia potenziale quando la carica di prova è a distanza infinita, poiché essa non subisce più lavoro.
Pertanto, la costante arbitraria si sceglie pari a $0$.

L'energia potenziale soddisfa il principio di conservazione degli effetti, quindi anche il potenziale lo soddisfa: se ci sono più cariche, i potenziali si sommano (come scalari)
\begin{equation}
    \spot(\p) = \sum_{i=1}^n \frac{q_i}{4\pi \eps_0 \norm{\p - \p_i}}
\end{equation}

Come si calcola $\E$ a partire da $\spot$?
\Cref{eq:spot_differenziali} si riscrive come
\begin{equation}
    E_x \de x + E_y \de y + E_z \de z = - \de \spot
\end{equation}
Fissando $y$ e $z$,
\begin{equation}
    E_x \de x = - \de \spot \implies E_x = - \parder[\spot]{x}
\end{equation}
Ripetendo per $y$ e $z$ si ottiene
\begin{equation}
\label{eq:meno_grad_spot}
    \E = - \grad \spot
\end{equation}

Dal fatto che $\E$ è conservativo, si deduce che la circuitazione di $\E$ (il lavoro della forza elettrica per unità di carica lungo una curva chiusa) è sempre nulla.

Inoltre, poiché $\E$ è conservativo, allora è anche irrotazionale:
\begin{equation}
    \curl \E = \vt{0}
\end{equation}

Fisicamente, questo si intuisce tramite il teorema di Stokes: se ogni circuitazione è nulla, allora è nullo anche il flusso del rotore attraverso qualunque superficie.

In particolare, calcoliamo la circuitazione lungo una spira rettangolare infinitesima perpendicolare all'asse $x$:
\begin{equation}
    \oint_\gamma \E \lde = \pts{\parder[E_z]{y} - \parder[E_y]{z}} \de y \de z = \pts{\curl \E} \sde
\end{equation}
È il teorema di Stokes per una curva chiusa infinitesima.
L'integrale a destra è sempre nullo, quindi deve essere sempre nullo anche il rotore.


% ## Interazione a distanza

% A seguito della scoperta della legge di gravitazione universale, non piaceva l'interazione a distanza
% L'introduzione di un campo permette invece di spiegare ciò tramite la propagazione di una proprietà dello spazio

\section{Distribuzione continua di carica}

Densità di carica $\rho$:
\begin{equation}
    \de q = \rho \, \de V
\end{equation}
definendo
\begin{equation}
    \ver{r'} \coloneq \frac{\p - \p'}{\norm{\p - \p'}}
\end{equation}
Si ottengono delle espressioni per $\E$ e $\spot$ in funzione della distribuzione di carica nello spazio:
\begin{gather}
    \de \E(\p) = \frac{\de q}{4\pi \eps_0 \norm{\p - \p'}^2} \ver{r'}
    \implies
    \E(\p) = \frac{1}{4\pi \eps_0} \int_V \frac{\rho(\p')}{\norm{\p - \p'}^2} \ver{r'} \, \de V' \\
    \de \spot(\p) = \frac{\de q}{4\pi \eps_0 \norm{\p - \p'}}
    \implies
    V(\p) = \frac{1}{4\pi \eps_0} \int_V \frac{\rho(\p')}{\norm{\p - \p'}} \de V'
\end{gather}
$V$ è il volume all'interno del quale la densità di carica è non nulla (ciò su cui occorre integrare).

\section{Legge di Gauss}

Sia $V$ un volume tale che $S = \partial V$ sia una superficie chiusa connessa.

Flusso di $\E$ attraverso $S$:
\begin{equation}
    \de \Phi = \E \sde, \qquad \Phi = \oint_S \E \sde
\end{equation}

\adddrawio[][0.4]{gauss_proof}

Calcolo per una carica $q$ in $\p' \in V$ con $R = \norm{\p - \p'}$
(una rappresentazione bidimensionale è data in \cref{fig:gauss_proof})
\begin{equation}
    \de \Phi = \frac{q}{4\pi \eps_0} \frac{\ver{r'} \cdot \uvt{n}}{R^2} \de S = \frac{q}{4\pi \eps_0} \frac{\de S \cos \theta}{R^2} = \frac{q}{4\pi \eps_0} \de \Omega
\end{equation}
$\Omega$ è l'angolo solido.
\begin{equation}
    \Phi = \oint_S \frac{q}{4\pi \eps_0} \de \Omega
    = \frac{q}{4\pi \eps_0} 4\pi
    = \frac{q}{\eps_0}
\end{equation}

Se ci sono più cariche o un volume di carica, per il principio di sovrapposizione degli effetti, è sufficiente considerare la carica totale.
Si ottiene quindi il formato integrale della legge di Gauss:
\begin{equation}
\label{eq:gauss_integrale}
    \Phi = \oint_S \E \sde = \frac{1}{\eps_0} \int_V \rho \, \de V
\end{equation}

Il teorema di Gauss applicato al flusso del campo elettrico risulta in
\begin{equation}
    \oint_S \E \sde = \int_V \diver \E \, \de V
\end{equation}
Poiché l'\cref{eq:gauss_integrale} deve valere per ogni volume $V$, occorre eguagliare gli integrandi e si ottiene il formato differenziale della legge di Gauss:
\begin{equation}
    \diver \E = \frac{\rho}{\eps_0}
\end{equation}

Attenzione: legge di Gauss $\ne$ teorema di Gauss $=$ teorema della divergenza.

Diamo un'idea del perché valga il teorema di Gauss.

\adddrawio[][0.6]{gauss_cube}

Consideriamo un volume infinitesimo cubico come in \cref{fig:gauss_cube}, con vertice di coordinate minori $\p$, di lati $\de x$, $\de y$, $\de z$ e volume $\de V = \de x \de y \de z$.
Se $\de \p \parallel \ux$,
\begin{equation}
    E_x(\p + \de \p) \approx E_x(\p) + \grad E_x(\p) \cdot \de \p = E_x(\p) + \parder[E_x(\p)]{x} \de x
\end{equation}
Il flusso attraverso le facce parallele al piano $zy$ è
\begin{equation}
    E_x(\p + \de \p) \de y \de z - E_x(\p) \de y \de z = \parder[E_x(\p)]{x} \de x \de y \de z
\end{equation}
Ripetendo per $y$ e $z$ e sommando si ottiene il flusso totale:
\begin{equation}
    \de \Phi = (\diver \E) \de V
\end{equation}

Ora, accostando piccoli volumi, i contributi adiacenti dei flussi si semplificano. Per un volume generico vale
\begin{equation}
    \Phi = \int_V \diver \E \, \de V
\end{equation}

\section{Equazione di Poisson}

Osserva che le due seguenti leggi:
\begin{gather}
\label{eq:gauss_differenziale}
    \diver \E = \frac{\rho}{\eps_0} \\
    \curl \E = \vt{0}
\end{gather}
sono conseguenze dei soli tre principi dell'elettrostatica e sono equivalenti a essi, ne sono una riscrittura analitica.

Dai tre principi siamo passati alle due leggi sopra.
Ora, è possibile ridurre queste leggi in un'unica equazione, sostituendo l'\cref{eq:meno_grad_spot} nell'\cref{eq:gauss_differenziale}:
\begin{equation}
    \lapl \spot = - \frac{\rho}{\eps_0}
\end{equation}
La soluzione di quest'equazione differenziale è proprio
\begin{equation}
    \spot(\p) = \frac{1}{4\pi \eps_0} \int_V \frac{\rho(\p')}{\norm{\p - \p'}} \de V'
    % = \frac{1}{\eps_0} \mathcal{S} \rho(r)
\end{equation}

% Tuttavia, la forma differenziale è più usata numericamente (:-/)

\section{Conduttori e condensatori}

Conduttore: materiale in cui le cariche possono muoversi liberamente.

In un conduttore, le cariche si posizionano sulla superficie.
Per questo, all'interno, $\E = \vt{0}$ e $\rho = 0$.
Ne segue anche che il potenziale è costante per l'intero conduttore.

Sulla superficie, $\E \parallel \uvt{n}$.
Se non fosse così, si avrebbe una componente tangenziale del campo elettrico e le cariche si muoverebbero sulla superficie.

È conveniente definire la densità superficiale (o ``areica'') $\sigma = \de q / \de S$.

Il potenziale a cui è il conduttore è
\begin{equation}
    \spot(\p) = \frac{1}{4\pi \eps_0} \int_S \frac{\sigma(\p')}{\norm{\p - \p'}} \de S'
\end{equation}

Rapporto potenziale/carica per le cariche sulla superficie:
\begin{equation}
    \frac{1}{C} = \frac{\spot(\p)}{Q} = \frac{1}{4\pi \eps_0} \int_S \frac{\sigma(\p')/Q}{\norm{\p - \p'}} \de S', \quad \p \in S
\end{equation}
$C$ è detta \important{capacità} ed è una costante che dipende solo dalla forma della cuperficie $S$ poiché l'integrale è puramente geometrico.

Consideriamo due conduttori con lo stesso quantitativo di carica di segno opposto $Q = Q_+ = -Q_- > 0$.
Si definiscono
\begin{itemize}
    \item armetura: ciascuno dei due conduttori
    \item condensatore: il sistema completo
\end{itemize}

Le linee di campo escono dall'armatura positiva ed entrano in quella negativa.

La differenza di potenziale (d.d.p.) vale

\begin{equation}
    \Delta \spot = \frac{Q_+}{C_+} - \frac{Q_-}{C_-} = Q \underbrace{\pts{\frac{1}{C_+} + \frac{1}{C_-}}}_{1/C}
\end{equation}

$C$ è la capacità del condensatore e si misura in farad: $[C] = \unit{\farad} = \unit{\coulomb\per\volt}$.

\subsection{Condensatore a facce piane parallele}

Consideriamo due piani conduttori infiniti paralleli di superficie $S$ a distanza $d$.

``Infiniti'' significa che $d$ è trascurabile rispetto alla loro superficie, $d \ll \sqrt{S}$.

Le distribuzioni di densità di carica superficiale sono uguali in modulo: $\sigma = \sigma_+ = -\sigma_- > 0$.

Per calcolare $C$, serve calcolare $V = V_+ - V_-$.

Poiché il piano è infinito, il campo elettrico è perpendicolare al piano.

\adddrawio[][0.3]{gauss_plane}

Considerando un cilindro di area di base $A$ perpendicolare al piano e che lo attraversa, come in \cref{fig:gauss_plane}.
Detto $E$ il modulo del campo elettrico presso le basi del cilindro, si applica la legge di Gauss:
\begin{equation}
    \Phi = 2 E A = \frac{\sigma A}{\eps_0} \implies E = \frac{\sigma}{2 \eps_0}
\end{equation}
In particolare, il campo è uniforme da ognuna delle due parti del piano.

Sommando il contributo dell'altro piano (che, tra le armature, è concorde), il campo tra le armature è, in modulo,
\begin{equation}
    E = \frac{\sigma}{\eps_0}
\end{equation}
Otteniamo la capacità del condensatore:
\begin{equation}
    V = - \int_{-,\gamma}^+ \E \lde = \frac{\sigma d}{\eps_0} \implies
    C = \frac{Q}{V} = \frac{\sigma S}{\frac{\sigma d}{\eps_0}} = \eps_0 \frac{S}{d}
\end{equation}

\subsection{Energia del campo elettrico}

Quale lavoro serve per spostare una carica $Q > 0$ da un'armatura all'altra?
\begin{gather}
    \de L = V \de q \\
    L = \int_0^Q V \de q = \int_0^Q \frac{q}{C} \de q = \frac{Q^2}{2 C} = \frac{1}{2} C V^2
\end{gather}
Nel caso di un condensatore a facce piane parallele:
\begin{equation}
    L = \frac{1}{2} \underbrace{\eps_0 \frac{S}{d}}_C {\underbrace{(E d)}_V}^2 = \frac{1}{2} \eps_0 S d E^2
\end{equation}
Osservando che $S d$ è il volume tra le facce del condensatore, l'energia per unità di volume è
\begin{equation}
    w_E = \frac{1}{2} \eps_0 E^2
\end{equation}

Questa è l'energia del campo elettrico.
Questa formula, in realtà, vale sempre, non solo per questo specifico caso.
Alla presenza di campo elettrico nello spazio è associata una certa densità di energia elettrica $w_E$.

Il lavoro svolto per spostare le cariche (ad esempio, da generatori di tensione o di corrente) risulta nella creazione di un campo elettrico, quindi si trasforma in questa forma di energia.

\section{Corrente elettrica}

Consideriamo un cilindro con due facce cariche in modo opposto: al suo interno ho un campo elettrico uniforme.
Sia $n$ il numero di cariche libere $q$ per unità di volume.
Messe in moto dal campo elettrico, hanno velocità $\vt{v}$.

% ($\vt{v}$ dovrebbe aumentare linearmente, in quanto il moto è uniformemente accelerato.
% In realtà è costante, poiché gli elettroni perdono energia urtando gli atomi del conduttore. Vedi \autoref{sec:legge_ohm}.)

La quantità di carica $\Delta Q$ che attraversa una sezione $S$ in un tempo $\Delta t$ è
\begin{equation}
    \Delta Q = n q \Delta t \vt{v} \cdot \vt{S}
\end{equation}
poiché il volume delle cariche che raggiungono la superficie è $(\vt{v}\Delta t) \cdot \vt{S}$.

Corrente elettrica: quantità di carica che attraversa una sezione per unità di tempo:
\begin{equation}
    I \coloneq \frac{\Delta Q}{\Delta t} = nq\vt{v} \cdot \vt{S}
\end{equation}
Unità di misura: ampere $\unit{\ampere} = \unit{\coulomb\per\second}$.

Densità di corrente: corrente per unità di superficie:
\begin{equation}
    \dcurr = nq\vt{v}
\end{equation}

$I$ è il flusso di $\dcurr$ attraverso $S$:
\begin{equation}
    \de I = \dcurr \sde, \qquad I = \int_S \dcurr \sde
\end{equation}

\subsection{Principio di conservazione della carica}

Se all'interno di un volume $V$ la carica è costante, la corrente totale attraverso la superficie è nulla:
\begin{equation}
    I = \oint_{\partial V} \dcurr \sde = 0
\end{equation}

\subsection{Legge di Ohm}
\label{sec:legge_ohm}

Consideriamo un conduttore e accendiamo un campo elettrico (lo si fa mettendo cariche positive da una parte e negative dall'altra).
Le cariche nel conduttore vengono messe in moto e si genera una corrente $\de I = \dcurr \sde$.

La corrente fa sì che le cariche positive interne si avvicinino a quelle negative che generano il campo, e viceversa.
Alla fine, il campo si sarà annullato.

Per permettere che attraverso un conduttore scorra corrente indefinitamente, serve svolgere del lavoro per riportare le cariche dall'altra parte (rispettando il principio di conservazione dell'energia).

Il lavoro svolto dal campo elettrico è
\begin{equation}
    \de L = \force \lde = q \E \lde = - q \de \spot
\end{equation}
Considerando una quantità di carica positiva $\Delta Q$ (che si muove verso le cariche negative) e definendo $V = \spot_+ - \spot_-$, il lavoro risulta direttamente proporzionale alla corrente.

Chiamiamo $R$ la costante di proporzionalità:
\begin{equation}
    L = \Delta Q \, V = \Delta Q  R I
\end{equation}

Si ottiene la \important{legge di Ohm}:
\begin{equation}
    V = R I
\end{equation}

$R$ è detta \important{resistenza} e l'unità di misura è l'ohm: $[R] = \unit{\ohm} = \unit{\volt\per\ampere}$.

Se il conduttore è un filo, si misura che in tutte le sezioni del filo la corrente è la stessa.
Poiché $I = \rho v S$ e la sezione del filo è costante, la velocità sarà la stessa.
Quindi, \important{la velocità delle cariche è costante}, è detta \important{velocità di deriva} e non aumenta linearmente come ci si aspetterebbe da un moto uniformemente accelerato (dovuto a un campo elettrico costante).
Parte dell'energia del campo elettrico, infatti, viene \important{dissipata in calore} per \important{effetto Joule}, che ha l'effetto di scaldare il filo.

Questo meccanismo può essere modellizzato tramite urti delle cariche in moto contro gli atomi del materiale.
È un modello ``sbagliato'', ma dà risultati corretti.

Tra un urto e l'altro, il moto è uniformemente accelerato, ma la velocità acquisita viene persa a ogni urto.

La legge di Ohm è una legge di conservazione dell'energia: $V$ quantifica il lavoro del campo elettrico, $RI$ la dissipazione termica.
$R$ svolge il ruolo della viscosità.

Per garantire che scorra una corrente $I$, è necessario un generatore di tensione che compensi l'energia dissipata.
Questo può essere chimico (pila o batteria) o meccanico (forza elettromotrice).

La resistenza è una grandezza estensiva e dipende dalla geometria del materiale.
Dette $l$ la lunghezza del conduttore e $S$ la sezione,
\begin{gather}
    V = RI
    \implies E l = R J S \\
    \implies J = \frac{l}{RS} E = \sigma E
\end{gather}
$\sigma$ non dipende dalla geometria ed è una proprietà del materiale: \important{conducibilità elettrica}, $[\sigma] = \unit{\per\ohm\per\metre}$

Posso anche calcolare la velocità:
\begin{equation}
    v = \frac{J}{nq} = \frac{l}{n q R S} E \sim \qty{e-4}{\metre\per\second}
\end{equation}
Quest'ordine di grandezza vale per molti materiali.

Per $T = \qty{300}{\kelvin}$, la velocità termica (cioè la velocità quadratica media) di un ``gas ideale di elettroni'' sarebbe:
\begin{equation}
    v_\text{th} = \sqrt{\frac{3 k_B T}{m_e}} \sim \qty{e5}{\metre\per\second}
\end{equation}
Questo mostra che la velocità netta delle cariche è molto minore della velocità termica e che la corrente elettrica deriva da un moto additivo dovuto al campo elettrico, debolissimo rispetto al moto termico.

\chapter{Elettrostatica -- esercitazioni}

\section{Ripasso sull'elettrostatica}

\begin{equation}
    \E(\p) = \frac{1}{4\pi\eps_0} \int_V \frac{\rho(\p')}{\norm{\p - \p'}^2} \ver{\p'} \, \de V'
\end{equation}
\begin{equation}
    \spot(\p) = \frac{1}{4\pi\eps_0} \int_V \frac{\rho(\p')}{\norm{\p - \p'}} \de V'
\end{equation}
La singola componente $\alpha \in \{x, y, z\}$ del campo elettrico è:
\begin{equation}
    E_\alpha(x,y,z) = \frac{1}{4\pi\eps_0} \int_V \frac{\rho(x',y',z')}{(x - x')^2 + (y - y')^2 + (z - z')^2} (\ver{\p'})_\alpha \, \de V'
\end{equation}

Conservatività:
\begin{gather}
    \E = - \grad \spot \\
    \E \lde = - \de V \iff \int_A^B \E \lde = \spot(\p_A) - \spot(\p_B) \\
    \curl \E = \vt{0}
    \iff
    \oint_\gamma \E \lde = 0
\end{gather}

Legge di Gauss ($S$ superficie chiusa):
\begin{equation}
    \diver \E = \frac{\rho}{\eps_0}
    \iff
    \Phi_S(\E) \coloneq \oint_S \E \sde = \frac{q}{\eps_0}
\end{equation}

\section{Campo elettrico dovuto a distribuzioni di carica}

\subsection{Anello}

\adddrawio[][0.9]{efield_ring}

Anello $C$ di raggio $R$ e carica totale $q$ distribuita uniformemente.
Ha centro $\vt{0}$ e giace sul piano $yz$.
Trovare $\E(x, 0, 0)$ e $\spot(x, 0, 0)$.

Per simmetria, $\E(x, 0, 0) \parallel \ux$, cioè $\E(x, 0, 0) = E_x(x, 0, 0) \ux$.

Per trovare $E_x(x, 0, 0)$,
\begin{itemize}
    \item L'integrale è di linea e la densità lineare è $q / (2\pi R)$
    \item La componente $x$ di $\ver{\p'}$ è $\cos \theta = \frac{x}{\sqrt{x^2 + R^2}}$, con $\theta$ l'angolo tra l'asse $x$ e le rette tra i punti su $C$ e $(x, 0, 0)$.
    \item $x' = 0$
    \item $y = z = 0$
    \item $(y')^2 + (z')^2 = R^2$
\end{itemize}

\begin{equation}
\begin{aligned}
    E_x(x, 0, 0) & = \frac{1}{4\pi\eps_0} \int_C \frac{q/(2 \pi R)}{x^2 + R^2} \frac{x}{\sqrt{x^2 + R^2}} \de l' = \\
    & = \frac{qx}{8\pi^2\eps_0 R (x^2 + R^2)^{3/2}} \int_C \de l' = \\
    & = \frac{qx}{8\pi^2\eps_0 R (x^2 + R^2)^{3/2}} 2\pi R = \\
    & = \frac{q}{4\pi\eps_0} \frac{x}{(x^2 + R^2)^{3/2}}
\end{aligned}
\end{equation}

Analogamente, il potenziale risulta:
\begin{equation}
    \spot(x, 0, 0) = \frac{q}{4\pi\eps_0} \frac{1}{\sqrt{x^2 + R^2}}
\end{equation}

Si osserva $\E = - \grad \spot$.

Inoltre, per $x \gg R$,
\begin{gather}
    E_x(x, 0, 0) \approx \frac{q}{4\pi\eps_0 x^2} \\
    V(x, 0, 0) \approx \frac{q}{4\pi\eps_0 \abs{x}} \\
\end{gather}
ovvero, campo e potenziale dovuti a una carica puntiforme $q$.

\subsection{Disco}

Come prima, ma con carica di superficie su un disco $D$.

Ricaviamo prima il potenziale.

L'integrale si svolge in coordinate polari:
\begin{gather}
    \de S = r \, \de r \, \de \phi \\
\begin{aligned}
    \spot(x, 0, 0) & = \frac{1}{4\pi\eps_0} \int_D \frac{q / (\pi R^2)}{\sqrt{x^2 + (y')^2 + (z')^2}} \de S' = \\
    & = \frac{1}{4\pi\eps_0} \int_{\phi = 0}^{2\pi} \int_{r = 0}^R \frac{q / (\pi R^2)}{\sqrt{x^2 + r^2}} r \, \de r \, \de \phi = \\
    & = \frac{q}{4\pi^2\eps_0 R^2} \int_{0}^{2\pi} \de \phi \int_0^R \frac{r}{\sqrt{x^2 + r^2}} \, \de r = \\
    & = \frac{q}{2\pi\eps_0 R^2} \int_0^R \frac{r}{\sqrt{x^2 + r^2}} \, \de r
\end{aligned}
\end{gather}

L'integrale in $r$ è
\begin{equation}
    \int_0^R \frac{r}{\sqrt{x^2 + r^2}} \, \de r
    = \frac{1}{2} \int_0^R \frac{2r}{\sqrt{x^2 + r^2}} \, \de r
    = \frac{1}{2} \left[ 2 \sqrt{x^2 + r^2} \right]_{r=0}^R
    = \sqrt{x^2 + R^2} - \abs{x}
\end{equation}

Per cui,
\begin{equation}
    \spot(x, 0, 0) = \frac{q}{2\pi\eps_0} \frac{\sqrt{x^2 + R^2} - \abs{x}}{R^2}
\end{equation}
Poiché $\E(x, 0, 0) = E_x(x, 0, 0) \ux$,
\begin{equation}
    E_x(x, 0, 0) = - \parder[\spot]{x}(x, 0, 0) = \frac{q}{2\pi\eps_0 R^2} \pts{\sgn{x} - \frac{x}{\sqrt{x^2 + R^2}}}
\end{equation}
Per studiare $x \gg R$, qui serve andare al secondo ordine in $R / x$:
\begin{equation}
\begin{gathered}
    E_x(x, 0, 0) = \frac{q}{2\pi\eps_0 R^2} \pts{1 - \frac{x}{\abs{x}\sqrt{1 + \frac{R^2}{x^2}}}} \approx \\
    \approx \frac{q}{2\pi\eps_0 R^2} \pts{1 - \pts{1 - \frac{1}{2} \frac{R^2}{x^2}}}
    = \frac{q}{4\pi\eps_0 x^2}
\end{gathered}
\end{equation}

Per $\abs{x} \ll R$ o $R \to +\infty$,
\begin{equation}
    E_x(x, 0, 0) \approx \frac{q}{2\pi\eps_0 R^2} \sgn{x} = \frac{\sigma}{2\eps_0} \sgn{x}
\end{equation}
È il campo di un piano infinito con densità di carica superficiale $\sigma = q / (\pi R^2)$.

\subsection{Piano infinito}

Piano infinito che giace nel piano $yz$ e ha densità di carica superficiale $\sigma$.

Come mostrato in precedenza:
\begin{gather}
    \E(x, y, z) = \frac{\sigma}{2\eps_0} \sgn{x} \ux \\
    \spot(x, y, z) = -\frac{\sigma}{2\eps_0} \abs{x}
\end{gather}

\subsubsection{Acceleratore di carica}
\label{sec:acceleratore_carica}

Un acceleratore di carica permette di accelerare cariche nel vuoto e determinarne la velocità.

Svolgiamo il bilancio energetico per una particella di carica $-q$ inizialmente sull'armatura A (negativa) di un condensatore a facce piane parallele.
La differenza di potenziale tra le armeture è $\Delta V = V_B - V_A > 0$.
\begin{equation}
    -q V_A = \frac{1}{2} m v^2 - q V_B
    \implies K = \frac{1}{2} m v^2 = q \Delta V
    \implies v = \sqrt{\frac{2 q \Delta V}{m}}
\end{equation}
$v$ è la velocità della particella quando raggiunge o supera l'armatura B.

Per $\Delta V = \qty{10}{\volt}$, si avranno $K = \qty{10}{\electronvolt}$ e $v = \qty{1.9e6}{\metre\per\second} \approx \num{e-2} c_0$.

\subsubsection{Separatore elettrostatico}

\addfigure[][0.8]{book/separatore_elettrostatico}

È possibile deviare una particella di carica $-q$ di massa $m$ dopo averla accelerata fino a una velocità iniziale $v_0$, facendola passare attraverso armature perpendicolari alle prime a una distanza $h$ dall'armatura positiva.
Il moto sarà parabolico.

Siano $a$ la lunghezza del separatore e $L$ una lunghezza successiva.
Ci si chiede la differenza di quota $d$.

Poniamo l'armatura negativa in basso.
Il campo e la forza sono $\E = -E \uy$ e $\force = q E \uy$.

Con l'origine all'inizio del separatore, la traiettoria è
\begin{gather}
    y(x) = \frac{1}{2} \frac{q E}{m} t^2 = \frac{q E x^2}{2 m v_0^2} \\
    \tan \alpha = \evalat{\der[y]{x}}{x = a} = \frac{q E a}{m v_0^2}, \qquad d = L \tan \alpha + h
\end{gather}

\subsection{Sfera}

Si consideri una sfera di carica $q$, raggio $R$ e centro nell'origine.

Per simmetria, il campo è radiale: $\E(\p) = E(r)\ver{r}$

Consideriamo una superficie sferica $\Sigma$ di raggio $r$.
Allora,
\begin{equation}
\begin{gathered}
    \Phi_\Sigma(\E) = \oint_\Sigma \E \sde = \oint_\Sigma E(r) \de S= 4\pi r^2 E(r) = \frac{q_r}{\eps_0} \\
    \implies E(r) = \frac{q_r}{4\pi\eps_0 r^2}
\end{gathered}
\end{equation}
dove $q_r$ è la carica all'interno di $\Sigma$.
Quindi,
\begin{equation}
    E(r) = \begin{cases}
        0 & \text{se } r < R \\
        \dfrac{q}{4\pi\eps_0 r^2} & \text{se } r > R
    \end{cases}
\end{equation}
Ovvero, se si è all'esterno, è come se tutta la carica fosse nel centro della sfera.

Il salto attraverso la superficie è $[E(r)]_R = \sigma/\eps_0$, quello di un piano infinito con campo nullo da una delle due parti.

Il potenziale è costante all'interno della sfera (poiché il campo è nullo) e, per determinare \textit{quale} costante, lo si impone continuo attraverso $r = R$:
\begin{equation}
\label{eq:potenziale_sfera}
    V(r) = \begin{cases}
        \frac{q}{4\pi\eps_0 R} & \text{se } r \le R \\
        \frac{q}{4\pi\eps_0 r} & \text{se } r > R
    \end{cases}
\end{equation}


\subsection{Palla}

Palla omogenea di carica $q$, raggio $R$ e centro nell'origine.

All'esterno, è tutto identico al caso della sfera carica.

Ora, se $r < R$, $q_r$ non è più nulla, ma
\begin{equation}
    q_r = q \frac{r^3}{R^3}
\end{equation}
Per cui
\begin{equation}
    E(r) = \frac{q_r}{4\pi\eps_0 r^2}
    = \frac{q}{4\pi\eps_0 R^3} r
    = \frac{\rho}{3\eps_0} r
\end{equation}
dove $\rho = q / (\frac{4}{3} \pi R^3)$ è la densità di carica.

Calcoliamo il potenziale all'interno:
\begin{equation}
\begin{gathered}
    \spot(r) - \spot(R) = \int_r^R \E \lde = \int_r^R E(r') \, dr' = \frac{\rho}{6\eps_0} \pts{R^2 - r^2} \\
    \implies \spot(r) = \frac{\rho}{6\eps_0} \pts{R^2 - r^2} + \frac{q}{4\pi\eps_0 R} = \frac{q}{8\pi\eps_0 R} \pts{3 - \frac{r^2}{R^2}}
\end{gathered}
\end{equation}


\subsection{Cilindro}

\adddrawio[Il cilindro di raggio $R$ è da intendersi infinito.]{efield_cylinder}

Cilindro infinito di raggio $R$ con asse coincidente con l'asse $z$ e densità di carica $\rho$.

Poiché il cilindro è infinito, il campo non ha componente $z$.

Sia $\lambda = \pi R^2 \rho$ la densità lineare di carica.
Si considera il flusso attraverso un cilindro (limitato) $\Sigma$ di raggio $r$ e altezza $h$, in modo che il flusso sia nullo sulle basi:
\begin{equation}
    q_r = h \lambda, \quad \text{se } r > R
\end{equation}
\begin{equation}
\begin{gathered}
    \Phi_\Sigma(\E) = \oint_\Sigma \E \sde = \oint_\Sigma E(r) \de S = 2\pi r h E(r) = \frac{q_r}{\eps_0} \\
    \implies E(r) = \frac{\lambda}{2\pi\eps_0 r}
\end{gathered}
\end{equation}

Se $r < R$,
\begin{equation}
    E(r) = \frac{\lambda}{2\pi\eps_0 R^2} r
\end{equation}

Il potenziale all'interno è
\begin{equation}
    V(r) = \frac{\lambda}{4\pi\eps_0 R^2} \pts{R^2 - r^2}
\end{equation}

All'esterno non si può fissare nullo il potenziale in $r = +\infty$.
È comunque possibile quantificare le differenze di potenziale:
\begin{equation}
    \spot(r_A) - \spot(r_B) = \int_A^B \E \lde = \frac{\lambda}{2\pi\eps_0} \ln\frac{r_B}{r_A}
\end{equation}

In un certo senso vale:
\begin{equation}
    V(r) = \begin{cases}
        \frac{\lambda}{4\pi\eps_0 R^2} \pts{R^2 - r^2} & r \le R \\
        - \frac{\lambda}{2\pi\eps_0} \ln\frac{r}{R} & r > R
    \end{cases}
\end{equation}

\section{Coordinate sferiche}

\addfigure[Da \shorturl{commons.wikimedia.org/wiki/File:3D\_Spherical.svg}.][0.4]{wikimedia/spherical_coordinates.svg.png}

\begin{subequations}
\begin{gather}
    r \in \left[0, +\infty\right), \quad \theta \in \left[0, \pi\right], \quad \phi \in \left[0, 2\pi\right) \\
    \begin{cases}
        x = r \sin \theta \cos \phi \\
        y = r \sin \theta \sin \phi \\
        z = r \cos \theta
    \end{cases} \\
    \begin{cases}
        r = \sqrt{x^2 + y^2 + z^2} \\
        \theta = \arccos\pts{z/\!\sqrt{x^2 + y^2 + z^2}} \\
        \tan \phi = y/x
    \end{cases}
\end{gather}
\end{subequations}

Matrice jacobiana:
\begin{equation}
    J = \begin{bmatrix}
        \parder[x]{r} & \parder[x]{\theta} & \parder[x]{\phi} \\
        \parder[y]{r} & \parder[y]{\theta} & \parder[y]{\phi} \\
        \parder[z]{r} & \parder[z]{\theta} & \parder[z]{\phi}
    \end{bmatrix}\!, \qquad
    \abs{\det J} = r^2 \sin \theta
\end{equation}

Differenziale di volume:
\begin{equation}
    \de V = \de x \, \de y \, \de z
    = \abs{\det J} \, \de r \, \de \theta \, \de \phi
    = r^2 \sin \theta \, \de r \, \de \theta \, \de \phi
\end{equation}

Gradiente:
\begin{gather}
    \grad f(r, \theta, \phi) = \parder[f(r, \theta, \phi)]{x} \ux + \parder[f(r, \theta, \phi)]{y} \uy + \parder[f(r, \theta, \phi)]{z} \uz \\
    \parder[f(r, \theta, \phi)]{x} =
    \parder[r]{x} \, \parder[f]{r} +
    \parder[\theta]{x} \, \parder[f]{\theta} +
    \parder[\phi]{x} \, \parder[f]{\phi}
\end{gather}
Risulta:
\begin{equation}
    \grad f(r, \theta, \phi) = \parder[f]{r} \ver{r} + \frac{1}{r}\parder[f]{\theta} \ver{\theta} + \frac{1}{r \sin \theta} \parder[f]{\phi} \ver{\phi}
\end{equation}




\section{Dipolo elettrico}

\subsection{Definizione}

Momento di dipolo elettrico per due cariche $-q$ e $q$ con $q > 0$ in posizioni $\p^-$ e $\p^+$ poste a distanza $a$ fissa:
\begin{equation}
    \vt{p} \coloneq q \vt{a} = q (\p^+ - \p^-) = q a \ver{- \to +}
\end{equation}

Il momento di dipolo $\vt{P}$ di un oggetto carico, intuitivamente, va dal ``centro di massa" delle cariche negative a quello delle positive.

\begin{equation}
    \vt{P} = \int_V \rho(\p) \, \p \, \de V
\end{equation}


\subsection{Potenziale e campo}

Potenziale dovuto a un dipolo elettrico con $\vt{p} \parallel \uz$.
Siano $r_1 = \norm{\p - \p^+}$, $r_2 = \norm{\p - \p^-}$:
\begin{equation}
    \spot(\p) = \frac{q}{4\pi\eps_0} \pts{\frac{1}{r_1} - \frac{1}{r_2}}
    = \frac{q}{4\pi\eps_0} \frac{r_2 - r_1}{r_1 r_2}
\end{equation}


Considerando $r \gg a$, otteniamo l'approssimazione al primo ordine in $a/r$ per $r_1 r_2$ e $r_2 - r_1$.
\begin{subequations}
\begin{gather}
    r_1 = \norm{\p - \frac{1}{2}\vt{a}} = \sqrt{r^2 + \frac{1}{4} a^2 - r a \cos \theta} \approx r \sqrt{1 - \frac{a}{r}\cos\theta} \\
    r_2 = \norm{\p + \frac{1}{2}\vt{a}} = \sqrt{r^2 + \frac{1}{4} a^2 + r a \cos \theta} \approx r \sqrt{1 + \frac{a}{r}\cos\theta}
\end{gather}
\end{subequations}

Quindi
\begin{subequations}
\begin{gather}
    r_1 r_2 \approx r^2 \sqrt{1 - \frac{a^2}{r^2}\cos^2 \theta} \approx r^2 \\
\begin{gathered}
    r_2 - r_1 \approx r \pts{\sqrt{1 + \frac{a}{r}\cos\theta} - \sqrt{1 - \frac{a}{r}\cos\theta}} \approx \\
    \approx r \pts{1 + \frac{1}{2} \frac{a}{r} \cos \theta - 1 + \frac{1}{2} \frac{a}{r} \cos \theta}
    = a \cos \theta
\end{gathered}
\end{gather}
\end{subequations}

Dunque, per $r \gg a$ (\important{dipolo ideale}),
\begin{gather}
    \spot(\p) = \frac{q}{4\pi \eps_0} \frac{r_2 - r_1}{r_1 r_2}
    \approx \frac{q}{4\pi \eps_0} \frac{a \cos \theta}{r^2}
    = \frac{1}{4\pi \eps_0} \frac{\vt{p} \cdot \ver{r}}{r^2} \\
\begin{gathered}
\label{eq:campo_dipolo_elettrico}
    \E(\p) = - \grad \spot(\p)
    = -\parder[\spot]{r} \ver{r} - \frac{1}{r}\parder[\spot]{\theta} \ver{\theta} - \cancel{\frac{1}{r \sin \theta} \parder[\spot]{\phi} \ver{\phi}} = \\
    = \frac{q a \cos\theta}{2\pi \eps_0 r^3} \ver{r} + \frac{q a \sin\theta}{4\pi \eps_0 r^3} \ver{\theta}
    = \frac{p}{4\pi \eps_0 r^3} \pts{2 \cos\theta \, \ver{r} + \sin\theta \, \ver{\theta}}
\end{gathered}
\end{gather}

$\parder[\spot]{\phi}$ è nulla per simmetria cilindrica.

Si osserva che:
\begin{itemize}
    \item Sull'asse $z$, $\E(\p) \parallel \uz$ con lo stesso verso (in quanto $\theta = 0$ e $\ver{r} = \uz$).
    \item Sul piano $xy$, $\E(\p) \parallel \uz$ con verso opposto (in quanto $\theta = \pi/2$ e $\ver{\theta} = -\uz$).
\end{itemize}

Se si considera il campo esatto, senza l'approssimazione, si parla di \important{dipolo fisico}.
Con il dipolo ideale non è possibile descrivere il campo tra le due cariche.


\subsection{Potenziale a grandi distanze}

Calcoliamo il potenziale a grande distanza da un corpo carico caratterizzato da una densità di carica $\rho$ in un colume $V$.

Usiamo la seguente espressione per la distanza tra due punti $\p$ e $\p'$:
\begin{equation}
    \norm{\p - \p'} = \sqrt{\norm{\p - \p'}^2} = \sqrt{r^2 + (r')^2 - 2 \p \cdot \p'} = r \pts{1 + \frac{(r')^2}{r^2} - 2 \frac{\p \cdot \p'}{r^2}}^{\!1/2}
\end{equation}

Se il corpo è vicino all'origine, possiamo usare l'approssimazione $r \gg r'$.
Risulta
\begin{equation}
\begin{gathered}
    \spot(\p)
    % = \frac{1}{\eps_0} \slp \rho(\p)
    = \frac{1}{4\pi\eps_0} \int_V \frac{\rho(\p')}{r} \pts{1 + \frac{(r')^2}{r^2} - 2 \frac{\p \cdot \p'}{r^2}}^{\!-1/2} \de V' \approx \\
    \approx \frac{1}{4\pi\eps_0} \int_V \frac{\rho(\p')}{r} \pts{1 + \frac{\p \cdot \p'}{r^2}} \de V' = \\
    = \frac{1}{4\pi\eps_0} \int_V \frac{\rho(\p')}{r} \de V' + \frac{1}{4\pi\eps_0} \int_V \frac{\rho(\p') \p \cdot \p'}{r^3} \de V' = \\
    = \frac{1}{4\pi\eps_0 r} \int_V \rho(\p') \, \de V' + \frac{1}{4\pi\eps_0 r^2} \frac{\p}{r} \cdot \int_V \rho(\p') \, \p' \, \de V'= \\
    = \frac{Q}{4\pi\eps_0 r} + \frac{\vt{P} \cdot \ver{r}}{4\pi\eps_0 r^2}
\end{gathered}
\end{equation}

Il momento di dipolo $\vt{P}$ è il responsabile del potenziale elettrostatico a grande distanza per oggetti complessivamente neutri.
Ogni oggetto neutro, a grande distanza, si può quindi rappresentare come un dipolo.


\subsection{Dipolo in campo elettrico}

Considerando un dipolo in un campo $\E$ uniforme, la forza totale $\force$ sul dipolo è nulla e il momento della coppia di forze sulle cariche è
\begin{equation}
    \vt{M} = (\vt{r}^+ - \vt{r}^-) \times \force
    = \vt{a} \times q \E
    = \vt{p} \times \E
\end{equation}

Lavoro ed energia potenziale relativi alla rotazione dall'angolo $\theta_0$ all'angolo $\theta$ rispetto alla direzione del campo elettrico:
\begin{equation}
    W = \int_{\theta_0}^\theta \norm{\vt{M}(\theta')} \de \theta'
    = \int_{\theta_0}^\theta p E \sin\theta' \de \theta'
    = - pE \pts{\cos\theta - \cos\theta_0}
\end{equation}
\begin{equation}
    U_e(\theta) = - \vt{p} \cdot \E = - p E \cos\theta
\end{equation}
La configurazione più stabile è quella con $\vt{p} \parallel \E$ con verso concorde.

Se il campo elettrico non è uniforme, $\vt{E}(\p^+) = \vt{E}(\p^-) + (J \E) \vt{a}$, dove $J \E$ è la matrice jacobiana del campo elettrico valutata in $\p^+ \approx \p^-$, che agisce tramite prodotto matrice-vettore.
\begin{gather}
    \force = q \vt{E}(\p^+) - q \vt{E}(\p^-) = q (J \E) \vt{a} = (J \E) \vt{p} \iff \\
    \iff \begin{cases}
        F_x = \vt{p} \cdot \grad E_x = p_x \parder[E_x]{x} + p_y \parder[E_x]{y} + p_z \parder[E_x]{z} \\
        F_y = \vt{p} \cdot \grad E_y = p_x \parder[E_y]{x} + p_y \parder[E_y]{y} + p_z \parder[E_y]{z} \\
        F_z = \vt{p} \cdot \grad E_z = p_x \parder[E_z]{x} + p_y \parder[E_z]{y} + p_z \parder[E_z]{z}
    \end{cases}
\end{gather}
coerentemente con $\force = - \grad U_e$.


\section{Condensatori}

\subsection{Condensatore sferico}

\adddrawio{spherical_capacitor}

Ricordando l'\cref{eq:potenziale_sfera}, la capacità di una sfera carica di raggio $R$ è
\begin{equation}
    C = \frac{q}{V} = 4\pi\eps_0 R
\end{equation}
Se $R = \qty{6e6}{\metre}$ (il raggio della Terra), allora $C = \qty{6e-4}{\farad}$.

Sembrerebbe difficile raggiungere alte capacità con piccoli dispositivi, eppure esistono in commercio dispositivi con capacità \qty{e3}{\farad}.

Consideriamo una sfera e guscio sferico concentrici con raggi $R_1 < R_2 < R_3$.
Caricando con $+q$ la sfera interna, le cariche sulle superfici saranno $+q$, $-q$ e $+q$ (nulle all'interno dei conduttori).
Il campo tra le armature è radiale e dipende solo dalla sfera interna.
\begin{equation}
    \E(\p) = \frac{q}{4\pi\eps_0 r^2} \ver{r}
\end{equation}
La differenza di potenziale tra le armature è
\begin{equation}
    V_1 - V_2 = \frac{q}{4\pi\eps_0} \pts{\frac{1}{R_1} - \frac{1}{R_2}}
\end{equation}
Per cui,
\begin{equation}
    C = 4\pi\eps_0 \frac{R_1 R_2}{R_2 - R_1}
\end{equation}
\begin{itemize}
    \item Per $R_2 \to +\infty$, fisicamente si ha il caso della sfera isolata e $C = 4\pi\eps_0 R_1$.
    \item Per $R_2 - R_1 \eqcolon h \ll R_1$, si ha $R_1 \approx R_2 \approxcolon R$ e
    \begin{equation}
        C = 4\pi\eps_0 \frac{R^2}{h} = \eps_0 \frac{A}{h} \to +\infty
    \end{equation}
    e si ha la stessa capacità che nel caso del condensatore a facce piane parallele.
\end{itemize}

Per ottenere dispositivi con enorme capacità, si creano nanostrutture che rendono l'area effettiva dell'ordine dei \unit{\kilo\metre\squared} per ogni \unit{\centi\metre\squared} di area ``apparente''.

``Condensatore'' = condensa energia elettrostatica in un volume.


\subsection{Condensatore cilindrico}

Per considerazioni analoghe, il campo tra le armature è
\begin{gather}
    E(r) = \frac{\lambda}{2\pi\eps_0 r} \\
    V_1 - V_2 = \frac{\lambda}{2\pi\eps_0} \ln\frac{R_2}{R_1} \\
    C = \frac{2\pi\eps_0 d}{\ln(R_2/R_1)}
\end{gather}
$d$ è la lunghezza del condensatore, per cui $q = \lambda d$.

Per $R_2 - R_1 \eqcolon h \ll R_1$, $R_1 \approx R_2 \approxcolon R$ e
\begin{equation}
    C = \frac{2\pi\eps_0 d}{\ln(1 + h/R)} = \frac{2\pi\eps_0 d R}{h}
    = \eps_0 \frac{A}{h}
\end{equation}
di nuovo come un condensatore a facce piane parallele.

\subsection{Energia elettrostatica}

\begin{equation}
    U_e = \frac{1}{2} C V^2 = \frac{1}{2} Q V = \frac{1}{2} \frac{Q^2}{C}
\end{equation}

\begin{equation}
    U_e = \int_V w_E \, \de V = \int_V \frac{1}{2}\eps_0 E^2 \de V
\end{equation}
Verifichiamo l'uguaglianza per un condensatore sferico:
\begin{equation}
\begin{gathered}
    U_e = \int_V \frac{1}{2}\eps_0 \frac{q^2}{16\pi^2\eps_0^2 r^4} \de V
    = \frac{q^2}{32\pi^2\eps_0} \int_V \frac{1}{r^4} \de V = \\
    = \frac{q^2}{32\pi^2\eps_0} 4\pi \int_{R_1}^{R_2} \frac{1}{r^2} \de r
    = \frac{q^2}{8\pi\eps_0} \pts{\frac{1}{R_1} - \frac{1}{R_2}}
    = \frac{1}{2} q V
\end{gathered}
\end{equation}
usando $\de V = 4\pi r^2 \de r$.

\section{Materiale dielettrico}

\addfigure[Da \href{https://commons.wikimedia.org/wiki/File:Dielettrico.png}{Papa November Brendy}, \href{https://creativecommons.org/licenses/by-sa/3.0}{CC BY-SA 3.0}, tramite Wikimedia Commons.][0.35]{wikimedia/dielettrico}

Finora, abbiamo trattato l'elettrostatica nel vuoto o per i conduttori.
Ora trattiamo i dielettrici: materiali isolanti (cioè che non perettono il moto di cariche libere) che modificano alcune proprietà elettrostatiche.

Consideriamo due condensatori a facce piane parallele identici.
Se tra le armature del secondo è inserito un dielettrico, le capacità non saranno uguali:
\begin{equation}
    \frac{C}{C_0} = \eps_r > 1
\end{equation}
$\eps_r$ è la \important{costante dielettrica relativa} e dipende dal materiale.
Inserire un dielettrico, quindi, permette di ``condensare'' meglio l'energia.

\important{Rigidità dielettrica} $E_\mathrm{max}$: campo al quale avviene la \important{rottura del dielettrico}, cioè per cui il dielettrico è costretto a condurre corrente (e si brucia).

Nel mezzo (= non nel vuoto), è sufficiente sostituire ovunque la costante dielettrica nel vuoto $\eps_0$ con la costante dielettrica $\eps = \eps_r \eps_0$.

A parità di carica $Q$, la differenza di potenziale tra le armature è minore che nel vuoto:
\begin{equation}
    V = \frac{Q}{C} = \frac{Q}{\eps_r C_0} = \frac{V_0}{\eps_r} < V_0
\end{equation}

A parità di potenziale $V$ la carica accumulata è maggiore:
\begin{equation}
    Q = CV = \eps_r C_0 V = \eps_r Q_0 > Q_0
\end{equation}
Ovvero, per ottenere lo stesso potenziale occorre spostare più carica.

Sempre a parità di potenziale, l'energia accumulata è maggiore, poiché si accumula più carica elettrica:
\begin{equation}
    U = \eps_r U_0 > U_0
\end{equation}

Un dielettrico è neutro e non conduttore, ma le sue particelle possono essere dipoli.
Classifichiamo i dielettrici in
\begin{itemize}
    \item dielettrici polari, con momento di dipolo intrinseco (ad esempio, sostanze con molecole polari come \ce{H2O});
    \item dielettrici apolari (ad esempio, sostanze con molecole apolari come \ce{CO2}).
\end{itemize}

Applicando un campo $\E_0$ a un dielettrico, i dipoli tendono a orientarsi secondo il campo.
Il campo elettrico totale sarà la somma di $\E_0$ e del campo generato dai dipoli, che ha verso opposto: $\E = \E_0 + \E'$, $E = E_0 - E' < E_0$.

Nel caso di dielettrici apolari, il campo $\E_0$ può generare dei dipoli indotti (e orientarli).

\important{Polarizzazione}:
\begin{equation}
    \vt{P}(\p) = n(\p) \vt{p}(\p)
\end{equation}
$n$ è la densità volumica dei dipoli.

I dipoli, sulle superfici dei dielettrici, espongono delle cariche ``spaiate'' e non mobili.
Quindi, si può scrivere il momento di dipolo totale
\begin{equation}
    \int_V \vt{P} \de V = \vt{P} S l
\end{equation}
dove $S$ è la superficie delle armature e $l$ è la distanza reciproca.
$\vt{P} S$, dimensionalmente, è una carica: quella che si vede sulla superficie del dielettrico.

Quindi il momento di dipolo totale è anche $Ql$.

Se le due armature hanno tra loro un angolo $\theta$ ma sono sufficientemente lontane, si può considerare ancora $\vt{P}$ uniforme e
\begin{equation}
    \int_V \vt{P} \de V = \vt{P} S l \cos \theta
\end{equation}

In generale, la densità di carica del dielettrico sulla superficie è
\begin{equation}
    \sigma\bound = \vt{P} \cdot \uvt{n}
\end{equation}
dove $\uvt{n}$ è la normale uscente dal dielettrico.

La densità totale sulla superficie delle armature, quindi, è $\sigma = \sigma\free - \sigma\bound$ ($\sigma\free$ è la carica libera).

Il campo elettrico nel dielettrico è
\begin{equation}
    E = \frac{\sigma}{\eps_0} = \frac{\sigma\free - \sigma\bound}{\eps_0}
\end{equation}
Ricordando $\sigma\free = \eps_0 E_0$, definiamo il campo di spostamento $\D$:
\begin{equation}
    \D \coloneq \eps_0 \E_0 = \eps_0 \E + \vt{P}
\end{equation}

Valgono le seguenti relazioni:
\begin{subequations}
\begin{alignat}{2}
    \sigma\bound & = \vt{P} \cdot \uvt{n}
    & \qquad & \text{$\uvt{n}$ uscente dal dielettrico} \\
    \sigma\free & = \pts{\D_2 - \D_1} \cdot \uvt{n}
    & \qquad & \text{$\uvt{n}$ dal mezzo 1 al mezzo 2}
\end{alignat}
\end{subequations}

La polarizzazione $\vt{P}$ è legata alle proprietà del dielettrico:
\begin{equation}
    \vt{P} = \eps_0 \chi_e \E
\end{equation}
$\chi_e$ è la \important{suscettività elettrica} e misura la risposta del dielettrico.
È adimensionata.

\begin{equation}
    \D = \eps_0\E + \eps_0 \chi_e \E = \eps_0 \underbrace{(1 + \chi_e)}_{\eps_r} \E = \eps \E
\end{equation}
Poiché $\eps_r \ge 1$, allora $\chi_e \ge 0$.

Flussi
\begin{subequations}
\begin{gather}
    \eps_0 \oint_S \E \sde = q = q\free + q\bound \\
    \oint_S \D \sde = \eps_0 \oint_S \E_0 \sde = q\free \\
    \oint_S \vt{P} \sde = -q\bound
\end{gather}
\end{subequations}
Ma anche
\begin{gather}
    \eps \oint_S \E \sde = q\free \\
    \oint_S \vt{P} \sde = \eps_0 \chi_e \oint \E \sde = \chi_e q
\end{gather}
Formati differenziali:
\begin{subequations}
\begin{gather}
    \diver \E = \frac{\rho}{\eps_0} = \frac{\rho\free}{\eps} \\
    \diver \D = \rho\free \\
    \diver \vt{P} = -\rho\bound = \chi_e \rho
\end{gather}
\end{subequations}

\subsection{Legge di Curie}

La suscettività dipende dalla temperatura secondo la legge di Curie:
\begin{equation}
    \chi_e = A + \frac{B}{T}
\end{equation}
per opportuni coefficienti $A$ e $B$.
Questo significa che scaldando il mezzo, la risposta dielettrica diminuisce.
Infatti, più difficilmente i dipoli del mezzo riescono a mantenersi orientati.

Considerando un singolo dipolo, la forza elastica di richiamo eguaglia la forza elettrica, che tende a separare le cariche
\begin{equation}
    q E = k x = \omega_0^2 m x
\end{equation}
La risposta è oscillatoria e determina un momento di dipolo di modulo
\begin{equation}
    p = q x = \frac{q^2 E}{\omega_0^2 m}
\end{equation}
Considerando una densità volumica $n$ di dipoli, la polarizzazione sarà
\begin{equation}
    n p = \frac{n q^2 E}{\omega_0^2 m} = \eps_0 \frac{n q^2}{\eps_0 \omega_0^2 m} E = \eps_0 \chi_{e,\text{apolare}} E
\end{equation}

Il coefficiente $A$ è quindi legato alla componente apolare del mezzo.
La componente polare della suscettività elettrica, invece, è quella che dipende dalla temperatura.

\subsection{Esempio di calcolo}

Considerando una sfera carica immersa in un dielettrico, questo schermerà le cariche sulla superficie della sfera orientandovi le cariche negative dei dipoli.
Il campo elettrico all'esterno sarà minore di un fattore $\eps_r$ e
\begin{gather}
    \vt{P} = \frac{\eps_r - 1}{\eps_r} \frac{q}{4\pi r^2} \ver{r} \\
    \implies \sigma\bound = \vt{P} \cdot \uvt{n}
    = -\frac{\eps_r - 1}{\eps_r} \frac{q}{4\pi r^2}
\end{gather}
Nota che $\uvt{n} = - \ver{r}$, cioè entrante, poiché la superficie è dal punto di vista del dielettrico.

Moltiplicando per $4\pi r^2$ si ottiene la carica affacciata sul dielettrico:
\begin{equation}
    q\bound = \sigma\bound 4\pi r^2 = - \frac{\eps_r - 1}{\eps_r} q
\end{equation}

\chapter{Magnetostatica}

\section{Principi della magnetostatica}

\subsection{Campo magnetico}

L'evidenza più basilare è costituita da barrette di magnetite che interagiscono reciprocamente.

Queste, quando affiancate, si comportano esattamente come un dipolo elettrico:
\begin{itemize}
    \item In preseza di un campo elettrico, il dipolo si orienta fino a diventare parallelo.
    \item Due dipoli affiancati si dispongono in direzione opposta.
    \item Due dipoli adiacenti si dispongono in direzione concorde.
\end{itemize}
Tuttavia, mentre un dipolo elettrico genera un campo elettrico, le barrette di magnetite no (non sono cariche nemmeno localmente).

Immaginiamo di riempire lo spazio con molti dipoli piccoli: si allineeranno sulle linee di campo.

La limatura di ferro si dispone, attorno alle barrette, secondo linee di campo simili a quelle di un dipolo elettrico.

C'è, quindi, un'altra proprietà della materia: il magnetismo

Si parla (provvisoriamente) di cariche magnetiche nord e sud e di campo magnetico $\B$ (tangente a queste linee di campo).

Il campo magnetico imprime una forza a cariche in movimento.

\important{Forza di Lorentz}:
\begin{equation}
    \force = q \vt{v} \times \B
\end{equation}

Grazie alla forza di Lorentz si definisce il campo magnetico (unità di misura: tesla $\unit{\tesla} = \unit{\newton\second\per\coulomb\per\metre}$).

Tuttavia, questa legge non definisce la sorgente del campo.
Inoltre, non si può parlare di carica magnetica poiché sperimentalmente non si riescono a isolare le ``cariche'' nord e sud, vanno sempre a coppie.

Poiché risultato di un prodotto vettoriale, $\force \perp \vt{v} \implies P = \force \cdot \vt{v} = 0$: il campo magnetico non svolge lavoro.

Quindi, non varia l'energia cinetica e neanche il modulo della velocità.

I tipi di moto possibili per una particella carica di massa $m$ in presenza di un campo magnetico sono i seguenti:
\begin{itemize}
    \item Se $\vt{v} \perp \B$, il moto è circolare uniforme di raggio $R$:
        \begin{equation}
            F = qvB = m \frac{v^2}{R} \implies R = \frac{mv}{qB}
        \end{equation}
    \item Se $\vt{v} \parallel \B$, il moto è rettilineo uniforme
    \item Ogni altro caso è una combinazione dei precedenti: moto a elica con $R = mv\sin{\alpha}/(qB)$
\end{itemize}

Questo rende possibile la spettrometria di massa (vedi \autoref{sec:spettrometro_massa}).

Inoltre, la Terra ha un campo magnetico e intrappola le particelle cariche che si avvicinano all'atmosfera terrestre, tranne vicino ai poli (nascono così le aurore boreali).

\subsection{Esperimento di Ampère}

La limatura di ferro si distribuisce secondo linee di campo in presenza di un filo conduttore attraverso cui passa corrente.
Se la corrente viene spenta, la limatura torna uniforme.
Quindi, la sorgente del campo magnetico è la carica in moto.

\important{Legge elementare di Ampère}: il campo magnetico generato da una carica $q$ in posizione $\p'$ e in moto con velocità istantanea $\vt{v}$ è
\begin{equation}
    \B(\p) = \frac{\mu_0}{4\pi} \frac{q \vt{v}}{\norm{\p - \p'}^2} \times \ver{\p'}
\end{equation}
con $\ver{\p'} = (\p - \p')/\norm{\p - \p'}$.
$\mu_0$ è detta \important{permeabilità magnetica del vuoto}.

Osserviamo che
\begin{itemize}
    \item La forma di quest'equazione è molto simile a quella della legge di Coulomb.
    \item $\B$ ha smmetria cilindrica rispetto alla retta individuata da $\vt{v}$.
    \item Le linee di campo sono tutte chiuse.
    \item vale la seguente relazione:
        \begin{equation}
            \B = \mu_0 \eps_0 \vt{v} \times \E
        \end{equation}
\end{itemize}

\subsection{Riassunto dei principi}

\begin{itemize}
    \item Forza di Lorentz; è il principio con cui si definisce il campo magnetico e l'equivalente della forza di Coulomb.
    \item Legge elementare di Ampère; il campo magnetico, che agisce su cariche in moto, è generato da cariche in moto.
    \item Principio di sovrapposizione degli effetti.
\end{itemize}

La magnetostatica ha due diverse equazioni su
\begin{itemize}
    \item come il campo è generato, sorgenti (legge elementare di Ampère)
    \item come il campo agisce (legge di Lorentz)
\end{itemize}

Per l'elettrostatica, invece, l'equazione è unica (la legge di Coulomb definisce sia il campo elettrico sia la carica; sia la sorgente del campo sia il suo effetto).


\section{Campo magnetico e fili conduttori}

Detta $n$ la densità volumica di cariche, la densità di carica sarà $\rho = n q$.
Ricordando $\dcurr = \rho \vt{v}$ e considerando che un tratto infinitesimo di filo ha volume $S \, \de l$,
\begin{equation}
\label{eq:corrente_infinitesima}
    \de q \, \vt{v} = \rho \, \de V \, \vt{v} = \dcurr \, \de V =  \pts{\int_S \dcurr \sde} \de \vt{l} = I \de \vt{l}
\end{equation}

\subsection{Campo magnetico generato da una corrente}

A partire dalla legge elementare di Ampère e usando l'\cref{eq:corrente_infinitesima}, si ottiene la \important{Legge di Ampère-Laplace}:
\begin{equation}
\label{eq:ampere_laplace}
    \de\B(\p) = \frac{\mu_0}{4\pi} \frac{I \de \vt{l} \times \ver{\p'}}{\norm{\p -\p'}^2}
    = \frac{\mu_0}{4\pi} \frac{\dcurr \times \ver{\p'}}{\norm{\p -\p'}^2} \de V
\end{equation}

Formato integrale:
\begin{equation}
\label{eq:ampere_laplace_integrale}
    \B(\p) = \frac{\mu_0 I}{4\pi} \!\int_\text{filo} \frac{\de \vt{l}' \times \ver{\p'}}{\norm{\p -\p'}^2}
    = \frac{\mu_0}{4\pi} \!\int_\text{filo} \frac{\dcurr(\p') \times \ver{\p'}}{\norm{\p -\p'}^2} \de V'
\end{equation}
Correla la geometria del filo al campo magnetico generato.

\subsection{Campo magnetico esercitato su un filo}

Riutilizzando l'\cref{eq:corrente_infinitesima},
\begin{equation}
    \de\force = I \de\vt{l} \times \B
\end{equation}

Formato integrale:
\begin{equation}
    \force = I\!\int_\text{filo} \de\vt{l}' \times \B(\p')
\end{equation}

\subsection{Forza esercitata reciprocamente}
\label{sec:def_ampere}

\adddrawio{biot_savart}{0.3}

\important{Legge di Biot-Savart}: considerando un filo rettilineo infinito con corrente $I$, il campo magnetico a una distanza $R$ è
\begin{equation}
\label{eq:biot_savart}
    \B = \frac{\mu_0 I}{2\pi R} \ver{\theta}
\end{equation}
Di nuovo, ``infinito'' significa che $R \ll$ lunghezza del filo.

A distanza $R$ si colloca parallelo un altro filo rettilineo di lunghezza $L'$ e con corrente $I'$.

Per il principio di azione e reazione, la forza che i due fili esercitano l'uno sull'altro è la stessa.

Quella esercitata sul secondo filo è
\begin{equation}
    \force = I'\!\int_\text{filo 2} \de\vt{l} \times \B = I' L' \norm{\B} \ver{R} = \frac{\mu_0 I I' L'}{2\pi R} \ver{R}
\end{equation}
con $\B$ generato dal primo filo e $\ver{R}$ rivolto dal secondo filo verso il primo (la forza è attrattiva).

Poiché è molto semplice avere fili con corrente e invece molto difficile avere cariche nette stabili, storicamente è stato definito prima l'ampere e poi il coulomb.

\qty{1}{\ampere} è, per definizione, la corrente che percorre due fili a distanza di \qty{1}{\metre} e che si attirano con una forza di \qty{2e-7}{\newton} per ogni metro di lunghezza dei fili.

Per cui, $\mu_0 = \qty{4\pi e-7}{\newton\per\coulomb\squared\second\squared}$, e da questo si definiscono $\eps_0$ e il coulomb.

Definendo una tra carica e corrente, si definiscono sia $\mu_0$ sia $\eps_0$, che non sono indipendenti.
(In particolare, risulterà $\mu_0\eps_0 c^2 = 1$.)

\section{Legge di Gauss per il magnetismo}

Partendo dall'\cref{eq:ampere_laplace_integrale} scritta con la densità di corrente, prendiamo la divergenza (rispetto alla variabile $\p$)
\begin{equation}
    \diver \B(\p) = \frac{\mu_0}{4\pi} \!\int_V \diver \frac{\dcurr(\p') \times \ver{\p'}}{\norm{\p - \p'}^2} \de V'
\end{equation}
La divergenza può entrare nell'integrale poiché è in $\p$, mentre l'integrale è in $\p'$.

Si usa la formula
\begin{equation}
    \diver (\vt{A} \times \vt{B}) = \vt{B} \cdot (\curl \vt{A}) - \vt{A} \cdot (\curl \vt{B})
\end{equation}

\begin{equation}
    \diver \B(\p) = \frac{\mu_0}{4\pi} \!\int_V \pts{\frac{\ver{\p'}}{\norm{\p - \p'}^2} \cdot \pts{\curl \dcurr(\p')} - \dcurr(\p') \cdot \pts{\curl \frac{\ver{\p'}}{\norm{\p - \p'}^2}}} \de V'
\end{equation}

Il primo termine è nullo perché $\dcurr$ non è funzione di $\p$, il secondo è anch'esso nullo perché è un campo centrale del tipo $1/r^2$, dunque è conservativo.

Risulta
\begin{equation}
    \diver \B = 0
\end{equation}
Ovvero, $\B$ è \important{solenoidale} e tutte le linee di campo sono chiuse.

Formato integrale tramite il teorema di Gauss:
\begin{equation}
    \oint_S \B \sde = 0
\end{equation}

\subsection{Potenziale vettore}

Il fatto che il campo magnetico sia solenoidale permette di introdurre il potenziale vettore $\vpot$:
\begin{equation}
    \B = \curl \vpot
\end{equation}

È il corrispettivo del potenziale scalare $\spot$, che esprime $\curl \E = \vt{0}$ così come il potenziale vettore esprime $\diver \B = 0$.

$\vpot$ non è unico, ma varia a meno del gradiente di un campo scalare.
Se $\vpot$ è un possibile potenziale vettore, andrebbe bene anche $\vpot + \grad \phi$ per un campo scalare $\phi$ qualsiasi, poiché
\begin{equation}
    \curl (\vpot + \grad \phi) = \curl \vpot = \B
\end{equation}

Vogliamo trovarne un'espressione.

Tramite la seguente relazione (gradiente in $\p$)
\begin{equation}
    - \grad\pts{\frac{1}{\norm{\p - \p'}}} = \frac{\ver{\p'}}{\norm{\p - \p'}^2}
\end{equation}
riscriviamo come segue la legge di Ampère-Laplace:
\begin{equation}
    \B(\p)
    = \frac{\mu_0}{4\pi} \int_V \frac{\dcurr(\p') \times \ver{\p'}}{\norm{\p - \p'}^2} \de V'
    = \frac{\mu_0}{4\pi} \int_V \grad\pts{\frac{1}{\norm{\p - \p'}}} \times \dcurr(\p') \, \de V'
\end{equation}
La perdita del segno meno è compensata dall'aver scambiato i fattori del prodotto vettoriale.

Si usa la seguente indentità del calcolo vettoriale:
\begin{equation}
    \grad f \times \dcurr = \curl (f \dcurr)  - f \, \curl \dcurr
\end{equation}
con $f = 1/{\norm{\p - \p'}}$.
\begin{gather}
    \B(\p) =
    \frac{\mu_0}{4\pi} \int_V \curl \pts{\frac{\dcurr(\p')}{\norm{\p - \p'}}} \de V' -
    \frac{\mu_0}{4\pi} \int_V \frac{1}{\norm{\p - \p'}} \nabla \times \dcurr(\p') \de V' = \\
    = \frac{\mu_0}{4\pi} \curl \int_V \frac{\dcurr(\p')}{\norm{\p - \p'}} \de V'
\end{gather}

Abbiamo espresso $\B$ come rotore di un campo vettoriale, che quindi è $\vpot$:
\begin{equation}
\label{eq:potenziale_vettore}
    \vpot(\p) = \frac{\mu_0}{4\pi} \int_V \frac{\dcurr(\p')}{\norm{\p - \p'}} \de V'
    % = \mu_0 \slp \dcurr(\p)
\end{equation}

Si ricorda l'analoga espressione per il potenziale scalare:
\begin{equation}
    \spot(r) = \frac{1}{4\pi \eps_0} \int_V \frac{\rho(\p')}{\norm{\p - \p'}} \de V'
    % = \frac{1}{\eps_0} \slp \rho(\p)
\end{equation}
Per analogia, quindi, anche per ogni componente del potenziale vettore vale l'equazione di Poisson:
\begin{gather}
    \vlapl \vpot(\p) = - \mu_0 \dcurr(\p) \\
    \lapl \spot(\p) = - \frac{1}{\eps_0} \rho(\p)
\end{gather}
Si tratta complessivamente di quattro equazioni differenziali scalari.

\section{Legge di Ampère}

Calcoliamo il rotore di $\B$:
\begin{equation}
    \curl \B = \curl \curl \vpot = - \vlapl \vpot + \cancel{\grad (\diver \vpot)} = \mu_0 \dcurr
\end{equation}
Infatti, definire $\vpot$ come nell'\cref{eq:potenziale_vettore} implica che $\vpot$ è solenoidale.

Formato integrale (tramite il teorema di Stokes):
\begin{equation}
\label{eq:legge_di_ampere}
    \oint_\gamma \B \lde = \mu_0 I
\end{equation}
dove $I$ è la \textbf{corrente concatenata a $\gamma$}, ovvero la corrente attraverso una qualunque superficie $S$ che abbia bordo $\gamma$, orientata secondo il verso di percorrenza di $\gamma$.

Simmetria tra campi e sorgenti:
\begin{itemize}
    \item divergenze: $\rho$ sorgente, quindi è nulla quella di $\B$
    \item rotori: $\dcurr$ sorgente, quindi nullo quello di $\E$
    \item la sorgente minima del campo elettrico è la carica
    \item la sorgente minima del campo magnetico è un dipolo $\iff$ non esistono monopoli magnetici (è un'evidenza sperimentale, nella teoria sarebbero possibili)
\end{itemize}

\section{Esempi di calcolo}

\subsection{Filo rettilineo}

Per simmetria le linee di campo sono circonferenze centrate nel filo; lo si vede anche dal prodotto vettoriale nella legge di Ampère-Laplace.

Si ha simmetria cilindrica e $B = \norm{\B}$ è costante se il punto dista $R$ dal filo

\begin{equation}
    \oint_\text{circ. $R$} \B \lde
    = \oint_\text{circ. $R$} B \de l
    = 2\pi R B
    = \mu_0 I \implies B
    = \frac{\mu_0 I}{2\pi R}
\end{equation}

È una conferma della legge di Biot-Savart (\cref{eq:biot_savart}).

\subsection{Solenoide}

= filo ``arrotolato'' attorno a un cilindro.

Consideriamolo di lunghezza $L$ e formato da $N$ giri del filo.

Se $L \gg$ del diametro, il campo magnetico è nullo all'esterno.

Si applica la legge di Ampère su un a sezione rettangolare che interseca la superficie del cilindro.

La curva è $\gamma$ e il lato parallelo all'asse del solenoide è $h \ll L$.

\begin{equation}
    \oint_\gamma \B \lde = B(R) h = \mu_0 \pts{h \frac{N}{L} I} \implies B = \mu_0 \frac{N}{L} I
\end{equation}

I contributi dei lati che attraversano il cilindro si elidono per simmetria e il lato esterno dà contributo zero.

\section{Riassunto: elettrostatica e magnetostatica}

Dagli esperimenti alle leggi:
\begin{itemize}
    \item evidenze sperimentali
    \item principi
    \item legge in formato integrale (valgono complessivamente rispetto a un volume o superficie), sono quelle che si ottengono con esperimenti
    \item legge in formato differenziale (valgono punto per punto), molto più semplici da trattare
\end{itemize}

Leggi sul campo elettrico:
\begin{itemize}
    \item legge di Gauss: $\diver \E = \rho/\eps_0$
    \item conservatività: $\curl \E = \vt{0}$
    \begin{itemize}
        \item potenziale scalare: $\E = -\grad \spot$
    \end{itemize}
\end{itemize}
Riassunte nell'equazione di Poisson: $\lapl \spot = -\rho/\eps_0$.

Leggi sul campo magnetico:
\begin{itemize}
    \item legge di Gauss: $\diver \B = 0$
    \begin{itemize}
        \item potenziale vettore: $\B = \curl \vpot$
    \end{itemize}
    \item legge di Ampère: $\curl \B = \mu_0 \dcurr$
\end{itemize}
Riassunte nell'equazione $\vlapl \vpot = -\mu_0 \dcurr$.

Parlando della sorgente del campo magnetico, abbiamo trascurato il campo elettrico generato dalle cariche in moto poiché i fili conduttori sono complessivamente neutri.

Un'altra osservazione da sospendere riguarda le contraddizioni rispetto alla relatività galileiana: il campo magnetico misurato in un sistema in quiete con una carica è nullo, ma non lo è in un sistema (sempre inerziale) in moto rispetto al primo. Risolveremo queste contraddizioni nel \autoref{sec:relativita}.

\chapter{Magnetostatica -- esercitazioni}

\section{Dispositivi con campo magnetico}

Scomponendo $\vt{v}$ nelle componenti $\vt{v}_\parallel \parallel \B$ e $\vt{v}_\perp \perp \B$,
\begin{equation}
    \force = q\vt{v} \times \B, \qquad
    F = \abs{q}vB \sin\theta = \abs{q} v_\perp B
\end{equation}

Il moto di una particella carica nel piano $\perp \B$ è circolare uniforme:
\begin{subequations}
\begin{gather}
    r = \frac{m v_\perp}{\abs{q}B} \\
    \omega = \frac{v_\perp}{r} = \frac{\abs{q} B}{m} \\
    T = \frac{2\pi}{\omega} = \frac{2\pi m}{\abs{q} B}
\end{gather}
\end{subequations}
Nella direzione $\parallel \B$, il moto è rettilineo uniforme di velocità $v_\parallel$.

Considerando entrambi i moti, il passo dell'elica è $v_\parallel T$.



\subsection{Selettore di velocità}

Consideriamo una sorgente di ioni in varie velocità.
% Facendo passare le particelle attraverso un condensatore a facce piane parallele perpendicolari alla velocità delle particelle, è possibile accelerarle.
Impostiamo una regione di spazio con campo elettrico e magnetico perpendicolari tra loro e alla velocità.
Le uniche particelle ad attraversare completamente questa regione saranno quelle che non deviano, ovvero per cui
\begin{equation}
    \force = q(\E + \vt{v} \times \B) = \vt{0}
    \implies \vt{v} \times \B = - \E
\end{equation}
In modulo e supponendo $\vt{v} \parallel \B$,
\begin{equation}
    vB = E \implies v = \frac{E}{B}
\end{equation}
$\B$ è probabilmente fornito da un magnete, quindi è difficile da modificare.
Per questo, si opera su $\E$ tramite un condensatore a facce piane parallele.

Esempio: vogliamo selezionare il catione argon \ce{Ar+}, che ha una massa $m = \qty{39.9}{\dalton}$ e una carica pari alla carica elementare.

Accelerando gli ioni con una differenza di potenziale $\Delta V = \qty{e3}{\volt}$, la velocità di \ce{Ar+} risulta
\begin{equation}
    v = \sqrt{\frac{2q\Delta V}{m}} = \qty{7e4}{\metre\per\second}
\end{equation}
Con $B = \qty{0.1}{\tesla}$, occorre impostare $E = vB = \qty{7e3}{\volt\per\metre}$.

\subsection{Spettrometro di massa}
\label{sec:spettrometro_massa}

È possibile misurare le masse degli ioni, nota la carica, misurando il raggio della semicirconferenza che percorrono in una regione con campo magnetico.

Se la velocità $v$ è dovuta a una d.d.p. $\Delta V$,
\begin{equation}
    v = \sqrt{\frac{2 \abs{q} \Delta V}{m}}
    \quad \implies \quad
    r = \frac{mv}{\abs{q}B} = \frac{1}{B}\sqrt{\frac{2 m \Delta V}{\abs{q}}}
\end{equation}

\adddrawio{mass_spectrometry}{0.7}

Consideriamo argon-40 e argon-38 ionizzati, $\Delta V = \qty{1}{\kilo\volt}$, $B = \qty{0.1}{\tesla}$:
\begin{equation}
    r_{40} = \qty{28.8}{\centi\metre}, \quad r_{38} = \qty{28.0}{\centi\metre}
\end{equation}
La distanza tra le due misure sul rilevatore è $2 \abs{r_{40} - r_{38}} = \qty{1.6}{\centi\metre}$.

In realtà, lo spettrometro di massa dà informazioni sul rapporto carica/massa, quindi bisogna conoscere la carica (spesso dovuta a ionizzazione singola).


\subsection{Ciclotrone}

\addfigure[Da \shorturl{commons.wikimedia.org/wiki/File:Ciclotrone.png}.]{wikimedia/ciclotrone}{0.8}

Permette di accelerare molto le particelle.

Due armature a semicerchio (``semicilindro'') pieno di raggio $R$ con una differenza di potenziale e un campo elettrico esterno.
Uno ione che parte dalla superficie di una delle armature accelera verso l'altra, con la stessa velocità descrive una semicirconferenza e accelera ulteriormente verso la prima; lì, descrive una semicirconferenza di raggio maggiore, e così via fino a un'uscita.

La differenza di potenziale tra le armature deve invertirsi ogni mezzo giro.
Questo è facile, poiché il tempo di percorrenza in un'armatura è metà del periodo del moto circolare e non dipende dalla velocità:
\begin{equation}
    t = \frac{T}{2} = \frac{\pi m}{\abs{q} B}
\end{equation}
È sufficiente
\begin{equation}
    V(t) = V_0 \sin(\omega_{RF} t), \quad \omega_{RF} = \frac{\abs{q}B}{m}
\end{equation}
$\omega_{RF}$ si chiama \important{frequenza ciclotronica}.

La velocità massima di uscita è limtiata dalla dimensione del ciclotrone:
\begin{gather}
    v_\mathrm{max} = \frac{\abs{q}B R}{m} \\
    K_\mathrm{max} = \frac{q^2 B^2 R^2}{2m}
\end{gather}

Il raggio dei ciclotroni più grandi non misura più di qualche metro, poiché è difficile imporre $B$ esattamente uniforme in un'area grande.

Esempio: un protone con $R = \qty{1}{\metre}$ e $B = \qty{1}{\tesla}$ (generato, ad esempio, con bobine):
\begin{equation}
    v_\mathrm{max} \approx \qty{e8}{\metre\per\second}, \quad K_\mathrm{max} \approx \qty{e-11}{\joule} \approx \qty{50}{\mega\electronvolt}
\end{equation}
In realtà, poiché la velocità è relativistica, in questo caso questo il calcolo risulta sbagliato.





\section{Campo magnetico generato da corrente}

Legge di Ampère-Laplace:
\begin{subequations}
\begin{gather}
    \de \B(\p) = \frac{\mu_0}{4\pi} \frac{I \de \vt{l}' \times \ver{r'}}{\norm{\p - \p'}^2} \\
    \B(\p) = \frac{\mu_0 I}{4\pi} \int_\mathrm{filo} \frac{\de \vt{l}' \times \ver{r'}}{\norm{\p - \p'}^2}
\end{gather}
\end{subequations}

\subsection{Spira circolare}

\adddrawio{bfield_ring}{0.8}

Spira circolare di raggio $R$ sul piano $yz$ con centro nell'origine percorsa da corrente $I$ nel verso concorde con l'asse $x$.
Calcoliamo il campo magnetico $\B(x, 0, 0)$.

Per simmetria, $\B(x, 0, 0) \parallel \ux$, cioè $\B(x, 0, 0) = B_x(x, 0, 0)\ux$.

Usando $\de \vt{l} \perp \ver{r'}$,

\begin{equation}
\begin{gathered}
\label{eq:campo_spira}
    B_x(x, 0, 0) = \frac{\mu_0}{4\pi} \int_\mathrm{anello} \frac{\norm{I \de \vt{l} \times \ver{r'}} \cos \theta}{\norm{\p - \p'}^2}
    = \frac{\mu_0 I}{4\pi} \int_\mathrm{anello} \frac{\de l \frac{R}{\sqrt{x^2 + R^2}}}{x^2 + R^2} = \\
    = \frac{\mu_0 I}{4\pi} \frac{R}{\pts{x^2 + R^2}^{3/2}} \int_\mathrm{anello} \de l
    = \frac{\mu_0 I}{4\pi} \frac{R}{\pts{x^2 + R^2}^{3/2}} 2 \pi R
    = \frac{\mu_0 I}{2} \frac{R^2}{\pts{x^2 + R^2}^{3/2}}
\end{gathered}
\end{equation}

La funzione è pari in $x$, quindi $\B$ sull'asse $x$ è sempre diretto nel verso delle $x$ positive.
$B_x(x, 0, 0)$ è massimo per $x = 0$, al centro della spira:
\begin{equation}
    \B(\vt{0}) = \frac{\mu_0 I}{2 R} \ux
\end{equation}



\subsection{Solenoide}

Solenoide di raggio $R$ lungo $d$ con $N$ spire, cioè con una densità di spire $n = N/d$, e percorso da corrente $I$.
È centrato nell'origine e orientato lungo l'asse $x$

Dal momento che si tratta di spire affiancate, il campo magnetico sull'asse $x$ sarà parallelo all'asse $x$. Detta $r' = \sqrt{(x - x')^2 + R^2}$, sarà
\begin{equation}
    \B_\text{spira in $x'$}(x, 0, 0) = \frac{\mu_0 I R^2}{2 (r')^3} \ux
\end{equation}
In termini di densità,
\begin{equation}
    \de\B(x, 0, 0) = \frac{\mu_0 I R^2}{2 (r')^3} n \, \de x' \, \ux
\end{equation}

\adddrawio{solenoid}{0.9}

Si opera il cambio di variabile da $x'$ a $\phi$ (l'angolo tra il segmento di lunghezza $r'$ e l'asse $x$):
\begin{gather}
    x - x' = R \cot\phi \implies \de x' = \frac{R}{\sin^2\phi} \, \de \phi \\
    r' = \sqrt{(x - x')^2 + R^2} = \sqrt{R^2 \cot^2\phi + R^2} = \frac{R}{\sin\phi}
\end{gather}

Per cui
\begin{gather}
    \de\B(x, 0, 0) = \frac{\mu_0 n I}{2} \sin\phi \, \de \phi \, \ux \\
\begin{gathered}
    \implies \B(x, 0, 0) = \int_\mathrm{solenoide} \de\B(x, 0, 0)
    = \frac{\mu_0 n I}{2} \int_{\phi_1}^{\phi_2} \sin\phi \, \de \phi \, \ux = \\
    = \frac{\mu_0 n I}{2} \pts{\cos\phi_1 - \cos\phi_2} \ux
    = \frac{\mu_0 n I}{2} \pts{
        \frac{\frac{d}{2} + x}{\sqrt{\pts{\frac{d}{2}+x}^2 + R^2}}
        + \frac{\frac{d}{2} - x}{\sqrt{\pts{\frac{d}{2}-x}^2 + R^2}}
    } \ux
\end{gathered}
\end{gather}

Per $x = 0$ si ha il campo massimo:
\begin{equation}
    \B(\vt{0}) = \mu_0 n I \frac{\frac{d}{2}}{\sqrt{\pts{\frac{d}{2}}^2 + R^2}} \ux
\end{equation}

Per $R \ll d$ si ha il limite del \important{solenoide ideale}:
\begin{equation}
    \B(\p) = \mu_0 n I \ux, \quad \text{$\p$ all'interno del solenoide}
\end{equation}

Meno il solenoide è ideale, più è forte il campo all'esterno del solenoide lungo l'asse $x$.
Nei solenoidi reali, $\B$ ha carattere ideale soprattutto al centro (e solo se vicino all'asse del solenoide).

Nei solenoidi ideali le linee di campo sono rette parallele, quindi $\B = \vt{0}$ all'esterno.

Nei solenoidi reali, il campo all'esterno è scarso ma presente.



\section{Forza magnetica su fili}

\begin{subequations}
\begin{gather}
    \force = I \!\int \de \vt{l} \times \B \\
    \de \force = I \de \vt{l} \times \B
\end{gather}
\end{subequations}

Se $\de \vt{l} \perp \B$,
\begin{equation}
    \force = I \vt{L} \times \B
    \implies
    F = I L B
\end{equation}


\subsection{Dinamometro a bilancia}

\addfigure{book/dinamometro_a_bilancia}{0.4}

Bilancia con massa su un piatto e un circuito presso l'altro braccio.

Il circuito consiste in una spira rigida per cui passa una corrente nota $I$ in presenza di un campo magnetico uniforme perpendicolare.
Poiché i due lati verticali danno contributi opposti, la forza risultante sulla spira è pari a quella sul solo lato orizzontale: $F = ILB$, e il verso della corrente è scelto in modo che la forza sia verso il basso.

È quindi possibile misurare la massa con grande precisione (e tarare lo strumento, nota la massa, con grande precisione).

Esempio: se $m = \qty{0.5}{\gram}$, $I = \qty{1}{\ampere}$ e $L = \qty{5}{\centi\metre}$, allora $B = \qty{0.1}{\tesla}$.

\subsection{Spira in un campo magnetico}

\addfigure{book/spira_semicircolare}{0.4}

Spira che segue il perimetro di un semicerchio di raggio $R$ poggiato sull'asse $x$ e nel piano $xy$, in presenza di un campo magnetico $\B = B \uy$.
Una corrente $I$ scorre in verso antiorario. Siano $P$ e $Q$ gli estremi del lato orizzontale.

La forza sul lato orizzontale è
\begin{equation}
    I \!\int_P^Q \de \vt{l} \times \B
    = I \!\int_P^Q \de l \ux \times B \uy
    = I B \!\int_P^Q \de l \, \uz
    = I B \!\int_{-R}^R \de x \, \uz
    = 2IBR \uz
\end{equation}

La forza sulla semicirconferenza è
\begin{equation}
\begin{gathered}
    I \!\int_Q^P \de \vt{l} \times \B
    = I \!\int_Q^P \de l \ver{\theta} \times B \uy
    = -IB \int_Q^P \sin\theta dl \, \uz = \\
    = -IB \int_0^\pi \sin\theta R \de \theta \, \uz
    = -2IBR \uz
\end{gathered}
\end{equation}

Il modulo è lo stesso.
Infatti, la forza dipende solo dalla distanza tra gli estremi del filo nella direzione perpendicolare a $\B$.

Ogni circuito chiuso percorso da corrente sente una forza totale $\force = \vt{0}$.
Tuttavia, sente una coppia di forze.
Considerando una spira rettangolare di lati $a$ e $b$ percorsa da corrente nel verso $ABCD$, con $\overline{AB} = b$ e $\overline{AD} = a$, per cui la forza netta sia sentita dai lati lunghi $a$,
\begin{equation}
    \vt{M} = \vt{b} \times \force_{AD} = \vt{b} \times \pts{I\vt{a}  \times \B}
    = I \vt{S} \times \B
\end{equation}
$\vt{S}$ è il vettore superficie (normale a essa).

Il risultato è generale e vale per ogni circuito piano chiuso in un campo magnetico uniforme.

Si definisce il momento di dipolo magnetico:
\begin{equation}
    \vt{m} = I \vt{S}
\end{equation}
Quindi
\begin{equation}
    \vt{M} = \int_S \de\vt{m} \times \B
\end{equation}

Per l'\cref{eq:campo_spira}, il campo magnetico di una spira con $\vt{m} \parallel \ux$, per $x \gg R$, è
\begin{equation}
    \B(x, 0, 0) = \frac{\mu_0 I R^2}{2 x^3} \ux = \frac{\mu_0}{4\pi} \frac{2\vt{m}}{x^3}
\end{equation}

Si noti l'analogia con il campo elettrico di un dipolo elettrico con $\vt{p} \parallel \ux$ (vedi \cref{eq:campo_dipolo_elettrico}):
\begin{equation}
    \E(x, 0, 0) = \frac{1}{4\pi\eps_0} \frac{2\vt{p}}{x^3}
\end{equation}

Energia potenziale:
\begin{equation}
    U = - \vt{m} \cdot \B = - I \vt{S} \cdot \B = -I \Phi(\B)
\end{equation}

Oltre a queste analogie con il dipolo elettrico, le linee di campo tendono a diventare identiche se i dipoli tendono a essere ideali.


\subsection{Dipolo magnetico}

\adddrawio{magnetic_dipole}{0.6}

Rappresentiamo il dipolo magnetico come una spira quadrata con corrente $I$, lato $L$ e area $S = L^2$ che giace sul piano $xy$ nell'origine.

Potenziale vettore:
\begin{equation}
    \vpot = \frac{\mu_0}{4\pi} \int_V \frac{\dcurr(\p')}{\norm{\p - \p'}} \de V'
\end{equation}
La componente $\alpha \in \set{x, y, z}$ è
\begin{equation}
    A_\alpha(\p) = \frac{\mu_0}{4\pi} \int_V \frac{J_\alpha(\p')}{\norm{\p - \p'}} \de V'
\end{equation}

Se $\p$ è nel piano $yz$, ad esempio, i contributi dovuti ai lati $\parallel \uy$ sono uguali in modulo e si elidono.
Anche la componente $z$ è nulla, poiché $\dcurr$ giace su $xy$.
\begin{equation}
    \vpot(\p) = A_x(\p) \ux, \quad \text{se $\p \in$ piano $yz$}
\end{equation}

Con $r = \norm{\p - \p'} \gg L$,
\begin{equation}
\begin{gathered}
    A_x(\p) = \frac{\mu_0}{4\pi} \int_V \frac{J_x(\p')}{\norm{\p - \p'}} \de V'
    = \frac{\mu_0}{4\pi} \int_V \pts{\frac{J_x(\p')}{r_+} - \frac{J_x(\p')}{r_-}} \de V' = \\
    = \frac{\mu_0}{4\pi} \pts{\frac{J_x}{r_+} - \frac{J_x}{r_-}} SL
    = \frac{\mu_0}{4\pi} J_x S L \frac{r_- - r_+}{r_- r_+}
    = \frac{\mu_0}{4\pi} I L \frac{r_- - r_+}{r_- r_+}
\end{gathered}
\end{equation}

Detto $\theta$ l'angolo tra $\p$ e $\uz$, $r_+ - r_- \approx L \sin \theta$.
Inoltre $r_- r_+ \approx r^2$

\begin{gather}
    A_x = -\frac{\mu_0}{4\pi} I L^2 \frac{\sin\theta}{r^2}
\end{gather}

Ricordando $\vt{m} = IL^2 \uz$ e $\vt{m} \times \ver{r} = -IL^2 \sin\theta \ux$,

\begin{gather}
    \vpot(\p) = \frac{\mu_0}{4\pi} \frac{\vt{m} \times \ver{r}}{r^2}
\end{gather}

Questo vuol dire che a grande distanza la forma del circuito non ha importanza, è sufficiente conoscere il momento magnetico.

Il campo magnetico sarà
\begin{equation}
\begin{gathered}
    \B(\p) = \curl \vpot(\p) = \frac{\mu_0}{4\pi} \curl \pts{\vt{m} \times \frac{\ver{r}}{r^2}}
    % = \frac{\mu_0}{4\pi} \pts{\vt{m} \pts{\diver \frac{\ver{r}}{r^2}} - \vt{m} \cdot \grad \frac{\ver{r}}{r^2}} = \\
    = \frac{\mu_0}{4\pi} I S \curl \pts{\uz \times \frac{\p}{r^3}} = \\
    = \frac{\mu_0}{4\pi} I S \curl \frac{-y\ux + x\uy}{r^3}
\end{gathered}
\end{equation}

Ci concentriamo ora sul rotore.
Si nota che, per $\alpha \in \{x, y, z\}$,
\begin{equation}
    \parder{\alpha} \frac{1}{r^3} = -3 \frac{\alpha}{r^5}
\end{equation}

Allora,
\begin{equation}
\begin{gathered}
    \curl \pts{-\frac{y}{r^3} \ux + \frac{x}{r^3}\uy}
    = \begin{vmatrix}
        \ux & \uy & \uz \\
        \parder{x} & \parder{y} & \parder{z} \\
        -\frac{y}{r^3} & \frac{x}{r^3} & 0
    \end{vmatrix} = \\
    = \frac{3 x z}{r^5} \ux + \frac{3 y z}{r^5} \uy + \pts{\frac{2}{r^3} - \frac{3 x^2}{r^5} - \frac{3 y^2}{r^5}} \uz = \\
    = \frac{1}{r^3} \pts{
        3 \frac{x}{r} \frac{z}{r} \ux +
        3 \frac{y}{r} \frac{z}{r} \uy +
        2 \uz
        - 3 \frac{x^2}{r^2} \uz
        - 3 \frac{y^2}{r^2} \uz
        - 3 \frac{z^2}{r^2} \uz
        + 3 \frac{z^2}{r^2} \uz
    } = \\
    = \frac{1}{r^3} \pts{
        3 \frac{z}{r} \pts{
            \frac{x}{r} \ux +
            \frac{y}{r} \uy +
            \frac{z}{r} \uz
        } - \uz
    } = \\
    = \frac{1}{r^3} \pts{3 \frac{z}{r} \ver{r} - \uz}
    = \frac{3 \pts{\uz \cdot \ver{r}} \ver{r} - \uz}{r^3}
\end{gathered}
\end{equation}

Pertanto, ricordando $\vt{m} = I S \uz$,
\begin{equation}
    \B(\p) = \frac{\mu_0}{4\pi} I S \frac{3 \pts{\uz \cdot \ver{r}} \ver{r} - \uz}{r^3}
    = \frac{\mu_0}{4\pi} \frac{3 \pts{\vt{m} \cdot \ver{r}} \ver{r} - \vt{m}}{r^3}
\end{equation}

La formula è analoga a quella per il campo elettrico generato da un dipolo elettrico:
\begin{equation}
    \E(\p) = \frac{1}{4\pi\eps_0} \frac{3 (\vt{p} \cdot \ver{r}) \ver{r} - \vt{p}}{r^3}
\end{equation}



\section{Circuitazione del campo magnetico}

Legge di Ampère-Laplace:
\begin{equation}
    \de \B(\p) = \frac{\mu_0}{4\pi} \frac{I \de \vt{l}' \times \ver{r'}}{\norm{\p - \p'}^2}
\end{equation}
Ricorda l'\cref{eq:corrente_infinitesima}:
\begin{equation}
    \dcurr = n q \vt{v} = \rho \vt{v}, \quad \de Q \, \vt{v} = \dcurr \de V = I \de\vt{l}
\end{equation}
con $n$ densità volumica di particelle di carica $q$.

La circuitazione del campo magnetico in una circonferenza intorno a un filo rettilineo per cui passa corrente $I$ è
\begin{equation}
    \oint_\gamma \B \lde = \oint_\gamma \frac{\mu_0 I}{2\pi R} \ver{\theta} \lde = \frac{\mu_0 I}{2\pi R} \oint_\gamma \de l = \mu_0 I
\end{equation}
Il primo passaggio è la \important{legge di Biot-Savart}.

In realtà, il risultato è generale $\to$ \important{legge di Ampère}:
\begin{equation}
    \oint_{\partial S} \B \lde = \mu_0 I
\end{equation}
dove $I$ è la corrente totale attraverso la superficie aperta $S$ (cioè la corrente concatenata alla curva chiusa $\partial S$).

In forma differenziale:
\begin{equation}
    \curl \B = \mu_0 \dcurr
\end{equation}

\subsection{Campo magnetico all'interno di un filo conduttore}

Filo rettilineo infinito di raggio $R$ e corrente $I$, cosicché $I = \pi R^2\dcurr$.

Si applica la legge di Ampère per una circonferenza perpendicolare al filo di raggio $r < R$.
Sia $\B(\p) = B(r) \ver{\theta}$.

\begin{equation}
    \oint_\gamma \B \lde = 2\pi r B(r)
    = \mu_0 \pi r^2 \dcurr
    = \mu_0 \frac{r^2}{R^2} I
    \implies B(r) = \frac{\mu_0 I}{2\pi R^2} r
\end{equation}

Ovvero, $B(r) \propto r$ per $r < R$ e $\propto 1/r$ per $r > R$.

\subsection{Cavo coassiale}

\adddrawio{coaxial_cable}{0.4}

Usato per connessioni, perché permette di isolare i campi elettromagnetici.

Cavo di raggio $R_2$ sulla cui superficie esterna scorre una corrente $-I$ e con un cavo interno di raggio $R_1 < R_2$ sulla cui superficie scorre una corrente $I$.

Poiché la corrente concatenata è nulla, $\B = \vt{0}$ sia all'esterno che all'interno del cavo più interno.

Tra i due cilindri, $B(r) = \frac{\mu_0 I}{2\pi r}$.

In sintesi:
\begin{equation}
    B(r) = \frac{\mu_0 I}{2\pi r} \, [R_1 < r < R_2]
\end{equation}

\subsection{Solenoide toroidale}

È un solenoide chiuso a toro di raggio interno $a$ e raggio esterno $b$.
Ha $N$ spire, ciascuna con corrente $I$.

\begin{equation}
    B(r) = \frac{\mu_0 N I}{2\pi r} \, [a < r < b]
\end{equation}
Se $b - a \ll a$, allora $a \approx b$ e il tratto non nullo si può approssimare costante.

\section{Magnetismo nei materiali}

In presenza di un campo magnetico esterno, i dipoli magnetici si allineano concordemente.

\important{Magnetizzazione}:
\begin{equation}
    \vt{M} = n\vt{m}
\end{equation}
con $n$ densità volumica di dipoli magnetici.

Momento magnetico totale (considerando un cilindro):
\begin{equation}
    \int_V \vt{M} \de V = \vt{M} S l
\end{equation}
$\vt{M} l$ corrisponde a una corrente che scorre attorno al cilindro seguendo la regola della mano destra rispetto al verso di $\vt{M}$, la \important{corrente di magnetizzazione}:
\begin{equation}
    \vt{I}_\text{mag} = l\vt{M} \times \uvt{n}, \quad
    I_\text{mag} = M l
\end{equation}

La corrente totale sarà
\begin{equation}
    I_\text{tot} = I\free + I_\text{mag}
\end{equation}
Immaginiamo di avvolgere il cilindro in un solenoide con densità lineare di spire $n$.
La corrente per unità di lunghezza sarà
\begin{gather}
    \frac{I_\text{tot}}{l} = n I\free + M \\
    \implies B = \mu_0(nI\free + M)
\end{gather}

\important{Campo magnetizzante}:
\begin{equation}
    \H \coloneq \frac{1}{\mu_0} \B_0 = \frac{1}{\mu_0} \B - \vt{M}, \quad \norm{\H} = n I\free
\end{equation}
cosicché
\begin{equation}
    \B = \mu_0 (\H + \vt{M}) = \B_0 + \mu_0\vt{M}
\end{equation}

Magnetizzazione e campo magnetizzante sono collegati dalla \important{suscettività magnetica} $\chi_m$:
\begin{gather}
    \vt{M} = \chi_m \vt{H} \\
    \B = \mu_0 (\H + \vt{M}) = \mu_0 \underbrace{(1 + \chi_m)}_{\mu_r} \H = \mu \H
\end{gather}
$\mu_r$ è la \important{permeabilità magnetica relativa}, $\mu = \mu_r \mu_0$ la permeabilità magnetica nel mezzo.

Se $\mu_r \ge 1$, allora $\chi_m \ge 0$.

\important{Legge di Curie}:
\begin{equation}
    \chi_m \propto \frac{1}{T}
\end{equation}

Circuitazioni:
\begin{gather}
    \frac{1}{\mu_0} \oint_\gamma \B \lde = I_\text{tot} = I\free + I_\text{mag} \\
    \oint_\gamma \H \lde = I\free \\
    \oint_\gamma \vt{M} \lde = I_\text{mag}
\end{gather}
Ma anche
\begin{gather}
    \oint_\gamma \B \lde = \mu I\free \\
    \oint_\gamma \vt{M} \lde = \chi_m \oint_\gamma \H \lde = \chi_m I\free
\end{gather}

Formato differenziale:
\begin{gather}
    \curl \B = \mu_0 \dcurr_\text{tot} = \mu \dcurr\free \\
    \curl \H = \dcurr\free \\
    \curl \vt{M} = \dcurr_\text{mag} = \chi_m \dcurr\free
\end{gather}

\subsection{Tipi di materiali}

\begin{table}[!h]
\centering
\begin{tabular}{|c|c|c|c|}
\hline
Materiali & $\chi_m$ & $\mu_r$ & Esempi \\
\hline
diamagnetici & $< 0$ & $\approx 1$ & molti materiali \\
% \hline
paramagnetici & $> 0$ & $\approx 1$ & \ce{O2}, alcuni metalli \\
% \hline
ferromagnetici & $\gg 0$ & $\gg 1$ & \ce{Fe}, \ce{Ni}, \ce{Co} \\
% \hline
antiferromagnetici & $= 0$ & $= 1$ & {FeO}, \ce{MnO}, \ce{CoO} \\
% \hline
ferrimagnetici & $> 0$ & $> 1$ & alcuni ossidi complessi\\
\hline
\end{tabular}
\caption{Tipi di materiali magnetici}
\label{tab:materiali_magnetici}
\end{table}

La risposta dei materiali diamagnetici, $\mu_r < 1$ di poco, è quella generica che si ottiene con ogni elettrone (che è una carica negativa).
$\B$ è meno intenso di $\B_0$.

Anche i paramagneti hanno normalmente $\vt{M} = \vt{0}$, ma in presenza di un campo magnetico esterno i dipoli si orientano, per cui $\mu_r > 1$ di poco.
$\B$ è più intenso di $\B_0$.

Nei ferromagneti, esistono delle regioni dette $\important{domini}$ in cui i dipoli sono allineati, per cui $\vt{M} \ne \vt{0}$.
In presenza di $\B_0$ i domini si orientano e $\B$ risulta molto più intenso.
Talvolta generano autonomamente un campo magnetico.

Gli antiferromagneti hanno domini con dipoli orientati in senso opposto, per cui $\vt{M} = \vt{0}$.

I materiali ferrimagnetici combinano effetti ferromagnetici (più intensi) e antiferromagnetici (più debili).

Le calamite si creano spegnendo $\H$ dopo aver raggiunto una forte magnetizzazione.
Per distruggere una calamita, l'unica possibilità è scaldarla in modo da disordinare i dipoli.

\addfigure[\important{Ciclo di isteresi} con variabili $\H$ e $\vt{M}$. Da \href{https://commons.wikimedia.org/wiki/File:Isteresi.png}{TFra6}, \href{https://creativecommons.org/licenses/by-sa/4.0}{CC BY-SA 4.0}, tramite Wikimedia Commons.]{wikimedia/isteresi}{0.6}

\chapter{Campi variabili nel tempo}

\section{Induzione elettromagnetica}

Campo elettrico e magnetico devono essere campi correlati, dal momento che dipendono dai sistemi di riferimento (cioè, se le cariche siano in moto o no).
Inoltre, non sono coerenti con la relatività galileiana.

Al tempo in cui si cercava il legame tra i due campi, Faraday si accorse che, quando la corrente in un circuito veniva accesa o spenta, si rilevava una corrente attraverso un secondo circuito vicino, quindi l'esistenza di un campo elettrico.

Se la corrente $I(t)$ varia nel tempo, allora anche $\B \propto I(t)$ varia nel tempo.

Il risultato di innumerevoli esperimenti successivi svolti da Faraday è il seguente:

Consideriamo un filo conduttore chiuso su se stesso (quindi, non percorso da corrente) in presenza di un campo magnetico $\B(t)$.
Si misura una corrente attraverso il filo, e poiché $V = RI$ è come se ci fosse una differenza di potenziale dovuta a una forza elettromotrice (f.e.m.)\ indotta $\fem_i$, definita come
\begin{equation}
    \fem_i \coloneq \oint_\gamma \E \lde
\end{equation}
Non si mette il segno meno davanti all'integrale (diversamente che nell'\cref{eq:spot_integrale}), in modo che la circuitazione sia positiva se il campo è nella direzione della corrente.

Consideriamo una qualsiasi superficie aperta $S \subset \R^3$ che ha il filo chiuso $\gamma$ come bordo e il verso positivo della corrente coerente con la direzione di $\uvt{n}$.

\important{Legge di induzione elettromagnetica}, o \important{legge di Faraday-Henry-Lentz} (formato integrale):
\begin{equation}
    \fem_i = - \der{t} \Phi_S(\B)
\end{equation}
È una legge sperimentale, quindi un principio.
Inoltre, vale per ogni curva $\gamma$ nel vuoto, non è necessario che sia un reale filo conduttore.

Riscrittura del formato integrale e formato differenziale (tramite il teorema di Stokes):
\begin{gather}
    \oint_{\partial S} \E \lde = -\der{t} \int_S \B \sde \\
    \curl \E = - \parder[\B]{t}
\end{gather}

Quindi, \important{la presenza di campi magnetici variabili nel tempo genera campi elettrici non conservativi}.

Una carica in moto accelerato genera:
\begin{itemize}
    \item Un campo elettrico conservativo, in quanto carica;
    \item Un campo magnetico variabile, in quanto corrente variabile.
    \begin{itemize}
        \item Quindi, un campo elettrico non conservativo.

        Le linee di questo campo non conservativo descrivono delle eliche intorno alle linee di campo magnetico.
    \end{itemize}
\end{itemize}

Questi campi elettrici non conservativi hanno linee di campo chiuse, poiché non nascono da cariche.
In seguito risulterà che, nel vuoto, sono localmente perpendicolari a quelle del campo magnetico.

La legge di induzione elettromagnetica generalizza la conservatività del campo elettrostatico in presenza di campi variabili nel tempo.

L'equazione ``duale'', che i fisici iniziarono subito a cercare, venne individuata solo decenni dopo da Maxwell e dimostrata da Hertz (vedi \autoref{sec:ampere_maxwell}).

\subsection{Autoinduzione}

Spira chiusa $\gamma$ percorsa da corrente $I$ dovuta a un generatore di tensione $V_0$ che genera un campo elettrico $\E_0$.

Viene generato un campo magnetico $\B$.

Se il circuito ha un interruttore, vuol dire che la corrente varia nel tempo da $I = 0$ a $I = V_0/R$.
Quindi, varia anche $\B$ e questo genera un campo elettrico $\E_\text{ind}$ che si sovrappone a $\E_0$.

Il campo magnetico in $\p$ è (legge di Ampère-Laplace):
\begin{equation}
    \B(\p) = \frac{\mu_0 I}{4\pi} \oint_\gamma \frac{\de \vt{l}' \times \ver{r'}}{\norm{\p - \p'}^2}
\end{equation}

Considerando una superficie $S$ delimitata da $\gamma$, il flusso di $\B$ attraverso $S$ è
\begin{equation}
    \Phi_S(\B) = I \underbrace{\frac{\mu_0}{4\pi} \int_S \oint_\gamma \frac{\de \vt{l}' \times \ver{r'}}{\norm{\p - \p'}^2} \sde}_L = L I
\end{equation}
$L$ si dice \important{coefficiente di autoinduzione} e dipende solo dalla geometria del circuito. L'unità di misura è l'henry, $[L] = \unit{\henry}$.
Il flusso del campo magnetico si misura in weber, $[\Phi(\B)] = \unit{\weber}$.

$L$ quantifica l'effetto che un circuito percorso da corrente variabile ha su se stesso.

Quando l'interruttore viene chiuso, $\B$ cresce nella direzione concorde col verso della corrente, quindi aumenta anche il suo flusso.
\begin{equation}
    \fem_i = - \der{t} \Phi_S(\B) = - L \der[I]{t}
\end{equation}
Il campo elettrico indotto, quindi, si oppone a quello originario.

\subsection{Circuiti RL}

Il lavoro del generatore è speso in calore e nella generazione del campo magnetico.
Descriviamo il bilancio energetico per una carica $\de q$ in moto:
\begin{itemize}
    \item lavoro erogato dal generatore di tensione: $V_0 \de q$
    \item lavoro dissipato in calore per effetto Joule: $R I \de q$
    \item lavoro speso per generare il campo magnetico: $-\fem_i \de q = L \der[I]{t} \de q$
\end{itemize}
Semplificando $\de q$,
\begin{equation}
    V_0 = R I + L \der[I]{t}
\end{equation}

La soluzione di questa equazione differenziale in $I$ con condizione iniziale $I(0) = 0$ è
\begin{equation}
    I(t) = \frac{V_0}{R} \pts{1 - \exp{-\frac{R}{L} t}}
\end{equation}
Ha pendenza $V_0/L$ per $t \to 0^+$ e tende a $V_0/R$ per $t \to +\infty$.

Alla fine, l'effetto autoinduttivo svanisce e il campo magnetico rimane costante.

\subsection{Energia di campo magnetico}
\label{sec:energia_campo_magnetico}

Circuito RL in cui l'induttore è un solenoide

Il coefficiente di autoinduzione è
\begin{equation}
    L = \frac{\Phi(\B)}{I} = \frac{S N B}{I} = \mu_0 S \frac{N^2}{l}
\end{equation}
$S$ è la sezione del solenoide, $l$ la lunghezza, $N$ il numero di spire.

L'energia assorbita nel transitorio dall'induttore è
\begin{equation}
    W = \int_0^t P(t') \de t'
    = \int_0^t \der{t'}(L I) I \de t'
    = \frac{1}{2} L I^2
    = \frac{1}{2} \frac{1}{\mu_0} \underbrace{\mu_0^2 \frac{N^2}{l^2} I^2}_{B^2} l S = \frac{1}{2} \frac{B^2}{\mu_0} l S
\end{equation}

Quindi, la densità di energia del campo magnetico è:
\begin{equation}
    w_B = \frac{1}{2} \frac{\norm{\B}^2}{\mu_0}
\end{equation}
Questo risultato è generale, per qualunque campo magnetico.

L'energia accumulata nel campo magnetico viene restituita al circuito quando si spegne il generatore di tensione, continuando a generare della corrente (analogamente al processo di scarica del condensatore, che restituisce l'energia di campo elettrico).

\subsection{Mutua induzione}

\addfigure{book/mutua_induzione}{0.6}

Due circuiti con fili $\gamma_1$ e $\gamma_2$ percorsi da corrente $I_1 = V_1/R_1$, $I_2 = V_2/R_2$.

Il campo magnetico generato da $I_1$ è
\begin{equation}
    \B_1(\p) = \frac{\mu_0 I_1}{4\pi} \oint_{\gamma_1} \frac{\de \vt{l}_1 \times \ver{r_1}}{\norm{\p - \p_1}^2}
\end{equation}
Il suo flusso attraverso una superficie $S_2$ che abbia bordo $\gamma_2$ è
\begin{equation}
    \Phi_{S_2}(\B_1) = I_1 \underbrace{\frac{\mu_0}{4\pi} \int_{S_2} \oint_{\gamma_1} \frac{\de \vt{l}_1 \times \ver{r_1}}{\norm{\p_2 - \p_1}^2} \sde_2}_{M_{21}} = M_{21} I_1
\end{equation}
$M_{21}$ è il \important{coefficiente di mutua induzione}.

Questo vuol dire che, accendendo il circuito 1, questo genera un campo magnetico attraverso $S_2$, e quindi una f.e.m.\ indotta nel circuito 2.
Finito il transitorio, sia $I_1$ sia $I_2$ sono costanti.

Svolgiamo alcuni bilanci energetici.

L'energia spesa dal generatore di tensione del circuito 1 per accendere la corrente nel circuito 2 è
\begin{equation}
\begin{gathered}
    \energy_{21}
    = \int \de \energy_{21}
    = \int \abs{\fem_2} \de q_2 = \\
    = \int M_{21} \der[I_1]{t} \de q_2
    = \int_0^{I_1} M_{21} I_2 \de I_1
    = M_{21} I_1 I_2
\end{gathered}
\end{equation}
Analogamente, l'energia spesa per accendere la corrente di mutua induzione nel circuito 1 è
\begin{equation}
    \energy_{12} = M_{12} I_1 I_2
\end{equation}
Nel caso dell'autoinduzione,
\begin{equation}
\begin{gathered}
    \energy_1
    = \int \de \energy_1
    = \int \abs{\fem_1} \de q_1 = \\
    = \int L_1 \der[I_1]{t} \de q_1
    = \int_0^{I_1} L_1 I_1 \de I_1
    = \frac{1}{2} L_1 I_1^2
\end{gathered}
\end{equation}

Mostriamo ora che $M_{12} = M_{21}$.

Consideriamo uno scenario 1:
\begin{itemize}
    \item Accendo $I_1$ con $I_2 = 0$
        \begin{equation}
            W_1 = \frac{1}{2} L_1 I_1^2
        \end{equation}
    \item Accendo $I_2$ con $I_1$ costante
        \begin{equation}
            W_{12} + W_2 = M_{12} I_1 I_2 + \frac{1}{2} L_2 I_2^2
        \end{equation}
        $W_{12}$ serve a mantenere $I_1$ costante mentre $I_2$ varia.
\end{itemize}

Scenario 2:
\begin{itemize}
    \item Accendo $I_2$ con $I_1 = 0$
        \begin{equation}
            W_2 = \frac{1}{2} L_2 I_2^2
        \end{equation}
    \item Accendo $I_1$ con $I_2$ costante
        \begin{equation}
            W_{21} + W_1 = M_{21} I_1 I_2 + \frac{1}{2} L_1 I_1^2
        \end{equation}
        $W_{21}$ serve a mantenere $I_2$ costante mentre $I_1$ varia.
\end{itemize}

Poiché alla fine ho raggiunto la stessa situazone,
\begin{equation}
    W_1 + W_{12} + W_2 = W_2 + W_{21} + W_1
    \implies
    M_{12} = M_{21} \eqcolon M
\end{equation}

Con $\B = \B_1 + \B_2$,
\begin{subequations}
\begin{gather}
    \begin{bmatrix} \Phi_{S_1}(\B) \\ \Phi_{S_2}(\B) \end{bmatrix} =
    \begin{bmatrix}
        L_1 & M \\ M & L_2
    \end{bmatrix}
    \begin{bmatrix} I_1 \\ I_2 \end{bmatrix}
    \\
    \begin{bmatrix} \fem_{i,1} \\ \fem_{i,2} \end{bmatrix} =
    - \begin{bmatrix}
        L_1 & M \\ M & L_2
    \end{bmatrix}
    \der{t} \begin{bmatrix} I_1 \\ I_2 \end{bmatrix}
\end{gather}
\end{subequations}

Questo rivela che i due circuiti sono accoppiati ed è possibile comunicare informazione trasportando energia da un circuito all'altro.

Ciò è realizzato con antenne (per trasferire informazione) e trasformatori (per trasferire energia).


\section{Principio di conservazione della carica}

Consideriamo un volume $V$ delimitato da una superficie $S = \partial V$.
La carica $q$ all'interno può aumentare o diminuire a seconda che le cariche escano o entrino.
$\uvt{n}$ è definito.

La carica netta che attraversa $S$ per unità di tempo è la corrente uscente:
\begin{equation}
    I = \oint_S \dcurr \sde
\end{equation}
Questa deve essere opposto alla variazione di carica all'interno di $V$ per unità di tempo.

Il principio di conservazione della carica, in formato integrale e differenziale, è dunque il seguente:
\begin{subequations}
\begin{gather}
    \der[q]{t} + I = 0 \\
    \parder[\rho]{t} + \diver \dcurr = 0
\end{gather}
\end{subequations}

Si passa dall'una all'altra forma tramite il teorema di Gauss e le definizioni di densità di carica e densità di corrente:
\begin{equation}
\begin{gathered}
    \der[q]{t} + I
    = \der{t} \int_V \rho \, \de V + \oint_S \dcurr \sde = \\
    = \int_V \parder[\rho]{t} \de V + \int_V \diver \dcurr \, \de V
    = \int_V \pts{\parder[\rho]{t} + \diver \dcurr} \de V
    = 0
\end{gathered}
\end{equation}

Considerando la legge di Gauss:
\begin{gather}
\begin{gathered}
    \eps_0 \oint_S \E \sde = q \\
    \implies
    \eps_0 \der{t} \oint_S \E \sde = -\der[q]{t} = -\oint_S \dcurr \sde \\
    \implies
    \oint_S \eps_0 \parder[\E]{t} \sde + \oint_S \dcurr \sde
\end{gathered} \\
\label{eq:flusso_corrente_generalizzata}
    \implies \oint_S \pts{\dcurr + \eps_0 \parder[\E]{t}} \sde = 0
\end{gather}
È una riscrittura della legge di conservazione della carica.

Inoltre, si osserva che $\eps_0 \parder[\E]{t}$ è dimensionalmente una densità di carica:
\begin{equation}
    \dimension \eps_0 \parder[\E]{t} = \dimension \dcurr = \mathsf{L}^{-2} \mathsf{I}
\end{equation}
Si definisce la \important{densità di corrente generalizzata}:
\begin{equation}
    \dcurr + \eps_0 \parder[\E]{t}
\end{equation}
Il suo flusso, che è nullo attraverso una superficie chiusa, è la \important{corrente generalizzata}.

Si pensi a un condensatore che si carica.
Considerando una superficie chiusa che include l'armatura positiva, la corrente è negativa (poiché entrante) ma il flusso (uscente) del campo elettrico aumenta, quindi la variazione nel tempo di tale flusso è positiva.

\subsection{Circuito RC}

Consideriamo un circuito con
un generatore di tensione $V_0$,
un condensatore a facce piane parallele di capacità $C$
e un resistore di resistenza $R$.

Per ogni carica $\de q$, l'energia erogata dal generatore viene dissipata in calore e spesa per generare un campo elettrico.
Semplificando i $\de q$,
\begin{equation}
    V_0 = R \der[q]{t} + \frac{1}{C} q
\end{equation}

La soluzione con condizione iniziale $q(0) = 0$ è
\begin{subequations}
\begin{gather}
    q(t) = C V_0 \pts{1 - \exp{-\frac{t}{RC}}} \\
    I(t) = \frac{V_0}{R} \exp{-\frac{t}{RC}}
\end{gather}
\end{subequations}

È ora possibile dare un'interpretazione fisica alle leggi di Kirchhoff:
\begin{itemize}
    \item KVL $\longleftrightarrow$ legge di conservazione dell'energia.
    \item KCL $\longleftrightarrow$ legge di conservazione della carica.
\end{itemize}

\section{Legge di Ampère-Maxwell}
\label{sec:ampere_maxwell}

% Applichiamo la legge di Ampère a una curva $\gamma$, usando una superficie $S$ che passa attraverso le armature del condensatore.

La legge di Ampère (\cref{eq:legge_di_ampere}) è valida solo se i campi sono statici, altrimenti viola la conservazione della carica.

Infatti, applichiamola a una superficie $S$ che tende a chiudersi, cosicché il suo bordo $\gamma$ tenda a diventare un punto.
La circuitazione di $\B$ dovrà anch'essa tendere a zero:
\begin{gather}
    \oint_\gamma \B \lde \to 0
    \implies \mu_0 \int_S \dcurr \sde \to 0
    \implies \oint_S \dcurr \sde = 0
\end{gather}
Al limite, infatti, la superficie diventa chiusa.

Ma questo contraddice l'\cref{eq:flusso_corrente_generalizzata} e vale solo se la corrente è stazionaria.

La legge di Ampère, ora, viola la legge di conservazione della carica.

Per correggere l'assurdo, si sostituisce alla densità di corrente la densità di corrente generalizzata nella legge di Ampère.

\begin{equation}
    \mu_0 \int_S \pts{\dcurr + \eps_0 \parder[\E]{t}} \sde
    = \mu_0 I + \mu_0\eps_0 \der{t} \Phi_S(\E)
\end{equation}

\important{Legge di Ampère-Maxwell}:
\begin{subequations}
\begin{gather}
    \oint_\gamma \B \lde
    = \mu_0 I + \mu_0\eps_0 \der{t} \Phi_S(\E) \\
    \curl \B = \mu_0 \pts{\dcurr + \eps_0 \parder[\E]{t}}
\end{gather}
\end{subequations}

Il segno $+$, diversamente dalla legge di induzione elettromagnetica, indica che $\B$ aumenta con verso concorde alla regola della mano destra quando il flusso di $\E$ aumenta.

L'esempio del condensatore citato sopra mostra anche che, tra le armature di un condensatore in carica, si genera un campo magnetico.

\section{Equazioni di Maxwell}

Principi:
\begin{itemize}
    \item Legge di Coulomb
    \item Conservazione della carica
    \item Sovrapposizione per $\E$
    \item Sovrapposizione per $\B$
    \item Legge elementare di Ampère
    \item Legge di Faraday-Henry
    \item Corrente di spostamento
    \item Forza di Lorentz
\end{itemize}
\medskip

Formato differenziale:
\begin{align}
    \diver \E & = \frac{\rho}{\eps_0} \\
    \diver \B & = 0 \\
    \curl \E & = - \parder[\B]{t} \\
    \curl \B & = \mu_0 \pts{\dcurr + \eps_0 \parder[\E]{t}}
\end{align}

Formato integrale:
\begin{gather}
    \oint_S \E \sde = \frac{q}{\eps_0} \\
    \oint_S \B \sde = 0 \\
    \oint_\gamma \E \lde = - \der{t} \int_{S(\gamma)} \B \sde \\
    \oint_\gamma \B \lde = \mu_0 I + \mu_0\eps_0 \der{t} \int_{S(\gamma)} \E \sde
\end{gather}
dove $S(\gamma)$ è una superficie aperta con bordo $\gamma$.

Sorgenti:
\begin{gather}
    q = \int_V \rho \, \de V \\
    I = \int_S \dcurr \sde
\end{gather}

Nel mezzo:
\begin{align}
    \diver \D & = \rho\free \\
    \diver \B & = 0 \\
    \curl \E & = - \parder[\B]{t} \\
    \curl \H & = \dcurr\free + \parder[\D]{t}
\end{align}
% Ovvero, $\E$ e $\H$ per le circuitazioni, $\D$ e $\B$ per i flussi.
Equazioni costitutive:
\begin{gather}
    \D = \eps_0 \E + \vt{P} = \eps \E \\
    \B = \mu_0 \H + \mu_0 \vt{M} = \mu \H
\end{gather}
% Le seconde uguaglianze valgono se il mezzo è lineare.

\chapter{Campi variabili nel tempo -- esercitazioni}

\section{Autoinduzione}

\begin{subequations}
\begin{gather}
    \oint_\gamma \E \lde = - \der{t} \Phi(\B) \\
    \curl \E = - \parder[\B]{t}
\end{gather}
\end{subequations}

Nei circuiti:
\begin{equation}
    \fem_i = - \der{t} \Phi(\B) = - L \der[I]{t}
\end{equation}
% se vale l'approssimazione a parametri concentrati: $\Delta t \gg d/c$.

Secondo la forza di Lorentz, vale anche
\begin{equation}
    \fem_i = \int_\gamma \E_i \lde
    = \int_\gamma \pts{\vt{v} \times \B} \lde
\end{equation}

Esempio:

Bobina di $N = 100$ spire di raggio $A = \qty{10}{\centi\metre}$ e resistenza $R = \qty{1}{\ohm}$.
Campo magnetico $\B(t) = \alpha t \uvt{n}$, con $\alpha = \qty{e-3}{\tesla\per\second}$.
Calcolare la corrente $I$ indotta nel circuito.

\begin{equation}
\begin{gathered}
    I = \frac{\fem}{R}
    = - \frac{1}{R} \der{t} \Phi(\B)
    = - \frac{1}{R} \der{t} \int_S \B \sde
    = - \frac{1}{R} \der{t}\pts{B \int_S \de S} = \\
    = - \frac{1}{R} \der{t}\pts{\alpha t N \pi A^2}
    = - \frac{\alpha N \pi A^2}{R}
    = -\qty{3.14}{\milli\ampere}
\end{gathered}
\end{equation}
misurata in verso antiorario se $\uvt{n}$ è uscente.

La corrente indotta genera un campo magnetico che contrasta la variazione di flusso.

Quanta carica scorre nella spira in $\Delta t = \qty{100}{\second}$?
\begin{equation}
    q = \int_{0}^{\Delta t} I(t) \de t = I \Delta t = \qty{3.14e-1}{\coulomb}
\end{equation}

La potenza dissipata dalla spira è
\begin{equation}
    P = \fem_i I = \qty{9e-6}{\watt}
\end{equation}


\subsection{Generatore di corrente alternata}

Campo magnetico uniforme $\B$ di modulo $B$ (generato, ad esempio, da un magnete) e spira di area $A$ la cui normale forma un angolo $\theta$ con $\B$.
La spira è messa in moto di rotazione uniforme con legge oraria $\theta(t) = \omega t$.

\begin{equation}
    \fem_i = - \der{t} \Phi(\B)
    = - \der{t} \pts{A B \cos(\omega t)}
    = A B \omega \sin(\omega t)
\end{equation}

Alternativamente, si può usare la forza di Lorentz:
\begin{equation}
    \fem_i = \int_\gamma \pts{\vt{v} \times \B} \lde
\end{equation}
Se la spira è rettangolare, i lati (lunghi $a$) perpendicolari all'asse di rotazione danno ciascuno contributo nullo, poiché $\vt{v}$ varia da un certo valore in corrispondenza di un estremo al suo opposto presso l'altro estremo del lato.
I lati (lunghi $b$) paralleli all'asse di rotazione hanno valore
\begin{equation}
    \int_A^B (\vt{v} \times \B) \lde
    = \int_A^B \pts{\frac{a}{2}\omega} B \sin(\omega t) \ver{AB} \lde
    = \frac{1}{2} abB \omega \sin\omega
\end{equation}
Analogamente,
\begin{equation}
    \int_C^D (\vt{v} \times \B) \lde
    = \frac{1}{2} abB \omega \sin\omega
\end{equation}
Quindi, $\fem_i = abB \omega \sin\omega$.

Si ottengono una tensione alternata e una corrente alternata:
\begin{subequations}
\begin{gather}
    V(t) = A B \omega \sin(\omega t) = V_0 \sin(\omega t) \\
    I(t) = \frac{\fem_i}{R} = \frac{AB \omega}{R} \sin(\omega t) = I_0 \sin(\omega t)
\end{gather}
\end{subequations}

Ruotando la spira, la forza di Lorentz (proporzionale alla velocità) fa sì che si senta una forza di attrito.
Bisogna fornire un lavoro meccanico con continuità, che sia convertito in energia elettrica.

Esempio:

$N = 20$ spire circolari con raggio $r = \qty{20}{\centi\metre}$, frequenza $f = \qty{50}{\hertz}$ e campo magnetico $B = \qty{0.4}{\tesla}$.

\begin{equation}
    \fem_i = N \pi r^2 \cdot B \cdot 2\pi f \cdot \sin(2\pi f t)
    = (\qty{316}{\volt}) \sin(\qty{314}{\per\second} \cdot t)
\end{equation}

Il \important{valore efficace} di una grandezza $X(t)$ periodica nel tempo di periodo $T$ è
\begin{equation}
    X\eff = \sqrt{\langle X(t)^2 \rangle_T}
    = \sqrt{\frac{1}{T} \int_0^T X(t)^2 \de t}
\end{equation}

\begin{equation}
    V\eff = AB\omega \sqrt{\frac{1}{\pi} \int_0^\pi \sin^2(\omega t) \de t}
    = \frac{1}{\sqrt{2}} A B \omega
    = \frac{V_0}{\sqrt{2}}
\end{equation}

Nell'esempio sopra, $V\eff = \qty{223}{\volt}$.

\subsection{Barretta in moto}

Circuito rettangolare con resistenza $R$ e una barretta lunga $L$ in moto a velocità $v$ espandendo il rettangolo.
Campo magnetico $B$ perpendicolare al circuito.

La f.e.m.\ è misurata con segno tale che $\uvt{n}$ ha lo stesso verso di $\B$.

Legge di autoinduzione:
\begin{equation}
    \fem_i = - \der{t} \Phi(B) = - v L B
\end{equation}

Forza di Lorentz:
\begin{equation}
    \fem_i = \int_\gamma \pts{\vt{v} \times \B} \lde
    = \int_\text{barretta} v B (-\ver{l}) \lde
    = - v L B
\end{equation}

Con la convenzione degli utilizzatori:
\begin{equation}
    I = \frac{v L B}{R}
\end{equation}

La forza esterna $\force$ per mantenere la barretta in moto rettilineo uniforme deve compensare la forza magnetica sulla barretta:
\begin{equation}
    \force = - I \vt{L} \times \B
    = \frac{vLB}{R} (L B \ver{v})
    = \frac{L^2 B^2}{R} \vt{v}
\end{equation}

La potenza da fornire è
\begin{equation}
    P = \force \cdot \vt{v} = \frac{v^2 L^2 B^2}{R}
\end{equation}

È proprio la potenza dissipata per effetto Joule:
\begin{equation}
    P = V_\text{resistore} I = - \fem_i I = \frac{v^2 L^2 B^2}{R}
\end{equation}

\subsection{Autoinduzione di secondo ordine}

Spira di raggio $A = \qty{10}{\centi\metre}$, resistenza $R = \qty{1}{\ohm}$ e attraversata da un campo magnetico perpendicolare $B = \alpha t^2$, con $\alpha = \qty{e-3}{\tesla\per\second\squared}$.
\begin{equation}
    \fem_i = - \pi A^2 \cdot 2 \alpha t
    \quad \implies \quad
    I = - \frac{2\pi A^2 \alpha}{R} t
\end{equation}

La corrente è variabile nel tempo e genera un campo magnetico indotto $B_i$.
Supponiamo per semplicità che il suo valore sulla superficie della spira sia ovunque uguale a quello nel centro:
\begin{equation}
    B_i = \frac{\mu_0 I}{2A} = - \frac{\mu_0 \pi A \alpha}{R}t
\end{equation}
La corrente indotta da un fenomeno secondario di autoinduzione è
\begin{equation}
    I_2 = -\frac{1}{R} \cdot \pi A^2 \cdot \pts{- \frac{\mu_0 \pi A \alpha}{R}}
    = \frac{\mu_0 \pi^2 A^3 \alpha}{R^2}
\end{equation}
$I_2$ scorre in modo da rafforzare il campo magnetico $B$ originario e genera a sua volta un campo magnetico $B_2$ con lo stesso verso di $B$ ma costante, e che quindi non induce altre correnti.
\begin{equation}
    B_2 = \frac{\mu_0 I_2}{2 A} = \frac{\mu_0^2 \pi^2 A^2 \alpha}{2 R^2}
\end{equation}

Calcoliamo i rapporti tra le grandezze per $t = \qty{0.1}{\second}$:
\begin{gather}
    \frac{I_2}{I} = \frac{B_2}{B_i} = -\frac{\mu_0 \pi A}{2 R t}
    = \num{2e-6}
\end{gather}
Inoltre, $I_2$ è sovrastimata.
Per questo, i fenomeni autoinduttivi di secondo ordine sono spesso trascurabili.
Il campo magnetico terrestre è dell'ordine di $\qty{e-5}{\tesla}$, quindi oscura tutti i campi magnetici dovuti ad autoinduzione di secondo ordine.

\subsection{Legge di Felici}

Spira in moto in presenza di un campo magnetico.
La carica che attraversa la spira tra i tempi $t_1$ e $t_2$ è
\begin{equation}
    q = \int_{t_1}^{t_2} \pts{-\frac{1}{R} \der[\Phi(\B)]{t}}\de t
    = -\frac{1}{R} \int_{\Phi_1}^{\Phi_2} \de \Phi
    = \frac{\Phi_1 - \Phi_2}{R}
\end{equation}

Esempio:

Spira di area $S$ e resistenza $R$ in un campo magnetico $\B$ non uniforme, ma che si può considerare uniforme sull'area della spira istante per istante.
Se ribalto la spira, la carica netta spostata è
\begin{equation}
    q = \frac{\Phi_1 - \Phi_2}{R} = \frac{2 B S}{R}
\end{equation}
È quindi possibile misurare un campo magnetico misurando (e integrando) la corrente che si genera mentre si ribalta una piccola spira:
\begin{equation}
    B = \frac{qR}{2S}
\end{equation}

\section{Mutua induzione}

\begin{gather}
    \Phi_{S_1}(\B_2) = M I_2 \\
    M = \frac{\mu_0}{4\pi} \int_{S_1} \oint_{\gamma_2} \frac{\de \vt{l}_2 \times \ver{r_2}}{\norm{\p_1 - \p_2}^2} \sde_1
\end{gather}
con $\ver{r_2} = (\p_1 - \p_2) / \norm{\p_1 - \p_2}$.

L'energia magnetica per accendere entrambi i circuiti è
\begin{equation}
    E_m = \frac{1}{2} \Phi_1 I_1 + \frac{1}{2} \Phi_2 I_2
    = \frac{1}{2} L_1 I_1^2 + \frac{1}{2} L_2 I_2^2 + M I_1 I_2
\end{equation}

Avendo $N$ circuiti,
\begin{equation}
    E_m = \frac{1}{2} \sum_{i = 1}^N \Phi_i I_i
    = \frac{1}{2} \sum_{i = 1}^N \sum_{j = 1}^N M_{ij} I_i I_j
\end{equation}
dove, per ogni $i, j = 1, \ldots, N$, $M_{ii} = L_i$ e $M_{ij} = M_{ji}$.

\subsection{Filo e spira}

Filo rettilineo infinito e spira di raggio $R = \qty{1}{\milli\metre}$ a distanza $r = \qty{1}{\metre}$.
Calcolare il coefficiente di mutua induzione.

Si suppone una corrente $I$ attraverso il filo e si calcola il flusso attraverso la superficie $S$ della spira:
\begin{equation}
\begin{gathered}
    M = \frac{\Phi_S(\B_\text{filo})}{I}
    = \int_S \frac{\mu_0}{2\pi r'} \de S'
    \approx \frac{\mu_0 R^2}{2 r}
    = \qty{6.28e-13}{\henry}
\end{gathered}
\end{equation}
L'approssimazione $r' \approx r$ segue da $R \ll r$.

\subsection{Spire concentriche}

Spire concentriche di raggi $R_1 = \qty{1}{\metre}$ e $R_2 = \qty{1}{\centi\metre}$.
Calcolare il coefficiente di mutua induzione.

Si calcola il flusso attraverso la superficie $S$ della spira piccola del campo dovuto a una corrente nella spira grande.

\begin{equation}
    M = \frac{\Phi_S(\B_\text{grande})}{I}
    \approx \int_S \frac{\mu_0}{2 R_1} \ver{n} \sde
    = \int_S \frac{\mu_0}{2 R_1} \de S
    = \frac{\mu_0 \pi R_2^2}{2 R_1}
    = \qty{2\pi^2 e-11}{\henry}
\end{equation}

Si approssima il campo magnetico nella spira piccola come se fosse uniforme e uguale al suo valore nel centro.

\subsection{Solenoide e spira}

Solenoide ideale lungo $L = \qty{5}{\metre}$ e con $N = 500$ spire.
Spira parallela alla sezione del solenoide e al suo interno, di diametro $d = \qty{10}{\milli\metre}$.

Flusso del campo nel solenoide attraverso la superficie $S$ della spira:
\begin{equation}
    M = \frac{\Phi_S(\B_\text{sol})}{I}
    = \int_S \mu_0 \frac{N}{L} \de S
    = \frac{\mu_0 N \pi d^2}{4 L}
    = \qty{9.87e-9}{\henry}
\end{equation}

\subsection{Spire concentriche in moto}

$R_1 = \qty{50}{\centi\metre}$, $R_2 = \qty{3}{\centi\metre}$, $I_1 = \qty{20}{\ampere}$ nella spira grande.
La spira piccola ha resistenza $Z = \qty{0.02}{\ohm}$ ed è in moto in direzione perpendicolare al piano su cui giace con $v = \qty{0.02}{\metre\per\second}$.
Quanto vale la f.e.m.\ nella spira piccola dopo $t_0 = \qty{2}{\second}$?
\begin{gather}
    M(t) = \frac{\Phi_{S_2}(\B)}{I_1}
    \approx \int_{S_2} \frac{\mu_0 R_1^2}{2 (R_1^2 + z(t)^2)^{3/2}} \uz \sde
    = \frac{\mu_0 \pi R_1^2 R_2^2}{2 (R_1^2 + z(t)^2)^{3/2}} \\
    \fem_2 = \evalat{\der{t} \pts{M(t) I_1}}{t_0}
    = -\frac{\mu_0 I_1 \pi R_1^2 R_2^2}{2}
    \frac{3}{2}
    \frac{2 z(t_0) v}{(R_1^2 + z(t_0)^2)^{5/2}}
    = -\qty{6.7e-10}{\volt}
\end{gather}

Si può mostrare che l'autoinduzione nella spira piccola è del tutto trascurabile.


\subsection{Spira quadrata e due fili}

Trovare il coefficiente di mutua induzione $M$ tra i fili a distanza $d$ e la spira quadrata di lato $h$ (ovvero, tra la spira e un filo e tra la spira e l'altro filo).

Spira con uno qualsiasi dei due fili:
\begin{equation}
\begin{gathered}
    M = \frac{\Phi_S(\B)}{I}
    = \int_S \frac{\mu_0}{2\pi r'} \de S'
    = \frac{\mu_0}{2\pi} \int_{x = d/2 - h/2}^{d/2 + h/2} \int_{y = 0}^h \frac{1}{x} \, \de y \, \de x = \\
    = \frac{\mu_0 h}{2\pi} \Big[\!\ln x\Big]_{d/2 - h/2}^{d/2 + h/2}
    = \frac{\mu_0 h}{2\pi} \ln \frac{d + h}{d - h}
\end{gathered}
\end{equation}

Se nei due fili la corrente scorre in versi opposti, l'induzione totale è nulla.

\chapter{Onde elettromagnetiche}

\begin{itemize}
    \item Legge di induzione elettromagnetica $\to$ campi magnetici variabili generano campi elettrici perpendicolari ai campi inducenti.
    \item Legge di Ampère-Maxwell $\to$ campi elettrici variabili generano campi magnetici perpendicolari ai campi inducenti.
\end{itemize}

In un punto $\p$ nel vuoto, siano $\E(\p) \parallel \uy \perp \B(\p) \parallel \uz$.

Si applica Ampère-Maxwell per un piccolo quadrato $\perp \uy$:
\begin{equation}
\label{eq:ampere_maxwell_quadratino}
\begin{gathered}
    \oint_\gamma \B \lde
    = \mu_0\eps_0 \der{t} \Phi_S(\E) \\
    \implies
    -\pts{\B + \parder[\B]{x} \de x} \cdot \de z \uz + \B \cdot \de z \uz
    = \mu_0 \eps_0 \parder[\E]{t} \cdot \de z \de x \uy \\
    \implies
    - \parder[\B]{x} \cdot \de x \de z \uz
    = \mu_0 \eps_0 \parder[\E]{t} \cdot \de x \de z \uy \\
    \implies
    - \parder[B]{x} = \mu_0 \eps_0 \parder[E]{t}
\end{gathered}
\end{equation}

Si applica Faraday-Henry per un piccolo quadrato $\parallel \uz$:
\begin{equation}
\label{eq:faraday_henry_quadratino}
\begin{gathered}
    \oint_\gamma \E \lde
    = -\der{t} \Phi_S(\B) \\
    \implies
    \pts{\E + \parder[\E]{x} \de x} \cdot \de y \uy - \E \cdot \de y \uy
    = - \parder[\B]{t} \cdot \de x \de y \uz \\
    \implies
    \parder[\E]{x} \cdot \de x \de y \uy
    = - \parder[\B]{t} \cdot \de x \de y \uz \\
    \implies
    \parder[E]{x} = - \parder[B]{t}
\end{gathered}
\end{equation}

Riassumendo,
\begin{equation}
    \begin{cases}
        - \parder[B]{x} = \mu_0 \eps_0 \parder[E]{t} \\
        \parder[E]{x} = - \parder[B]{t}
    \end{cases}
\end{equation}

Derivando ulteriormente la \eqref{eq:ampere_maxwell_quadratino} (AM) rispetto a $t$ e la \eqref{eq:faraday_henry_quadratino} (FH) rispetto a $x$,
\begin{equation}
    \begin{cases}
        - \partial^2_{tx} B = \mu_0 \eps_0 \parder[E][2]{t} \\
        \parder[E][2]{x} = - \partial^2_{xt} B
    \end{cases}
    \implies \parder[E][2]{x} = \mu_0 \eps_0 \parder[E][2]{t}
\end{equation}

Derivando ulteriormente AM rispetto a $x$ e FH rispetto a $t$,
\begin{equation}
    \begin{cases}
        - \parder[B][2]{x} = \mu_0 \eps_0 \partial^2_{tx} E \\
        \partial^2_{xt} E = - \parder[B][2]{t}
    \end{cases}
    \implies \parder[B][2]{x} = \mu_0 \eps_0 \parder[B][2]{t}
\end{equation}

Cioè
\begin{gather}
    \parder[E][2]{x} = \mu_0 \eps_0 \parder[E][2]{t} \\
    \parder[B][2]{x} = \mu_0 \eps_0 \parder[B][2]{t}
\end{gather}

$\E$ e $\B$ sono nuovamente disaccoppiati, e per ciascuno vale l'\important{equazione di d'Alembert}:
\begin{equation}
\label{eq:dalembert_1d}
    \parder[y][2]{x} = \frac{1}{v^2} \parder[y][2]{t}
\end{equation}

con $\mu_0 \eps_0 = 1/v^2$.

\section{Derivazione in forma differenziale}

Usando la seguente identità dell'analisi vettoriale
\begin{equation}
    \curl \curl \vt{F} = \grad \pts{\diver \vt{F}} - \vlapl \vt{F}
\end{equation}
si giunge alla forma generale dell'equazione di d'Alembert per $\E$ e $\B$ nel vuoto:
\begin{gather}
    \curl \E = - \parder[\B]{t} \\
    \implies
    \curl \curl \E = - \curl \parder[\B]{t} \\
    \implies
    \cancel{\grad \pts{\diver \E}} - \vlapl \E = - \parder{t} \curl \B \\
    \implies
    \vlapl \E = \parder{t} \pts{\mu_0 \eps_0 \parder[\E]{t}} \\
    \implies
    \vlapl \E = \frac{1}{c^2} \parder[\E][2]{t}
\end{gather}

Analogamente per $\B$.

\section{Soluzioni dell'equazione di d'Alembert}

La soluzione della \eqref{eq:dalembert_1d}, in una dimensione, è
\begin{equation}
    y(x, t) = f_1(x - vt) + f_2(x + vt)
\end{equation}
Ovvero, la somma di funzioni $f_1$ e $f_2$ generiche che traslano nello spazio in direzione $x$ con una certa velocità.
Si tratta di onde, segnali che si spostano nello spazio.

Quindi, $\E$ e $\B$ si propagano nello spazio come onde, con velocità
\begin{equation}
    c = \frac{1}{\sqrt{\mu_0 \eps_0}} = \qty{299792458}{\metre\per\second}
\end{equation}
$c$ è la \important{velocità dell'onda elettromagnetica nel vuoto}.

Mentre le altre onde hanno un supporto, queste si propagano nel vuoto e sono perturbazioni di campi.
Storicamente, vi furono due reazioni:
\begin{itemize}
    \item Si tratta solo di un risultato matematico senza realtà fisica
    \item Esiste un supporto, l'\textit{etere}, che permette la propagazione delle onde elettromagnetiche
\end{itemize}

Le onde elettromagnetiche sono sempre solo trasversali (cioè, perpendicolari alla direzione di propagazione), non longitudinali.
Se la direzione di propagazione varia, $\E$ e $\B$ rimangono perpendicolari tra loro e alla direzione di propagazione.

Le funzioni $f_1$ e $f_2$ vengono considerate sinusoidali:
\begin{equation}
    f(x - c t) = A \sin\pts{k (x - c t)} = A \sin(k x - \omega t)
    = A \sin\pts{2\pi \pts{\frac{x}{\lambda} - \frac{t}{T}}}
\end{equation}
$A$ è l'ampiezza, $k = 2\pi / \lambda$ è il \important{numero d'onda}, $\omega = 2\pi / T$ è la $\important{pulsazione}$, $k x - \omega t$ è la \important{fase}.
Segue che
\begin{equation}
    c = \frac{\omega}{k} = \frac{\lambda}{T}
\end{equation}
Infatti, grazie alla legge di Fourier, qualunque funzione si può scrivere come serie di seni e coseni.

Le onde con $f$ sinusoidale sono dette \important{onde armoniche}.
Ogni moto di carica non rettilineo si può approssimare moto circolare (poiché la traiettoria si può approssimare come arco di circonferenza), quindi ogni carica in moto emette ``localmente'' onde elettromagnetiche sinusoidali.

Anche le cariche nei materiali emettono onde elettromagnetiche sinusoidali, poiché sono in moti oscillatori.
\begin{subequations}
\begin{gather}
    \E(x, t) = \E_0 \sin(k x - \omega t) \\
    \B(x, t) = \B_0 \sin(k x - \omega t)
\end{gather}
\end{subequations}

Usando la \eqref{eq:faraday_henry_quadratino},
\begin{equation}
\begin{gathered}
    \parder[E]{x} = - \parder[B]{t}
    \implies
    \der[E]{(x - c t)} \der[(x - c t)]{x} = - \der[B]{(x - c t)} \der[(x - c t)]{t} \\
    \implies
    \der[E]{(x - c t)} = c \der[B]{(x - c t)}
    \implies
    E = c B
\end{gathered}
\end{equation}

\important{Fronte d'onda}: superficie a fase costante.
È $\parallel \E \times \B$ (cioè perpendicolare sia a $\E$ si a $\B$) e si sposta a velocità $c$.

In conclusione, nel vuoto:
\begin{subequations}
\begin{gather}
    \vlapl \E = \frac{1}{c^2} \parder[\E][2]{t} \\
    \vlapl \B = \frac{1}{c^2} \parder[\B][2]{t} \\
    \E \perp \B \\
    \norm{\E} = c\norm{\B}
\end{gather}
\end{subequations}


\section{Tipi di onde}

\subsection{Onda sferica}

Un'onda sferica si genera quando una sorgente puntiforme emette onde in un mezzo isotropo ed è caratterizzata da $\vt{k} \parallel \p$ (ponendo la sorgente nell'origine).
Infatti, la direzione di propagazione è necessariamente radiale rispetto alla sorgente.
\begin{equation}
    \E(\p, t) = \E_0(r) \sin(k r - \omega t)
\end{equation}
L'ampiezza dipende da $r$ per ragioni di conservazione dell'energia (vedi \autoref{sec:intensita_onda_sferica}).
La fase dipende solo dal modulo di $\p$ poiché i fronti d'onda sono, appunto, a simmetria sferica.

Per conoscere l'onda dovuta a una sorgente generica, occorre considerare varie sorgenti puntiformi e integrare sulle onde sferiche che queste generano.

\subsection{Onda piana}

Sono le soluzioni (non banali) più semplici alle equazioni di Maxwell nel vuoto.

Un'onda piana è tale che le direzioni di $\E$ e $\B$ sono costanti.
\begin{subequations}
\begin{gather}
    \E(\vt{x}, t) = \E_0 \sin(\vt{k} \cdot \vt{x} - \omega t)
    = \imag{\E_0 \exp{\im(\vt{k} \cdot \vt{x} - \omega t)}} \\
    \B(\vt{x}, t) = \B_0 \sin(\vt{k} \cdot \vt{x} - \omega t)
    = \imag{\B_0 \exp{\im(\vt{k} \cdot \vt{x} - \omega t)}}
\end{gather}
\end{subequations}
$\vt{k}$ è il \important{vettore d'onda}, di modulo $k$ (il numero d'onda) e che ha per direzione la direzione di propagazione.
Nel caso di un'onda piana, è costante.

Un'onda piana è \important{polarizzata linearmente}, cioè sia $\E$ sia $\B$ oscillano in piani fissi.

Onda \important{polarizzata circolarmente}: $\E$ e $\B$, anziché oscillare in modulo, ruotano in moto circolare uniforme.

Le onde piane non esistono, visto che le sorgenti non sono mai infinite.
Sono approssimazioni di onde sferiche per percorsi brevi rispetto alla distanza dalla sorgente.

\subsection{Onda generica}

Nello spazio, l'equazione di un onda è
\begin{equation}
    A = f(\phi) = f(\vt{r} \cdot \uvt{u} - vt)
\end{equation}
$A$ è la quantità perturbata, $\vt{r}$ è il punto sul fronte d'onda, $\uvt{u}$ ha la direzione verso cui l'onda si propaga a velocità $v$.

I \important{raggi} sono le curve tangenti a $\uvt{u}$.
$\uvt{u}$ e i raggi sono ovunque normali ai fronti d'onda.

\section{Onde elettromagnetiche nel mezzo}

Se il mezzo in cui l'onda si propaga è \important{isotropo}, i raggi sono rette.

In un mezzo omogeneo isotropo (in assenza di cariche o correnti nette), è sufficiente sostituire $\mu_0$ ed $\eps_0$ con $\mu$ ed $\eps$.

La velocità dell'onda è
\begin{equation}
    v = \frac{1}{\sqrt{\mu \eps}} = \frac{c}{\sqrt{\mu_r \eps_r}}
\end{equation}

Molto spesso $\mu_r \approx 1$ (in tutti i materiali tranne quelli ferromagnetici, che però sono spesso anche conduttori).
Quindi, $\eps_r$ ha la maggiore importanza.

Si definisce l'\important{indice di rifrazione}:
\begin{equation}
    n \coloneq \sqrt{\mu_r \eps_r} \approx \sqrt{\eps_r} > 1
\end{equation}

Per cui
\begin{equation}
    v = \frac{c}{n}
\end{equation}

Considerando un'onda piana, nel mezzo variano ampiezza e lunghezza d'onda/numero d'onda, ma non pulsazione/frequenza/periodo.
\begin{gather}
    v = \frac{c}{n} \\
    k = n k_0
    \quad \iff \quad
    \lambda = \frac{\lambda_0}{n} \\
    \omega = \omega_0
    \quad \iff \quad
    T = T_0
    \quad \iff \quad
    \nu = \nu_0
\end{gather}

Per dimostrarlo, si considera una carica $q$ nell'origine con velocità $\vt{v}_p(t) = v_0 \sin(\omega t) \uy$.
In un punto sull'asse $x$, si ha che $\B(x, 0, 0) \parallel \uz$ e $\E(x, 0, 0) \parallel \uy \parallel \vt{v}$.
Questo vale sempre: il campo elettrico è parallelo al moto della carica.

Detta $v$ la velocità di propagazione dei campi, possiamo valutare l'andamento di $B$ (e quindi di $E = c B$) grazie alla legge elementare di Ampère:
\begin{equation}
    B(x, t) \propto q v_p\pts{t - \frac{x}{v}} = - q v_0 \sin(k x - \omega t)
\end{equation}
Il termine $k x$ è dovuto all'\important{effetto di ritardo}.

Si può concludere che:
\begin{itemize}
    \item La pulsazione $\omega$ del campo deriva dal moto della carica, non dal mezzo di propagazione.
    \item Il numero d'onda $k = \omega / v$ deriva dall'effetto di ritardo, e quindi dipende dal mezzo.
\end{itemize}


\section{Energia del campo elettromagnetico}

Siano $E \coloneq \norm{\E}$ e $B \coloneq \norm{\B}$.

Densità di energia totale del campo elettromagnetico è la somma delle densità di energia dei campi elettrostatico e magnetostatico (anche se non sono statici):
\begin{equation}
    w\EM = \frac{1}{2} \eps_0 E^2 + \frac{1}{2} \frac{B^2}{\mu_0}
\end{equation}
Inoltre,
\begin{equation}
    \eps_0 E^2 = \eps_0 c^2 B^2 = \frac{B^2}{\mu_0}
\end{equation}
Quindi ciascuno dei due campi contribuisce per metà energia.
Allora,
\begin{equation}
    w\EM = \eps_0 E^2 = \frac{B^2}{\mu_0}
\end{equation}

\section{Esperimento di Hertz}

Svolto nel 1888, serve a dimostrare l'esistenza delle onde elettromagnetiche.

Si collegano due sfere metalliche a un trasformatore, in modo che si carichino e scarichino ripetutamente.

Alla rottura del dielettrico (aria), avviene un arco elettrico e le sfere si scaricano.

Ponendo una spira con amperometro in un punto tra tra le sfere e una lastra di metallo posta lontano si può capire se si propagano campi.
Una lastra metallica svolge il ruolo di uno specchio riflettente per le onde elettromagnetiche.

Presso la spira, si sovrappongono onde progressive (incidenti) e regressive (riflesse):
\begin{subequations}
\begin{gather}
    E(x, t) = E_{0,i}\exp{\im(kx - \omega t)} + E_{0,r}\exp{-\im(kx + \omega t)} \\
    B(x, t) = B_{0,i}\exp{\im(kx - \omega t)} + B_{0,r}\exp{-\im(kx + \omega t)}
\end{gather}
\end{subequations}
$\omega t$ è sempre con segno meno perché l'onda torna indietro nello spazio, non nel tempo.

Sulla superficie del metallo il campo elettrico totale deve essere nullo (poiché non ha componente normale alla lastra).

Questo, con $x = 0$ (cioè sulla lastra), implica $E_{0,i} = - E_{0,r} \eqcolon E_0$.

La direzione di propagazione, che è $\parallel \E \times \B$, deve invece cambiare verso.
Quindi, se $\E$ diventa opposto, $\B$ deve rimanere quale è: $B_{0,i} = B_{0,r} \eqcolon B_0$.
\begin{subequations}
\label{eq:onde_stazionarie}
\begin{gather}
    E(x, t) = E_0\exp{\im(kx - \omega t)} - E_0\exp{-\im(kx + \omega t)}
    = 2 E_0 \sin(kx) \cos(\omega t) \\
    B(x, t) = B_0\exp{\im(kx - \omega t)} + B_0\exp{-\im(kx + \omega t)}
    = 2 B_0 \cos(kx) \cos(\omega t)
\end{gather}
\end{subequations}

Usando il fatto che
\begin{equation}
\begin{gathered}
    \exp{\im (A + B)} = \cos(A + B) + \im \sin(A + B)
    = \\
    = \cos(A) \cos(B) - \sin(A) \sin(B) + \im \cos(A) \sin(B) + \im \sin(A) \cos(B)
\end{gathered}
\end{equation}

Le \eqref{eq:onde_stazionarie} sono \important{onde stazionarie}, quindi esistono dei nodi.
Posizionando lì la spira, se questa è sufficientemente piccola, l'amperometro non legge mai corrente.

La spira deve essere piccola rispetto alla lunghezza d'onda, quindi occorre scegliere una frequenza sufficientemente bassa.

In questo modo si misura anche la velocità della luce, poiché $\omega$ è scelta e $\lambda/2$ è la distanza tra i nodi.
Questa può essere misurata spostando la spira via da un nodo e trovando il successivo.



\section{Pressione di radiazione}

Il campo elettromagnetico trasporta non solo energia, ma anche quantità di moto.

Lo si verifica con un dinamometro a torsione, illuminando degli specchi dopo averli chiusi in una bolla di vetro per evitare altre perturbazioni.

Consideriamo un'onda piana che incide perpendicolarmente un materiale.
Un'elettrone del materiale (di carica elementare $e$) viene accelerato dal campo elettrico in verso opposto a $\E$ acquisendo una velocità $v$.
Viene poi accelerato dal campo magnetico verso l'interno del materiale.

La forza perpendicolare risultante $F_x$ genera il trasferimento della quantità di moto $p$:
\begin{gather}
    F_x = \der[p_x]{t} = e v B \\
    \der[U]{t} = F_E v = e E v = e c B v = c \der[p]{t}
    \implies \Delta U = c \Delta p
\end{gather}
$U$ è l'energia dell'elettrone e varia solo in funzione della forza elettrica poiché $\B$ non compie lavoro.

La densità di quantità di moto di un'onda elettromagnetica, quindi, è
\begin{equation}
    p\EM = \frac{w\EM}{c}
\end{equation}

Se l'onda è totalmente assorbita dal materiale, la pressione esercitata sul materiale è
\begin{equation}
    \text{pressione} = \frac{F_\perp}{A} = \frac{1}{A} \frac{\Delta p}{\Delta t}
    = \frac{1}{A} \frac{p\EM A c \Delta t}{\Delta t}
    = w\EM
\end{equation}
$A c \Delta t = A \Delta x$ è il volume che incide sulla superficie in un tempo $\Delta t$.

Se l'onda viene totalmente riflessa la pressione è doppia, poiché è doppia la variazione di quantità di moto.

Se la radiazione incide con un angolo $\theta$ rispetto alla normale, occorre moltiplicare per $\cos^2\theta$ (un coseno viene dalla forza, l'altro dal volume che incide sulla superficie nel tempo $\Delta t$).
\begin{equation}
    \text{pressione} = \frac{F \cos\theta}{A} = \frac{1}{A} \frac{\Delta p \cos \theta}{\Delta t}
    = \frac{1}{A} \frac{p\EM \pts{A c \Delta t \cos \theta} \cos \theta}{\Delta t}
    = \cos^2(\theta) w\EM
\end{equation}


\section{Vettore di Poynting}

Volume $V$ con $S = \partial V$ in presenza di campi elettrico e magnetico e con, all'interno, delle cariche $q$ in densità volumica $n$ in moto con velocità $\vt{v}$.

La densità di carica è $\rho = qn$, la densità di corrente $\dcurr = qn\vt{v}$.

Qual è la potenza che il campo elettromagnetico trasferisce alle cariche in moto?
\begin{gather}
    \de P = n q \de V \E \cdot \vt{v} = n q \vt{v} \cdot \E \,\de V = \dcurr \cdot \E \,\de V \\
    \implies P = \int_V \dcurr \cdot \E \,\de V
\end{gather}

Dalla quarta equazione di Maxwell:
\begin{equation}
    \curl \B = \mu_0 \dcurr + \mu_0 \eps_0 \parder[\E]{t}
    \implies
    \dcurr = \frac{1}{\mu_0} \curl \B - \eps_0 \parder[\E]{t}
\end{equation}

Per cui
\begin{equation}
\label{eq:potenza1}
\begin{gathered}
    P = \int_V \pts{\frac{1}{\mu_0} \curl \B - \eps_0 \parder[\E]{t}} \cdot \E \,\de V = \\
    = \frac{1}{\mu_0} \int_V (\curl \B) \cdot \E \,\de V - \eps_0 \int_V \parder[\E]{t} \cdot \E \,\de V
\end{gathered}
\end{equation}

Si usa la seguente identità:
\begin{equation}
    \diver (\E \times \B) = (\curl \E) \cdot \B - \E \cdot (\curl \B)
\end{equation}

La \eqref{eq:potenza1} prosegue come:
\begin{equation}
\begin{gathered}
    P = \frac{1}{\mu_0} \int_V (\curl \E) \cdot \B \,\de V
    - \frac{1}{\mu_0} \int_V \diver (\E \times \B) \,\de V
    - \eps_0 \int_V \parder[\E]{t} \cdot \E \,\de V = \\
    = - \frac{1}{\mu_0} \int_V \parder[\B]{t} \cdot \B \,\de V
    - \frac{1}{\mu_0} \int_V \diver (\E \times \B) \,\de V
    - \eps_0 \int_V \parder[\E]{t} \cdot \E \,\de V = \\
    = - \int_V \pts{\frac{1}{2 \mu_0} \parder[\B^2]{t} + \frac{\eps_0}{2} \parder[\E^2]{t}} \de V
    - \frac{1}{\mu_0} \int_V \diver (\E \times \B) \,\de V = \\
    = - \der{t} \int_V \pts{\frac{1}{2 \mu_0} \B^2 + \frac{\eps_0}{2} \E^2} \de V
    - \oint_{\partial V} \frac{\E \times \B}{\mu_0} \sde = \\
    = -\der[U\EM]{t} - \flux{\partial V}{\poy}
\end{gathered}
\end{equation}

Nell'ultimo passaggio è stato definito il \important{vettore di Poynting}:
\begin{gather}
    \poy \coloneq \frac{\E \times \B}{\mu_0} \\
    \norm{\poy} = \frac{E B}{\mu_0} = \frac{E^2}{\mu_0 c} = \eps_0 c E^2
\end{gather}
Si tratta della potenza per unità di superficie di un'onda elettromagnetica.

Il bilancio energetico che ne risulta costituiscce il \important{teorema di Poynting}:
\begin{gather}
    -\der[U\EM]{t} = P + \flux{\partial V}{\poy}
\end{gather}
Ovvero, la perdita di energia di campo elettromagnetico (a sinistra) è dovuta a
\begin{itemize}
    \item lavoro svolto sulle cariche
    \item uscita dell'onda attraverso la superficie
\end{itemize}

Inoltre, l'energia ``stanziale'' (per unità di volume) $w\EM$ e l'energia (per unità di tempo e si superficie) espressa dal vettore di Poynting sono la stessa a meno di costanti: entrambe proporzionali al quadrato dell'ampiezza dell'onda.

Un altro modo di derivare il vettore di Poynting è il seguente: si considera un cilindro di base $A$ e lunghezza $c \Delta t$ con asse parallelo alla velocità dell'onda $\vt{c}$.

la potenza per unità di superficie che esce da una faccia del cilindro è
\begin{equation}
    \frac{1}{A} \der[U]{t} = \frac{w\EM A c \de t}{A \de t} = w\EM c = \eps_0 E^2 c
\end{equation}
Considerare questa potenza come vettore:
\begin{equation}
    \eps_0 E^2 \vt{c} = \frac{\E \times \B}{\mu_0} = \poy
\end{equation}

Il vettore di Poynting permette inoltre di definire l'\important{intensità} di un'onda elettromagnetica:
\begin{equation}
    I \coloneq \avg{\norm{\poy}}
\end{equation}
La media è considerata sul periodo dell'onda (sinusoidale).
Se l'onda elettromagnetica è visibile, si tratta proprio dell'intensità avvertita dall'occhio.

\subsection{Intensità di un'onda piana}

Onda piana con direzione di propagazione $x$ con polarizzazione rettilinea lungo $y$.
\begin{equation}
    \E(\p, t) = \E_0 \sin(\vt{k} \cdot \p - \omega t)
\end{equation}
Calcoliamo il valore medio nel tempo del vettore di Poynting:
\begin{equation}
\begin{gathered}
    \avg{\vt{S}} = \frac{1}{T} \int_0^T \vt{S} \, \de t
    = \frac{1}{T} \int_0^T \frac{E^2}{\mu_0 c} \ver{k} \, \de t = \\
    = \frac{1}{T} \int_0^T \frac{E_0^2}{\mu_0 c} \sin^2(\vt{k} \cdot \p - \omega t) \ver{k} \, \de t
    = \frac{1}{T} \frac{E_0^2}{\mu_0 c} \ver{k} \int_0^T \sin^2(\vt{k} \cdot \p - \omega t) \, \de t
\end{gathered}
\end{equation}
Sostituzione:
\begin{equation}
\begin{gathered}
    y(t) = \vt{k} \cdot \p - \omega t, \quad
    \de y = -\omega \de t = -\frac{2 \pi}{T} \de t, \quad \\
    t \in [0, \, T] \implies y \in [a, \, a - 2\pi]
    \text{ con } a \coloneq \vt{k} \cdot \p
\end{gathered}
\end{equation}
Segue
\begin{gather}
    \avg{\poy} = \frac{E_0^2}{2 \pi \mu_0 c} \ver{k} \int_{a - 2\pi}^a \sin^2(y) \, \de y
    = \frac{E_0^2}{2 \mu_0 c} \ver{k}
    = \frac{1}{2} \eps_0 c E_0^2 \ver{k}
\end{gather}

Analogamente,
\begin{equation}
    I = \avg{\norm{\poy}} = \frac{E_0^2}{2 \mu_0 c}
    = \frac{1}{2} \eps_0 c E_0^2
\end{equation}
È l'intensità di un'onda piana, \important{proporzionale al quadrato dell'ampiezza} dell'onda.


\subsection{Intensità di un'onda sferica}
\label{sec:intensita_onda_sferica}

Presso un certo punto dello spazio in presenza di un'onda sferica, l'energia per unità di tempo e di superficie è
\begin{equation}
    \norm{\poy} = \frac{1}{\mu_0 c} E^2 = \frac{1}{\mu_0 c} E_0^2 \sin^2(kr - \omega t)
\end{equation}

Se la sorgente dell'onda è puntiforme a distanza $r$ ed emette una potenza $W(t)$, la potenza che fuoriesce da una sfera di raggio $r$ dopo essere stata emessa dalla sorgente è
\begin{equation}
    4 \pi r^2 \norm{\poy} = \frac{4 \pi r^2}{\mu_0 c} E_0^2 \sin^2(kr - \omega t)
\end{equation}

Questa non è sempre uguale alla potenza emessa dalla sorgente perché c'è un ritardo temporale $r/v$.
Tuttavia, si può considerare la potenza media su un periodo:
\begin{gather}
    W_\text{avg} = \avg{W(t)} = \avg{\frac{4 \pi r^2}{\mu_0 c} E_0^2 \sin^2(kr - \omega t)}
    = 4\pi r^2 I
    = \frac{2 \pi r^2}{\mu_0 c} E_0^2 \\
    \implies E_0 = \sqrt{\frac{W_\text{avg} \mu_0 c}{2\pi}} \frac{1}{r}
\end{gather}

\chapter{Onde elettromagnetiche -- esercitazioni}

\section{Onda sferica}

Un'onda sferica di genera quando una sorgente puntiforme emette onde in un mezzo isotropo ed è caratterizzata da $\vt{k} \parallel \p$ (ponendo la sorgente nell'origine).
\begin{gather}
    \E(\p, t) = \E_0 \sin(k r - \omega t) \\
    \vt{S} = \frac{E_0^2}{\mu_0 c} \sin^2(kr - \omega t) \ver{r}
\end{gather}

Il flusso di $\vt{S}$ attraverso una superficie sferica $\Sigma$ di raggio $r$ è la seguente potenza:
\begin{equation}
\begin{gathered}
    P(r) = \int_\Sigma \vt{S} \sde = \int_{\Sigma} \frac{E_0^2}{\mu_0 c} \sin^2(kr - \omega t) \ver{r} \sde = \\
    = \int_{\phi = 0}^{2\pi} \int_{\theta = 0}^{\pi} \frac{E_0^2}{\mu_0 c} \sin^2(kr - \omega t) \, r^2 \sin\theta \, \de\theta \de\phi
    = \frac{E_0^2}{\mu_0 c} \sin^2(kr - \omega t) \, 4 \pi r^2
\end{gathered}
\end{equation}

La potenza media in un periodo vale
\begin{equation}
    \avg{P} = \frac{1}{T} \int_0^T P(t) \de t
    = \frac{E_0^2}{\mu_0 c} 4 \pi r^2 \frac{1}{T} \int_0^T \sin^2(kr - \omega t) \de t
    = \frac{E_0^2}{2\mu_0 c} 4 \pi r^2
    = 4 \pi r^2 I
\end{equation}

La potenza deve essere indipendente dal raggio, e questo vale se e solo se
\begin{equation}
    E_0 \propto \frac{1}{r}
\end{equation}

Esercizio: il campo elettrico raccolto da un ricevitore radio a una distanza $d = \qty{500}{\metre}$ dalla sorgente ha un'ampiezza massima $E_0 = \qty{0.1}{\volt\per\metre}$. Calcolare l'intensità dell'onda presso il ricevitore e la potenza emessa dalla sorgente.
\begin{gather}
    I = \frac{E_0^2}{2 \mu_0 c} = \qty{1.33e-5}{\watt\per\metre\squared} \\
    \avg{P} = 4 \pi d^2 I = \qty{41.7}{\watt}
\end{gather}

\chapter{Interferenza e diffrazione}

\section{Interferenza}

A grande distanza da una sorgente di onde sferiche si pone uno schermo con $N$ fenditure allineate a reciproca distanza $d$.

Ciascuna delle fenditure agisce come sorgente puntiforme di onde sferiche.
% uno schermo parallelo alla congiungente tra le sorgenti a distanza $D$.

Poiché provengono dallo stesso fronte d'onda della stessa onda (che si approssima a onda piana data la grande distanza delle fenditure dalla sorgente), le nuove onde sferiche saranno \important{coerenti}, cioè:
\begin{itemize}
    \item Stessa polarizzazione (i campi elettrici sono tutti paralleli al piano)
    % perpendicolari alla congiungente
    \item Stessa ampiezza massima $\E_0$.
    \item Stessa pulsazione $\omega$.
    \item Stessa fase iniziale $\phi_0 = 0$.
\end{itemize}

Consideriamo i raggi $r_1, \ldots, r_N$ dalle sorgenti allo stesso punto $P$ posto su un ulteriore schermo paralleo al primo e a distanza $L \gg N d$ dalle fenditure.

I campi elettrici in $P$ sono, per $i = 1, \ldots, N$:
\begin{equation}
    \E_i(P, t) = \E_0 \exp{\im (k r_i - \omega t)}
\end{equation}

In virtù del principio di sovrapposizione degli effetti, il campo elettrico totale in $P$ è
\begin{equation}
    \E(P, t) = \E_0 \sum_{i = 1}^N \exp{\im (k r_i - \omega t)}
\end{equation}

Si definiscono le fasi
\begin{equation}
    \phi_i = k r_i - \omega t
\end{equation}
Le differenze di fase (ad esempio, tra le prime due, $\phi \coloneq \phi_2 - \phi_1 = k(r_2 - r_1)$) sono costanti nel tempo.
Inoltre, poiché le fenditure sono equispaziate, allora le differenze di fase tra onde successive sono approssimativamente uguali tra loro (e uguali a $\phi$ come appena definita).

\addsvg[][0.7]{efield_sum}

I campi elettrici $\E_i(P, t)$ si possono rappresentare sul piano complesso e, visto che la differenza di fase è costante, si fissa $t$ in modo da mette $\E_1$ sull'asse reale per semplicità.

Per sommare dei complessi, disponiamo i vettori in punta-coda e consideriamo il circocentro delle punte e code.
Detto $R$ il raggio della circoscritta, si può ottenere l'ampiezza $E_R$ del campo elettrico risultante:
\begin{equation}
    E_R = 2 R \sin\pts{\frac{N \phi}{2}}, \quad E_0 = 2 R \sin\frac{\phi}{2}
    \implies E_R = E_0 \frac{\sin N \frac{\phi}{2}}{\sin \frac{\phi}{2}}
\end{equation}

L'intensità, quindi, è
\begin{equation}
    I(P) = \avg{\norm{\poy(P)}} = \frac{E_R^2}{2 \mu_0 c} = \frac{E_0^2}{2 \mu_0 c} \frac{\sin^2 N\frac{\phi}{2}}{\sin^2 \frac{\phi}{2}} = I_0 \frac{\sin^2 N\frac{\phi}{2}}{\sin^2 \frac{\phi}{2}}
\end{equation}

Inoltre, poiché $L \gg d$, definendo $\theta \in [-\pi / 2, \pi / 2]$ come l'angolo a cui si trova $P$ rispetto alla perpendicolare allo schermo,
\begin{gather}
    r_2 - r_1 \approx d \sin \theta \\
    \implies \frac{\phi}{2} = \frac{k (r_2 - r_1)}{2} = \frac{\pi}{\lambda} (r_2 - r_1) \approx \pi \frac{d}{\lambda} \sin \theta
\end{gather}

Quindi
\begin{equation}
\label{eq:reticolo}
    I(\theta) = I_0 \frac{\sin^2 \pts{\pi \frac{N d}{\lambda} \sin \theta}}{\sin^2 \pts{\pi \frac{d}{\lambda} \sin \theta}}
\end{equation}

Questa funzione quantifica quanta luce si vede sullo schermo in funzione di $\theta$.

Va notato che il tempo non compare nell'equazione, quindi questa intensità si osserva fissa sullo schermo.

Inoltre, le figure di interferenza sono distinte per ogni lunghezza d'onda proveniente dalla sorgente: le lunghezze d'onda maggiori vengono deviate di più.
Il prisma, invece, fa il contrario: devia maggiormente le lunghezze d'onda minori.

\subsection{Reticolo di interferenza}

Si parla di reticolo di interferenza quando si hanno $N$ sorgenti separate da una distanza $d$, ovvero esattamente il caso descritto dall'\cref{eq:reticolo}.

Si osservano:
\begin{itemize}
    \item Massimi principali in cui $I = N^2 I_0$ intervallati da regioni scure:
        \begin{equation}
            \sin\theta_M = \frac{\lambda}{d} m, \quad m \in \Z
        \end{equation}
        Sono i punti in cui numeratore e denominatore si annullano a dare una forma indeterminata $[0/0]$, che risulta essere un massimo.

        $m$ è detto \important{ordine del massimo}.
    \item $N - 1$ minimi in cui $I = 0$ tra ogni coppia di massimi principali:
        \begin{equation}
            \sin\theta_0 = \frac{\lambda}{N d} m', \quad
            m' \in \Z, \quad
            N \nmid m'
        \end{equation}
        Sono i punti in cui si annulla il numeratore ma non il denominatore, poiché in quei casi si ha un massimo principale.
    \item $N - 2$ massimi secondari tra ogni coppia di massimi principali:
        \begin{equation}
            \sin\theta_m \approx \frac{\lambda}{N d} \pts{m'' + \frac{1}{2}}, \quad
            m'' \in Z, \quad
            N \nmid m'', \, N \nmid m''+1
        \end{equation}
        Vanno esclusi quelli che sarebbero in prossimità di un massimo principale.
\end{itemize}

Infine,
\begin{gather}
    \abs{\sin\theta_M} < 1 \implies \abs{m} < \frac{d}{\lambda} \\
    \abs{\sin\theta_0} < 1 \implies \abs{m'} < \frac{N d}{\lambda}
\end{gather}
ovvero, esistono un numero finito di massimi e minimi.

\addfigure{book/figura_reticolo}

\subsection{Doppia fenditura}
\label{sec:doubleslit}

Con $N = 2$ e $d$ la distanza tra le fenditure, l'\cref{eq:reticolo} si semplifica come
\begin{equation}
    I = I_0 \frac{\sin^2 \phi}{\sin^2 \frac{\phi}{2}}
    = 4 I_0 \cos^2 \frac{\phi}{2}
    = 4 I_0 \cos^2 \pts{\pi \frac{d}{\lambda} \sin\theta}
\end{equation}

Si osservano:
\begin{itemize}
    \item Massimi principali in cui $I = 4 I_0$:
        \begin{equation}
            \sin\theta_M = \frac{\lambda}{d} m, \quad m \in \Z
        \end{equation}
        Si tratta dei punti in cui $\cos^2(\cdots) = 1$.
    \item Un minimo in cui $I = 0$ tra ogni coppia di massimi principali:
        \begin{equation}
            \sin\theta_0 = \frac{\lambda}{d} \pts{m + \frac{1}{2}}, \quad
            m \in \Z
        \end{equation}
        % $m'$ non può assumere valori multipli di $N$, poiché in quei casi si ha una forma indeterminata $[0/0]$ che in realtà è un massimo.
\end{itemize}

L'\important{esperimento di Young} consiste nel mostrare la figura di interferenza a bande luminose dovuta a una doppia fenditura.
Fu svolto all'inizio dell'Ottocento sulla base della teoria di Huygens (che risaliva al Seicento), poiché diventò possibile produrre onde coerenti.
Per ottenere delle onde coerenti, il fronte d'onda proveniente da una singola fenditura veniva diviso in due fronti d'onda, che quindi erano coerenti tra loro.
In questo modo si mostrava la natura ondulatoria della luce.


\section{Singola fenditura}

Considerando una singola fenditura di ampiezza $a$ molto grande, un'onda piana che le vada contro passerebbe oltre.
Diminuendo $a$, inizialmente la zona illuminata diminuisce.
Al di sotto di una certa dimensione (quando $a$ è confrontabile della lunghezza d'onda) la fenditura si comporta come una sorgente puntiforme e illumina tutto lo schermo.

Huygens aveva osservato questo fenomeno con le onde del mare.

\important{Principio di Huygens}: ogni fronte d'onda si propaga come se ogni tratto infinitesimo del fronte d'onda fosse una sorgente puntiforme.

\addsvg[][0.7]{efield_sum_single_slit}

Applicando questo principio a una piccola fenditura, ogni punto diventa sorgente di un'onda con ampiezza $\de\E$ e differenza di fase $\de\phi$ con la sorgente successiva.
Sommando i fasori come prima, la spezzata diventa un arco di circonferenza su cui insiste un angolo $\phi = k a \sin\theta$.
Sia $E_\text{max}$ la lunghezza dell'arco di circonferenza:
\begin{gather}
    E_R = 2 R \sin \frac{\phi}{2}, \quad
    E_\text{max} = R \phi
    \implies
    E_R = E_\text{max} \frac{\sin\frac{\phi}{2}}{\frac{\phi}{2}} \\
    I = I_0 \pts{\frac{\sin\pts{\pi \frac{a}{\lambda} \sin\theta}}{\pi \frac{a}{\lambda} \sin\theta}}^{\!2}
\end{gather}

Si osservano:
\begin{itemize}
    \item Minimi in cui $I = 0$:
        \begin{equation}
            \sin \theta_0 = \pm \frac{\lambda}{a} m, \quad
            m \in \Z^+
        \end{equation}
    \item Un massimo centrale in $\theta = 0$ in cui $I = I_0$.
    Questo massimo identifica una regione luminosa delimitata dai due minimi di ordine 1, cioè per
        \begin{equation}
            -\frac{\lambda}{a} < \theta < \frac{\lambda}{a}
        \end{equation}
    \item Flebili massimi tra i minimi:
        \begin{equation}
            \sin \theta_M \approx \pm \frac{\lambda}{a} \pts{m + \frac{1}{2}}, \quad
            m \in \Z^+
        \end{equation}
\end{itemize}

\addfigure[Figura di diffrazione per una fenditura di ampiezza $b$.][0.7]{book/figura_singola}

Questo fenomeno è detto \important{diffrazione} e si verifica sempre, anche per $a / \lambda$ molto grande o molto piccolo.
\begin{itemize}
    \item Se $a \gg \lambda$ si vede solo e distintamente il massimo centrale e l'ombra intorno.
    \item Per $a \to \lambda^+$, $\theta_0 \to \pm \pi/2$, cioè la fenditura tende a illuminare tutto lo schermo.
    \item Per $a \le \lambda$, la fenditura tende a diventare una sorgente puntiforme.
\end{itemize}
Negli ultimi due casi non ci sono minimi.
Per questo, solitamente, si parla di diffrazione solo per $a > \lambda$.

La diffrazione è la ragione per cui non vediamo oggetti troppo piccoli: la pupilla funge da fenditura e la retina rileva solo la figura di diffrazione.
Esiste un limite alla capacità di risolvere i punti vicini, che si supera aumentando le dimensioni delle fenditure (siano queste pupille o lenti di telescopi).

\chapter{Interferenza e diffrazione -- esercitazioni}

% Definizione di intensità:
% \begin{equation}
%     I = \avg{\norm{\vt{S}}} = \frac{E^2}{2 \mu_0 c}
% \end{equation}

% Quando un onda piana incide uno schermo con $N$ fenditure puntiformi, tutte queste fenditure si comportano come sorgenti coerenti.

% Su un secondo schermo, si vede luce là dove le $N$ sorgenti hanno interferenza costruttiva.

% Ci poniamo nella condizione per cui lo schermo è a grande distanza $L$ dalle fenditure.

% Definendo $I_0$ il fattore di proporzionalità,
% \begin{equation}
% \label{eq:intensitynsource}
%     I = I_0 \frac{\sin^2\frac{N \pi d \sin\theta}{\lambda}}{\sin^2\frac{\pi d \sin\theta}{\lambda}}
% \end{equation}
% dove $d$ è la distanza tra le fenditure e $\theta$ è l'angolo rispetto all'orizzontale sullo schermo, l'unico parametro per determinare la posizione del punto di osservazione, per cui
% \begin{equation}
%     - \frac{\pi}{2} < \theta < \frac{\pi}{2}
% \end{equation}

% Definendo $x = \pi d \sin \theta / \lambda$, la funzione
% \begin{equation}
%     f(x) = \frac{\sin^2 N x}{\sin^2 x}
% \end{equation}
% determina il profilo dell'intensità.

% $f$ ha dei massimi quando il numeratore è massimo.

% Di questi, uno ogni $N$ è un massimo principale con valore $N^2 I_0$, che annulla anche il denominatore.

% Tra due massimi principali, ci sono $N - 1$ minimi in cui la funzione si annulla (là dove si annulla il numeratore ma non il denominatore).

% Massimi principali:
% \begin{equation}
%     d \sin \theta_M = m \lambda, \quad m \in \Z
% \end{equation}
% $m$ è detto \important{ordine del massimo}.

% Minimi ($N - 1$ tra ogni coppia di massimi principali):
% \begin{equation}
%     N d \sin \theta_0 = m' \lambda, \quad
%     m' \in \Z, \quad N \nmid m'
% \end{equation}

% Massimi secondari ($N - 2$ tra ogni coppia di massimi principali):
% \begin{equation}
%     N d \sin \theta_m = \pts{m'' + \frac{1}{2}} \lambda, \quad m'' \in \Z, \quad N, N-1 \nmid m''
% \end{equation}

Se sono richieste le posizioni $x$ sullo schermo, queste si trovano come
\begin{equation}
    x = L \tan \theta
\end{equation}

Se $L \gg \abs{x}$,
\begin{equation}
    x \approx L \sin\theta
\end{equation}
È per questo che i massimi e minimi delle figure di interferenza e diffrazione risultano equispaziati.

\section{Inteferenza tra due sorgenti}

Sorgenti isofrequenziali:
\begin{subequations}
\begin{gather}
    \E_1 = \E_{01} \sin(\vt{k}_1 \cdot \p_1 - \omega t + \phi_1) \\
    \E_2 = \E_{02} \sin(\vt{k}_2 \cdot \p_2 - \omega t + \phi_2)
\end{gather}
\end{subequations}

Differenza di fase:
\begin{equation}
    \phi \coloneq \vt{k}_1 \cdot \p_1 + \phi_1 - \vt{k}_2 \cdot \p_2 - \phi_2
\end{equation}

Intensità risultante:
\begin{equation}
    I_\text{tot} = I_1 + I_2 + 2 \sqrt{I_1 I_2} \cos \phi
\end{equation}

Se le sorgenti sono identiche,
\begin{gather}
    I_1 = I_2 \eqcolon I_0 \\
    \implies I_\text{tot} = 4 I_0 \frac{1 + \cos \phi}{2}
    = 4 I_0 \cos^2 \frac{\phi}{2}
\end{gather}

\section{Interferenza}

Esercizio: reticolo di diffrazione con $N = 100$ fenditure e una distanza tra le fenditure di $d = \qty{50}{\micro\metre}$, luce rossa con $\lambda_r = \qty{700}{\nano\metre}$ e luce blu con $\lambda_b = \qty{500}{\nano\metre}$, schermo posto a distanza $L = \qty{2}{\metre}$.
Quanto distano i due massimi di ordine $m = 1$ per i due colori?
\begin{equation}
\begin{gathered}
    \Delta x = x_r - x_b = L \pts{\tan \arcsin \theta_r - \tan \arcsin \theta_b} \approx \\
    \approx L \pts{\arcsin \theta_r - \arcsin \theta_b}
    = L \frac{\lambda_r}{d} m - L \frac{\lambda_b}{d} m
    = \qty{8}{\milli\metre}
\end{gathered}
\end{equation}

Esercizio: doppia fenditura (esperimento di Young), $d = \qty{1.5e-4}{\metre}$, $L = \qty{1.4}{\metre}$, $\lambda = \qty{643}{\nano\metre}$.
Nel punto sullo schermo in $x = \qty{1.8e-2}{\metre}$ si ha un massimo o un minimo?

$x \ll L$, quindi $\sin \theta \approx \tan \theta = x/L$:
\begin{equation}
    \cos^2\pts{\frac{\phi}{2}}
    = \cos^2\pts{\pi \frac{d}{\lambda} \sin\theta}
    \approx \cos^2\pts{\pi \frac{d}{\lambda} \frac{x}{L}} = 1
\end{equation}
L'interferenza è costruttiva.

Esercizio: doppia fenditura (esperimento di Young), $d = \qty{e-5}{\metre}$, $L = \qty{1}{\metre}$, $\lambda = \qty{500}{\nano\metre}$.
In che posizione $x$ si trova il terzo minimo di interferenza?

Con $m = 3$,
\begin{equation}
    x = L \tan \theta
    = L \tan \arcsin \pts{\frac{\lambda}{d} \pts{m - \frac{1}{2}}}
    = \qty{12.6}{\centi\metre}
\end{equation}
\textit{A posteriori}, si vede che si sarebbe anche potuta usare l'approssimazione per angoli piccoli.

Esercizio: reticolo di diffrazione con $4000$ tratti al centimetro.
Il massimo di secondo ordine ($m = 2$) è deviato di $\theta = \qty{34}{\degree}$.
Quanto vale $\lambda$?
\begin{gather}
    d = \pts{\qty{4000}{\per\centi\meter}}^{-1} = \qty{2.5}{\micro\metre} \\
    \lambda = \frac{d \sin{\theta}}{m} = \qty{699}{\nano\metre}
\end{gather}

Esercizio: quattro antenne in fila a una distanza l'una dall'altra di $d = \qty{10}{\metre}$ irradiano (in un mezzo isotropo) una potenza $P = \qty{100}{\kilo\watt}$ a frequenza $\nu = \qty{30}{\mega\hertz}$.
Come varia l'intensità a una distanza $L = \qty{10}{\kilo\metre}$?

Si usa l'\cref{eq:reticolo}.

Poiché il mezzo è isotropo, le sorgenti irradiano onde sferiche.
Allora, l'intensità dovuta a una singola sorgente è
\begin{equation}
    I_0 = \frac{P}{4\pi L^2} = \qty{7.96e-5}{\watt\per\meter\squared}
\end{equation}

I massimi principali sono ad angoli
\begin{equation}
    \theta = \arcsin \frac{m \lambda}{d}
    = \arcsin \frac{m c}{\nu d}
    = \arcsin m
\end{equation}
Poiché $-\pi/2 \le \theta \le \pi/2$, $m \in \{-1, 0, 1\}$.
Gli unici quattro massimi principali sono nelle quattro direzioni rispetto all'allineamento delle antenne.

I minimi saranno $4 - 1 = 3$ per quadrante, in angoli
\begin{gather}
    \theta
    = \arcsin \frac{m' c}{4 \nu d}
    = \arcsin \frac{m'}{4} \\
    \theta_1 = \qty{14.5}{\degree}, \quad
    \theta_2 = \qty{30.0}{\degree}, \quad
    \theta_3 = \qty{48.6}{\degree}
\end{gather}

I massimi secondari saranno due per quadrante:
\begin{gather}
    \theta
    = \arcsin \frac{(m'' + 0.5) c}{4 \nu d}
    = \arcsin \frac{2m'' + 1}{8} \\
    \theta_1 = \qty{22.0}{\degree}, \quad
    \theta_2 = \qty{38.7}{\degree}
\end{gather}

\addfigure[Pattern di radiazione.][0.6]{book/quattro_antenne}

Esercizio: reticolo di diffrazione con $1000$ fenditure al centimetro. Due onde incidenti, $\lambda_1 = \qty{650}{\nano\metre}$ e $\lambda_2 = \qty{640}{\nano\metre}$.
Qual è la separazione angolare tra i due massimi del primo ordine?
\begin{gather}
    d = \pts{\qty{1000}{\per\centi\meter}}^{-1} = \qty{e-5}{\metre} \\
    \Delta \theta = \theta_1 - \theta_2
    \approx \frac{m \lambda_1}{d} - \frac{m \lambda_2}{d}
    = \frac{m}{d}(\lambda_1 - \lambda_2)
    = \qty{e-3}{\radian}
\end{gather}

È possibile osservare la separazione tra i due picchi?

\section{Risoluzione}

\important{Criterio di Rayleigh}: è possibile distinguere due massimi se il secondo cade al di là dei minimi che circondano il primo, o viceversa.
\begin{gather}
    \theta_{\text{max}1} = \frac{\lambda_1}{d}
    \qquad
    \theta_{\text{max}2} = \frac{\lambda_2}{d} \\
    \theta_{\text{min}1^+} = \frac{N+1}{N} \frac{\lambda_1}{d}
    \qquad
    \theta_{\text{min}2^+} = \frac{N+1}{N} \frac{\lambda_2}{d} \\
    \theta_{\text{min}1^-} = \frac{N-1}{N} \frac{\lambda_1}{d}
    \qquad
    \theta_{\text{min}2^-} = \frac{N-1}{N} \frac{\lambda_2}{d}
\end{gather}

Quindi, non è indifferente quale onda scelgo per i minimi.

Sia $\lambda_1 < \lambda_2$.
Le due opzioni sono:
\begin{itemize}
    \item massimo di 2 sul minimo di destra di 1:
    \begin{equation}
        \frac{\lambda_2}{d} = \frac{N+1}{N} \frac{\lambda_1}{d}
        = \frac{\lambda_1}{d} + \frac{\lambda_1}{N_{L1} d}
    \end{equation}
    \item massimo di 1 sul minimo di sinistra di 2:
    \begin{equation}
        \frac{\lambda_1}{d} = \frac{N-1}{N} \frac{\lambda_2}{d}
        = \frac{\lambda_2}{d} - \frac{\lambda_2}{N_{L2} d}
    \end{equation}
\end{itemize}
$N_{L1}$ e $N_{L2}$ sono gli $N$ limite per soddisfare le due condizioni.

Risulta
\begin{subequations}
\begin{gather}
    N_{L1} = \frac{\lambda_1}{\lambda_2 - \lambda_1} \\
    N_{L2} = \frac{\lambda_2}{\lambda_2 - \lambda_1}
\end{gather}
\end{subequations}

Si sceglie di considerare
\begin{equation}
    N_L = \frac{N_{L1} + N_{L2}}{2} = \frac{\frac{\lambda_1 + \lambda_2}{2}}{\lambda_2 - \lambda_1}
    = \frac{\lambda}{\lambda_2 - \lambda_1}
\end{equation}

Esercizio: reticolo con $N = 50$ fenditure e $\lambda_1 = \qty{400}{\nano\metre}$.
Quali valori può assumere $\lambda_2$ affinché non si possano distinguere i massimi di secondo ordine?
\begin{equation}
    \lambda_2 \in \pts{\frac{N m - 1}{N m} \lambda_1, \, \frac{N m + 1}{N m} \lambda_1}
    = \pts{\qty{396}{\nano\metre}, \qty{404}{\nano\metre}}
\end{equation}

\section{Diffrazione}

% Si parla di diffrazione quando si ha una singola fenditura di ampiezza $a > \lambda$.

% Nell'ipotesi che $a \ll L$, la distribuzione dell'intensità sarà
% \begin{equation}
%     I_\text{tot} = I_0 \pts{\frac{\sin \frac{\pi a \sin\theta}{\lambda}}{\frac{\pi a \sin\theta}{\lambda}}}^2
% \end{equation}

% I massimi diversi da quello centrale sono subito molto flebili.

% Minimi:
% \begin{equation}
%     a \sin\theta = m \lambda, \quad
%     m \in \Z, \, m \ne 0
% \end{equation}
% Se $a < \lambda$, non ci sono minimi e non si parla di diffrazione.

% Massimi secondari:
% \begin{equation}
%     a \sin\theta = \pts{m' + \frac{1}{2}} \lambda, \quad
%     m' \in \Z, \, m' \ne 0
% \end{equation}

Esercizio: $a = \qty{6e-4}{\metre}$, $\lambda = \qty{400}{\nano\metre}$, $L = \qty{1.5}{\metre}$.
A che angolo e in che posizione si trova il minimo di ordine 2?
\begin{gather}
    \theta = \arcsin \frac{m \lambda}{a} = \qty{1.33e-3}{\radian} \\
    x = L \tan \theta \approx L \sin \theta = \qty{2}{\milli\metre}
\end{gather}

L'intensità nel caso del reticolo di diffrazione con $a > \lambda$ va descritta considerando sia la diffrazione che l'interferenza:
\begin{equation}
    I_\text{tot} = I_0
    \underbrace{\pts{\frac{\sin\pts{\pi \frac{a}{\lambda} \sin\theta}}{\pi \frac{a}{\lambda} \sin\theta}}^2}_\text{diffrazione}
    \underbrace{\frac{\sin^2 \pts{N \pi \frac{d}{\lambda} \sin \theta}}{\sin^2 \pts{\pi \frac{d}{\lambda} \sin \theta}}}_\text{interferenza}
\end{equation}

\addfigure[][0.8]{book/figura_interferenza_e_diffrazione}

Esercizio: doppia fenditura con $d = \qty{30}{\micro\metre}$, $a = \qty{3}{\micro\metre}$, $\lambda = \qty{500}{\nano\metre}$.
A occhio nudo si vedono solo le regioni con $I > \qty{5}{\percent} I_\text{max}$.
Quanti picchi si vedono?

Capiamo quali massimi di diffrazioni rispettano la condizione richiesta.
Per $m = 1$,
\begin{equation}
    \frac{I}{N^2 I_0} = \pts{\frac{\sin\pts{\pi \frac{a}{\lambda} \sin\theta}}{\pi \frac{a}{\lambda} \sin\theta}}^2
    = \pts{\frac{\sin \frac{3}{2} \pi}{\frac{3}{2} \pi}}^2
    = \qty{4.5}{\percent} < \qty{5}{\percent}
\end{equation}
Quindi, si studia solo il massimo centrale di diffrazione.

Il primo minimo di diffrazione è in $\theta$ tale che $\sin \theta = \lambda / a$.

L'ordine $m$ maggiore del massimo di interferenza che sta entro il minimo di diffrazione si trova imponendo
\begin{equation}
    \frac{\lambda}{a} > m \frac{\lambda}{d}
    \implies
    m_\text{max} = \left\lceil \frac{d}{a} \right\rceil - 1 = 9
\end{equation}
Quindi, i massimi candidati sono $1 + 2 \cdot 9 = 19$.

Cercando caso per caso, si osserva che il massimo di ordine 8 dà $\qty{5.5}{\percent}$ e quello di ordine 9 dà $\qty{1.2}{\percent}$.
Quindi, si vedono $1 + 2 \cdot 8 = 17$ massimi.

\chapter{Relatività -- esercitazioni}
\label{sec:relativita}

\section{Esperimento di Michelson e Morley}

La differenza di fase tra due onde coerenti in un certo punto a distanza $r_1$ dalla prima sorgente e a $r_2$ dalla seconda è
\begin{equation}
    \Delta \phi = (k r_2 - \omega t + \phi_2) - (k r_2 - \omega t + \phi_1) = k (r_2 - r_1) + \phi_2 - \phi_1
\end{equation}

\redtext{Immagine interferometro}

La lastra D serve a compensare la dispersione del raggio verso l'alto dovuta alla riflessione presso la lastra B.

La direzione orizzontale è coerente con la velocità della Terra, quindi l'interferenza risultante doveva dipendere dalla velocità della terra, che determina la differenza dei cammini.

Riguardo il raggio $L_2$:
\begin{gather}
    (c t_3)^2 = L_2^2 + (u t_3)^2 \implies t_3 = \frac{L}{\sqrt{c^2 - u^2}} \\
    t_4 = t_3 \\
    \implies \Delta t_2 = t_3 + t_4  = \frac{2 L}{c} \frac{1}{\sqrt{1 - \frac{u^2}{c^2}}}
\end{gather}

Riguardo il raggio $L_1$
\begin{gather}
    c t_1 = L_1 + u t_1 \implies t_1 = \frac{L}{c - u} \\
    c t_2 = L_1 - u t_2 \implies t_2 = \frac{L}{c + u} \\
    \implies \Delta t_1 = t_1 + t_2 = \frac{2 L}{c} \frac{1}{1 - \frac{u^2}{c^2}}
\end{gather}

Sperimentalmente, però, non si osservò nessuno sfasamento.

Questo perché la composizione delle velocità non avviene come previsto dalle trasformazioni di Galileo.

Si definisce il \important{fattore di Lorentz}:
\begin{gather}
    \gamma(v) \coloneq \frac{1}{\sqrt{1 - \frac{v^2}{c^2}}}
    = 1 + \frac{1}{2} \frac{v^2}{c^2} + o\pts{\frac{v^2}{c^2}}, \quad \text{per } \frac{v}{c} \to 0 \\
    \abs{v} \ll c \implies \gamma \approx 1
\end{gather}

Ma se $c$ è invariante rispetto ai sistemi di riferimento, allora non può esserlo lo spazio.
Ipotizzando la seguente contrazione delle lunghezze per i sistemi in moto a velocità $v$,
\begin{equation}
    L' = \frac{L}{\gamma}
\end{equation}
Effettivamente si ottiene l'uguaglianza tra $\Delta t_1$ e $\Delta t_2$:
\begin{equation}
    \implies \Delta t_1 = \frac{2 L'}{c} \frac{1}{1 - \frac{u^2}{c^2}}
    = \frac{2 L}{c} \frac{\sqrt{1 - \frac{u^2}{c^2}}}{1 - \frac{u^2}{c^2}}
    = \frac{2 L}{c} \frac{1}{\sqrt{1 - \frac{u^2}{c^2}}}
    = \Delta t_2
\end{equation}
Ne segue anche una dilatazione dei tempi:
\begin{equation}
    \Delta t = \gamma \Delta t'
\end{equation}

\section{Trasformazioni di Galileo}

Consideriamo due sistemi di riferimento $S_1: x, y, z$ e $S_2: x', y', z'$.
$S_2$ è in moto con velocità $\vt{v}$ rispetto a $S_1$.
In fisica classica valgono le trasformazioni di Galileo:
\begin{itemize}
    \item Osservatore 1:
        \begin{equation}
            \begin{cases}
                x = x' + v_x t \\
                y = y' + v_y t \\
                z = z' + v_z t \\
                t = t'
            \end{cases}
        \end{equation}
    \item Osservatore 2:
        \begin{equation}
            \begin{cases}
                x' = x - v_x t \\
                y' = y - v_y t \\
                z' = z - v_z t \\
                t' = t
            \end{cases}
        \end{equation}
\end{itemize}

Le equazioni di Maxwell non sono compatibili con queste trasformazioni.

Consideriamo un filo rettilineo con densità di carica $\lambda$ e, a distanza $r$, una carica $q$.
La forza di Coulomb è
\begin{equation}
    F = \frac{q \lambda}{2\pi \eps_0 r}
\end{equation}
Tuttavia, da un sistema di riferimento in moto a velocità $v$, vedo il filo carico in moto, e dunque una corrente elettrica $\lambda v$ e una forza magnetica attrattiva dovuta a un campo magnetico $\mu_0 \lambda v / (2\pi r)$:
\begin{equation}
\label{eq:frel}
    F = \frac{q \lambda}{2\pi \eps_0 r} - q v \frac{\mu_0 \lambda v}{2\pi r}
    = \frac{q \lambda}{2\pi r \eps_0} \pts{1 - \frac{v^2}{c^2}}
    = \frac{q \lambda}{2\pi r \eps_0} \gamma^{-2}
\end{equation}

Poiché le equazioni di Maxwell non sono valide in ogni sistema di riferimento inerziale, vorrebbe dire che ne esiste uno privilegiato.

In realtà, risulta che le trasformazioni di Galileo valgono solo a velocità $\abs{v} \ll c$ e sono un caso limite delle trasformazioni di Lorentz.
Con $\vt{v} = v \ux$,
\begin{itemize}
    \item Osservatore 1:
    \begin{equation}
        \begin{cases}
            x = \gamma (x' + v t) \\
            t = \gamma \pts{t' + \dfrac{v}{c^2} x'}
        \end{cases}
    \end{equation}
    \item Osservatore 2:
    \begin{equation}
        \begin{cases}
            x' = \gamma (x - v t) \\
            t' = \gamma \pts{t - \dfrac{v}{c^2} x}
        \end{cases}
    \end{equation}
\end{itemize}

Con queste trasformazioni, $c$ è la stessa in tutti i sistemi di riferimento.

Immaginiamo due specchi paralleli a distanza $d$ con un raggio di luce che va da uno all'altro dei due.

In un sistema di riferimento in cui il raggio è verticale, il raggio percorre una distanza $d$ in un tempo $\Delta t$.

In uno in cui il raggio ha un angolo, questo percorre una distanza $L > d$ in un tempo $\Delta t'$.

Il cateto parallelo agli specchi si misura nel sistema di riferimento in moto.

Inoltre,
\begin{gather}
    L^2 = d^2 + (v \Delta t')^2 \\
    \implies c^2 \Delta t'^2 = c^2 \Delta t^2 + v^2 \Delta t'^2 \\
    \implies \Delta t'^2 = \Delta t^2 + \frac{v^2}{c^2} \Delta t'^2 \\
    \implies \Delta t' = \gamma \Delta t > \Delta t
\end{gather}

Si ha una \important{dilatazione dei tempi} e si perde il concetto di simultaneità degli eventi: lo stesso evento può avvenire in momenti diversi a seconda dell'osservatore.

Analogamente, vale la \important{contrazione delle lunghezze}:
\begin{equation}
    c = \frac{L}{\Delta t} = \frac{L'}{\Delta t'}
    \implies L' = \frac{L}{\gamma} < L
\end{equation}

È solo per $v > c/2$ circa che $\gamma$ diventa percettibilmente maggiore di 1.
La relatività, quindi, vale solo per velocità elevatissime.

Per quanto riguarda l'esempio precedente, a dover essere corrette sono
\begin{itemize}
    \item la densità di carica, poiché è diversa la lunghezza:
    \begin{equation}
        \lambda' = \gamma \lambda
    \end{equation}
    \item la forza, che è derivata della quantità di moto rispetto al tempo:
    \begin{equation}
        \Delta p' = \frac{q \lambda'}{2\pi \eps_0 r} \gamma^{-2} \Delta t'
        = \frac{q \lambda \gamma}{2\pi \eps_0 r} \gamma^{-2} \gamma \Delta t
        = \frac{q \lambda}{2\pi \eps_0 r} \Delta t
        = \Delta p
    \end{equation}
\end{itemize}

In conclusione, il comportamento del sistema non cambia tra i due sistemi di riferimento.

\section{Equivalenza massa-energia}

Nel sistema $O$, due onde elettromagnetiche colpiscono una particella di massa $m$, propagandosi l'una contro l'altra in verticale.
In un sistema $O'$ in moto $v$ orizzontale, i raggi sono a un angolo $\alpha$ con la verticale.

In $O$, la particella non cambia la sua quantità di moto.

In $O'$, le quantità di moto iniziale e finale sono
\begin{gather}
    p_i = m v_i + 2 \frac{E}{c} \sin\alpha \\
    p_f = m v_f
\end{gather}

Poiché in $O$ la particella non cambia velocità, non deve farlo neanche in $O'$.

Del resto, come in $O$, anche in $O'$ deve valere $p_i = p_f$.

Si ottiene l'assurdo $E = 0$.

L'errore risiede nell'assumere che la massa sia la stessa.
Invece,
\begin{equation}
    p_f = p_i, \, v_f = v_i \eqcolon v
    \implies m_f v = m_i v + 2 \frac{E}{c} \sin\alpha
\end{equation}

Usando anche $\sin \alpha = v/c$,
\begin{equation}
    m_f = m_i + 2 \frac{E}{c^2}
    \implies
    2E = (m_f - m_i) c^2
\end{equation}
La massa della particella aumenta di una quantità equivalente all'energia delle due onde.

In generale, la massa finale dopo un trasferimento di energia che porta il corpo a velocità $v$ per una particella di massa a riposo $m_0$ è
\begin{equation}
    m = \gamma m_0
\end{equation}
Nel limite $v \ll c$,
\begin{equation}
    m = m_0 + \frac{1}{c^2} \frac{1}{2} m_0 v^2 + o\pts{\frac{v^2}{c^2}}
\end{equation}
cioè si ottiene l'espressione dell'energia cinetica

\section{Paradosso dei gemelli}

Di due gemelli di 20 anni sulla Terra uno parte per un lontano pianeta, viaggiando a velocità $v = 0.8 c$ per un tempo $\Delta t = \qty{10}{anni}$.

Considerando andata e ritorno, al ritorno il gemello rimasto sulla terra avrà 40 anni.
Il gemello che torna, invece, avrà 32 anni.

Tuttavia, nel sistema di riferimento dell'astronave, la situazione dovrebbe essere opposta: è il gemello sulla terra a essere invecchiato meno, poiché è la Terra a essere in moto rispetto all'astronave.

Quindi, dovrei essere in grado di scegliere un sistema di riferimento corretto, mentre l'altro sarà sbagliato (perché uno dei due gemelli è effettivamente invecchiato di più).

Il problema sta nel presupporre il concetto di simultaneità.

Siano $T: (x, t)$ il sistema di riferimento della Terra, $A: (x', t')$ quello dell'andata e $R: (x'', t'')$ quello del ritorno.

Il tempo in cui il gemello che parte arriva sul pianeta è diverso in $T$ e in $A$ e quello in cui il gemello arriva e inizia a tornare è diverso tra $A$ e $R$.

\chapter{Inadeguatezza della fisica classica}

Fine Ottocento, nel contesto della seconda Rivoluzione industriale e dopo aver definito le equazioni di Maxwell.
Fino ad allora, le teorie fisiche (di Newton e Maxwell) non avevano fallito.

La meccanica classica descrive oggetti materiali che possono essere descritti come scomponibili e portano proprietà che possono essere descritte come onde.

L'elettromagnetismo introduce la carica, una proprietà ulteriore.
Le interazioni si risolvono comunque in forze, che quindi possono essere meccaniche o elettromagnetiche.

Essendo un onda, il campo elettromagnetico lo si interpreta come perturbazione di un mezzo chiamato ``etere''.

L'esperimento di Michelson e Morley, tuttavia, mostra che questo tessuto connettivo non esiste.

Inoltre, nelle equazioni di Maxwell esiste un parametro velocità ($c$) che non cambia mai.
Un dipolo in moto e uno in quiete generano campi elettromagnetici che si muovono alla stessa velocità: la velocità della luce non si compone secondo la relatività galileiana.

Einstein derivò le trasformazioni di Lorentz partendo dalla meccanica e non dall'elettromagnetismo, principalmente perché la meccanica era ritenuta vera da tre secoli mentre l'elettromagnetismo era una teoria  più nuova.
Il ragionamento dalla meccanica, però, è più complicato.

Caveat: l'applicazione di una teoria (che è assoluta, matematica) presuppone la modellizzazione del sistema fisico.
I fisici, per questo, pensavano che non fossero sbagliate le teorie, ma i modelli della materia.


\section{Corpo nero}

Tutti i corpi a temperatura finita ($T > 0$) emettono un campo elettromagnetico, poiché le cariche nella materia si muovono di moto accelerato (con oscillazioni meccaniche).

Quando questa radiazione diventa visibile si ha incandescenza e all'aumentare della temperatura il colore passa dal rosso al giallo, perché aumenta la frequenza delle onde emesse.

Maxwell definì il corpo nero per modellizzare una lampadina a incandescenza.

\important{Corpo nero}: cavità di conduttore ideale (pareti perfettamente riflettenti) a temperatura fissa.

Il campo elettromagnetico emesso dalle pareti è continuamente riassorbito e riemesso (cioè, riflesso), quindi \important{è in equilibrio termodinamico con le pareti} stesse.

Per misurare l'energia emessa, il corpo nero è dotato di un piccolo foro (sufficientemente piccolo da non perturbare l'equilibrio).

L'energia emessa viene misurata per ogni frequenza.
L'energia per unità di volume a una certa frequenza si scrive come
\begin{equation}
\label{eq:densita_frequenza}
    I(\nu) \de \nu = g(\nu) \avg{\energy} \de \nu
\end{equation}
dove $g(\nu) \de \nu$ indica il numero di onde a frequenza $\nu$ per unità di volume e $\avg{\energy}$ è l'energia media di un onda a frequenza $\nu$.

Assumiamo che il corpo nero sia un cubo di lato $L$.
Lungo ogni direzione si avranno un'onda progressiva e una regressiva con stessa frequenza e polarizzazione.

Per il principio di sovrapposizione degli effetti, nella direzione $x$,
\begin{equation}
    E(x, t) = E_p \exp{\im (k x - \omega t)} + E_r \exp{-\im (k x + \omega t)}
\end{equation}

Presso il conduttore ci sarebbe dissipazione di energia per effetto Joule, a meno che
\begin{equation}
    E(0, t) = E(L, t) \equiv 0
\end{equation}

Imponendo queste condizioni al contorno, risulta
\begin{gather}
    E_p + E_r = 0 \\
    \exp{\im k L} = \exp{-\im k L} \implies \frac{k L}{\pi} \in \Z
\end{gather}

Riassumendo, lungo la direzione $x$,
\begin{equation}
    k_x = \frac{\pi}{L} n_x, \quad n_x \in \Z^+
\end{equation}
Ci limitiamo a $n_x > 0$ poiché $n_x = 0$ non ha senso e i valori negativi corrispondono a scambiare le onde progressive e regressive. Inoltre, onde che avessero $k$ diverso si eliderebbero per interferenza distruttiva.

Per contare le onde, si ragiona nello \important{spazio reciproco}, con i numeri d'onda presso gli assi $k_x$, $k_y$, $k_z$.

Ogni punto dello spazio reciproco corrisponde a un vettore d'onda.

Le onde ammesse sono quelle con vettore d'onda $\vt{k} = (n_x, n_y, n_z) \pi/L$, nell'ottante positivo. A ogni onda corrisponde un punto, a cui corrisponde (asintoticamente) un cubetto.

A ogni frequenza corrisponde un modulo di $\vt{k}$.
Il numero di onde di frequenza compresa tra $0$ e $\nu$ è il numero di cubetti di lato $\pi / L$ in un'ottante di sfera di raggio $k = \norm{\vt{k}} = 2\pi \nu / c$.
In realtà, queste sono solo le onde polarizzate in una certa direzione.
Serve almeno un'altra direzione di polarizzazione per ottenere tutte le onde, quindi occorre moltiplicare per due.
\begin{equation}
    N = 2 \cdot \frac{1}{8} \cdot \frac{\frac{4}{3} \pi k^3}{\pts{\frac{\pi}{L}}^3}
    % = \frac{1}{3 \pi^2 c^3} \omega^3 L^3
    = \frac{8 \pi}{3 c^3} \nu^3 L^3
\end{equation}

La densità, rispetto alla frequenza, di onde per unità di volume è:
\begin{equation}
    g(\nu) = \der{\nu} \pts{\frac{N}{L^3}}
    = \frac{8 \pi}{c^3} \nu^2
\end{equation}
% \begin{equation}
%     g(\omega) = \der{\omega} \pts{\frac{N}{L^3}}
%     = \frac{1}{\pi^2 c^3} \omega^2
% \end{equation}

\Cref{eq:densita_frequenza} ora è
\begin{equation}
    I(\nu) = \frac{8 \pi}{c^3} \nu^2 \avg{\energy}
\end{equation}

Poiché l'onda viene generata da un'oscillazione armonica di cariche, la sua energia deve corrispondere all'energia meccanica (potenziale e cinetica) media di un oscillatore armonico a temperatura $T$.

La frequenza $\omega$ degli oscillatori è un parametro puramente meccanico e dipende dalla costante elastica e dalla massa delle cariche ($k = m \omega^2$).

L'energia media del sistema termodinamico è
\begin{equation}
\label{eq:energia_media}
    \avg{\energy} = \int_0^{+\infty} \energy \, F(\energy) \, \de \energy
\end{equation}
dove $F(\energy) \, \de \energy$ è la frazione di particelle che hanno energia in $[\energy, \, \energy + \de \energy]$.

La densità $F$ è funzione della velocità, poiché lo è l'energia: $F = F(v_x, v_y, v_z)$.
Queste tre componenti sono statisticamente indipendenti, per cui esistono densità $f$ per cui
\begin{equation}
    F(v_x, v_y, v_z) = f_x(v_x) f_y(v_z) f_z(v_z)
\end{equation}
Poiché lo spazio è isotropo, $f_x = f_y = f_z \eqcolon f$.

Per la stessa ragione, poiché non può dipendere dalla scelta delle coordinate, $F$ deve dipendere solo dal modulo di $\vt{v}$ o, equivalentemente, dal suo quadrato, cioè $v_x^2 + v_y^2 + v_z^2$:
\begin{equation}
    F(v_x^2 + v_y^2 + v_z^2) = f(v_x) f(v_z) f(v_z)
\end{equation}
L'unica funzione che soddisfa è l'esponenziale:
\begin{equation}
    F(\vt{v}) = A \exp{B \pts{v_x^2 + v_y^2 + v_z^2}}
\end{equation}
con $\dimension{B} = \mathsf{T}^2 \mathsf{L}^{-2}$.

Nel caso del gas ideale,
\begin{gather}
    \int_0^{+\infty} \frac{1}{2} m \vt{v}^2 F(\vt{v}) \, \de \vt{v} = \frac{3}{2} k_B T \\
    \implies F(\vt{v}) = A \exp{-\frac{m}{2 k_B T} \vt{v}^2}
\end{gather}
È la \important{distribuzione di Maxwell}.
In termini di energia ($A$ è diversa), si chiama \important{distribuzione di Boltzmann} ed è soddisfatta da qualunque sistema termodinamico all'equilibrio:
\begin{gather}
    F(\energy) = A \exp{-\frac{\energy}{k_B T}} \\
    \int_0^{+\infty} F(\energy) \de \energy = 1
    \implies A = \frac{1}{\int_0^{+\infty} \exp{-\frac{\energy}{k_B T}} \de \energy}
\end{gather}

Segue che
\begin{equation}
    \avg{\energy} = \int_0^{+\infty} A \energy \exp{-\frac{\energy}{k_B T}} \de \energy
    = \frac{\int_0^{+\infty} \energy \exp{-\frac{\energy}{k_B T}} \de \energy}{\int_0^{+\infty} \exp{-\frac{\energy}{k_B T}} \de \energy}
    = k_B T
\end{equation}

Infatti, classicamente a ogni grado di libertà (cioè a ogni termine quadratico nell'espressione dell'energia meccanica totale) corrisponde un'energia $\frac{1}{2} k_B T$, e l'oscillatore armonico monodimensionale ha due gradi di libertà: energia cinetica e potenziale.

\Cref{eq:densita_frequenza} ora è
\begin{equation}
    I(\nu) = \frac{8\pi}{c^3} k_B T \nu^2
\end{equation}

Sperimentalmente, $I(\nu)$ ha una forma a campana.
Questa legge, invece, indica una parabola ed è adeguata solo per la regione a frequenze basse (fin quasi al massimo della curva sperimentale).

Nelle curve sperimentali si osserva anche che il massimo di $I(\nu)$ è direttamente proporzionale alla temperatura.

Le frequenze in cui si trova il massimo se la temperatura è circa quella della superficie solare sono quelle della luce visibile.

Planck, per far tornare i risultati teorici con i dati sperimentali, ipotizza la seguente distribuzione di energia:
\begin{equation}
    \energy = j h \nu, \quad j \in \N
\end{equation}

Sostituendo nell'\cref{eq:energia_media} e passando al discreto:
\begin{equation}
    \avg{\energy} = \sum_{j = 0}^{\infty} j h \nu A \exp{-\frac{j h \nu}{k_B T}}
\end{equation}
Si impone la condizione di normalizzazione sulla distribuzione $F$:
\begin{equation}
    \sum_{j = 0}^{\infty} A \exp{-\frac{j h \nu}{k_B T}} = 1
    \implies A = \frac{1}{\sum_j \exp{-\frac{j h \nu}{k_B T}}}
\end{equation}
Dunque, definendo $\beta \coloneq 1/(k_B T)$:
\begin{equation}
    \avg{\energy} = \frac{\sum_j j h \nu \exp{-\beta j h \nu}}{\sum_j \exp{-\beta j h \nu}}
\end{equation}

Osserviamo che la derivata rispetto a $\beta$ del denominatore è
\begin{equation}
    \parder{\beta} \sum_{j = 0}^\infty \exp{-\beta j h \nu}
    = - \sum_{j = 0}^\infty j h \nu \exp{-\beta j h \nu} \implies \avg{\energy}
    = -\frac{\parder{\beta} \sum_j \exp{-\beta j h \nu}}{\sum_j \exp{-\beta j h \nu}}
\end{equation}

La somma è una serie geometrica e vale:
\begin{gather}
    \sum_{j = 0}^\infty \exp{-\beta j h \nu} = \frac{1}{1 - \exp{-\beta h \nu}} \\
    \implies
    \avg{\energy} = -\frac{\displaystyle\parder{\beta} \dfrac{1}{1 - \exp{-\beta h \nu}}}{\dfrac{1}{1 - \exp{-\beta h \nu}}}
    = \frac{h \nu}{\exp{\dfrac{h \nu}{k_B T}} - 1}
\end{gather}

Aggiorniamo $I(\nu)$:
\begin{equation}
    I(\nu) = \frac{8\pi h}{c^3} \frac{\nu^3}{\exp{\dfrac{h \nu}{k_B T}} - 1}
\end{equation}

Fittando $h$ ai dati sperimentali, si ottiene lo stesso valore per tutte le curve (cioè per ogni temperatura):
\begin{equation}
    h = 2\pi \cdot \qty{1.05e-34}{\joule\second}
\end{equation}

La distribuzione corretta arriva al prezzo di aver violato la continuità della fisica classica, aver introdotto un oscillatore che assume solo valori discreti di energia.



\section{Effetto fotoelettrico}

È l'effetto per cui, quando un'onda elettromagnetica colpisce una superficie metallica, quest'ultima emette elettroni.
È dovuto al fatto che questi sentono la forza di Lorentz dovuta al campo elettrico e possono essere strappati se il campo è sufficientemente forte.

L'onda è caratterizzata da
\begin{itemize}
    \item ampiezza (quindi, intensità $I$)
    \item frequenza $\nu$
\end{itemize}

L'effetto è caratterizzato da
\begin{itemize}
    \item energia cinetica $E_k$ degli elettroni emessi
    \item numero di elettroni emessi $N_e$
    \item tempo di uscita (o di attesa) $t$
\end{itemize}

Classicamente, ci si aspetta:
\begin{itemize}
    \item $E_k$ in uscita che cresce all'aumentare dell'intensità dell'onda.
    \item $t$ che diminuisce all'aumentare dell'intensità dell'onda.
    \item $N_e$ legato all'intensità (in modo anche complicato).
    \item che $\nu$ sia ininfluente.
\end{itemize}

Invece, Hertz trovò che:
\begin{itemize}
    \item Esiste una \important{frequenza di soglia}...
    \begin{itemize}
        \item al di sotto della quale gli elettroni non sono mai emessi, per ogni intensità
        \item al di sopra della quale sono sempre emessi, per ogni intensità.
    \end{itemize}
    \item L'energia cinetica degli elettroni
    \begin{itemize}
        \item dipende linearmente dalla frequenza
        \item non dipende dall'intensità
    \end{itemize}
        \begin{equation}
        \label{eq:fotoelettrico}
            E_k = h \nu - \phi
        \end{equation}
    \item Il tempo di uscita è sempre nullo:
        \begin{equation}
            t = 0
        \end{equation}
    \item Il numero di elettroni emessi ha il seguente andamento:
        \begin{equation}
            N_e \propto \frac{I}{h \nu}
        \end{equation}
\end{itemize}

A trovare che la costante di proporzionalità era proprio $h$ fu Einstein.

Fu chiaro che $h$ era una costante fondamentale, e venne detta \important{costante di Planck}.

Osservando l'\cref{eq:fotoelettrico}, sembra un urto elastico con una particella in cui l'elettrone guadagna energia $h \nu$.

In un certo senso, è il campo elettromagnetico (e non solo l'oscillatore armonico) a essere discreto.

L'entità che discretizza l'onda elettromagnetica si chiama \important{fotone}, immaginabile come la particella di energia $h \nu$ che urta gli elettroni.

\section{Quantità di moto di un'onda elettromagnetica}

Richiamiamo l'equazione per la densità di quantità di moto di un'onda elettromagnetica:
\begin{equation}
    p\EM  = \frac{w\EM}{c}
\end{equation}

La quantità di moto, classicamente, è associata a una massa.

Ora è possibile immaginare che il campo elettromagnetico sia costituito da fotoni, particelle di energia $h \nu$.
Quando queste urtano il materiale, ``si attaccano'' all'elettrone e gli trasferiscono energia e quantità di moto.

La massa dell'elettrone non varia nel processo, quindi il fotone non ha massa.

Detta $n_f$ la densità volumica di fotoni, la quantità di moto del singolo fotone è
\begin{equation}
\label{eq:momento_fotone}
    p = \frac{p\EM}{n_f} = \frac{w\EM}{n_f c} = \frac{h \nu}{c} = \frac{h}{\lambda} = \hbar k
\end{equation}

\subsection{Momento angolare di un'onda elettromagnetica}

Consideriamo un disco con sfere cariche sul bordo e un solenoide presso l'asse collegato a un interruttore tramite un timer.

Quando il circuito si accende, il flusso del campo magnetico genera un campo elettrico che agisce sulle cariche, mettendo il disco in rotazione con una certa velocità angolare.
Quindi, poiché il disco ha momento d'inerzia non nullo, il sistema acquisisce un momento angolare anche se ce l'aveva originariamente apparentemente nullo.

Quindi, il campo elettromagnetico deve avere anche un momento angolare.

Da queste considerazioni seguirà che:
\begin{itemize}
    \item Massa ed energia sono equivalenti.
    \item Un fotone non può che muoversi a velocità $c$.
\end{itemize}


\section{Modello atomico}

\subsection{Esperimento di Rutherford}

I chimici erano convinti che la materia fosse discreta, mentre le teorie fisiche la descrivevano come continua.

Vigeva il \important{modello di Thomson}: gran parte della massa della materia deve essere costituita di una componente positiva, in cui erano immersi gli elettroni di massa molto trascurabile.

Rutherford svolse un esperimento con una lamina d'oro sottilissima ($\sim \unit{\micro\metre}$).
Bombardò la lamina con particelle $\alpha$, positive: sarebbero dovute passare attraverso senza grandi deviazioni, invece una piccola frazione (dell'ordine di una su $10^4$) veniva riflessa.
Rutherford concluse che la carica positiva era raggruppata in nuclei di grande massa.

Il rapporto tra le dimensioni di queste particelle positive rispetto a quelle degli elettroni doveva essere paragonabile al rapporto tra le particelle $\alpha$ riflesse.

La distanza tra due nuclei risultava dell'ordine di \qty{e-10}{\metre}, le dimensioni dei nuclei \qty{e-15}{\metre}.

La materia è essenzialmente vuota, il fatto che sembri ``piena'' deriva dai campi elettromagnetici.

Viene sviluppato un \important{modello planetario} per gli atomi, in moto circolare uniforme dovuto alla legge di Coulomb.

Tuttavia, nel suo modo accelerato, l'elettrone dovrebbe generare campo elettromagnetico e perdere energia, infine collassando sul nucleo.


\subsection{Spettri di emissione e assorbimento}

In un gas scaldato gli urti tra le molecole che hanno energia differente trasferiscono energia da un atomo all'altro.
Questa energia viene emessa o assorbita dagli elettroni degli atomi, che emettono o assorbono onde elettromagnetiche.

Rydberg identificò una legge empirica per le righe spettrali dell'idrogeno:
\begin{equation}
\label{eq:rydberg}
    \frac{1}{\lambda} = R \pts{\frac{1}{m^2} - \frac{1}{n^2}}, \quad n > m, \quad n, m \in \Z^+
\end{equation}
$R$ è detta \important{costante di Rydberg}:
\begin{equation}
    R = \qty{1.09e7}{\per\metre}
\end{equation}
Quindi, gli spettri di emissione sono righe discrete, mentre quelli di assorbimento sono i loro complementari.

\subsection{Modello atomico di Bohr}

Il modello di Bohr vuole spiegare:
\begin{itemize}
    \item Le dimensioni dell'atomo
    \item La stabilità dell'atomo
    \item L'energia di ionizzazione e l'effetto fotoelettrico
    \item Il carattere discreto degli spettri di emissione e di assorbimento
\end{itemize}

Consideriamo l'atomo di idrogeno, costituito da un protone di carica $e > 0$ e massa $m_p$ e un'elettrone di carica $-e$ e massa $m \ll m_p$ che orbita attorno al protone in moto circolare uniforme a distanza $r$ e con velocità $v$, grazie a una forza centripeta elettrostatica (la forza di Coulomb):
\begin{equation}
    F = m \frac{v^2}{r} = \frac{e^2}{4 \pi \eps_0 r^2}
\end{equation}
Momento angolare:
\begin{equation}
    L = r m v
\end{equation}
Energia meccanica totale ($H$ per Hamilton):
\begin{equation}
    H = \frac{1}{2} m v^2 + (-e) \frac{e}{4 \pi \eps_0 r} = - \frac{e^2}{8 \pi \eps_0 r}
\end{equation}


Gli spettri di emissione suggerivano una discretizzazione.
Bohr impose la discretizzazione del momento angolare dell'elettrone:
\begin{equation}
    L = n \hbar, \quad n \in \Z
\end{equation}
Si introduce qui la \important{costante di Planck ridotta}:
\begin{equation}
    \hbar \coloneq \frac{h}{2\pi}
\end{equation}

Si ricava la velocità
\begin{gather}
    r m v = n \hbar
    \implies
    m v^2 = \frac{e^2}{4 \pi \eps_0} \frac{m v}{n \hbar}
    \implies v = \frac{e^2}{4 \pi \eps_0 \hbar} \frac{1}{n}
\end{gather}

Si ricava il raggio dell'orbita:
\begin{gather}
    r = \frac{n \hbar}{m v} = \frac{n \hbar}{m} \frac{4 \pi \eps_0 \hbar}{e^2} n = \frac{4\pi \eps_0 \hbar^2}{m e^2} n^2 = r_1 n^2
\end{gather}

Si ricava l'energia meccanica:
\begin{equation}
\label{eq:energia_bohr}
    H = - \frac{e^2}{8 \pi \eps_0} \frac{m e^2}{4\pi \eps_0 \hbar^2} \frac{1}{n^2} = - \frac{m e^4}{2 (4\pi \eps_0)^2 \hbar^2} \frac{1}{n^2} = -\frac{H_1}{n^2}
\end{equation}

Quindi dobbiamo avere $n \ne 0$.
Inoltre, il segno influisce solo sul verso della velocità.
Consideriamo quindi $n$ intero positivo:
\begin{equation}
    n \in \Z^+
\end{equation}

Per $n \to +\infty$, $H \to 0^-$.
Per $n = 1$, l'energia è minima e l'elettrone ruota con la velocità massima e lungo l'orbita minore, il cui raggio (\important{raggio di Bohr}) è $r_1 = \qty{5.3e-11}{\metre}$.

Non è possibile che l'elettrone abbia energia minore di $-H_1$: l'ipotesi \textit{ad hoc} di Bohr implica la discretizzazione anche dell'energia e il fatto che gli elettroni non possono perdere energia indefinitamente.
$H_1$ vale
\begin{equation}
    H_1 = \qty{13.6}{\electronvolt}
\end{equation}
e corrisponde all'energia necessaria per portare l'elettrone dallo stato fondamentale a un punto lontano dall'atomo ($H = 0$).
È l'energia di ionizzazione per l'idrogeno allo stato fondamentale.

L'unità di misura \important{elettronvolt} (\unit{\electronvolt}) è l'energia di un elettrone che attraversa la differenza di potenziale di \qty{1}{\volt}.
\begin{equation}
    \qty{1}{\electronvolt}
    = e \cdot \qty{1}{\volt}
    = (\qty{1.6e-19}{\coulomb}) (\qty{1}{\volt})
\end{equation}

La formula di Rydberg, nel caso dell'emissione, corrisponde alla variazione di energia dal livello $n$ al livello $m$:
\begin{equation}
    H_n - H_m = H_1 \pts{\frac{1}{m^2} - \frac{1}{n^2}}, \quad 1 \le m < n
\end{equation}
Questa differenza di energia diventa radiazione elettromagnetica, quindi viene emessa come fotone:
\begin{equation}
    h \nu = \frac{h c}{\lambda} = H_1 \pts{\frac{1}{m^2} - \frac{1}{n^2}}
\end{equation}
La costante di Rydberg, calcolata empiricamente, risulta correttamente prevista dalla teoria.
Confrontando con l'\cref{eq:rydberg}, infatti, risulta
\begin{equation}
    R = \frac{H_1}{h c}
\end{equation}


\section{Calore specifico}

La capacità termica $C$ di un corpo alla temperatura $T$ è il rapporto tra il calore $\delta Q$ fornito al corpo e la variazione di temperatura $\de T$ che ne risulta:
\begin{equation}
    C = \frac{\delta Q}{\de T} = \frac{\de U + \delta L}{\de T}
\end{equation}
Il calore specifico è la capacità termica per unità di massa, o quantità di sostanza $n$, o numero di particelle $N$.

Si definisce il calore specifico isocoro molare per le trasformazioni isocore (per un solido o liquido, quasi sempre):
\begin{equation}
    c_V = \frac{1}{n} \der[U]{T}
\end{equation}

\subsection{Molecola biatomica}

Nel caso della molecola di idrogeno \ce{H2} allineata lungo l'asse $x$, l'energia totale è
\begin{equation}
    U = \frac{1}{2} M v^2
    + \frac{1}{2} I_y \omega_y^2
    + \frac{1}{2} I_z \omega_z^2
    + \frac{1}{2} M v_r^2
    + \frac{1}{2} k_e x^2
\end{equation}
$M$ è la massa di entrambi i nuclei, mentre il momento d'inerzia $I_x$ è trascurabile. Il sistema, quindi, ha sette gradi di libertà.

Quindi,
\begin{equation}
    U = n N_A \frac{7}{2} k_B T \implies c_V = \frac{7}{2} R
\end{equation}
$R = N_A k_B$ qui è la costante universale dei gas perfetti.

La fisica classica prevede questo valore sempre.
Tuttavia, i due gradi di libertà corrispondenti alle oscillazioni si misurano solo per temperatura molto alte (purché non sufficientemente alte da rompere le molecole).
A temperatura ambiente $c_V = 5/2 \, R$. A temperature ancora più basse si perdono anche i gradi di libertà della rotazione: $c_V = 3/2 \, R$.

\redtext{Grafico $c_V(T)$}

Si svolge un'ipotesi \textit{ad hoc}, sostituendo al momento angolare e all'energia dell'oscillatore armonico le espressioni con le ipotesi di Planck e Bohr:
\begin{equation}
    U = \frac{1}{2} M v^2 + \frac{(n \hbar)^2}{2 I} + j \hbar \omega
\end{equation}

Consideriamo un termostato a temperatura $T = T_0 + \de T$, mentre la molecola di idrogeno ha temperatura $T_0$.

Dopo l'interazione con il termostato, la molecola avrà temperatura $T$.
Al massimo, il termostato potrà trasferire energia $3/2 \, k_B T$ (i gas monoatomici non mostrano problemi, quindi usiamo un termostato classico a gas monoatomico).
\begin{equation}
    \Delta U \le \frac{3}{2} k_B T
\end{equation}
Quest'energia deve distribuirsi equamente su tutti i gradi di libertà della molecola.

Ma questo è contraddetto dagli esperimenti, come detto sopra.

Il modo di far quadrare i conti prevede che
\begin{itemize}
    \item l'energia di traslazione vari nel continuo
    \item l'energia di rotazione assuma valori discreti $(n \hbar)^2 / (2 I)$
    \item l'energia di vibrazione assuma valori discreti $j \hbar \omega$
\end{itemize}
Ma allora, la ragione per il fenomeno sopra è che l'energia, per poter andare distribuita sulla rotazione o vibrazione, deve superare una soglia minima dovuta ai ``salti''.
È anche possibile calcolare le temperature di transizione.

\subsection{Solido cristallino}

Consideriamo un cristallo atomico con struttura cubica.
Ogni atomo è circondato da sei atomi.

Se un atomo riceve energia termica, inizia a oscillare lungo una direzione, respinto dalle coppie di atomi nella direzione lungo cui oscilla.
L'energia media di oscillazione vale il triplo di quella di un singolo oscillatore, poiché corrisponde a tre oscillatori arminici (uno per ogni direzione):
\begin{equation}
    \avg{\energy} = \frac{3 h \nu}{\exp{\dfrac{h \nu}{k_B T}} - 1}
\end{equation}
Ma quindi, il calore specifico isocoro molare vale
\begin{equation}
    c_V = \der{T} \frac{3 N_A h \nu}{\exp{\dfrac{h \nu}{k_B T}} - 1}
\end{equation}

\redtext{Grafico $c_V(T)$}

In particolare,
\begin{gather}
    c_V
    % \propto \frac{1}{T^2} \exp{- \frac{h \nu}{k_B T}}
    \to 0, \quad \text{per } T \to 0^+ \\
    c_V \to 3 R, \quad \text{per } T \to +\infty
\end{gather}
Il modello classico darebbe invece $c_V = 3 R$ per ogni temperatura.


\section{Diffrazione nei cristalli}

\subsection{Diffrazione di raggi X}

Consideriamo un cristallo con struttura cubica e passo reticolare $d \sim \qty{e0}{\angstrom}$.
Mandando un'onda elettromagnetica a una angolo $\theta$ dal piano, si osserva diffrazione su uno schermo lontano perpendicolare all'onda riflessa.

Similmente al caso di diffrazione da doppia fenditura (vedi \autoref{sec:doubleslit}),
\begin{equation}
    I = 4 I_0 \cos^2 \frac{k (r_2 - r_1)}{2}
\end{equation}
La differenza dei cammini stavolta è
\begin{gather}
    r_2 - r_1 = 2 d \sin \theta \\
    \implies I = 4 I_0 \cos^2 \pts{\pi \frac{2 d \sin \theta}{\lambda}}
\end{gather}

La condizione per i massimi di interferenza è nota come \important{legge di Bragg}:
\begin{equation}
    2 d \sin \theta_M= \lambda n, \quad n \in \Z
\end{equation}

\subsection{Diffrazione di elettroni}

Si tentò di replicare la diffrazione nei cristalli con gli elettroni.
Era difficile poiché occorreva mandare un elettrone alla volta e creare il vuoto per distanze relativamente lunghe.

Gli elettroni vengono accelerati con un condensatore a facce piane parallele (vedi \autoref{sec:acceleratore_carica}) fino ad avere una certa energia cinetica $E_k$.

Si osserva una figura di interferenza da cui è possibile calcolare la lunghezza d'onda tramite la legge di Bragg, come se si trattasse di un'onda elettromagnetica.
Graficando la quantità di moto degli elettroni $p = \sqrt{2 m E_k}$ rispetto a $1 / \lambda$ si ottiene una retta di pendenza $h$ e quindi la seguente ralazione:
\begin{equation}
    p = \frac{h}{\lambda}
\end{equation}

Questo risultato è coerente con l'\cref{eq:momento_fotone} per i fotoni e indica che anche l'elettrone esibisce dualismo onda-particella: a seconda del constesto, può essere descritto come un'onda o come una particella.

Le stesse conclusioni si potrebbero svolgere in generale, affermando che ogni particella (caratterizzata da $E$ e $p$) può anche essere descritta come un'onda (caratterizzata da $\nu$ e $\lambda$) e viceversa e che valgano le \important{relazioni di de Broglie}
\begin{subequations}
\begin{gather}
    E = h \nu \\
    p = \frac{h}{\lambda}
\end{gather}
\end{subequations}

Ma allora, è opportuno introdurre una \important{funzione d'onda} $\Psi(\p, t)$ che descriva la dinamica dell'elettrone in quanto onda, così come la legge oraria $\p(t)$ ne descrive la dinamica in quanto particella.


\section{Conclusione}

Caratteristiche comuni alle osservazioni discusse:
\begin{itemize}
    \item Gli effetti avvengono per dimensioni su scala atomica o subatomica.
    \item Nuova costante universale $h$, dal valore numerico piccolissimo.
    \item Grandezze fisiche quantizzate (cioè discretizzate):
    \begin{itemize}
        \item energia di un oscillatore armonico
        \item momento angolare
        \item energia del campo elettromagnetico
        \item quantità di moto del campo elettromagnetico
    \end{itemize}
    \item Dualismo onda-particella:
    \begin{itemize}
        \item onde descritte come particelle (effetto fotoelettrico e quantità di moto dei fotoni)
        \item particelle descritte come onde (diffrazione di elettroni)
    \end{itemize}
\end{itemize}

Discretizzando il momento angolare e l'energia dell'oscillatore armonico si spiegano tutti questi fenomeni.

Rimane da spiegare da cosa derivano queste discretizzazioni e che significato fisico abbia la funzione d'onda (cosa si propaga?).

La discretizzazione significa, ad esempio, che non ha neanche senso \textit{immaginare} un elettrone che disti dal nucleo una certa distanza $r$ qualsiasi.
Nel caso dell'oscillatore, significa che una massa che oscilla non può assumere determinate ampiezze e determinate velocità massime.

\chapter{Meccanica quantistica}

\section{Equazione di Schrödinger}

Occore ora indicare un'equazione differenziale che definisca la funzione d'onda $\Psi(\p, t)$, analogamente a come la legge di Newton definisce la legge oraria $\p(t)$.

% Descriviamo la diffrazione di elettroni in termini di $\Psi$:
% \begin{subequations}
% \begin{gather}
%     \Psi_1(r_1, t) = A \exp{\im (k r_1 - \omega t)} \\
%     \Psi_2(r_2, t) = A \exp{\im (k r_2 - \omega t)}
% \end{gather}
% \end{subequations}
% \begin{gather}
%     \phi = k r_1 - \omega t - k r_2 + \omega t = k (r_1 - r_2) \\
%     I \propto \norm{\Psi_1 + \Psi_2}^2 = 4 A^2 \cos^2 \frac{k (r_1 - r_2)}{2}
% \end{gather}

Bisogna partire dal presupposto che l'elettrone sia un'onda, quindi deve valere l'equazione di d'Alembert:
\begin{equation}
    \lapl \Psi = \frac{1}{v_f^2} \parder[\Psi][2]{t}
\end{equation}
$v_f$ è la velocità di fase dell'onda, non la velocità della particella $v_p$.

Presupponiamo per $\Psi$ la seguente forma:
\begin{equation}
    \Psi(\p, t) = A \exp{\im (k r - \omega t)}
\end{equation}
Bisogna ora usare il principio di de Broglie.

Si calcola la derivata seconda rispetto al tempo di $\Psi$:
\begin{equation}
    \parder[\Psi][2]{t} = - \omega^2 \Psi
    \implies \lapl \Psi = - \frac{\omega^2}{v_f^2} \Psi
\end{equation}

Per il principio di de Broglie,
\begin{gather}
    p = k \hbar \implies
    \frac{\omega^2}{v_f^2} = k^2 = \frac{p^2}{\hbar^2} \\
    \lapl \Psi = - \frac{p^2}{\hbar^2} \Psi
\end{gather}

Sia $E = T + V$ l'energia meccanica totale della particella, con $T$ energia cinetica e $V$ energia potenziale.

Se $v_p \ll c$ allora non è necessario usare le relazioni relativistiche per energia cinetica e quantità di moto.
Se la particella è puntiforme, vale
\begin{equation}
    p = m v_p, \quad
    T = \frac{1}{2} m v_p^2 = \frac{p^2}{2 m}
\end{equation}

Quindi,
\begin{gather}
    \lapl \Psi = - \frac{2 m T}{\hbar^2} \Psi = - \frac{2 m}{\hbar^2} (E - V) \Psi \\
    \implies - \frac{\hbar}{2 m} \lapl \Psi + V \Psi = E \Psi
\end{gather}

Inoltre, usando nuovamente il principio di de Broglie nella forma $E = \hbar \omega$,
\begin{equation}
    \parder[\Psi]{t}
    = - \im \omega \Psi
    = - \im \frac{E}{\hbar} \Psi
    \implies
    E \Psi = \im \hbar \parder[\Psi]{t}
\end{equation}

Sostituendo, si ottiene l'\important{equazione di Schrödinger}:
\begin{equation}
    \boxed{
    -\frac{\hbar^2}{2 m} \lapl \Psi(\p, t) + V(\p, t) \Psi(\p, t) = \im \hbar \parder{t} \Psi(\p, t)
    }
\end{equation}
valida per
\begin{itemize}
    \item masse puntiformi
    \item velocità non relativistiche
\end{itemize}

Le ``forze'' (o meglio, l'ambiente) sono codificate nel termine dell'energia potenziale $V$, quindi è necessario che siano tutte conservative (ma tutte le interazioni fondamentali sono conservative, quindi su scala microscopica questa richiesta è sempre rispettata).

L'equazione di Schrödinger sostituisce l'equazione di Newton:
\begin{itemize}
    \item Secondo Newton, un sistema fisico evolve in quanto i punti materiali che lo costituiscono si muovono lungo curve $\p(t)$ nello spazio euclideo $\R^3$.
    \item Secondo l'equazione di Schrödinger, il sistema fisico evolve secondo la funzione d'onda $\Psi(\p, t)$.
\end{itemize}

Heisenberg aveva contemporaneamente prodotto una descrizione analoga ma differente, un diverso formalismo basato sull'algebra lineare che forniva le stesse previsioni.

\section{Interpretazione di Copenhagen}

Sorge la necessità di interpretare $\Psi$: a cosa corrisponde?

La prima cosa che viene in mente è che $\Psi$ rappresenti comunque il ``corpo'' della particella, come se la sua massa fosse dispersa in un volume maggiore, ma questo porterebbe a paradossi.

Born fornì una migliore interpretazione confrontando la diffrazione della luce e quella degli elettroni.

Nello scenario della doppia fenditura (\autoref{sec:doubleslit}), il campo elettrico in $P$ è
\begin{equation}
    \E(P) = \E_1 + \E_2
\end{equation}

Sullo schermo si vede l'intensità dell'onda:
\begin{equation}
    I(P) \propto \norm{\E}^2 = \norm{\E_1 + \E_2}^2
\end{equation}

La sorgente emette potenza $W$.
Quindi, $I(P)/W$ è la frazione di potenza per unità di superficie che arriva in $P$:
\begin{equation}
\label{eq:probphot}
    \frac{I(P)}{W}
    = \frac{N_f(P) h \nu}{N_0 h \nu}
    = \boxed{\frac{N_f(P)}{N_0}
    \propto \norm{\E_1 + \E_2}^2}
\end{equation}
dove $N_f(P)$ è il numero di fotoni per unità di superficie in $P$ e $N_0$ è il numero di fotoni emessi dalla sorgente.

$N_f(P)/N_0$ può anche essere interpretata come densità di probabilità (rispetto alla superficie sullo schermo) che un fotone emesso dalla sorgente arrivi in $P$.

Questa è un'interpretazione ``termodinamica'': la si può svolgere poiché i fotoni emessi dalla sorgente sono moltissimi, è una considerazione statistica.

Se al posto dei fotoni consideriamo gli elettroni, l'\cref{eq:probphot} diventa:
\begin{equation}
    \frac{N_e(P)}{N_0} = \abs{\Psi_1 + \Psi_2}^2
\end{equation}

Quindi, il modulo quadro di $\Psi$ rappresenta (almeno) la densità di probabilità in senso frequentistico (cioè, la frazione delle osservazioni sui casi totali) di trovare elettroni in $P$.

Ovvero,
\begin{equation}
    \mathrm{P}(\text{particella} \in \de\p) = \abs{\Psi(\p, t)}^2 \de\p
\end{equation}
$\de\p$ rappresenta un elemento di volume.

Questa interpretazione, ad oggi, è l'unica che è coerente con gli esperimenti.

Ogni altro tentativo di dare un significato fisico a $\Psi$ risulta contraddittorio logicamente o sperimentalmente.

In particolare, non è vero che la particella ha, ad esempio, una traiettoria che non si può conoscere per motivi pratici (come nel caso della termodinamica).
Il concetto di traiettoria risulta inadeguato in un senso fondamentale.
Non è in alcun modo possibile studiare con certezza la dinamica della particella: la statistica, in questo caso, non entra in gioco per circoscrivere i limiti di chi sta studiando il sistema, ma quelli del sistema stesso.

L'interpretazione di Born, poiché adottata anche da Heisenberg e discussa a Copenhagen, viene detta \important{interpretazione di Copenhagen}.


\section{Vettore di stato}

Nella meccanica classica, tutte le grandezze fisiche sono funzione della posizione e della quantità di moto, cioè le si ricava dalla legge oraria.

Come è possibile, allora, estrarre da $\Psi$ le grandezze fisiche?
A questa domanda risposero Heisenberg e Dirac.

Matematicamente, $\Psi$ è non nulla nella regione in cui è possibile trovare l'elettrone e deve essere derivabile rispetto a $\p$ e a $t$.

Se è possibile che l'elettrone sia solo nella regione $\Omega$, allora
\begin{equation}
    \p \notin \Omega \lor \p \in \partial \Omega \implies \Psi(\p, t) = 0
\end{equation}

Poiché la probabilità deve essere complessivamente $1$,
\begin{equation}
    \int_\Omega \abs{\Psi}^2 \de\p = 1
\end{equation}

In particolare, $\Psi$ deve essere una funzione \important{a quadrato sommabile} in $\Omega$.
Si scrive anche $\Psi \in L^2(\Omega)$.

Mentre il precedente integrale era reale, il seguente è in generale un numero complesso:
\begin{equation}
    \inner{\Psi_1}{\Psi_2} \coloneqq \int_\Omega \conj{\Psi_1} \Psi_2 \de\p \in \C
\end{equation}
Questa espressione costituisce un prodotto scalare.
Le funzioni d'onda, quindi, sono elementi dello spazio di Hilbert $L^2(\Omega)$.

Il formalismo di Heisenberg si basava sulla versione in vettori colonna $\C^{n,1}$ di questo spazio vettoriale.
Le formulazioni sono equivalenti poiché l'algebra lineare garantisce che due spazi vettoriali con la stessa dimensione siano isomorfi.

Dirac indica gli elementi di questo spazio vettoriale come \textit{ket} $\ket{\Psi}$, in modo che giustapposti ai \textit{bra} $\bra{\Psi}$ diano i prodotti scalari $\braket{\Psi_1}{\Psi_2}$.

Comunque sia interpretato (come funzione di uno spazio di Hilbert, vettore colonna complesso o ket), $\Psi$ è elemento di uno spazio vettoriale dotato di prodotto scalare e identifica lo stato del sistema: è un \important{vettore di stato}.

\section{Informazioni fisiche}

Le particelle sono descritte da uno vettore di stato che ``vive'' in uno spazio vettoriale e l'evoluzione di un sistema fisico corrisponde a una traiettoria in questo spazio vettoriale.
È la proiezione di questi stati a dare le informazioni fisiche nello spazio fisico $\R^3$.

Questa teoria fisica non è mai stata falsificata sperimentalmente (da fine anni '30 del Novecento).

Ogni misura della posizione della particella fornisce una posizione $\p_i$.
Prese $N$ misure, posso definire la posizione media
\begin{equation}
    \avg{\p} = \frac{1}{N} \sum_{j = 1}^N \p_j
\end{equation}

Il numero di volte in cui si trova l'elettrone in $\p_i$ è $N \abs{\Psi(\p_i)}^2$, per cui
\begin{equation}
    \avg{\p} = \frac{1}{N} \sum_{j = 1}^N \p_j N \abs{\Psi(\p_i)}^2 = \sum_{i = 1}^N \p_i \abs{\Psi(\p_i)}^2
\end{equation}
Nel continuo:
\begin{equation}
    \avg{\p} = \int_\Omega \p \abs{\Psi(\p)}^2 \de\p
    = \int_\Omega \conj{\Psi(\p)} \pts{\p \Psi(\p)} \de\p
    = \inner{\Psi}{\p \Psi}
\end{equation}
Questa è un'espressione per estrarre da $\Psi$ il valor medio della posizione.

Per quanto riguarda la quantità di moto:
\begin{equation}
    \avg{\vt{p}} = m \der{t} \inner{\Psi}{\p \Psi}
\end{equation}
In una dimensione, $\Omega = [a, b] \subset \R$ e:
\begin{equation}
    \avg{p_x} = m \der{t} \inner{\Psi}{x \Psi}
    = m \int_a^b x \parder{t}\pts{\conj{\Psi} \Psi} \de x
\end{equation}

Ora, consideriamo l'equazione di Schrödinger, moltiplichiamo per $\conj{\Psi}$ e prendiamo il complesso coniugato:
\begin{gather}
    \im \hbar \parder{t} \Psi = -\frac{\hbar^2}{2 m} \lapl \Psi + V \Psi \\
\label{eq:conj1}
    \implies \im \hbar \conj{\Psi} \parder{t} \Psi = -\frac{\hbar^2}{2 m} \conj{\Psi} \lapl \Psi + \conj{\Psi} V \Psi \\
\label{eq:conj2}
    \implies -\im \hbar \Psi \parder{t} \conj{\Psi} = -\frac{\hbar^2}{2 m} \Psi \lapl \conj{\Psi} + \Psi V \conj{\Psi}
\end{gather}
Sottraendo membro a membro l'\cref{eq:conj2} dall'\cref{eq:conj1},
\begin{gather}
    \im \hbar \pts{\conj{\Psi} \parder{t} \Psi + \Psi \parder{t} \conj{\Psi}}
    = -\frac{\hbar^2}{2 m} \pts{\conj{\Psi} \lapl \Psi - \Psi \lapl \conj{\Psi}} \\
    \implies \im \hbar \parder{t} \pts{\conj{\Psi} \Psi}
    = -\frac{\hbar^2}{2 m} \pts{\conj{\Psi} \lapl \Psi - \Psi \lapl \conj{\Psi}} \\
    \implies  \parder{t} \pts{\conj{\Psi} \Psi} = \frac{\im \hbar}{2 m} \pts{\conj{\Psi} \lapl \Psi - \Psi \lapl \conj{\Psi}}
\end{gather}

Per cui,
\begin{gather}
    \avg{p_x}
    = m \int_a^b x \parder{t}\pts{\conj{\Psi} \Psi} \de x = \\
    = \frac{\im \hbar}{2} \int_a^b x \pts{\conj{\Psi} \parder[][2]{x} \Psi - \Psi \parder[][2]{x} \conj{\Psi}} \de x = \\
    = \frac{\im \hbar}{2} \int_a^b x \, \parder{x} \pts{\conj{\Psi} \parder{x} \Psi - \Psi \parder{x} \conj{\Psi}} \de x = \\
    = \frac{\im \hbar}{2} \pts{
        \left[
            x \pts{
                \conj{\Psi} \parder{x} \Psi
                - \Psi \parder{x} \conj{\Psi}
            }
        \right]_{x = a}^b
        - \int_a^b \pts{
            \conj{\Psi} \parder{x} \Psi
            - \Psi \parder{x} \conj{\Psi}
        } \de x
    } = \\
    = -\frac{\im \hbar}{2} \int_a^b \pts{
        \conj{\Psi} \parder{x} \Psi
        - \Psi \parder{x} \conj{\Psi}
    } \de x = \\
    = -\frac{\im \hbar}{2} \pts{
        \int_a^b \conj{\Psi} \parder{x} \Psi \,\de x
        - \left[\Psi \conj{\Psi}\right]_{x = a}^b
        + \int_a^b \conj{\Psi} \parder{x} \Psi \,\de x
    } = \\
    = - \im \hbar \int_a^b \conj{\Psi} \parder{x} \Psi \,\de x
    = \inner{\Psi}{-\im \hbar \parder{x} \Psi}
\end{gather}
I termini da valutare nelle integrazioni per parti sono nulli poiché $\Psi$ è nulla agli estremi dell'intervallo.

In tre dimensioni,
\begin{equation}
    \avg{\vt{p}} = \inner{\Psi}{-\im \hbar \grad \Psi}
\end{equation}

In generale, i valori medi delle grandezze fisiche si estraggono applicando degli operatori lineari a $\Psi$ e poi prendendo il prodotto scalare con $\conj{\Psi}$.

Risulta che è sempre questo il caso, anche per altre grandezze fisiche.
Infatti, ogni grandezza fisica è funzione di $\p$ e $\vt{p}$.
Quindi basta sostituire gli operatori per $\p$ e $\vt{p}$, che abbiamo visto essere $\p$ stessa e $-\im \hbar \grad$.
\begin{align}
    \p & \to \hat{\p} = \p \\
    \vt{p} & \to \hat{\vt{p}} = -\im \hbar \grad \\
    F(\p, \vt{p}) & \to \hat{F}(\p, -\im \hbar \grad)
\end{align}

Inoltre, i valori medi delle grandezze fisiche devono essere reali.
Cioè, per una grandezza fisica $F$ associata a un operatore $\hat{F}$,
\begin{gather}
    \conj{\avg{F}} = \avg{F}
    \implies
    \inner{\Psi}{\hat{F} \Psi} = \conj{\inner{\Psi}{\hat{F} \Psi}} = \inner{\hat{F} \Psi}{\Psi}
\end{gather}
Poiché, detto $\herm{\hat{F}}$ l'aggiunto di $\hat{F}$, vale in generale
\begin{equation}
    \inner{\Psi}{\hat{F} \Psi} = \inner{\herm{\hat{F}} \Psi}{\Psi}
\end{equation}
allora per gli operatori che corrispondono a grandezze fisiche deve valere $\hat{F} = \herm{\hat{F}}$, ovvero devono essere \textbf{hermitiani} (si dice anche ``\textbf{autoaggiunti}'').

Operatori hermitiani hanno autovalori reali.

In conclusione,
\begin{itemize}
    \item Le grandezze fisiche classiche diventano operatori lineari hermitiani.
    \item Il valore medio delle misure di una grandezza fisica $F$ è
        \begin{equation}
            \avg{F} = \inner{\Psi}{\hat{F} \Psi}
            = \int_\Omega \conj{\Psi} \hat{F} \Psi \de\p
            \in \R
        \end{equation}
\end{itemize}


\section{Equazione di Schrödinger stazionaria}

Otteniamo l'operatore associato all'energia totale (hamiltoniana) di un sistema:
\begin{gather}
    H = \frac{p^2}{2 m} + V(\p, t) \\
    \hat{H} = \frac{(-\im \hbar \grad)^2}{2 m} + V(\p, t)
    = -\frac{\hbar^2}{2 m} \lapl + V(\p, t)
\end{gather}
Sostituendo nell'equazione di Schrödinger,
\begin{equation}
\label{eq:schr_hamiltoniana}
    \hat{H} \Psi(\p, t) = \im \hbar \parder{t} \Psi(\p, t)
\end{equation}
L'energia di un sistema, quindi, è legata all'evoluzione del sistema nel tempo.

$\hat{H}$ dipende dal tempo, ma solo in quanto $V$ dipende dal tempo.
Un esempio di energia potenziale che dipende dal tempo è quella dovuta alla presenza di un campo elettromagnetico variabile nel tempo.
Occupiamoci ora di potenziali statici, che esprimono il fatto che l'ambiente è statico.
Possiamo supporre che l'\cref{eq:schr_hamiltoniana} abbia soluzione a variabili separabili:
\begin{gather}
    \Psi(\p, t) = T(t) R(\p) \\
    T \hat{H} R = \im \hbar R \der[T]{t} \\
\label{eq:variabili_separate}
    \frac{\im \hbar}{T} \der[T]{t} = \frac{\hat{H} R}{R}
\end{gather}
Poiché nell'\cref{eq:variabili_separate} i due membri sono funzioni di variabili diverse ma sempre costanti, allora ciascuno dei due è uguale alla stessa costante $A$:
\begin{equation}
    \begin{cases}
        \frac{1}{T} \der[T]{t} = -\im \frac{A}{\hbar} \\
        \hat{H} R = A R
    \end{cases}
\end{equation}
Si risolve la prima equazione differenziale:
\begin{equation}
    T(t) = B \exp{-\im \frac{A}{\hbar} t}
\end{equation}
Al variare di $V(\p)$, quindi, la componente temporale in $\Psi$ è sempre la stessa.

La seconda equazione, invece, è un'equazione agli autovalori per l'operatore $\hat{H}$.
Le soluzioni $R$ sono gli sutostati di $\hat{H}$.

Calcoliamo il valor medio dell'energia:
\begin{equation}
\begin{gathered}
    \avg{E} = \int_\Omega \conj{\Psi} \hat{H} \Psi \de\p
    = \int_\Omega \conj{\Psi} T(t) \hat{H} R(\p) \de\p = \\
    = \int_\Omega \conj{\Psi} T(t) A R(\p) \de\p
    = A \int_\Omega \conj{\Psi} \Psi \de\p
    = A
\end{gathered}
\end{equation}
$A$, quindi, è l'energia media totale del sistema;
rinominiamola $E$.

Nel caso in cui l'energia potenziale $V$ sia statica, quindi, si risolve l'\important{equazione di Schrödinger stazionaria}:
\begin{equation}
    \hat{H} \Psi(\p) = E \Psi(\p)
\end{equation}
che ha per autovalori le possibili energie del sistema e in modo che le funzioni d'onda che risolvono l'equazione di Schrödinger completa siano:
\begin{equation}
    \Psi(\p, t) = \exp{-\im \frac{E}{\hbar} t} \Psi(\p)
\end{equation}


\section{Misura fisica}

Immaginiamo di voler misurare una grandezza $F$.
Il valore medio previsto dalla teoria è
\begin{equation}
    \avg{F} = \inner{\Psi}{\hat{F}\Psi}
\end{equation}

Il valore medio ottenuto sperimentalmente dopo aver svolto $N$ misure è
\begin{equation}
    \avg{F}_s = \frac{1}{N} \sum_{j = 1}^N n_j f_j
\end{equation}
Dove $\{f_j\}$ sono gli esiti delle misure e $\{n_j\}$ sono le frequenze con cui si ottengono gli esiti.
Analogamente, $n_j / N$ è la probabilità (frequentistica) di ottenere l'esito $f_j$.

Tra i due risultati ci sarà un errore $\Delta F$:
\begin{equation}
    \avg{F} = \avg{F}_s \pm \Delta F
\end{equation}
Trattiamo $\Delta F$ come deviazione standard:
\begin{equation}
    \pts{\Delta F}^2 = \frac{1}{N} \sum_{j = 1}^N \pts{f_j - \avg{F}_s}^2
\end{equation}

Oltre all'indeterminazione sperimentale (casuale, strumentale e sistematica) c'è un'indeterminazione intrinseca alla meccanica quantistica, che non è epistemica ma un elemento del sistema fisico.
\begin{equation}
    \pts{\Delta F}^2 = \inner{\Psi}{\pts{\hat{F} - \avg{F}}^2 \Psi}
\end{equation}

L'operatore sopra deve essere hermitiano:
\begin{equation}
    \inner{\Psi}{\pts{\hat{F} - \avg{F}}^2 \Psi}
    = \inner{\pts{\hat{F} - \avg{F}} \Psi}{\pts{\hat{F} - \avg{F}} \Psi}
    = \abs{\pts{\hat{F} - \avg{F}} \Psi}^2
\end{equation}
Quindi, l'indeterminazione quantistica è nulla se e solo se è nullo l'argomento del modulo:
\begin{equation}
\label{eq:indeterminazione_nulla}
\begin{gathered}
    \Delta F = 0
    \iff \hat{F} \Psi = \avg{F} \Psi \iff \\
    \iff \avg{F} \text{ è un autovalore di } \hat{F}
    \iff \Psi \text{ è un autostato di } \hat{F}
\end{gathered}
\end{equation}

Poiché $\hat{F}$ è hermitiano, i suoi autostati $\{\Psi_i\}$ possono essere scelti ortonormali rispetto al prodotto scalare $\inner{\cdot}{\cdot}$ e costituiscono quindi una base per i vettori di stato possibili.
Ogni soluzione $\Psi$ si può quindi scrivere come combinazione lineare degli autostati:
\begin{gather}
    \Psi = \sum_i a_i \Psi_i \\
\begin{gathered}
    \avg{F}
    = \inner{\Psi}{\hat{F} \Psi}
    = \inner{\sum_i a_i \Psi_i}{\hat{F} \sum_i a_i \Psi_i}
    = \inner{\sum_i a_i \Psi_i}{\sum_i a_i \hat{F} \Psi_i} = \\
    = \sum_{i,j} \conj{a_j} a_i \inner{\Psi_j}{\hat{F} \Psi_i}
    = \sum_{i,j} \conj{a_j} a_i f_i \inner{\Psi_j}{\Psi_i}
    = \sum_{i,j} \conj{a_j} a_i f_i \delta_{j,i}
    = \sum_i \abs{a_i}^2 f_i
\end{gathered}
\end{gather}
$\inner{\Psi_j}{\Psi_i} = \delta_{j,i}$ poiché ortonormali.

Abbiamo, ora,
\begin{subequations}
\begin{gather}
    \avg{F} = \sum_i \abs{a_i}^2 f_i \\
    \avg{F}_s = \sum_i \frac{n_i}{N} f_i
\end{gather}
\end{subequations}
\important{Ogni misura fisica è un autovalore}, anche perché una misura determina esattamente lo stato e deve valere l'\cref{eq:indeterminazione_nulla}.

Dopo aver effettuato una misura, quindi, \important{il sistema è collassato in un autostato}.

Prima della misura, invece, lo stato del sistema non era necessariamente un autostato e si avevano solo dei pesi probabilistici $\{\abs{a_i}^2\}$ sull'esito della misura.
La grandezza fisica, quindi, non era definita.

\section{Indeterminazione}

Se una misura altera lo stato del sistema, è possibile misurare due grandezze fisiche $F$ e $G$ contemporaneamente e con precisione arbitraria?
Poiché lo stato del sistema è unico, deve valere:
\begin{equation}
    \begin{cases}
        \hat{F} \Psi = f \Psi \\
        \hat{G} \Psi = g \Psi
    \end{cases}
\end{equation}
Le grandezze devono quindi avere autostati comuni.

Inoltre,
\begin{gather}
    \begin{cases}
        \hat{F} \hat{G} \Psi_i = \hat{F} g_i \Psi_i = g_i \hat{F} \Psi_i = f_i g_i \Psi_i \\
        \hat{G} \hat{F} \Psi_i = \hat{G} f_i \Psi_i = f_i \hat{G} \Psi_i = f_i g_i \Psi_i
    \end{cases} \\
    \implies
    \hat{F} \hat{G} \Psi_i = \hat{G} \hat{F} \Psi_i
    \implies \pts{\hat{F} \hat{G} - \hat{G} \hat{F}} \Psi_i = 0
\end{gather}
Questo vale per ogni autostato, quindi l'operatore deve essere identicamente nullo.

Definiamo le \important{parentesi di commutazione} $[\cdot, \cdot]$:
\begin{equation}
    [\hat{F}, \hat{G}] \coloneq \hat{F} \hat{G} - \hat{G} \hat{F}
\end{equation}

Si conclude che deve valere:
\begin{equation}
    [\hat{F}, \hat{G}] = 0
\end{equation}
Due grandezze fisiche possono essere misurate contemporaneamente con precisione arbitraria se e solo se gli operatori associati commutano.

Ad esempio:
\begin{gather}
    [\hat{x}, \hat{p}_x] \Psi = - x \im \hbar \parder{x} \Psi - \pts{-\im \hbar \parder{x} \pts{x \Psi}} = \im \hbar \Psi
    \implies [\hat{x}, \hat{p}_x] = \im \hbar
\end{gather}
Quindi, posizione e quantità di moto lungo la stessa componente non commutano.

Vale, in particolare, il \important{principio di indeterminazione di Heisenberg}:
\begin{equation}
    \Delta x \Delta p_x \ge \frac{\hbar}{2}
\end{equation}

% \begin{equation}
%     \Delta F \Delta G \geq \frac{1}{2} \abs{\avg{[\hat{F}, \hat{G}]}}
% \end{equation}


\section{Principi della meccanica quantistica}

\begin{itemize}
    \item Principio di corrispondenza: la meccanica quantistica deve tendere alla meccanica classica per sistemi macroscopici.
    \item Assioma dell'osservabile: i sistemi fisici sono descritti da un vettore di stato (nell'equazione di Schrödinger, è una funzione a valori complessi; nella formulazione di Dirac, un \textit{ket}; Heisenberg ha usato un altro formalismo).
    \begin{itemize}
        \item Interpretazione di Copenhagen: il vettore di stato è connesso alla probabilità in senso frequentistico, cioè la frazione di volte in cui una misura dà un certo risultato.
    \end{itemize}
    \item Assioma dell'operatore: le grandezze fisiche sono associate a operatori lineari hermitiani.
    Questi, come le grandezze classiche, sono funzione degli operatori posizione e quantità di moto.

    Il valore medio di una grandezza fisica si estrae applicando l'operatore e prendendo il prodotto scalare con il vettore di stato.
    \item Principio del collasso: una misura fa collassare il vettore di stato in un autostato dell'operatore associato alla grandezza misurata.

    Ne seguono:
    \begin{itemize}
        \item Principio di indeterminazione
        \item Principio di complementarietà: esistono diversi massimi insiemi di informazioni che posso conoscere di un sistema fisico, poiché serve che il sistema collassi in uno stato che sia autostato di ogni grandezza misurata.
    \end{itemize}
    \item Principio di Pauli: due fermioni (particelle che hanno spin semintero) non possono occupare lo stesso stato quantico.
    Si basa interamente su evidenza sperimentale, non può essere dedotto teoricamente.
    \item Principio di indistinguibilità delle particelle: due particelle identiche non possono essere distinte, ``etichettate''.
    Si pensi alle onde: non ha senso chiedersi se due onde che collidono si sovrappongono o si riflettono.

    È per questo che non vale più la distribuzione di Maxwell-Boltzmann: si basa sul fatto che ogni particella abbia un'``identità''.
\end{itemize}


Fondazione assiomatica della meccanica quantistica:
\begin{itemize}
    \item Principio dell'osservabile
    \item Principio dell'operatore
    \item Principio dell'evoluzione temporale: lo stato fisico evolve in quanto evolve il vettore di stato, purché inizialmente si conosca l'informazione massima sul sistema.
\end{itemize}

\chapter{Meccanica quantistica -- esercitazioni}

\section{Riassunto della teoria}

Relazioni di de Broglie:
\begin{equation}
    \begin{cases}
        E = h \nu = \hbar \omega \\
        p = \frac{h}{\lambda} = \hbar k
    \end{cases}
\end{equation}

Funzione d'onda di Schrödinger:
\begin{equation}
    \Psi = A \exp{\im (\vt{k} \cdot \p - \omega t)}
    = A \exp{\frac{\im (\vt{p} \cdot \p - Et)}{\hbar}}
\end{equation}

Equazione di Schrödinger completa:
\begin{equation}
    \im \hbar \parder{t} \Psi(\p, t) = - \frac{\hbar^2}{2 m} \lapl \Psi(\p, t) + V(\p, t) \Psi(\p, t)
\end{equation}
Vale con le ipotesi di
\begin{itemize}
    \item particella puntiforme
    \item $v \ll c$
    \item forze conservative
\end{itemize}

Separazione delle variabili:
\begin{equation}
    \Psi(\p, t) = R(\p) T(t)
\end{equation}

$T$ è armonica:
\begin{equation}
    T(t) = \exp{-\im E t/\hbar}
\end{equation}

L'equzione di Schrödinger stazionaria è:
\begin{equation}
\label{eq:schrstaz}
    \hat{H} \Psi(\p) = E \Psi(\p)
\end{equation}

$\hat{H}$ è l'hamiltoniana ed è un operatore lineare, quindi l'\cref{eq:schrstaz} è un'equazione agli autovalori.

È possibile scrivere decomporre $\Psi$ nella base degli autovettori di $\hat{H}$:
\begin{equation}
    \Psi = \sum_i a_i \Psi_i
\end{equation}

Svolgendo la misura, lo stato $\Psi	$ collassa in uno degli autostati:
\begin{gather}
    \hat{H} \Psi_i = E_i \Psi_i \\
    \abs{a_i}^2 = \mathrm{P}(\text{la particella sarà misurata nello stato $\Psi_i$})
\end{gather}

\section{Buca di potenziale}

È un caso di confinamento della particella in una regione di spazio.

In una dimensione,
\begin{equation}
    V(x) = \begin{cases}
        0 & 0 < x < L \\
        +\infty & \text{altrimenti}
    \end{cases}
\end{equation}
Ovvero, la particella è confinata nell'intervallo $(0, L)$.
Poiché non può uscire, $\Psi = 0$ al di fuori dell'intervallo e la rinormalizzazione interesserà solo quest'ultimo.

L'hamiltoniana classica è:
\begin{equation}
    H = \frac{p^2}{2 m} + V(x)
\end{equation}

Per ottenere l'hamiltoniana quantistica, si sostituisce $p$ con $\hat{p} = -\im \hbar \parder{x}$:

\begin{equation}
    \hat{H} = -\frac{\hbar^2}{2 m} \parder[][2]{x} + V(x)
\end{equation}

L'equazione di Schrödinger stazionaria risulta:
\begin{gather}
    -\frac{\hbar^2}{2 m} \parder[][2]{x} \Psi = E \Psi \\
    \parder[][2]{x} \Psi = -\frac{2 m E}{\hbar^2} \Psi
\end{gather}

Solitamente si definisce il fattore $\alpha$:
\begin{equation}
    \alpha \coloneq \frac{\sqrt{2 m E}}{\hbar}
\end{equation}

L'equazione diventa
\begin{equation}
    \parder[][2]{x} \Psi + \alpha^2 \Psi = 0
\end{equation}

$\Psi = A \exp{\lambda x}$, quindi l'equazione caratteristica è:
\begin{equation}
    \lambda^2 + \alpha^2 = 0
    \implies
    \lambda = \pm \im \alpha
\end{equation}

La soluzione generale è
\begin{equation}
    \Psi(x) = A \exp{\im \alpha x} + B \exp{-\im \alpha x}
\end{equation}
ovvero, la somma di un'onda progressiva e una regressiva.

I valori di $A$ e $B$ si determinano imponendo le condizioni al contorno:
\begin{gather}
    \begin{cases}
        \Psi(0) = 0 \implies B = -A \\
        \Psi(L) = 0 \implies A \exp{\im \alpha L} + B \exp{-\im \alpha L} = 0
    \end{cases} \\
    \implies \exp{\im \alpha L} - \exp{-\im \alpha L} = 0 \\
    \implies 2 \im \sin(\alpha L) = 0
    \implies \alpha L = n \pi
\end{gather}

Questo mostra che i valori di energia possibili sono quantizzati:
\begin{equation}
    \boxed{E_n = \frac{\hbar^2 \alpha^2}{2 m} = \frac{\pi^2 \hbar^2}{2 m L^2} n^2, \quad
    n \in \Z^+}
\end{equation}
Si esclude $n = 0$ \textit{a posteriori} poiché l'\cref{eq:autostati_scatola} non può essere la funzione nulla.

Trovati gli autovalori, si determinano gli autostati:
\begin{equation}
    \hat{H} \Psi_n = E_n \Psi_n
\end{equation}

Abbiamo già mostrato che le soluzioni sono nella forma:
\begin{equation}
    \Psi(x) = A \exp{\im \alpha x} - A \exp{-\im \alpha x}
    = 2 \im A \sin(\alpha x)
\end{equation}
Occorre imporre la normalizzazione:
\begin{gather}
    \int_0^L \abs{\Psi(\p)}^2 \de x
    = \int_0^L \abs{2 A \im \sin(\alpha x)}^2 \de x
    = 4 A^2 \int_0^L \sin^2(\alpha x) \de x
    = 2 A^2 L = 1 \\
    \implies
    A = \sqrt{\frac{1}{2 L}}
\end{gather}
È stato usato il fatto che $\sin(\alpha L) = 0$, e quindi $L$ è multiplo del periodo di $\sin^2(\alpha x)$.

Gli autostati, quindi, sono
\begin{equation}
\label{eq:autostati_scatola}
    \Psi_n(x) = \im \sqrt{\frac{2}{L}} \sin\pts{\frac{n \pi}{L} x}
\end{equation}

La densità di probabilità di trovare la particella in $x$ è
\begin{equation}
    \abs{\Psi_n(x)}^2 = \frac{2}{L} \sin^2\pts{\frac{n \pi}{L} x}
\end{equation}

In tre dimensioni, il numero quantico diventa un vettore $\vt{n} = (n_x, n_y, n_z)$ e gli autostati sono
\begin{equation}
    \Psi_n(\p) = 2 A \im \sin\pts{\frac{n_x \pi}{L} x} \sin\pts{\frac{n_y \pi}{L} y} \sin\pts{\frac{n_z \pi}{L} z}
\end{equation}

Livelli energetici:
\begin{equation}
    E_\vt{n} = \frac{\hbar^2}{2 m L^2} \pts{n_x^2 + n_y^2 + n_z^2}
\end{equation}
Sono detti \important{degeneri}, poiché più stati possono corrispondere allo stesso livello energetico.


\subsection{Molecola come buca di potenziale}

Modellizziamo la molecola \ce{H2} come due protoni di carica $e$ e massa $m_p = \qty{1.67e-27}{\kilo\gram}$ e un elettrone di carica $-e$ e massa $m_e = \qty{9.11e-31}{\kilo\gram}$.

L'equazione di Schrödinger stazionaria è nuovamente
\begin{gather}
    \parder[][2]{x} \Psi + \alpha^2 \Psi = 0
    \implies
    \Psi(x) = A \exp{\im \alpha x} + B \exp{-\im \alpha x} \\
    E_n = \frac{\pi^2 \hbar^2}{2 m L^2} n^2
\end{gather}

Se vogliamo trovare le velocità:
\begin{equation}
    E_n = \frac{1}{2} m v_n^2 \implies v_n
    = \frac{\pi \hbar}{m L} n \propto \frac{1}{L}
\end{equation}
Cioè, un confinamento maggiore porta a energie e velocità maggiori.

Con $L = \qty{1}{\meter}$, le velocità dei protoni saranno dell'ordine \qty{e-7}{\metre\per\second}, quelle degli elettroni dell'ordine \qty{e-4}{\metre\per\second}.

Con una buca di potenziale adatta alla descrizione di un atomo, come \qtyrange{0.5}{5}{\angstrom}, le velocità saranno dell'ordine \qty{e4}{\metre\per\second}.

\subsection{Transizioni energetiche}

Una particella può assorbire energia tramite un fotone e aumentare il suo livello energetico, cioè passare a uno stato eccitato.

Similmente, può emettere un fotone e diminuire il proprio livello energetico.

Esercizio: un ettrone in un atomo assorbe un fotone.
Rappresentiamo l'atomo come una buca di potenziale di lato $L = \qty{5}{\angstrom}$.
Qual è la lunghezza d'onda $\lambda$ del fotone?
\begin{equation}
    E_f = E_2 - E_1 = \frac{\hbar^2 \pi^2}{2 m L^2} (2^2 - 1^2), \quad
    \lambda = \frac{h c}{E_f} = \qty{275}{\nano\metre}
\end{equation}

Esercizio: qual è la dimensione della buca di potenziale per cui la diseccitazione di un elettrone da $n_2$ a $n_1$ corrisponde a un fotone con lunghezza d'onda $\lambda$?
\begin{equation}
    E_f = \frac{h c}{\lambda} = E_2 - E_1 = \frac{\hbar^2 \pi^2}{2 m L^2} (n_2^2 - n_1^2)
    \implies
    L = \sqrt{\frac{\lambda h (n_2^2 - n_1^2)}{8 m c}}
\end{equation}

Esercizio: all'interno di un nucleo ($L = \qty{e-15}{\metre}$), un protone assorbe un fotone e passa da $n_1 = 1$ a $n_2 = 2$.
Qual è la lunghezza d'onda del fotone?
\begin{equation}
    E_f = E_2 - E_1 = \frac{\hbar^2 \pi^2}{2 m L^2} (2^2 - 1^2), \quad
    \lambda = \frac{h c}{E_f} = \qty{2e-15}{\metre}
\end{equation}
I fotoni scambiati dai nuclei sono raggi gamma, molto energetici.

\section{Gradino di potenziale}

\begin{equation}
    V(x) = \begin{cases}
        0 & x < 0 \text{ (regione I)} \\
        V_0 & x \ge 0 \text{ (regione II)}
    \end{cases}
\end{equation}
con $V_0 > E$.
\begin{subequations}
\begin{gather}
    \hat{H}_\mathrm{I} = -\dfrac{\hbar^2}{2 m} \lapl \\
    \hat{H}_\mathrm{II} = -\dfrac{\hbar^2}{2 m} \lapl + V_0
\end{gather}
\end{subequations}

Definiamo i seguenti parametri
\begin{equation}
    \alpha = \sqrt{\frac{2 m E}{\hbar^2}}, \quad
    \beta = \sqrt{\frac{2 m (E - V_0)}{\hbar^2}} \eqcolon \im \gamma
\end{equation}
in modo che $\alpha, \gamma \in \R$, mentre $\beta$ è immaginario.
\begin{subequations}
\begin{gather}
    \Psi_\mathrm{I}(x) = A \exp{\im \alpha x} + B \exp{-\im \alpha x} \\
    \Psi_\mathrm{II}(x) = C \exp{-\gamma x} + D \exp{\gamma x}
\end{gather}
\end{subequations}

Imponiamo le seguenti condizioni:
\begin{itemize}
    \item $\int_{-\infty}^{+\infty} \abs{\Psi_\mathrm{I} + \Psi_\mathrm{II}}^2 \de x = 1 \implies D = 0$
    \item $\Psi \text{ continua} \implies \Psi_\mathrm{I}(0) = \Psi_\mathrm{II}(0) \implies A + B = C$
    \item $\Psi \text{ derivabile} \implies \Psi_\mathrm{I}'(0) = \Psi_\mathrm{II}'(0) \implies \im \alpha (A - B) = - \gamma C$
\end{itemize}

Interessano in particolare i rapporti $C/A$ e $B/A$: sono legati alle probabilità che l'onda sia trasmessa o riflessa, rispettivamente.

\begin{subequations}
\begin{align}
    B = C - A
    \implies \im \alpha A - \im \alpha (C - A) = - \gamma C
    \implies C = \frac{2 \im \alpha}{\im \alpha - \gamma} A \\
    C = A - B
    \implies \im \alpha A - \im \alpha B = - \gamma (A + B)
    \implies B = \frac{\im \alpha + \gamma}{\im \alpha - \gamma} A
\end{align}
\end{subequations}

Si definisce il coefficiente di riflessione:
\begin{equation}
    R = \abs{\frac{B}{A}}^2 = \frac{\alpha^2 + \gamma^2}{\alpha^2 + \gamma^2} = 1
\end{equation}
Cioè, la particella viene sempre riflessa.

Allo stesso tempo, la probabilità di trovare la particella nella regione II è
\begin{equation}
    \abs{\Psi_\mathrm{II}}^2 = \abs{C}^2 \exp{-2 \gamma x} \ne 0
\end{equation}
Non è contraddittorio poiché la particella, effettivamente, non viene trasmessa.

Questo è l'\important{effetto tunnel}.

\section{Buca di potenziale a pareti finite}

\begin{equation}
    V(x) = V_0 \, [x < 0 \lor x > L]
\end{equation}

A differenza del caso con pareti infinite, la probabilità di trovare la particella al di là delle pareti è non nulla, proprio a causa dell'effetto tunnel.

\section{Barriera di potenziale}

\begin{equation}
    V(x) = V_0 \, [0 < x < L]
\end{equation}
con $V_0 > E$.

Definiamo le zone I ($x < 0$), II ($0 < x < L$) e III ($x > L$).

Zone I e III:
\begin{equation}
    \parder[][2]{x} \Psi + \frac{2 m E}{\hbar^2} \Psi
    = \parder[][2]{x} \Psi + \alpha^2 \Psi
    = 0
\end{equation}
Zona II:
\begin{equation}
    \parder[][2]{x} \Psi + \frac{2 m (E - V_0)}{\hbar^2} \Psi
    = \parder[][2]{x} \Psi - \gamma^2 \Psi
    = 0
\end{equation}
avendo definito
\begin{subequations}
\begin{gather}
    \alpha = \sqrt{\frac{2 m E}{\hbar^2}} \\
    \gamma = \sqrt{\frac{2 m (V_0 - E)}{\hbar^2}}
\end{gather}
\end{subequations}

Soluzioni:
\begin{itemize}
    \item Regione I:
    \begin{equation}
        \Psi_\mathrm{I}(x) = A \exp{\im \alpha x} + B \exp{-\im \alpha x}
    \end{equation}
    \item Regione II:
    \begin{equation}
        \Psi_\mathrm{II}(x) = F \exp{-\gamma x} + G \exp{\gamma x}
    \end{equation}
    \item Regione III:
    \begin{equation}
        \Psi_\mathrm{III}(x) = C \exp{\im \alpha x}
    \end{equation}
\end{itemize}

Imponiamo continuità e derivabilità in $0$ e $L$:
\begin{gather}
    \Psi_\mathrm{I}(0) = \Psi_\mathrm{II}(0)
    \implies A + B = F + G \\
    \Psi_\mathrm{II}(L) = \Psi_\mathrm{III}(L)
    \implies F \exp{-\gamma L} + G \exp{\gamma L} = C \exp{\im \alpha L} \\
    \Psi_\mathrm{I}'(0) = \Psi_\mathrm{II}'(0) \implies
    \im \alpha A - \im \alpha B = -\gamma F + \gamma G \\
    \Psi_\mathrm{II}'(L) = \Psi_\mathrm{III}'(L) \implies
    -\gamma F \exp{-\gamma L} + \gamma G \exp{\gamma L} = \im \alpha C \exp{\im \alpha L}
\end{gather}

\redtext{
Poiché
\begin{equation}
    \int_{-\infty}^{+\infty} \abs{\Psi_\mathrm{I} + \Psi_\mathrm{II} + \Psi_\mathrm{III}}^2 \de x = 1
\end{equation}
si deve avere $G = 0$.
}

Si ottiene
\begin{equation}
    \frac{C}{A} = \frac{4 \im \alpha \gamma \exp{(\gamma - \im \alpha) L}}{(\gamma^2 - \alpha^2) (1 - \exp{2 \gamma L}) + 2 \im \alpha \gamma (1 + \exp{2 \gamma L})}
\end{equation}
e la probabilità di trasmissione è
\begin{gather}
    T = \abs{\frac{C}{A}}^2 \sim \exp{-2 \gamma L}
\end{gather}

Se le grandezze fisiche hanno dimensioni classiche, quindi, questa probabilità è praticamente nulla.

Nel caso micorscopico, invece, ad esempio con un elettrone con energia $E = \qty{5}{\electronvolt}$, $V_0 = \qty{6}{\electronvolt}$ e $L = \qty{0.5}{\angstrom}$, si ottiene $T = \qty{20}{\percent}$.

Per una particella che proviene da sinistra, la funzione d'onda sarà sinusoidale nella regione I, un esponenziale decrescente nella regione II e nuovamente una sinusoide nella regione III, di ampiezza pari alla precedente.

\section{Rotore rigido}

È un modello che si usa per rappresentare sistemi quantistici in cui è presente una rotazione.

Esempio: massa $m$ che orbita a distanza $r$ fissa intorno a un asse.

Il momento angolare classico è $\vt{L} = \vt{r} \times \vt{p}$.

In meccanica quantistica, il momento angolare è un operatore:
\begin{equation}
    \hat{\vt{L}} = \hat{\vt{r}} \times \hat{\vt{p}}
    = -\im \hbar \vt{r} \times \grad
\end{equation}

L'energia cinetica classica è $E = L^2 / (2 I)$, con $I = m r^2$ momento di inerzia.
Quindi,
\begin{equation}
    \hat{H} = \frac{\hat{L}^2}{2 I}
\end{equation}

Equazione di Schrödinger stazionaria:
\begin{equation}
    \hat{L}^2 \Psi = \underbrace{2 I E}_\lambda \Psi
\end{equation}

Si osserva anche che, $\forall i, j \in \{x, y, z\}, i \ne j$:
\begin{equation}
    [\hat{L}_i, \hat{L}_j] \ne 0
\end{equation}
Quindi, non è possibile conoscere due componenti del momento angolare simultaneamente (se due operatori non commutano, vale la disuguaglianza di Heisenberg).

Ad esempio:
\begin{equation}
    [\hat{\vt{x}}, \hat{\vt{p}}] = \im \hbar \ne 0
    \implies
    \Delta x \Delta p \ge \frac{\hbar}{2}
\end{equation}

Però vale sempre che
\begin{equation}
    [\hat{L}^2, \hat{L}_i] = 0
\end{equation}
Quindi posso conoscere una componente e il modulo.

Si può quindi associare all'equazione di Schrödinger l'equzione stazionaria per la componente $z$:
\begin{equation}
    \begin{cases}
        \hat{L}^2 \Psi = \lambda \Psi \\
        \hat{L}_z \Psi = M_z \Psi
    \end{cases}
\end{equation}

Gli autovalori sono:
\begin{gather}
    \lambda = l (l + 1) \hbar^2, \quad l \in \N \\
    M_z = m \hbar, \quad m \in \{-l, -l + 1, \ldots, l - 1, l\}
\end{gather}

Si introducono quindi i numeri quantici $l$ e $m$.

Ricordando che $\lambda = 2 I E$,
\begin{equation}
    E = \frac{\hbar^2}{2 I} l (l + 1)
\end{equation}

Quindi, $E$ è identificato da $l$ e si possono avere $2l + 1$ stati quantici diversi (identificati da $m$) associati allo stesso livello energetico.

L'energia del fotone associato alla transizione da $E_l$ a $E_{l-1}$ è
\begin{equation}
    h \nu = \frac{\hbar^2}{2 I} \pts{l (l + 1) - (l - 1) (l - 1 + 1)} = \frac{\hbar^2}{I} l
\end{equation}

Alla fine, abbiamo i seguenti \important{numeri quantici}:
\begin{table}[!h]
    \centering
    \begin{tabular}{|c|c|c|c|}
        \hline
        Nome & Simbolo & Operatore & Autovalori \\
        \hline
        Numero quantico principale & $n$ & $\hat{H}$ & $E_n$ \\
        Numero quantico azimutale & $l$ & $\hat{L}^2$ & $\hbar^2 l (l + 1)$ \\
        Numero quantico magnetico & $m$ & $\hat{L}_z$ & $m \hbar$ \\
        \hline
    \end{tabular}
    \caption{Numeri quantici e relativi operatori e autovalori}
    \label{tab:numeri_quantici}
\end{table}

\section{Oscillatore armonico}

\begin{equation}
    V(x) = \frac{1}{2} m \omega^2 x^2
\end{equation}

Operatore hamiltoniano:
\begin{equation}
    \hat{H} = -\frac{\hbar^2}{2 m} \parder[][2]{x} + \frac{1}{2} m \omega^2 x^2
\end{equation}

Equazione stazionaria di Schrödinger:
\begin{equation}
    \parder[][2]{x} \Psi + \pts{- \frac{m^2 \omega^2}{\hbar^2} x^2 + \frac{2 m E}{\hbar^2}} \Psi = 0
\end{equation}

Autovalori:
\begin{equation}
    E_n = \hbar \omega \pts{n + \frac{1}{2}}
\end{equation}

Autostati:
\begin{equation}
    \Psi(\xi) = H_n(\xi) \exp{-\frac{\xi^2}{2}}, \quad
    \xi = \sqrt{\frac{m \omega}{\hbar}} x
\end{equation}
usando i polinomi di Hermite:
\begin{equation}
    H_n(x) = (-1)^n \exp{x^2} \parder[][n]{x} \exp{-x^2}
\end{equation}

\redtext{Grafico di $V(x)$ con livelli energetici $E_n$}

\redtext{Grafici di $\Psi$ per $n = 0, 1, 2$}

Nota che $E_n \ne 0$ sempre, cioè non è possibile che l'oscillatore sia fermo, o sarebbe violato il principio di indeterminazione di Heisenberg per posizione e quantità di moto.

\section{Atomo monoelettronico}

Atomo con $Z$ protoni e un elettrone a distanza $r$.
\begin{gather}
    V(r) = -\frac{Z e^2}{4 \pi \eps_0 r} \\
    H = \frac{p^2}{2 m} - \frac{Z e^2}{4 \pi \eps_0 r} \\
    \hat{H} \Psi(\p) = \pts{-\frac{\hbar^2}{2 m} \lapl - \frac{Z e^2}{4 \pi \eps_0 r}} \Psi(\p) = E \Psi(\p)
\end{gather}

La soluzione in coordinate sferiche è
\begin{equation}
    \Psi_{n, l, m}(r, \theta, \phi) = \Phi_{n, l}(r) Y_l^m(\theta, \phi) \eqcolon \ket{n, l, m}
\end{equation}
le funzioni $Y_l^m$ sono le armoniche sferiche.

Le tre equazioni stazionarie sono:
\begin{equation}
    \begin{cases}
        \hat{H} \Psi_{n,l,m} = E_n \Psi_{n,l,m} \\
        \hat{L}^2 \Psi_{n,l,m} = \hbar^2 l (l + 1) \Psi_{n,l,m} \\
        \hat{L}_z \Psi_{n,l,m} = \hbar m \Psi_{n,l,m}
    \end{cases}
\end{equation}

Ogni livello energetico $E_n$ ha $n$ possibili valori di $l$ e ogni sottolivello (individuato da $n$ ed $l$) ha $2 l + 1$ possibili valori di $m$:
\begin{subequations}
\begin{gather}
    n \in \Z^+ \\
    l \in \{0, 1, \ldots, n - 1\} \\
    m \in \{-l, -l + 1, \ldots, l - 1, l\}
\end{gather}
\end{subequations}

Gli autovalori $E_n$ sono
\begin{equation}
    E_n = -\frac{m Z^2 e^4}{2 \pts{4\pi \eps_0}^2 \hbar^2 n^2} \propto -\frac{Z^2}{n^2}
\end{equation}
È la stessa espressione dell'\cref{eq:energia_bohr}, l'energia di un elettrone secondo il modello di Bohr (lì, con $Z = 1$).


Ogni possibile stato $\ket{n, l, m}$ corrisponde a un orbitale.

Si indicano con numeri i livelli di energetici e con $s, p, d, f$ i valori di $l = 0, 1, 2, 3$ (le lettere proseguono poi in ordine alfabetico da $g$ in avanti).
Ad esempio, $3p$ è il sottolivello con $n = 3$, $l = 1$ e $m = -1, 0, 1$.

La forma degli orbitali dipende dalle armoniche sferiche:
\begin{table}[!h]
    \centering
    \begin{tabular}{|c|c|c|c|}
        \hline
        $l$ & Orbitali & Forma & Immagine \\
        \hline
        0 & 1 orbitale $s$ & sferica & \redtext{immagine}
        \\
        1 & 3 orbitali $p$ & bilobata & \redtext{immagine}
        \\
        \hline
    \end{tabular}
    \caption{Orbitali atomici per diversi valori di $l$}
    \label{tab:orbitali}
\end{table}

\subsection{Esperimento di Stern-Gerlach e spin}

Pauli introdusse un quarto numero quantico legato a una proprietà (lo \important{spin}) che esprime il momento angolare intrinseco di una particella.
Questo numero quantico è detto \important{numero quantico di spin} e si indica con $m_s$.

Consideriamo un fascio di elettroni che attraversa una regione con campo magnetico $\B(\p) = B(z) \uz$ e colpisce uno schermo.
Classicamente, mi aspetterei una striscia continua di elettroni sullo schermo:
\begin{gather}
    \parder[B_z]{z} \ne 0 \\
    U = - \vt{\mu} \cdot \B \\
    \force = - \grad U = \mu_z \parder[\B]{z}
\end{gather}
Invece, si riconoscono due fasci.
Anche lo spin, quindi, è discreto e può assumere solo due valori opposti.

Questo fenomeno avviene lungo ogni direzione e non è possibile conoscere simultaneamente le componenti di spin lungo due direzioni diverse (come nel caso del momento angolare, non commutano).

Lo spin (questo momento angolare intrinseco) si comporta come un momento angolare orbitale.
\begin{gather}
    \hat{s}^2 \Psi = \hbar^2 s (s + 1) \Psi \\
    \hat{s}_z \Psi = \hbar m_s \Psi
\end{gather}
Inoltre, per gli elettroni, vale sempre
\begin{equation}
    s = \frac{1}{2} \implies m_s = \pm \frac{1}{2}
\end{equation}

Lo stato di un elettrone, quindi, si scrive come $\ket{n, l, m, m_s}$.
I primi tre numeri quantici individuano l'orbitale, l'ultimo caratterizza lo spin della particella.

Il principio di esclusione di Pauli afferma (in particolare) che due elettroni non possono avere lo stesso stato quantico.

Esempio: i due elettroni dell'atomo di elio (che ha configurazione elettronica $1s^2$) avranno stati $\ket{1, 0, 0, \pm 1/2}$.

È anche possibile distinguere due tipi di particelle:
\begin{itemize}
    \item \important{Fermioni}, con spin semintero, che obbediscono al principio di esclusione di Pauli
    \item \important{Bosoni}, con spin intero, che \important{non} obbediscono al principio di esclusione di Pauli

    Più bosoni possono quindi occupare lo stesso stato quantico, anche in gran numero.
\end{itemize}

\section{Nucleo atomico}

Un nucleo $_Z^A X$ è identificato dal numero di protoni $Z$, dal numero di neutroni $N$ e dal numero di massa $A = Z + N$.

Il nucleo è descritto da questo potenziale:
\begin{equation}
    V(r) = \begin{cases}
        -E_\text{legame} & r < R \\
        \dfrac{Z e^2}{4 \pi \eps_0 r} & r > R \\
    \end{cases}
\end{equation}

Definiamo anche $C$, il valore per $r = R^+$:
\begin{equation}
    C \eqcolon \frac{Z e^2}{4 \pi \eps_0 R}
\end{equation}

L'energia di legame si esprime con la formula:
\begin{equation}
    E_\text{legame} = \Delta m c^2
\end{equation}
dove $\Delta m$ è la differenza di massa tra i nucleoni liberi e quelli legati nel nucleo.

Infatti, quando delle particelle si legano a formare un nucleo, si ha una perdita di massa (\important{difetto di massa}), che si trasforma in energia di legame.

\subsection{Decadimento alfa}

La particella $\alpha$ è un nucleo di elio, cioè due protoni e due neutroni.
\begin{equation}
    \alpha = \ce{^4_2He}
\end{equation}

Decadimento alfa:
\begin{equation}
    \ce{^{A}_{Z}X -> ^{A-4}_{Z-2}Y + \alpha}
\end{equation}

Esempio:
\begin{equation}
    \ce{^{226}_{88}Ra -> ^{222}_{86}Rn + \alpha}
\end{equation}

Il radio ha $C_{\ce{Ra}} = \qty{50}{\mega\electronvolt}$ e $R_{\ce{Ra}} = \qty{e-15}{\metre}$.
Si osserva che la particella $\alpha$ emessa ha una energia cinetica di circa \qty{10}{\mega\electronvolt}.

Il problema è stato descritto da Gamow.
\begin{gather}
    V(R_B) = \frac{(Z - 2)e \cdot 2 e}{4 \pi \eps_0 R_B} = \qty{10}{\mega\electronvolt} \\
    \abs{\frac{C}{A}}^2 = \num{e-31} \\
    f_\text{urti} = \frac{v_\alpha}{2 R_{\ce{Ra}}}
\end{gather}

$v_\alpha$ si stima con il principio di indeterminazione:
\begin{gather}
    \Delta x \Delta p \approx \frac{\hbar}{2}
    \implies
    \Delta (2 R_{\ce{Ra}}) (m_\alpha \Delta v_\alpha) \approx \frac{\hbar}{2} \\
    \implies
    v_\alpha \sim \qty{e5}{\metre\per\second}
    \implies
    f_\text{urti} \sim \qty{e20}{urti\per\second}
\end{gather}

Esercizio: quanti decadimenti al secondo avvengono in un campione di \qty{1}{\gram} di radio?
\begin{gather}
    N = \frac{m}{{M\!M}_{\ce{Ra}}} N_A = \qty{e21}{atomi} \\
    f_\text{urti} \cdot N \cdot \abs{\frac{C}{A}}^2 = \qty{e10}{decadimenti\per\second}
\end{gather}

\subsection{Fusione nucleare}

\redtext{Grafico di $V(r)$ per la fusione nucleare}

Fusione di due protoni in un nucleo di deuterio:
\begin{equation}
    \ce{^1_1H + ^1_1H -> ^2_1H}
\end{equation}

L'energia potenziale massima è \qty{14}{\mega\electronvolt}, quella che un protone deve avere per svolgere la fusione è \qty{1.9}{\kilo\electronvolt}, quindi il fenomeno deve avvenire per effetto tunnel.

\begin{equation}
    E_k = \frac{3}{2} k_B T = \frac{e^2}{4 \pi \eps_0 r} \implies r \sim \qty{e-13}{\metre}
\end{equation}

\begin{equation}
    \abs{\frac{C}{A}}^2 = \num{e-36}
\end{equation}

Successivamente alla fusione di due protoni, si ha la fusione di un protone con il deuterio:
\begin{gather}
    \ce{^1_1H + ^1_1H -> ^2_1H} \\
    \ce{^2_1H + ^1_1H -> ^3_2He} \\
    \ce{^3_2He + ^1_1H -> ^4_2He}
\end{gather}

\subsection{Fissione nucleare}

\begin{equation}
    \ce{^{235}_{92}U + n -> ^{141}_{56}Ba + ^{92}_{36}Kr + 3 n + \qty{200}{\mega\electronvolt}}
\end{equation}

I tre neutroni emessi colpiscono altri nuclei di uranio e il processo si autoalimenta in una reazione a catena.

Serve però un campione di uranio arricchito, cioè con una percentuale maggiore del normale di isotopi fissili.

\section{Esercizi vari}

Esercizio: trovare i possibili valori di energia per una particella di massa $m = \qty{e-30}{\kilo\gram}$ e carica $Q = \qty{e-19}{\coulomb}$ in moto a distanza $R = \qty{1}{\nano\metre}$ da un punto.
\begin{gather}
    \hat{H} \Psi = E \Psi \implies \frac{\hat{L}^2}{2 I} \Psi = E \Psi
    \implies
    \hat{L}^2 \Psi = 2 I E \Psi \\
    2 I E = \hbar^2 l (l + 1) \implies E = \frac{\hbar^2}{2 m R^2} l (l + 1)
\end{gather}

Calcolare la frequenza del fotone che permette il passaggio dal secondo livello al livello fondamentale.
\begin{equation}
    \nu = \frac{E_2 - E_0}{h} = \frac{\hbar^2}{2 m R^2 h} \pts{2 (2 + 1) - 0 (0 + 1)}
    = \frac{3 \hbar}{2 \pi m R^2}
    = \qty{5e13}{\hertz}
\end{equation}

Esercizio:
calcolare $[x, p_y]$.
\begin{gather}
    \Psi(\vt{x}) = \exp{\im \vt{k} \cdot \vt{x}} = \exp{\frac{\im \vt{p} \cdot \vt{x}}{\hbar}} \\
    \grad \Psi(\vt{x}) = \im \frac{\vt{p}}{\hbar} \Psi(\vt{x})
    \implies \vt{p} \Psi(\vt{x}) = -\im \hbar \grad \Psi(\vt{x})
    \implies \hat{\vt{p}} = -\im \hbar \grad \\
    [\hat{x}, \hat{p}_y] = \hat{x} \hat{p}_y - \hat{p}_y \hat{x} = -\im \hbar x \parder{y} + \im \hbar \parder{y} x = 0
\end{gather}

Esercizio:
determinare la massa $m$ di un corpo vincolato a una molla di costante elastica $k = \qty{4e-8}{\newton\per\metre}$ se, transendo tra due livelli contigui, la lunghezza d'onda del fotone associato è $\lambda = \qty{400}{\nano\metre}$.
\begin{gather}
    \hat{H} = -\frac{\hbar^2}{2 m} \parder[][2]{x} + \frac{1}{2} m \omega^2 x^2 \\
    \hat{H} \Psi = E \Psi \implies
    \parder[][2]{x} \Psi = \pts{\frac{m^2 \omega^2}{\hbar^2} x^2 - \frac{2 m E}{\hbar^2}} \Psi \\
    E_n = \pts{n + \frac{1}{2}} \hbar \omega \\
    E_{n+1} - E_n = \hbar \omega = \hbar \sqrt{\frac{k}{m}} = \frac{h c}{\lambda} \\
    m = \frac{k \lambda^2}{4 \pi^2 c^2} = \qty{1.8e-39}{\kilo\gram}
\end{gather}

Esercizio:
un elettrone è in una buca di potenziale con barriere infinite di lato $L = \qty{0.1}{\nano\metre}$.
Trovare l'energia del secondo livello eccitato e la lunghezza d'onda del fotone che va assorbito per saltare dal livello fondamentale al secondo livello eccitato e scrivere la funzione d'onda per questi due livelli.
\begin{gather}
    \hat{H} = -\frac{\hbar^2}{2 m} \lapl, \quad \text{ per } x \in [0, L] \\
    \bigg(\parder[][2]{x} + \underbrace{\frac{2 m E}{\hbar^2}}_{\alpha^2}\bigg) \Psi = 0 \\
    \Psi(x) = A \exp{\lambda x} \implies \lambda^2 + \alpha^2 = 0 \implies \lambda = \pm \im \alpha \\
    \Psi(x) = A \exp{\im \alpha x} + B \exp{-\im \alpha x} \\
    \Psi(0) = 0 \implies A + B = 0 \\
    \Psi(L) = 0 \implies \exp{\im \alpha L} - \exp{-\im \alpha L} = 0 \implies \sin(\alpha L) = 0 \\
    \implies \alpha = \frac{n \pi}{L}, \, n \in \N \\
    E_n = \frac{\hbar^2 \pi^2}{2 m L^2} n^2 \\
    \int_0^L \abs{\Psi}^2 \de x = 1 \implies 4 A^2 \int_0^L \sin^2(\alpha x) \de x = 1 \implies A = \frac{1}{\sqrt{2 L}} \\
    \Psi_n(x) = 2 \im \frac{1}{\sqrt{2 L}} \sin\pts{\frac{n \pi x}{L}} \\
    \implies n \ge 1
\end{gather}
Il secondo livello eccitato, quindi, è quello per $n = 3$.
\begin{gather}
    E_3 = \frac{\hbar^2 \pi^2}{2 m L^2} 3^2 = \qty{338}{\electronvolt} \\
    \lambda = \frac{h c}{E_3 - E_1} = \frac{2 h c m L^2}{\hbar^2 \pi^2 \cdot (9 - 1)} = \qty{4.13}{\nano\metre} \\
    \Psi_1(x) = \im \sqrt{\frac{2}{L}} \sin\pts{\frac{\pi x}{L}} \\
    \Psi_3(x) = \im \sqrt{\frac{2}{L}} \sin\pts{\frac{3 \pi x}{L}}
\end{gather}

Esercizio:
molecola di azoto \ce{N2} con atomi a distanza fissa $d$.
Trovare le energie dello stato fondamentale e dei primi due stati eccitati.
Si consideri solo la rotazione.

Sia $m$ la massa di un nucleone e $M = 14 m$ la massa di un nucleo di azoto.
\begin{gather}
    I = 2 M \pts{\frac{d}{2}}^2 \\
    \hat{H} = \frac{\hat{L}^2}{2 I} \\
    \hat{L}^2 \Psi = 2 I E \Psi \\
    2 I E = \hbar^2 l (l + 1) \implies E
    = \frac{\hbar^2}{2 I} l (l + 1)
    = \frac{\hbar^2}{M d^2} l (l + 1)
    = \frac{\hbar^2}{14 m d^2} l (l + 1) \\
    E_0 = 0 \\
    E_1 = \frac{\hbar^2}{7 m d^2} \\
    E_2 = \frac{3 \hbar^2}{7 m d^2}
\end{gather}

Esercizio:
quanto vale $E_1 - E_0$ per una molecola di ossigeno \ce{O2} con atomi a distanza fissa $d = \qty{1.2}{\angstrom}$ e massa $M = \qty{2.2e-26}{\kilo\gram}$?
\begin{gather}
    I = 2 M \pts{\frac{d}{2}}^2 \\
    2 I E = \hbar^2 l (l + 1) \implies E_l = \frac{\hbar^2}{2 I} l (l + 1) \\
    E_1 - E_0 = \frac{\hbar^2}{2 I} \pts{1 (1 + 1) - 0 (0 + 1)}
    = \frac{2 \hbar^2}{M d^2}
    = \qty{439}{\micro\electronvolt}
\end{gather}

Esercizio:
trovare la massa degli atomi a distanza $d = \qty{2}{\nano\metre}$ in una molecola, sapendo che la lunghezza d'onda del fotone associato al salto tra i livelli 0 e 1 è $\lambda = \qty{500}{\nano\metre}$.
\begin{gather}
    E_1 - E_0 = \frac{\hbar^2}{2 I} \pts{1 (1 + 1) - 0 (0 + 1)} = \frac{h c}{\lambda} \\
    I = \frac{\hbar^2 \lambda}{hc} = \frac{\hbar \lambda}{2 \pi c} = 2 M \pts{\frac{d}{2}}^2 \\
    M = \frac{\hbar \lambda}{\pi c d^2} = \qty{1.4e-31}{\kilo\gram}
\end{gather}

\chapter{Fisica dello stato solido}

Abbiamo risolto l'equazione di Schrödinger nel caso di una particella che
\begin{itemize}
    \item trasla (conferma la discretizzazione)
    \item oscilla (conferma l'ipotesi \textit{ad hoc} di Planck, giusta: $\Delta E = \hbar \omega n$)
    \item ruota (conferma l'ipotesi \textit{ad hoc} di Bohr, sbagliata: $L = \hbar \sqrt{l (l + 1)}$, e non $L = \hbar l$).
\end{itemize}
Nel caso dell'effetto tunnel non si ha alcuna discretizzazione perché, dal momento che la particella può essere in tutto $\R$, non ci sono condizioni al contorno.

Studiamo ora il comportamento degli elettroni nei solidi.

\section{Modellizzazione}

Consideriamo un cristallo (struttura ordinata e periodica) a struttura cubica (ogni atomo è circondato da sei atomi, secondo i vertici di un cubo).

Non è possibile risolvere analiticamente il problema descrivendo tutti gli elettroni del sistema, poiché sono in numero dell'ordine di grandezza del numero di Avogadro.

Per non perdere di vista il messaggio fisico, si svolgono cinque ipotesi semplificative.
\begin{enumerate}
    \item $V$ non è l'energia potenziale coulombiana, ma una schiera di buche di potenziale.

    $V$ dovrebbe essere:
    \begin{equation}
        V(\p) = \sum_i -\frac{Z e^2}{4 \pi \eps_0 \norm{\p - \p_i}}
    \end{equation}
    Tuttavia, l'elettrone si trova in uno stato fondamentale molto profondo (tanto più quanto maggiore è $Z$), quindi vede una barriera praticamente verticale che può superare solo in caso di ionizzazione.

    L'energia potenziale si può quindi approssimare con una buca di potenziale a pareti finite per ogni nucleo.

    \item Il problema è monodimensionale (si può scegliere una direzione arbitraria poiché lo spazio è isotropo).
    \item Il solido ha dimensione infinita, per garantire la periodicità e limitare le definizioni a tratti
    \item Il solido contiene un solo elettrone, oltre ai nuclei positivi.
    \item Il solido è a temperatura $T = \qty{0}{\kelvin}$, così che i nuclei siano fermi.
\end{enumerate}

Descriveremo la soluzione nel caso di questo ``solido'', e poi elimineremo queste semplificazioni per avvicinarci al comportamento di un solido reale.

Questo solido è quindi rappresentabile tramite una disposizione di buche di potenziale profonde $V_0$, di ampiezza $a$ e a distanza $b$.
\begin{equation}
    % V(x) = V_0 \sum_{k = -\infty}^{\infty} [(a + b)k + a < x < (a + b)k + a + b] \\
    V(x) = \begin{cases}
        0 & x \in [0, a] \\
        V_0 & x \in [a, a + b]
    \end{cases}
\end{equation}
estesa per periodicità a tutto $\R$.
Qui, $V_0$ è l'\important{energia di ionizzazione} e $a + b \approx a$ è il \important{passo reticolare}.

% Si potrà rappresentare un cristallo a struttura cubica definendo un vettore $\vt{a}$ in modo che tutti i nuclei siano in posizioni $n \vt{a}$.
% Le tre componenti di $\vt{a}$ sono i passi reticolari nelle tre direzioni.

Condizioni:
\begin{subequations}
\label{eq:condcristallo}
    \begin{gather}
        V_0 \gg E \\
        b \ll a
    \end{gather}
\end{subequations}
con $E$ energia dell'elettrone.
Inoltre, $V$ è periodico di periodo $a + b$.

\section{Soluzione dell'equazione di Schrödinger}

Siano ``regione I'' le barriere e ``regione II'' le buche.

Equazioni di Schrödinger nelle regioni 1 e 2:
\begin{subequations}
\label{eq:schrcristallo}
    \begin{gather}
        -\frac{\hbar^2}{2 m} \parder[][2]{x} \Psi_\mathrm{I} = E \Psi_\mathrm{I}
        \implies
        \parder[][2]{x} \Psi_\mathrm{I} = - \underbrace{\frac{2 m E}{\hbar^2}}_{\alpha^2} \Psi_\mathrm{I} \\
        -\frac{\hbar^2}{2 m} \parder[][2]{x} \Psi_\mathrm{II} + V_0 \Psi_\mathrm{II} = E \Psi_\mathrm{II}
        \implies
        \parder[][2]{x} \Psi_\mathrm{II} = \underbrace{\frac{2 m (V_0 - E)}{\hbar^2}}_{\gamma^2} \Psi_\mathrm{II}
    \end{gather}
\end{subequations}

Un elettrone che si trova inizialmente in una buca, dopo un certo tempo ha probabilità finita di trovarsi ovunque.

Quindi, mi aspetto una collettività di elettroni che interagiscono con una collettività di nuclei.

Bloch dimostrò (\important{teorema di Bloch}) che quando l'equazione di Schrödinger ha un potenziale periodico, la soluzione è del tipo
\begin{equation}
\label{eq:bloch}
    \Psi(x) = u(x) \exp{\im k x}
\end{equation}
dove $k \in \R$ è un parametro di adimensionamento con dimensioni $\dimension{k} = \mathsf{L}^{-1}$ e $u$ ha la stessa periodicità del potenziale:
\begin{equation}
    u(x + a + b) = u(x) \quad \forall x \in \R
\end{equation}

Sostituendo nelle due soluzioni $\Psi_\mathrm{I}$ e $\Psi_\mathrm{II}$ nella \eqref{eq:schrcristallo}, si ottiengono due equazioni differenziali in $u_\mathrm{I}$ e $u_\mathrm{II}$
\begin{subequations}
    \begin{gather}
        \parder[][2]{x} u_\mathrm{I} + 2\im k \parder{x} u_\mathrm{I} - (k^2 - \alpha^2) u_\mathrm{I} = 0 \\
        \parder[][2]{x} u_\mathrm{II} + 2\im k \parder{x} u_\mathrm{II} - (k^2 + \gamma^2) u_\mathrm{II} = 0
    \end{gather}
\end{subequations}
con soluzioni
\begin{subequations}
    \begin{gather}
        u_\mathrm{I}(x) = \exp{-\im k x} \pts{A \exp{\im \alpha x} + B \exp{-\im \alpha x}} \\
        u_\mathrm{II}(x) = \exp{-\im k x} \pts{C \exp{\gamma x} + D \exp{-\gamma x}}
    \end{gather}
\end{subequations}

Condizioni al contorno: continuità di $\Psi$ e della sua derivata prima all'inizio e alla fine di ogni barriera.
\begin{subequations}
    \begin{gather}
        u_\mathrm{I}(0) = u_\mathrm{II}(0) \qquad
        \evalat{\der[u_\mathrm{I}]{x}}{x = 0} = \evalat{\der[u_\mathrm{II}]{x}}{x = 0} \\
    \label{eq:boundaryuii}
        u_\mathrm{I}(a) = u_\mathrm{II}(-b) \qquad
        \evalat{\der[u_\mathrm{I}]{x}}{x = a} = \evalat{\der[u_\mathrm{II}]{x}}{x = -b}
    \end{gather}
\end{subequations}

Sostituire $a$ con $-b$ per $u_\mathrm{II}$ nella \eqref{eq:boundaryuii} permette di tenere automaticamente in considerazione il fatto che $u$ è periodica e che le condizioni al contorno vanno imposte per ogni buca.

Risulta un sistema lineare omogeneo di quattro equazioni in quattro incognite:
\begin{equation}
    \begin{bmatrix}
        a_{11} & a_{22} & a_{13} & a_{14} \\
        a_{21} & a_{22} & a_{23} & a_{24} \\
        a_{31} & a_{32} & a_{33} & a_{34} \\
        a_{41} & a_{42} & a_{43} & a_{44}
    \end{bmatrix}
    \begin{bmatrix}
        A \\ B \\ C \\ D
    \end{bmatrix}
    = \begin{bmatrix}
        0 \\ 0 \\ 0 \\ 0
    \end{bmatrix}
\end{equation}
I coefficienti dipendono da $k$, $\alpha$ e $\gamma$.

Perché il sistema abbia soluzioni non banali è necessario che la matrice dei coefficienti non abbia rango massimo.
Porre il determinante uguale a zero corrisponde alla seguente equazione:
\begin{equation}
    \frac{\gamma^2 - \alpha^2}{2 \alpha \gamma} \sinh(\gamma b) \sin(\alpha a) + \cosh(\gamma b) \cos(\alpha a) = \cos\pts{k(a + b)}
\end{equation}
Si usano le seguenti approssimazioni derivate dalle condizioni nella \eqref{eq:condcristallo}:
\begin{subequations}
    \begin{gather}
        b \ll a \implies \cosh(\gamma b) \approx 1, \ \sinh(\gamma b) \approx \gamma b \\
        E \ll V_0 \implies \alpha \ll \gamma
    \end{gather}
\end{subequations}
che risultano in
\begin{equation}
\label{eq:bande}
    \begin{gathered}
        \frac{\gamma^2 - \cancel{\alpha^2}}{2 \alpha \cancel{\gamma}} \cancel{\gamma} b \sin(\alpha a) + \cancel{\cosh(\gamma b)} \cos(\alpha a) = \cos\pts{k(a + \cancel{b})} \\
        \implies \frac{\gamma^2 b}{2 \alpha} \sin(\alpha a) + \cos(\alpha a) = \underbrace{\frac{m a b V_0}{\hbar^2}}_P \frac{\sin(\alpha a)}{\alpha a} + \cos(\alpha a) = \cos(ka) \\
        \implies
        \boxed{P \, \frac{\sin(\alpha a)}{\alpha a} + \cos(\alpha a) = \cos(ka)}
    \end{gathered}
\end{equation}
Questa è la condizione che l'elettrone deve rispettare per esistere nel solido e indica quanto devono valere il parametro di Bloch $k$ e l'energia $E$ dell'elettrone (implicita in $\alpha$).

Se $V_0 = 0$, l'elettrone non vedrebbe alcun potenziale e sarebbe libero.
Varrebbe $P = 0$ e la \eqref{eq:bande} diventerebbe
\begin{gather}
    \cos(\alpha a) = \cos(k a)
    \implies
    k^2 = \alpha^2 = \frac{2 m E}{\hbar^2} = \frac{2 m}{\hbar^2} \frac{p^2}{2 m}
    \implies
    p = \hbar k
\end{gather}
$k$ ha quindi un significato fisico legato alla quantità di moto dell'elettrone.

\section{Schema a bande}

In generale, la \eqref{eq:bande} relaziona $E$ e $k$ (ma non è si può esplicitare rispetto a nessuna delle due variabili).
Non è quindi possibile che l'elettrone abbia energia e quantità di moto arbitrarie.
Il luogo dei punti $(k, E)$ che soddisfano la \eqref{eq:bande} (non nel caso di elettrone libero) è:

\redtext{Figura}

Il periodo delle curve è $2 \pi / a$, quindi i punti di massimo e minimo di $E$ rispetto a $k$ sono per
\begin{equation}
    k = \frac{\pi}{a} n, \quad n \in \N
\end{equation}

Lo stesso grafico sarebbe da interpretare in tre dimensioni, con $\vt{k} = (k_x, k_y, k_z)$.
Ogni elettrone sarà caratterizzato da una quaterna $(E, \vt{k})$.

Sono le informazioni contenute della funzione d'onda.
Questo significa che, di un elettrone in un solido, posso misurare solamente energia e quantità di moto senza indeterminazione.

In realtà manca lo spin, che non è considerato dall'equazione di Schrödinger, poiché questa tratta solo masse puntiformi e questo sarebbe contraddittorio con il concetto di momento angolare intrinseco.

Quindi, lo stato è caratterizzato da
\begin{equation}
    (E, \vt{k}, s) \longleftrightarrow \ket{\Psi, \uparrow}
\end{equation}
A destra è rappresentata la notazione di Dirac, che evidenzia come $\Psi$ e $\uparrow$ siamo elementi di spazi vettoriali diversi.
\begin{table}[!h]
    \centering
    \begin{tabular}{|c|c|c|}
        \hline
        Elettrone in un... & Caratteristiche fisiche \\
        \hline
        atomo & 4: $(n, l, m, m_s)$ \\
        solido & 5: $(E, k_x, k_y, k_z, s)$ \\
        \hline
    \end{tabular}
    \caption{Caratteristiche fisiche degli elettroni in un atomo e in un solido}
    \label{tab:caratteristiche}
\end{table}

Un elettrone in un solido, ora, è propriamente un onda (lo si vede dalla forma della \eqref{eq:bloch}).

La ``posizione'' di un onda è meglio descritta nello \important{spazio reciproco}, in cui ``vive'' il numero d'onda $k$, anziché nello spazio reale dove  ``vive'' la posizione.

La periodicità in $k$ vuol dire che
\begin{equation}
    \Psi(x; k) = \Psi\pts{x; k + \frac{2\pi}{a}}
\end{equation}
$k$ è un parametro, non l'argomento della funzione.
Questo vuol dire che al crescere della quantità di moto l'energia non necessariamente aumenta, ma si ripete.
La funzione d'onda, inoltre, si ripeterà anch'essa.

Per questa ragione, la quantità sull'asse delle ascisse viene talvolta chiamata ``pseudo-quantità di moto''.

È quindi possibile studiare $\Psi(x)$ in un unica oscillazione fondamentale: per $k \in [-\pi / a, \pi / a]$, regione detta \important{prima zona di Brillouin}.

Il grafico $E$ vs. $k$ nella prima zona di Brillouin è detto \important{schema a bande}:

\redtext{Figura}

Ogni curva rappresenta una \important{banda di energia} ``nel senso della fisica'': ci sono zone di energia in cui l'elettrone può esistere (bande di energia \important{permesse}) e zone in cui non può (bande di energia \important{proibite}).

Le proiezioni delle bande sull'asse $E$ costituiscono le bande ``nel senso dell'elettronica''.

Ogni banda, inoltre, corrisponde fisicamente da un livello energetico nel senso degli orbitali (la prima banda all'$1s$, la seconda a $2s$ e $2p$ ecc.).

\section{Solido finito}

Ora rimuoviamo l'ipotesi di estensione infinita del solido.

Un solido unidimensionale infinito, poiché periodico, può essere descritto come un'infinità di solidi uguali di lunghezza $L = N a$, che è un multiplo intero del passo reticolare $a$:
\begin{gather}
    \Psi(x) = \Psi(x + L) = \Psi(x + N a) \\
    \implies u(x) \exp{\im k x} = u(x + L) \exp{\im k (x + L)} \\
    \implies 1 = \exp{\im k L} \implies 2\pi \mid k L \\
    \implies k = \frac{2\pi}{L} n, \quad n \in \Z
\end{gather}
Poiché siamo in una dimensione,
\begin{equation}
    k_x = \frac{2\pi}{L} n_x
\end{equation}

In un intervallo $[-\pi / a, \pi / a]$ ci sono quindi solo alcuni valori discreti di $k$ possibili, separati a distanza $2\pi / L$.
I valori ammessi sono in numero
\begin{equation}
    N = \frac{\frac{2\pi}{a}}{\frac{2\pi}{L}} = \frac{L}{a}
\end{equation}
Ovvero, il rapporto tra la lunghezza del cristallo e il passo reticolare.
Questo numero è tanto alto che il carattere discreto delle bande non è osservabile sperimentalmente.

\section{Più elettroni}

Rimuoviamo l'ipotesi di un solo elettrone.

Un elettrone solo, nello schema a bande, si collocherà nel punto di minima energia, in $(E = 0, \, \vt{k} = \vt{0}, \, \uparrow)$.
Un secondo elettrone si collocherà nella stessa posizione ma, per il principio di Pauli, con spin opposto: $(E = 0, \, \vt{k} = \vt{0}, \, \downarrow)$.
Nota che ogni elettrone introdotto vede la periodicità del potenziale, quindi non ha senso pensare che possa scegliere un altro dei minimi periodici.

Come si aggiungono altri elettroni?
Il terzo si dovrà collocare in un punto contiguo (le bande sono discrete se il solido è finito) e avrà energia non nulla.
Questo vuol dire che allo zero assoluto possiamo immaginare i nuclei fermi ma gli elettroni avranno comunque energia e quantità di moto non nulle.

Se la banda ha un massimo di energia al centro anziché un minimo, allora quattro elettroni potranno posizionarsi a energia minima negli estremi della banda.

In tutto, ogni banda può ospitare $2 N$ elettroni.

Esempio: un atomo di sodio \ce{Na} ha 11 elettroni.
Se ne posizionano $2 N$ in ciascuna delle prime cinque bande e $N$ nell'ultima.
L'ultima banda sarà piena a metà e l'energia massima occupata si chiama \important{energia di Fermi} $E_F$.
I materiali cristallini che si riempiono come il sodio sono i \important{conduttori}.

Altro esempio: silicio \ce{Si}, 14 elettroni.
$2 N$ in ciascuna delle prime sette bande.
L'ottava è del tutto vuota.
I materiali cristallini che si riempiono come il silicio sono gli \important{isolanti}.
L'ultima banda piena si chiama banda di valenza (BV), la prima banda vuota si chiama banda di conduzione (BC).
La distanza tra il minimo della banda di conduzione ($E_c$) e il massimo della banda di valenza ($E_v$) si chiama \important{energy gap} o banda proibita ($E_g = E_c - E_v$).

Un semiconduttore è un materiale cristallino di ``tipo isolante'' che ha basso energy gap e può essere drogato (cioè, le sue cariche possono essere modulate tramite la sostituzione di alcuni atomi con altri elementi).

\subsection{Approssimazione parabolica}

Ora, svolgiamo lo sviluppo in serie della funzione $E(k)$ per una certa banda:
\begin{equation}
    E(k) = E(0) + \evalat{\der[E]{k}}{k = 0} k + \frac{1}{2} \evalat{\der[E][2]{k}}{k = 0} k^2 + o(k^2), \quad k \to 0
\end{equation}

In un minimo la derivata prima è nulla, quindi se la banda ha un minimo in $k = 0$ e si sceglie lì il riferimento per l'energia, si ha
\begin{equation}
\label{eq:approssparabolica}
    \boxed{
    E(k) \approx \frac{1}{2} \evalat{\der[E][2]{k}}{0} k^2
    }
\end{equation}
Questa approssimazione è detta \important{approssimazione parabolica} ed è buona per ogni punto della banda tranne la zona in cui la curva flette, agli estremi.
Si può riscrivere come
\begin{equation}
    E(k) \approx p^2 \pts{\frac{1}{2 \hbar^2} \evalat{\der[E][2]{k}}{0}} \\
    \implies \frac{1}{2 m} \approx \frac{1}{2 m^*} \coloneq \frac{1}{2 \hbar^2} \evalat{\der[E][2]{k}}{0}
\end{equation}
$m^*$ si chiama \important{massa efficace} ed è una massa surrettizia definita in modo che la particella si possa descrivere come libera.
\begin{equation}
    m^* = \frac{\hbar^2}{\evalat{\der[E][2]{k}}{0}}
\end{equation}

Normalmente la massa efficace è dello stesso ordine di grandezza della massa dell'elettrone, ma può essere anche molto diversa, soprattutto con semiconduttori particolari.

\subsection{Densità degli stati}

Definiamo ora il numero di stati energetici disponibili in un internallo $\Delta E$ intorno all'energia $E$.
Al $\Delta E$ corrisponde un $\Delta k_x$ con $2 \cdot \Delta k_x / (2\pi / L)$ stati (il fattore due è dovuto ai due possibili valori di spin).
Per considerare tutte le direzioni, immaginiamo un cubetto nello spazio reciproco che ha per vertici otto valori di $\vt{k}$ possibili.

Il numero di stati $N(k)$ che hanno quantità di moto minore di $k = \norm{\vt{k}}$, ovvero in $[0, \, k]$, è il numero di tali cubetti in una sfera di raggio $k$:
\begin{equation}
    N(k) = 2 \cdot \frac{\frac{4}{3} \pi k^3}{\pts{\frac{2\pi}{L}}^3}
\end{equation}

A energia $E(k)$,
\begin{gather}
    E(k) = \frac{\hbar^2 k^2}{2 m^*}
    \implies k = \sqrt{\frac{2 m^* E}{\hbar^2}} \\
    \implies N(k) = 2 \cdot \frac{\frac{4}{3} \pi \pts{\frac{2 m^* E}{\hbar^2}}^{3/2}}{\pts{\frac{2\pi}{L}}^3}
    = \frac{L^3}{3 \pi^2} \pts{\frac{2 m^* E}{\hbar^2}}^{3/2} \\
    \implies g(E) \coloneq \der[N]{E} = \frac{L^3}{2 \pi^2} \pts{\frac{2 m^*}{\hbar}}^{3/2} \sqrt{E} \propto \sqrt{E}
\end{gather}
$g(E)$ è la densità degli stati elettronici all'energia $E$.

\redtext{Grafico $E$ vs. $g(E)$}

\section{Distribuzione di Fermi-Dirac}

Rimuoviamo ora l'ultima ipotesi, quella della temperatura allo zero assoluto.

Consideriamo l'ultima banda di un conduttore.

Suddividiamo l'intervallo di energia della banda in intervallini $\Delta E_i$ alle energie $E_i$.
Il numero di stati in ogni intervallino è $g_i \coloneq g(E_i) \Delta E_i$.

Se $T > \qty{0}{\kelvin}$, i nuclei iniziano a muoversi e urtano gli elettroni.
Allo zero assoluto gli elettroni erano in equilibrio termodinamico con i nuclei, pur avendo energia non nulla.
Adesso, invece, deve avvernire un trasferimento di energia dai nuclei agli elettroni.
Con l'energia ricevuta dagli urti, alcuni elettroni saltano oltre il livello di Fermi.

Ogni intervallino è caratterizzato da un energia per elettrone $E_i$, numero di stati $g_i$ e numero di elettroni $n_i$.
Usiamo gli unici tre vincoli fisici indiscutibili in questa situazione:
\begin{itemize}
    \item Principio di Pauli (quantistico)

    Il numero di elettroni in un intervallino di energia non può eccedere il numero di stati disponibili:
    \begin{equation}
        n_i \leq g_i
    \end{equation}
    \item Conservazione del numero di elettroni (classico)
    \begin{equation}
        N = \sum_i n_i \text{ costante}
    \end{equation}
    \item Conservazione dell'energia (classico)

    L'energia totale deve essere pari alla somma delle energie di ogni livello:
    \begin{equation}
        U = \sum_i E_i n_i \text{ costante}
    \end{equation}
\end{itemize}

Se non si raggiungesse un equilibrio dinamico, osserveremmo i materiali variare la loro disposizione di elettroni (e quindi anche la loro chimica e le loro proprietà) nel tempo, ma così non è.
Fermi, allora, postula l'esistenza di una distribuzione (oggi detta distribuzione di Fermi-Dirac) degli elettroni negli stati disponibili che si può realizzare in un numero grandissimo di combinazioni, in modo da trascurare le altre:
\begin{equation}
    f(E_i; T) = \frac{n_i}{g_i}
\end{equation}

Il ragionamento è analogo al caso classico della teoria cinetica dei gas: la distribuzione uniforme nello spazio è la più probabile;
non è impossibile che, ad esempio, metà contenitore sia vuoto, ma questo si realizza in un modo solo.
% La situazione che osserviamo sempre è quella immensamente più probabile.

Per ricavare la distribuzione, aggiungeremo un quarto vincolo non classico: l'indistinguibilità delle particelle.

Il numero di modi in cui posso disporre $n_i$ elettroni \important{indistinguibili} in $g_i$ stati è
\begin{equation}
    \binom{g_i}{n_i} = \frac{g_i!}{n_i! (g_i - n_i)!}
\end{equation}

Il numero totale di modi $W$ di disporre gli elettroni in tutti i livelli è una funzione delle variabili $\{n_i\}$:
\begin{equation}
    % W(\{n_i\}) = \prod_i \binom{g_i}{n_i}
    W(\ldots, n_i, \ldots) = \prod_i \binom{g_i}{n_i}
\end{equation}
Questa funzione va massimizzata sotto i vincoli dati prima.

Usiamo il primo principio della termodinamica e la definizione di variazione di entropia:
\begin{equation}
    \de U = \delta Q = T \de S
    \implies \de U - T \de S = \de (U - T S) = 0
\end{equation}
L'entropia è una misura del disordine di un sistema:
\begin{equation}
    S = k_B \ln W
\end{equation}
Definiamo l'\important{energia libera} $F$:
\begin{equation}
\begin{gathered}
    F \coloneq U - T S = \sum_i n_i E_i - k_B T \ln \prod_i \binom{g_i}{n_i} = \\
    = \sum_i n_i E_i - k_B T \sum_i \pts{\ln g_i! - \ln n_i! - \ln{(g_i - n_i)!}}
\end{gathered}
\end{equation}
L'energia libera deve essere costante:
\begin{equation}
    \de F = \sum_i \parder[F]{n_i} \delta n_i = 0
\end{equation}
$\delta n_i$, dal punto di vista fisico, rappresenta lo spostamento degli elettroni da uno stato all'altro, quindi la loro somma deve essere nulla (poiché il numero totale di elettroni è costante):
\begin{equation}
    \sum_i \delta n_i = 0
\end{equation}
Fermi intuì che, allora, la derivata di $F$ rispetto a ciascuna variabile $n_i$ deve essere una costante, il \important{potenziale chimico} $\mu$:
\begin{gather}
    \mu = \parder[F]{n_i} \quad \text{costante } \forall i \\
    \implies \de F = \sum_i \parder[F]{n_i} \delta n_i = \mu \sum \delta n_i = 0
\end{gather}


Tenendo presente l'equivalenza asintotica per $n \to+\infty$ nota come approssimazione di Stirling,
\begin{gather}
    n! \sim \sqrt{2 \pi n} \pts{\frac{n}{e}}^n \\
    \implies \ln n! \sim \frac{1}{2} \ln(2 \pi n) +  n \ln n - n \sim n \ln n
\end{gather}
imponiamo che la derivata di $F$ rispetto a $n_i$ sia $\mu$:
\begin{gather}
\begin{gathered}
    \mu = \parder[F]{n_i}
    = E_i + k_B T \parder{n_i} \pts{\ln n_i! + \ln{(g_i - n_i)!}} \approx \\
    \approx E_i + k_B T \parder{n_i} \pts{n_i \ln n_i + (g_i - n_i) \ln(g_i - n_i)} = \\
    = E_i + k_B T \ln \frac{n_i}{g_i - n_i}
\end{gathered} \\
    \implies n_i = \frac{g_i}{1 + \exp{\dfrac{E_i - \mu}{k_B T}}}
\end{gather}
Per cui la distribuzione di Fermi-Dirac cercata è
\begin{equation}
    f(E_i; T) = \frac{1}{1 + \exp{\dfrac{E_i - \mu}{k_B T}}}
\end{equation}
Tornando a valori di energia continui:
\begin{equation}
    \boxed{
    f(E; T) = \frac{1}{1 + \exp{\dfrac{E - \mu}{k_B T}}}
    }
\end{equation}
È una sigmoide decrescente centrata in $\mu$:

\redtext{Grafico $f(E; T)$ vs. $E$ per $E \ge 0$ e segnati $\mu$ in ascissa e $1/2$ e $1$ in ordinata}

Per $T \to 0$, la distribuzione diventa una funzione a gradino centrata in $\mu$.
Quindi, $\mu$ è proprio l'energia di Fermi.

Consideriamo l'ultima banda (quella semipiena) per un cristallo conduttore.
Possiamo ora determinare la densità di elettroni rispetto all'energia, moltiplicando la densità degli stati e la distrinuzione di Fermi-Dirac:

\redtext{Grafici $E$ vs. $g(E)$, $f(E)$ e $g(E) f(E)$}

La regione in cui $f(E)$ passa da $1$ a $0$ è ampia solo circa $2 k_B T$ attorno a $E_F$.
A $T = \qty{300}{\kelvin}$, questo valore corrisponde a circa \qty{0.05}{\electronvolt}.

\section{Calcolo del livello di Fermi}

Supponiamo di essere a $T = \qty{0}{\kelvin}$.

Detto $n_\text{el}$ il numero di elettroni nell'ultima banda per ogni atomo, il numero di elettroni totali nell'ultima banda è
\begin{gather}
    N = n_\text{el} N_\text{atomi} = \int_0^{E_F} g(E) \de E = \\
    = \frac{L^3}{2 \pi^2} \pts{\frac{2 m^*}{\hbar}}^{3/2} \int_0^{E_F} \sqrt{E} \de E
    = \frac{L^3}{2 \pi^2} \pts{\frac{2 m^* E_F}{\hbar}}^{3/2}
\end{gather}
Qui, lo zero dell'energia corrisponde all'inizio dell'ultima banda.
Si può quindi trovare il livello di Fermi:
\begin{equation}
    \boxed{
    E_F = \frac{\hbar^2}{2 m^*} \pts{3 \pi^2 n_\text{el} \frac{N_\text{atomi}}{L^3}}^{2/3}
    }
\end{equation}

% Nel caso del sodio, ad esempio, $n_\text{el} = 1$ e risulta $E_F = \qty{4.7}{\electronvolt}$.

Si osserva anche che, in questa scala e nel caso di un conduttore, la distribuzione di Fermi-Dirac è approssimabile con una funzione a gradino centrata in $E_F$.

\section{Conduttore in un campo elettrico}

Accendiamo un campo elettrico $\vt{C}$ nel verso $-\uy$.
La forza elettrica determina una variazione di quantità di moto di ogni elettrone:
\begin{equation}
\label{eq:campocristallo}
    e C = \frac{\Delta p_x}{\Delta t}
    \implies
    \frac{e C}{\hbar} \Delta t = \frac{\Delta p_x}{\hbar} = \Delta k_x
\end{equation}
Le quantità di moto aumentano tutte nel verso positivo di $k_x$.

Si sposta prima l'elettrone con $k_x$ maggiore liberando il suo stato, poi quello successivo, e così via.
Gli stati con $k_x$ più negativi vengono lasciati liberi.
Lo schema a bande, alla fine, è riempito in modo asimmetrico.
Gli elettroni che hanno stati appaiati con uno stato con $k_x$ opposto non contribuiscono a quantità di moto netta.
Gli elettroni spaiati, invece, danno origine a una corrente elettrica netta (in direzione opposta al ``moto'' degli elettroni, che hanno carica negativa).

Il $\Delta t$ nella \eqref{eq:campocristallo} è un tempo di urto $\tau$: il tempo medio prima di urtare contro un nucleo.
Dopo ogni urto, gli elettroni vengono accelerati nuovamente.
Per questa ragione, le quantità di moto variano ralativamente poco e la velocità di deriva è circa quella che corrisponde al livello di Fermi (interpretato come energia cinetica):
\begin{equation}
    v_F = \sqrt{\frac{2 E_F}{m}} \sim \qty{e5}{\metre\per\second}
\end{equation}
Da quest'ordine di grandezza, considerando la distanza tra i nuclei, risulta $\tau \sim \qty{e-12}{\second}$.

È anche possibile stimare il tempo di urto in modo non classico.
Infatti, la frequenza di urto $1/\tau$ è proporzionale alla densità dei nuclei.
I nuclei, che sono in vibrazione, corrispondono a onde elastiche.
Si introducono quindi delle particelle, dette \important{fononi}, che rappresentano le onde con cui gli elettroni interagiscono (cioè, contro cui ``urtano'').

L'energia media di oscillazione di un atomo in un solido è
\begin{equation}
    E_\text{osc} = \frac{3}{\exp{\frac{h \nu}{k_B T}} - 1} h \nu
\end{equation}
Se ho $N$ nuclei, il numero di fononi può essere considerato
\begin{equation}
    N_\text{fon} = \frac{3 N}{\exp{\frac{h \nu}{k_B T}} - 1}
\end{equation}
ognuno con energia $h \nu$.

$\tau$ si stima quindi come
\begin{equation}
    \tau \propto \frac{1}{N_\text{fon}}
\end{equation}


\backmatter

\end{document}

\end{document}
