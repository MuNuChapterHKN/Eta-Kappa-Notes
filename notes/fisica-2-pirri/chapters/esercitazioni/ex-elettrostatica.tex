\chapter{Elettrostatica -- esercitazioni}

\section{Ripasso sull'elettrostatica}

\begin{equation}
    \E(\p) = \frac{1}{4\pi\eps_0} \int_V \frac{\rho(\p')}{\norm{\p - \p'}^2} \ver{\p'} \, \de V'
\end{equation}
\begin{equation}
    \spot(\p) = \frac{1}{4\pi\eps_0} \int_V \frac{\rho(\p')}{\norm{\p - \p'}} \de V'
\end{equation}
La singola componente $\alpha \in \{x, y, z\}$ del campo elettrico è:
\begin{equation}
    E_\alpha(x,y,z) = \frac{1}{4\pi\eps_0} \int_V \frac{\rho(x',y',z')}{(x - x')^2 + (y - y')^2 + (z - z')^2} (\ver{\p'})_\alpha \, \de V'
\end{equation}

Conservatività:
\begin{gather}
    \E = - \grad \spot \\
    \E \lde = - \de V \iff \int_A^B \E \lde = \spot(\p_A) - \spot(\p_B) \\
    \curl \E = \vt{0}
    \iff
    \oint_\gamma \E \lde = 0
\end{gather}

Legge di Gauss ($S$ superficie chiusa):
\begin{equation}
    \diver \E = \frac{\rho}{\eps_0}
    \iff
    \Phi_S(\E) \coloneq \oint_S \E \sde = \frac{q}{\eps_0}
\end{equation}

\section{Campo elettrico dovuto a distribuzioni di carica}

\subsection{Anello}

\adddrawio{efield_ring}{0.9}

Anello $C$ di raggio $R$ e carica totale $q$ distribuita uniformemente.
Ha centro $\vt{0}$ e giace sul piano $yz$.
Trovare $\E(x, 0, 0)$ e $\spot(x, 0, 0)$.

Per simmetria, $\E(x, 0, 0) \parallel \ux$, cioè $\E(x, 0, 0) = E_x(x, 0, 0) \ux$.

Per trovare $E_x(x, 0, 0)$,
\begin{itemize}
    \item L'integrale è di linea e la densità lineare è $q / (2\pi R)$
    \item La componente $x$ di $\ver{\p'}$ è $\cos \theta = \frac{x}{\sqrt{x^2 + R^2}}$, con $\theta$ l'angolo tra l'asse $x$ e le rette tra i punti su $C$ e $(x, 0, 0)$.
    \item $x' = 0$
    \item $y = z = 0$
    \item $(y')^2 + (z')^2 = R^2$
\end{itemize}

\begin{equation}
\begin{aligned}
    E_x(x, 0, 0) & = \frac{1}{4\pi\eps_0} \int_C \frac{q/(2 \pi R)}{x^2 + R^2} \frac{x}{\sqrt{x^2 + R^2}} \de l' = \\
    & = \frac{qx}{8\pi^2\eps_0 R (x^2 + R^2)^{3/2}} \int_C \de l' = \\
    & = \frac{qx}{8\pi^2\eps_0 R (x^2 + R^2)^{3/2}} 2\pi R = \\
    & = \frac{q}{4\pi\eps_0} \frac{x}{(x^2 + R^2)^{3/2}}
\end{aligned}
\end{equation}

Analogamente, il potenziale risulta:
\begin{equation}
    \spot(x, 0, 0) = \frac{q}{4\pi\eps_0} \frac{1}{\sqrt{x^2 + R^2}}
\end{equation}

Si osserva $\E = - \grad \spot$.

Inoltre, per $x \gg R$,
\begin{gather}
    E_x(x, 0, 0) \approx \frac{q}{4\pi\eps_0 x^2} \\
    V(x, 0, 0) \approx \frac{q}{4\pi\eps_0 \abs{x}} \\
\end{gather}
ovvero, campo e potenziale dovuti a una carica puntiforme $q$.

\subsection{Disco}

Come prima, ma con carica di superficie su un disco $D$.

Ricaviamo prima il potenziale.

L'integrale si svolge in coordinate polari:
\begin{gather}
    \de S = r \, \de r \, \de \phi \\
\begin{aligned}
    \spot(x, 0, 0) & = \frac{1}{4\pi\eps_0} \int_D \frac{q / (\pi R^2)}{\sqrt{x^2 + (y')^2 + (z')^2}} \de S' = \\
    & = \frac{1}{4\pi\eps_0} \int_{\phi = 0}^{2\pi} \int_{r = 0}^R \frac{q / (\pi R^2)}{\sqrt{x^2 + r^2}} r \, \de r \, \de \phi = \\
    & = \frac{q}{4\pi^2\eps_0 R^2} \int_{0}^{2\pi} \de \phi \int_0^R \frac{r}{\sqrt{x^2 + r^2}} \, \de r = \\
    & = \frac{q}{2\pi\eps_0 R^2} \int_0^R \frac{r}{\sqrt{x^2 + r^2}} \, \de r
\end{aligned}
\end{gather}

L'integrale in $r$ è
\begin{equation}
    \int_0^R \frac{r}{\sqrt{x^2 + r^2}} \, \de r
    = \frac{1}{2} \int_0^R \frac{2r}{\sqrt{x^2 + r^2}} \, \de r
    = \frac{1}{2} \left[ 2 \sqrt{x^2 + r^2} \right]_{r=0}^R
    = \sqrt{x^2 + R^2} - \abs{x}
\end{equation}

Per cui,
\begin{equation}
    \spot(x, 0, 0) = \frac{q}{2\pi\eps_0} \frac{\sqrt{x^2 + R^2} - \abs{x}}{R^2}
\end{equation}
Poiché $\E(x, 0, 0) = E_x(x, 0, 0) \ux$,
\begin{equation}
    E_x(x, 0, 0) = - \parder[\spot]{x}(x, 0, 0) = \frac{q}{2\pi\eps_0 R^2} \pts{\sgn{x} - \frac{x}{\sqrt{x^2 + R^2}}}
\end{equation}
Per studiare $x \gg R$, qui serve andare al secondo ordine in $R / x$:
\begin{equation}
\begin{gathered}
    E_x(x, 0, 0) = \frac{q}{2\pi\eps_0 R^2} \pts{1 - \frac{x}{\abs{x}\sqrt{1 + \frac{R^2}{x^2}}}} \approx \\
    \approx \frac{q}{2\pi\eps_0 R^2} \pts{1 - \pts{1 - \frac{1}{2} \frac{R^2}{x^2}}}
    = \frac{q}{4\pi\eps_0 x^2}
\end{gathered}
\end{equation}

Per $\abs{x} \ll R$ o $R \to +\infty$,
\begin{equation}
    E_x(x, 0, 0) \approx \frac{q}{2\pi\eps_0 R^2} \sgn{x} = \frac{\sigma}{2\eps_0} \sgn{x}
\end{equation}
È il campo di un piano infinito con densità di carica superficiale $\sigma = q / (\pi R^2)$.

\subsection{Piano infinito}

Piano infinito che giace nel piano $yz$ e ha densità di carica superficiale $\sigma$.

Come mostrato in precedenza:
\begin{gather}
    \E(x, y, z) = \frac{\sigma}{2\eps_0} \sgn{x} \ux \\
    \spot(x, y, z) = -\frac{\sigma}{2\eps_0} \abs{x}
\end{gather}

\subsubsection{Acceleratore di carica}
\label{sec:acceleratore_carica}

Un acceleratore di carica permette di accelerare cariche nel vuoto e determinarne la velocità.

Svolgiamo il bilancio energetico per una particella di carica $-q$ inizialmente sull'armatura A (negativa) di un condensatore a facce piane parallele.
La differenza di potenziale tra le armeture è $\Delta V = V_B - V_A > 0$.
\begin{equation}
    -q V_A = \frac{1}{2} m v^2 - q V_B
    \implies K = \frac{1}{2} m v^2 = q \Delta V
    \implies v = \sqrt{\frac{2 q \Delta V}{m}}
\end{equation}
$v$ è la velocità della particella quando raggiunge o supera l'armatura B.

Per $\Delta V = \qty{10}{\volt}$, si avranno $K = \qty{10}{\electronvolt}$ e $v = \qty{1.9e6}{\metre\per\second} \approx \num{e-2} c_0$.

\subsubsection{Separatore elettrostatico}

\addfigure{book/separatore_elettrostatico}{0.8}

È possibile deviare una particella di carica $-q$ di massa $m$ dopo averla accelerata fino a una velocità iniziale $v_0$, facendola passare attraverso armature perpendicolari alle prime a una distanza $h$ dall'armatura positiva.
Il moto sarà parabolico.

Siano $a$ la lunghezza del separatore e $L$ una lunghezza successiva.
Ci si chiede la differenza di quota $d$.

Poniamo l'armatura negativa in basso.
Il campo e la forza sono $\E = -E \uy$ e $\force = q E \uy$.

Con l'origine all'inizio del separatore, la traiettoria è
\begin{gather}
    y(x) = \frac{1}{2} \frac{q E}{m} t^2 = \frac{q E x^2}{2 m v_0^2} \\
    \tan \alpha = \evalat{\der[y]{x}}{x = a} = \frac{q E a}{m v_0^2}, \qquad d = L \tan \alpha + h
\end{gather}

\subsection{Sfera}

Si consideri una sfera di carica $q$, raggio $R$ e centro nell'origine.

Per simmetria, il campo è radiale: $\E(\p) = E(r)\ver{r}$

Consideriamo una superficie sferica $\Sigma$ di raggio $r$.
Allora,
\begin{equation}
\begin{gathered}
    \Phi_\Sigma(\E) = \oint_\Sigma \E \sde = \oint_\Sigma E(r) \de S= 4\pi r^2 E(r) = \frac{q_r}{\eps_0} \\
    \implies E(r) = \frac{q_r}{4\pi\eps_0 r^2}
\end{gathered}
\end{equation}
dove $q_r$ è la carica all'interno di $\Sigma$.
Quindi,
\begin{equation}
    E(r) = \begin{cases}
        0 & \text{se } r < R \\
        \dfrac{q}{4\pi\eps_0 r^2} & \text{se } r > R
    \end{cases}
\end{equation}
Ovvero, se si è all'esterno, è come se tutta la carica fosse nel centro della sfera.

Il salto attraverso la superficie è $[E(r)]_R = \sigma/\eps_0$, quello di un piano infinito con campo nullo da una delle due parti.

Il potenziale è costante all'interno della sfera (poiché il campo è nullo) e, per determinare \textit{quale} costante, lo si impone continuo attraverso $r = R$:
\begin{equation}
\label{eq:potenziale_sfera}
    V(r) = \begin{cases}
        \frac{q}{4\pi\eps_0 R} & \text{se } r \le R \\
        \frac{q}{4\pi\eps_0 r} & \text{se } r > R
    \end{cases}
\end{equation}


\subsection{Palla}

Palla omogenea di carica $q$, raggio $R$ e centro nell'origine.

All'esterno, è tutto identico al caso della sfera carica.

Ora, se $r < R$, $q_r$ non è più nulla, ma
\begin{equation}
    q_r = q \frac{r^3}{R^3}
\end{equation}
Per cui
\begin{equation}
    E(r) = \frac{q_r}{4\pi\eps_0 r^2}
    = \frac{q}{4\pi\eps_0 R^3} r
    = \frac{\rho}{3\eps_0} r
\end{equation}
dove $\rho = q / (\frac{4}{3} \pi R^3)$ è la densità di carica.

Calcoliamo il potenziale all'interno:
\begin{equation}
\begin{gathered}
    \spot(r) - \spot(R) = \int_r^R \E \lde = \int_r^R E(r') \, dr' = \frac{\rho}{6\eps_0} \pts{R^2 - r^2} \\
    \implies \spot(r) = \frac{\rho}{6\eps_0} \pts{R^2 - r^2} + \frac{q}{4\pi\eps_0 R} = \frac{q}{8\pi\eps_0 R} \pts{3 - \frac{r^2}{R^2}}
\end{gathered}
\end{equation}


\subsection{Cilindro}

\adddrawio[Il cilindro di raggio $R$ è da intendersi infinito.]{efield_cylinder}{0.4}

Cilindro infinito di raggio $R$ con asse coincidente con l'asse $z$ e densità di carica $\rho$.

Poiché il cilindro è infinito, il campo non ha componente $z$.

Sia $\lambda = \pi R^2 \rho$ la densità lineare di carica.
Si considera il flusso attraverso un cilindro (limitato) $\Sigma$ di raggio $r$ e altezza $h$, in modo che il flusso sia nullo sulle basi:
\begin{equation}
    q_r = h \lambda, \quad \text{se } r > R
\end{equation}
\begin{equation}
\begin{gathered}
    \Phi_\Sigma(\E) = \oint_\Sigma \E \sde = \oint_\Sigma E(r) \de S = 2\pi r h E(r) = \frac{q_r}{\eps_0} \\
    \implies E(r) = \frac{\lambda}{2\pi\eps_0 r}
\end{gathered}
\end{equation}

Se $r < R$,
\begin{equation}
    E(r) = \frac{\lambda}{2\pi\eps_0 R^2} r
\end{equation}

Il potenziale all'interno è
\begin{equation}
    V(r) = \frac{\lambda}{4\pi\eps_0 R^2} \pts{R^2 - r^2}
\end{equation}

All'esterno non si può fissare nullo il potenziale in $r = +\infty$.
È comunque possibile quantificare le differenze di potenziale:
\begin{equation}
    \spot(r_A) - \spot(r_B) = \int_A^B \E \lde = \frac{\lambda}{2\pi\eps_0} \ln\frac{r_B}{r_A}
\end{equation}

In un certo senso vale:
\begin{equation}
    V(r) = \begin{cases}
        \frac{\lambda}{4\pi\eps_0 R^2} \pts{R^2 - r^2} & r \le R \\
        - \frac{\lambda}{2\pi\eps_0} \ln\frac{r}{R} & r > R
    \end{cases}
\end{equation}

\section{Coordinate sferiche}

\addfigure[Da \shorturl{commons.wikimedia.org/wiki/File:3D\_Spherical.svg}.]{wikimedia/spherical_coordinates.svg.png}{0.4}

\begin{subequations}
\begin{gather}
    r \in \left[0, +\infty\right), \quad \theta \in \left[0, \pi\right], \quad \phi \in \left[0, 2\pi\right) \\
    \begin{cases}
        x = r \sin \theta \cos \phi \\
        y = r \sin \theta \sin \phi \\
        z = r \cos \theta
    \end{cases} \\
    \begin{cases}
        r = \sqrt{x^2 + y^2 + z^2} \\
        \theta = \arccos\pts{z/\!\sqrt{x^2 + y^2 + z^2}} \\
        \tan \phi = y/x
    \end{cases}
\end{gather}
\end{subequations}

Matrice jacobiana:
\begin{equation}
    J = \begin{bmatrix}
        \parder[x]{r} & \parder[x]{\theta} & \parder[x]{\phi} \\
        \parder[y]{r} & \parder[y]{\theta} & \parder[y]{\phi} \\
        \parder[z]{r} & \parder[z]{\theta} & \parder[z]{\phi}
    \end{bmatrix}\!, \qquad
    \abs{\det J} = r^2 \sin \theta
\end{equation}

Differenziale di volume:
\begin{equation}
    \de V = \de x \, \de y \, \de z
    = \abs{\det J} \, \de r \, \de \theta \, \de \phi
    = r^2 \sin \theta \, \de r \, \de \theta \, \de \phi
\end{equation}

Gradiente:
\begin{gather}
    \grad f(r, \theta, \phi) = \parder[f(r, \theta, \phi)]{x} \ux + \parder[f(r, \theta, \phi)]{y} \uy + \parder[f(r, \theta, \phi)]{z} \uz \\
    \parder[f(r, \theta, \phi)]{x} =
    \parder[r]{x} \, \parder[f]{r} +
    \parder[\theta]{x} \, \parder[f]{\theta} +
    \parder[\phi]{x} \, \parder[f]{\phi}
\end{gather}
Risulta:
\begin{equation}
    \grad f(r, \theta, \phi) = \parder[f]{r} \ver{r} + \frac{1}{r}\parder[f]{\theta} \ver{\theta} + \frac{1}{r \sin \theta} \parder[f]{\phi} \ver{\phi}
\end{equation}




\section{Dipolo elettrico}

\subsection{Definizione}

Momento di dipolo elettrico per due cariche $-q$ e $q$ con $q > 0$ in posizioni $\p^-$ e $\p^+$ poste a distanza $a$ fissa:
\begin{equation}
    \vt{p} \coloneq q \vt{a} = q (\p^+ - \p^-) = q a \ver{- \to +}
\end{equation}

Il momento di dipolo $\vt{P}$ di un oggetto carico, intuitivamente, va dal ``centro di massa" delle cariche negative a quello delle positive.

\begin{equation}
    \vt{P} = \int_V \rho(\p) \, \p \, \de V
\end{equation}


\subsection{Potenziale e campo}

Potenziale dovuto a un dipolo elettrico con $\vt{p} \parallel \uz$.
Siano $r_1 = \norm{\p - \p^+}$, $r_2 = \norm{\p - \p^-}$:
\begin{equation}
    \spot(\p) = \frac{q}{4\pi\eps_0} \pts{\frac{1}{r_1} - \frac{1}{r_2}}
    = \frac{q}{4\pi\eps_0} \frac{r_2 - r_1}{r_1 r_2}
\end{equation}


Considerando $r \gg a$, otteniamo l'approssimazione al primo ordine in $a/r$ per $r_1 r_2$ e $r_2 - r_1$.
\begin{subequations}
\begin{gather}
    r_1 = \norm{\p - \frac{1}{2}\vt{a}} = \sqrt{r^2 + \frac{1}{4} a^2 - r a \cos \theta} \approx r \sqrt{1 - \frac{a}{r}\cos\theta} \\
    r_2 = \norm{\p + \frac{1}{2}\vt{a}} = \sqrt{r^2 + \frac{1}{4} a^2 + r a \cos \theta} \approx r \sqrt{1 + \frac{a}{r}\cos\theta}
\end{gather}
\end{subequations}

Quindi
\begin{subequations}
\begin{gather}
    r_1 r_2 \approx r^2 \sqrt{1 - \frac{a^2}{r^2}\cos^2 \theta} \approx r^2 \\
\begin{gathered}
    r_2 - r_1 \approx r \pts{\sqrt{1 + \frac{a}{r}\cos\theta} - \sqrt{1 - \frac{a}{r}\cos\theta}} \approx \\
    \approx r \pts{1 + \frac{1}{2} \frac{a}{r} \cos \theta - 1 + \frac{1}{2} \frac{a}{r} \cos \theta}
    = a \cos \theta
\end{gathered}
\end{gather}
\end{subequations}

Dunque, per $r \gg a$ (\important{dipolo ideale}),
\begin{gather}
    \spot(\p) = \frac{q}{4\pi \eps_0} \frac{r_2 - r_1}{r_1 r_2}
    \approx \frac{q}{4\pi \eps_0} \frac{a \cos \theta}{r^2}
    = \frac{1}{4\pi \eps_0} \frac{\vt{p} \cdot \ver{r}}{r^2} \\
\begin{gathered}
\label{eq:campo_dipolo_elettrico}
    \E(\p) = - \grad \spot(\p)
    = -\parder[\spot]{r} \ver{r} - \frac{1}{r}\parder[\spot]{\theta} \ver{\theta} - \cancel{\frac{1}{r \sin \theta} \parder[\spot]{\phi} \ver{\phi}} = \\
    = \frac{q a \cos\theta}{2\pi \eps_0 r^3} \ver{r} + \frac{q a \sin\theta}{4\pi \eps_0 r^3} \ver{\theta}
    = \frac{p}{4\pi \eps_0 r^3} \pts{2 \cos\theta \, \ver{r} + \sin\theta \, \ver{\theta}}
\end{gathered}
\end{gather}

$\parder[\spot]{\phi}$ è nulla per simmetria cilindrica.

Si osserva che:
\begin{itemize}
    \item Sull'asse $z$, $\E(\p) \parallel \uz$ con lo stesso verso (in quanto $\theta = 0$ e $\ver{r} = \uz$).
    \item Sul piano $xy$, $\E(\p) \parallel \uz$ con verso opposto (in quanto $\theta = \pi/2$ e $\ver{\theta} = -\uz$).
\end{itemize}

Se si considera il campo esatto, senza l'approssimazione, si parla di \important{dipolo fisico}.
Con il dipolo ideale non è possibile descrivere il campo tra le due cariche.


\subsection{Potenziale a grandi distanze}

Calcoliamo il potenziale a grande distanza da un corpo carico caratterizzato da una densità di carica $\rho$ in un colume $V$.

Usiamo la seguente espressione per la distanza tra due punti $\p$ e $\p'$:
\begin{equation}
    \norm{\p - \p'} = \sqrt{\norm{\p - \p'}^2} = \sqrt{r^2 + (r')^2 - 2 \p \cdot \p'} = r \pts{1 + \frac{(r')^2}{r^2} - 2 \frac{\p \cdot \p'}{r^2}}^{\!1/2}
\end{equation}

Se il corpo è vicino all'origine, possiamo usare l'approssimazione $r \gg r'$.
Risulta
\begin{equation}
\begin{gathered}
    \spot(\p)
    % = \frac{1}{\eps_0} \slp \rho(\p)
    = \frac{1}{4\pi\eps_0} \int_V \frac{\rho(\p')}{r} \pts{1 + \frac{(r')^2}{r^2} - 2 \frac{\p \cdot \p'}{r^2}}^{\!-1/2} \de V' \approx \\
    \approx \frac{1}{4\pi\eps_0} \int_V \frac{\rho(\p')}{r} \pts{1 + \frac{\p \cdot \p'}{r^2}} \de V' = \\
    = \frac{1}{4\pi\eps_0} \int_V \frac{\rho(\p')}{r} \de V' + \frac{1}{4\pi\eps_0} \int_V \frac{\rho(\p') \p \cdot \p'}{r^3} \de V' = \\
    = \frac{1}{4\pi\eps_0 r} \int_V \rho(\p') \, \de V' + \frac{1}{4\pi\eps_0 r^2} \frac{\p}{r} \cdot \int_V \rho(\p') \, \p' \, \de V'= \\
    = \frac{Q}{4\pi\eps_0 r} + \frac{\vt{P} \cdot \ver{r}}{4\pi\eps_0 r^2}
\end{gathered}
\end{equation}

Il momento di dipolo $\vt{P}$ è il responsabile del potenziale elettrostatico a grande distanza per oggetti complessivamente neutri.
Ogni oggetto neutro, a grande distanza, si può quindi rappresentare come un dipolo.


\subsection{Dipolo in campo elettrico}

Considerando un dipolo in un campo $\E$ uniforme, la forza totale $\force$ sul dipolo è nulla e il momento della coppia di forze sulle cariche è
\begin{equation}
    \vt{M} = (\vt{r}^+ - \vt{r}^-) \times \force
    = \vt{a} \times q \E
    = \vt{p} \times \E
\end{equation}

Lavoro ed energia potenziale relativi alla rotazione dall'angolo $\theta_0$ all'angolo $\theta$ rispetto alla direzione del campo elettrico:
\begin{equation}
    W = \int_{\theta_0}^\theta \norm{\vt{M}(\theta')} \de \theta'
    = \int_{\theta_0}^\theta p E \sin\theta' \de \theta'
    = - pE \pts{\cos\theta - \cos\theta_0}
\end{equation}
\begin{equation}
    U_e(\theta) = - \vt{p} \cdot \E = - p E \cos\theta
\end{equation}
La configurazione più stabile è quella con $\vt{p} \parallel \E$ con verso concorde.

Se il campo elettrico non è uniforme, $\vt{E}(\p^+) = \vt{E}(\p^-) + (J \E) \vt{a}$, dove $J \E$ è la matrice jacobiana del campo elettrico valutata in $\p^+ \approx \p^-$, che agisce tramite prodotto matrice-vettore.
\begin{gather}
    \force = q \vt{E}(\p^+) - q \vt{E}(\p^-) = q (J \E) \vt{a} = (J \E) \vt{p} \iff \\
    \iff \begin{cases}
        F_x = \vt{p} \cdot \grad E_x = p_x \parder[E_x]{x} + p_y \parder[E_x]{y} + p_z \parder[E_x]{z} \\
        F_y = \vt{p} \cdot \grad E_y = p_x \parder[E_y]{x} + p_y \parder[E_y]{y} + p_z \parder[E_y]{z} \\
        F_z = \vt{p} \cdot \grad E_z = p_x \parder[E_z]{x} + p_y \parder[E_z]{y} + p_z \parder[E_z]{z}
    \end{cases}
\end{gather}
coerentemente con $\force = - \grad U_e$.


\section{Condensatori}

\subsection{Condensatore sferico}

\adddrawio{spherical_capacitor}{0.4}

Ricordando l'\cref{eq:potenziale_sfera}, la capacità di una sfera carica di raggio $R$ è
\begin{equation}
    C = \frac{q}{V} = 4\pi\eps_0 R
\end{equation}
Se $R = \qty{6e6}{\metre}$ (il raggio della Terra), allora $C = \qty{6e-4}{\farad}$.

Sembrerebbe difficile raggiungere alte capacità con piccoli dispositivi, eppure esistono in commercio dispositivi con capacità \qty{e3}{\farad}.

Consideriamo una sfera e guscio sferico concentrici con raggi $R_1 < R_2 < R_3$.
Caricando con $+q$ la sfera interna, le cariche sulle superfici saranno $+q$, $-q$ e $+q$ (nulle all'interno dei conduttori).
Il campo tra le armature è radiale e dipende solo dalla sfera interna.
\begin{equation}
    \E(\p) = \frac{q}{4\pi\eps_0 r^2} \ver{r}
\end{equation}
La differenza di potenziale tra le armature è
\begin{equation}
    V_1 - V_2 = \frac{q}{4\pi\eps_0} \pts{\frac{1}{R_1} - \frac{1}{R_2}}
\end{equation}
Per cui,
\begin{equation}
    C = 4\pi\eps_0 \frac{R_1 R_2}{R_2 - R_1}
\end{equation}
\begin{itemize}
    \item Per $R_2 \to +\infty$, fisicamente si ha il caso della sfera isolata e $C = 4\pi\eps_0 R_1$.
    \item Per $R_2 - R_1 \eqcolon h \ll R_1$, si ha $R_1 \approx R_2 \approxcolon R$ e
    \begin{equation}
        C = 4\pi\eps_0 \frac{R^2}{h} = \eps_0 \frac{A}{h} \to +\infty
    \end{equation}
    e si ha la stessa capacità che nel caso del condensatore a facce piane parallele.
\end{itemize}

Per ottenere dispositivi con enorme capacità, si creano nanostrutture che rendono l'area effettiva dell'ordine dei \unit{\kilo\metre\squared} per ogni \unit{\centi\metre\squared} di area ``apparente''.

``Condensatore'' = condensa energia elettrostatica in un volume.


\subsection{Condensatore cilindrico}

Per considerazioni analoghe, il campo tra le armature è
\begin{gather}
    E(r) = \frac{\lambda}{2\pi\eps_0 r} \\
    V_1 - V_2 = \frac{\lambda}{2\pi\eps_0} \ln\frac{R_2}{R_1} \\
    C = \frac{2\pi\eps_0 d}{\ln(R_2/R_1)}
\end{gather}
$d$ è la lunghezza del condensatore, per cui $q = \lambda d$.

Per $R_2 - R_1 \eqcolon h \ll R_1$, $R_1 \approx R_2 \approxcolon R$ e
\begin{equation}
    C = \frac{2\pi\eps_0 d}{\ln(1 + h/R)} = \frac{2\pi\eps_0 d R}{h}
    = \eps_0 \frac{A}{h}
\end{equation}
di nuovo come un condensatore a facce piane parallele.

\subsection{Energia elettrostatica}

\begin{equation}
    U_e = \frac{1}{2} C V^2 = \frac{1}{2} Q V = \frac{1}{2} \frac{Q^2}{C}
\end{equation}

\begin{equation}
    U_e = \int_V w_E \, \de V = \int_V \frac{1}{2}\eps_0 E^2 \de V
\end{equation}
Verifichiamo l'uguaglianza per un condensatore sferico:
\begin{equation}
\begin{gathered}
    U_e = \int_V \frac{1}{2}\eps_0 \frac{q^2}{16\pi^2\eps_0^2 r^4} \de V
    = \frac{q^2}{32\pi^2\eps_0} \int_V \frac{1}{r^4} \de V = \\
    = \frac{q^2}{32\pi^2\eps_0} 4\pi \int_{R_1}^{R_2} \frac{1}{r^2} \de r
    = \frac{q^2}{8\pi\eps_0} \pts{\frac{1}{R_1} - \frac{1}{R_2}}
    = \frac{1}{2} q V
\end{gathered}
\end{equation}
usando $\de V = 4\pi r^2 \de r$.

\section{Materiale dielettrico}

\addfigure[Da \href{https://commons.wikimedia.org/wiki/File:Dielettrico.png}{Papa November Brendy}, \href{https://creativecommons.org/licenses/by-sa/3.0}{CC BY-SA 3.0}, tramite Wikimedia Commons.]{wikimedia/dielettrico}{0.35}

Finora, abbiamo trattato l'elettrostatica nel vuoto o per i conduttori.
Ora trattiamo i dielettrici: materiali isolanti (cioè che non perettono il moto di cariche libere) che modificano alcune proprietà elettrostatiche.

Consideriamo due condensatori a facce piane parallele identici.
Se tra le armature del secondo è inserito un dielettrico, le capacità non saranno uguali:
\begin{equation}
    \frac{C}{C_0} = \eps_r > 1
\end{equation}
$\eps_r$ è la \important{costante dielettrica relativa} e dipende dal materiale.
Inserire un dielettrico, quindi, permette di ``condensare'' meglio l'energia.

\important{Rigidità dielettrica} $E_\mathrm{max}$: campo al quale avviene la \important{rottura del dielettrico}, cioè per cui il dielettrico è costretto a condurre corrente (e si brucia).

Nel mezzo (= non nel vuoto), è sufficiente sostituire ovunque la costante dielettrica nel vuoto $\eps_0$ con la costante dielettrica $\eps = \eps_r \eps_0$.

A parità di carica $Q$, la differenza di potenziale tra le armature è minore che nel vuoto:
\begin{equation}
    V = \frac{Q}{C} = \frac{Q}{\eps_r C_0} = \frac{V_0}{\eps_r} < V_0
\end{equation}

A parità di potenziale $V$ la carica accumulata è maggiore:
\begin{equation}
    Q = CV = \eps_r C_0 V = \eps_r Q_0 > Q_0
\end{equation}
Ovvero, per ottenere lo stesso potenziale occorre spostare più carica.

Sempre a parità di potenziale, l'energia accumulata è maggiore, poiché si accumula più carica elettrica:
\begin{equation}
    U = \eps_r U_0 > U_0
\end{equation}

Un dielettrico è neutro e non conduttore, ma le sue particelle possono essere dipoli.
Classifichiamo i dielettrici in
\begin{itemize}
    \item dielettrici polari, con momento di dipolo intrinseco (ad esempio, sostanze con molecole polari come \ce{H2O});
    \item dielettrici apolari (ad esempio, sostanze con molecole apolari come \ce{CO2}).
\end{itemize}

Applicando un campo $\E_0$ a un dielettrico, i dipoli tendono a orientarsi secondo il campo.
Il campo elettrico totale sarà la somma di $\E_0$ e del campo generato dai dipoli, che ha verso opposto: $\E = \E_0 + \E'$, $E = E_0 - E' < E_0$.

Nel caso di dielettrici apolari, il campo $\E_0$ può generare dei dipoli indotti (e orientarli).

\important{Polarizzazione}:
\begin{equation}
    \vt{P}(\p) = n(\p) \vt{p}(\p)
\end{equation}
$n$ è la densità volumica dei dipoli.

I dipoli, sulle superfici dei dielettrici, espongono delle cariche ``spaiate'' e non mobili.
Quindi, si può scrivere il momento di dipolo totale
\begin{equation}
    \int_V \vt{P} \de V = \vt{P} S l
\end{equation}
dove $S$ è la superficie delle armature e $l$ è la distanza reciproca.
$\vt{P} S$, dimensionalmente, è una carica: quella che si vede sulla superficie del dielettrico.

Quindi il momento di dipolo totale è anche $Ql$.

Se le due armature hanno tra loro un angolo $\theta$ ma sono sufficientemente lontane, si può considerare ancora $\vt{P}$ uniforme e
\begin{equation}
    \int_V \vt{P} \de V = \vt{P} S l \cos \theta
\end{equation}

In generale, la densità di carica del dielettrico sulla superficie è
\begin{equation}
    \sigma\bound = \vt{P} \cdot \uvt{n}
\end{equation}
dove $\uvt{n}$ è la normale uscente dal dielettrico.

La densità totale sulla superficie delle armature, quindi, è $\sigma = \sigma\free - \sigma\bound$ ($\sigma\free$ è la carica libera).

Il campo elettrico nel dielettrico è
\begin{equation}
    E = \frac{\sigma}{\eps_0} = \frac{\sigma\free - \sigma\bound}{\eps_0}
\end{equation}
Ricordando $\sigma\free = \eps_0 E_0$, definiamo il campo di spostamento $\D$:
\begin{equation}
    \D \coloneq \eps_0 \E_0 = \eps_0 \E + \vt{P}
\end{equation}

Valgono le seguenti relazioni:
\begin{subequations}
\begin{alignat}{2}
    \sigma\bound & = \vt{P} \cdot \uvt{n}
    & \qquad & \text{$\uvt{n}$ uscente dal dielettrico} \\
    \sigma\free & = \pts{\D_2 - \D_1} \cdot \uvt{n}
    & \qquad & \text{$\uvt{n}$ dal mezzo 1 al mezzo 2}
\end{alignat}
\end{subequations}

La polarizzazione $\vt{P}$ è legata alle proprietà del dielettrico:
\begin{equation}
    \vt{P} = \eps_0 \chi_e \E
\end{equation}
$\chi_e$ è la \important{suscettività elettrica} e misura la risposta del dielettrico.
È adimensionata.

\begin{equation}
    \D = \eps_0\E + \eps_0 \chi_e \E = \eps_0 \underbrace{(1 + \chi_e)}_{\eps_r} \E = \eps \E
\end{equation}
Poiché $\eps_r \ge 1$, allora $\chi_e \ge 0$.

Flussi
\begin{subequations}
\begin{gather}
    \eps_0 \oint_S \E \sde = q = q\free + q\bound \\
    \oint_S \D \sde = \eps_0 \oint_S \E_0 \sde = q\free \\
    \oint_S \vt{P} \sde = -q\bound
\end{gather}
\end{subequations}
Ma anche
\begin{gather}
    \eps \oint_S \E \sde = q\free \\
    \oint_S \vt{P} \sde = \eps_0 \chi_e \oint \E \sde = \chi_e q
\end{gather}
Formati differenziali:
\begin{subequations}
\begin{gather}
    \diver \E = \frac{\rho}{\eps_0} = \frac{\rho\free}{\eps} \\
    \diver \D = \rho\free \\
    \diver \vt{P} = -\rho\bound = \chi_e \rho
\end{gather}
\end{subequations}

\subsection{Legge di Curie}

La suscettività dipende dalla temperatura secondo la legge di Curie:
\begin{equation}
    \chi_e = A + \frac{B}{T}
\end{equation}
per opportuni coefficienti $A$ e $B$.
Questo significa che scaldando il mezzo, la risposta dielettrica diminuisce.
Infatti, più difficilmente i dipoli del mezzo riescono a mantenersi orientati.

Considerando un singolo dipolo, la forza elastica di richiamo eguaglia la forza elettrica, che tende a separare le cariche
\begin{equation}
    q E = k x = \omega_0^2 m x
\end{equation}
La risposta è oscillatoria e determina un momento di dipolo di modulo
\begin{equation}
    p = q x = \frac{q^2 E}{\omega_0^2 m}
\end{equation}
Considerando una densità volumica $n$ di dipoli, la polarizzazione sarà
\begin{equation}
    n p = \frac{n q^2 E}{\omega_0^2 m} = \eps_0 \frac{n q^2}{\eps_0 \omega_0^2 m} E = \eps_0 \chi_{e,\text{apolare}} E
\end{equation}

Il coefficiente $A$ è quindi legato alla componente apolare del mezzo.
La componente polare della suscettività elettrica, invece, è quella che dipende dalla temperatura.

\subsection{Esempio di calcolo}

Considerando una sfera carica immersa in un dielettrico, questo schermerà le cariche sulla superficie della sfera orientandovi le cariche negative dei dipoli.
Il campo elettrico all'esterno sarà minore di un fattore $\eps_r$ e
\begin{gather}
    \vt{P} = \frac{\eps_r - 1}{\eps_r} \frac{q}{4\pi r^2} \ver{r} \\
    \implies \sigma\bound = \vt{P} \cdot \uvt{n}
    = -\frac{\eps_r - 1}{\eps_r} \frac{q}{4\pi r^2}
\end{gather}
Nota che $\uvt{n} = - \ver{r}$, cioè entrante, poiché la superficie è dal punto di vista del dielettrico.

Moltiplicando per $4\pi r^2$ si ottiene la carica affacciata sul dielettrico:
\begin{equation}
    q\bound = \sigma\bound 4\pi r^2 = - \frac{\eps_r - 1}{\eps_r} q
\end{equation}
