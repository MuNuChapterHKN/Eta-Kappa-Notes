\chapter[Campi variabili nel tempo -- es.]{Campi variabili nel tempo -- esercitazioni}

\section{Autoinduzione}

\begin{subequations}
\begin{gather}
    \oint_\gamma \E \lde = - \der{t} \Phi(\B) \\
    \curl \E = - \parder[\B]{t}
\end{gather}
\end{subequations}

Nei circuiti:
\begin{equation}
    \fem_i = - \der{t} \Phi(\B) = - L \der[I]{t}
\end{equation}
% se vale l'approssimazione a parametri concentrati: $\Delta t \gg d/c$.

Secondo la forza di Lorentz, vale anche
\begin{equation}
    \fem_i = \int_\gamma \E_i \lde
    = \int_\gamma \pts{\vt{v} \times \B} \lde
\end{equation}

Esempio:

Bobina di $N = 100$ spire di raggio $A = \qty{10}{\centi\metre}$ e resistenza $R = \qty{1}{\ohm}$.
Campo magnetico $\B(t) = \alpha t \uvt{n}$, con $\alpha = \qty{e-3}{\tesla\per\second}$.
Calcolare la corrente $I$ indotta nel circuito.

\begin{equation}
\begin{gathered}
    I = \frac{\fem}{R}
    = - \frac{1}{R} \der{t} \Phi(\B)
    = - \frac{1}{R} \der{t} \int_S \B \sde
    = - \frac{1}{R} \der{t}\pts{B \int_S \de S} = \\
    = - \frac{1}{R} \der{t}\pts{\alpha t N \pi A^2}
    = - \frac{\alpha N \pi A^2}{R}
    = -\qty{3.14}{\milli\ampere}
\end{gathered}
\end{equation}
misurata in verso antiorario se $\uvt{n}$ è uscente.

La corrente indotta genera un campo magnetico che contrasta la variazione di flusso.

Quanta carica scorre nella spira in $\Delta t = \qty{100}{\second}$?
\begin{equation}
    q = \int_{0}^{\Delta t} I(t) \de t = I \Delta t = \qty{3.14e-1}{\coulomb}
\end{equation}

La potenza dissipata dalla spira è
\begin{equation}
    P = \fem_i I = \qty{9e-6}{\watt}
\end{equation}


\subsection{Generatore di corrente alternata}

\adddrawio[][0.6]{ac_generator}

Campo magnetico uniforme $\B$ di modulo $B$ (generato, ad esempio, da un magnete) e spira di area $A$ la cui normale forma un angolo $\theta$ con $\B$.
La spira è messa in moto di rotazione uniforme con legge oraria $\theta(t) = \omega t$.

\begin{equation}
    \fem_i = - \der{t} \Phi(\B)
    = - \der{t} \pts{A B \cos(\omega t)}
    = A B \omega \sin(\omega t)
\end{equation}

Alternativamente, si può usare la forza di Lorentz:
\begin{equation}
    \fem_i = \int_\gamma \pts{\vt{v} \times \B} \lde
\end{equation}
Se la spira è rettangolare, i lati (lunghi $a$) perpendicolari all'asse di rotazione danno ciascuno contributo nullo, poiché $\vt{v}$ varia da un certo valore in corrispondenza di un estremo al suo opposto presso l'altro estremo del lato.
I lati (lunghi $b$) paralleli all'asse di rotazione hanno valore
\begin{equation}
    \int_A^B (\vt{v} \times \B) \lde
    = \int_A^B \pts{\frac{a}{2}\omega} B \sin(\omega t) \ver{AB} \lde
    = \frac{1}{2} abB \omega \sin\omega
\end{equation}
Analogamente,
\begin{equation}
    \int_C^D (\vt{v} \times \B) \lde
    = \frac{1}{2} abB \omega \sin\omega
\end{equation}
Quindi, $\fem_i = abB \omega \sin\omega$.

Si ottengono una tensione alternata e una corrente alternata:
\begin{subequations}
\begin{gather}
    V(t) = A B \omega \sin(\omega t) = V_0 \sin(\omega t) \\
    I(t) = \frac{\fem_i}{R} = \frac{AB \omega}{R} \sin(\omega t) = I_0 \sin(\omega t)
\end{gather}
\end{subequations}

Ruotando la spira, la forza di Lorentz (proporzionale alla velocità) fa sì che si senta una forza di attrito.
Bisogna fornire un lavoro meccanico con continuità, che sia convertito in energia elettrica.

Esempio:

$N = 20$ spire circolari con raggio $r = \qty{20}{\centi\metre}$, frequenza $f = \qty{50}{\hertz}$ e campo magnetico $B = \qty{0.4}{\tesla}$.

\begin{equation}
    \fem_i = N \pi r^2 \cdot B \cdot 2\pi f \cdot \sin(2\pi f t)
    = (\qty{316}{\volt}) \sin(\qty{314}{\per\second} \cdot t)
\end{equation}

Il \important{valore efficace} di una grandezza $X(t)$ periodica nel tempo di periodo $T$ è
\begin{equation}
    X\eff = \sqrt{\langle X(t)^2 \rangle_T}
    = \sqrt{\frac{1}{T} \int_0^T X(t)^2 \de t}
\end{equation}

\begin{equation}
    V\eff = AB\omega \sqrt{\frac{1}{\pi} \int_0^\pi \sin^2(\omega t) \de t}
    = \frac{1}{\sqrt{2}} A B \omega
    = \frac{V_0}{\sqrt{2}}
\end{equation}

Nell'esempio sopra, $V\eff = \qty{223}{\volt}$.

\subsection{Barretta in moto}

\addfigure[][0.6]{book/barretta_in_moto}

Circuito rettangolare con resistenza $R$ e una barretta lunga $L$ in moto a velocità $v$ espandendo il rettangolo.
Campo magnetico $B$ perpendicolare al circuito.

La f.e.m.\ è misurata con segno tale che $\uvt{n}$ ha lo stesso verso di $\B$.

Legge di autoinduzione:
\begin{equation}
    \fem_i = - \der{t} \Phi(B) = - v L B
\end{equation}

Forza di Lorentz:
\begin{equation}
    \fem_i = \int_\gamma \pts{\vt{v} \times \B} \lde
    = \int_\text{barretta} v B (-\ver{l}) \lde
    = - v L B
\end{equation}

Con la convenzione degli utilizzatori:
\begin{equation}
    I = \frac{v L B}{R}
\end{equation}

La forza esterna $\force$ per mantenere la barretta in moto rettilineo uniforme deve compensare la forza magnetica sulla barretta:
\begin{equation}
    \force = - I \vt{L} \times \B
    = \frac{vLB}{R} (L B \ver{v})
    = \frac{L^2 B^2}{R} \vt{v}
\end{equation}

La potenza da fornire è
\begin{equation}
    P = \force \cdot \vt{v} = \frac{v^2 L^2 B^2}{R}
\end{equation}

È proprio la potenza dissipata per effetto Joule:
\begin{equation}
    P = V_\text{resistore} I = - \fem_i I = \frac{v^2 L^2 B^2}{R}
\end{equation}

\subsection{Autoinduzione di secondo ordine}

Spira di raggio $A = \qty{10}{\centi\metre}$, resistenza $R = \qty{1}{\ohm}$ e attraversata da un campo magnetico perpendicolare $B = \alpha t^2$, con $\alpha = \qty{e-3}{\tesla\per\second\squared}$.
\begin{equation}
    \fem_i = - \pi A^2 \cdot 2 \alpha t
    \quad \implies \quad
    I = - \frac{2\pi A^2 \alpha}{R} t
\end{equation}

La corrente è variabile nel tempo e genera un campo magnetico indotto $B_i$.
Supponiamo per semplicità che il suo valore sulla superficie della spira sia ovunque uguale a quello nel centro:
\begin{equation}
    B_i = \frac{\mu_0 I}{2A} = - \frac{\mu_0 \pi A \alpha}{R}t
\end{equation}
La corrente indotta da un fenomeno secondario di autoinduzione è
\begin{equation}
    I_2 = -\frac{1}{R} \cdot \pi A^2 \cdot \pts{- \frac{\mu_0 \pi A \alpha}{R}}
    = \frac{\mu_0 \pi^2 A^3 \alpha}{R^2}
\end{equation}
$I_2$ scorre in modo da rafforzare il campo magnetico $B$ originario e genera a sua volta un campo magnetico $B_2$ con lo stesso verso di $B$ ma costante, e che quindi non induce altre correnti.
\begin{equation}
    B_2 = \frac{\mu_0 I_2}{2 A} = \frac{\mu_0^2 \pi^2 A^2 \alpha}{2 R^2}
\end{equation}

Calcoliamo i rapporti tra le grandezze per $t = \qty{0.1}{\second}$:
\begin{gather}
    \frac{I_2}{I} = \frac{B_2}{B_i} = -\frac{\mu_0 \pi A}{2 R t}
    = \num{2e-6}
\end{gather}
Inoltre, $I_2$ è sovrastimata.
Per questo, i fenomeni autoinduttivi di secondo ordine sono spesso trascurabili.
Il campo magnetico terrestre è dell'ordine di $\qty{e-5}{\tesla}$, quindi oscura tutti i campi magnetici dovuti ad autoinduzione di secondo ordine.

\subsection{Legge di Felici}

Spira in moto in presenza di un campo magnetico.
La carica che attraversa la spira tra i tempi $t_1$ e $t_2$ è
\begin{equation}
    q = \int_{t_1}^{t_2} \pts{-\frac{1}{R} \der[\Phi(\B)]{t}}\de t
    = -\frac{1}{R} \int_{\Phi_1}^{\Phi_2} \de \Phi
    = \frac{\Phi_1 - \Phi_2}{R}
\end{equation}

Esempio:

Spira di area $S$ e resistenza $R$ in un campo magnetico $\B$ non uniforme, ma che si può considerare uniforme sull'area della spira istante per istante.
Se ribalto la spira, la carica netta spostata è
\begin{equation}
    q = \frac{\Phi_1 - \Phi_2}{R} = \frac{2 B S}{R}
\end{equation}
È quindi possibile misurare un campo magnetico misurando (e integrando) la corrente che si genera mentre si ribalta una piccola spira:
\begin{equation}
    B = \frac{qR}{2S}
\end{equation}

\section{Mutua induzione}

\begin{gather}
    \Phi_{S_1}(\B_2) = M I_2 \\
    M = \frac{\mu_0}{4\pi} \int_{S_1} \oint_{\gamma_2} \frac{\de \vt{l}_2 \times \ver{r_2}}{\norm{\p_1 - \p_2}^2} \sde_1
\end{gather}
con $\ver{r_2} = (\p_1 - \p_2) / \norm{\p_1 - \p_2}$.

L'energia magnetica per accendere entrambi i circuiti è
\begin{equation}
    E_m = \frac{1}{2} \Phi_1 I_1 + \frac{1}{2} \Phi_2 I_2
    = \frac{1}{2} L_1 I_1^2 + \frac{1}{2} L_2 I_2^2 + M I_1 I_2
\end{equation}

Avendo $N$ circuiti,
\begin{equation}
    E_m = \frac{1}{2} \sum_{i = 1}^N \Phi_i I_i
    = \frac{1}{2} \sum_{i = 1}^N \sum_{j = 1}^N M_{ij} I_i I_j
\end{equation}
dove, per ogni $i, j = 1, \ldots, N$, $M_{ii} = L_i$ e $M_{ij} = M_{ji}$.

\subsection{Filo e spira}

Filo rettilineo infinito e spira di raggio $R = \qty{1}{\milli\metre}$ a distanza $r = \qty{1}{\metre}$.
Calcolare il coefficiente di mutua induzione.

Si suppone una corrente $I$ attraverso il filo e si calcola il flusso attraverso la superficie $S$ della spira:
\begin{equation}
\begin{gathered}
    M = \frac{\Phi_S(\B_\text{filo})}{I}
    = \int_S \frac{\mu_0}{2\pi r'} \de S'
    \approx \frac{\mu_0 R^2}{2 r}
    = \qty{6.28e-13}{\henry}
\end{gathered}
\end{equation}
L'approssimazione $r' \approx r$ segue da $R \ll r$.

\subsection{Spire concentriche}

Spire concentriche di raggi $R_1 = \qty{1}{\metre}$ e $R_2 = \qty{1}{\centi\metre}$.
Calcolare il coefficiente di mutua induzione.

Si calcola il flusso attraverso la superficie $S$ della spira piccola del campo dovuto a una corrente nella spira grande.

\begin{equation}
    M = \frac{\Phi_S(\B_\text{grande})}{I}
    \approx \int_S \frac{\mu_0}{2 R_1} \ver{n} \sde
    = \int_S \frac{\mu_0}{2 R_1} \de S
    = \frac{\mu_0 \pi R_2^2}{2 R_1}
    = \qty{2\pi^2 e-11}{\henry}
\end{equation}

Si approssima il campo magnetico nella spira piccola come se fosse uniforme e uguale al suo valore nel centro.

\subsection{Solenoide e spira}

Solenoide ideale lungo $L = \qty{5}{\metre}$ e con $N = 500$ spire.
Spira parallela alla sezione del solenoide e al suo interno, di diametro $d = \qty{10}{\milli\metre}$.

Flusso del campo nel solenoide attraverso la superficie $S$ della spira:
\begin{equation}
    M = \frac{\Phi_S(\B_\text{sol})}{I}
    = \int_S \mu_0 \frac{N}{L} \de S
    = \frac{\mu_0 N \pi d^2}{4 L}
    = \qty{9.87e-9}{\henry}
\end{equation}

\subsection{Spire concentriche in moto}

\adddrawio[][0.4]{concentric_coils}

$R_1 = \qty{50}{\centi\metre}$, $R_2 = \qty{3}{\centi\metre}$, $I_1 = \qty{20}{\ampere}$ nella spira grande.
La spira piccola ha resistenza $Z = \qty{0.02}{\ohm}$ ed è in moto in direzione perpendicolare al piano su cui giace con $v = \qty{0.02}{\metre\per\second}$.
Quanto vale la f.e.m.\ nella spira piccola dopo $t_0 = \qty{2}{\second}$?
\begin{gather}
    M(t) = \frac{\Phi_{S_2}(\B)}{I_1}
    \approx \int_{S_2} \frac{\mu_0 R_1^2}{2 (R_1^2 + z(t)^2)^{3/2}} \uz \sde
    = \frac{\mu_0 \pi R_1^2 R_2^2}{2 (R_1^2 + z(t)^2)^{3/2}} \\
    \fem_2 = \evalat{\der{t} \pts{M(t) I_1}}{t_0}
    = -\frac{\mu_0 I_1 \pi R_1^2 R_2^2}{2}
    \frac{3}{2}
    \frac{2 z(t_0) v}{(R_1^2 + z(t_0)^2)^{5/2}}
    = -\qty{6.7e-10}{\volt}
\end{gather}

Si può mostrare che l'autoinduzione nella spira piccola è del tutto trascurabile.


\subsection{Spira quadrata e due fili}

Trovare il coefficiente di mutua induzione $M$ tra i fili a distanza $d$ e una spira quadrata in mezzo di lato $h$ (ovvero, il coefficiente di mutua induzione tra la spira e un filo e tra la spira e l'altro filo).

Spira con uno qualsiasi dei due fili:
\begin{equation}
\begin{gathered}
    M = \frac{\Phi_S(\B)}{I}
    = \int_S \frac{\mu_0}{2\pi r'} \de S'
    = \frac{\mu_0}{2\pi} \int_{x = d/2 - h/2}^{d/2 + h/2} \int_{y = 0}^h \frac{1}{x} \, \de y \, \de x = \\
    = \frac{\mu_0 h}{2\pi} \Big[\!\ln x\Big]_{d/2 - h/2}^{d/2 + h/2}
    = \frac{\mu_0 h}{2\pi} \ln \frac{d + h}{d - h}
\end{gathered}
\end{equation}

Se nei due fili la corrente scorre in versi opposti, l'induzione totale è nulla.
