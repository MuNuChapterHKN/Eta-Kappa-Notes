\chapter{Elettrostatica}

\section{Principi dell'elettrostatica}

L'elettrostatica nasce dalla constatazione che esiste una proprietà della materia diversa dalla massa che si manifesta attraverso forze attrattive e repulsive.
Questa proprietà è detta \important{carica elettrica}, posseduta da un corpo materiale come è posseduta la massa, e ha un segno.

\subsection{Legge di Coulomb}

La forza elettrica è evidentemente più forte di quella gravitazionale e per quantificarla devo raggiungere il limite della carica puntiforme: sferette cariche poste a grande distanza (come fossero masse puntiformi).

Così facendo, si scopre la \important{legge di Coulomb}: due cariche $q_1$ e $q_2$ poste a distanza $r$ esercitano l'una sull'altra una forza pari a
\begin{equation}
    \force = k_e \frac{q_1 q_2}{r^2} \ver{r}
\end{equation}
dove $k_e$ è una costante di proporzionalità e di conversione tra unità di misura e $\ver{r}$ è il vettore unitario che punta dalla carica che esercita la forza a quella che la sente.
Se la forza è attrattiva, allora le cariche sono di segno opposto (cioè $q_1 q_2 < 0$).

La legge di Coulomb permette di definire la grandezza fisica ``carica elettrica'' e la sua unità di misura.
Questo è possibile poiché nell'equazione figurano solo grandezze fisiche già definite, come la forza e la distanza.

Si fissa il valore numerico di $k_e$ a $\{k_e\} = 10^{-7} (c / (\qty{1}{\metre\per\second}))^2$ (dove $c$ è la velocità della luce nel vuoto) e si ottiene l'unità di carica \qty{1}{\coulomb}.

La costante ha un valore numerico ``brutto'' perché in realtà il coulomb è stato definito dopo l'ampere (vedi \autoref{sec:def_ampere}).

Fatto questo, l'unità di misura di $k_e$ risulta essere $[k_e] = \unit{\newton\metre\squared\per\coulomb\squared}$.

Si definisce anche $\eps_0$, la \important{costante dielettrica del vuoto}, in modo che valga
\begin{equation}
    k_e = \frac{1}{4\pi \eps_0}
\end{equation}

Fatto ciò, la legge di Coulomb, da legge empirica, diventa un principio.

(Le leggi empiriche sono quelle che non si dimostrano, ma si osservano direttamente.)

\subsection{Conservazione della carica}

Si osserva anche un'altra proprietà: \important{la carica elettrica si conserva}.
Cioè, considerando un volume $V \subset \R^3$ delimitato da una superficie $S$, le cariche entrano o escono attraverso $S$, ma non nascono né spariscono lì dentro.

Di preciso, a conservarsi è la carica netta (cioè, considerata con segno).

\subsection{Principio di sovrapposizione degli effetti}

Un terzo principio è il \important{principio di sovrapposizione degli effetti}: se in presenza di una carica $q_0$ in posizione $\p_0$ ci sono altre $n$ cariche $q_i$ in posizioni $\p_i$, per $i = 1, \ldots, n$, la forza totale su $q_0$ è la somma vettoriale delle forze che le altre cariche esprimono singolarmente su $q_0$.
\begin{equation}
    \force_0 = \sum_{i=1}^n \frac{q_0 q_i}{4 \pi \eps_0 \norm{\p_0 - \p_i}^2} \ver{r_i}
\end{equation}
dove $\ver{r_i}$ è il versore che va da $\p_i$ a $\p_0$:
\begin{equation}
    \ver{r_i} = \frac{\p_0 - \p_i}{\norm{\p_0 - \p_i}}
\end{equation}

Attenzione: il principio di sovrapposizione non è qualcosa di implicito nella legge di Coulomb, ma va verificato sperimentalmente.
Infatti, in generale, ci sono interazioni che non lo rispettano.

Grazie al principio di sivrapposizione, di può definire il \important{campo elettrico}: la forza che agisce per unità di carica in un punto dello spazio, con unità di misura $\unit{\newton\per\coulomb}$.

Il campo elettrico generato dalle $n$ cariche di prima valutato in $\p_0$ è
\begin{equation}
    \E(\p_0) = \frac{\force_0}{q_0} = \sum_{i=1}^n \frac{q_i}{4 \pi \eps_0 \norm{\p_0 - \p_i}^2} \ver{r_i}
\end{equation}

È una proprietà definita in ogni punto dello spazio, e si misura tramite cariche campione.

Il campo elettrico venne introdotto da Faraday come semplice strumento matematico.
Tuttavia, sembra che sia molto più fondamentale di così dal punto di vista fisico, così come altri campi che incontreremo in seguito.
Sembra infatti che sia una proprietà non solo nello spazio, ma \textit{dello} spazio.

In sintesi:

L'elettrostatica è lo studio delle proprietà elettriche della materia ferma nello spazio.
Si basa su tre principi:
\begin{itemize}
    \item legge di Coulomb
    \item conservazione della carica
    \item principio di sovrapposizione degli effetti
\end{itemize}

\section{Conservatività del campo elettrico}

Se la carica $q$ che genera il campo è nell'origine, il campo elettrico in un punto generico $P$ a distanza $r$ dall'origine è
\begin{equation}
    \E(P) = \frac{q}{4\pi\eps_0 r^2} \ver{r}
\end{equation}
con $\ver{r}$ versore radiale verso $P$.

Se vi è una carica $q_0$ in $P$, la forza che agisce su essa è $q_0 \E(P)$.

\adddrawio[Carica $q_0$ in moto lungo una curva $\gamma$ in presenza del campo elettrico generato dalla carica $q$.]{efield_work}{0.4}

Se la carica viene spostata lungo un arco $\gamma$ dal punto $A$ al punto $B$ a distanze $r_A$ e $r_B$ dall'origine, la forza elettrica svolge un lavoro e si dimostra che questo lavoro dipende solo dalla componente radiale dello spostamento:
\begin{equation}
\label{eq:potenziale_elettrostatico}
    \begin{gathered}
        L_{AB, \gamma} = \int_\gamma \force \lde = \frac{q_0 q}{4\pi \eps_0} \int_{r_A}^{r_B} \frac{1}{r^2} \de r = - \frac{q_0 q}{4\pi \eps_0} \pts{\frac{1}{r_B} - \frac{1}{r_A}} = \\
        = -q_0 \pts{\frac{q}{4\pi \eps_0 r_B} - \frac{q}{4\pi \eps_0 r_A}}
    \end{gathered}
\end{equation}

Poiché $L_{AB,\gamma}$ non dipende da $\gamma$, \important{il campo elettrico è conservativo} (cioè genera forze conservative).

I due termini nell'\cref{eq:potenziale_elettrostatico}, se moltiplicati per $q_0$, sono energie che determinano una variazione di energia potenziale.

Trascurando la carica campione, si ottiene il potenziale elettrico $\spot(P)$: l'energia potenziale per unità di carica, con unità di misura volt: $\unit{\volt} = \unit{\joule\per\coulomb}$.

Si ha quindi
\begin{equation}
\label{eq:spot_integrale}
    \int_A^B \E \lde = -\int_A^B \de \spot
\end{equation}
% \begin{equation}
%     \int_A^B \E \lde = \frac{1}{q_0} \int_A^B \force \lde = \frac{1}{q_0} (- q_0) \pts{\frac{q}{4\pi \eps_0 r_B} - \frac{q}{4\pi \eps_0 r_A}} = - (\spot_B - \spot_A) = -\int_A^B \de \spot
% \end{equation}
In termini di differenziali,
\begin{equation}
\label{eq:spot_differenziali}
    \E \lde = - \de \spot
\end{equation}
Questa è una legge di conservazione dell'energia (energia cinetica a sinistra, ``meno energia potenziale'' a destra).

Quindi, $[\E] = \unit{\volt\per\metre}$, preferito a \unit{\newton\per\coulomb}, che non è usato.

Risolvendo l'\cref{eq:spot_integrale} per $\spot$ si ottiene
\begin{equation}
    \spot = \frac{q}{4\pi \eps_0 r} + \text{costante}
\end{equation}

Dal punto di vista ficico, ha senso porre a zero l'energia potenziale quando la carica di prova è a distanza infinita, poiché essa non subisce più lavoro.
Pertanto, la costante arbitraria si sceglie pari a $0$.

L'energia potenziale soddisfa il principio di conservazione degli effetti, quindi anche il potenziale lo soddisfa: se ci sono più cariche, i potenziali si sommano (come scalari)
\begin{equation}
    \spot(\p) = \sum_{i=1}^n \frac{q_i}{4\pi \eps_0 \norm{\p - \p_i}}
\end{equation}

Come si calcola $\E$ a partire da $\spot$?
\Cref{eq:spot_differenziali} si riscrive come
\begin{equation}
    E_x \de x + E_y \de y + E_z \de z = - \de \spot
\end{equation}
Fissando $y$ e $z$,
\begin{equation}
    E_x \de x = - \de \spot \implies E_x = - \parder[\spot]{x}
\end{equation}
Ripetendo per $y$ e $z$ si ottiene
\begin{equation}
\label{eq:meno_grad_spot}
    \E = - \grad \spot
\end{equation}

Dal fatto che $\E$ è conservativo, si deduce che la circuitazione di $\E$ (il lavoro della forza elettrica per unità di carica lungo una curva chiusa) è sempre nulla.

Inoltre, poiché $\E$ è conservativo, allora è anche irrotazionale:
\begin{equation}
    \curl \E = \vt{0}
\end{equation}

Fisicamente, questo si intuisce tramite il teorema di Stokes: se ogni circuitazione è nulla, allora è nullo anche il flusso del rotore attraverso qualunque superficie.

In particolare, calcoliamo la circuitazione lungo una spira rettangolare infinitesima perpendicolare all'asse $x$:
\begin{equation}
    \oint_\gamma \E \lde = \pts{\parder[E_z]{y} - \parder[E_y]{z}} \de y \de z = \pts{\curl \E} \sde
\end{equation}
È il teorema di Stokes per una curva chiusa infinitesima.
L'integrale a destra è sempre nullo, quindi deve essere sempre nullo anche il rotore.


% ## Interazione a distanza

% A seguito della scoperta della legge di gravitazione universale, non piaceva l'interazione a distanza
% L'introduzione di un campo permette invece di spiegare ciò tramite la propagazione di una proprietà dello spazio

\section{Distribuzione continua di carica}

Densità di carica $\rho$:
\begin{equation}
    \de q = \rho \, \de V
\end{equation}
definendo
\begin{equation}
    \ver{r'} \coloneq \frac{\p - \p'}{\norm{\p - \p'}}
\end{equation}
Si ottengono delle espressioni per $\E$ e $\spot$ in funzione della distribuzione di carica nello spazio:
\begin{gather}
    \de \E(\p) = \frac{\de q}{4\pi \eps_0 \norm{\p - \p'}^2} \ver{r'}
    \implies
    \E(\p) = \frac{1}{4\pi \eps_0} \int_V \frac{\rho(\p')}{\norm{\p - \p'}^2} \ver{r'} \, \de V' \\
    \de \spot(\p) = \frac{\de q}{4\pi \eps_0 \norm{\p - \p'}}
    \implies
    V(\p) = \frac{1}{4\pi \eps_0} \int_V \frac{\rho(\p')}{\norm{\p - \p'}} \de V'
\end{gather}
$V$ è il volume all'interno del quale la densità di carica è non nulla (ciò su cui occorre integrare).

\section{Legge di Gauss}

Sia $V$ un volume tale che $S = \partial V$ sia una superficie chiusa connessa.

Flusso di $\E$ attraverso $S$:
\begin{equation}
    \de \Phi = \E \sde, \qquad \Phi = \oint_S \E \sde
\end{equation}

\adddrawio{gauss_proof}{0.4}

Calcolo per una carica $q$ in $\p' \in V$ con $R = \norm{\p - \p'}$
(una rappresentazione bidimensionale è data in \cref{fig:gauss_proof})
\begin{equation}
    \de \Phi = \frac{q}{4\pi \eps_0} \frac{\ver{r'} \cdot \uvt{n}}{R^2} \de S = \frac{q}{4\pi \eps_0} \frac{\de S \cos \theta}{R^2} = \frac{q}{4\pi \eps_0} \de \Omega
\end{equation}
$\Omega$ è l'angolo solido.
\begin{equation}
    \Phi = \oint_S \frac{q}{4\pi \eps_0} \de \Omega
    = \frac{q}{4\pi \eps_0} 4\pi
    = \frac{q}{\eps_0}
\end{equation}

Se ci sono più cariche o un volume di carica, per il principio di sovrapposizione degli effetti, è sufficiente considerare la carica totale.
Si ottiene quindi il formato integrale della legge di Gauss:
\begin{equation}
\label{eq:gauss_integrale}
    \Phi = \oint_S \E \sde = \frac{1}{\eps_0} \int_V \rho \, \de V
\end{equation}

Il teorema di Gauss applicato al flusso del campo elettrico risulta in
\begin{equation}
    \oint_S \E \sde = \int_V \diver \E \, \de V
\end{equation}
Poiché l'\cref{eq:gauss_integrale} deve valere per ogni volume $V$, occorre eguagliare gli integrandi e si ottiene il formato differenziale della legge di Gauss:
\begin{equation}
    \diver \E = \frac{\rho}{\eps_0}
\end{equation}

Attenzione: legge di Gauss $\ne$ teorema di Gauss $=$ teorema della divergenza.

Diamo un'idea del perché valga il teorema di Gauss.

\adddrawio{gauss_cube}{0.6}

Consideriamo un volume infinitesimo cubico come in \cref{fig:gauss_cube}, con vertice di coordinate minori $\p$, di lati $\de x$, $\de y$, $\de z$ e volume $\de V = \de x \de y \de z$.
Se $\de \p \parallel \ux$,
\begin{equation}
    E_x(\p + \de \p) \approx E_x(\p) + \grad E_x(\p) \cdot \de \p = E_x(\p) + \parder[E_x(\p)]{x} \de x
\end{equation}
Il flusso attraverso le facce parallele al piano $zy$ è
\begin{equation}
    E_x(\p + \de \p) \de y \de z - E_x(\p) \de y \de z = \parder[E_x(\p)]{x} \de x \de y \de z
\end{equation}
Ripetendo per $y$ e $z$ e sommando si ottiene il flusso totale:
\begin{equation}
    \de \Phi = (\diver \E) \de V
\end{equation}

Ora, accostando piccoli volumi, i contributi adiacenti dei flussi si semplificano. Per un volume generico vale
\begin{equation}
    \Phi = \int_V \diver \E \, \de V
\end{equation}

\section{Equazione di Poisson}

Osserva che le due seguenti leggi:
\begin{gather}
\label{eq:gauss_differenziale}
    \diver \E = \frac{\rho}{\eps_0} \\
    \curl \E = \vt{0}
\end{gather}
sono conseguenze dei soli tre principi dell'elettrostatica e sono equivalenti a essi, ne sono una riscrittura analitica.

Dai tre principi siamo passati alle due leggi sopra.
Ora, è possibile ridurre queste leggi in un'unica equazione, sostituendo l'\cref{eq:meno_grad_spot} nell'\cref{eq:gauss_differenziale}:
\begin{equation}
    \lapl \spot = - \frac{\rho}{\eps_0}
\end{equation}
La soluzione di quest'equazione differenziale è proprio
\begin{equation}
    \spot(\p) = \frac{1}{4\pi \eps_0} \int_V \frac{\rho(\p')}{\norm{\p - \p'}} \de V'
    % = \frac{1}{\eps_0} \mathcal{S} \rho(r)
\end{equation}

% Tuttavia, la forma differenziale è più usata numericamente (:-/)

\section{Conduttori e condensatori}

Conduttore: materiale in cui le cariche possono muoversi liberamente.

In un conduttore, le cariche si posizionano sulla superficie.
Per questo, all'interno, $\E = \vt{0}$ e $\rho = 0$.
Ne segue anche che il potenziale è costante per l'intero conduttore.

Sulla superficie, $\E \parallel \uvt{n}$.
Se non fosse così, si avrebbe una componente tangenziale del campo elettrico e le cariche si muoverebbero sulla superficie.

È conveniente definire la densità superficiale (o ``areica'') $\sigma = \de q / \de S$.

Il potenziale a cui è il conduttore è
\begin{equation}
    \spot(\p) = \frac{1}{4\pi \eps_0} \int_S \frac{\sigma(\p')}{\norm{\p - \p'}} \de S'
\end{equation}

Rapporto potenziale/carica per le cariche sulla superficie:
\begin{equation}
    \frac{1}{C} = \frac{\spot(\p)}{Q} = \frac{1}{4\pi \eps_0} \int_S \frac{\sigma(\p')/Q}{\norm{\p - \p'}} \de S', \quad \p \in S
\end{equation}
$C$ è detta \important{capacità} ed è una costante che dipende solo dalla forma della cuperficie $S$ poiché l'integrale è puramente geometrico.

Consideriamo due conduttori con lo stesso quantitativo di carica di segno opposto $Q = Q_+ = -Q_- > 0$.
Si definiscono
\begin{itemize}
    \item armetura: ciascuno dei due conduttori
    \item condensatore: il sistema completo
\end{itemize}

Le linee di campo escono dall'armatura positiva ed entrano in quella negativa.

La differenza di potenziale (d.d.p.) vale

\begin{equation}
    \Delta \spot = \frac{Q_+}{C_+} - \frac{Q_-}{C_-} = Q \underbrace{\pts{\frac{1}{C_+} + \frac{1}{C_-}}}_{1/C}
\end{equation}

$C$ è la capacità del condensatore e si misura in farad: $[C] = \unit{\farad} = \unit{\coulomb\per\volt}$.

\subsection{Condensatore a facce piane parallele}

Consideriamo due piani conduttori infiniti paralleli di superficie $S$ a distanza $d$.

``Infiniti'' significa che $d$ è trascurabile rispetto alla loro superficie, $d \ll \sqrt{S}$.

Le distribuzioni di densità di carica superficiale sono uguali in modulo: $\sigma = \sigma_+ = -\sigma_- > 0$.

Per calcolare $C$, serve calcolare $V = V_+ - V_-$.

Poiché il piano è infinito, il campo elettrico è perpendicolare al piano.

\adddrawio{gauss_plane}{0.3}

Considerando un cilindro di area di base $A$ perpendicolare al piano e che lo attraversa, come in \cref{fig:gauss_plane}.
Detto $E$ il modulo del campo elettrico presso le basi del cilindro, si applica la legge di Gauss:
\begin{equation}
    \Phi = 2 E A = \frac{\sigma A}{\eps_0} \implies E = \frac{\sigma}{2 \eps_0}
\end{equation}
In particolare, il campo è uniforme da ognuna delle due parti del piano.

Sommando il contributo dell'altro piano (che, tra le armature, è concorde), il campo tra le armature è, in modulo,
\begin{equation}
    E = \frac{\sigma}{\eps_0}
\end{equation}
Otteniamo la capacità del condensatore:
\begin{equation}
    V = - \int_{-,\gamma}^+ \E \lde = \frac{\sigma d}{\eps_0} \implies
    C = \frac{Q}{V} = \frac{\sigma S}{\frac{\sigma d}{\eps_0}} = \eps_0 \frac{S}{d}
\end{equation}

\subsection{Energia del campo elettrico}

Quale lavoro serve per spostare una carica $Q > 0$ da un'armatura all'altra?
\begin{gather}
    \de L = V \de q \\
    L = \int_0^Q V \de q = \int_0^Q \frac{q}{C} \de q = \frac{Q^2}{2 C} = \frac{1}{2} C V^2
\end{gather}
Nel caso di un condensatore a facce piane parallele:
\begin{equation}
    L = \frac{1}{2} \underbrace{\eps_0 \frac{S}{d}}_C {\underbrace{(E d)}_V}^2 = \frac{1}{2} \eps_0 S d E^2
\end{equation}
Osservando che $S d$ è il volume tra le facce del condensatore, l'energia per unità di volume è
\begin{equation}
    w_E = \frac{1}{2} \eps_0 E^2
\end{equation}

Questa è l'energia del campo elettrico.
Questa formula, in realtà, vale sempre, non solo per questo specifico caso.
Alla presenza di campo elettrico nello spazio è associata una certa densità di energia elettrica $w_E$.

Il lavoro svolto per spostare le cariche (ad esempio, da generatori di tensione o di corrente) risulta nella creazione di un campo elettrico, quindi si trasforma in questa forma di energia.

\section{Corrente elettrica}

Consideriamo un cilindro con due facce cariche in modo opposto: al suo interno ho un campo elettrico uniforme.
Sia $n$ il numero di cariche libere $q$ per unità di volume.
Messe in moto dal campo elettrico, hanno velocità $\vt{v}$.

% ($\vt{v}$ dovrebbe aumentare linearmente, in quanto il moto è uniformemente accelerato.
% In realtà è costante, poiché gli elettroni perdono energia urtando gli atomi del conduttore. Vedi \autoref{sec:legge_ohm}.)

La quantità di carica $\Delta Q$ che attraversa una sezione $S$ in un tempo $\Delta t$ è
\begin{equation}
    \Delta Q = n q \Delta t \vt{v} \cdot \vt{S}
\end{equation}
poiché il volume delle cariche che raggiungono la superficie è $(\vt{v}\Delta t) \cdot \vt{S}$.

Corrente elettrica: quantità di carica che attraversa una sezione per unità di tempo:
\begin{equation}
    I \coloneq \frac{\Delta Q}{\Delta t} = nq\vt{v} \cdot \vt{S}
\end{equation}
Unità di misura: ampere $\unit{\ampere} = \unit{\coulomb\per\second}$.

Densità di corrente: corrente per unità di superficie:
\begin{equation}
    \dcurr = nq\vt{v}
\end{equation}

$I$ è il flusso di $\dcurr$ attraverso $S$:
\begin{equation}
    \de I = \dcurr \sde, \qquad I = \int_S \dcurr \sde
\end{equation}

\subsection{Principio di conservazione della carica}

Se all'interno di un volume $V$ la carica è costante, la corrente totale attraverso la superficie è nulla:
\begin{equation}
    I = \oint_{\partial V} \dcurr \sde = 0
\end{equation}

\subsection{Legge di Ohm}
\label{sec:legge_ohm}

Consideriamo un conduttore e accendiamo un campo elettrico (lo si fa mettendo cariche positive da una parte e negative dall'altra).
Le cariche nel conduttore vengono messe in moto e si genera una corrente $\de I = \dcurr \sde$.

La corrente fa sì che le cariche positive interne si avvicinino a quelle negative che generano il campo, e viceversa.
Alla fine, il campo si sarà annullato.

Per permettere che attraverso un conduttore scorra corrente indefinitamente, serve svolgere del lavoro per riportare le cariche dall'altra parte (rispettando il principio di conservazione dell'energia).

Il lavoro svolto dal campo elettrico è
\begin{equation}
    \de L = \force \lde = q \E \lde = - q \de \spot
\end{equation}
Considerando una quantità di carica positiva $\Delta Q$ (che si muove verso le cariche negative) e definendo $V = \spot_+ - \spot_-$, il lavoro risulta direttamente proporzionale alla corrente.

Chiamiamo $R$ la costante di proporzionalità:
\begin{equation}
    L = \Delta Q \, V = \Delta Q  R I
\end{equation}

Si ottiene la \important{legge di Ohm}:
\begin{equation}
    V = R I
\end{equation}

$R$ è detta \important{resistenza} e l'unità di misura è l'ohm: $[R] = \unit{\ohm} = \unit{\volt\per\ampere}$.

Se il conduttore è un filo, si misura che in tutte le sezioni del filo la corrente è la stessa.
Poiché $I = \rho v S$ e la sezione del filo è costante, la velocità sarà la stessa.
Quindi, \important{la velocità delle cariche è costante}, è detta \important{velocità di deriva} e non aumenta linearmente come ci si aspetterebbe da un moto uniformemente accelerato (dovuto a un campo elettrico costante).
Parte dell'energia del campo elettrico, infatti, viene \important{dissipata in calore} per \important{effetto Joule}, che ha l'effetto di scaldare il filo.

Questo meccanismo può essere modellizzato tramite urti delle cariche in moto contro gli atomi del materiale.
È un modello ``sbagliato'', ma dà risultati corretti.

Tra un urto e l'altro, il moto è uniformemente accelerato, ma la velocità acquisita viene persa a ogni urto.

La legge di Ohm è una legge di conservazione dell'energia: $V$ quantifica il lavoro del campo elettrico, $RI$ la dissipazione termica.
$R$ svolge il ruolo della viscosità.

Per garantire che scorra una corrente $I$, è necessario un generatore di tensione che compensi l'energia dissipata.
Questo può essere chimico (pila o batteria) o meccanico (forza elettromotrice).

La resistenza è una grandezza estensiva e dipende dalla geometria del materiale.
Dette $l$ la lunghezza del conduttore e $S$ la sezione,
\begin{gather}
    V = RI
    \implies E l = R J S \\
    \implies J = \frac{l}{RS} E = \sigma E
\end{gather}
$\sigma$ non dipende dalla geometria ed è una proprietà del materiale: \important{conducibilità elettrica}, $[\sigma] = \unit{\per\ohm\per\metre}$

Posso anche calcolare la velocità:
\begin{equation}
    v = \frac{J}{nq} = \frac{l}{n q R S} E \sim \qty{e-4}{\metre\per\second}
\end{equation}
Quest'ordine di grandezza vale per molti materiali.

Per $T = \qty{300}{\kelvin}$, la velocità termica (cioè la velocità quadratica media) di un ``gas ideale di elettroni'' sarebbe:
\begin{equation}
    v_\text{th} = \sqrt{\frac{3 k_B T}{m_e}} \sim \qty{e5}{\metre\per\second}
\end{equation}
Questo mostra che la velocità netta delle cariche è molto minore della velocità termica e che la corrente elettrica deriva da un moto additivo dovuto al campo elettrico, debolissimo rispetto al moto termico.
