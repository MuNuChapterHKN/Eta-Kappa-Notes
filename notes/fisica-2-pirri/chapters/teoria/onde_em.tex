\chapter{Onde elettromagnetiche}

\begin{itemize}
    \item Legge di induzione elettromagnetica $\to$ campi magnetici variabili generano campi elettrici perpendicolari ai campi inducenti.
    \item Legge di Ampère-Maxwell $\to$ campi elettrici variabili generano campi magnetici perpendicolari ai campi inducenti.
\end{itemize}

In un punto $\p$ nel vuoto, siano $\E(\p) \parallel \uy \perp \B(\p) \parallel \uz$.

\addfigure{book/waves_proof_2}

Si applica Ampère-Maxwell per un piccolo quadrato $\perp \uy$:
\begin{equation}
\label{eq:ampere_maxwell_quadratino}
\begin{gathered}
    \oint_\gamma \B \lde
    = \mu_0\eps_0 \der{t} \Phi_S(\E) \\
    \implies
    -\pts{\B + \parder[\B]{x} \de x} \cdot \de z \uz + \B \cdot \de z \uz
    = \mu_0 \eps_0 \parder[\E]{t} \cdot \de z \de x \uy \\
    \implies
    - \parder[\B]{x} \cdot \de x \de z \uz
    = \mu_0 \eps_0 \parder[\E]{t} \cdot \de x \de z \uy \\
    \implies
    - \parder[B]{x} = \mu_0 \eps_0 \parder[E]{t}
\end{gathered}
\end{equation}

\addfigure{book/waves_proof_1}

Si applica Faraday-Henry per un piccolo quadrato $\parallel \uz$:
\begin{equation}
\label{eq:faraday_henry_quadratino}
\begin{gathered}
    \oint_\gamma \E \lde
    = -\der{t} \Phi_S(\B) \\
    \implies
    \pts{\E + \parder[\E]{x} \de x} \cdot \de y \uy - \E \cdot \de y \uy
    = - \parder[\B]{t} \cdot \de x \de y \uz \\
    \implies
    \parder[\E]{x} \cdot \de x \de y \uy
    = - \parder[\B]{t} \cdot \de x \de y \uz \\
    \implies
    \parder[E]{x} = - \parder[B]{t}
\end{gathered}
\end{equation}

Riassumendo,
\begin{equation}
    \begin{cases}
        - \parder[B]{x} = \mu_0 \eps_0 \parder[E]{t} \\
        \parder[E]{x} = - \parder[B]{t}
    \end{cases}
\end{equation}

Derivando ulteriormente l'\cref{eq:ampere_maxwell_quadratino} (AM) rispetto a $t$ e l'\cref{eq:faraday_henry_quadratino} (FH) rispetto a $x$,
\begin{equation}
    \begin{cases}
        - \partial^2_{tx} B = \mu_0 \eps_0 \parder[E][2]{t} \\
        \parder[E][2]{x} = - \partial^2_{xt} B
    \end{cases}
    \implies \parder[E][2]{x} = \mu_0 \eps_0 \parder[E][2]{t}
\end{equation}

Derivando ulteriormente AM rispetto a $x$ e FH rispetto a $t$,
\begin{equation}
    \begin{cases}
        - \parder[B][2]{x} = \mu_0 \eps_0 \partial^2_{tx} E \\
        \partial^2_{xt} E = - \parder[B][2]{t}
    \end{cases}
    \implies \parder[B][2]{x} = \mu_0 \eps_0 \parder[B][2]{t}
\end{equation}

Cioè
\begin{gather}
    \parder[E][2]{x} = \mu_0 \eps_0 \parder[E][2]{t} \\
    \parder[B][2]{x} = \mu_0 \eps_0 \parder[B][2]{t}
\end{gather}

$\E$ e $\B$ sono nuovamente disaccoppiati, e per ciascuno vale l'\important{equazione di d'Alembert}:
\begin{equation}
\label{eq:dalembert_1d}
    \parder[y][2]{x} = \frac{1}{v^2} \parder[y][2]{t}
\end{equation}

con $\mu_0 \eps_0 = 1/v^2$.

\section{Derivazione in forma differenziale}

Usando la seguente identità dell'analisi vettoriale
\begin{equation}
    \curl \curl \vt{F} = \grad \pts{\diver \vt{F}} - \vlapl \vt{F}
\end{equation}
si giunge alla forma generale dell'equazione di d'Alembert per $\E$ e $\B$ nel vuoto:
\begin{gather}
    \curl \E = - \parder[\B]{t} \\
    \implies
    \curl \curl \E = - \curl \parder[\B]{t} \\
    \implies
    \cancel{\grad \pts{\diver \E}} - \vlapl \E = - \parder{t} \curl \B \\
    \implies
    \vlapl \E = \parder{t} \pts{\mu_0 \eps_0 \parder[\E]{t}} \\
    \implies
    \vlapl \E = \frac{1}{c^2} \parder[\E][2]{t}
\end{gather}

Analogamente per $\B$.

\section{Soluzioni dell'equazione di d'Alembert}

La soluzione dell'\cref{eq:dalembert_1d}, in una dimensione, è
\begin{equation}
    y(x, t) = f_1(x - vt) + f_2(x + vt)
\end{equation}
Ovvero, la somma di funzioni $f_1$ e $f_2$ generiche che traslano nello spazio in direzione $x$ con una certa velocità.
Si tratta di onde, segnali che si spostano nello spazio.

Quindi, $\E$ e $\B$ si propagano nello spazio come onde, con velocità
\begin{equation}
    c = \frac{1}{\sqrt{\mu_0 \eps_0}} = \qty{299792458}{\metre\per\second}
\end{equation}
$c$ è la \important{velocità dell'onda elettromagnetica nel vuoto}.

Mentre le altre onde hanno un supporto, queste si propagano nel vuoto e sono perturbazioni di campi.
Storicamente, vi furono due reazioni:
\begin{itemize}
    \item Si tratta solo di un risultato matematico senza realtà fisica
    \item Esiste un supporto, l'\textit{etere}, che permette la propagazione delle onde elettromagnetiche
\end{itemize}

Le onde elettromagnetiche sono sempre solo trasversali (cioè, perpendicolari alla direzione di propagazione), non longitudinali.
Se la direzione di propagazione varia, $\E$ e $\B$ rimangono perpendicolari tra loro e alla direzione di propagazione.

Le funzioni $f_1$ e $f_2$ vengono considerate sinusoidali:
\begin{equation}
    f(x - c t) = A \sin\pts{k (x - c t)} = A \sin(k x - \omega t)
    = A \sin\pts{2\pi \pts{\frac{x}{\lambda} - \frac{t}{T}}}
\end{equation}
$A$ è l'ampiezza, $k = 2\pi / \lambda$ è il \important{numero d'onda}, $\omega = 2\pi / T$ è la $\important{pulsazione}$, $k x - \omega t$ è la \important{fase}.
Segue che
\begin{equation}
    c = \frac{\omega}{k} = \frac{\lambda}{T}
\end{equation}
Infatti, grazie alla legge di Fourier, qualunque funzione si può scrivere come serie di seni e coseni.

Le onde con $f$ sinusoidale sono dette \important{onde armoniche}.
Ogni moto di carica non rettilineo si può approssimare moto circolare (poiché la traiettoria si può approssimare come arco di circonferenza), quindi ogni carica in moto emette ``localmente'' onde elettromagnetiche sinusoidali.

Anche le cariche nei materiali emettono onde elettromagnetiche sinusoidali, poiché sono in moti oscillatori.
\begin{subequations}
\begin{gather}
    \E(x, t) = \E_0 \sin(k x - \omega t) \\
    \B(x, t) = \B_0 \sin(k x - \omega t)
\end{gather}
\end{subequations}

Usando l'\cref{eq:faraday_henry_quadratino},
\begin{equation}
\begin{gathered}
    \parder[E]{x} = - \parder[B]{t}
    \implies
    \der[E]{(x - c t)} \der[(x - c t)]{x} = - \der[B]{(x - c t)} \der[(x - c t)]{t} \\
    \implies
    \der[E]{(x - c t)} = c \der[B]{(x - c t)}
    \implies
    E = c B
\end{gathered}
\end{equation}

\important{Fronte d'onda}: superficie a fase costante.
È $\parallel \E \times \B$ (cioè perpendicolare sia a $\E$ si a $\B$) e si sposta a velocità $c$.

In conclusione, nel vuoto:
\begin{subequations}
\begin{gather}
    \vlapl \E = \frac{1}{c^2} \parder[\E][2]{t} \\
    \vlapl \B = \frac{1}{c^2} \parder[\B][2]{t} \\
    \E \perp \B \\
    \norm{\E} = c\norm{\B}
\end{gather}
\end{subequations}


\section{Tipi di onde}

\subsection{Onda sferica}

Un'onda sferica si genera quando una sorgente puntiforme emette onde in un mezzo isotropo ed è caratterizzata da $\vt{k} \parallel \p$ (ponendo la sorgente nell'origine).
Infatti, la direzione di propagazione è necessariamente radiale rispetto alla sorgente.
\begin{equation}
    \E(\p, t) = \E_0(r) \sin(k r - \omega t)
\end{equation}
L'ampiezza dipende da $r$ per ragioni di conservazione dell'energia (vedi \autoref{sec:intensita_onda_sferica}).
La fase dipende solo dal modulo di $\p$ poiché i fronti d'onda sono, appunto, a simmetria sferica.

Per conoscere l'onda dovuta a una sorgente generica, occorre considerare varie sorgenti puntiformi e integrare sulle onde sferiche che queste generano.

\subsection{Onda piana}

Sono le soluzioni (non banali) più semplici alle equazioni di Maxwell nel vuoto.

Un'onda piana è tale che le direzioni di $\E$ e $\B$ sono costanti.
\begin{subequations}
\begin{gather}
    \E(\vt{x}, t) = \E_0 \sin(\vt{k} \cdot \vt{x} - \omega t)
    = \imag{\E_0 \exp{\im(\vt{k} \cdot \vt{x} - \omega t)}} \\
    \B(\vt{x}, t) = \B_0 \sin(\vt{k} \cdot \vt{x} - \omega t)
    = \imag{\B_0 \exp{\im(\vt{k} \cdot \vt{x} - \omega t)}}
\end{gather}
\end{subequations}
$\vt{k}$ è il \important{vettore d'onda}, di modulo $k$ (il numero d'onda) e che ha per direzione la direzione di propagazione.
Nel caso di un'onda piana, è costante.

Un'onda piana è \important{polarizzata linearmente}, cioè sia $\E$ sia $\B$ oscillano in piani fissi.

\addfigure[Onda elettromagnetica polarizzata circolarmente.]{book/polarizzazione_circolare}

Onda \important{polarizzata circolarmente}: $\E$ e $\B$, anziché oscillare in modulo, ruotano in moto circolare uniforme.

Le onde piane non esistono, visto che le sorgenti non sono mai infinite.
Sono approssimazioni di onde sferiche per percorsi brevi rispetto alla distanza dalla sorgente.

\addfigure[Onde piane, cilindriche e sferiche che si propagano in un mezzo isotropo.]{book/onde_isotropo}

\subsection{Onda generica}

Nello spazio, l'equazione di un onda è
\begin{equation}
    A = f(\phi) = f(\vt{r} \cdot \uvt{u} - vt)
\end{equation}
$A$ è la quantità perturbata, $\vt{r}$ è il punto sul fronte d'onda, $\uvt{u}$ ha la direzione verso cui l'onda si propaga a velocità $v$.

I \important{raggi} sono le curve tangenti a $\uvt{u}$.
$\uvt{u}$ e i raggi sono ovunque normali ai fronti d'onda.

\section{Onde elettromagnetiche nel mezzo}

Se il mezzo in cui l'onda si propaga è \important{isotropo}, i raggi sono rette.

In un mezzo omogeneo isotropo (in assenza di cariche o correnti nette), è sufficiente sostituire $\mu_0$ ed $\eps_0$ con $\mu$ ed $\eps$.

La velocità dell'onda è
\begin{equation}
    v = \frac{1}{\sqrt{\mu \eps}} = \frac{c}{\sqrt{\mu_r \eps_r}}
\end{equation}

Molto spesso $\mu_r \approx 1$ (in tutti i materiali tranne quelli ferromagnetici, che però sono spesso anche conduttori).
Quindi, $\eps_r$ ha la maggiore importanza.

Si definisce l'\important{indice di rifrazione}:
\begin{equation}
    n \coloneq \sqrt{\mu_r \eps_r} \approx \sqrt{\eps_r} > 1
\end{equation}

Per cui
\begin{equation}
    v = \frac{c}{n}
\end{equation}

Considerando un'onda piana, nel mezzo variano ampiezza e lunghezza d'onda/numero d'onda, ma non pulsazione/frequenza/periodo.
\begin{gather}
    v = \frac{c}{n} \\
    k = n k_0
    \quad \iff \quad
    \lambda = \frac{\lambda_0}{n} \\
    \omega = \omega_0
    \quad \iff \quad
    T = T_0
    \quad \iff \quad
    \nu = \nu_0
\end{gather}

Per dimostrarlo, si considera una carica $q$ nell'origine con velocità $\vt{v}_p(t) = v_0 \sin(\omega t) \uy$.
In un punto sull'asse $x$, si ha che $\B(x, 0, 0) \parallel \uz$ e $\E(x, 0, 0) \parallel \uy \parallel \vt{v}$.
Questo vale sempre: il campo elettrico è parallelo al moto della carica.

Detta $v$ la velocità di propagazione dei campi, possiamo valutare l'andamento di $B$ (e quindi di $E = c B$) grazie alla legge elementare di Ampère:
\begin{equation}
    B(x, t) \propto q v_p\pts{t - \frac{x}{v}} = - q v_0 \sin(k x - \omega t)
\end{equation}
Il termine $k x$ è dovuto all'\important{effetto di ritardo}.

Si può concludere che:
\begin{itemize}
    \item La pulsazione $\omega$ del campo deriva dal moto della carica, non dal mezzo di propagazione.
    \item Il numero d'onda $k = \omega / v$ deriva dall'effetto di ritardo, e quindi dipende dal mezzo.
\end{itemize}


\section{Energia del campo elettromagnetico}

Siano $E \coloneq \norm{\E}$ e $B \coloneq \norm{\B}$.

Densità di energia totale del campo elettromagnetico è la somma delle densità di energia dei campi elettrostatico e magnetostatico (anche se non sono statici):
\begin{equation}
    w\EM = \frac{1}{2} \eps_0 E^2 + \frac{1}{2} \frac{B^2}{\mu_0}
\end{equation}
Inoltre,
\begin{equation}
    \eps_0 E^2 = \eps_0 c^2 B^2 = \frac{B^2}{\mu_0}
\end{equation}
Quindi ciascuno dei due campi contribuisce per metà energia.
Allora,
\begin{equation}
    w\EM = \eps_0 E^2 = \frac{B^2}{\mu_0}
\end{equation}

\section{Esperimento di Hertz}

Svolto nel 1888, serve a dimostrare l'esistenza delle onde elettromagnetiche.

Si collegano due sfere metalliche a un trasformatore, in modo che si carichino e scarichino ripetutamente.

Alla rottura del dielettrico (aria), avviene un arco elettrico e le sfere si scaricano.

Ponendo una spira con amperometro in un punto tra tra le sfere e una lastra di metallo posta lontano si può capire se si propagano campi.
Una lastra metallica svolge il ruolo di uno specchio riflettente per le onde elettromagnetiche.

Presso la spira, si sovrappongono onde progressive (incidenti) e regressive (riflesse):
\begin{subequations}
\begin{gather}
    E(x, t) = E_{0,i}\exp{\im(kx - \omega t)} + E_{0,r}\exp{-\im(kx + \omega t)} \\
    B(x, t) = B_{0,i}\exp{\im(kx - \omega t)} + B_{0,r}\exp{-\im(kx + \omega t)}
\end{gather}
\end{subequations}
$\omega t$ è sempre con segno meno perché l'onda torna indietro nello spazio, non nel tempo.

Sulla superficie del metallo il campo elettrico totale deve essere nullo (poiché non ha componente normale alla lastra).

Questo, con $x = 0$ (cioè sulla lastra), implica $E_{0,i} = - E_{0,r} \eqcolon E_0$.

La direzione di propagazione, che è $\parallel \E \times \B$, deve invece cambiare verso.
Quindi, se $\E$ diventa opposto, $\B$ deve rimanere quale è: $B_{0,i} = B_{0,r} \eqcolon B_0$.
\begin{subequations}
\label[pluralequation]{eq:onde_stazionarie}
\begin{gather}
    E(x, t) = E_0\exp{\im(kx - \omega t)} - E_0\exp{-\im(kx + \omega t)}
    = 2 E_0 \sin(kx) \cos(\omega t) \\
    B(x, t) = B_0\exp{\im(kx - \omega t)} + B_0\exp{-\im(kx + \omega t)}
    = 2 B_0 \cos(kx) \cos(\omega t)
\end{gather}
\end{subequations}

Usando il fatto che
\begin{equation}
\begin{gathered}
    \exp{\im (A + B)} = \cos(A + B) + \im \sin(A + B)
    = \\
    = \cos(A) \cos(B) - \sin(A) \sin(B) + \im \cos(A) \sin(B) + \im \sin(A) \cos(B)
\end{gathered}
\end{equation}

\Cref{eq:onde_stazionarie} sono \important{onde stazionarie}, quindi esistono dei nodi.
Posizionando lì la spira, se questa è sufficientemente piccola, l'amperometro non legge mai corrente.

La spira deve essere piccola rispetto alla lunghezza d'onda, quindi occorre scegliere una frequenza sufficientemente bassa.

In questo modo si misura anche la velocità della luce, poiché $\omega$ è scelta e $\lambda/2$ è la distanza tra i nodi.
Questa può essere misurata spostando la spira via da un nodo e trovando il successivo.



\section{Pressione di radiazione}

Il campo elettromagnetico trasporta non solo energia, ma anche quantità di moto.

Lo si verifica con un dinamometro a torsione, illuminando degli specchi dopo averli chiusi in una bolla di vetro per evitare altre perturbazioni.

Consideriamo un'onda piana che incide perpendicolarmente un materiale.
Un'elettrone del materiale (di carica elementare $e$) viene accelerato dal campo elettrico in verso opposto a $\E$ acquisendo una velocità $v$.
Viene poi accelerato dal campo magnetico verso l'interno del materiale.

La forza perpendicolare risultante $F_x$ genera il trasferimento della quantità di moto $p$:
\begin{gather}
    F_x = \der[p_x]{t} = e v B \\
    \der[U]{t} = F_E v = e E v = e c B v = c \der[p]{t}
    \implies \Delta U = c \Delta p
\end{gather}
$U$ è l'energia dell'elettrone e varia solo in funzione della forza elettrica poiché $\B$ non compie lavoro.

La densità di quantità di moto di un'onda elettromagnetica, quindi, è
\begin{equation}
    p\EM = \frac{w\EM}{c}
\end{equation}

Se l'onda è totalmente assorbita dal materiale, la pressione esercitata sul materiale è
\begin{equation}
    \text{pressione} = \frac{F_\perp}{A} = \frac{1}{A} \frac{\Delta p}{\Delta t}
    = \frac{1}{A} \frac{p\EM A c \Delta t}{\Delta t}
    = w\EM
\end{equation}
$A c \Delta t = A \Delta x$ è il volume che incide sulla superficie in un tempo $\Delta t$.

Se l'onda viene totalmente riflessa la pressione è doppia, poiché è doppia la variazione di quantità di moto.

\addfigure{book/pressione_di_radiazione}

Se la radiazione incide con un angolo $\theta$ rispetto alla normale, occorre moltiplicare per $\cos^2\theta$ (un coseno viene dalla forza, l'altro dal volume che incide sulla superficie nel tempo $\Delta t$).
\begin{equation}
    \text{pressione} = \frac{F \cos\theta}{A} = \frac{1}{A} \frac{\Delta p \cos \theta}{\Delta t}
    = \frac{1}{A} \frac{p\EM \pts{A c \Delta t \cos \theta} \cos \theta}{\Delta t}
    = \cos^2(\theta) w\EM
\end{equation}


\section{Vettore di Poynting}

Volume $V$ con $S = \partial V$ in presenza di campi elettrico e magnetico e con, all'interno, delle cariche $q$ in densità volumica $n$ in moto con velocità $\vt{v}$.

La densità di carica è $\rho = qn$, la densità di corrente $\dcurr = qn\vt{v}$.

Qual è la potenza che il campo elettromagnetico trasferisce alle cariche in moto?
\begin{gather}
    \de P = n q \de V \E \cdot \vt{v} = n q \vt{v} \cdot \E \,\de V = \dcurr \cdot \E \,\de V \\
    \implies P = \int_V \dcurr \cdot \E \,\de V
\end{gather}

Dalla quarta equazione di Maxwell:
\begin{equation}
    \curl \B = \mu_0 \dcurr + \mu_0 \eps_0 \parder[\E]{t}
    \implies
    \dcurr = \frac{1}{\mu_0} \curl \B - \eps_0 \parder[\E]{t}
\end{equation}

Per cui
\begin{equation}
\label{eq:potenza1}
\begin{gathered}
    P = \int_V \pts{\frac{1}{\mu_0} \curl \B - \eps_0 \parder[\E]{t}} \cdot \E \,\de V = \\
    = \frac{1}{\mu_0} \int_V (\curl \B) \cdot \E \,\de V - \eps_0 \int_V \parder[\E]{t} \cdot \E \,\de V
\end{gathered}
\end{equation}

Si usa la seguente identità:
\begin{equation}
    \diver (\E \times \B) = (\curl \E) \cdot \B - \E \cdot (\curl \B)
\end{equation}

\Cref{eq:potenza1} prosegue come:
\begin{equation}
\begin{gathered}
    P = \frac{1}{\mu_0} \int_V (\curl \E) \cdot \B \,\de V
    - \frac{1}{\mu_0} \int_V \diver (\E \times \B) \,\de V
    - \eps_0 \int_V \parder[\E]{t} \cdot \E \,\de V = \\
    = - \frac{1}{\mu_0} \int_V \parder[\B]{t} \cdot \B \,\de V
    - \frac{1}{\mu_0} \int_V \diver (\E \times \B) \,\de V
    - \eps_0 \int_V \parder[\E]{t} \cdot \E \,\de V = \\
    = - \int_V \pts{\frac{1}{2 \mu_0} \parder[\B^2]{t} + \frac{\eps_0}{2} \parder[\E^2]{t}} \de V
    - \frac{1}{\mu_0} \int_V \diver (\E \times \B) \,\de V = \\
    = - \der{t} \int_V \pts{\frac{1}{2 \mu_0} \B^2 + \frac{\eps_0}{2} \E^2} \de V
    - \oint_{\partial V} \frac{\E \times \B}{\mu_0} \sde = \\
    = -\der[U\EM]{t} - \flux{\partial V}{\poy}
\end{gathered}
\end{equation}

Nell'ultimo passaggio è stato definito il \important{vettore di Poynting}:
\begin{gather}
    \poy \coloneq \frac{\E \times \B}{\mu_0} \\
    \norm{\poy} = \frac{E B}{\mu_0} = \frac{E^2}{\mu_0 c} = \eps_0 c E^2
\end{gather}
Si tratta della potenza per unità di superficie di un'onda elettromagnetica.

Il bilancio energetico che ne risulta costituiscce il \important{teorema di Poynting}:
\begin{gather}
    -\der[U\EM]{t} = P + \flux{\partial V}{\poy}
\end{gather}
Ovvero, la perdita di energia di campo elettromagnetico (a sinistra) è dovuta a
\begin{itemize}
    \item lavoro svolto sulle cariche
    \item uscita dell'onda attraverso la superficie
\end{itemize}

Inoltre, l'energia ``stanziale'' (per unità di volume) $w\EM$ e l'energia (per unità di tempo e si superficie) espressa dal vettore di Poynting sono la stessa a meno di costanti: entrambe proporzionali al quadrato dell'ampiezza dell'onda.

Un altro modo di derivare il vettore di Poynting è il seguente: si considera un cilindro di base $A$ e lunghezza $c \Delta t$ con asse parallelo alla velocità dell'onda $\vt{c}$.

la potenza per unità di superficie che esce da una faccia del cilindro è
\begin{equation}
    \frac{1}{A} \der[U]{t} = \frac{w\EM A c \de t}{A \de t} = w\EM c = \eps_0 E^2 c
\end{equation}
Considerare questa potenza come vettore:
\begin{equation}
    \eps_0 E^2 \vt{c} = \frac{\E \times \B}{\mu_0} = \poy
\end{equation}

Il vettore di Poynting permette inoltre di definire l'\important{intensità} di un'onda elettromagnetica:
\begin{equation}
    I \coloneq \avg{\norm{\poy}}
\end{equation}
La media è considerata sul periodo dell'onda (sinusoidale).
Se l'onda elettromagnetica è visibile, si tratta proprio dell'intensità avvertita dall'occhio.

\subsection{Intensità di un'onda piana}

Onda piana con direzione di propagazione $x$ con polarizzazione rettilinea lungo $y$.
\begin{equation}
    \E(\p, t) = \E_0 \sin(\vt{k} \cdot \p - \omega t)
\end{equation}
Calcoliamo il valore medio nel tempo del vettore di Poynting:
\begin{equation}
\begin{gathered}
    \avg{\vt{S}} = \frac{1}{T} \int_0^T \vt{S} \, \de t
    = \frac{1}{T} \int_0^T \frac{E^2}{\mu_0 c} \ver{k} \, \de t = \\
    = \frac{1}{T} \int_0^T \frac{E_0^2}{\mu_0 c} \sin^2(\vt{k} \cdot \p - \omega t) \ver{k} \, \de t
    = \frac{1}{T} \frac{E_0^2}{\mu_0 c} \ver{k} \int_0^T \sin^2(\vt{k} \cdot \p - \omega t) \, \de t
\end{gathered}
\end{equation}
Sostituzione:
\begin{equation}
\begin{gathered}
    y(t) = \vt{k} \cdot \p - \omega t, \quad
    \de y = -\omega \de t = -\frac{2 \pi}{T} \de t, \quad \\
    t \in [0, \, T] \implies y \in [a, \, a - 2\pi]
    \text{ con } a \coloneq \vt{k} \cdot \p
\end{gathered}
\end{equation}
Segue
\begin{gather}
    \avg{\poy} = \frac{E_0^2}{2 \pi \mu_0 c} \ver{k} \int_{a - 2\pi}^a \sin^2(y) \, \de y
    = \frac{E_0^2}{2 \mu_0 c} \ver{k}
    = \frac{1}{2} \eps_0 c E_0^2 \ver{k}
\end{gather}

Analogamente,
\begin{equation}
    I = \avg{\norm{\poy}} = \frac{E_0^2}{2 \mu_0 c}
    = \frac{1}{2} \eps_0 c E_0^2
\end{equation}
È l'intensità di un'onda piana, \important{proporzionale al quadrato dell'ampiezza} dell'onda.


\subsection{Intensità di un'onda sferica}
\label{sec:intensita_onda_sferica}

Presso un certo punto dello spazio in presenza di un'onda sferica, l'energia per unità di tempo e di superficie è
\begin{equation}
    \norm{\poy} = \frac{1}{\mu_0 c} E^2 = \frac{1}{\mu_0 c} E_0^2 \sin^2(kr - \omega t)
\end{equation}

Se la sorgente dell'onda è puntiforme a distanza $r$ ed emette una potenza $W(t)$, la potenza che fuoriesce da una sfera di raggio $r$ dopo essere stata emessa dalla sorgente è
\begin{equation}
    4 \pi r^2 \norm{\poy} = \frac{4 \pi r^2}{\mu_0 c} E_0^2 \sin^2(kr - \omega t)
\end{equation}

Questa non è sempre uguale alla potenza emessa dalla sorgente perché c'è un ritardo temporale $r/v$.
Tuttavia, si può considerare la potenza media su un periodo:
\begin{gather}
    W_\text{avg} = \avg{W(t)} = \avg{\frac{4 \pi r^2}{\mu_0 c} E_0^2 \sin^2(kr - \omega t)}
    = 4\pi r^2 I
    = \frac{2 \pi r^2}{\mu_0 c} E_0^2 \\
    \implies E_0 = \sqrt{\frac{W_\text{avg} \mu_0 c}{2\pi}} \frac{1}{r}
\end{gather}
