\chapter{Interferenza e diffrazione}

\section{Interferenza}

A grande distanza da una sorgente di onde sferiche si pone uno schermo con $N$ fenditure allineate a reciproca distanza $d$.

Ciascuna delle fenditure agisce come sorgente puntiforme di onde sferiche.
% uno schermo parallelo alla congiungente tra le sorgenti a distanza $D$.

Poiché provengono dallo stesso fronte d'onda della stessa onda (che si approssima a onda piana data la grande distanza delle fenditure dalla sorgente), le nuove onde sferiche saranno \important{coerenti}, cioè:
\begin{itemize}
    \item Stessa polarizzazione (i campi elettrici sono tutti paralleli al piano)
    % perpendicolari alla congiungente
    \item Stessa ampiezza massima $\E_0$.
    \item Stessa pulsazione $\omega$.
    \item Stessa fase iniziale $\phi_0 = 0$.
\end{itemize}

Consideriamo i raggi $r_1, \ldots, r_N$ dalle sorgenti allo stesso punto $P$ posto su un ulteriore schermo paralleo al primo e a distanza $L \gg N d$ dalle fenditure.

I campi elettrici in $P$ sono, per $i = 1, \ldots, N$:
\begin{equation}
    \E_i(P, t) = \E_0 \exp{\im (k r_i - \omega t)}
\end{equation}

In virtù del principio di sovrapposizione degli effetti, il campo elettrico totale in $P$ è
\begin{equation}
    \E(P, t) = \E_0 \sum_{i = 1}^N \exp{\im (k r_i - \omega t)}
\end{equation}

Si definiscono le fasi
\begin{equation}
    \phi_i = k r_i - \omega t
\end{equation}
Le differenze di fase (ad esempio, tra le prime due, $\phi \coloneq \phi_2 - \phi_1 = k(r_2 - r_1)$) sono costanti nel tempo.
Inoltre, poiché le fenditure sono equispaziate, allora le differenze di fase tra onde successive sono approssimativamente uguali tra loro (e uguali a $\phi$ come appena definita).

\addsvg{efield_sum}{0.7}

I campi elettrici $\E_i(P, t)$ si possono rappresentare sul piano complesso e, visto che la differenza di fase è costante, si fissa $t$ in modo da mette $\E_1$ sull'asse reale per semplicità.

Per sommare dei complessi, disponiamo i vettori in punta-coda e consideriamo il circocentro delle punte e code.
Detto $R$ il raggio della circoscritta, si può ottenere l'ampiezza $E_R$ del campo elettrico risultante:
\begin{equation}
    E_R = 2 R \sin\pts{\frac{N \phi}{2}}, \quad E_0 = 2 R \sin\frac{\phi}{2}
    \implies E_R = E_0 \frac{\sin N \frac{\phi}{2}}{\sin \frac{\phi}{2}}
\end{equation}

L'intensità, quindi, è
\begin{equation}
    I(P) = \avg{\norm{\poy(P)}} = \frac{E_R^2}{2 \mu_0 c} = \frac{E_0^2}{2 \mu_0 c} \frac{\sin^2 N\frac{\phi}{2}}{\sin^2 \frac{\phi}{2}} = I_0 \frac{\sin^2 N\frac{\phi}{2}}{\sin^2 \frac{\phi}{2}}
\end{equation}

Inoltre, poiché $L \gg d$, definendo $\theta \in [-\pi / 2, \pi / 2]$ come l'angolo a cui si trova $P$ rispetto alla perpendicolare allo schermo,
\begin{gather}
    r_2 - r_1 \approx d \sin \theta \\
    \implies \frac{\phi}{2} = \frac{k (r_2 - r_1)}{2} = \frac{\pi}{\lambda} (r_2 - r_1) \approx \pi \frac{d}{\lambda} \sin \theta
\end{gather}

Quindi
\begin{equation}
\label{eq:reticolo}
    I(\theta) = I_0 \frac{\sin^2 \pts{\pi \frac{N d}{\lambda} \sin \theta}}{\sin^2 \pts{\pi \frac{d}{\lambda} \sin \theta}}
\end{equation}

Questa funzione quantifica quanta luce si vede sullo schermo in funzione di $\theta$.

Va notato che il tempo non compare nell'equazione, quindi questa intensità si osserva fissa sullo schermo.

Inoltre, le figure di interferenza sono distinte per ogni lunghezza d'onda proveniente dalla sorgente: le lunghezze d'onda maggiori vengono deviate di più.
Il prisma, invece, fa il contrario: devia maggiormente le lunghezze d'onda minori.

\subsection{Reticolo di interferenza}

Si parla di reticolo di interferenza quando si hanno $N$ sorgenti separate da una distanza $d$, ovvero esattamente il caso descritto dall'\cref{eq:reticolo}.

Si osservano:
\begin{itemize}
    \item Massimi principali in cui $I = N^2 I_0$ intervallati da regioni scure:
        \begin{equation}
            \sin\theta_M = \frac{\lambda}{d} m, \quad m \in \Z
        \end{equation}
        Sono i punti in cui numeratore e denominatore si annullano a dare una forma indeterminata $[0/0]$, che risulta essere un massimo.

        $m$ è detto \important{ordine del massimo}.
    \item $N - 1$ minimi in cui $I = 0$ tra ogni coppia di massimi principali:
        \begin{equation}
            \sin\theta_0 = \frac{\lambda}{N d} m', \quad
            m' \in \Z, \quad
            N \nmid m'
        \end{equation}
        Sono i punti in cui si annulla il numeratore ma non il denominatore, poiché in quei casi si ha un massimo principale.
    \item $N - 2$ massimi secondari tra ogni coppia di massimi principali:
        \begin{equation}
            \sin\theta_m \approx \frac{\lambda}{N d} \pts{m'' + \frac{1}{2}}, \quad
            m'' \in Z, \quad
            N \nmid m'', \, N \nmid m''+1
        \end{equation}
        Vanno esclusi quelli che sarebbero in prossimità di un massimo principale.
\end{itemize}

Infine,
\begin{gather}
    \abs{\sin\theta_M} < 1 \implies \abs{m} < \frac{d}{\lambda} \\
    \abs{\sin\theta_0} < 1 \implies \abs{m'} < \frac{N d}{\lambda}
\end{gather}
ovvero, esistono un numero finito di massimi e minimi.

\addfigure{book/figura_reticolo}{0.6}

\subsection{Doppia fenditura}
\label{sec:doubleslit}

Con $N = 2$ e $d$ la distanza tra le fenditure, l'\cref{eq:reticolo} si semplifica come
\begin{equation}
    I = I_0 \frac{\sin^2 \phi}{\sin^2 \frac{\phi}{2}}
    = 4 I_0 \cos^2 \frac{\phi}{2}
    = 4 I_0 \cos^2 \pts{\pi \frac{d}{\lambda} \sin\theta}
\end{equation}

Si osservano:
\begin{itemize}
    \item Massimi principali in cui $I = 4 I_0$:
        \begin{equation}
            \sin\theta_M = \frac{\lambda}{d} m, \quad m \in \Z
        \end{equation}
        Si tratta dei punti in cui $\cos^2(\cdots) = 1$.
    \item Un minimo in cui $I = 0$ tra ogni coppia di massimi principali:
        \begin{equation}
            \sin\theta_0 = \frac{\lambda}{d} \pts{m + \frac{1}{2}}, \quad
            m \in \Z
        \end{equation}
        % $m'$ non può assumere valori multipli di $N$, poiché in quei casi si ha una forma indeterminata $[0/0]$ che in realtà è un massimo.
\end{itemize}

L'\important{esperimento di Young} consiste nel mostrare la figura di interferenza a bande luminose dovuta a una doppia fenditura.
Fu svolto all'inizio dell'Ottocento sulla base della teoria di Huygens (che risaliva al Seicento), poiché diventò possibile produrre onde coerenti.
Per ottenere delle onde coerenti, il fronte d'onda proveniente da una singola fenditura veniva diviso in due fronti d'onda, che quindi erano coerenti tra loro.
In questo modo si mostrava la natura ondulatoria della luce.


\section{Singola fenditura}

Considerando una singola fenditura di ampiezza $a$ molto grande, un'onda piana che le vada contro passerebbe oltre.
Diminuendo $a$, inizialmente la zona illuminata diminuisce.
Al di sotto di una certa dimensione (quando $a$ è confrontabile della lunghezza d'onda) la fenditura si comporta come una sorgente puntiforme e illumina tutto lo schermo.

Huygens aveva osservato questo fenomeno con le onde del mare.

\important{Principio di Huygens}: ogni fronte d'onda si propaga come se ogni tratto infinitesimo del fronte d'onda fosse una sorgente puntiforme.

\addsvg{efield_sum_single_slit}{0.7}

Applicando questo principio a una piccola fenditura, ogni punto diventa sorgente di un'onda con ampiezza $\de\E$ e differenza di fase $\de\phi$ con la sorgente successiva.
Sommando i fasori come prima, la spezzata diventa un arco di circonferenza su cui insiste un angolo $\phi = k a \sin\theta$.
Sia $E_\text{max}$ la lunghezza dell'arco di circonferenza:
\begin{gather}
    E_R = 2 R \sin \frac{\phi}{2}, \quad
    E_\text{max} = R \phi
    \implies
    E_R = E_\text{max} \frac{\sin\frac{\phi}{2}}{\frac{\phi}{2}} \\
    I = I_0 \pts{\frac{\sin\pts{\pi \frac{a}{\lambda} \sin\theta}}{\pi \frac{a}{\lambda} \sin\theta}}^{\!2}
\end{gather}

Si osservano:
\begin{itemize}
    \item Minimi in cui $I = 0$:
        \begin{equation}
            \sin \theta_0 = \pm \frac{\lambda}{a} m, \quad
            m \in \Z^+
        \end{equation}
    \item Un massimo centrale in $\theta = 0$ in cui $I = I_0$.
    Questo massimo identifica una regione luminosa delimitata dai due minimi di ordine 1, cioè per
        \begin{equation}
            -\frac{\lambda}{a} < \theta < \frac{\lambda}{a}
        \end{equation}
    \item Flebili massimi tra i minimi:
        \begin{equation}
            \sin \theta_M \approx \pm \frac{\lambda}{a} \pts{m + \frac{1}{2}}, \quad
            m \in \Z^+
        \end{equation}
\end{itemize}

\addfigure[Figura di diffrazione per una fenditura di ampiezza $b$.]{book/figura_singola}{0.7}

Questo fenomeno è detto \important{diffrazione} e si verifica sempre, anche per $a / \lambda$ molto grande o molto piccolo.
\begin{itemize}
    \item Se $a \gg \lambda$ si vede solo e distintamente il massimo centrale e l'ombra intorno.
    \item Per $a \to \lambda^+$, $\theta_0 \to \pm \pi/2$, cioè la fenditura tende a illuminare tutto lo schermo.
    \item Per $a \le \lambda$, la fenditura tende a diventare una sorgente puntiforme.
\end{itemize}
Negli ultimi due casi non ci sono minimi.
Per questo, solitamente, si parla di diffrazione solo per $a > \lambda$.

La diffrazione è la ragione per cui non vediamo oggetti troppo piccoli: la pupilla funge da fenditura e la retina rileva solo la figura di diffrazione.
Esiste un limite alla capacità di risolvere i punti vicini, che si supera aumentando le dimensioni delle fenditure (siano queste pupille o lenti di telescopi).
