% !TEX root = main.tex

\documentclass[italian]{HKNdocument}
% Packages
\usepackage{listings}                    % Code highlighting
\usepackage{xcolor}                      % Custom colors
\usepackage{longtable}                   % Breakable tables
\usepackage{ulem}                        % Underline
\usepackage{contour}                     % Border around text
\usepackage{tcolorbox}                   % Custom boxes

% Primary (Accent) Colors
% Primary (Accent) Colors
\definecolor{accentYellow}{RGB}{254, 196, 41}  % #FEC421
\definecolor{accentRed}{RGB}{236, 45, 36}      % #EC2D24

% Secondary Colors
\definecolor{supportOrange}{RGB}{242, 183, 5}  % #F2B705
\definecolor{supportDarkBlue}{RGB}{55, 81, 113} % #375171

% Background Colors
\definecolor{backgroundLight}{RGB}{242, 242, 242} % #F2F2F2

% Text & Border Colors
\definecolor{textGrayBlue}{RGB}{100, 117, 140}   % #64758C
\definecolor{textGrayMedium}{RGB}{146, 154, 166}  % #929AA6
\definecolor{textGrayLight}{RGB}{184, 187, 191}   % #B8BBBF



% Listings style
\lstdefinestyle{hkn}{
  basicstyle=\ttfamily\small\color{textGrayBlue},                         % Base style (size and font)
  keywordstyle=\bfseries\color{accentRed},           % Keywords in red (important, eye-catching)
  identifierstyle=\color{supportDarkBlue},               % Identifiers in blue (clear distinction)
  commentstyle=\color{textGrayMedium},                 % Comments in gray-blue (less prominent)
  stringstyle=\color{supportOrange},                 % Strings in orange (warm and readable)
  numberstyle=\ttfamily\scriptsize\color{textGrayMedium}, % Line numbers in gray (non-intrusive)
  backgroundcolor=\color{backgroundLight},           % Light background for contrast
  rulecolor=\color{textGrayLight},                   % Soft gray border for structure
  frame=single,                          % Border around code (single, double, shadowbox, none)
  framerule=0.8pt,                       % Border thickness
  frameround=tttt,                       % Round all corners
  framesep=5pt,                          % Distance between border and code
  rulesep=2pt,                           % Distance between border and code line
  numbers=left,                          % Line number position (left, right, none)
  stepnumber=1,                          % Line number interval
  numbersep=10pt,                        % Distance between line numbers and code
  xleftmargin=30pt,                      % Left margin
  xrightmargin=30pt,                     % Right margin
  resetmargins=true,                     % Reset margins
  numberblanklines=false,                % Number blank lines
  firstnumber=auto,                      % Initial line number
  columns=fixed,                         % Fixed column width
  showstringspaces=false,                % Show spaces in strings
  tabsize=2,                             % Tab size
  breaklines=true,                       % Automatic line break for long lines
  breakatwhitespace=true,                % Line break at whitespace
  breakautoindent=true,                  % Automatic indentation after line break
  escapeinside={(*@}{@*)}                % LaTeX commands in code
}

% Underline settings
\renewcommand{\ULdepth}{1.8pt} % Underline depth
\contourlength{0.8pt}

% Custom underline command
\newcommand{\myuline}[1]{%
\uline{\phantom{#1}}%
\llap{\contour{white}{#1}}%
}

% tcolorbox color settings
\definecolor{tcolorboxLeftColor}{RGB}{2, 65, 191}
\definecolor{tcolorboxBackTitleColor}{RGB}{119, 152, 255}
\definecolor{tcolorboxBackColor}{RGB}{210, 226, 255}

% Custom boxes
\newtcolorbox[auto counter, number within=chapter]{definition}[1]{
  title={\iflanguage{italian}{Definizione}{Definition}\par~\arabic{\tcbcounter}.~#1},
  boxrule=0mm,                       % Bordo principale (disabilitato)
  leftrule=1mm,                    % Bordo sinistro principale
  arc=2mm,
  colframe=accentRed,       % Colore bordo
  colbacktitle=textGrayMedium,
  colback=backgroundLight,        % Colore sfondo
  fonttitle=\bfseries,
  rounded corners=all,               % Bordi arrotondati
  }

\newtcolorbox[auto counter, number within=chapter]{theorem}[1]{
  title={\iflanguage{italian}{Teorema}{Theorem}~\arabic{\tcbcounter}.~#1},
  boxrule=0mm,                       % Bordo principale (disabilitato)
  leftrule=1mm,                    % Bordo sinistro principale
  arc=2mm,
  colframe=accentYellow,       % Colore bordo
  colbacktitle=textGrayMedium,
  colback=backgroundLight,        % Colore sfondo
  fonttitle=\bfseries,
  rounded corners=all,               % Bordi arrotondati
}

\newtcolorbox[auto counter, number within=chapter]{corollary}[1]{
  title={\iflanguage{italian}{Corollario}{Corollary}~\arabic{\tcbcounter}.~#1},
  boxrule=0mm,                       % Bordo principale (disabilitato)
  leftrule=1mm,                    % Bordo sinistro principale
  arc=2mm,
  colframe=supportOrange,       % Colore bordo
  colbacktitle=textGrayMedium,
  colback=backgroundLight,        % Colore sfondo
  fonttitle=\bfseries,
  rounded corners=all,               % Bordi arrotondati
}

\newtcolorbox[auto counter, number within=chapter]{exercise}[1]{
  title={\iflanguage{italian}{Esercizio}{Exercise}~\arabic{\tcbcounter}.~#1},
  boxrule=0mm,                       % Bordo principale (disabilitato)
  leftrule=1mm,                    % Bordo sinistro principale
  arc=2mm,
  colframe=supportDarkBlue,       % Colore bordo
  colbacktitle=textGrayMedium,
  colback=backgroundLight,        % Colore sfondo
  fonttitle=\bfseries,
  rounded corners=all,               % Bordi arrotondati
}

\newtcolorbox[auto counter, number within=chapter]{observation}[1]{
  title={\iflanguage{italian}{Osservazione}{Observation}~\arabic{\tcbcounter}.~#1},
  boxrule=0mm,                       % Bordo principale (disabilitato)
  leftrule=1mm,                    % Bordo sinistro principale
  arc=2mm,
  colframe=textGrayBlue,       % Colore bordo
  colbacktitle=textGrayMedium,
  colback=backgroundLight,        % Colore sfondo
  fonttitle=\bfseries,
  rounded corners=all,               % Bordi arrotondati
  }


\usepackage{cancel}

\title{Quantum physics}
\author{Luca Oddenino}
\docdate{Second semester 2024/2025}
\docversion{1.0}

\begin{document}

\frontmatter
\maketitle
\thispagestyle{empty}
\def\tempa{italian}

\ifx\DocumentLanguage\tempa
    {\Large\textbf{\#WeAreHKN}}
    \vspace{0.2cm}
    \par
    Il \textbf{Mu Nu Chapter} di \textbf{IEEE-Eta Kappa Nu (HKN)} è la Honor Society del Politecnico di Torino dedicata agli studenti di eccellenza nei campi dell'ingegneria elettrica, elettronica, informatica e delle telecomunicazioni. IEEE-HKN è la più antica e prestigiosa honor society internazionale per ingegneri elettrici ed elettronici, fondata nel 1904. Il nostro capitolo promuove l'eccellenza accademica, il servizio alla comunità e lo sviluppo professionale attraverso attività di tutoring, workshop e progetti collaborativi.

    \vspace{0.2cm}
    {\large\textbf{Progetto Gratuito per la Condivisione della Conoscenza}}
    \vspace{-0.15cm}
    \par
    Questo è un progetto completamente gratuito volto alla distribuzione libera di materiale didattico di qualità. Crediamo nell'importanza di rendere accessibile a tutti una formazione di eccellenza e offriamo gratuitamente queste risorse per supportare il percorso di studio di ogni studente.

    \vspace{0.2cm}
    {\large\textbf{Scopri le nostre iniziative:}}
    \vspace{-0.15cm}
    \par
    \begin{itemize}[topsep=0pt,itemsep=0pt,parsep=0pt,partopsep=0pt]
        \item \textbf{Sito web:} \href{https://hknpolito.org/}{https://hknpolito.org/}
        \item \textbf{Instagram:} \href{https://www.instagram.com/hknpolito/}{@hknpolito}
        \item \textbf{Repository GitHub:} \href{https://github.com/MuNuChapterHKN/Eta-Kappa-Notes/}{https://github.com/MuNuChapterHKN/Eta-Kappa-Notes/}
    \end{itemize}

    \vspace{0.2cm}
    \textbf{Segnalazioni ed Errori}
    \vspace{-0.15cm}
    \par
    Hai trovato un errore o vorresti contribuire al miglioramento di questo materiale? Consulta il documento \href{https://github.com/MuNuChapterHKN/Eta-Kappa-Notes/blob/main/CONTRIBUTING.md}{CONTRIBUTING.md} nella nostra repository GitHub oppure scrivici a: \href{mailto:responsabile.tutoring@hknpolito.it}{responsabile.tutoring@hknpolito.it}

    \vfill

    \begin{center}
        \begin{minipage}{\linewidth}
            \begin{minipage}{0.6\textwidth}
                {\Large\textbf{Licenza Creative Commons}}
            \end{minipage}
            \begin{minipage}{0.39\linewidth}
                \raggedleft
                \includegraphics[width=0.65\textwidth]{\doclicense}
            \end{minipage}
        \end{minipage}
    \end{center}

    Questo lavoro è distribuito con una \textbf{Licenza Creative Commons Attribuzione - Non commerciale - Non opere derivate 4.0 International}.
    Il lavoro è stato prodotto da \DDauthors\ifx\DDeditors\empty\else, editato da \DDeditors\fi{} e pubblicizzato da \DDorganization.

    \textbf{Puoi:}
    \begin{itemize}[topsep=0pt,itemsep=0pt,parsep=0pt,partopsep=0pt]
        \item \textbf{Condividere:} copiare, distribuire e trasmettere il lavoro
    \end{itemize}

    \textbf{Alle seguenti condizioni:}
    \begin{itemize}[topsep=0pt,itemsep=0pt,parsep=0pt,partopsep=0pt]
        \item \textbf{Attribuzione:} devi attribuire il lavoro nei modi indicati dall'autore o da chi ti ha dato la licenza (ma non in un modo che suggerisca che loro approvino te o il tuo uso del lavoro).
        \item \textbf{Non commerciale:} non puoi usare questo lavoro per fini commerciali.
        \item \textbf{Non opere derivate:} se modifichi, trasformi o sviluppi questo lavoro, non puoi distribuire il risultato.
    \end{itemize}

    Maggiori informazioni sul sito di \href{https://creativecommons.org/licenses/by-nc-nd/4.0/}{Creative Commons}.
\else
    % Information about HKN organization
    {\Large\textbf{\#WeAreHKN}}
    \vspace{0.2cm}
    \par
    The \textbf{Mu Nu Chapter} of \textbf{IEEE-Eta Kappa Nu (HKN)} is the Honor Society of Politecnico di Torino dedicated to outstanding students in electrical, electronic, computer, and telecommunications engineering. IEEE-HKN is the oldest and most prestigious international honor society for electrical and electronic engineers, founded in 1904. Our chapter promotes academic excellence, community service, and professional development through tutoring activities, workshops, and collaborative projects.

    \vspace{0.3cm}
    {\large\textbf{Free Project for Knowledge Sharing}}
    \par
    This is a completely free project aimed at the open distribution of high-quality educational materials. We believe in the importance of making excellent education accessible to everyone and offer these resources free of charge to support every student's learning journey.

    \vspace{0.3cm}
    {\large\textbf{Discover our initiatives:}}
    \par
    \begin{itemize}[topsep=0pt,itemsep=0pt,parsep=0pt,partopsep=0pt]
        \item \textbf{Website:} \href{https://hknpolito.org/}{https://hknpolito.org/}
        \item \textbf{Instagram:} \href{https://www.instagram.com/hknpolito/}{@hknpolito}
        \item \textbf{GitHub Repository:} \href{https://github.com/MuNuChapterHKN/Eta-Kappa-Notes/}{https://github.com/MuNuChapterHKN/Eta-Kappa-Notes/}
    \end{itemize}

    \vspace{0.3cm}
    \textbf{Error Reports and Contributions}
    \par
    Found an error or would like to contribute to improving this material? Check the \href{https://github.com/MuNuChapterHKN/Eta-Kappa-Notes/blob/main/CONTRIBUTING.md}{CONTRIBUTING.md} document in our GitHub repository or write to us at: \href{mailto:responsabile.tutoring@hknpolito.it}{responsabile.tutoring@hknpolito.it}

    \vfill

    \begin{center}
        \begin{minipage}{\linewidth}
            \begin{minipage}{0.6\textwidth}
                {\Large\textbf{Creative Commons License}}
            \end{minipage}
            \begin{minipage}{0.39\linewidth}
                \raggedleft
                \includegraphics[width=0.65\textwidth]{\doclicense}
            \end{minipage}
        \end{minipage}
    \end{center}

    This work is licensed under a \textbf{Creative Commons Attribution - NonCommercial - NoDerivatives 4.0 International}.
    The work was produced by \DDauthors\ifx\DDeditors\empty\else, edited by \DDeditors\fi{}, and publicized by \DDorganization.

    \textbf{You are free:}
    \begin{itemize}[topsep=0pt,itemsep=0pt,parsep=0pt,partopsep=0pt]
        \item \textbf{to Share:} to copy, distribute and transmit the work
    \end{itemize}

    \textbf{Under the following conditions:}
    \begin{itemize}[topsep=0pt,itemsep=0pt,parsep=0pt,partopsep=0pt]
        \item \textbf{Attribution:} you must attribute the work in the manner specified by the author or licensor (but not in any way that suggests that they endorse you or your use of the work).
        \item \textbf{Noncommercial:} you may not use this work for commercial purposes.
        \item \textbf{No derivatives:} if you alter, transform, or build upon this work, you may not distribute the resulting work.
    \end{itemize}

    More information on the \href{https://creativecommons.org/licenses/by-nc-nd/4.0/}{Creative Commons website}.
\fi

\clearpage

\tableofcontents
\clearpage
\mainmatter

\section*{1 Introduction}
Let's begin with a concise introduction to quantum mechanics and its fundamental principles.

\subsection*{1.1 Heuristic derivation of Schrödinger's equation}
The cornerstone of quantum mechanics is Schrödinger's equation. As a fundamental postulate of quantum theory, it cannot be rigorously derived from more basic principles. However, we can develop intuition for its form through heuristic arguments that connect classical and quantum concepts.

We start with de Broglie's revolutionary hypothesis, which associates a wavelength with any particle. While classical mechanics characterizes particles by mass $m$, position $\vec{r}$, and velocity $\dot{\vec{r}}$ (allowing for deterministic predictions), quantum mechanics describes particles with wave-like properties and probabilistic behavior. These principles can be expressed as:

\[
\begin{cases}\lambda=\frac{h}{p} & \text { de Broglie's principle }  \tag{1.1}\\ \Psi(\vec{x}, t)=A \mathrm{e}^{i(k \vec{x}-\omega t)} & \text { wave function }\end{cases}
\]

Here, Planck's constant $h \simeq 6.602 \cdot 10^{-34} \mathrm{~J} \cdot \mathrm{s}$ appears as a fundamental quantum parameter. Since we're treating particles as waves, a natural starting point might be the classical d'Alembert wave equation:

\begin{equation*}
\frac{\partial^{2} \Psi}{\partial t^{2}}=\sigma \nabla^{2} \Psi \tag{1.2}
\end{equation*}

Substituting our wave function from (1.1) yields:

\begin{align*}
-\omega^{2} \Psi & =-\sigma k^{2} \Psi \\
\omega^{2} & =\sigma k^{2} \\
\omega^{2} h^{2} & =\sigma\left(\frac{2 \pi}{\lambda}\right)^{2} h^{2}  \tag{1.3}\\
\omega^{2} \hbar^{2} & =\sigma\left(\frac{h}{\lambda}\right)^{2} \\
E^{2} & =\sigma p^{2}
\end{align*}


The proportionality relation $E^2 = \sigma p^2$ derived from the d'Alembert equation is fundamentally incompatible with the nonrelativistic energy-momentum relation we seek. This discrepancy reveals a crucial insight: the quantum wave equation governing nonrelativistic particles must differ in structure from classical wave equations.

The d'Alembert equation's second-order time derivative produces a relativistic-like dispersion relation where energy scales with momentum (rather than its square), appropriate for massless particles like photons ($E = pc$) but unsuitable for massive particles at nonrelativistic speeds. This mathematical inconsistency was one of the profound challenges faced by early quantum theorists, including Schrödinger himself in 1925-1926.

Let's pursue an alternative approach by considering a first-order time derivative:

\begin{equation*}
-i \frac{\partial \Psi}{\partial t}=\sigma \nabla^{2} \Psi \tag{1.4}
\end{equation*}

The complex factor $-i$ is crucial and represents a fundamental departure from classical wave equations. Substituting our wave function:

\begin{align*}
-i(-i\omega) \Psi & =-\sigma k^{2} \Psi \\
\omega \Psi & =-\sigma k^{2} \Psi \\
\omega & =\sigma k^{2} \\
\omega h & =\sigma\left(\frac{2 \pi}{\lambda}\right)^{2} h  \tag{1.5}\\
\omega \hbar & =\sigma\left(\frac{h}{\lambda}\right)^{2} \\
E & =\sigma \frac{p^{2}}{\hbar}
\end{align*}

This relation aligns with classical mechanics when we set:

\begin{equation*}
\sigma=\frac{\hbar}{2 m} \Rightarrow E=\frac{p^{2}}{2 m} \tag{1.6}
\end{equation*}

This nonrelativistic kinetic energy expression ($E=\frac{p^2}{2m}$) correctly describes particles like electrons under typical laboratory conditions where $v \ll c$. For reference, an electron accelerated through 100 eV reaches approximately 0.02c, where relativistic effects remain negligible (contributing corrections of only about 0.02%).

Rearranging equation (1.4) with our value of $\sigma$, we obtain the time-dependent Schrödinger equation for a free particle:

\begin{align*}
i \hbar \frac{\partial \Psi}{\partial t} & =-\frac{\hbar^{2}}{2 m} \nabla^{2} \Psi \quad \text{(3-D equation)} \\
i \hbar \frac{\partial \Psi}{\partial t} & =-\frac{\hbar^{2}}{2 m} \frac{\partial^{2} \Psi}{\partial x^{2}} \quad \text{(1-D equation)} \tag{1.7}
\end{align*}

This equation was formulated by Erwin Schrödinger in 1926 and represents one of the most significant achievements in theoretical physics. Unlike Newton's deterministic equations, Schrödinger's equation describes the evolution of probability amplitudes, fundamentally changing our understanding of physical reality.

Two critical observations about these equations reveal their physical significance:

\begin{enumerate}
  \item The spatial differential operator corresponds directly to the kinetic energy operator. For a plane wave with wavevector components $k_1, k_2, k_3$:
\end{enumerate}

\begin{align*}
-\frac{\hbar^{2}}{2 m} \nabla^{2} \Psi & =-\frac{\hbar^{2}}{2 m}\left(k_{1}^{2}+k_{2}^{2}+k_{3}^{2}\right) \Psi  \tag{1.8}\\
& =\frac{p^{2}}{2 m} \Psi=E \Psi
\end{align*}

This demonstrates how the Laplacian operator extracts kinetic energy information from the wave function, providing a direct link between the mathematical formalism and physical observables. The eigenvalue $E$ represents the energy measured in experiments, with discrete values emerging naturally for bound systems—explaining atomic spectra that had puzzled classical physicists.


\begin{enumerate}
  \setcounter{enumi}{1}
  \item The gradient operator, when properly scaled, yields the momentum operator—another fundamental connection between wave properties and particle attributes:
\end{enumerate}

\begin{equation*}
-i \hbar \vec{\nabla} \Psi=-\hbar \vec{k} \Psi=\vec{p} \Psi \tag{1.9}
\end{equation*}

These mathematical entities—the Laplacian and gradient operators—represent quantum mechanical operators that extract physical information from the wave function. Operators are transformations that act on wave functions to yield observable quantities, forming the mathematical backbone of quantum theory.

The relationship between these operators mirrors classical physics in a profound way. In Newtonian mechanics, the kinetic energy and momentum of a particle are related by:

\begin{equation*}
E=\frac{p^{2}}{2 m} \tag{1.10}
\end{equation*}

When we substitute the quantum operators, we find this relationship is preserved:

\begin{equation*}
\frac{1}{2 m}(-i \hbar \vec{\nabla}) \cdot(-i \hbar \vec{\nabla})=-\frac{\hbar^{2}}{2 m} \vec{\nabla} \cdot \vec{\nabla}=-\frac{\hbar^{2}}{2 m} \nabla^{2} \tag{1.11}
\end{equation*}

This calculation reveals how the squared momentum operator transforms into the Laplacian, demonstrating the mathematical correspondence principle that connects classical and quantum formulations. The appearance of $\hbar$ (Planck's constant divided by $2\pi$, approximately $1.055 \times 10^{-34}$ J s) in these expressions marks the scale at which quantum effects become significant.

In classical Hamiltonian mechanics, the total energy includes both kinetic and potential contributions:

\begin{equation*}
H=\frac{p^{2}}{2 m}+U \tag{1.12}
\end{equation*}

This structure carries over directly to quantum mechanics through the operator formalism. The correspondence allows us to define two fundamental quantum operators:

\begin{itemize}
  \item \textbf{Momentum operator:} The quantum analog of classical momentum
\end{itemize}

\begin{equation*}
\hat{p}=-i \hbar \vec{\nabla} \tag{1.13}
\end{equation*}

This operator explains the wave-particle duality first proposed by de Broglie. When applied to a wave function, it extracts information about the particle's momentum. The operator's eigenvalues represent the possible momentum values that could be measured in an experiment.

\begin{itemize}
  \item \textbf{Hamiltonian (energy operator):} The quantum analog of total energy
\end{itemize}

\begin{equation*}
\hat{H}=-\frac{\hbar^{2}}{2 m} \nabla^{2}+U(\vec{x}) \tag{1.14}
\end{equation*}

The Hamiltonian operator encapsulates both kinetic energy (through the Laplacian term) and potential energy (through the function $U(\vec{x})$). This operator generates time evolution and determines energy eigenvalues, playing a central role in quantum dynamics comparable to that of the Hamiltonian function in classical mechanics.

With these operators defined, we can express the complete form of the time-dependent Schrödinger equation:

\section*{Schrödinger's equation:}

\begin{equation*}
i \hbar \frac{\partial \Psi(\vec{x}, t)}{\partial t}=-\frac{\hbar^{2}}{2 m} \nabla^{2} \Psi(\vec{x}, t)+U(\vec{x}) \Psi(\vec{x}, t) \tag{1.15}
\end{equation*}

This equation, published by Erwin Schrödinger in 1926, represents one of the most significant achievements in theoretical physics of the 20th century. It earned Schrödinger the Nobel Prize in Physics in 1933 (shared with Paul Dirac) and has been experimentally validated across countless phenomena, from atomic spectra to quantum tunneling.

For one-dimensional systems, the equation simplifies to:

\begin{equation*}
i \hbar \frac{\partial \Psi(x, t)}{\partial t}=-\frac{\hbar^{2}}{2 m} \frac{\partial^{2} \Psi(x, t)}{\partial x^{2}}+U(x) \Psi(x, t) \tag{1.16}
\end{equation*}

One-dimensional models provide valuable insights into quantum behavior while remaining mathematically tractable. Systems like the particle in a box, harmonic oscillator, and potential barriers are often first analyzed in one dimension before extending to more complex geometries.

Using operator notation, the equation takes an elegant, compact form:

\begin{equation*}
i \hbar \frac{\partial \Psi}{\partial t}=\hat{H} \Psi \tag{1.17}
\end{equation*}

This formulation highlights that the Hamiltonian operator generates time evolution of the wave function, analogous to how the Hamiltonian function generates time evolution in classical mechanics through canonical equations. The equation is a partial differential equation of first order in time and second order in space, requiring both initial conditions and boundary conditions for unique solutions.

For multi-particle systems with masses $m_{1}, m_{2}, \ldots, m_{n}$ at positions $\vec{r}_{1}, \vec{r}_{2}, \ldots, \vec{r}_{n}$, the Hamiltonian becomes considerably more complex:

\begin{equation*}
\hat{H}=\sum_{i=1}^{n} \frac{\hat{p}_i^{2}}{2 m_{i}}+\sum_{i=1}^{n} U_{i}\left(\vec{r}_{i}\right)+\sum_{i=1}^{n} \sum_{j \neq i}^{n} U_{i j}\left(\left|\vec{r}_{i}-\vec{r}_{j}\right|\right) \tag{1.18}
\end{equation*}

This expression includes three distinct contributions:
1. The kinetic energy terms for each particle
2. External potential energy terms acting on individual particles
3. Interaction potential terms between particle pairs

The resulting many-body Schrödinger equation poses formidable mathematical challenges. For a system of $n$ particles in three dimensions, the wave function depends on $3n$ spatial coordinates, making exact solutions computationally intractable for more than a few particles. This "exponential wall" necessitates approximation methods like the Hartree-Fock approach, density functional theory, and various numerical techniques that form the basis of computational quantum chemistry and condensed matter physics.

\subsection*{1.2 Superposition}
The superposition principle represents another fundamental aspect of quantum mechanics, arising directly from the linearity of the Schrödinger equation. This principle has no classical analog and leads to many counterintuitive quantum phenomena.

When we solve Schrödinger's equation for a particular system, we often obtain multiple wavefunctions $\Psi_{j}$ (where $j=1,2, \ldots, n$) that satisfy the equation. The superposition principle states that any linear combination of these solutions also represents a valid quantum state. This follows mathematically from the linearity of the differential operators in the Schrödinger equation.

\textbf{Proof.} Start by defining a superposition of two solutions:

\begin{equation*}
\Psi=c_{1} \Psi_{1}+c_{2} \Psi_{2} \tag{1.19}
\end{equation*}

where $c_1$ and $c_2$ are complex coefficients. To verify that $\Psi$ is also a solution to Schrödinger's equation, we substitute it directly:
I apologize for the confusion. Let me provide the content in English:


\begin{equation*}
\Psi=c_{1} \Psi_{1}+c_{2} \Psi_{2} \tag{1.19}
\end{equation*}

Substituting this superposition into Schrödinger's equation:

\begin{align*}
i \hbar \frac{\partial \Psi}{\partial t} &= i \hbar \frac{\partial}{\partial t}\left(c_{1} \Psi_{1}+c_{2} \Psi_{2}\right) \\
&= i \hbar\left(c_{1} \frac{\partial \Psi_{1}}{\partial t}+c_{2} \frac{\partial \Psi_{2}}{\partial t}\right) \tag{1.20}
\end{align*}

Since $\Psi_1$ and $\Psi_2$ are solutions to Schrödinger's equation, we know that:
$i \hbar \frac{\partial \Psi_1}{\partial t} = \hat{H}\Psi_1$ and $i \hbar \frac{\partial \Psi_2}{\partial t} = \hat{H}\Psi_2$

Therefore:
\begin{align*}
i \hbar \frac{\partial \Psi}{\partial t} &= i \hbar\left(c_{1} \frac{\partial \Psi_{1}}{\partial t}+c_{2} \frac{\partial \Psi_{2}}{\partial t}\right) \\
&= c_{1} \hat{H} \Psi_{1}+c_{2} \hat{H} \Psi_{2} \\
&= \hat{H}\left(c_{1} \Psi_{1}+c_{2} \Psi_{2}\right) \\
&= \hat{H} \Psi
\end{align*}

This proves that $\Psi$ satisfies Schrödinger's equation. The linearity of the Hamiltonian operator allows us to distribute it across the terms, confirming that any linear combination of solutions is also a solution.

By induction, this extends to any number of solutions:

\begin{equation*}
\Psi=\sum_{j=1}^{n} c_{j} \Psi_{j} \text{ is a solution } \Longleftrightarrow \text{ every } \Psi_{j} \text{ is a solution } \tag{1.21}
\end{equation*}

The superposition principle has profound implications, explaining quantum interference and the behavior seen in the double-slit experiment.

\subsection*{1.3 Quantum representation of dynamics}
What is the meaning of the wave function?

In classical physics, dynamics follows a deterministic framework. When we solve Hamilton's equations:

\[
\left\{\begin{array}{l}
\frac{\partial \mathcal{H}}{\partial q_{\alpha}}=-\dot{p}_{\alpha}  \tag{1.22}\\
\frac{\partial \mathcal{H}}{\partial p_{\alpha}}=\dot{q}_{\alpha}
\end{array}\right.
\]

We can precisely predict the system's evolution at any time. Given initial conditions, the future trajectory is entirely determined.

Quantum mechanics represents a radical departure from determinism. The wave function introduces an inherently probabilistic description of physical systems, fundamentally changing our understanding of nature at the microscopic level.

The probabilistic interpretation, known as the Copenhagen interpretation, was developed in Copenhagen during the formative years of quantum theory. While alternative interpretations exist, the Copenhagen approach remains standard due to its consistent agreement with experiments.

In this framework, the wave function $\Psi(\vec{x}, t)$ represents the complete state of the system. From it, we define the probability density $\rho$:

\begin{equation*}
\rho:=|\Psi(\vec{x}, t)|^{2} = \Psi^*(\vec{x}, t)\Psi(\vec{x}, t) \tag{1.23}
\end{equation*}

For an infinitesimal volume $\mathrm{d}V$ within a region, the probability of finding the particle in that volume element is:

\begin{equation*}
\mathrm{d}P=\rho\,\mathrm{d}V \tag{1.24}
\end{equation*}

This probabilistic interpretation was initially controversial but has withstood extensive experimental verification, including tests of Bell's inequalities that support the quantum mechanical description over local hidden variable theories.

For a finite volume $V_{0}$, the probability of finding the particle within that region is calculated by integrating the probability density over the entire volume:

\begin{equation*}
P\left(V_{0}\right)=\int_{V_{0}} \rho \mathrm{~d} V \tag{1.25}
\end{equation*}

This integral represents a key connection between the mathematical formalism and experimental measurements. When we perform position measurements on identically prepared quantum systems, the statistical distribution of results approaches this probability distribution in the limit of many measurements.

If the particle is confined within a volume $V$ (for example, in a potential well with infinite barriers), then the particle must be found somewhere within this volume. This leads to the normalization condition:

\begin{equation*}
\int_{V} \rho \mathrm{~d} V \stackrel{!}{=} 1 \tag{1.26}
\end{equation*}

The exclamation mark above the equal sign emphasizes that this is a constraint on physically acceptable wave functions rather than a derived result. Wave functions that don't satisfy this normalization condition don't represent physically realizable quantum states. This requirement guides the selection of appropriate solutions to Schrödinger's equation when boundary conditions allow for multiple mathematical solutions.

A remarkable feature of quantum dynamics is that while the wave function evolves in time according to Schrödinger's equation, the total probability remains constant—a property known as probability conservation. This reflects the physical principle that particles are neither created nor destroyed during quantum evolution.

To prove this conservation property, we start with the explicit definition of $\rho$:

\begin{equation*}
\rho=|\Psi|^{2}=\Psi \Psi^{*} \tag{1.27}
\end{equation*}

Taking the time derivative:

\begin{equation*}
\frac{\partial \rho}{\partial t}=\frac{\partial}{\partial t}(\Psi \Psi^{*})=\Psi \frac{\partial \Psi^{*}}{\partial t}+\Psi^{*} \frac{\partial \Psi}{\partial t} \tag{1.28}
\end{equation*}

This equation expresses how the probability density changes over time. To establish conservation, we need to show that the total integrated probability remains constant despite local changes in the distribution.


Exploiting Schrödinger's equation and its complex conjugate:

\begin{align*}
i \hbar \frac{\partial \Psi}{\partial t} &= -\frac{\hbar^{2}}{2 m} \nabla^{2} \Psi+U(\vec{x}) \Psi \\
-i \hbar \frac{\partial \Psi^{*}}{\partial t} &= -\frac{\hbar^{2}}{2 m} \nabla^{2} \Psi^{*}+U(\vec{x}) \Psi^{*} \tag{1.29}
\end{align*}

The second equation can be rewritten as:
\begin{align*}
i \hbar \frac{\partial \Psi^{*}}{\partial t} &= \frac{\hbar^{2}}{2 m} \nabla^{2} \Psi^{*}-U(\vec{x}) \Psi^{*}
\end{align*}

Substituting these expressions into equation (1.28):

\begin{align*}
\frac{\partial \rho}{\partial t} &= \Psi \frac{\partial \Psi^{*}}{\partial t}+\Psi^{*} \frac{\partial \Psi}{\partial t} \\
&= \Psi\left(\frac{-i\hbar^{2}}{2m i\hbar} \nabla^{2} \Psi^{*}+\frac{iU(\vec{x})}{i\hbar} \Psi^{*}\right)+\Psi^{*}\left(\frac{-i\hbar^{2}}{i\hbar 2 m} \nabla^{2} \Psi+\frac{iU(\vec{x})}{i\hbar} \Psi\right) \tag{1.30}\\
&= \frac{\hbar^{2}}{2m i\hbar}\left(\Psi \nabla^{2} \Psi^{*}-\Psi^{*} \nabla^{2} \Psi\right)
\end{align*}

The potential terms cancel out exactly, revealing that probability conservation holds even in the presence of external forces. This is a profound result showing that quantum evolution preserves probability regardless of the specific potential.

Using a vector calculus identity for the Laplacian terms:

\begin{equation*}
\Psi \nabla^{2} \Psi^{*}-\Psi^{*} \nabla^{2} \Psi=\vec{\nabla} \cdot\left(\Psi \vec{\nabla} \Psi^{*}-\Psi^{*} \vec{\nabla} \Psi\right) \tag{1.31}
\end{equation*}

This identity transforms the difference of Laplacians into a divergence term, allowing us to express the time evolution in terms of a flow.

Substituting this result:

\begin{equation*}
\frac{\partial \rho}{\partial t}=\frac{1}{i\hbar}\frac{\hbar^{2}}{2 m} \vec{\nabla} \cdot\left(\Psi \vec{\nabla} \Psi^{*}-\Psi^{*} \vec{\nabla} \Psi\right) \tag{1.32}
\end{equation*}

We can define a vector quantity representing the flow of probability:

\begin{equation*}
\vec{j}:=\frac{\hbar}{2 m i}\left(\Psi^{*} \vec{\nabla} \Psi-\Psi \vec{\nabla} \Psi^{*}\right) \tag{1.33}
\end{equation*}

This quantity $\vec{j}$ is the probability current density, analogous to fluid flow in classical mechanics. It represents how probability "flows" through space as the wave function evolves. The direction of $\vec{j}$ indicates the direction of probability flow, while its magnitude indicates the rate of flow.

With this definition, equation (1.32) takes the elegant form of a continuity equation:

\begin{equation*}
\frac{\partial \rho}{\partial t}=-\vec{\nabla} \cdot \vec{j} \tag{1.34}
\end{equation*}

This continuity equation expresses a fundamental conservation law: the rate at which probability density changes at any point equals the negative divergence of the probability current. In other words, probability can neither be created nor destroyed—it can only flow from one region to another.


If we substitute the general expression for a complex wave function $\Psi=|\Psi| \mathrm{e}^{i \theta}$ into our expression for the probability current, we can gain further insight into its physical meaning:

\begin{align*}
\vec{j} &= \frac{\hbar}{2 m i}\left(\Psi^{*} \vec{\nabla} \Psi-\Psi \vec{\nabla} \Psi^{*}\right) \\
&= \frac{\hbar}{2 m i}\left(|\Psi| \mathrm{e}^{-i \theta} \vec{\nabla}\left(|\Psi| \mathrm{e}^{i \theta}\right)-|\Psi| \mathrm{e}^{i \theta} \vec{\nabla}\left(|\Psi| \mathrm{e}^{-i \theta}\right)\right) \\
&= \frac{\hbar}{2 m i}(|\Psi| \mathrm{e}^{-i \theta}(\vec{\nabla}|\Psi|+i|\Psi| \vec{\nabla} \theta)-\underbrace{|\Psi| \mathrm{e}^{i \theta}(\vec{\nabla}|\Psi|-i|\Psi| \vec{\nabla} \theta)}_{\text {complex conjugate }}) \tag{1.35} \\
\end{align*}

The gradient terms involving $|\Psi|$ cancel out when we complete the calculation:

\begin{align*}
\vec{j} &= \frac{\hbar}{2 m i}\left(2 i|\Psi|^{2} \vec{\nabla} \theta\right) \\
&= \frac{\hbar}{m}|\Psi|^{2} \vec{\nabla} \theta
\end{align*}

This result reveals a profound connection between quantum mechanics and fluid dynamics. The term $\vec{v}=\frac{\hbar}{m} \vec{\nabla} \theta$ represents a velocity field associated with the flow of probability, while $|\Psi|^2 = \rho$ is the probability density. Thus:

\begin{equation*}
\vec{j}=\rho \vec{v} \tag{1.36}
\end{equation*}

This form mirrors the mass current density in fluid mechanics, suggesting that quantum probability flows like a fluid, with the phase gradient of the wave function determining the velocity field. This hydrodynamic analogy was explored by Madelung in 1926 and forms the basis of the quantum hydrodynamic interpretation.

We can now prove that the total probability $P$ remains constant over time.

Proof. The time derivative of the total probability in volume $V$ is:

\begin{align*}
\frac{\partial}{\partial t} \int_{V} \rho \mathrm{~d}^{3} \vec{x} &= \int_{V} \frac{\partial \rho}{\partial t} \mathrm{~d}^{3} \vec{x} \tag{1.37} \\
&= -\int_{V} \vec{\nabla} \cdot \vec{j} \mathrm{~d}^{3} \vec{x}
\end{align*}

Applying the divergence theorem (Gauss's theorem), this becomes:

\begin{align*}
\frac{\partial}{\partial t} \int_{V} \rho \mathrm{~d}^{3} \vec{x} &= -\oint_{\partial V} \vec{j} \cdot \mathrm{d} \vec{\Sigma} = 0
\end{align*}

The surface integral vanishes because for physically relevant wave functions, $|\Psi|^2$ approaches zero at the boundaries of the volume (for bound states) or falls off rapidly enough (for scattering states). Consequently, $\vec{j}$ also vanishes at the boundary, making the surface integral zero.

This proves that the total probability within the volume remains constant during time evolution—a fundamental property ensuring that quantum mechanics preserves the probabilistic interpretation.

For proper normalization, we require the total probability to be exactly 1. If a wave function yields:

\begin{equation*}
\int_{V} \mathrm{~d}^{3} x|\Psi|^{2}=c \tag{1.38}
\end{equation*}

where $c$ is some positive constant, we normalize by scaling the wave function:
$\Psi \rightarrow \frac{\Psi}{\sqrt{c}}$

This ensures:

\begin{equation*}
\int_{V} \mathrm{~d}^{3} x\left|\frac{\Psi}{\sqrt{c}}\right|^{2}=1 \tag{1.39}
\end{equation*}

The normalization procedure is essential for maintaining the probabilistic interpretation, allowing us to interpret $|\Psi|^2$ as a proper probability density.

For infinite volumes where $V=\mathbb{R}^{3}$, we still require the wave function to be normalizable, meaning it must decay sufficiently rapidly at infinity. This constraint excludes certain mathematical solutions to Schrödinger's equation that, while mathematically valid, don't represent physically realizable quantum states.


For a wave function that asymptotically behaves as:

\begin{equation*}
\Psi \simeq \frac{f(r)}{r^{k}} \quad \text { with } \quad|f(r)|<C \tag{1.40}
\end{equation*}

where $C$ is some positive constant, we need to determine conditions on $k$ that ensure normalizability. Substituting this expression into the probability current:

\begin{align*}
\vec{j} &= \frac{\hbar}{2 m i}\left(\frac{f^{*}(r)}{r^{k}} \vec{\nabla} \frac{f(r)}{r^{k}}-\frac{f(r)}{r^{k}} \vec{\nabla} \frac{f^{*}(r)}{r^{k}}\right)
\end{align*}

Applying the product rule to evaluate the gradients:

\begin{align*}
\vec{j} &= \frac{\hbar}{2 m i}\left(\frac{f^{*}(r)}{r^{k}}\left(\frac{\vec{\nabla} f(r)}{r^{k}}-\frac{k f(r)}{r^{k+1}}\hat{r}\right)-\frac{f(r)}{r^{k}}\left(\frac{\vec{\nabla} f^{*}(r)}{r^{k}}-\frac{k f(r)^{*}}{r^{k+1}}\hat{r}\right)\right) \\
&= \frac{\hbar}{2 m i}\left(\frac{f^{*}(r) \vec{\nabla} f(r)}{r^{2 k}}-\frac{f^{*}(r) k f(r)}{r^{2 k+1}}\hat{r}-\frac{f(r) \vec{\nabla} f^{*}(r)}{r^{2 k}}+\frac{f(r) k f^{*}(r)}{r^{2 k+1}}\hat{r}\right) \tag{1.41}
\end{align*}

The terms containing $\hat{r}$ cancel, leaving:

\begin{align*}
\vec{j} &= \frac{\hbar}{2 m i}\left(\frac{f^{*}(r) \vec{\nabla} f(r)-f(r) \vec{\nabla} f^{*}(r)}{r^{2 k}}\right)
\end{align*}

To analyze whether the total probability is conserved in an infinite volume, we examine the flux through a large sphere surrounding the origin:

\begin{align*}
\frac{\partial}{\partial t} \int_{\mathbb{R}^{3}} \rho \mathrm{~d}^{3} \vec{x} &= -\oint_{\partial V} \vec{j} \cdot \mathrm{d} \vec{\Sigma} \\
&= -\oint_{S} \vec{j} \cdot \hat{r} \, r^{2} \sin\theta \, d\theta \, d\phi
\end{align*}

Substituting our expression for $\vec{j}$:

\begin{align*}
\frac{\partial}{\partial t} \int_{\mathbb{R}^{3}} \rho \mathrm{~d}^{3} \vec{x} &= -\oint_{S}\left[\frac{\hbar}{2 m i}\left(\frac{f^{*}(r) \vec{\nabla} f(r)-f(r) \vec{\nabla} f^{*}(r)}{r^{2 k}}\right) \cdot \hat{r} \, r^{2}\right] \sin\theta \, d\theta \, d\phi \tag{1.42} \\
&= -\oint_{S} \frac{\hbar}{2 m i}\left(\frac{f^{*}(r) \frac{\partial f(r)}{\partial r}-f(r) \frac{\partial f^{*}(r)}{\partial r}}{r^{2 k-2}}\right) \sin\theta \, d\theta \, d\phi
\end{align*}

For this integral to vanish as $r \to \infty$, the integrand must decay fast enough. Since $|f(r)|$ is bounded, the behavior is dominated by the $r^{2k-2}$ term in the denominator. The probability current density vanishes at infinity if $k > 1$, ensuring no probability escapes to infinity.

However, for the wave function to be normalizable, we need:
\begin{align*}
\int_{\mathbb{R}^{3}} |\Psi|^2 \, d^3x &= \int_0^\infty \int_0^\pi \int_0^{2\pi} \left|\frac{f(r)}{r^k}\right|^2 r^2 \sin\theta \, d\phi \, d\theta \, dr \\
&= 4\pi \int_0^\infty \frac{|f(r)|^2}{r^{2k}} r^2 \, dr \\
&= 4\pi \int_0^\infty \frac{|f(r)|^2}{r^{2k-2}} \, dr
\end{align*}

This integral converges only if $2k-2 > 1$, which means $k > \frac{3}{2}$. Since this condition is stronger than $k > 1$, we need $k > \frac{3}{2}$ to satisfy both requirements.

\subsection*{1.4 Physical operators}
In quantum mechanics, mathematical operators represent physical observables. These operators can be classified into three main categories based on how they act on wave functions:

\begin{enumerate}
  \item Multiplication operators: These act by simply multiplying the wave function by a function of coordinates. Key examples include:
\end{enumerate}

\begin{itemize}
  \item The position operator, derived from the position vector $\vec{x}=x_{1} \hat{\vec{u}}_{1}+x_{2} \hat{\vec{u}}_{2}+x_{3} \hat{\vec{u}}_{3}$, becomes $\hat{x}=\hat{x_{1}} \hat{\vec{u}}_{1}+\hat{x_{2}} \hat{\vec{u}}_{2}+\hat{x_{3}} \hat{\vec{u}}_{3}$ where each component acts as $\hat{x}_i\Psi(\vec{x}) = x_i\Psi(\vec{x})$
  \item Any function of position $f(\vec{x})$ becomes a multiplicative operator $\hat{f}$ acting as $\hat{f}\Psi(\vec{x}) = f(\vec{x})\Psi(\vec{x})$
  \item The potential energy operator $\hat{U}(\vec{x})$ acts by multiplication: $\hat{U}(\vec{x})\Psi(\vec{x}) = U(\vec{x})\Psi(\vec{x})$
\end{itemize}

\begin{enumerate}
  \setcounter{enumi}{1}
  \item Differential operators: These involve derivatives of various orders. Important examples include:
\end{enumerate}

\begin{itemize}
  \item The momentum operator $\hat{p}_{j}=-i \hbar \frac{\partial}{\partial x_{j}}$, which acts on wave functions through differentiation
  \item The kinetic energy operator $\hat{T} = \frac{\hat{p}^2}{2m} = -\frac{\hbar^2}{2m}\nabla^2$
\end{itemize}


\begin{enumerate}
  \setcounter{enumi}{2}
  \item Mixed operators: These combine both multiplication and differentiation operations. Key examples include:
\end{enumerate}

\begin{itemize}
  \item The Hamiltonian operator $\hat{H}=-\frac{\hbar^{2}}{2 m} \nabla^{2}+U(\vec{x})$, representing the total energy
  \item The angular momentum operator $\hat{\vec{L}}=\hat{\vec{x}} \times \hat{\vec{p}}$, which describes rotational motion
\end{itemize}

To work with vector operators like angular momentum, we introduce the Levi-Civita symbol (also known as the permutation symbol or antisymmetric tensor):

\[
\epsilon_{i j k}:= \begin{cases}
1 & \text{if $(i,j,k)$ is an even permutation of $(1,2,3)$} \tag{1.43}\\
-1 & \text{if $(i,j,k)$ is an odd permutation of $(1,2,3)$} \\
0 & \text{if any index is repeated}
\end{cases}
\]

This means $\epsilon_{123} = \epsilon_{231} = \epsilon_{312} = 1$ and $\epsilon_{132} = \epsilon_{321} = \epsilon_{213} = -1$, while all other combinations with repeated indices are zero.

We adopt Einstein's summation convention where repeated indices automatically imply summation over that index. For example:

\begin{equation*}
\epsilon_{i j k} a_{i}=\sum_{i=1}^3 \epsilon_{i j k} a_{i} \tag{1.44}
\end{equation*}

The Levi-Civita symbol possesses several remarkable properties. One particularly useful identity relates products of two Levi-Civita symbols to determinants of Kronecker delta functions:

\[
\epsilon_{i j k} \epsilon_{l m n}=\begin{vmatrix}
\delta_{i l} & \delta_{i m} & \delta_{i n} \tag{1.45}\\
\delta_{j l} & \delta_{j m} & \delta_{j n} \\
\delta_{k l} & \delta_{k m} & \delta_{k n}
\end{vmatrix}
\]

This identity proves extremely useful in tensor manipulations and simplifications of vector calculus expressions.

Another important property connects the Levi-Civita symbol to the cross product of vectors:

\[
\epsilon_{i j k} \hat{\vec{u}}_{i} a_{j} b_{k}=\vec{a} \times \vec{b}=\begin{vmatrix}
\hat{\vec{u}}_{1} & \hat{\vec{u}}_{2} & \hat{\vec{u}}_{3} \tag{1.46}\\
a_{1} & a_{2} & a_{3} \\
b_{1} & b_{2} & b_{3}
\end{vmatrix}
\]

This provides a compact tensor notation for cross products, which is particularly valuable when working with angular momentum operators.

To verify this identity, we can expand the summation explicitly:


\begin{align*}
\epsilon_{i j k} \hat{\vec{u}}_{i} a_{j} b_{k} &= \sum_{i=1}^{3} \sum_{j=1}^{3} \sum_{k=1}^{3} \epsilon_{i j k} \hat{\vec{u}}_{i} a_{j} b_{k} \\
&= \hat{\vec{u}}_{1} \sum_{j=1}^{3} \sum_{k=1}^{3} \epsilon_{1 j k} a_{j} b_{k} + \hat{\vec{u}}_{2} \sum_{j=1}^{3} \sum_{k=1}^{3} \epsilon_{2 j k} a_{j} b_{k} + \hat{\vec{u}}_{3} \sum_{j=1}^{3} \sum_{k=1}^{3} \epsilon_{3 j k} a_{j} b_{k} \tag{1.47}
\end{align*}

Examining the first term in detail, we note that $\epsilon_{1jk}$ is non-zero only when $(j,k)$ is either $(2,3)$ or $(3,2)$:

\begin{align*}
\hat{\vec{u}}_{1} \sum_{j=1}^{3} \sum_{k=1}^{3} \epsilon_{1 j k} a_{j} b_{k} &= \hat{\vec{u}}_{1}(\epsilon_{123} a_{2} b_{3} + \epsilon_{132} a_{3} b_{2}) \tag{1.48} \\
&= \hat{\vec{u}}_{1}(1 \cdot a_{2} b_{3} + (-1) \cdot a_{3} b_{2}) \\
&= \hat{\vec{u}}_{1}(a_{2} b_{3} - a_{3} b_{2})
\end{align*}

This is precisely the first component of the cross product $\vec{a} \times \vec{b}$. Similarly, the second and third terms yield the second and third components:
$\hat{\vec{u}}_{2}(a_{3} b_{1} - a_{1} b_{3})$ and $\hat{\vec{u}}_{3}(a_{1} b_{2} - a_{2} b_{1})$.

Thus, equation (1.46) is indeed the cross product of vectors $\vec{a}$ and $\vec{b}$, expressed using the Levi-Civita symbol.

Using this notation, we can elegantly express the components of the angular momentum operator:

\begin{equation*}
\hat{L}_{i} = (\hat{\vec{x}} \times \hat{\vec{p}})_{i} = \epsilon_{i j k} \hat{x}_{j} \hat{p}_{k} \tag{1.49}
\end{equation*}

Expanding this definition:
\begin{align*}
\hat{L}_1 &= \hat{x}_2\hat{p}_3 - \hat{x}_3\hat{p}_2 = \hat{y}\hat{p}_z - \hat{z}\hat{p}_y \\
\hat{L}_2 &= \hat{x}_3\hat{p}_1 - \hat{x}_1\hat{p}_3 = \hat{z}\hat{p}_x - \hat{x}\hat{p}_z \\
\hat{L}_3 &= \hat{x}_1\hat{p}_2 - \hat{x}_2\hat{p}_1 = \hat{x}\hat{p}_y - \hat{y}\hat{p}_x
\end{align*}

This compact form highlights the geometric nature of angular momentum as a cross product of position and momentum operators.

To formalize the mathematical framework of quantum mechanics, we define the Hilbert space $L^{2}(V)$, which contains all square-integrable functions:

\begin{equation*}
L^{2}(V) := \left\{\Psi(\vec{x},t) : \int_{V} d^{3}x |\Psi(\vec{x},t)|^{2} < \infty \right\} \tag{1.50}
\end{equation*}

This space provides the mathematical foundation for quantum states, ensuring that probability interpretations remain valid.

For a generic linear operator $\hat{A}$ acting on functions in this space:

\begin{equation*}
\Psi \in L^{2}(V) \Rightarrow \hat{A}\Psi = \phi \in L^{2}(V) \tag{1.51}
\end{equation*}

This requirement ensures that operators transform physically meaningful states into other physically meaningful states.

Quantum operators satisfy several important algebraic properties:

\begin{itemize}
  \item Additivity: $(\hat{A}+\hat{B})\Psi = \hat{A}\Psi + \hat{B}\Psi$
  \item Linearity: $\hat{A}(c_{1}\Psi_{1} + c_{2}\Psi_{2}) = c_{1}\hat{A}\Psi_{1} + c_{2}\hat{A}\Psi_{2}$
  \item Associativity: $(\hat{A}\hat{B})\Psi = \hat{A}(\hat{B}\Psi) = \hat{A}\phi$ where $\phi = \hat{B}\Psi$
  \item Non-commutativity: $\hat{A}\hat{B} \neq \hat{B}\hat{A}$ in general, a property with profound physical implications
  \item Identity operator: $\hat{\mathbb{I}}\Psi = \Psi$ for all $\Psi$
  \item Zero operator: $\hat{O}\Psi = 0$ for all $\Psi$
\end{itemize}

The non-commutativity of operators is particularly significant in quantum mechanics, as it leads directly to uncertainty principles and complementarity between conjugate observables.


\subsection*{1.5 Commutators}

In quantum mechanics, the commutator is a fundamental mathematical tool that captures the non-commutative nature of physical observables. For two operators $\hat{A}$ and $\hat{B}$, the commutator is defined as:

\begin{equation*}
[\hat{A}, \hat{B}] := \hat{A}\hat{B} - \hat{B}\hat{A} \tag{1.52}
\end{equation*}

When the commutator $[\hat{A}, \hat{B}] = 0$, we say the operators commute, meaning their order of application doesn't matter. However, when $[\hat{A}, \hat{B}] \neq 0$, the order becomes physically significant, leading to uncertainty relations between the corresponding observables.

Commutators possess several important algebraic properties:

\begin{enumerate}
  \item Bilinearity: The commutator is linear in both arguments.
\end{enumerate}

\begin{align*}
[c_{1}\hat{A} + c_{2}\hat{B}, \hat{C}] &= c_{1}[\hat{A}, \hat{C}] + c_{2}[\hat{B}, \hat{C}] \\
[\hat{A}, c_{1}\hat{B} + c_{2}\hat{C}] &= c_{1}[\hat{A}, \hat{B}] + c_{2}[\hat{A}, \hat{C}] \tag{1.53}
\end{align*}

This property allows us to distribute commutators across linear combinations of operators.

\begin{enumerate}
  \setcounter{enumi}{1}
  \item Leibniz rule: The commutator satisfies a product rule analogous to differentiation.
\end{enumerate}

\begin{equation*}
[\hat{A}, \hat{B}\hat{C}] = [\hat{A}, \hat{B}]\hat{C} + \hat{B}[\hat{A}, \hat{C}] \tag{1.54}
\end{equation*}

This property is particularly useful when computing commutators of composite operators.

\begin{enumerate}
  \setcounter{enumi}{2}
  \item Jacobi identity: The commutator satisfies a cyclic relation.
\end{enumerate}

\begin{equation*}
[\hat{A},[\hat{B}, \hat{C}]] + [\hat{B},[\hat{C}, \hat{A}]] + [\hat{C},[\hat{A}, \hat{B}]] = 0 \tag{1.55}
\end{equation*}

The Jacobi identity plays a crucial role in the theory of Lie algebras and is essential for establishing conservation laws in quantum mechanics.

These properties reveal a profound connection between quantum mechanics and classical mechanics. In classical Hamiltonian mechanics, Poisson brackets $\{f,g\}$ serve a role analogous to commutators in quantum mechanics. The correspondence between these structures is captured in the following relations:

\begin{align*}
[x_{j}, x_{k}] &= 0 \longleftrightarrow \{x_{j}, x_{k}\} = 0 \\
[\hat{p}_{j}, \hat{p}_{k}] &= 0 \longleftrightarrow \{p_{j}, p_{k}\} = 0 \tag{1.56} \\
[x_{j}, \hat{p}_{k}] &= ? \longleftrightarrow \{x_{j}, p_{k}\} = \delta_{j k}
\end{align*}

This correspondence, formalized in Dirac's quantum condition, suggests that quantum mechanics can be viewed as a deformation of classical mechanics where Poisson brackets are replaced by commutators, with $i\hbar$ serving as the deformation parameter.

The last relation raises a crucial question: what is the commutator between position and momentum operators? This fundamental commutation relation determines the uncertainty principle and lies at the heart of quantum behavior.


Let's compute the commutator between position and momentum operators, a fundamental relation in quantum mechanics.

For the position operator $x_j$ and momentum operator $\hat{p}_k$, the commutator is:

\begin{equation*}
[x_j, \hat{p}_k] = x_j\hat{p}_k - \hat{p}_k x_j \tag{1.57}
\end{equation*}

Using the position representation of the momentum operator $\hat{p}_k = -i\hbar\frac{\partial}{\partial x_k}$, we apply this commutator to a general wave function $\Psi$:

\begin{align*}
[x_j, \hat{p}_k]\Psi &= x_j\left(-i\hbar\frac{\partial\Psi}{\partial x_k}\right) - \left(-i\hbar\frac{\partial}{\partial x_k}\right)(x_j\Psi) \\
&= -i\hbar x_j\frac{\partial\Psi}{\partial x_k} + i\hbar\frac{\partial}{\partial x_k}(x_j\Psi)
\end{align*}

Applying the product rule to the second term:

\begin{align*}
[x_j, \hat{p}_k]\Psi &= -i\hbar x_j\frac{\partial\Psi}{\partial x_k} + i\hbar\left(\frac{\partial x_j}{\partial x_k}\Psi + x_j\frac{\partial\Psi}{\partial x_k}\right) \tag{1.58} \\
&= -i\hbar x_j\frac{\partial\Psi}{\partial x_k} + i\hbar\delta_{jk}\Psi + i\hbar x_j\frac{\partial\Psi}{\partial x_k}
\end{align*}

Where we've used $\frac{\partial x_j}{\partial x_k} = \delta_{jk}$, the Kronecker delta. The first and third terms cancel, leaving:

\begin{align*}
[x_j, \hat{p}_k]\Psi &= i\hbar\delta_{jk}\Psi
\end{align*}

Since this holds for any wave function $\Psi$, we conclude:

\begin{equation*}
[x_j, \hat{p}_k] = i\hbar\delta_{jk} \tag{1.59}
\end{equation*}

This canonical commutation relation is one of the most fundamental in quantum mechanics. It directly leads to the Heisenberg uncertainty principle and distinguishes quantum from classical mechanics.

Next, let's examine the commutator between the momentum operator $\hat{p}_j$ and a general function of position $f(x_1,x_2,x_3)$:

\begin{equation*}
[\hat{p}_j, f] = \hat{p}_j f - f\hat{p}_j \tag{1.60}
\end{equation*}

Applying this commutator to a wave function:

\begin{align*}
[\hat{p}_j, f]\Psi &= \hat{p}_j(f\Psi) - f(\hat{p}_j\Psi) \\
&= -i\hbar\frac{\partial}{\partial x_j}(f\Psi) - f\left(-i\hbar\frac{\partial\Psi}{\partial x_j}\right)
\end{align*}

Using the product rule for differentiation:

\begin{align*}
[\hat{p}_j, f]\Psi &= -i\hbar\left(\frac{\partial f}{\partial x_j}\Psi + f\frac{\partial\Psi}{\partial x_j}\right) + i\hbar f\frac{\partial\Psi}{\partial x_j} \tag{1.61} \\
&= -i\hbar\frac{\partial f}{\partial x_j}\Psi
\end{align*}

Therefore:

\begin{equation*}
[\hat{p}_j, f] = -i\hbar\frac{\partial f}{\partial x_j}
\end{equation*}

This result generalizes the position-momentum commutation relation and is essential for calculating the evolution of observables in quantum mechanics. It shows that momentum operators act as generators of infinitesimal translations in position space.

The commutation relations (1.59) and (1.61) form the algebraic foundation of quantum mechanics and lead directly to the uncertainty principle. They encode the fundamental non-commutativity of conjugate observables, which has no classical analog and is responsible for many uniquely quantum phenomena.


From our previous calculation, we have established:

\begin{equation*}
[\hat{p}_j, f] = -i\hbar\frac{\partial f}{\partial x_j} \tag{1.62}
\end{equation*}

This relation is essential for understanding how momentum operators interact with position-dependent functions in quantum mechanics.

Next, we'll determine the commutation relation between angular momentum and linear momentum operators. Recall that the components of angular momentum are given by:

\begin{equation*}
\hat{L}_i = (\hat{\vec{x}} \times \hat{\vec{p}})_i = \epsilon_{irs}x_r\hat{p}_s \tag{1.63}
\end{equation*}

To find the commutator between $\hat{L}_i$ and $\hat{p}_n$, we write:

\begin{equation*}
[\hat{L}_i, \hat{p}_n] = [\epsilon_{irs}x_r\hat{p}_s, \hat{p}_n] \tag{1.64}
\end{equation*}

Expanding this using the Leibniz rule for commutators:

\begin{align*}
[\hat{L}_i, \hat{p}_n] &= \epsilon_{irs}[x_r\hat{p}_s, \hat{p}_n] \\
&= \epsilon_{irs}(x_r[\hat{p}_s, \hat{p}_n] + [x_r, \hat{p}_n]\hat{p}_s) \tag{1.65}
\end{align*}

We know that momentum components commute with each other, so $[\hat{p}_s, \hat{p}_n] = 0$. Using our earlier result $[x_r, \hat{p}_n] = i\hbar\delta_{rn}$:

\begin{align*}
[\hat{L}_i, \hat{p}_n] &= \epsilon_{irs}(0 + i\hbar\delta_{rn}\hat{p}_s) \\
&= i\hbar\epsilon_{irs}\delta_{rn}\hat{p}_s \\
&= i\hbar\epsilon_{ins}\hat{p}_s
\end{align*}

This compact expression shows how angular momentum transforms linear momentum under rotations. It reveals that angular momentum serves as the generator of rotations in quantum mechanics.

Now let's calculate the commutator between angular momentum and the squared momentum operator $\hat{p}^2 = \sum_{k=1}^3 \hat{p}_k^2$:

\begin{equation*}
[\hat{L}_i, \hat{p}^2] = \left[\hat{L}_i, \sum_{k=1}^3 \hat{p}_k^2\right] \tag{1.66}
\end{equation*}

Using the linearity property of commutators:

\begin{equation*}
[\hat{L}_i, \hat{p}^2] = \sum_{k=1}^3 [\hat{L}_i, \hat{p}_k^2] \tag{1.67}
\end{equation*}

For each term in the sum, we can apply the Leibniz rule:

Continuing with our calculation of $[\hat{L}_i, \hat{p}^2]$, we analyze each term in the sum using the Leibniz rule:

\begin{equation*}
[\hat{L}_i, \hat{p}_k^2] = [\hat{L}_i, \hat{p}_k\hat{p}_k] = [\hat{L}_i, \hat{p}_k]\hat{p}_k + \hat{p}_k[\hat{L}_i, \hat{p}_k] \tag{1.68}
\end{equation*}

Using our previously derived result:

\begin{equation*}
[\hat{L}_i, \hat{p}_k] = i\hbar\epsilon_{iks}\hat{p}_s \tag{1.69}
\end{equation*}

We substitute this into equation (1.68):

\begin{equation*}
[\hat{L}_i, \hat{p}_k^2] = (i\hbar\epsilon_{iks}\hat{p}_s)\hat{p}_k + \hat{p}_k(i\hbar\epsilon_{iks}\hat{p}_s) \tag{1.70}
\end{equation*}

Now, returning to the full commutator with $\hat{p}^2$:

\begin{equation*}
[\hat{L}_i, \hat{p}^2] = \sum_{k=1}^{3}(i\hbar\epsilon_{iks}\hat{p}_s)\hat{p}_k + \sum_{k=1}^{3}\hat{p}_k(i\hbar\epsilon_{iks}\hat{p}_s) \tag{1.71}
\end{equation*}

Since $k$ and $s$ are summation indices, we can rename them in the second sum, swapping $k$ and $s$:

\begin{equation*}
[\hat{L}_i, \hat{p}^2] = \sum_{k=1}^{3}(i\hbar\epsilon_{iks}\hat{p}_s)\hat{p}_k + \sum_{s=1}^{3}\hat{p}_s(i\hbar\epsilon_{isk}\hat{p}_k) \tag{1.72}
\end{equation*}

Using the antisymmetry property of the Levi-Civita symbol: $\epsilon_{isk} = -\epsilon_{iks}$, the second term becomes:

\begin{align*}
\sum_{s=1}^{3}\hat{p}_s(i\hbar\epsilon_{isk}\hat{p}_k) &= \sum_{s=1}^{3}\hat{p}_s(-i\hbar\epsilon_{iks}\hat{p}_k) \\
&= -\sum_{s=1}^{3}(i\hbar\epsilon_{iks}\hat{p}_s\hat{p}_k)
\end{align*}

Therefore:

\begin{align*}
[\hat{L}_i, \hat{p}^2] &= \sum_{k=1}^{3}(i\hbar\epsilon_{iks}\hat{p}_s\hat{p}_k) - \sum_{s=1}^{3}(i\hbar\epsilon_{iks}\hat{p}_s\hat{p}_k) \\
&= i\hbar\epsilon_{iks}\sum_{k=1}^{3}(\hat{p}_s\hat{p}_k - \hat{p}_s\hat{p}_k) \\
&= 0
\end{align*}

This remarkable result shows that $[\hat{L}_i, \hat{p}^2] = 0$, meaning angular momentum commutes with the squared momentum operator. This is a manifestation of rotational invariance - the magnitude of momentum is unchanged by rotations, though its direction changes.

The vanishing commutator $[\hat{L}_i, \hat{p}^2] = 0$ has profound physical significance:

1. It implies that angular momentum and the energy of a free particle (proportional to $\hat{p}^2$) can be simultaneously measured with arbitrary precision.

2. It confirms that rotational symmetry preserves kinetic energy, as expected from physical intuition.

3. It establishes $\hat{p}^2$ as a rotational invariant, consistent with its scalar nature.

This commutation relation is part of a broader pattern in quantum mechanics where symmetry operations commute with the corresponding invariant quantities, reflecting fundamental conservation laws through Noether's theorem.


The final calculation confirms that angular momentum commutes with squared momentum:

\begin{equation*}
[\hat{L}_i, \hat{p}^2] = \sum_{k=1}^{3}(i\hbar\epsilon_{iks}\hat{p}_s\hat{p}_k) - \sum_{k=1}^{3}(i\hbar\epsilon_{iks}\hat{p}_s\hat{p}_k) = 0 \tag{1.73}
\end{equation*}

Through similar analysis, we can establish several other important commutation relations:

\begin{align*}
[\hat{L}^2, \hat{p}_n] &= 0 \\
[\hat{L}^2, \hat{L}_j] &= 0 \\
[\hat{L}_j, x^2] &= 0 \tag{1.74} \\
[\hat{L}_j, \hat{L}_k] &= i\hbar\epsilon_{jkm}\hat{L}_m
\end{align*}

These relations have profound physical implications:
- $[\hat{L}^2, \hat{p}_n] = 0$ indicates that the magnitude of angular momentum commutes with linear momentum components
- $[\hat{L}^2, \hat{L}_j] = 0$ shows that the magnitude of angular momentum commutes with its components
- $[\hat{L}_j, x^2] = 0$ demonstrates that angular momentum components commute with the square of position
- $[\hat{L}_j, \hat{L}_k] = i\hbar\epsilon_{jkm}\hat{L}_m$ reveals that angular momentum components do not commute with each other, forming an $SU(2)$ algebra

\subsection*{1.6 Canonical quantization rule}

The pattern emerging from our calculations leads to a fundamental principle in quantum mechanics:

Definition 1.1. The canonical quantization rule (CQR) establishes a correspondence between classical and quantum mechanics:

\begin{align*}
\text{Classical} &\longleftrightarrow \text{Quantum} \\
x_j &\longleftrightarrow x_j \\
p_j &\longleftrightarrow \hat{p}_j = -i\hbar\frac{\partial}{\partial x_j} \tag{1.75} \\
\{x_j, p_k\} = \delta_{jk} &\longleftrightarrow [x_j, \hat{p}_k] = i\hbar\delta_{jk}
\end{align*}

Where $\{A,B\}$ denotes the classical Poisson bracket, and $[A,B]$ is the quantum commutator.

This correspondence can be extended to general observables. For classical observables:

\begin{align*}
A &= A(q_\alpha, p_\alpha) \\
B &= B(q_\alpha, p_\alpha) \tag{1.76} \\
C &= \{A, B\}
\end{align*}

The canonical quantization rule gives us:

\begin{equation*}
[\hat{A}, \hat{B}] = i\hbar\hat{C} \tag{1.77}
\end{equation*}

This principle, formulated by Dirac, provides a systematic way to convert classical theories into quantum ones. It reveals quantum mechanics as a deformation of classical mechanics with $\hbar$ as the deformation parameter. When $\hbar \to 0$, we recover classical physics.

However, the canonical quantization rule faces ordering ambiguities when quantizing products of non-commuting observables (e.g., $xp$ could become $\hat{x}\hat{p}$ or $\hat{p}\hat{x}$ or some combination). Various prescriptions exist to resolve these ambiguities, including Weyl ordering and normal ordering.

\subsection*{1.7 Expectation value}

To connect quantum operators with measurable quantities, we need to define how to extract physical predictions from wave functions. This begins with defining an appropriate scalar product on the Hilbert space $L^2(V)$ of square-integrable functions.


For wave functions $\phi, \Psi \in L^2(V)$, the scalar product is defined as:

\begin{equation*}
(\phi, \Psi) := \int_V d^3x \, \phi^*(x) \Psi(x) \tag{1.78}
\end{equation*}

This inner product satisfies all requirements of a Hermitian scalar product, particularly sesquilinearity:

\begin{equation*}
(a\phi, b_1\Psi_1 + b_2\Psi_2) = a^*b_1(\phi, \Psi_1) + a^*b_2(\phi, \Psi_2) \tag{1.79}
\end{equation*}

For functions $\xi, \phi, \Psi \in L^2(V)$ where $\phi = \hat{A}\Psi$ for some operator $\hat{A}$, the scalar product becomes:

\begin{equation*}
(\xi, \phi) = \int_V d^3x \, \xi^*(x)(\hat{A}\Psi)(x) \tag{1.80}
\end{equation*}

When $\xi = \Psi$, this defines the expectation value of operator $\hat{A}$:

\begin{equation*}
(\Psi, \phi) = (\Psi, \hat{A}\Psi) = \int_V d^3x \, \Psi^*(x)(\hat{A}\Psi)(x) := \langle\hat{A}\rangle \tag{1.81}
\end{equation*}

The expectation value $\langle\hat{A}\rangle$ represents the statistical average of measurements of the physical observable corresponding to $\hat{A}$ when the system is in state $\Psi$.

From the properties of the scalar product and the Hermiticity of physical operators, several important relations follow:

\begin{itemize}
  \item $(\phi, \hat{A}\Psi) = (\hat{A}^\dagger\phi, \Psi) = (\hat{A}\phi, \Psi)$ for Hermitian operators
  \item $(\Psi, \hat{A}\Psi)^* = (\hat{A}\Psi, \Psi) = (\Psi, \hat{A}^\dagger\Psi) = (\Psi, \hat{A}\Psi)$
\end{itemize}

The second property confirms that expectation values of Hermitian operators are real numbers, as required for physical observables.

For operator products, we have:
\begin{itemize}
  \item $(\phi, (\hat{A}\hat{B})^\dagger\Psi) = (\hat{A}\hat{B}\phi, \Psi) = (\hat{B}\phi, \hat{A}^\dagger\Psi) = (\phi, \hat{B}^\dagger\hat{A}^\dagger\Psi)$
\end{itemize}

This shows that $(\hat{A}\hat{B})^\dagger = \hat{B}^\dagger\hat{A}^\dagger$, a fundamental property of adjoint operators.

\subsection*{1.8 The Ehrenfest Theorem}

To establish the Ehrenfest theorem, we first verify that the Hamiltonian operator is Hermitian. A Hermitian operator must satisfy:

\begin{equation*}
(\phi, \hat{H}^\dagger\Psi) = (\Psi, \hat{H}\phi)^* \stackrel{!}{=} (\phi, \hat{H}\Psi) \tag{1.82}
\end{equation*}

for all $\phi, \Psi \in L^2(V)$.

Proof:


To prove the Hermiticity of the Hamiltonian operator, we examine:

\begin{align*}
(\Psi, \hat{H}\phi)^* &= \int_V d^3\vec{x}(\Psi^*\hat{H}\phi)^* \\
&= \int_V d^3\vec{x}\left[\Psi^*\left(-\frac{\hbar^2}{2m}\nabla^2 + U\right)\phi\right]^* \\
&= \int_V d^3\vec{x}\Psi\left(-\frac{\hbar^2}{2m}\nabla^2 + U\right)\phi^* \tag{1.83} \\
&= -\frac{\hbar^2}{2m}\int_V d^3\vec{x}\Psi\nabla^2\phi^* + \int_V d^3\vec{x}\Psi U\phi^*
\end{align*}

Using integration by parts twice on the first term and applying boundary conditions where wave functions vanish at the boundary:

\begin{align*}
-\frac{\hbar^2}{2m}\int_V d^3\vec{x}\Psi\nabla^2\phi^* &= -\frac{\hbar^2}{2m}\int_{\partial V}d\vec{\Sigma}\cdot\Psi\vec{\nabla}\phi^* + \frac{\hbar^2}{2m}\int_V d^3\vec{x}\vec{\nabla}\Psi\cdot\vec{\nabla}\phi^* \tag{1.84} \\
&= \frac{\hbar^2}{2m}\int_{\partial V}d\vec{\Sigma}\cdot(\vec{\nabla}\Psi)\phi^* - \frac{\hbar^2}{2m}\int_V d^3\vec{x}\phi^*\nabla^2\Psi
\end{align*}

The surface integrals vanish because $\Psi$ and $\phi$ approach zero at infinity. Substituting back:

\begin{align*}
(\Psi, \hat{H}\phi)^* &= -\frac{\hbar^2}{2m}\int_V d^3\vec{x}\phi^*\nabla^2\Psi + \int_V d^3\vec{x}\phi^*U\Psi \\
&= \int_V d^3\vec{x}\phi^*\left(-\frac{\hbar^2}{2m}\nabla^2 + U\right)\Psi \\
&= (\phi, \hat{H}\Psi) \tag{1.85}
\end{align*}

This confirms that $\hat{H}$ is indeed Hermitian, as required for a physical observable. $\square$

Now we derive the Ehrenfest theorem by examining the time evolution of expectation values:

\begin{align*}
\frac{\partial\langle\hat{A}\rangle}{\partial t} &= \int_V d^3x\frac{\partial}{\partial t}(\Psi^*\hat{A}\Psi) \\
&= \int_V d^3x(\partial_t\Psi^*)\hat{A}\Psi + \int_V d^3x\Psi^*(\partial_t\hat{A})\Psi + \int_V d^3x\Psi^*\hat{A}(\partial_t\Psi) \tag{1.86}
\end{align*}

Using the Schrödinger equation:
$i\hbar\frac{\partial\Psi}{\partial t} = \hat{H}\Psi$ and $-i\hbar\frac{\partial\Psi^*}{\partial t} = \hat{H}\Psi^*$

\begin{align*}
\frac{\partial\langle\hat{A}\rangle}{\partial t} &= -\frac{1}{i\hbar}(\hat{H}\Psi, \hat{A}\Psi) + \frac{1}{i\hbar}(\Psi, \hat{A}\hat{H}\Psi) + (\Psi, \partial_t\hat{A}\Psi) \\
&= -\frac{1}{i\hbar}(\Psi, \hat{H}\hat{A}\Psi) + \frac{1}{i\hbar}(\Psi, \hat{A}\hat{H}\Psi) + (\Psi, \partial_t\hat{A}\Psi) \\
&= \frac{1}{i\hbar}(\Psi, (\hat{A}\hat{H} - \hat{H}\hat{A})\Psi) + (\Psi, \partial_t\hat{A}\Psi) \tag{1.87} \\
&= \frac{1}{i\hbar}(\Psi, [\hat{A}, \hat{H}]\Psi) + (\Psi, \partial_t\hat{A}\Psi) \\
&= \frac{1}{i\hbar}\langle[\hat{A}, \hat{H}]\rangle + \langle\partial_t\hat{A}\rangle
\end{align*}

This is the Ehrenfest theorem, which shows striking similarity to classical mechanics. In classical Hamiltonian mechanics, the time evolution of a function $A$ is given by:


The Ehrenfest theorem reveals a profound correspondence between quantum and classical mechanics. In classical mechanics, the time evolution of a function $A$ is given by:

\begin{equation*}
\frac{dA}{dt} = \{H,A\} + \frac{\partial A}{\partial t} \tag{1.88}
\end{equation*}

where $\{H,A\}$ is the Poisson bracket. Comparing with equation (1.87), we see that quantum commutators $\frac{1}{i\hbar}[\hat{A},\hat{H}]$ play the role analogous to classical Poisson brackets $\{H,A\}$.

Let's examine specific applications of the Ehrenfest theorem:

For the position operator $x_j$:

\begin{equation*}
\frac{\partial\langle x_j\rangle}{\partial t} = \frac{1}{i\hbar}\langle[x_j,\hat{H}]\rangle = \frac{1}{i\hbar}\left\langle\left[x_j,-\frac{\hat{p}^2}{2m}+U\right]\right\rangle = \frac{1}{i\hbar}\frac{1}{2m}\langle[x_j,\hat{p}^2]\rangle \tag{1.89}
\end{equation*}

Since $x_j$ commutes with all position operators and only fails to commute with $\hat{p}_j$, we have:

\begin{equation*}
\frac{\partial\langle x_j\rangle}{\partial t} = \frac{1}{i\hbar 2m}\langle[x_j,\hat{p}_j^2]\rangle \stackrel{\text{Leibniz}}{=} \frac{1}{i\hbar 2m}\langle\hat{p}_j[x_j,\hat{p}_j]+[x_j,\hat{p}_j]\hat{p}_j\rangle = \frac{1}{m}\langle\hat{p}_j\rangle \tag{1.90}
\end{equation*}

Using the commutation relation $[x_j,\hat{p}_j] = i\hbar$.

For the momentum operator $\hat{p}_j$:

\begin{equation*}
\frac{\partial\langle\hat{p}_j\rangle}{\partial t} = \frac{1}{i\hbar}\langle[\hat{p}_j,\hat{H}]\rangle = \frac{1}{i\hbar}\left\langle\left[\hat{p}_j,-\frac{\hat{p}^2}{2m}+U\right]\right\rangle \tag{1.91}
\end{equation*}

The first term vanishes since momentum operators commute with each other: $[\hat{p}_j,\hat{p}^2] = 0$.

For the potential energy term, we use our earlier result for the commutator of momentum with a function:

\begin{equation*}
[\hat{p}_j,U] = [\hat{p}_j,U(x_1,x_2,x_3)] = -i\hbar\frac{\partial U}{\partial x_j} \tag{1.92}
\end{equation*}

Therefore:

\begin{equation*}
\frac{\partial\langle\hat{p}_j\rangle}{\partial t} = -\left\langle\frac{\partial U}{\partial x_j}\right\rangle \tag{1.93}
\end{equation*}

These results are the quantum analogs of Newton's equations of motion. In classical mechanics:


The Ehrenfest theorem leads directly to the quantum analogs of Newton's laws:

\[
\begin{array}{r}
\frac{dx_j}{dt} = \{H, x_j\} = \frac{p_j}{m} \\
\frac{dp_j}{dt} = \{H, p_j\} = -\frac{\partial U}{\partial x_j} \tag{1.94}
\end{array}
\]

This correspondence is formalized in Ehrenfest's theorem:

Theorem 1.1. The expectation values of physical operators follow the same equations of motion as their classical counterparts.

This theorem applies exactly when the potential $U$ contains at most quadratic terms in position coordinates. For a general potential of the form:

\begin{equation*}
U = \sum_i \sum_k U_{ik}x_i x_k + \sum_i c_i x_i \tag{1.95}
\end{equation*}

Differentiating with respect to coordinate $x_n$:

\begin{align*}
W := \frac{\partial U}{\partial x_n} &= \sum_i \sum_k U_{ik}\frac{\partial}{\partial x_n}(x_i x_k) + \sum_i c_i \frac{\partial x_i}{\partial x_n} \\
&= \sum_i \sum_k U_{ik}x_k\frac{\partial x_i}{\partial x_n} + \sum_i \sum_k U_{ik}x_i\frac{\partial x_k}{\partial x_n} + \sum_i c_i\frac{\partial x_i}{\partial x_n} \\
&= \sum_i \sum_k U_{ik}x_k\delta_{in} + \sum_i \sum_k U_{ik}x_i\delta_{kn} + \sum_i c_i\delta_{in} \\
&= \sum_k(U_{nk} + U_{kn})x_k + c_n \tag{1.96}
\end{align*}

Since $W$ is linear in the coordinates, we can show that:

\begin{equation*}
\langle W(x_n)\rangle = W(\langle x_i\rangle) \tag{1.97}
\end{equation*}

This equality wouldn't hold for potentials with higher-order terms. However, in the semiclassical limit (large quantum numbers), the factorization approximation $\langle\hat{A}\hat{B}\rangle \approx \langle\hat{A}\rangle\langle\hat{B}\rangle$ becomes valid, extending Ehrenfest's theorem to more general potentials.

This aligns with Bohr's Correspondence Principle, which states that quantum mechanics must reduce to classical mechanics in the limit of large quantum numbers or macroscopic systems.

\section*{2 Solutions to the Schrödinger's equation}

This section explores general solutions to the Schrödinger equation and their physical interpretation.

\subsection*{2.1 Stationary solution}

The separation of variables technique is a powerful method for solving partial differential equations like the Schrödinger equation. This approach assumes that solutions can be expressed as products of functions, each depending on a single variable.


The separation of variables method assumes a solution of the form:

\begin{equation*}
\Psi(\vec{x}, t) = f(t)\Psi(\vec{x}) \tag{2.1}
\end{equation*}

Substituting this into the Schrödinger equation:

\begin{equation*}
i\hbar\Psi\frac{\partial f}{\partial t} = -\frac{\hbar^2}{2m}f\nabla^2\Psi + Uf\Psi \tag{2.2}
\end{equation*}

Dividing by $f\Psi$ (assuming both are non-zero):

\begin{align*}
\frac{1}{f(t)}i\hbar\frac{\partial f}{\partial t} &= \frac{1}{\Psi}\left(-\frac{\hbar^2}{2m}\nabla^2\Psi + U\Psi\right) \\
\frac{1}{f(t)}i\hbar\frac{\partial f}{\partial t} &= \frac{1}{\Psi}\hat{H}\Psi \tag{2.3}
\end{align*}

Since the left side depends only on time and the right side only on spatial coordinates, both must equal a constant, which we denote as $E$. This leads to two separate equations:

\[
\left\{\begin{array}{l}
i\hbar\frac{\partial f}{\partial t} = Ef(t) \\
\hat{H}\Psi = E\Psi
\end{array}\right. \Rightarrow \left\{\begin{array}{l}
f(t) = e^{-\frac{iE}{\hbar}t} \\
\hat{H}\Psi_E = E\Psi_E
\end{array}\right. \tag{2.4}
\]

The first equation has the solution $f(t) = e^{-\frac{iE}{\hbar}t}$, while the second is an eigenvalue equation for the Hamiltonian operator. The energy $E$ represents an eigenvalue, and $\Psi_E$ the corresponding eigenfunction.

These solutions, called stationary states, have special significance in quantum mechanics as they represent states of definite energy. While the spatial probability distribution $|\Psi_E(\vec{x})|^2$ remains constant in time, the phase evolves uniformly.

\subsection*{2.2 Superposition principle}

The stationary solution has the form:

\begin{equation*}
\Psi(\vec{x}, t) = \exp\left(-\frac{iE}{\hbar}t\right)\Psi_E(\vec{x}) \tag{2.5}
\end{equation*}

Since the Schrödinger equation is linear, any linear combination of solutions is also a solution. This leads to the superposition principle, allowing us to write a general solution as:

\begin{equation*}
\Psi(\vec{x}, t) = \sum_n \underbrace{c(E_n)}_{c_n}\exp\left(-\frac{iE_n}{\hbar}t\right)\Psi_{E_n}(\vec{x}) \tag{2.6}
\end{equation*}

Where $c_n$ are complex coefficients determining the contribution of each eigenstate to the total wavefunction.

To verify this is indeed a solution, we can substitute it into the Schrödinger equation. For the time derivative:


Verifying the superposition principle by substituting into the Schrödinger equation:

For the time derivative:
\begin{align*}
-i\hbar\frac{\partial\Psi(\vec{x},t)}{\partial t} &= -i\hbar\frac{\partial}{\partial t}\sum_n c_n\exp\left(-\frac{iE_n}{\hbar}t\right)\Psi_{E_n}(\vec{x}) \\
&= \sum_n c_n\Psi_{E_n}(\vec{x})(-i\hbar)\frac{\partial}{\partial t}\left[\exp\left(-\frac{iE_n}{\hbar}t\right)\right] \tag{2.7} \\
&= \sum_n c_nE_n\exp\left(-\frac{iE_n}{\hbar}t\right)\Psi_{E_n}(\vec{x})
\end{align*}

For the Hamiltonian operating on the wavefunction:
\begin{equation*}
\hat{H}\Psi(\vec{x},t) = \hat{H}\sum_n c_n\exp\left(-\frac{iE_n}{\hbar}t\right)\Psi_{E_n}(\vec{x}) \tag{2.8}
\end{equation*}

Since the Hamiltonian operates only on the spatial part:
\begin{equation*}
\hat{H}\Psi(\vec{x},t) = \sum_n c_n\exp\left(-\frac{iE_n}{\hbar}t\right)\hat{H}\Psi_{E_n}(\vec{x}) \tag{2.9}
\end{equation*}

Using the eigenvalue equation $\hat{H}\Psi_{E_n} = E_n\Psi_{E_n}$:
\begin{equation*}
\hat{H}\Psi(\vec{x},t) = \sum_n c_n\exp\left(-\frac{iE_n}{\hbar}t\right)E_n\Psi_{E_n}(\vec{x}) \tag{2.10}
\end{equation*}

Comparing equations (2.7) and (2.10), we see they are identical, confirming that the superposition indeed satisfies the Schrödinger equation.

\subsection*{2.3 Formal solution to the Schrödinger's equation}

The time evolution of quantum states can be elegantly expressed using the quantum propagator, which connects the initial state to its future evolution:

\begin{equation*}
\underbrace{\Psi(\vec{x},t)}_{\text{propagated state}} = \underbrace{\exp\left(-\frac{i}{\hbar}\hat{H}t\right)}_{\text{propagator}}\underbrace{\Psi(\vec{x},0)}_{\text{initial state}} \tag{2.11}
\end{equation*}

The exponential operator can be expanded as a Taylor series:

\begin{align*}
\Psi(\vec{x},t) = \exp\left(-\frac{i}{\hbar}\hat{H}t\right)\Psi(\vec{x},0) &\stackrel{\text{Taylor}}{=} \sum_{n=0}^{\infty}\frac{\left(-\frac{i}{\hbar}t\right)^n}{n!}\hat{H}^n\Psi(\vec{x},0) \\
&= \sum_{n=0}^{\infty}\frac{\left(-\frac{i}{\hbar}t\right)^n}{n!}E^n\Psi(\vec{x},0) \tag{2.12} \\
&= \exp\left(-\frac{i}{\hbar}Et\right)\Psi(\vec{x},0)
\end{align*}

This formal solution demonstrates how the time-evolution operator $\exp\left(-\frac{i}{\hbar}\hat{H}t\right)$ acts as a propagator that evolves the initial state forward in time. For stationary states (eigenstates of the Hamiltonian), this simply adds a time-dependent phase factor.

For a general initial state that is a superposition of energy eigenstates, the propagator acts on each component independently, consistent with our earlier derivation using the superposition principle.

The propagator formalism offers several advantages:

1. It provides a compact mathematical expression for time evolution
2. It preserves probability (unitarity) by construction
3. It can be extended to more complex scenarios, including time-dependent Hamiltonians
4. It connects quantum mechanics to path integral formulations

This approach also highlights the fundamental role of the Hamiltonian as the generator of time translations in quantum mechanics, a direct consequence of energy conservation and time-translation symmetry.


The formal solution to the Schrödinger equation can be verified by direct substitution. For a superposition of energy eigenstates evolved by the propagator:

\begin{align*}
i\hbar\frac{\partial}{\partial t}\sum_n c_n\exp\left(-\frac{i}{\hbar}\hat{H}t\right)\Psi_{E_n} &= \sum_n c_n i\hbar\frac{\partial}{\partial t}\exp\left(-\frac{i}{\hbar}\hat{H}t\right)\Psi_{E_n} \\
&= \sum_n c_n\exp\left(-\frac{i}{\hbar}\hat{H}t\right)\hat{H}\Psi_{E_n} \\
&= \sum_n c_n\exp\left(-\frac{i}{\hbar}\hat{H}t\right)E_n\Psi_{E_n} \tag{2.13} \\
&= \exp\left(-\frac{i}{\hbar}\hat{H}t\right)\sum_n c_nE_n\Psi_{E_n}
\end{align*}

And for the Hamiltonian acting on the time-evolved state:
\begin{align*}
\hat{H}\sum_n c_n\exp\left(-\frac{i}{\hbar}\hat{H}t\right)\Psi_{E_n} &= \sum_n c_n\hat{H}\exp\left(-\frac{i}{\hbar}\hat{H}t\right)\Psi_{E_n} \\
&= \sum_n c_n\exp\left(-\frac{i}{\hbar}\hat{H}t\right)\hat{H}\Psi_{E_n} \\
&= \sum_n c_n\exp\left(-\frac{i}{\hbar}\hat{H}t\right)E_n\Psi_{E_n} \tag{2.14} \\
&= \exp\left(-\frac{i}{\hbar}\hat{H}t\right)\sum_n c_nE_n\Psi_{E_n}
\end{align*}

Since the right-hand sides of equations (2.13) and (2.14) are identical, the propagator formulation satisfies the Schrödinger equation.

Alternatively, we can verify this using the Taylor expansion of the propagator:
\begin{align*}
i\hbar\frac{\partial}{\partial t}\sum_{n=0}^{\infty}\frac{\left(-\frac{i}{\hbar}t\right)^n}{n!}\hat{H}^n\Psi(\vec{x},0) &= i\hbar\sum_{n=0}^{\infty}\frac{\partial}{\partial t}\frac{\left(-\frac{i}{\hbar}t\right)^n}{n!}\hat{H}^n\Psi(\vec{x},0) \\
&= i\hbar\sum_{n=0}^{\infty}\frac{n\left(-\frac{i}{\hbar}\right)^nt^{n-1}}{n!}\hat{H}^n\Psi(\vec{x},0) \\
&= \sum_{n=1}^{\infty}\frac{\left(-\frac{i}{\hbar}\right)^nt^{n-1}}{(n-1)!}\hat{H}^n\Psi(\vec{x},0) \tag{2.15} \\
&= \sum_{s=0}^{\infty}\frac{\left(-\frac{i}{\hbar}\right)^{s+1}t^s}{s!}\hat{H}^{s+1}\Psi(\vec{x},0) \\
&= \hat{H}\sum_{s=0}^{\infty}\frac{\left(-\frac{i}{\hbar}t\right)^s}{s!}\hat{H}^s\Psi(\vec{x},0) \\
&= \hat{H}\exp\left(-\frac{i}{\hbar}\hat{H}t\right)\Psi(\vec{x},0)
\end{align*}

This confirms that $\exp\left(-\frac{i}{\hbar}\hat{H}t\right)\Psi(\vec{x},0)$ satisfies the Schrödinger equation.

For systems with continuous energy spectra (rather than discrete eigenvalues), the superposition becomes an integral:

\begin{equation*}
\int_V c(E)\exp\left(-\frac{i}{\hbar}\hat{H}t\right)dE \tag{2.16}
\end{equation*}

This continuous spectrum formulation is particularly relevant for unbound states, such as free particles, where energy can take any value in a continuous range.


\section*{3 Free particle}

The free particle represents a foundational problem in quantum mechanics. While classically a free particle follows a simple trajectory:

\begin{equation*}
\vec{r}(t) = \vec{r}_0 + \vec{v}t \tag{3.1}
\end{equation*}

In quantum mechanics, we must solve the Schrödinger equation with the Hamiltonian:

\begin{equation*}
\hat{H} = -\frac{\hbar^2}{2m}\nabla^2 \tag{3.2}
\end{equation*}

where the potential energy term is zero. The time-independent Schrödinger equation becomes:

\begin{equation*}
\hat{H}\Psi = E\Psi \tag{3.3}
\end{equation*}

For a free particle, the energy spectrum is continuous, and the energy depends on the wave vector $\vec{k}$, which is related to momentum by $\vec{p} = \hbar\vec{k}$. The stationary solutions have the form:

\begin{equation*}
\Psi_{E(\vec{k})} = e^{-i\frac{E(\vec{k})}{\hbar}t}\Psi_k = e^{-i\frac{E(\vec{k})}{\hbar}t}\frac{e^{i\vec{k}\cdot\vec{x}}}{\sqrt{2\pi}} \tag{3.4}
\end{equation*}

The normalization factor $\frac{1}{\sqrt{2\pi}}$ ensures proper normalization in the context of continuous spectra, where we work with wave packets rather than individual plane waves.

The general solution is formed by superposing these plane wave solutions across all possible wave vectors:

\begin{align*}
\Psi &= \int_V d^3k\,a(k)e^{-i\frac{E(\vec{k})}{\hbar}t}\Psi_k \\
&= \int_V d^3k\,a(k)e^{-i\frac{E(\vec{k})}{\hbar}t}\frac{1}{\sqrt{2\pi}}e^{i\vec{k}\cdot\vec{x}} \tag{3.5} \\
&= \int_V d^3k\,a(k)\frac{1}{\sqrt{2\pi}}e^{i\left(\vec{k}\cdot\vec{x} - \frac{\hbar\vec{k}^2}{2m}t\right)}
\end{align*}

Defining the angular frequency $\omega := \frac{\hbar\vec{k}^2}{2m}$, which represents the energy divided by $\hbar$:

\begin{equation*}
\Psi = \int_V d^3k\,a(k)\frac{1}{\sqrt{2\pi}}e^{i(\vec{k}\cdot\vec{x} - \omega t)} \tag{3.6}
\end{equation*}

This integral represents a superposition of plane waves with different wave vectors, each propagating with its own frequency determined by the dispersion relation $\omega = \frac{\hbar k^2}{2m}$. The coefficient function $a(k)$ determines the contribution of each plane wave to the overall wavefunction.

Unlike classical particles, a quantum free particle is described by a wave packet that generally spreads over time. Even if initially localized, the wave packet will disperse as different momentum components travel at different speeds, a distinctly quantum phenomenon with no classical analog.

The probability density $|\Psi|^2$ evolves in time, showing how the particle's position becomes increasingly uncertain—a manifestation of Heisenberg's uncertainty principle connecting position and momentum uncertainties.


Verifying that our proposed solution satisfies the Schrödinger equation:

For the time derivative:
\begin{align*}
i\hbar\frac{\partial\Psi}{\partial t} &= i\hbar\frac{\partial}{\partial t}\int_V d^3k\,a(k)\frac{1}{\sqrt{2\pi}}e^{i(\vec{k}\cdot\vec{x}-\omega t)} \\
&= \int_V d^3k\,a(k)\frac{1}{\sqrt{2\pi}}(-\hbar\omega i)e^{i(\vec{k}\cdot\vec{x}-\omega t)} \\
&= \int_V d^3k\,a(k)\frac{1}{\sqrt{2\pi}}\hbar\omega e^{i(\vec{k}\cdot\vec{x}-\omega t)} \tag{3.7} \\
&= \int_V d^3k\,a(k)\frac{1}{\sqrt{2\pi}}\left(\frac{\hbar^2\vec{k}^2}{2m}\right)e^{i(\vec{k}\cdot\vec{x}-\omega t)}
\end{align*}

For the Hamiltonian acting on the wavefunction:
\begin{align*}
\hat{H}\Psi &= \hat{H}\int_V d^3k\,a(k)\frac{1}{\sqrt{2\pi}}e^{i(\vec{k}\cdot\vec{x}-\omega t)} \\
&= \int_V d^3k\,a(k)\frac{1}{\sqrt{2\pi}}\hat{H}e^{i(\vec{k}\cdot\vec{x}-\omega t)} \\
&= \int_V d^3k\,a(k)\frac{1}{\sqrt{2\pi}}\left(-\frac{\hbar^2}{2m}\nabla^2\right)e^{i(\vec{k}\cdot\vec{x}-\omega t)} \tag{3.8} \\
&= \int_V d^3k\,a(k)\frac{1}{\sqrt{2\pi}}\left(\frac{\hbar^2\vec{k}^2}{2m}\right)e^{i(\vec{k}\cdot\vec{x}-\omega t)}
\end{align*}

Since the results from equations (3.7) and (3.8) are identical, our solution satisfies the Schrödinger equation.

Now we consider the physical meaning of the coefficient function $a(k)$. For an electron beam from an emitter, $a(k)$ represents a distribution of wave vectors centered around a classical momentum $\vec{q}$. This can be modeled as a Gaussian distribution:

\begin{equation*}
a(k) = Ae^{-d^2(\vec{k}-\vec{q})^2} \tag{3.9}
\end{equation*}

When $\vec{k} \sim \vec{q}$, the particle behaves approximately like a classical particle. Substituting this expression for $a(k)$ into equation (3.6):

\begin{equation*}
\Psi = \int_{\mathbb{R}^3}d^3k\,Ae^{-d^2(\vec{k}-\vec{q})^2}\frac{1}{\sqrt{2\pi}}e^{i(\vec{k}\cdot\vec{x}-\omega t)} \tag{3.10}
\end{equation*}

Since the contributions along each dimension are independent, we can factorize this into three one-dimensional integrals:

\begin{equation*}
\Psi = \int_{\mathbb{R}^3}\prod_{j=1}^3 dk_j\,A_je^{-d^2(k_j-q_j)^2}\frac{1}{\sqrt{2\pi}}e^{i(k_jx_j-\frac{\hbar k_j^2}{2m}t)} \tag{3.11}
\end{equation*}


To solve this integral, we introduce a simplification by defining:

\begin{equation*}
s := \frac{\hbar t}{2m} \tag{3.12}
\end{equation*}

This allows us to rewrite our wavefunction as:

\begin{align*}
\Psi &= \int_{\mathbb{R}^3}\prod_{j=1}^3dk_j\,A_je^{-d^2(k_j-q_j)^2}\frac{1}{\sqrt{2\pi}}e^{i(k_jx_j-sk_j^2)} \\
&= \prod_{j=1}^3\frac{A_j}{\sqrt{2\pi}}\underbrace{\int_{\mathbb{R}}dk_j\,e^{-d^2(k_j-q_j)^2+i(k_jx_j-sk_j^2)}}_{I} \tag{3.13}
\end{align*}

To evaluate the integral $I$, we drop the indices temporarily for clarity:

\begin{align*}
\phi &= \int_{\mathbb{R}}dk\,e^{-d^2k^2+2d^2kq-d^2q^2+ikx-isk^2} \\
&= e^{-d^2q^2}\int_{\mathbb{R}}dk\,e^{-d^2k^2+2d^2kq+ikx-isk^2} \tag{3.14} \\
&= e^{-d^2q^2}\int_{\mathbb{R}}dk\,e^{-k^2(d^2+is)+k(2d^2q+ix)}
\end{align*}

This is a Gaussian integral that we can evaluate by defining:

\begin{align*}
\nu &:= d^2+is \\
\eta &:= 2d^2q+ix \tag{3.15}
\end{align*}

To complete the square in the exponent, we add and subtract $\frac{\eta^2}{4\nu}$:

\begin{align*}
\phi &= e^{-d^2q^2}\int_{\mathbb{R}}dk\,e^{-\nu k^2+\eta k} \\
&= e^{-d^2q^2}\int_{\mathbb{R}}dk\,e^{-\nu k^2+\eta k-\frac{\eta^2}{4\nu}+\frac{\eta^2}{4\nu}} \tag{3.16} \\
&= e^{-d^2q^2+\frac{\eta^2}{4\nu}}\int_{\mathbb{R}}dk\,e^{-\nu\left(k-\frac{\eta}{2\nu}\right)^2}
\end{align*}

This integral is a standard Gaussian form that can be evaluated using various methods, such as Feynman's trick or polar coordinates in two dimensions.


The solution to the Gaussian integral gives us:

\begin{equation*}
\phi = e^{-d^2q^2+\frac{\eta^2}{4\nu}}\sqrt{\frac{\pi}{\nu}} \tag{3.17}
\end{equation*}

Since this expression generally contains complex values, we need to calculate the probability density $\rho$ by evaluating:

\begin{equation*}
\rho = |\Psi|^2 = \left|\frac{A_1}{\sqrt{2\pi}}\phi_1\right|^2\left|\frac{A_2}{\sqrt{2\pi}}\phi_2\right|^2\left|\frac{A_3}{\sqrt{2\pi}}\phi_3\right|^2 \tag{3.18}
\end{equation*}

For this, we need to compute:

\begin{gather*}
|\phi|^2 = \left|\sqrt{\frac{\pi}{\nu}}e^{\text{Re}(z)+i\text{Im}(z)}\right|^2 = \frac{\pi}{|\nu|}e^{2\text{Re}(z)} \tag{3.19} \\
-d^2q^2+\frac{\eta^2}{4\nu} = -d^2q^2+\frac{(d^2-is)(2d^2q+ix)^2}{4(d^4+s^2)} = \\
= -d^2q^2+\frac{(d^2-is)(4d^4q^2+4id^2qx-x^2)}{4(d^4+s^2)} = \\
= -d^2q^2+\frac{d^2(4d^4q^2+4id^2qx-x^2)-is(4d^4q^2+4id^2qx-x^2)}{4(d^4+s^2)} = \tag{3.20} \\
= -d^2q^2+\frac{d^2(4d^4q^2+4sqx-x^2)}{4(d^4+s^2)}-i\frac{(4sd^4q^2+4d^4qx-sx^2)}{4(d^4+s^2)}
\end{gather*}

Therefore:

\begin{align*}
\text{Re}(z) &= -d^2q^2+\frac{d^2(4d^4q^2+4sqx-x^2)}{4(d^4+s^2)} \\
&= -d^2q^2+\frac{d^2(4d^4q^2+4s^2q^2-4s^2q^2+4sqx-x^2)}{4(d^4+s^2)} \\
&= -d^2q^2+\frac{d^2(4d^4q^2+4s^2q^2-(x-2sq)^2)}{4(d^4+s^2)} \\
&= -d^2q^2+\frac{4d^2q^2(d^4+s^2)-d^2(x-2sq)^2}{4(d^4+s^2)} \tag{3.21} \\
&= -d^2q^2+d^2q^2-\frac{d^2(x-2sq)^2}{4(d^4+s^2)} \\
&= -\frac{d^2(x-2sq)^2}{4(d^4+s^2)}
\end{align*}

Substituting the value of $s$ back:

\begin{equation*}
2sq = 2\frac{\hbar t}{2m}q = \frac{\hbar q}{m}t = vt \tag{3.22}
\end{equation*}

where $v = \frac{\hbar q}{m}$ is the velocity.

Finally, we can evaluate one component of the probability density:

\begin{align*}
\left|\frac{A_j}{\sqrt{2\pi}}\phi_j\right|^2 &= \frac{|A_j|^2}{2\pi}\frac{\pi}{|\nu|}e^{-\frac{d^2(x_j-v_jt)^2}{4(d^4+s^2)}} \\
&= \frac{|A_j|^2}{2\sqrt{d^4+s^2}}e^{-\frac{d^2(x_j-v_jt)^2}{4(d^4+s^2)}} \tag{3.23}
\end{align*}

This represents a Gaussian wave packet that moves with velocity $v_j$ along the $x_j$-axis while spreading out over time. At $t=0$, it is centered at $x=0$, and as time progresses, it both translates and spreads until it becomes essentially flat.

\section*{4 Hilbert spaces}
\subsection*{4.1 Vector space definition and properties}

We begin with a theoretical introduction to Hilbert spaces. First, let's recall the definition of a vector space:

Definition 4.1. A linear vector space $L$ over a field $F$ (usually either $F=\mathbb{R}$ or $F=\mathbb{C}$) is a set of elements (called vectors):

\begin{equation*}
|a\rangle, |b\rangle, |c\rangle, \ldots \tag{4.1}
\end{equation*}

equipped with two operations:
\begin{itemize}
  \item vector addition: $\forall|x\rangle,|y\rangle \in L \longrightarrow |x\rangle+|y\rangle=|z\rangle \in L$
  \item scalar multiplication: $\forall \lambda \in F, \forall|x\rangle \in L \longrightarrow \lambda|x\rangle=|v\rangle \in L$
\end{itemize}

These operations must satisfy the following axioms:
\begin{enumerate}
  \item $|x\rangle+|y\rangle=|y\rangle+|x\rangle$ (commutativity of addition)
  \item $(|x\rangle+|y\rangle)+|z\rangle=|x\rangle+(|y\rangle+|z\rangle)=|x\rangle+|y\rangle+|z\rangle$ (associativity of addition)
  \item $(\lambda\gamma)|x\rangle=\lambda(\gamma|x\rangle)=\lambda\gamma|x\rangle$ (associativity of multiplication)
  \item $(\lambda+\gamma)|x\rangle=\lambda|x\rangle+\gamma|x\rangle, \lambda(|x\rangle+|y\rangle)=\lambda|x\rangle+\lambda|y\rangle$ (distributivity of multiplication)
  \item $|0\rangle+|x\rangle=|x\rangle$ (zero element of addition)
  \item $|x'\rangle+|x\rangle=|0\rangle \Longrightarrow |x'\rangle=-|x\rangle$ (inverse element of addition)
\end{enumerate}

Definition 4.2. A set of vectors is said to be linearly independent if:


\begin{equation*}
\sum_{k=1}^{N} c_k|x_k\rangle = 0 \Longleftrightarrow c_k = 0 \;\forall k \tag{4.2}
\end{equation*}

Definition 4.3. A set $\mathcal{B}(L) = \{|x_k\rangle: k \in [1,N]\}$ of $N$ linearly independent vectors forms a basis if any vector $|v\rangle \in L$ can be written as a linear combination of vectors $|x_k\rangle$:

\begin{equation*}
|v\rangle = \sum_{k=1}^{N} v_k|x_k\rangle \tag{4.3}
\end{equation*}

Equivalently, a set of linearly independent vectors is said to be complete if it forms a basis.

Definition 4.4. A scalar product (or inner product) is an operation $(x,y)$ that associates a number from the field $F$ to any pair of vectors $|x\rangle$ and $|y\rangle$. From now on, we assume $F = \mathbb{C}$. The scalar product satisfies the following properties:

\begin{enumerate}
  \item $(x, c_1y_1 + c_2y_2) = c_1(x,y_1) + c_2(x,y_2)$
  \item $(c_1x_1 + c_2x_2, y) = c_1^*(x_1,y) + c_2^*(x_2,y)$
\end{enumerate}

Due to the second property (the product is antilinear in the first argument), such a scalar product is said to be sesquilinear. Note that if the field $F = \mathbb{R}$, the scalar product is simply bilinear because $c_i = c_i^*$.

We also define the dual vector:

\begin{equation*}
|v\rangle = \sum_{k=1}^{N} v_k|x_k\rangle \longleftrightarrow \langle v| = \sum_{k=1}^{N} v_k^*\langle x_k| \tag{4.4}
\end{equation*}

From this the scalar product can be defined as:

\begin{equation*}
\langle a|b\rangle := \sum_{j=1}^{N}\sum_{k=1}^{N} a_j^*b_k\langle x_j|x_k\rangle \tag{4.5}
\end{equation*}

And the norm is defined as:

\begin{equation*}
|a| := \sqrt{\langle a|a\rangle} = \sqrt{\sum_{j=1}^{N}\sum_{k=1}^{N} a_j^*a_k\langle x_j|x_k\rangle} \tag{4.6}
\end{equation*}

Definition 4.5. A set of vectors $\mathcal{B}(L)$ forms an orthonormal basis if $\mathcal{B}(L) = \{|k\rangle: k \in [1,N]\}$ is such that:

\[
\langle j|k\rangle = \delta_{jk} = \begin{cases}
0 & \text{if } j \neq k \\
1 & \text{if } j = k
\end{cases} \tag{4.7}
\]

Thus if we consider an orthonormal vector space the scalar product simplifies to:

\begin{equation*}
\sum_{j=1}^{N}\sum_{k=1}^{N} a_j^*b_k\langle x_j|x_k\rangle = \sum_{j=1}^{N}\sum_{k=1}^{N} a_j^*b_k\delta_{jk} = \sum_{k=1}^{N} a_k^*b_k \tag{4.8}
\end{equation*}

And also the norm reduces to:

\begin{equation*}
|a| := \sqrt{\langle a|a\rangle} = \sqrt{\sum_{k=1}^{N} a_k^*a_k} = \sqrt{\sum_{k=1}^{N}|a_k|^2} \tag{4.9}
\end{equation*}

To identify the difference between vectors and dual vectors introduced in the Dirac picture of quantum mechanics we can identify them as column vectors and row vectors. In particular the vectors in Dirac's notation also called ket are column vectors, instead the dual vectors also called bra are the row vectors:

\[
|a\rangle = \begin{bmatrix}
a_1 \\
a_2 \\
\ldots \\
a_N
\end{bmatrix} \quad
\langle a| = \begin{bmatrix}
a_1^*, & a_2^*, & \ldots, & a_N^*
\end{bmatrix} \tag{4.10}
\]

We denote the standard basis of elements as the set of vectors and dual vectors $|k\rangle, \langle k|$ such that:

\[
|k\rangle = \begin{bmatrix}
0 \\
\cdots \\
1 \\
0 \\
\cdots \\
0
\end{bmatrix} \quad
\langle k| = \begin{bmatrix}
0, & \ldots, & 1, & 0, & \ldots, & 0
\end{bmatrix} \tag{4.11}
\]

Where the only element equal to 1 is the k-th element.

If we choose $F$ as the set of real numbers $\mathbb{R}$ and the basis of vectors:

\begin{align*}
&\langle x_1| = (1,0,0) \\
&\langle x_2| = (0,1,0) \tag{4.12} \\
&\langle x_3| = (0,0,1)
\end{align*}

we are dealing with the Euclidean vector space which is the usual 3D physical space.


\subsection*{4.2 Hilbert space}

Let's finally give the definition of a Hilbert space:

Definition 4.6. A Hilbert space is an infinite-dimensional linear vector space over the field $F = \mathbb{C}$ whose vectors have finite norm. Given the standard basis we have:

\begin{equation*}
|v\rangle = \sum_{k}^{\infty} v_k|k\rangle \tag{4.13}
\end{equation*}

And so the norm squared must be:

\begin{equation*}
|v|^2 = \langle v|v\rangle = \sum_{k=1}^{\infty}|v_k|^2 < \infty \tag{4.14}
\end{equation*}

This means that the series generated from the square modulus of the vector $v$ is the result of a convergent series and so the modulus of the scalar product of any two vectors in a Hilbert space must be finite. This follows from the Cauchy-Schwarz inequality:

\begin{equation*}
|\langle a|b\rangle|^2 \leq |a|^2|b|^2 < \infty \tag{4.15}
\end{equation*}

Let us now focus on a particular Hilbert space. We said that all wave functions must satisfy the normalization condition:

\begin{equation*}
(\Psi, \Psi) = 1 \tag{4.16}
\end{equation*}

And so if we go back to the definition of $L^2$ (1.50) space we can notice that this space is a vector space that satisfies the properties to be a Hilbert space.

We can notice that the property of additivity of the Hilbert space is expected since the wave functions must satisfy the superposition principle. As we anticipated the space $L^2(V)$ has a scalar product (previously defined). We also have a partially different definition of a complete orthonormal basis for $L^2(V)$:

Definition 4.7. A set of functions $\mathcal{B} = \{\phi_k: k \in [1,\infty]\}$ is a complete (orthonormal) basis of $L^2(V)$ if any wave function is equivalent to its Fourier series:

\begin{equation*}
\Psi(\vec{x}) = \sum_{k=1}^{\infty} c_k\phi_k(\vec{x}) \quad \text{where} \quad c_k = (\phi_k, \Psi) \tag{4.17}
\end{equation*}

The formula for the coefficient can be proven using the fact that by construction $(\phi_i, \phi_j) = \delta_{ij}$:

Proof.
\begin{align*}
(\phi_k, \Psi) &= \int_V d^3x\,\phi_k^*\Psi = \int_V d^3x\,\phi_k^*\sum_{n=1}^{\infty}c_n\phi_n = \\
&= \sum_{n=1}^{\infty}c_n\int_V d^3x\,\phi_k^*\phi_n = \sum_{n=1}^{\infty}c_n(\phi_k, \phi_n) = \tag{4.18} \\
&= \sum_{n=1}^{\infty}c_n\delta_{kn} = c_k
\end{align*}

This means that given a wave function and a basis the wave function itself can be identified by an infinite vector containing all the $c_k$ coefficients. Another consequence of the Fourier expansion of the wave function is that the total probability is given by:

\begin{align*}
(\Psi, \Psi) &= \int_V d^3x\,\Psi^*\Psi = \int_V d^3x\,\sum_{n=1}^{\infty}c_n^*\phi_n^*\sum_{k=1}^{\infty}c_k\phi_k = \\
&= \sum_{k=1}^{\infty}\sum_{n=1}^{\infty}c_n^*c_k\int_V d^3x\,\phi_n^*\phi_k = \sum_{k=1}^{\infty}\sum_{n=1}^{\infty}c_n^*c_k(\phi_n, \phi_k) = \\
&= \sum_{k=1}^{\infty}\sum_{n=1}^{\infty}c_n^*c_k\delta_{nk} = \sum_{k=1}^{\infty}c_k^*c_k = \sum_{k=1}^{\infty}|c_k|^2 \tag{4.19} \\
&(\Psi, \Psi) = \sum_{k=1}^{\infty}|c_k|^2
\end{align*}

This identity is also known as the Parseval's equation (or Parseval's identity). This equation is true for a generic function only if the base $\mathcal{B}$ is complete, otherwise it can be proven that:

\begin{equation*}
(f, f) \geq \sum_{k=1}^{\infty}|f_k|^2 \tag{4.20}
\end{equation*}

In the case of wave functions if we take a complete basis we can also state:

\begin{equation*}
(\Psi, \Psi) = \sum_{k=1}^{\infty}|c_k|^2 = 1 \tag{4.21}
\end{equation*}

And so every term of the series $|c_k|^2$ can be interpreted as an elementary probability related to the eigenstate $\phi_k$ of the wave function. Also, we notice that in general we do not know what the k factor means and it could depend on various physical quantities such as the energy, the angular momentum, etc.

We can now continue our discussion about Hilbert spaces by stating a new theorem:

Theorem 4.1. (Fischer-Riesz) Given a Hilbert space with an orthonormal basis $\mathcal{B} = \{\phi_k \in L^2(V): k \in [1,\infty]\}$ the condition:

\begin{equation*}
\sum_{k=1}^{\infty}|c_k|^2 < \infty \tag{4.22}
\end{equation*}

is necessary and sufficient to grant the fact that there exist a function $\Psi \in L^2(V)$ such that:


The completeness relation in Hilbert space provides a powerful framework for understanding quantum systems. When working with an orthonormal basis, we can express any function in the space through its expansion coefficients:

\begin{equation*}
\Psi = \sum_{k=1}^{\infty} c_k\phi_k \tag{4.23}
\end{equation*}

\subsection*{4.3 Completeness relation}

For a finite-dimensional vector space $L$ with complete basis $\mathcal{B} = \{|k\rangle: k \in [1,N]\}$, the scalar product of any vector with a basis element yields that vector's component along the basis direction:

\begin{equation*}
\langle i|v\rangle = \sum_k v_k\langle i|k\rangle = v_i \tag{4.24}
\end{equation*}

This fundamental property allows us to reconstruct any vector through its projections:

\begin{equation*}
|v\rangle = \sum_k v_k|k\rangle = \sum_k |k\rangle\langle k|v\rangle = \mathbb{I}|v\rangle \tag{4.25}
\end{equation*}

From this, we derive the completeness relation:

\begin{equation*}
\mathbb{I} = \sum_k |k\rangle\langle k| \tag{4.26}
\end{equation*}

The satisfaction of this relation confirms that basis $\mathcal{B}$ is indeed complete. In matrix form, each term $|k\rangle\langle k|$ represents a matrix with a single non-zero element:

\[
|k\rangle\langle k| = \begin{bmatrix}
0 & 0 & \ldots & 0 & 0 \\
0 & \ldots & \ldots & \ldots & 0 \\
\ldots & \ldots & 1 & \ldots & \ldots \\
0 & \ldots & \ldots & \ldots & 0 \\
0 & 0 & \ldots & 0 & 0
\end{bmatrix} \tag{4.27}
\]

When summed over all values of $k$, these matrices yield the identity:

\[
\mathbb{I} = \begin{bmatrix}
1 & 0 & 0 & 0 & 0 \\
0 & 1 & 0 & 0 & 0 \\
0 & 0 & 1 & 0 & 0 \\
0 & 0 & 0 & 1 & 0 \\
0 & 0 & 0 & 0 & \ldots
\end{bmatrix} \tag{4.28}
\]

Extending to infinite-dimensional function spaces requires the Dirac delta function. Using the Fourier expansion of a wavefunction:

\begin{align*}
\int_V d^3y \left(\sum_k \phi_k^*(\vec{y})\phi_k(\vec{x})\right)\Psi(\vec{y}) &= \sum_k \left(\int_V d^3y\, \phi_k^*(\vec{y})\Psi(\vec{y})\right)\phi_k(\vec{x}) \\
&= \sum_k (\phi_k(\vec{y}), \Psi(\vec{y}))\phi_k(\vec{x}) \\
&= \sum_k c_k\phi_k(\vec{x}) = \Psi(\vec{x}) \tag{4.29}
\end{align*}

For this equation to hold, we must have:

\begin{equation*}
\sum_k \phi_k^*(\vec{y})\phi_k(\vec{x}) = \delta^3(\vec{y}-\vec{x}) \tag{4.30}
\end{equation*}

The Dirac delta function acts on functions in one dimension according to:

\begin{equation*}
\int_V dy\, f(y)\delta(y-x) = f(x) \tag{4.31}
\end{equation*}

In three dimensions, this generalizes to:

\begin{align*}
\int_{X \times Y \times Z} d^3y\, f(\vec{y})\delta^3(\vec{y}-\vec{x}) &= \int_X dy_1 \int_Y dy_2 \int_Z dy_3\, \delta(y_1-x_1)\delta(y_2-x_2)\delta(y_3-x_3)f(y_1,y_2,y_3) \\
&= f(x_1,x_2,x_3) \tag{4.32}
\end{align*}

Thus, the completeness relation in Hilbert space takes the form:

\begin{equation*}
\sum_{k=1}^{\infty} \phi_k^*(\vec{y})\phi_k(\vec{x}) = \delta^3(\vec{y}-\vec{x}) \tag{4.33}
\end{equation*}

\subsection*{4.4 Examples}
\subsection*{4.4.1 Plane wave basis (finite volume)}
Let's examine a box of volume $V = L^3$ to verify if the basis:


\begin{equation*}
\mathcal{B} = \left\{\phi_k = \frac{e^{i\vec{k}\cdot\vec{x}}}{\sqrt{L^3}}: \vec{k} = \frac{2\pi}{L}(n_1, n_2, n_3), n_i \in \mathbb{Z}\right\} \tag{4.34}
\end{equation*}

This set of plane waves exhibits L-periodic boundary conditions, making it particularly useful for analyzing quantum systems in finite volumes. The periodicity manifests as:

\begin{equation*}
\phi_k(x_1, x_2, x_3) = \phi_k(x_1+L, x_2, x_3) = \phi_k(x_1, x_2+L, x_3) = \phi_k(x_1, x_2, x_3+L) \tag{4.35}
\end{equation*}

We can verify this property by direct substitution. Taking the first periodicity condition:

\begin{align*}
\phi_k(x_1+L, x_2, x_3) &= \frac{e^{i(x_1k_1+Lk_1+x_2k_2+x_3k_3)}}{\sqrt{L^3}} \\
&= \frac{e^{i\vec{k}\cdot\vec{x}}}{\sqrt{L^3}}e^{iLk_1} \\
&= \frac{e^{i\vec{k}\cdot\vec{x}}}{\sqrt{L^3}}e^{i2\pi n_1} = \phi_k(x_1, x_2, x_3) \tag{4.36}
\end{align*}

Since $e^{i2\pi n_1} = 1$ for any integer $n_1$, the periodicity is confirmed.

To establish that these functions form an orthonormal basis, we must verify their orthogonality. The scalar product between two basis elements is:

\begin{align*}
(\phi_k, \phi_q) &= \int_V d^3x\, \frac{e^{-i\vec{k}\cdot\vec{x}}}{\sqrt{L^3}} \frac{e^{i\vec{q}\cdot\vec{x}}}{\sqrt{L^3}} \\
&= \int_V d^3x\, \frac{e^{i(\vec{q}-\vec{k})\cdot\vec{x}}}{L^3} \tag{4.37}
\end{align*}

For the case where $\vec{q} = \vec{k}$, this simplifies to:

\begin{equation*}
(\phi_k, \phi_q) = \frac{1}{L^3}\int_V d^3x = \frac{V}{V} = 1 \tag{4.38}
\end{equation*}

When $\vec{q} \neq \vec{k}$, the integral can be factorized into three one-dimensional integrals:

\begin{equation*}
\prod_{j=1}^3 \int_V dx_j\, \frac{e^{i(q_j-k_j)x_j}}{L^3} = \left.\frac{1}{L^3}\prod_{j=1}^3 \frac{e^{i(q_j-k_j)x_j}}{i(q_j-k_j)}\right|_{-L/2}^{L/2} \tag{4.39}
\end{equation*}

Evaluating each factor separately:

\begin{align*}
I &= \left.\frac{e^{i(q_j-k_j)x_j}}{i(q_j-k_j)}\right|_{-L/2}^{L/2} \\
&= \frac{e^{i(q_j-k_j)L/2}}{i(q_j-k_j)} - \frac{e^{-i(q_j-k_j)L/2}}{i(q_j-k_j)} \\
&= (\ldots)\sin((q_j-k_j)L/2) \tag{4.40} \\
&= (\ldots)\sin((\frac{2\pi}{L}n_j-\frac{2\pi}{L}m_j)L/2) \\
&= (\ldots)\sin(z_j\pi) = 0
\end{align*}

Since $z_j$ is an integer, $\sin(z_j\pi) = 0$, confirming orthogonality.

The completeness relation requires:

\begin{align*}
\sum_k \phi_k^*(\vec{x})\phi_k(\vec{y}) &= \frac{1}{L^3}\sum_k e^{-i\vec{k}\cdot\vec{x}}e^{i\vec{k}\cdot\vec{y}} \\
&= \frac{1}{L^3}\sum_k e^{i\vec{k}\cdot(\vec{y}-\vec{x})} = \delta^3(\vec{y}-\vec{x}) \tag{4.41}
\end{align*}

This identity, a fundamental result in functional analysis, confirms that our basis is complete.

\section*{5 Harmonic oscillator}
\subsection*{5.1 Description of the problem}

The harmonic oscillator represents one of the most important models in quantum mechanics. Its classical Hamiltonian is given by:

\begin{equation*}
\mathcal{H} = \frac{p^2}{2m} + \frac{1}{2}m\omega^2 x^2 \tag{5.1}
\end{equation*}

To transition to quantum mechanics, we substitute the appropriate operators:

\begin{align*}
&p \longrightarrow \hat{p} = -i\hbar\frac{\partial}{\partial x} \tag{5.2} \\
&x \longrightarrow \hat{x}
\end{align*}

The resulting Hamiltonian operator is:


\begin{equation*}
\hat{H} = -\frac{\hbar^2}{2m}\frac{\partial^2}{\partial x^2} + \frac{1}{2}m\omega^2\hat{x}^2 \tag{5.3}
\end{equation*}

The quantum harmonic oscillator represents one of the most elegant problems in quantum mechanics. To find its energy spectrum, we need to solve the eigenvalue equation:

\begin{equation*}
\hat{H}\Psi_E = E\Psi_E \tag{5.4}
\end{equation*}

\subsection*{5.2 Solution of the eigenvalue equation}

Rather than directly solving the differential equation, we'll employ the powerful spectrum generating algebra method by introducing ladder operators. These operators allow us to navigate between energy states with remarkable simplicity:

\begin{align*}
\hat{a} &= \sqrt{\frac{m\omega}{2\hbar}}\left(\hat{x} + i\frac{\hat{p}}{m\omega}\right) \tag{5.5} \\
\hat{a}^\dagger &= \sqrt{\frac{m\omega}{2\hbar}}\left(\hat{x} - i\frac{\hat{p}}{m\omega}\right)
\end{align*}

A characteristic length scale emerges naturally in this problem:

\begin{equation*}
\lambda = \sqrt{\frac{\hbar}{m\omega}} \tag{5.6}
\end{equation*}

Using this length scale, we can express the ladder operators in a more compact form:

\begin{align*}
\hat{a} &= \frac{1}{\sqrt{2}}\left(\frac{\hat{x}}{\lambda} + i\frac{\lambda\hat{p}}{\hbar}\right) \tag{5.7} \\
\hat{a}^\dagger &= \frac{1}{\sqrt{2}}\left(\frac{\hat{x}}{\lambda} - i\frac{\lambda\hat{p}}{\hbar}\right)
\end{align*}

The commutation relation between these operators reveals their fundamental property. First calculating $\hat{a}\hat{a}^\dagger$:

\begin{align*}
\hat{a}\hat{a}^\dagger &= \frac{1}{2}\left(\frac{\hat{x}}{\lambda} + i\frac{\lambda\hat{p}}{\hbar}\right)\left(\frac{\hat{x}}{\lambda} - i\frac{\lambda\hat{p}}{\hbar}\right) \\
&= \frac{1}{2}\left[\left(\frac{\hat{x}}{\lambda}\right)^2 - i\frac{\hat{x}}{\lambda}\frac{\lambda\hat{p}}{\hbar} + i\frac{\lambda\hat{p}}{\hbar}\frac{\hat{x}}{\lambda} + \left(\frac{\lambda\hat{p}}{\hbar}\right)^2\right] \\
&= \frac{1}{2}\left[\frac{\hat{x}^2}{\lambda^2} + \frac{\lambda^2\hat{p}^2}{\hbar^2} + \frac{i}{\hbar}[\hat{p},\hat{x}]\right] \tag{5.8} \\
&= \frac{1}{2}\left[\frac{\hat{x}^2}{\lambda^2} + \frac{\lambda^2\hat{p}^2}{\hbar^2} + 1\right]
\end{align*}

And similarly for $\hat{a}^\dagger\hat{a}$:

\begin{align*}
\hat{a}^\dagger\hat{a} &= \frac{1}{2}\left[\frac{\hat{x}^2}{\lambda^2} + \frac{\lambda^2\hat{p}^2}{\hbar^2} - 1\right] \tag{5.9}
\end{align*}

From these calculations, we find the commutation relation:

\begin{equation*}
[a,a^\dagger] = 1 \tag{5.10}
\end{equation*}

We can invert the ladder operator definitions to express position and momentum in terms of these operators:

\begin{align*}
\hat{x} &= \sqrt{\frac{\hbar}{2m\omega}}(\hat{a} + \hat{a}^\dagger) \\
\hat{p} &= -i\sqrt{\frac{m\omega\hbar}{2}}(\hat{a} - \hat{a}^\dagger) \tag{5.11}
\end{align*}

Substituting these expressions into the Hamiltonian reveals a remarkably simple form:

\begin{align*}
\hat{H} &= \frac{1}{2m}\hat{p}^2 + \frac{1}{2}m\omega^2\hat{x}^2 \\
&= \frac{1}{2m}\left[-i\sqrt{\frac{m\omega\hbar}{2}}(\hat{a} - \hat{a}^\dagger)\right]^2 + \frac{1}{2}m\omega^2\left[\sqrt{\frac{\hbar}{2m\omega}}(\hat{a} + \hat{a}^\dagger)\right]^2 \\
&= -\frac{\hbar\omega}{4}(\hat{a} - \hat{a}^\dagger)^2 + \frac{\hbar\omega}{4}(\hat{a} + \hat{a}^\dagger)^2 \\
&= \frac{\hbar\omega}{4}[(\hat{a} + \hat{a}^\dagger)^2 - (\hat{a} - \hat{a}^\dagger)^2] \tag{5.12} \\
&= \frac{\hbar\omega}{4}[(\hat{a}^2 + \hat{a}\hat{a}^\dagger + \hat{a}^\dagger\hat{a} + \hat{a}^{\dagger2}) - (\hat{a}^2 - \hat{a}\hat{a}^\dagger - \hat{a}^\dagger\hat{a} + \hat{a}^{\dagger2})] \\
&= \frac{\hbar\omega}{2}(\hat{a}\hat{a}^\dagger + \hat{a}^\dagger\hat{a}) \\
&= \hbar\omega\left(\hat{a}^\dagger\hat{a} + \frac{1}{2}\right)
\end{align*}

This leads us to define the number operator $\hat{n} = \hat{a}^\dagger\hat{a}$, yielding:

\begin{equation*}
\hat{H} = \hbar\omega\left(\hat{n} + \frac{1}{2}\right) \tag{5.13}
\end{equation*}

To determine the energy spectrum, we must understand how $\hat{n}$ acts on the eigenstates:


Let's explore how the number operator $\hat{n}$ acts on energy eigenstates $\Psi_E$. First, observe that:

\begin{equation*}
(\Psi_n, \hat{n}\Psi_n) = (\Psi_n, \hat{a}^\dagger\hat{a}\Psi_n) = (\hat{a}\Psi_n, \hat{a}\Psi_n) = |\hat{a}\Psi_n|^2 \geq 0 \tag{5.14}
\end{equation*}

This confirms that the expectation value of $\hat{n}$ must be positive. If we denote the eigenvalue of $\hat{n}$ as $n$, we have:

\begin{equation*}
\hat{n}\Psi_n = n\Psi_n \tag{5.15}
\end{equation*}

The operators we've introduced form an algebraic structure:

\begin{equation*}
\{\hat{a}, \hat{a}^\dagger, \hat{n}, \mathbb{I}\} \tag{5.16}
\end{equation*}

These operators satisfy the following commutation relations:

\begin{align*}
&[\hat{a}, \hat{a}^\dagger] = 1 \\
&[\hat{a}, \hat{n}] = \hat{a} \\
&[\hat{a}^\dagger, \hat{n}] = -\hat{a}^\dagger \tag{5.17} \\
&[\hat{n}, \mathbb{I}] = 0 \\
&[\hat{a}, \mathbb{I}] = 0 \\
&[\hat{a}^\dagger, \mathbb{I}] = 0
\end{align*}

This collection of operators forms an algebra:

\begin{gather*}
\mathcal{A} = \{\hat{a}, \hat{a}^\dagger, \hat{n}, \mathbb{I}\} \tag{5.18} \\
b = \alpha_1\hat{n} + \alpha_2\mathbb{I} + \xi\hat{a} + \xi^*\hat{a}^\dagger \in \mathcal{A} \\
c = \beta_1\hat{n} + \beta_2\mathbb{I} + \eta\hat{a} + \eta^*\hat{a}^\dagger \in \mathcal{A} \tag{5.19}
\end{gather*}

With the Hermiticity requirements:

\begin{align*}
&b = b^\dagger \\
&c = c^\dagger \tag{5.20}
\end{align*}

And closure under commutation:

\begin{equation*}
[b,c] = f \in \mathcal{A} \tag{5.21}
\end{equation*}

This mathematical structure constitutes the algebra of Hermitian operators, which provides powerful insights into the quantum harmonic oscillator's behavior.

The ladder operators have remarkable properties that allow us to navigate between energy eigenstates. Let's examine two key lemmas:

Lemma 5.1. $\hat{a}^\dagger\Psi_n$ is an eigenfunction of $\hat{n}$ associated with the eigenvalue $n+1$

Proof:
Starting with the eigenvalue equation:

\begin{equation*}
\hat{n}\Psi_n = n\Psi_n \tag{5.22}
\end{equation*}

We apply $\hat{n}$ to $\hat{a}^\dagger\Psi_n$:

\begin{align*}
\hat{n}(\hat{a}^\dagger\Psi_n) &= \hat{n}\hat{a}^\dagger\Psi_n - \hat{a}^\dagger\hat{n}\Psi_n + \hat{a}^\dagger\hat{n}\Psi_n \\
&= [\hat{n}, \hat{a}^\dagger]\Psi_n + \hat{a}^\dagger\hat{n}\Psi_n \\
&= (-\hat{a}^\dagger)\Psi_n + \hat{a}^\dagger(n\Psi_n) \tag{5.23} \\
&= -\hat{a}^\dagger\Psi_n + n\hat{a}^\dagger\Psi_n \\
&= (n+1)\hat{a}^\dagger\Psi_n
\end{align*}

This confirms that $\hat{a}^\dagger\Psi_n$ is indeed an eigenfunction of $\hat{n}$ with eigenvalue $n+1$. We can therefore write:

\begin{equation*}
\hat{a}^\dagger\Psi_n = d\Psi_{n+1} \Longrightarrow \Psi_{n+1} = \frac{\hat{a}^\dagger}{d}\Psi_n \tag{5.24}
\end{equation*}


From the normalization condition for eigenstates:

\begin{equation*}
(\Psi_{n+1}, \Psi_{n+1}) = 1 \tag{5.25}
\end{equation*}

We can determine the normalization constant $d$ in our relation $\hat{a}^\dagger\Psi_n = d\Psi_{n+1}$:

\begin{align*}
(\Psi_{n+1}, \Psi_{n+1}) &= \left(\frac{\hat{a}^\dagger}{d}\Psi_n, \frac{\hat{a}^\dagger}{d}\Psi_n\right) \\
&= \frac{1}{d^2}(\Psi_n, \hat{a}\hat{a}^\dagger\Psi_n) \\
&= \frac{1}{d^2}(\Psi_n, (\hat{n}+1)\Psi_n) \tag{5.26} \\
&= \frac{1}{d^2}(\Psi_n, \hat{n}\Psi_n) + \frac{1}{d^2}(\Psi_n, \Psi_n) \\
&= \frac{1}{d^2}(n+1)
\end{align*}

Setting this equal to 1 gives us:

\begin{equation*}
d = \sqrt{n+1} \tag{5.27}
\end{equation*}

Therefore:

\begin{equation*}
\hat{a}^\dagger\Psi_n = \sqrt{n+1}\Psi_{n+1} \tag{5.28}
\end{equation*}

This confirms that $\hat{a}^\dagger$ acts as a raising operator, increasing the energy level by one quantum.

Lemma 5.2. $\hat{a}\Psi_n$ is an eigenfunction of $\hat{n}$ associated with the eigenvalue $n-1$

Proof:
Starting with:

\begin{equation*}
\hat{n}\Psi_n = n\Psi_n \tag{5.29}
\end{equation*}

We apply $\hat{n}$ to $\hat{a}\Psi_n$:

\begin{align*}
\hat{n}(\hat{a}\Psi_n) &= \hat{n}\hat{a}\Psi_n - \hat{a}\hat{n}\Psi_n + \hat{a}\hat{n}\Psi_n \\
&= [\hat{n}, \hat{a}]\Psi_n + \hat{a}\hat{n}\Psi_n \\
&= (\hat{a})\Psi_n + \hat{a}(n\Psi_n) \tag{5.30} \\
&= \hat{a}\Psi_n + n\hat{a}\Psi_n \\
&= (n-1)\hat{a}\Psi_n
\end{align*}

This confirms that $\hat{a}\Psi_n$ is an eigenfunction of $\hat{n}$ with eigenvalue $n-1$. We can write:

\begin{equation*}
\hat{a}\Psi_n = c\Psi_{n-1} \Longrightarrow \Psi_{n-1} = \frac{\hat{a}}{c}\Psi_n \tag{5.31}
\end{equation*}

From the normalization condition:

\begin{equation*}
(\Psi_{n-1}, \Psi_{n-1}) = 1 \tag{5.32}
\end{equation*}

We determine the value of $c$:

\begin{align*}
(\Psi_{n-1}, \Psi_{n-1}) &= \left(\frac{\hat{a}}{c}\Psi_n, \frac{\hat{a}}{c}\Psi_n\right) \\
&= \frac{1}{c^2}(\Psi_n, a^\dagger a\Psi_n) \\
&= \frac{1}{c^2}(\Psi_n, \hat{n}\Psi_n) \tag{5.33} \\
&= \frac{1}{c^2}(\Psi_n, n\Psi_n) \\
&= \frac{n}{c^2}
\end{align*}

Setting this equal to 1 gives us:

\begin{equation*}
c = \sqrt{n} \tag{5.34}
\end{equation*}

Therefore:

\begin{equation*}
\hat{a}\Psi_n = \sqrt{n}\Psi_{n-1} \tag{5.35}
\end{equation*}

This establishes $\hat{a}$ as a lowering operator, decreasing the energy level by one quantum.

These ladder operators allow us to explore the entire spectrum by repeatedly applying them to known states.


The recursive relationship established by the ladder operators allows us to express any energy eigenstate in terms of the ground state:

\begin{equation*}
\hat{a}^\dagger\Psi_n = \sqrt{n+1}\Psi_{n+1} \Longrightarrow \Psi_{n+1} = \frac{\hat{a}^\dagger}{\sqrt{n+1}}\Psi_n \tag{5.36}
\end{equation*}

Applying this relation repeatedly, we can express any state $\Psi_n$ in terms of the ground state $\Psi_0$:

\begin{align*}
\Psi_n &= \frac{\hat{a}^\dagger}{\sqrt{n}}\Psi_{n-1} = \frac{(\hat{a}^\dagger)^2}{\sqrt{n}\sqrt{n-1}}\Psi_{n-2} = \ldots \tag{5.37} \\
\cdots &= \frac{(\hat{a}^\dagger)^n}{\sqrt{n!}}\Psi_0
\end{align*}

The ground state $\Psi_0$ has the special property that applying the lowering operator to it yields zero:

\begin{equation*}
\hat{a}\Psi_0 = \sqrt{0}\Psi_{-1} = 0 \tag{5.38}
\end{equation*}

This property allows us to determine the energy of the ground state:

\begin{equation*}
\hat{H}\Psi_0 = \hbar\omega\left(\hat{n}+\frac{1}{2}\right)\Psi_0 = \hbar\omega\hat{a}^\dagger\hat{a}\Psi_0 + \frac{1}{2}\hbar\omega\Psi_0 = \frac{\hbar\omega}{2}\Psi_0 \tag{5.39}
\end{equation*}

Thus, the ground state energy is $\frac{\hbar\omega}{2}$, demonstrating the presence of zero-point energy in the quantum harmonic oscillator.

To find the explicit form of the ground state wavefunction, we use the condition $\hat{a}\Psi_0 = 0$:

\begin{align*}
&\sqrt{\frac{m\omega}{2\hbar}}\left(x + i\frac{\hat{p}}{m\omega}\right)\Psi_0 = 0 \tag{5.40} \\
&x\Psi_0 + \frac{1}{m\omega}\frac{\partial\Psi_0}{\partial x} = 0
\end{align*}

This first-order differential equation has the solution:

\begin{equation*}
\Psi_0(x) = C\exp\left(-\frac{m\omega}{2\hbar}x^2\right) = C\exp\left(-\frac{x^2}{2\lambda^2}\right) \tag{5.41}
\end{equation*}

The normalization constant $C$ is determined by requiring:

\begin{equation*}
(\Psi_0, \Psi_0) = \int_{-\infty}^{\infty}dx\,C^2\exp\left(-\frac{x^2}{\lambda^2}\right) = \lambda C^2\sqrt{\pi} \stackrel{!}{=} 1 \tag{5.42}
\end{equation*}

Solving for $C$:

\begin{equation*}
C = \sqrt{\frac{1}{\lambda\sqrt{\pi}}} = \left(\frac{m\omega}{\hbar\pi}\right)^{-\frac{1}{4}} \tag{5.43}
\end{equation*}

\subsection*{5.3 Hermite polynomials}

To find the functional form of all energy eigenstates, we need to evaluate the expression:

\begin{equation*}
(\hat{a}^\dagger)^n = \frac{1}{\sqrt{2^n}}\left(\frac{\hat{x}}{\lambda} - i\frac{\lambda\hat{p}}{\hbar}\right)^n \tag{5.44}
\end{equation*}

This gives the general solution:

\begin{equation*}
\Psi_n = \frac{1}{\sqrt{2^n n!\lambda\sqrt{\pi}}}\left(\frac{\hat{x}}{\lambda} - i\frac{\lambda\hat{p}}{\hbar}\right)^n\exp\left(-\frac{x^2}{2\lambda^2}\right) \tag{5.45}
\end{equation*}

Making the substitution $y = \frac{x}{\lambda}$ (which implies $dx = \lambda dy$), the term in brackets becomes:

\begin{equation*}
\frac{x}{\lambda} - i\frac{\lambda\hat{p}}{\hbar} = \frac{x}{\lambda} - i\frac{\lambda}{\hbar}\left(-i\hbar\frac{d}{dx}\right) = y - \frac{d}{dy} \tag{5.46}
\end{equation*}

The exponential term can be rewritten as:

\begin{equation*}
\exp\left(-\frac{x^2}{2\lambda^2}\right) = \exp\left(\frac{y^2}{2}\right)\exp(-y^2) \tag{5.47}
\end{equation*}

Therefore, apart from normalization constants, the wavefunction involves repeated application of a differential operator:

\begin{equation*}
\underbrace{\left(y - \frac{d}{dy}\right)\left(y - \frac{d}{dy}\right)\ldots\left(y - \frac{d}{dy}\right)}_{n}\left[\exp\left(\frac{y^2}{2}\right)\exp(-y^2)\right] \tag{5.48}
\end{equation*}

For the first excited state ($n=1$), we calculate:

\begin{align*}
\left(y-\frac{d}{dy}\right)\left[\exp\left(\frac{y^2}{2}\right)\exp(-y^2)\right] &= \left(y-\frac{d}{dy}\right)\left[\exp\left(-\frac{y^2}{2}\right)\right] \\
&= y\exp\left(-\frac{y^2}{2}\right) - \frac{d}{dy}\left[\exp\left(\frac{y^2}{2}\right)\exp(-y^2)\right] \\
&= y\exp\left(-\frac{y^2}{2}\right) - y\exp\left(\frac{y^2}{2}\right)\exp(-y^2) + 2y\exp\left(\frac{y^2}{2}\right)\exp(-y^2) \\
&= y\exp\left(-\frac{y^2}{2}\right) - y\exp\left(-\frac{y^2}{2}\right) + 2y\exp\left(-\frac{y^2}{2}\right) \\
&= 2y\exp\left(-\frac{y^2}{2}\right) \tag{5.49}
\end{align*}

For the second excited state ($n=2$):

\begin{align*}
&\left(y-\frac{d}{dy}\right)\left[\left(y-\frac{d}{dy}\right)\exp\left(-\frac{y^2}{2}\right)\right] \\
&= \left(y-\frac{d}{dy}\right)\left[2y\exp\left(\frac{y^2}{2}\right)\exp(-y^2)\right] \\
&= y\cdot 2y\exp\left(-\frac{y^2}{2}\right) - \frac{d}{dy}\left[2y\exp\left(\frac{y^2}{2}\right)\exp(-y^2)\right] \tag{5.50} \\
&= 2y^2\exp\left(-\frac{y^2}{2}\right) - 2\exp\left(-\frac{y^2}{2}\right) - 2y\frac{d}{dy}\left[\exp\left(\frac{y^2}{2}\right)\exp(-y^2)\right] \\
&= 2y^2\exp\left(-\frac{y^2}{2}\right) - 2\exp\left(-\frac{y^2}{2}\right) - 2y\cdot y\exp\left(-\frac{y^2}{2}\right) + 4y^2\exp\left(-\frac{y^2}{2}\right) \\
&= (4y^2-2)\exp\left(-\frac{y^2}{2}\right)
\end{align*}

This pattern can be generalized by the formula:

\begin{equation*}
F(y) = (-1)^n\exp\left(\frac{y^2}{2}\right)\frac{d^n}{dy^n}\exp(-y^2) \tag{5.51}
\end{equation*}

Multiplying by $\exp\left(\frac{y^2}{2}\right)$, we obtain the Hermite polynomials:

\begin{equation*}
H_n(y) = (-1)^n\exp(y^2)\frac{d^n}{dy^n}\exp(-y^2) \tag{5.52}
\end{equation*}

Substituting back $y=\frac{x}{\lambda}$, we can express the wavefunction in terms of Hermite polynomials:

\begin{equation*}
\Psi_n(x) = \frac{1}{\sqrt{2^n n!\lambda\sqrt{\pi}}}\exp\left(-\frac{x^2}{2\lambda^2}\right)H_n\left(\frac{x}{\lambda}\right) \tag{5.53}
\end{equation*}

The probability density function is given by:

\begin{equation*}
\rho(x) = |\Psi_n(x)|^2 = \frac{1}{2^n n!\lambda\sqrt{\pi}}\exp\left(-\frac{x^2}{\lambda^2}\right)H_n^2\left(\frac{x}{\lambda}\right) \tag{5.54}
\end{equation*}

When plotted, the probability density shows maxima that follow a parabolic path, analogous to the classical harmonic oscillator confined between turning points. In quantum mechanics, the particle isn't strictly confined, but the probability of finding it beyond certain points becomes negligibly small.

In classical mechanics, the Hamiltonian:

\begin{equation*}
\mathcal{H} = \frac{p^2}{2m} + \frac{1}{2}m\omega^2 x^2 \tag{5.55}
\end{equation*}

yields the solution:

\begin{equation*}
x = A\sin(\omega y + \varphi) \tag{5.56}
\end{equation*}

where $A = \sqrt{\frac{2E}{m\omega^2}}$ defines the classical turning points $[-A,A]$. In quantum mechanics, these turning points correspond to where the probability density reaches its maximum, found at:

\begin{equation*}
A_n = \sqrt{\frac{2\hbar\omega(n+\frac{1}{2})}{m\omega^2}} = \sqrt{\frac{2E_n}{m\omega^2}} \tag{5.57}
\end{equation*}

This maintains the analogy with the classical case. For large quantum numbers, where $\frac{1}{2}$ becomes negligible:

\begin{equation*}
A_n = \sqrt{\frac{2\hbar\omega(n+\frac{1}{2})}{m\omega^2}} \approx \sqrt{\frac{2\hbar n}{m\omega}} \tag{5.58}
\end{equation*}

In the semiclassical limit (large $n$), the distance between probability peaks becomes smaller, and the overall distribution increasingly resembles the classical parabolic potential.

\subsection*{5.4 Uncertainty of the Harmonic oscillator}

For any observable $\hat{A}$, the expected value $\langle\hat{A}\rangle$ is subject to uncertainty in measurement:

\begin{gather*}
\langle\hat{A}\rangle \pm \Delta\hat{A} \tag{5.59} \\
\Delta\hat{A} = \sqrt{\langle\hat{A}^2\rangle - \langle\hat{A}\rangle^2} \tag{5.60}
\end{gather*}


In the case of the harmonic oscillator, we can calculate the uncertainties explicitly. For the position operator $\hat{x}$:

\begin{align*}
\langle\hat{x}\rangle &= (\Psi_n, \hat{x}\Psi_n) = \sqrt{\frac{\hbar}{2m\omega}}(\Psi_n, (\hat{a}+\hat{a}^\dagger)\Psi_n) \\
&= \sqrt{\frac{\hbar}{2m\omega}}(\Psi_n, \hat{a}\Psi_n) + \sqrt{\frac{\hbar}{2m\omega}}(\Psi_n, \hat{a}^\dagger\Psi_n) \\
&= \sqrt{\frac{\hbar}{2m\omega}}\sqrt{n}(\Psi_n, \Psi_{n-1}) + \sqrt{\frac{\hbar}{2m\omega}}\sqrt{n+1}(\Psi_n, \Psi_{n+1}) = 0
\end{align*}

The expected value of position is zero due to the orthogonality of the eigenstates. For the expectation value of $\hat{x}^2$:

\begin{align*}
\langle\hat{x}^2\rangle &= \frac{\hbar}{2m\omega}(\Psi_n, (\hat{a}+\hat{a}^\dagger)^2\Psi_n) \\
&= \frac{\hbar}{2m\omega}[(\Psi_n, \hat{a}^2\Psi_n) + (\Psi_n, \hat{a}\hat{a}^\dagger\Psi_n) + (\Psi_n, \hat{a}^\dagger\hat{a}\Psi_n) + (\Psi_n, (\hat{a}^\dagger)^2\Psi_n)] \\
&= \frac{\hbar}{2m\omega}[(n+1)(\Psi_n, \Psi_n) + n(\Psi_n, \Psi_n)] \\
&= \frac{\hbar}{2m\omega}(2n+1) \tag{5.61}
\end{align*}

Similarly for the momentum operator $\hat{p}$:

\begin{align*}
\langle\hat{p}\rangle &= (\Psi_n, \hat{p}\Psi_n) = -i\sqrt{\frac{m\omega\hbar}{2}}(\Psi_n, (\hat{a}-\hat{a}^\dagger)\Psi_n) \\
&= -i\sqrt{\frac{m\omega\hbar}{2}}((\Psi_n, \hat{a}\Psi_n) - (\Psi_n, \hat{a}^\dagger\Psi_n)) \\
&= -i\sqrt{\frac{m\omega\hbar}{2}}(\sqrt{n}(\Psi_n, \Psi_{n-1}) - \sqrt{n+1}(\Psi_n, \Psi_{n+1})) = 0
\end{align*}

And for $\hat{p}^2$:

\begin{align*}
\langle\hat{p}^2\rangle &= -\frac{m\omega\hbar}{2}(\Psi_n, (\hat{a}-\hat{a}^\dagger)^2\Psi_n) \\
&= -\frac{m\omega\hbar}{2}((\Psi_n, \hat{a}^2\Psi_n) - (\Psi_n, \hat{a}\hat{a}^\dagger\Psi_n) - (\Psi_n, \hat{a}^\dagger\hat{a}\Psi_n) + (\Psi_n, (\hat{a}^\dagger)^2\Psi_n)) \\
&= -\frac{m\omega\hbar}{2}(-(n+1)(\Psi_n, \Psi_n) - n(\Psi_n, \Psi_n)) \\
&= \frac{m\omega\hbar}{2}(2n+1) \tag{5.62}
\end{align*}

These results show that the uncertainty in both position and momentum increases with the energy level $n$. Notably, the product of uncertainties $\Delta x \cdot \Delta p = \frac{\hbar}{2}(2n+1)$ exceeds the minimum value allowed by the uncertainty principle.

\subsection*{5.5 Orthogonality of Hermite polynomials}

We now want to prove that the set of eigenfunctions $\Psi_n$ are orthonormal:

\begin{equation*}
\Psi_n(x) = \frac{1}{\sqrt{2^n n!\lambda\sqrt{\pi}}}\exp\left(-\frac{x^2}{2\lambda^2}\right)H_n\left(\frac{x}{\lambda}\right) \tag{5.63}
\end{equation*}

We need to establish:

\begin{equation*}
(\Psi_m, \Psi_n) = \delta_{mn} \tag{5.64}
\end{equation*}

Proof: Let's assume, without loss of generality, that $n \geq m$. We can express $\Psi_m$ as a finite power series:

\begin{equation*}
\Psi_m = \sum_{s=0}^m C_s\exp\left(-\frac{x^2}{2\lambda^2}\right)x^s \tag{5.65}
\end{equation*}

Multiplying and dividing by $\lambda^s$:

\begin{equation*}
\Psi_m = \sum_{s=0}^m C_s\lambda^s\exp\left(-\frac{x^2}{2\lambda^2}\right)\frac{x^s}{\lambda^s} \tag{5.66}
\end{equation*}

Using the substitution $y = \frac{x}{\lambda}$, the scalar product becomes:

\begin{equation*}
(\Psi_m, \Psi_n) = \left(\sum_{s=0}^m C_s\lambda^s\exp\left(-\frac{y^2}{2}\right)y^s, \Psi_n\right) = \sum_{s=0}^m C_s\lambda^s\left(\exp\left(-\frac{y^2}{2}\right)y^s, \Psi_n\right) \tag{5.67}
\end{equation*}

To analyze the inner product $\left(\exp\left(-y^2/2\right)y^s, \Psi_n\right)$, we use the expression for $\Psi_n$:

\begin{equation*}
\Psi_n(y) = C_n\exp\left(-\frac{y^2}{2}\right)H_n(y) = C_n(-1)^n\exp\left(\frac{y^2}{2}\right)\frac{d^n}{dy^n}\exp(-y^2) \tag{5.68}
\end{equation*}

Therefore:

\begin{equation*}
\left(\exp\left(-y^2/2\right)y^s, \Psi_n\right) = C_n(-1)^n\left(\exp\left(\frac{-y^2}{2}\right)y^s, \exp\left(\frac{y^2}{2}\right)\frac{d^n}{dy^n}\exp(-y^2)\right) \tag{5.69}
\end{equation*}

I understand now. Here's the revised LaTeX code with a different structure and explanations, keeping all mathematical expressions intact, without comments, and without adding content at the end:


\section*{Analysis of Inner Products in Harmonic Oscillator States}

The key expression we need to evaluate is:

\begin{equation*}
\left(\exp \left(\frac{-y^{2}}{2}\right) y^{s}, \exp \left(\frac{y^{2}}{2}\right) \frac{\mathrm{d}^{n}}{\mathrm{~d} y^{n}} \exp \left(-y^{2}\right)\right) \tag{5.70}
\end{equation*}

When written as an integral, this becomes:

\begin{equation*}
\int_{-\infty}^{+\infty} \mathrm{d} y \exp \left(\frac{-y^{2}}{2}\right) y^{s} \exp \left(\frac{y^{2}}{2}\right) \frac{\mathrm{d}^{n}}{\mathrm{~d} y^{n}} \exp \left(-y^{2}\right)=\int_{-\infty}^{+\infty} \mathrm{d} y y^{s} \frac{\mathrm{d}^{n}}{\mathrm{~d} y^{n}} \exp \left(-y^{2}\right) \tag{5.71}
\end{equation*}

Applying integration by parts gives:

\begin{equation*}
\int_{-\infty}^{+\infty} \mathrm{d} y y^{s} \frac{\mathrm{d}^{n}}{\mathrm{~d} y^{n}} \exp \left(-y^{2}\right)=\left.y^{s} \frac{\mathrm{~d}^{n}}{\mathrm{~d} y^{n}} \exp \left(-y^{2}\right)\right|_{-\infty} ^{+\infty}+s \int_{-\infty}^{+\infty} \mathrm{d} y y^{s-1} \frac{\mathrm{~d}^{n-1}}{\mathrm{~d} y^{n-1}} \exp \left(-y^{2}\right) \tag{5.72}
\end{equation*}

With repeated integration by parts, we obtain:

\begin{equation*}
s!\int_{-\infty}^{+\infty} \mathrm{d} y \frac{\mathrm{d}^{n-s}}{\mathrm{~d} y^{n-s}} \exp \left(-y^{2}\right) \tag{5.73}
\end{equation*}

For cases where $n \neq s$, this integral vanishes. With $s \leq m$ and $m \leq n$, we can only have $s = n$ when:

\begin{equation*}
s \leq m \leq n \rightarrow n \leq m \leq n \Longrightarrow m=n \tag{5.74}
\end{equation*}

This establishes the orthogonality relation:

\begin{equation*}
\left(\Psi_{m}, \Psi_{n}\right)=\delta_{m n} \tag{5.75}
\end{equation*}

\section*{6 Elementary Potential Problems}
\subsection*{6.1 Definiton of elementary potentials}

In quantum mechanics, we often approximate complex potentials with simpler forms. For instance, a smooth transition potential:

\begin{equation*}
W(x)=\frac{V_{0}}{2}\left(1+\tanh \left(\frac{x}{d}\right)\right) \tag{6.1}
\end{equation*}

Can be modeled as a step potential:

\begin{equation*}
V(x)=V_{0} \theta(x) \tag{6.2}
\end{equation*}

Where the Heaviside function is defined as:

\[
\theta(x)= \begin{cases}0 & \text { for } x<0  \tag{6.3}\\ 1 & \text { for } x \geq 0\end{cases}
\]

Two fundamental potential models are particularly useful:

The finite potential well:

\[
V_{-}(x)= \begin{cases}0 & \text { for } x<-a  \tag{6.4}\\ -V_{0} & \text { for }-a \leq x \leq a \\ 0 & \text { for } x>a\end{cases}
\]

This approximates potentials like:

\begin{equation*}
W(x)=\frac{-V_{0}}{\cosh \left(\frac{x}{d}\right)} \tag{6.5}
\end{equation*}

And the finite potential barrier:

\[
V_{+}(x)= \begin{cases}0 & \text { for } x<-a  \tag{6.6}\\ V_{0} & \text { for }-a \leq x \leq a \\ 0 & \text { for } x>a\end{cases}
\]

Which models potentials of the form:

\begin{equation*}
W(x)=\frac{V_{0}}{\cosh \left(\frac{x}{d}\right)} \tag{6.7}
\end{equation*}

Here's the revised LaTeX code with a different structure while maintaining all mathematical expressions:


\subsection*{6.2 Step Potential: Solving the Eigenvalue Problem}

The step potential divides space into two regions with different energy characteristics. Starting with the Hamiltonian eigenvalue equation:

\begin{gather*}
\hat{H} \Psi=E \Psi  \tag{6.8}\\
-\frac{\hbar^{2}}{2 m} \frac{d^{2} \Psi}{d x^{2}}+V_{0} \theta(x) \Psi=E \Psi \tag{6.9}
\end{gather*}

We can rewrite this as a piecewise differential equation system:

\[
\begin{cases}-\frac{\hbar^{2}}{2 m} \frac{\partial^{2} \Psi}{\partial x^{2}}=E \Psi, & x<0 \text { (region I) }  \tag{6.10}\\ -\frac{\hbar^{2}}{2 m} \frac{\partial^{2} \Psi}{\partial x^{2}}+V_{0} \Psi=E \Psi, & x \geq 0 \text { (region II) }\end{cases}
\]

For the first region, we obtain a free-particle solution:

\begin{align*}
& \text { Let } k=\sqrt{\frac{2 m E}{\hbar^{2}}}  \tag{6.11}\\
& \Psi_{1}=A \mathrm{e}^{i k x}+R \mathrm{e}^{-i k x}
\end{align*}

The coefficient A corresponds to the incident wave, while R represents the reflected component.

For the second region, unlike classical mechanics where total reflection occurs when $E < V_0$, quantum mechanics allows for different behaviors. The general solution is:

\begin{align*}
& \text { Let } q=\sqrt{\frac{2 m\left(E-V_{0}\right)}{\hbar^{2}}}  \tag{6.12}\\
& \Psi_{2}=T \mathrm{e}^{i q x}
\end{align*}

No wave from the right is included as this would be physically unrealistic.

Verifying our solutions by substitution into the Schrödinger equations:

\begin{align*}
k & =\sqrt{\frac{2 m E}{\hbar^{2}}}  \tag{6.13}\\
q & =\sqrt{\frac{2 m\left(E-V_{0}\right)}{\hbar^{2}}}
\end{align*}

The complete wavefunction can be expressed as:

\begin{equation*}
\Psi=\Psi_{1} \theta(-x)+\Psi_{2} \theta(x) \tag{6.14}
\end{equation*}

To determine the coefficients, we apply boundary conditions:

\begin{enumerate}
  \item Wavefunction continuity: $\Psi_{1}(0)=\Psi_{2}(0)$
  \item Derivative continuity: $\Psi_{1}^{\prime}(0)=\Psi_{2}^{\prime}(0)$
  \item Probability current continuity: $j_{1}(0)=j_{2}(0)$
\end{enumerate}

From the first condition:

\begin{equation*}
A+R=T \tag{6.15}
\end{equation*}

From the second condition:

\begin{equation*}
i k A-i k R=i q T \tag{6.16}
\end{equation*}

Solving these equations yields:

\begin{align*}
\frac{R}{A} & =\frac{k-q}{k+q} \\
\frac{T}{A} & =\frac{2 k}{k+q} \tag{6.17}
\end{align*}

Substituting into our wavefunction gives:

\begin{equation*}
\Psi(x)=A\left[\left(e^{i k x}+\frac{k-q}{k+q} e^{-i k x}\right) \theta(-x)+\frac{2 k}{k+q} e^{i q x} \theta(x)\right] \tag{6.18}
\end{equation*}


\subsection*{6.2 Probability Current Analysis for Step Potential}

To verify the condition $j_{1}(0)=j_{2}(0)$, we need to calculate the probability currents in both regions. Beginning with region I:

\begin{equation*}
j_{1}=\frac{\hbar}{2 m i}\left(\Psi_{1}^{*} \frac{\partial \Psi_{1}}{\partial x}-\Psi_{1} \frac{\partial \Psi_{1}^{*}}{\partial x}\right) \tag{6.19}
\end{equation*}

We evaluate each component individually:

\begin{align*}
\Psi_{1} & =A e^{i k x}+R e^{-i k x} \\
\frac{\partial \Psi_{1}}{\partial x} & =i k A e^{i k x}-i k R e^{-i k x} \\
\Psi_{1}^{*} & =A e^{-i k x}+R e^{i k x}  \tag{6.20}\\
\frac{\partial \Psi_{1}^{*}}{\partial x} & =-i k A e^{-i k x}+i k R e^{i k x}
\end{align*}

Substituting these expressions into the current formula:

\begin{align*}
j_{1}= & \frac{\hbar}{2 m i}\left[\left(A e^{-i k x}+R e^{i k x}\right)\left(i k A e^{i k x}-i k R e^{-i k x}\right)\right. \\
& \left.-\left(A e^{i k x}+R e^{-i k x}\right)\left(-i k A e^{-i k x}+i k R e^{i k x}\right)\right]= \\
= & \frac{\hbar}{2 m i}\left[i k A^{2}-i k A R e^{-2 i k x}+i k R A e^{2 i k x}-i k R^{2}\right.  \tag{6.21}\\
& \left.+i k A^{2}-i k A R e^{2 i k x}+i k R A e^{-2 i k x}-i k R^{2}\right]
\end{align*}

The oscillatory terms cancel out, yielding:

\begin{equation*}
j_{1}=\frac{\hbar k}{m}\left(A^{2}-R^{2}\right) \tag{6.22}
\end{equation*}

For region II, we compute:

\begin{equation*}
j_{2}=\frac{\hbar}{2 m i}\left(\Psi_{2}^{*} \frac{\partial \Psi_{2}}{\partial x}-\Psi_{2} \frac{\partial \Psi_{2}^{*}}{\partial x}\right) \tag{6.23}
\end{equation*}

With the following components:

\begin{align*}
\Psi_{2} & =T e^{i q x} \\
\frac{\partial \Psi_{2}}{\partial x} & =i q T e^{i q x}  \tag{6.24}\\
\Psi_{2}^{*} & =T e^{-i q x} \\
\frac{\partial \Psi_{2}^{*}}{\partial x} & =-i q T e^{-i q x}
\end{align*}

Inserting these into the current expression:

\begin{align*}
j_{2} & =\frac{\hbar}{2 m i}\left[\left(T e^{-i q x}\right)\left(i q T e^{i q x}\right)-\left(T e^{i q x}\right)\left(-i q T e^{-i q x}\right)\right] \\
& =\frac{\hbar}{2 m i}\left[i q T^{2}-\left(-i q T^{2}\right)\right]  \tag{6.25}\\
& =\frac{\hbar}{2 m i}\left(2 i q T^{2}\right) \\
& =\frac{\hbar q}{m} T^{2}
\end{align*}

The probability current can be divided into three distinct contributions:

\begin{align*}
j_{i n} & =\frac{\hbar k}{m} A^{2} \\
j_{R} & =\frac{\hbar k}{m} R^{2}  \tag{6.26}\\
j_{T} & =\frac{\hbar q}{m} T^{2}
\end{align*}

This leads to a conservation principle analogous to Kirchhoff's law:

\begin{equation*}
j_{1}=j_{2} \Longrightarrow j_{i n}=j_{R}+j_{T} \tag{6.27}
\end{equation*}

We can express the reflection and transmission ratios as:

\begin{align*}
\frac{j_{R}}{j_{\text {in }}} & =\frac{R^{2}}{A^{2}}=\left(\frac{k-q}{k+q}\right)^{2} \\
\frac{j_{T}}{j_{\text {in }}} & =\frac{q}{k} \frac{T^{2}}{A^{2}}=\frac{q}{k}\left(\frac{2 k}{k+q}\right)^{2}=\frac{4 k q}{(k+q)^{2}} \tag{6.28}
\end{align*}

For the high-energy limit where $E \gg V_{0}$:

\begin{equation*}
q=\sqrt{\frac{2 m\left(E-V_{0}\right)}{\hbar^{2}}} \simeq \sqrt{\frac{2 m E}{\hbar^{2}}}=k \tag{6.29}
\end{equation*}


\subsection*{6.2.1 Analysis of Energy Regimes for Step Potential}

For the high-energy case ($E \gg V_{0}$), the transmission and reflection ratios become:

\begin{align*}
& \frac{j_{R}}{j_{i n}}=\frac{R^{2}}{A^{2}}=\left(\frac{k-q}{k+q}\right)^{2} \simeq 0  \tag{6.30}\\
& \frac{j_{T}}{j_{i n}}=\frac{4 k q}{(k+q)^{2}} \simeq \frac{4 k^{2}}{4 k^{2}}=1
\end{align*}

This physical result aligns with our intuition: particles with energy significantly exceeding the barrier height pass through virtually unimpeded. The wavefunction simplifies to:

\begin{equation*}
\Psi \approx A \mathrm{e}^{i k x} \tag{6.31}
\end{equation*}

When examining the threshold case where $E \simeq V_{0}$:

\begin{equation*}
q=\sqrt{\frac{2 m\left(E-V_{0}\right)}{\hbar^{2}}} \simeq 0 \tag{6.32}
\end{equation*}

The transmission and reflection ratios become:

\begin{align*}
\frac{j_{R}}{j_{\text {in }}} & =\frac{R^{2}}{A^{2}}=\left(\frac{k-q}{k+q}\right)^{2} \simeq \frac{k^{2}}{k^{2}}=1  \tag{6.33}\\
\frac{j_{T}}{j_{\text {in }}} & =\frac{4 k q}{(k+q)^{2}} \simeq 0
\end{align*}

In this scenario, the wave experiences complete reflection with no transmission. The amplitude coefficients satisfy:

\begin{align*}
& \frac{R}{A}=\frac{k-q}{k+q} \simeq 1 \Longrightarrow R=A  \tag{6.34}\\
& \frac{T}{A}=\frac{2 k}{k+q} \simeq 2 \Longrightarrow T=2 A
\end{align*}

The regional wavefunctions become:

\begin{equation*}
\Psi_{1}=A \mathrm{e}^{i k x}+A \mathrm{e}^{-i k x}=2 A \cos (k x) \tag{6.35}
\end{equation*}

And:

\begin{equation*}
\Psi_{2}=T \mathrm{e}^{i q x} \simeq T=2 A \tag{6.36}
\end{equation*}

For the classically forbidden region where $E<V_{0}$, we redefine $q$ to maintain a real parameter:

\begin{equation*}
q=\sqrt{\frac{2 m\left(V_{0}-E\right)}{\hbar}} \tag{6.37}
\end{equation*}

The wavefunctions take the form:

\begin{align*}
& \Psi_{1}=A \mathrm{e}^{i k x}+R \mathrm{e}^{-i k x} \\
& \Psi_{2}=T \mathrm{e}^{-q x} \tag{6.38}
\end{align*}

The boundary conditions yield:

\[
\left\{\begin{array}{l}
A+R=T  \tag{6.39}\\
i k A-i k R=-q T
\end{array}\right.
\]


\subsection*{6.2.2 Derivation of Coefficients for Sub-barrier Case}

Solving the boundary condition equations for the case where $E<V_0$:

\begin{align*}
& \left\{\begin{array}{l}
A+R=T \\
k R-k A=-i q A+-i q R
\end{array}\right. \\
& \left\{\begin{array}{l}
A+R=T \\
R(k+i q)=A(k-i q)
\end{array}\right. \\
& \left\{\begin{array}{l}
A+R=T \\
\frac{R}{A}=\frac{k-i q}{k+i q}
\end{array}\right.  \tag{6.40}\\
& \left\{\begin{array}{l}
1+\frac{R}{A}=\frac{T}{A} \\
\frac{R}{A}=\frac{k-i q}{k+i q}
\end{array}\right. \\
& \left\{\begin{array}{l}
\frac{k-i q+k+i q}{k+i q}=\frac{T}{A} \\
\frac{R}{A}=\frac{k-i q}{k+i q}
\end{array}\right.
\end{align*}

This yields the amplitude ratios:

\begin{align*}
\frac{R}{A} & =\frac{k-i q}{k+i q} \\
\frac{T}{A} & =\frac{2 k}{k+i q} \tag{6.41}
\end{align*}

The complete wavefunction becomes:

\begin{equation*}
\Psi(x)=A\left[\left(e^{i k x}+\frac{k-i q}{k+i q} e^{-i k x}\right) \theta(-x)+\frac{2 k}{k+i q} e^{-q x} \theta(x)\right] \tag{6.42}
\end{equation*}

For region II, we analyze the probability current:

\begin{equation*}
j_{2}=\frac{\hbar}{2 m i}\left(\Psi_{2}^{*} \frac{\partial \Psi_{2}}{\partial x}-\Psi_{2} \frac{\partial \Psi_{2}^{*}}{\partial x}\right) \tag{6.43}
\end{equation*}

Evaluating each term:

\begin{align*}
\Psi_{2} & =T e^{-q x} \\
\frac{\partial \Psi_{2}}{\partial x} & =-q T e^{-q x} \\
\Psi_{2}^{*} & =T^{*} e^{-q x}  \tag{6.44}\\
\frac{\partial \Psi_{2}^{*}}{\partial x} & =-q T^{*} e^{-q x}
\end{align*}

Substituting into the current expression:

\begin{align*}
j_{2} & =\frac{\hbar}{2 m i}\left[\left(T e^{-q x}\right)\left(-q T^{*} e^{-q x}\right)-\left(T e^{-q x}\right)\left(-q T^{*} e^{-q x}\right)\right] \\
& =\frac{\hbar}{2 m i}\left[-q|T|^{2} e^{-2 q x}+q|T|^{2} e^{-2 q x}\right]=0 \tag{6.45}
\end{align*}

Since $j_2 = 0$, the conservation law requires $j_1 = 0$ as well. For region I:

\begin{equation*}
j_{1}=\frac{\hbar}{2 m i}\left(\Psi_{1}^{*} \frac{\partial \Psi_{1}}{\partial x}-\Psi_{1} \frac{\partial \Psi_{1}^{*}}{\partial x}\right) \tag{6.46}
\end{equation*}

With components:

\begin{align*}
\Psi_{1} & =A e^{i k x}+R e^{-i k x} \\
\frac{\partial \Psi_{1}}{\partial x} & =i k A e^{i k x}-i k R e^{-i k x}  \tag{6.47}\\
\Psi_{1}^{*} & =A^{*} e^{-i k x}+R^{*} e^{i k x} \\
\frac{\partial \Psi_{1}^{*}}{\partial x} & =-i k A^{*} e^{-i k x}+i k R^{*} e^{i k x}
\end{align*}

The current calculation yields:

\begin{align*}
j_{1}= & \frac{\hbar}{2 m i}\left[\left(A^{*} e^{-i k x}+R^{*} e^{i k x}\right)\left(i k A e^{i k x}-i k R e^{-i k x}\right)\right. \\
& \left.-\left(A e^{i k x}+R e^{-i k x}\right)\left(-i k A^{*} e^{-i k x}+i k R^{*} e^{i k x}\right)\right] \\
= & \frac{\hbar}{2 m i}\left[i k|A|^{2}-i k A^{*} R e^{-2 i k x}+i k R^{*} A e^{2 i k x}-i k|R|^{2}\right. \\
& \left.+i k|A|^{2}-i k A R^{*} e^{2 i k x}+i k R A^{*} e^{-2 i k x}-i k|R|^{2}\right] \\
= & \frac{\hbar}{2 m i}\left[2 i k|A|^{2}-2 i k|R|^{2}+i k\left(R^{*} A-A R^{*}\right) e^{2 i k x}+i k\left(A^{*} R-R A^{*}\right) e^{-2 i k x}\right] \\
= & \frac{\hbar}{2 m i}\left[2 i k\left(|A|^{2}-|R|^{2}\right)+i k\left(\left(R^{*} A-A R^{*}\right) e^{2 i k x}+\left(A^{*} R-R A^{*}\right) e^{-2 i k x}\right)\right]= \\
= & \frac{\hbar k}{m}\left(|A|^{2}-|R|^{2}\right) \tag{6.48}
\end{align*}


\subsection*{6.2.3 Current Conservation and Physical Interpretation}

Taking $A$ as real, the current conservation condition gives:

\begin{equation*}
j_{1}=\frac{\hbar k}{m}\left(A^{2}-|R|^{2}\right)=0 \Longrightarrow|A|=|R| \tag{6.49}
\end{equation*}

This equality between incident and reflected amplitudes indicates complete reflection for the sub-barrier case.

\subsection*{6.3 Finite Potential Barrier Analysis}

For a rectangular potential barrier, we divide space into three distinct regions:

\[
V= \begin{cases}0 & \text { for } x<-a  \tag{6.50}\\ V_{0} & \text { for }-a \leq x \leq a \\ 0 & \text { for } x>a\end{cases}
\]

The Schrödinger equation takes different forms in each region:

\[
\begin{cases}-\frac{\hbar^{2}}{2 m} \frac{\partial^{2} \Psi_{1}}{\partial x^{2}}=E \Psi_{1}, & x<-a \quad \text { (region I) }  \tag{6.51}\\ -\frac{\hbar^{2}}{2 m} \frac{\partial^{2} \Psi_{2}}{\partial x^{2}}+V_{0} \Psi_{2}=E \Psi_{2}, & -a \leq x \leq a \quad \text { (region II) } \\ -\frac{\hbar^{2}}{2 m} \frac{\partial^{2} \Psi_{3}}{\partial x^{2}}=E \Psi_{3}, & x>a \quad \text { (region III) }\end{cases}
\]

The general solutions in each region are:

\[
\begin{array}{ll}
\Psi_{1}(x)=A e^{i k x}+R e^{-i k x} & \text { for } x<-a \\
\Psi_{2}(x)=C e^{i q x}+D e^{-i q x} & \text { for }-a \leq x \leq a  \tag{6.52}\\
\Psi_{3}(x)=T e^{i k x} \quad \text { for } x>a &
\end{array}
\]

The wavevectors are defined as:

\begin{align*}
k & =\sqrt{\frac{2 m E}{\hbar^{2}}} \\
q & =\sqrt{\frac{2 m\left(E-V_{0}\right)}{\hbar^{2}}} \tag{6.53}
\end{align*}

At the boundaries between regions, we apply matching conditions:

\begin{align*}
& \Psi_{1}(-a)=\Psi_{2}(-a) \\
& \Psi_{1}^{\prime}(-a)=\Psi_{2}^{\prime}(-a) \\
& \Psi_{2}(a)=\Psi_{3}(a)  \tag{6.54}\\
& \Psi_{2}^{\prime}(a)=\Psi_{3}^{\prime}(a)
\end{align*}

These conditions yield the following equations:

\begin{equation*}
A e^{-i k a}+R e^{i k a}=C e^{-i q a}+D e^{i q a} \tag{6.55}
\end{equation*}

\begin{equation*}
-i k\left(A e^{-i k a}-R e^{i k a}\right)=-i q\left(C e^{-i q a}-D e^{i q a}\right) \tag{6.56}
\end{equation*}

\begin{equation*}
C e^{i q a}+D e^{-i q a}=T e^{i k a} \tag{6.57}
\end{equation*}

\begin{equation*}
i q\left(C e^{i q a}-D e^{-i q a}\right)=i k T e^{i k a} \tag{6.58}
\end{equation*}

Solving this system of equations, we obtain the reflection and transmission amplitudes:

\begin{equation*}
\frac{R}{A}=\mathrm{e}^{-2 i k a} \frac{i\left(q^{2}-k^{2}\right) \sin (2 q a)}{2 k q \cos (2 q a)-i\left(k^{2}+q^{2}\right) \sin (2 q a)} \tag{6.59}
\end{equation*}

\begin{equation*}
\frac{T}{A}=\frac{2 k q e^{-2 i k a}}{2 k q \cos (2 q a)-i\left(k^{2}+q^{2}\right) \sin (2 q a)} \tag{6.60}
\end{equation*}


\subsection*{6.3.1 Probability Current Analysis for Barrier Problem}

For the finite barrier, we can express the probability currents in each region. For region I:

\begin{equation*}
j_{1}=\frac{\hbar k}{m}\left(A^{2}-|R|^{2}\right) \tag{6.61}
\end{equation*}

While in region III:

\begin{equation*}
j_{3}=\frac{\hbar k}{m}|T|^{2} \tag{6.62}
\end{equation*}

Current conservation requires $j_1 = j_2 = j_3$, leading to:

\begin{equation*}
A^{2}=|R|^{2}+|T|^{2} \tag{6.63}
\end{equation*}

The probability current components can be identified as:

\begin{align*}
j_{i n} & =\frac{\hbar k}{m} A^{2} \\
j_{R} & =\frac{\hbar k}{m}|R|^{2}  \tag{6.64}\\
j_{T} & =\frac{\hbar k}{m}|T|^{2}
\end{align*}

The reflection and transmission coefficients become:

\begin{align*}
\frac{j_{R}}{j_{i n}} & =\frac{|R|^{2}}{A^{2}}=\frac{\left(q^{2}-k^{2}\right) \sin ^{2}(2 q a)}{4 k^{2} q^{2}+\left(k^{2}-q^{2}\right) \sin ^{2}(2 q a)} \\
\frac{j_{T}}{j_{i n}} & =\frac{|T|^{2}}{A^{2}}=\frac{4 k^{2} q^{2}}{4 k^{2} q^{2}+\left(k^{2}-q^{2}\right) \sin ^{2}(2 q a)} \tag{6.65}
\end{align*}

\subsection*{6.3.2 Limiting Cases for Barrier Transmission}

In the high-energy limit where $E \gg V_0$, we find $q \approx k$, resulting in:

\begin{align*}
& \frac{R}{A} \rightarrow 0 \\
& \frac{T}{A} \rightarrow 1 \tag{6.66}
\end{align*}

This corresponds to complete transmission through the barrier.

As $E$ approaches $V_0$ from above ($E \rightarrow V_0^+$):

\begin{align*}
& \frac{R}{A} \rightarrow \frac{-i k a}{1-i k a} \Longrightarrow \frac{j_{R}}{j_{i n}}=\frac{k^{2} a^{2}}{1+k^{2} a^{2}}  \tag{6.67}\\
& \frac{T}{A} \rightarrow \frac{1}{1-i k a} \Longrightarrow \frac{j_{R}}{j_{i n}}=\frac{1}{1+k^{2} a^{2}}
\end{align*}

The behavior depends critically on the dimensionless parameter $ka$, which relates to the ratio of barrier width to wavelength:

\begin{equation*}
k a=\frac{2 \pi a}{\lambda} \tag{6.68}
\end{equation*}

This gives rise to two distinct regimes:

\[
\begin{array}{ll}
k a \gg 1 & \text { thick barrier } \\
k a \ll 1 & \text { thin barrier } \tag{6.69}
\end{array}
\]

For a thick barrier ($ka \gg 1$), the reflection dominates:

\begin{align*}
\frac{R}{A} & =\frac{-i k a}{1-i k a} \rightarrow 1 \\
\frac{T}{A} & =\frac{1}{1-i k a} \rightarrow 0 \tag{6.70}
\end{align*}

For a thin barrier ($ka \ll 1$), transmission prevails:

\begin{align*}
\frac{R}{A} & =\frac{-i k a}{1-i k a} \rightarrow 0 \\
\frac{T}{A} & =\frac{1}{1-i k a} \rightarrow 1 \tag{6.71}
\end{align*}

A particularly interesting case occurs when $q=\frac{\pi n}{a}$, where the reflection coefficient vanishes completely:

\begin{align*}
\frac{R}{A} & =0 \\
\frac{T}{A} & =1 \tag{6.72}
\end{align*}


\subsection*{6.4 Quantum Tunneling Analysis}

When the particle energy is below the potential barrier height ($E < V_0$), classical physics predicts total reflection. However, quantum mechanics allows for the tunneling effect. To analyze this phenomenon, we redefine the parameter $q$ as:

\begin{equation*}
q \rightarrow i q \tag{6.73}
\end{equation*}

This substitution transforms our previous formulas through the following identities:

\begin{equation*}
\sin (i x)=i \sinh (x) \tag{6.75}
\end{equation*}

\begin{equation*}
\cos (i x)=\cosh (x) \tag{6.76}
\end{equation*}

The wavefunctions in the three regions become:

\begin{align*}
& \Psi_{1}(x)=A e^{i k x}+R e^{-i k x} \quad \text { for } x<-a \\
& \Psi_{2}(x)=C e^{q x}+D e^{-q x} \quad \text { for }-a \leq x \leq a  \tag{6.77}\\
& \Psi_{3}(x)=T e^{i k x} \quad \text { for } x>a
\end{align*}

Applying boundary conditions yields the modified reflection and transmission amplitudes:

\begin{align*}
\frac{R}{A} & =\mathrm{e}^{-2 i k a} \frac{\left(q^{2}-k^{2}\right) \sinh (2 q a)}{2 i k q \cosh (2 q a)+\left(k^{2}-q^{2}\right) \sinh (2 q a)} \\
\frac{T}{A} & =\frac{2 i k q e^{-2 i k a}}{2 i k q \cosh (2 q a)+\left(k^{2}-q^{2}\right) \sinh (2 q a)} \tag{6.78}
\end{align*}

The corresponding probability current ratios are:

\begin{align*}
\frac{j_{R}}{j_{i n}} & =\frac{\left(q^{2}+k^{2}\right)^{2} \sinh ^{2}(2 q a)}{4 k^{2} q^{2}+\left(q^{2}+k^{2}\right)^{2} \sinh ^{2}(2 q a)}  \tag{6.79}\\
\frac{j_{T}}{j_{i n}} & =\frac{4 k^{2} q^{2}}{4 k^{2} q^{2}+\left(q^{2}+k^{2}\right)^{2} \sinh ^{2}(2 q a)}
\end{align*}

In the limit where the barrier height greatly exceeds the particle energy ($V_0 \gg E$), we approach total reflection:

\begin{align*}
& \frac{R}{A} \simeq 1  \tag{6.80}\\
& \frac{T}{A} \simeq 0
\end{align*}

\section*{7 Heisenberg Uncertainty Principle}
\subsection*{7.1 Mathematical Foundation of Uncertainty Relations}

Unlike classical mechanics, where a particle's position and momentum can be simultaneously determined with arbitrary precision, quantum mechanics imposes fundamental limitations on measurement precision. We can only specify:

\begin{align*}
& \left\langle x_{j}\right\rangle \pm \Delta x_{j} \\
& \left\langle\hat{p}_{j}\right\rangle \pm \Delta \hat{p}_{j} \tag{7.1}
\end{align*}

This inherent uncertainty is not due to technological limitations or experimental errors but represents a fundamental property of nature. Our objective is to derive the general Heisenberg uncertainty principle:

\begin{equation*}
\left(\Delta \hat{A}_{1}\right)^{2}\left(\Delta \hat{A}_{2}\right)^{2} \geq \frac{1}{4}\left|\left\langle\left[\hat{A}_{1}, \hat{A}_{2}\right]\right\rangle\right|^{2} \tag{7.2}
\end{equation*}


\subsection*{7.1.1 Derivation of the General Uncertainty Principle}

Let's consider two physical observables represented by Hermitian operators $\hat{A}_1$ and $\hat{A}_2$:

\begin{align*}
& \hat{A}_{1}^{\dagger}=\hat{A}_{1} \\
& \hat{A}_{2}^{\dagger}=\hat{A}_{2} \tag{7.3}
\end{align*}

We define the uncertainty (or standard deviation) of a physical quantity as:

\begin{equation*}
(\Delta \hat{A})^{2}=\left(\Psi,(\hat{A}-\langle\hat{A}\rangle)^{2} \Psi\right) \tag{7.4}
\end{equation*}

For convenience, we introduce deviation operators:

\begin{align*}
& \hat{D}_{1}=\hat{A}_{1}-\left\langle\hat{A}_{1}\right\rangle \\
& \hat{D}_{2}=\hat{A}_{2}-\left\langle\hat{A}_{2}\right\rangle \tag{7.5}
\end{align*}

The uncertainties can then be expressed as:

\begin{align*}
& \left(\Delta \hat{A}_{1}\right)^{2}=\left(\Psi, \hat{D}_{1}^{2} \Psi\right) \\
& \left(\Delta \hat{A}_{2}\right)^{2}=\left(\Psi, \hat{D}_{2}^{2} \Psi\right) \tag{7.6}
\end{align*}

Expanding $\hat{D}_j^2$:

\begin{equation*}
\hat{D}_{j}^{2}=\left(\hat{A}_{j}-\left\langle\hat{A}_{j}\right\rangle\right)^{2}=\hat{A}_{j}^{2}-2\left\langle\hat{A}_{j}\right\rangle \hat{A}_{j}+\left\langle\hat{A}_{j}\right\rangle^{2} \tag{7.7}
\end{equation*}

Taking the expectation value:

\begin{equation*}
\left\langle\hat{D}_{j}^{2}\right\rangle=\left\langle\hat{A}_{j}^{2}\right\rangle-2\left\langle\hat{A}_{j}\right\rangle\left\langle\hat{A}_{j}\right\rangle+\left\langle\hat{A}_{j}\right\rangle^{2}=\left\langle\hat{A}_{j}^{2}\right\rangle-\left\langle\hat{A}_{j}\right\rangle^{2}=\sigma_{A_{j}}^{2} \tag{7.8}
\end{equation*}

The commutator of the deviation operators equals the commutator of the original operators:

\begin{align*}
{\left[\hat{D}_{1}, \hat{D}_{2}\right] } & =\left[\hat{A}_{1}-\left\langle\hat{A}_{1}\right\rangle, \hat{A}_{2}-\left\langle\hat{A}_{2}\right\rangle\right] \\
& =\left[\hat{A}_{1}, \hat{A}_{2}-\left\langle\hat{A}_{2}\right\rangle\right]-\left[\left\langle\hat{A}_{1}\right\rangle, \hat{A}_{2}-\left\langle\hat{A}_{2}\right\rangle\right] \\
& =\left[\hat{A}_{1}, \hat{A}_{2}\right]-\left[\hat{A}_{1},\left\langle\hat{A}_{2}\right\rangle\right]-\left[\left\langle\hat{A}_{1}\right\rangle, \hat{A}_{2}\right]+\left[\left\langle\hat{A}_{1}\right\rangle,\left\langle\hat{A}_{2}\right\rangle\right]  \tag{7.9}\\
& =\left[\hat{A}_{1}, \hat{A}_{2}\right]
\end{align*}

Since $\hat{D}_1$ and $\hat{D}_2$ are Hermitian, we can establish:

\begin{align*}
\left\langle\hat{D}_{1} \hat{D}_{2}\right\rangle^{*} & =\left(\Psi, \hat{D}_{1} \hat{D}_{2} \Psi\right)^{*}= \\
& =\left(\hat{D}_{1} \hat{D}_{2} \Psi, \Psi\right)=  \tag{7.10}\\
& =\left(\hat{D}_{2} \Psi, \hat{D}_{1} \Psi\right)= \\
& =\left(\Psi, \hat{D}_{2} \hat{D}_{1} \Psi\right)=\left\langle\hat{D}_{2} \hat{D}_{1}\right\rangle
\end{align*}

This gives us:

\begin{equation*}
\left\langle\hat{D}_{1} \hat{D}_{2}\right\rangle^{*}=\left\langle\hat{D}_{2} \hat{D}_{1}\right\rangle \tag{7.11}
\end{equation*}

Using these relations, we can evaluate:

\begin{align*}
\left|\left\langle\left[\hat{A}_{1}, \hat{A}_{2}\right]\right\rangle\right|^{2} & \stackrel{(1)}{=}\left|\left\langle\left[\hat{D}_{1}, \hat{D}_{2}\right]\right\rangle\right|^{2}=\left|\left\langle\hat{D}_{1} \hat{D}_{2}-\hat{D}_{2} \hat{D}_{1}\right\rangle\right|^{2}= \\
& =\left|\left\langle\hat{D}_{1} \hat{D}_{2}\right\rangle-\left\langle\hat{D}_{1} \hat{D}_{2}\right\rangle^{*}\right|^{2}=\left|2 i \operatorname{Im}\left(\left\langle\hat{D}_{1} \hat{D}_{2}\right\rangle\right)\right|^{2}= \\
& =4\left|\operatorname{Im}\left(\left\langle\hat{D}_{1} \hat{D}_{2}\right\rangle\right)\right|^{2} \leq 4\left|\operatorname{Re}\left(\left\langle\hat{D}_{1} \hat{D}_{2}\right\rangle\right)\right|^{2}+4\left|\operatorname{Im}\left(\left\langle\hat{D}_{1} \hat{D}_{2}\right\rangle\right)\right|^{2}= \\
& =4\left|\left\langle\hat{D}_{1} \hat{D}_{2}\right\rangle\right|^{2} \tag{7.12}
\end{align*}

This can be rewritten as:

\begin{equation*}
4\left|\left\langle\hat{D}_{1} \hat{D}_{2}\right\rangle\right|^{2}=4\left(\Psi,\left(\hat{D}_{1} \hat{D}_{2}\right) \Psi\right)^{2}=4\left(\hat{D}_{1} \Psi, \hat{D}_{2} \Psi\right)^{2} \tag{7.13}
\end{equation*}

Defining $\hat{D}_{1} \Psi=\phi_{1}$ and $\hat{D}_{2} \Psi=\phi_{2}$, we can apply the Cauchy-Schwarz inequality:

\begin{equation*}
\left(\phi_{1}, \phi_{2}\right)^{2} \leq\left|\phi_{1}\right|^2\left|\phi_{2}\right|^2 \tag{7.14}
\end{equation*}


\subsection*{7.1 Deriving the Uncertainty Relation}
This analysis leads to:

\begin{align*}
& 4\left(\Psi,\left(\hat{D}_{1} \hat{D}_{2}\right) \Psi\right)^{2} \stackrel{(2)}{\leq} 4\left|\hat{D}_{1} \Psi\right|\left|\hat{D}_{2} \Psi\right|= \\
& =4\left(\hat{D}_{1} \Psi, \hat{D}_{1} \Psi\right)\left(\hat{D}_{2} \Psi \hat{D}_{2} \Psi\right)=4\left(\Psi, \hat{D}_{1}^{2} \Psi\right)\left(\Psi \hat{D}_{2}^{2} \Psi\right)  \tag{7.15}\\
& =4\left\langle\hat{D}_{1}^{2}\right\rangle\left\langle\hat{D}_{2}^{2}\right\rangle=4\left(\Delta \hat{A}_{1}\right)^{2}\left(\Delta \hat{A}_{2}\right)^{2} \stackrel{(3)}{=} 4 \sigma_{A_{1}}^{2} \sigma_{A_{2}}^{2}
\end{align*}

Combining the previous results yields:

\begin{equation*}
\left|\left\langle\left[\hat{A}_{1}, \hat{A}_{2}\right]\right\rangle\right|^{2}=\left|\left\langle\left[\hat{D}_{1}, \hat{D}_{2}\right]\right\rangle\right|^{2} \leq 4\left|\hat{D}_{1} \Psi\right|\left|\hat{D}_{2} \Psi\right|=4 \sigma_{A_{1}}^{2} \sigma_{A_{2}}^{2} \tag{7.16}
\end{equation*}

Thus we arrive at the fundamental uncertainty relation:

\begin{equation*}
\sigma_{A_{1}}^{2} \sigma_{A_{2}}^{2} \geq \frac{1}{4}\left|\left\langle\left[\hat{A}_{1}, \hat{A}_{2}\right]\right\rangle\right|^{2} \tag{7.17}
\end{equation*}

\subsection*{7.2 Physical Implications}
The Heisenberg uncertainty principle reveals a fundamental limitation: increasing measurement precision for one observable necessarily reduces precision for another non-commuting observable. The only exception occurs when operators commute, yielding $\sigma_{A_{1}}^{2} \sigma_{A_{2}}^{2} \geq 0$, allowing simultaneous precise measurements.

The classical notion of simultaneously determining a particle's position and momentum with arbitrary precision fails in quantum mechanics. From the commutation relation:

\begin{equation*}
\left[x_{i}, p_{j}\right]=i \hbar \delta_{i j} \tag{7.18}
\end{equation*}

We derive:

\begin{equation*}
\sigma_{x_{i}}^{2} \sigma_{p_{j}}^{2} \geq \frac{1}{4}\left|\left\langle\left[x_{i}, p_{j}\right]\right\rangle\right|^{2}=\frac{\hbar^{2}}{4} \delta_{i j} \tag{7.19}
\end{equation*}

For the same coordinate direction ($i=j$), we obtain the iconic inequality:

\begin{equation*}
\Delta x \Delta p \geq \frac{\hbar}{2} \tag{7.20}
\end{equation*}

\subsection*{7.3 Properties of Commuting Operators}
For a wavefunction characterized by two quantum numbers $\alpha_1$ and $\alpha_2$, the following eigenvalue equations hold simultaneously if and only if the operators commute:

\begin{align*}
& \hat{A}_{1} \Psi_{\alpha_{1}, \alpha_{2}}=\alpha_{1} \Psi_{\alpha_{1}, \alpha_{2}} \\
& \hat{A}_{2} \Psi_{\alpha_{1}, \alpha_{2}}=\alpha_{2} \Psi_{\alpha_{1}, \alpha_{2}} \tag{7.21}
\end{align*}

This follows from:

\begin{equation*}
\left[\hat{A}_{1}, \hat{A}_{2}\right]=0 \Longleftrightarrow \hat{A}_{1} \hat{A}_{2} \Psi_{\alpha_{1}, \alpha_{2}}-\hat{A}_{2} \hat{A}_{1} \Psi_{\alpha_{1}, \alpha_{2}} \tag{7.22}
\end{equation*}

For this relation to hold, $\hat{A}_1$ and $\hat{A}_2$ must have eigenvalues $\alpha_1$ and $\alpha_2$.

Theorem 7.1. If two operators $\hat{A}$ and $\hat{B}$ satisfy $[\hat{A}, \hat{B}]=0$, they possess a common basis of eigenfunctions.

\section*{Non-degenerate case}
Proof. For an eigenstate $\Psi$ of $\hat{A}$ with $\hat{A} \Psi=\alpha \Psi$, consider $\Psi_{B}=\hat{B} \Psi$. Then:

\begin{equation*}
\hat{A} \Psi_{B}=\hat{A} \hat{B} \Psi=\hat{B} \hat{A} \Psi=\alpha \hat{B} \Psi=\alpha \Psi_{B} \tag{7.23}
\end{equation*}

Thus $\Psi_{B}$ is also an eigenstate of $\hat{A}$. Since $\Psi_B$ is an eigenstate, we must have $\hat{A} \Psi_{B}=\gamma \Psi_{B}$. This requires:

\begin{equation*}
\hat{B} \Psi=\beta \Psi \Longrightarrow \gamma=\alpha \beta \tag{7.24}
\end{equation*}

Therefore, $\Psi$ must be a simultaneous eigenstate of both $\hat{A}$ and $\hat{B}$.

\section*{Degenerate case}
Proof. For the degenerate case, we consider the basis:

\begin{equation*}
\mathcal{B}=\left\{\Psi_{i}: i \in[1, M], \hat{A} \Psi_{i}=\alpha_{i} \Psi_{i}\right\} \tag{7.25}
\end{equation*}


\subsection*{7.3 Analyzing Commuting Operators}
Continuing with the degenerate case:

\begin{equation*}
\hat{A} \hat{B} \Psi_{i}=\hat{B} \hat{A} \Psi_{i}=\alpha_{i} \hat{B} \Psi \tag{7.26}
\end{equation*}

While $\hat{B} \Psi_{i}$ represents a wavefunction, generally $\hat{B} \Psi_{i} \neq \beta \Psi_{i}$. Instead, $\hat{B} \Psi_{i}$ can be expressed as a linear combination:

\begin{equation*}
\hat{B} \Psi_{i}=\sum_{n=1}^{M} C_{i n} \Psi_{n} \tag{7.27}
\end{equation*}

Where $C_{i n}=\left(\Psi_{i}, \hat{B} \Psi_{n}\right)$. Examining the Hermiticity of $C_{in}$:

\begin{equation*}
C_{i n}^{\dagger}=C_{n i}^{*}=\left(\Psi_{n}, \hat{B} \Psi_{i}\right)^{*}=\left(\hat{B} \Psi_{i}, \Psi_{n}\right)=\left(\Psi_{i}, \hat{B}^{\dagger} \Psi_{n}\right)=\left(\Psi_{i}, \hat{B} \Psi_{n}\right)=C_{i n} \tag{7.28}
\end{equation*}

Thus $C_{i n}$ forms a Hermitian matrix. For any Hermitian matrix $C$, there exists a unitary matrix $U$ such that:

\begin{align*}
& U^{\dagger} C U=D \quad \text { where } D \text { is a diagonal matrix } \\
& U^{\dagger} U=\mathbb{I} \tag{7.29}
\end{align*}

In component form:

\begin{equation*}
U_{i j}^{\dagger} C_{j n} U_{n m}=U_{j i}^{*} C_{j n} U_{n m}=d_{i} \delta_{i j} \tag{7.30}
\end{equation*}

With the unitarity condition:

\begin{equation*}
U_{i k} U_{k j}^{\dagger}=\delta_{i j} \tag{7.31}
\end{equation*}

Defining new states $\phi_{i}=U_{j i}^{*} \Psi_{j}$, we prove these are eigenstates of $\hat{B}$:

\begin{equation*}
\hat{B} \phi_{i}=\hat{B} U_{j i}^{*} \Psi_{j}=U_{j i}^{*} \hat{B} \Psi_{j}=U_{j i}^{*} C_{j n} \Psi_{n}=U_{j i}^{*} C_{j n} \delta_{n m} \Psi_{m} \tag{7.32}
\end{equation*}

Substituting the identity matrix with our $U$ matrix definition:

\begin{equation*}
U_{j i}^{*} C_{j n} \delta_{n m} \Psi_{m}=\underbrace{U_{j i}^{*} C_{j n} U_{n k}}_{d_{i} \delta_{i k}} U_{k m}^{\dagger} \Psi_{m} \tag{7.33}
\end{equation*}

This yields:

\begin{equation*}
d_{i} \delta_{i k} U_{k m}^{\dagger} \Psi_{m}=d_{i} U_{m i}^{*} \Psi_{m}=d_{i} \phi_{i} \tag{7.34}
\end{equation*}

Confirming $\phi_{i}$ is an eigenstate of $\hat{B}$ with eigenvalue $d_i$. Additionally:

\begin{equation*}
\hat{A} \phi_{i}=\hat{A} U_{j i}^{*} \Psi_{j}=U_{j i}^{*} \hat{A} \Psi_{j}=U_{j i}^{*} \alpha \Psi_{j}=\alpha \phi_{i} \tag{7.35}
\end{equation*}

Thus $\phi_{i}$ is a simultaneous eigenstate of both $\hat{A}$ and $\hat{B}$.

\subsection*{7.4 Key Implications}
The commutation relation $[\hat{A}, \hat{B}]=0$ yields several significant consequences:

\begin{enumerate}
  \item If $[\hat{A}, \hat{H}]=0$, applying the Ehrenfest theorem we have:
\end{enumerate}


\subsection*{7.4 Consequences of Operator Commutation}

\begin{equation*}
i \hbar \frac{\mathrm{~d}\langle\hat{A}\rangle}{\mathrm{d} t}=\langle[\hat{A}, \hat{H}]\rangle=0 \Longrightarrow \frac{\mathrm{~d}\langle\hat{A}\rangle}{\mathrm{d} t}=0 \tag{7.36}
\end{equation*}

This demonstrates that when an operator commutes with the Hamiltonian, its expectation value remains constant over time.

Additional consequences include:
\begin{enumerate}
  \item If $[\hat{A}, \hat{H}]=0$, a common eigenbasis exists
  \item For commuting operators, the uncertainty relation reduces to:

\begin{equation*}
(\Delta \hat{A})^{2}(\Delta \hat{H})^{2} \geq 0 \tag{7.37}
\end{equation*}

  Allowing for states where both $\Delta \hat{A}=0$ and $\Delta \hat{H}=0$
  \item Complete sets of observables can be defined
  \item Coherent states become possible
\end{enumerate}

We now examine the last two points in detail.

\subsection*{7.5 Complete Observable Sets}
A collection of Hermitian operators $\hat{A}, \hat{B}, \ldots, \hat{P}$ forms a complete set when it contains the maximum possible number of mutually commuting operators.

Example 1: For a free particle Hamiltonian:

\begin{equation*}
\hat{H}=\frac{\hat{p}_{1}^{2}+\hat{p}_{2}^{2}+\hat{p}_{3}^{2}}{2 m} \tag{7.38}
\end{equation*}

Several operator sets can be considered:
\begin{enumerate}
  \item $x_{1}, x_{2}, x_{3}$ - mutually commuting but not with $\hat{H}$
  \item $\hat{p}_{1}, \hat{p}_{2}, \hat{p}_{3}$ - mutually commuting and commuting with $\hat{H}$
  \item $x_{1}, x_{2}, p_{3}$ - mutually commuting but not with $\hat{H}$
  \item $\hat{L}_{3}, \hat{L}^{2}, \hat{H}$ - mutually commuting
  \item $\hat{p}_{i}, \hat{L}_{i}, \hat{H}$ - mutually commuting
\end{enumerate}

While options 4 and 5 are valid, $\hat{p}_{1}, \hat{p}_{2}, \hat{p}_{3}$ is most practical. For plane waves:

\begin{equation*}
\hat{p}_{j} \phi_{\vec{k}}(\vec{x})=\hbar k_{j} \phi_{\vec{k}}(\vec{x}) \tag{7.39}
\end{equation*}

The Hamiltonian acts on these eigenfunctions as:

\begin{equation*}
\hat{H} \phi_{\vec{k}}(\vec{x})=\frac{1}{2 m} \sum_{j} \hat{p}_{j}^{2} \phi_{\vec{k}}(\vec{x})=\frac{\hbar k^{2}}{2 m} \phi_{\vec{k}}(\vec{x}) \tag{7.40}
\end{equation*}

Example 2: For a Hamiltonian with potential:

\begin{equation*}
\hat{H}=\frac{\hat{p}^{2}}{2 m}+U(\vec{x}) \tag{7.41}
\end{equation*}

Many commuting sets exist:
\begin{enumerate}
  \item $x_{1}, x_{2}, x_{3}$
  \item $\hat{p}_{1}, \hat{p}_{2}, \hat{p}_{3}$\\
$\cdots$\\
n. $\hat{L}_{3}, \hat{L}^{2}, \hat{H}$
\end{enumerate}

Without knowledge of the potential form, any set might be appropriate.

Example 3: For a spherically symmetric potential:

\begin{equation*}
\hat{H}=\frac{\hat{p}^{2}}{2 m}+U(|\vec{x}|) \tag{7.42}
\end{equation*}

The set $\hat{L}_{3}, \hat{L}^{2}, \hat{H}$ becomes particularly advantageous due to rotational symmetry. Note that since $\left[\hat{L}_{i}, \hat{L}_{j}\right]=i \hbar \epsilon_{i j k} \hat{L}_{k}$, we cannot simultaneously include both $\hat{L}_{2}$ and $\hat{L}_{3}$, but must use $\hat{L}^{2}$.

\subsection*{7.6 Quantum Coherent States}
Coherent states minimize the uncertainty relation:

\begin{equation*}
\Delta \hat{A} \Delta \hat{B}=\frac{1}{2}|[\hat{A}, \hat{B}]| \tag{7.43}
\end{equation*}

These states exhibit quasi-classical behavior with minimal observable uncertainties. For position and momentum:

\begin{equation*}
\Delta x \Delta \hat{p}=\frac{\hbar}{2} \tag{7.44}
\end{equation*}


\subsection*{7.6 Analysis of Quantum Coherent States}
Coherent states satisfy the uncertainty relation with the minimal possible product:

\begin{equation*}
\Delta \hat{A}=\sqrt{\left\langle\hat{A}^{2}\right\rangle-\langle\hat{A}\rangle^{2}} \tag{7.45}
\end{equation*}

For position and momentum, coherent states are defined as eigenstates of the annihilation operator:

\begin{equation*}
\hat{a} \Psi_{z}=z \Psi_{z} \tag{7.46}
\end{equation*}

Using the definition of the annihilation operator:

\begin{equation*}
\hat{a}=\frac{1}{\sqrt{2} \lambda}\left(x+i \frac{\lambda^{2} \hat{p}}{\hbar}\right) \tag{7.47}
\end{equation*}

The eigenvalue equation becomes:

\begin{equation*}
\frac{1}{\sqrt{2} \lambda}\left(x+\lambda^{2} \frac{\partial}{\partial x}\right) \Psi_{z}=z \Psi_{z} \tag{7.48}
\end{equation*}

This first-order differential equation can be solved by proposing:

\begin{equation*}
\Psi_{z}=c \mathrm{e}^{-\sigma x^{2}+i \eta x} \tag{7.49}
\end{equation*}

Substituting into the eigenvalue equation:

\begin{align*}
& \frac{1}{\sqrt{2} \lambda}\left[x+\lambda^{2} \frac{\partial}{\partial x}\right]\left(c e^{-\sigma x^{2}+i \eta x}\right)= \\
& =\frac{1}{\sqrt{2} \lambda}\left[x c e^{-\sigma x^{2}+i \eta x}+\lambda^{2} \frac{\partial}{\partial x}\left(c e^{-\sigma x^{2}+i \eta x}\right)\right]= \\
& =\frac{1}{\sqrt{2} \lambda}\left[x c e^{-\sigma x^{2}+i \eta x}+\lambda^{2} c(-2 \sigma x+i \eta) e^{-\sigma x^{2}+i \eta x}\right]  \tag{7.50}\\
& =\frac{c e^{\sigma x^{2}+i \eta x}}{\sqrt{2} \lambda}\left[x+\lambda^{2}(-2 \sigma x+i \eta)\right]
\end{align*}

Which gives:

\begin{equation*}
z c e^{-\sigma x^{2}+i \eta x}=\frac{c e^{-\sigma x^{2}+i \eta x}}{\sqrt{2} \lambda}\left[x-\lambda^{2}(2 \sigma x+i \eta)\right] \tag{7.51}
\end{equation*}

For this equation to be satisfied:

\begin{align*}
& 1-\lambda^{2} 2 \sigma=0 \Longrightarrow \sigma=\frac{1}{2 \lambda^{2}} \\
& z=\frac{i \lambda \eta}{\sqrt{2}} \Longrightarrow i \eta=\frac{\sqrt{2}}{\lambda} z \tag{7.52}
\end{align*}

Therefore, the coherent state wavefunction is:

\begin{equation*}
\Psi_{z}=c \exp \left(-\frac{x^{2}}{2 \lambda^{2}}+\frac{\sqrt{2}}{\lambda} z x\right) \tag{7.53}
\end{equation*}

To normalize this wavefunction, we define $\nu=\operatorname{Re}(z)$ so that $z+z^{*}=2 \nu$ and calculate:

\begin{equation*}
\Psi_{z} \Psi_{z}^{*}=c^{2} \exp \left(-\frac{x^{2}}{\lambda^{2}}+\frac{\sqrt{2}}{\lambda}\left(z+z^{*}\right) x\right)=c^{2} \exp \left(-\frac{x^{2}}{\lambda^{2}}+\frac{\sqrt{2}}{\lambda} \nu x\right) \tag{7.54}
\end{equation*}

Completing the square in the exponent:

\begin{equation*}
x^{2}-2 \sqrt{2} \lambda \nu x=(x-\sqrt{2} \lambda \nu)^{2}-2 \lambda^{2} \nu^{2} \tag{7.55}
\end{equation*}

The probability density becomes:

\begin{equation*}
\Psi_{z} \Psi_{z}^{*}=c^{2} \exp \left(-\frac{1}{\lambda^{2}}(x-\sqrt{2} \lambda \nu)^{2}+2 \nu^{2}\right) \tag{7.56}
\end{equation*}

This represents a Gaussian function centered at $x_{0}=\frac{\lambda \nu}{\sqrt{2}}$.

The normalization integral is:

\begin{align*}
& c^{2} \int_{-\infty}^{+\infty} \mathrm{d} x \exp \left(-\frac{1}{\lambda^{2}}(x-\sqrt{2} \lambda \nu)^{2}+2 \nu^{2}\right)= \\
& c^{2} \exp \left(2 \nu^{2}\right) \int_{-\infty}^{+\infty} \mathrm{d} x \exp \left(-\frac{1}{\lambda^{2}}(x-\sqrt{2} \lambda \nu)^{2}\right) \tag{7.57}
\end{align*}

I understand now - you want me to revise the LaTeX code without adding comments, while maintaining mathematical expressions exactly as they are. Here's the revised code:


This represents a gaussian integral with parameter $a=\frac{1}{\lambda^{2}}$ yielding $\sqrt{\frac{\pi}{a}}=\lambda \sqrt{\pi}$. The normalization condition requires:

\begin{equation*}
c^{2} \mathrm{e}^{2 \nu^{2}} \lambda \sqrt{\pi}=1 \Longrightarrow c=\frac{\mathrm{e}^{-\nu^{2}}}{\sqrt{\lambda \sqrt{\pi}}} \tag{7.58}
\end{equation*}

With the wavefunction normalized, we can determine the explicit form of $\Psi_{z}$. We know:

\begin{equation*}
\left(\Psi_{z}, \hat{a} \Psi_{z}\right)=z\left(\Psi_{z}, \Psi_{z}\right)=z \tag{7.59}
\end{equation*}

Additionally:

\begin{align*}
\left(\Psi_{z}, \hat{a} \Psi_{z}\right) & =\frac{1}{\sqrt{2} \lambda}\left(\Psi_{z}, x \Psi_{z}\right)+\frac{i \lambda}{\sqrt{2} \hbar}\left(\Psi_{z}, \hat{p} \Psi_{z}\right)  \tag{7.60}\\
& =\frac{1}{\sqrt{2} \lambda}\langle x\rangle+\frac{i \lambda}{\sqrt{2} \hbar}\langle\hat{p}\rangle
\end{align*}

Substituting into equation (7.53):

\begin{align*}
\Psi_{z} & =\frac{\mathrm{e}^{-\nu^{2}}}{\sqrt{\lambda \sqrt{\pi}}} \exp \left(-\frac{x^{2}}{2 \lambda^{2}}+\frac{\sqrt{2}}{\lambda} \frac{1}{\sqrt{2} \lambda}\langle x\rangle x+\frac{\sqrt{2}}{\lambda} \frac{i \lambda}{\sqrt{2} \hbar}\langle\hat{p}\rangle x\right)= \\
& =\frac{\mathrm{e}^{-\nu^{2}}}{\sqrt{\lambda \sqrt{\pi}}} \exp \left(-\frac{x^{2}}{2 \lambda^{2}}+\frac{1}{\lambda^{2}}\langle x\rangle x+\frac{i}{\hbar}\langle\hat{p}\rangle x\right)= \\
& =\frac{\mathrm{e}^{-\nu^{2}}}{\sqrt{\lambda \sqrt{\pi}}} \exp \left[-\frac{1}{2 \lambda^{2}}\left(x^{2}-2\langle x\rangle x\right)+\frac{i}{\hbar}\langle\hat{p}\rangle x\right]=  \tag{7.61}\\
& =\frac{\mathrm{e}^{-\nu^{2}}}{\sqrt{\lambda \sqrt{\pi}}} \exp \left[-\frac{1}{2 \lambda^{2}}(x-\langle x\rangle)^{2}+\frac{\langle x\rangle^{2}}{2 \lambda^{2}}+\frac{i}{\hbar}\langle\hat{p}\rangle x\right]
\end{align*}

For $\langle x\rangle$, the ladder operator properties give us:

\begin{equation*}
x=\frac{\lambda}{\sqrt{2}}\left(\hat{a}+\hat{a}^{\dagger}\right) \tag{7.62}
\end{equation*}

Since $z^{*}$ corresponds to $\hat{a}^{\dagger}$:

\begin{equation*}
\left(\Psi_{z}, \hat{a}^{\dagger} \Psi_{z}\right)^{*}=\left(\hat{a}^{\dagger} \Psi_{z}, \Psi_{z}\right)=\left(\Psi_{z}, \hat{a} \Psi_{z}\right)=z \tag{7.63}
\end{equation*}

Therefore:

\begin{align*}
\langle x\rangle & =\left(\Psi_{z}, \frac{\lambda}{\sqrt{2}}\left(\hat{a}+\hat{a}^{\dagger}\right) \Psi_{z}\right)=\frac{\lambda}{\sqrt{2}}\left[\left(\Psi_{z}, \hat{a} \Psi_{z}\right)+\left(\Psi_{z}, \hat{a}^{\dagger} \Psi_{z}\right)\right]=  \tag{7.64}\\
& =\frac{\lambda}{\sqrt{2}}\left(z+z^{*}\right)=\frac{2 \nu \lambda}{\sqrt{2}}
\end{align*}

Squaring this result:

\begin{equation*}
\langle x\rangle^{2}=2 \nu^{2} \lambda^{2} \tag{7.65}
\end{equation*}

Substituting into equation (7.61):

\begin{gather*}
\Psi_{z}=\frac{1}{\sqrt{\lambda \sqrt{\pi}}} \exp \left[-\frac{1}{2 \lambda^{2}}(x-\langle x\rangle)^{2}-\mathcal{\nu}^{2}+\mathcal{P}^{2}+\frac{i}{\hbar}\langle\hat{p}\rangle x\right]  \tag{7.66}\\
\Psi_{z}=\frac{1}{\sqrt{\lambda \sqrt{\pi}}} \exp \left(-\frac{1}{2 \lambda^{2}}(x-\langle x\rangle)^{2}\right) \exp \left(i \frac{\langle\hat{p}\rangle}{\hbar} x\right) \tag{7.67}
\end{gather*}

The second exponential represents a plane wave, while the first describes a gaussian distribution centered at $\langle x\rangle$ with variance $\sigma^{2}=\lambda^{2}$. This confines the highest probability region to a narrow band proportional to $\lambda$, explaining why coherent states approach deterministic behavior.

For the uncertainty relation $\Delta x \Delta p=\hbar / 2$, recall from the harmonic oscillator:

\begin{align*}
\Delta x & =\sqrt{\left\langle x^{2}\right\rangle-\langle x\rangle^{2}}=\sqrt{\left\langle x^{2}\right\rangle} \\
\Delta p & =\sqrt{\left\langle\hat{p}^{2}\right\rangle-\langle\hat{p}\rangle^{2}}=\sqrt{\left\langle p^{2}\right\rangle} \tag{7.68}
\end{align*}

```
\begin{equation*}
\Delta x \Delta p=\frac{\hbar}{2}(2 n+1) \geq \frac{\hbar}{2} \tag{7.69}
\end{equation*}

This relation represents the fundamental uncertainty principle for quantum states, establishing the minimum bound for simultaneous measurement precision.

\begin{align*}
\langle x\rangle & =\sqrt{\frac{\hbar}{2 m \omega}}\left(\Psi_{z},\left(\hat{a}+\hat{a}^{\dagger}\right) \Psi_{z}\right)= \\
& =\sqrt{\frac{\hbar}{2 m \omega}}\left[\left(\Psi_{z}, \hat{a} \Psi_{z}\right)+\left(\Psi_{z}, \hat{a}^{\dagger} \Psi_{z}\right)\right]=  \tag{7.70}\\
& =\sqrt{\frac{\hbar}{2 m \omega}}\left(z+z^{*}\right)
\end{align*}

Here we derive the expectation value of position for a coherent state by expressing the position operator in terms of creation and annihilation operators.

\begin{align*}
\langle\hat{p}\rangle & =-i \sqrt{\frac{\hbar m \omega}{2}}\left(\Psi_{z},\left(\hat{a}-\hat{a}^{\dagger}\right) \Psi_{z}\right)= \\
& =-i \sqrt{\frac{\hbar m \omega}{2}}\left[\left(\Psi_{z}, \hat{a} \Psi_{z}\right)-\left(\Psi_{z}, \hat{a}^{\dagger} \Psi_{z}\right)\right]=  \tag{7.71}\\
& =-i \sqrt{\frac{\hbar m \omega}{2}}\left(z-z^{*}\right) \\
\left\langle x^{2}\right\rangle & =\frac{\hbar}{2 m \omega}\left(\Psi_{z},\left(\hat{a}+\hat{a}^{\dagger}\right)^{2} \Psi_{z}\right) \\
& =\frac{\hbar}{2 m \omega}\left(\Psi_{z},\left(\hat{a}^{2}+\hat{a}^{\dagger 2}+\hat{a} \hat{a}^{\dagger}+\hat{a}^{\dagger} \hat{a}\right) \Psi_{z}\right) \\
& =\frac{\hbar}{2 m \omega}\left(\Psi_{z},\left(\hat{a}^{2}+\hat{a}^{\dagger 2}+\hat{a}^{\dagger} \hat{a}+1+\hat{a}^{\dagger} \hat{a}\right) \Psi_{z}\right)  \tag{7.72}\\
& =\frac{\hbar}{2 m \omega}\left(z^{2}+z^{* 2}+2 z^{*} z+1\right) \\
& =\frac{\hbar}{2 m \omega}\left[\left(z+z^{*}\right)^{2}+1\right] \\
\left\langle\hat{p}^{2}\right\rangle & =-\frac{\hbar m \omega}{2}\left(\Psi_{z},\left(\hat{a}-\hat{a}^{\dagger}\right)^{2} \Psi_{z}\right) \\
& =-\frac{\hbar m \omega}{2}\left(\Psi_{z},\left(\hat{a}^{2}-\hat{a} \hat{a}^{\dagger}-\hat{a}^{\dagger} \hat{a}+\hat{a}^{\dagger 2}\right) \Psi_{z}\right) \\
& =-\frac{\hbar m \omega}{2}\left(\Psi_{z},\left(\hat{a}^{2}-\hat{a}^{\dagger} \hat{a}-1-\hat{a}^{\dagger} \hat{a}+\hat{a}^{\dagger 2}\right) \Psi_{z}\right)  \tag{7.73}\\
& =-\frac{\hbar m \omega}{2}\left(z^{2}+z^{* 2}-2 z^{*} z-1\right) \\
& =-\frac{\hbar m \omega}{2}\left[\left(z-z^{*}\right)^{2}-1\right]
\end{align*}

The calculations of squared expectation values reveal the variance properties of coherent states.

\begin{align*}
\Delta x & =\sqrt{\frac{\hbar}{2 m \omega}\left[\left(z+z^{*}\right)^{2}+1\right]-\frac{\hbar}{2 m \omega}\left(z+z^{*}\right)^{2}}=\sqrt{\frac{\hbar}{2 m \omega}}  \tag{7.74}\\
\Delta p & =\sqrt{-\frac{\hbar m \omega}{2}\left[\left(z-z^{*}\right)^{2}-1\right]+\frac{\hbar m \omega}{2}\left(z-z^{*}\right)^{2}}=\sqrt{\frac{\hbar m \omega}{2}}
\end{align*}

Through algebraic simplification, we find that the uncertainties in position and momentum are independent of the specific coherent state parameter.

\begin{equation*}
\Delta x \Delta p=\sqrt{\frac{\hbar}{2 m \omega} \frac{\hbar m \omega}{2}}=\frac{\hbar}{2} \tag{7.75}
\end{equation*}

This remarkable result demonstrates that coherent states achieve the minimum possible uncertainty product allowed by quantum mechanics.

\subsection*{7.7 Construction of coherent states via eigenfunctions}
We may alternatively build coherent states beginning with the harmonic oscillator eigenfunctions:

\begin{equation*}
\frac{1}{\sqrt{2^{n} n!}} \frac{1}{\sqrt{\lambda \sqrt{\pi}}} \exp \left(-\frac{x^{2}}{2 \lambda^{2}}\right) H_{n}\left(\frac{x}{\lambda}\right) \tag{7.76}
\end{equation*}

Our goal is to identify a quantum state $\Psi_{z}$ that satisfies the eigenvalue equation:

\begin{equation*}
\hat{a} \Psi_{z}=z \Psi_{z} \tag{7.77}
\end{equation*}

We can express this state through a Fourier expansion using the complete set of eigenfunctions $\Psi_{n}$:

\begin{equation*}
\Psi_{z}=\sum_{n=0}^{\infty} c_{n} \Psi_{n} \quad \text { s.t. } \quad \hat{a} \Psi_{z}=z \Psi_{z} \tag{7.78}
\end{equation*}

Applying the annihilation operator to this expansion yields:

\begin{equation*}
\hat{a} \Psi_{z}=\hat{a} \sum_{n=0}^{\infty} c_{n} \Psi_{n}=\sum_{n=0}^{\infty} c_{n} \hat{a} \Psi_{n}=\sum_{n=0}^{\infty} c_{n} \sqrt{n} \Psi_{n-1} \tag{7.79}
\end{equation*}

Through an index transformation $n \rightarrow n+1$, we can rewrite this as:

\begin{equation*}
\hat{a} \Psi_{z}=\sum_{n=0}^{\infty} c_{n+1} \sqrt{n+1} \Psi_{n} \tag{7.80}
\end{equation*}

From the eigenvalue condition, we also have:

\begin{equation*}
\hat{a} \Psi_{z}=z \Psi_{z}=z \sum_{n=0}^{\infty} c_{n} \Psi_{n} \tag{7.81}
\end{equation*}

```
\begin{equation*}
z \sum_{n=0}^{\infty} c_{n} \Psi_{n}=\sum_{n=0}^{\infty} c_{n+1} \sqrt{n+1} \Psi_{n} \tag{7.82}
\end{equation*}

Comparing coefficients on both sides of this equation reveals a fundamental relationship between successive terms in our expansion.

\begin{equation*}
z c_{n}=\sqrt{n+1} c_{n+1} \Longrightarrow c_{n}=\frac{z c_{n-1}}{\sqrt{n}} \tag{7.83}
\end{equation*}

This recursive relation provides a systematic way to determine all coefficients from the initial value $c_0$.

\begin{equation*}
c_{n}=\frac{z c_{n-1}}{\sqrt{n}}=\frac{z}{\sqrt{n}} \frac{z c_{n-2}}{\sqrt{n-1}}=\ldots=\frac{z^{n}}{\sqrt{n!}} c_{0} \tag{7.84}
\end{equation*}

Through iterative application of the recurrence relation, we express each coefficient in terms of the parameter $z$ and the initial coefficient $c_0$.

\begin{equation*}
\left(\Psi_{z}, \Psi_{z}\right) \stackrel{!}{=} 1 \tag{7.85}
\end{equation*}

The normalization requirement ensures our coherent state has unit probability across all space.

\begin{align*}
\left(\Psi_{z}, \Psi_{z}\right) & =\left(\sum_{n} \frac{z^{n}}{\sqrt{n!}} c_{0} \Psi_{n}, \sum_{m} \frac{z^{m}}{\sqrt{m!}} c_{0} \Psi_{m}\right)=c_{0}^{2} \sum_{n} \sum_{m} \frac{z^{n} z^{* m}}{\sqrt{n!m!}}\left(\Psi_{n}, \Psi_{m}\right)= \\
& =c_{0}^{2} \sum_{n} \sum_{m} \frac{z^{n} z^{* m}}{\sqrt{n!m!}} \delta_{n m}=c_{0}^{2} \sum_{n} \frac{\left(|z|^{2}\right)^{n}}{n!}= \\
& =c_{0}^{2} \exp \left(|z|^{2}\right) \tag{7.86}
\end{align*}

The orthonormality of the basis states simplifies the double sum to a recognizable exponential series.

\begin{equation*}
c_{0}^{2} \exp \left(|z|^{2}\right)=1 \Longrightarrow c_{0}=\exp \left(-\frac{|z|^{2}}{2}\right) \tag{7.87}
\end{equation*}

Solving for the normalization constant yields a characteristic exponential form that ensures proper probability interpretation.

\begin{equation*}
\Psi_{z}=\mathrm{e}^{-|z|^{2} / 2} \sum_{n} \frac{z^{n}}{\sqrt{n!}} \Psi_{n} \tag{7.88}
\end{equation*}

The complete coherent state representation elegantly combines the basis states with coefficients determined by the parameter $z$.

\subsection*{7.8 Temporal evolution of coherent states}
We now examine how coherent states evolve under the influence of the harmonic oscillator Hamiltonian:

\begin{align*}
\exp \left(-i \frac{t}{\hbar} H\right) \Psi_{z} & =\exp \left(-|z|^{2} / 2\right) \sum_{n} \frac{z^{n}}{\sqrt{n!}} \exp \left(-i \frac{t}{\hbar} H\right) \Psi_{n}= \\
& =\exp \left(-|z|^{2} / 2\right) \sum_{n} \frac{z^{n}}{\sqrt{n!}} \exp \left(-i \frac{t}{\hbar} \hbar \omega\left(\hat{n}+\frac{1}{2}\right)\right) \Psi_{n}= \\
& =\exp \left(-|z|^{2} / 2\right) \sum_{n} \frac{z^{n}}{\sqrt{n!}} \exp \left(-i t \omega\left(n+\frac{1}{2}\right)\right) \Psi_{n}= \\
& =\exp \left(-\frac{|z|^{2}+i \omega t}{2}\right) \sum_{n} \frac{z^{n}}{\sqrt{n!}}[\exp (-i t \omega)]^{n} \Psi_{n}= \\
& =\exp \left(-\frac{|z|^{2}+i \omega t}{2}\right) \sum_{n} \frac{z^{n}(t)}{\sqrt{n!}} \Psi_{n}= \tag{7.89}
\end{align*}

The time evolution maintains the coherent state structure with a time-dependent parameter $z(t)=z \exp (-i t \omega)$.

We observe that the expectation values follow classical-like trajectories:

$\langle x\rangle=\sqrt{\frac{\hbar}{2 m \omega}}\left(\Psi_{z},\left(a^{\dagger}+a\right) \Psi_{z}\right)=\sqrt{\frac{\hbar}{2 m \omega}}\left(z^{*}(t)+z(t)\right)=\sqrt{\frac{2 \hbar}{m \omega}}\left|z_{0}\right| \cos \left(\omega t-\phi_{0}\right)$

\begin{equation*}
\langle p\rangle=\sqrt{\frac{\hbar m \omega}{2}}\left(\Phi_{z},\left(a-a^{\dagger}\right) \Phi_{z}\right)=\sqrt{\frac{\hbar m \omega}{2}}\left(z^{*}(t)-z(t)\right)=\sqrt{\frac{\hbar m \omega}{2}}\left|z_{0}\right| \sin \left(\omega t-\phi_{0}\right) \tag{7.90}
\end{equation*}

These expressions demonstrate that coherent states exhibit dynamics remarkably similar to classical harmonic oscillation with amplitude $A=\sqrt{\frac{2 \hbar}{m \omega}}\left|z_{0}\right|$.

\subsection*{7.9 Complementary approach to coherent state dynamics}
An alternative derivation of coherent state evolution begins with the ansatz:

\begin{equation*}
\Psi_{z}(x, t)=\mathrm{e}^{i \alpha} \Psi_{z}(x) \tag{7.91}
\end{equation*}

Where the parameter $z$ evolves according to:

\begin{equation*}
z(t)=z_{0} \mathrm{e}^{-i \omega t} \tag{7.92}
\end{equation*}

```
\begin{align*}
i \hbar \frac{\partial}{\partial t}\left(\mathrm{e}^{i \alpha} \Psi_{z}(x)\right) & =\mathrm{e}^{i \alpha} H \Psi_{z}(x) \\
i \hbar \mathrm{e}^{i \alpha} \frac{\partial \Psi_{z}(x)}{\partial t}-\hbar \dot{\alpha} \mathrm{e}^{i \alpha} \Psi_{z}(x) & =\mathrm{e}^{i \alpha} H \Psi_{z}(x) \tag{7.93}
\end{align*}

The Schrödinger equation governs how our quantum state evolves in time, capturing both the phase evolution and the change in state structure.

\begin{align*}
& -\hbar \dot{\alpha} \mathrm{e}^{i \alpha} \Psi_{z}(x)+i \hbar \mathrm{e}^{i \alpha} \frac{\partial}{\partial t} \sum_{n=0}^{\infty} \frac{z^{n}}{\sqrt{n!}} \Psi_{n}= \\
& =-\hbar \dot{\alpha} \mathrm{e}^{i \alpha} \Psi_{z}(x)+i \hbar \mathrm{e}^{i \alpha} \sum_{n=0}^{\infty} n \frac{\dot{z}}{z} \frac{z^{n}}{\sqrt{n!}} \Psi_{n}= \\
& =-\hbar \dot{\alpha} \mathrm{e}^{i \alpha} \Psi_{z}(x)+i \hbar \mathrm{e}^{i \alpha} \sum_{n=0}^{\infty} n \frac{-i \omega \cancel{z}}{\cancel{z}} \frac{z^{n}}{\sqrt{n!}} \Psi_{n}=  \tag{7.94}\\
& =-\hbar \dot{\alpha} \mathrm{e}^{i \alpha} \Psi_{z}(x)+\mathrm{e}^{i \alpha} \sum_{n=0}^{\infty}(\hbar \omega n) \frac{z^{n}}{\sqrt{n!}} \Psi_{n}=
\end{align*}

Substituting our time-dependent parameter into the coherent state expansion reveals how each energy component evolves.

\begin{align*}
& \mathrm{e}^{i \alpha} H \Psi_{z}(x)=\mathrm{e}^{i \alpha} \hbar \omega\left(\hat{n}+\frac{1}{2}\right) \sum_{n=0}^{\infty} \frac{z^{n}}{\sqrt{n!}} \Psi_{n}=  \tag{7.95}\\
& =\mathrm{e}^{i \alpha} \sum_{n=0}^{\infty} \hbar \omega n \frac{z^{n}}{\sqrt{n!}} \Psi_{n}+\frac{\hbar \omega}{2} \mathrm{e}^{i \alpha} \Psi_{z}
\end{align*}

The Hamiltonian action on our coherent state separates into terms representing excited states and the zero-point energy contribution.

\begin{align*}
-\hbar \dot{\alpha} e^{i \alpha} \Psi_{z}(x)+e^{i \alpha} \sum_{n=0}^{\infty}(\hbar \omega n) \frac{z^{n}}{\sqrt{n!}} \Psi_{n} & =e^{i \alpha} \sum_{n=0}^{\infty} \hbar \omega n \frac{z^{n}}{\sqrt{n!}} \Psi_{n}+\frac{\hbar \omega}{2} e^{i \alpha} \Psi_{z}  \tag{7.96}\\
-\hbar \dot{\alpha} & =\frac{\hbar \omega}{2}
\end{align*}

Comparing terms in the Schrödinger equation determines the phase evolution of our coherent state.

\begin{equation*}
\alpha=-\frac{\omega}{2} t \tag{7.97}
\end{equation*}

The phase accumulates linearly with time, reflecting the zero-point energy contribution to the coherent state dynamics.

\section*{8 Quantum angular momentum theory}
\subsection*{8.1 Angular momentum operators and properties}
We now explore the quantum mechanical treatment of angular momentum, a fundamental concept with wide-ranging applications. The angular momentum operator is defined as:

\begin{equation*}
\hat{L}=\hat{x} \wedge \hat{p}=-i \hbar x \wedge \vec{\nabla} \tag{8.1}
\end{equation*}

The individual components can be expressed using the Levi-Civita symbol:

\begin{equation*}
\hat{L}_{i}=-i \hbar \epsilon_{i j k} x_{j} \frac{\partial}{\partial x_{k}} \tag{8.2}
\end{equation*}

These operators satisfy important commutation relations with position and momentum:

\begin{align*}
{\left[\hat{L}_{i}, \hat{x}_{n}\right] } & =\left[\epsilon_{i j k} \hat{x}_{j} \hat{p}_{k}, \hat{x}_{n}\right]=\epsilon_{i j k} \hat{x}_{j}\left[\hat{p}_{k}, \hat{x}_{n}\right]+\epsilon_{i j k}\left[\hat{x}_{j}, \hat{x}_{n}\right] \hat{p}_{k}= \\
& =-\epsilon_{i j k} x_{j} i \hbar \delta_{k n}=-i \hbar \epsilon_{i j n} \hat{x}_{j}=i \hbar \epsilon_{i n j} \hat{x}_{j}  \tag{8.3}\\
{\left[\hat{L}_{i}, \hat{p}_{n}\right] } & =\left[\epsilon_{i j k} \hat{x}_{j} \hat{p}_{k}, \hat{p}_{n}\right]=\epsilon_{i j k} \hat{x}_{j}\left[\hat{p}_{k}, \hat{p}_{n}\right]+\epsilon_{i j k}\left[\hat{x}_{j}, \hat{p}_{n}\right] \hat{p}_{k}= \\
& =\epsilon_{i j k} \hat{p}_{k} i \hbar \delta_{j n}=i \hbar \epsilon_{i n k} \hat{p}_{k}  \tag{8.4}\\
{\left[\hat{L}_{i}, \hat{x}^{2}\right] } & =\left[\epsilon_{i j k} \hat{x}_{j} \hat{p}_{k}, \hat{x}^{2}\right]=\epsilon_{i j k} x_{j}\left[\hat{p}_{k}, \hat{x}^{2}\right]+\epsilon_{i j k}\left[x_{j}, \hat{x}^{2}\right] \hat{p}_{k}=0  \tag{8.5}\\
{\left[\hat{L}_{i}, \hat{p}^{2}\right] } & =\left[\epsilon_{i j k} \hat{x}_{j} \hat{p}_{k}, \hat{p}^{2}\right]=\epsilon_{i j k} x_{j}\left[\hat{p}_{k}, \hat{p}^{2}\right]+\epsilon_{i j k}\left[x_{j}, \hat{p}^{2}\right] \hat{p}_{k}=0 \tag{8.6}
\end{align*}

For notational simplicity, we'll now refer to the operators without hats. Angular momentum is particularly important because many physical systems exhibit rotational symmetry, leading to conservation laws expressed through commutation relations:

\begin{align*}
& {\left[L^{2}, H\right]=0} \\
& {\left[L_{j}, H\right]=0} \tag{8.7}
\end{align*}

While $L^2$ commutes with each component $L_j$, the components themselves do not commute with each other. This allows us to simultaneously specify eigenvalues of $L^2$ and only one component (conventionally $L_3$), creating a complete set of commuting observables with the Hamiltonian.

\subsection*{8.2 Angular momentum eigenvalue structure}
To analyze the angular momentum spectrum, we introduce ladder operators analogous to those used in the harmonic oscillator:

\begin{align*}
& L_{+}=L_{1}+i L_{2} \\
& L_{-}=L_{1}-i L_{2} \tag{8.8}
\end{align*}
```
\begin{equation*}
L_{ \pm}^{\dagger}=\left(L_{1} \pm i L_{2}\right)^{\dagger}=\left(L_{1} \mp i L_{2}\right)=L_{\mp} \tag{8.9}
\end{equation*}

The hermiticity properties of these ladder operators reveal their complementary nature, similar to creation and annihilation operators.

\begin{equation*}
\left[L_{3}, L_{ \pm}\right]=\left[L_{3}, L_{1} \pm i L_{2}\right]=\left[L_{3}, L_{1}\right] \pm i\left[L_{3}, L_{2}\right]=i \hbar L_{2} \pm i\left(-i \hbar L_{1}\right)= \pm \hbar\left(\hbar L_{1} \pm i L_{2}\right)= \pm \hbar L_{ \pm} \tag{8.10}
\end{equation*}

This commutation relation demonstrates how the ladder operators shift eigenvalues of $L_3$ by precisely one quantum unit.

\begin{align*}
{\left[L_{+}, L_{-}\right] } & =\left[L_{1}+i L_{2}, L_{1}-i L_{2}\right]=\left[L_{1}, L_{1}\right]-i\left[L_{1}, L_{2}\right]+i\left[L_{2}, L_{1}\right]+\left[L_{2}, L_{2}\right]= \\
& =i\left[L_{2}, L_{1}\right]+i\left[L_{2}, L_{1}\right]=2 i\left(-i \hbar L_{3}\right)=2 \hbar L_{3} \tag{8.11}
\end{align*}

The commutator between raising and lowering operators connects back to $L_3$, forming a closed algebraic structure.

\begin{equation*}
L_{3} \phi_{m}=\hbar m \phi_{m} \tag{8.12}
\end{equation*}

Eigenfunctions of $L_3$ are characterized by the magnetic quantum number $m$, which quantifies the component of angular momentum along a specified axis.

\begin{equation*}
L^{2} \phi_{m}=\hbar \ell(\ell+1) \phi_{m} \tag{8.13}
\end{equation*}

The total angular momentum operator $L^2$ has eigenvalues determined by the quantum number $\ell$, with the characteristic $\ell(\ell+1)$ form.

\begin{align*}
L_{3}\left(L_{+} \phi_{m}\right) & =L_{3} L_{+} \phi_{m}=(\underbrace{L_{3} L_{+}-L_{+} L_{3}}_{\left[L_{3}, L_{+}\right]=\hbar L_{+}}+L_{+} L_{3})=  \tag{8.14}\\
& =L_{+}\left(\hbar+L_{3}\right) \phi_{m}=\hbar(m+1) L_{+} \phi_{m}
\end{align*}

The raising operator $L_+$ transforms an eigenstate of $L_3$ into another eigenstate with magnetic quantum number increased by one unit.

\begin{align*}
L_{3}\left(L_{-} \phi_{m}\right) & =L_{3} L_{-} \phi_{m}=(\underbrace{L_{3} L_{-}-L_{-} L_{3}}_{\left[L_{3}, L_{-}\right]=-\hbar L_{-}}+L_{-} L_{3})=  \tag{8.15}\\
& =L_{-}\left(-\hbar+L_{3}\right) \phi_{m}=\hbar(m-1) L_{+} \phi_{m}
\end{align*}

Similarly, the lowering operator $L_-$ transforms an eigenstate into one with magnetic quantum number decreased by one unit.

\begin{equation*}
L_{ \pm} \phi_{m}=\lambda_{ \pm} \phi_{m \pm 1} \tag{8.16}
\end{equation*}

The ladder operators connect adjacent states in the angular momentum spectrum with proportionality constants $\lambda_\pm$.

\begin{equation*}
\left(L_{ \pm} \phi_{m}, L_{ \pm} \phi_{m}\right) \stackrel{!}{=} 1 \tag{8.17}
\end{equation*}

The normalization condition helps determine the precise values of these proportionality constants.

\begin{equation*}
\left(L_{ \pm} \phi_{m}, L_{ \pm} \phi_{m}\right)=\left(\phi_{m}, L_{\mp} L_{ \pm} \phi_{m}\right) \tag{8.18}
\end{equation*}

Using the adjoint property, we can express the normalization in terms of operator products acting on the original state.

\begin{align*}
L_{-} L_{+} & =\left(L_{1}+i L_{2}\right)\left(L_{1}-i L_{2}\right)=L_{1}^{2}+i L_{2} L_{1}-i L_{1} L_{2}+L_{2}^{2}= \\
& =L^{3}-L_{3}^{2}+i\left[L_{2}, L_{1}\right]=L^{2}-L_{3}^{2}+i\left(i \hbar L_{3}\right)=L^{2}-L_{3}^{2}-\hbar L_{3} \tag{8.19}
\end{align*}

The product of lowering and raising operators can be expressed in terms of the fundamental operators $L^2$ and $L_3$.

\begin{align*}
\left(\phi_{m}, L_{-} L_{+} \phi_{m}\right) & =\left(\phi_{m},\left(L^{2}-L_{3}^{2}-\hbar L_{3}\right) \phi_{m}\right)= \\
& =\left(\phi_{m}, L^{2} \phi_{m}\right)-\left(\phi_{m}, L_{3}^{2} \phi_{m}\right)-\hbar\left(\phi_{m}, L_{3} \phi_{m}\right)=  \tag{8.20}\\
& =\hbar^{2} \ell(\ell+1)-\hbar^{2} m^{2}-\hbar^{2} m=\hbar^{2}[\ell(\ell+1)-m(m+1)]
\end{align*}

This calculation reveals how the normalization depends on both quantum numbers $\ell$ and $m$.

\begin{equation*}
\left(L_{+} \phi_{m}, L_{+} \phi_{m}\right)=\lambda_{+}^{2}\left(\phi_{m+1}, \phi_{m+1}\right)=\lambda_{+}^{2} \tag{8.21}
\end{equation*}

```
And so $L_{+}$ and $L_{-}$ act on $\phi_{m}=Y_{\ell, m}$ as:

\[
\left\{\begin{array}{l}
L_{+} Y_{\ell, m}=\hbar \sqrt{\ell(\ell+1)-m(m+1)} Y_{\ell, m+1}  \tag{8.22}\\
L_{-} Y_{\ell, m}=\hbar \sqrt{\ell(\ell+1)-m(m-1)} Y_{\ell, m-1}
\end{array}\right.
\]

These explicit formulas demonstrate precisely how the ladder operators transform angular momentum eigenstates, connecting states with different magnetic quantum numbers.

\begin{equation*}
\ell(\ell+1)-m(m+1) \geq 0 \quad \text { and } \quad \ell(\ell+1)-m(m-1) \geq 0 \tag{8.23}
\end{equation*}

Physical constraints require that the expressions under the square roots remain non-negative, which limits the allowed values of quantum numbers.

\begin{equation*}
-\ell \leq m \leq \ell \tag{8.24}
\end{equation*}

This fundamental constraint establishes that for each value of $\ell$, there are exactly $2\ell+1$ possible values of the magnetic quantum number $m$.

\begin{equation*}
L_{-} \phi_{\ell}=\hbar \sqrt{\ell(\ell+1)-\ell(\ell-1)} \phi_{\ell-1}=\hbar \sqrt{\ell^{\chi}+\ell-\ell^{\chi}+\ell} \phi_{\ell-1}=\hbar \sqrt{2 l} \phi_{\ell-1} \tag{8.25}
\end{equation*}

Starting from the maximum value of $m=\ell$, we can systematically generate all lower states through repeated application of the lowering operator.

\begin{align*}
\left(L_{-}\right)^{2} \phi_{\ell} & =\hbar \sqrt{2 l} L_{+} \phi_{\ell-1}=\hbar \sqrt{2 l}[\hbar \sqrt{\ell(\ell+1)-(\ell-1)(\ell-2)}] \phi_{\ell-2}= \\
& =\hbar^{2} \sqrt{2 l} \sqrt{\ell^{2}+\ell-\ell^{2}+3 l-2} \phi_{\ell-2}=\hbar^{2} \sqrt{2 l} \sqrt{2(2 l-1)} \phi_{\ell-2} \tag{8.26}
\end{align*}

Continuing this process reveals patterns in the coefficients that emerge when descending through the angular momentum ladder.

\begin{equation*}
L_{-}^{s} \phi_{\ell}=\hbar^{s} \sqrt{s!} \sqrt{\frac{(2 l)!}{(2 l-s)!}} \phi_{\ell-s} \tag{8.27}
\end{equation*}

This general formula describes how repeated application of the lowering operator connects states separated by $s$ units of angular momentum.

\begin{align*}
L_{+}^{s} Y_{\ell, 0} & =\hbar^{s} \sqrt{\frac{(\ell+s)!}{(\ell-s)!}} Y_{\ell,-s} \\
L_{-}^{s} Y_{\ell, 0} & =\hbar^{s} \sqrt{\frac{(\ell+s)!}{(\ell-s)!}} Y_{\ell, s} \tag{8.28}
\end{align*}

Starting from the $m=0$ state, we can generate all other states symmetrically through appropriate applications of raising and lowering operators.

\subsection*{8.3 Spherical coordinate representation}
For problems with rotational symmetry, spherical coordinates provide the most natural framework:

\[
\begin{cases}x_{1} & =r \sin \theta \cos \varphi  \tag{8.29}\\ x_{2} & =r \sin \theta \sin \varphi \\ x_{3} & =r \cos \theta\end{cases}
\]

This coordinate transformation allows us to express angular momentum operators in a form that highlights their geometric interpretation.

\begin{align}
\left\{\begin{array}{l}
L_{1} = i \hbar\left(\sin \varphi \frac{\partial}{\partial \theta}+\cot \theta \cos \varphi \frac{\partial}{\partial \varphi}\right)  \tag{8.30}\\
L_{2} = i \hbar\left(-\cos \varphi \frac{\partial}{\partial \theta}+\cot \theta \sin \varphi \frac{\partial}{\partial \varphi}\right) \\
L_{3} = -i \hbar \frac{\partial}{\partial \varphi}
\end{array}\right.
\end{align}

The simplicity of $L_3$ in this representation explains why it is conventionally chosen as the component to diagonalize alongside $L^2$.

\[
\left\{\begin{array}{l}
L_{+}=\hbar e^{i \varphi}\left(\frac{\partial}{\partial \theta}+i \cot \theta \frac{\partial}{\partial \varphi}\right)  \tag{8.31}\\
L_{-}=\hbar e^{-i \varphi}\left(-\frac{\partial}{\partial \theta}+i \cot \theta \frac{\partial}{\partial \varphi}\right) \\
L^{2}=-\hbar^{2}\left[\frac{1}{\sin \theta} \frac{\partial}{\partial \theta}\left(\sin \theta \frac{\partial}{\partial \theta}\right)+\frac{1}{\sin ^{2} \theta} \frac{\partial^{2}}{\partial \varphi^{2}}\right]
\end{array}\right.
\]
```
Now we assume that we can separate the contribution of $\varphi$ and $\theta$ :

\begin{equation*}
Y_{\ell, m}(\theta, \varphi)=\Phi_{m}(\varphi) F_{\ell, m}(\theta) \tag{8.32}
\end{equation*}

This separation of variables approach simplifies solving for the eigenfunctions by treating the azimuthal and polar dependencies independently.

\begin{equation*}
-i \hbar \frac{\partial}{\partial \varphi}\left(\Phi_{m}(\varphi) F_{\ell, m}(\theta)\right)=-F_{\ell, m} i \hbar \frac{\partial \Phi_{m}}{\partial \varphi} \tag{8.33}
\end{equation*}

Applying the $L_3$ operator to our separated function affects only the azimuthal component.

\begin{equation*}
L_{3} Y_{\ell, m}=\hbar m Y_{\ell, m} \tag{8.34}
\end{equation*}

The eigenvalue equation for $L_3$ establishes the quantum number $m$ as the defining characteristic of the azimuthal behavior.

\begin{equation*}
-F_{\ell, m} i \hbar \frac{\partial \Phi_{m}}{\partial \varphi}=\hbar m \Phi_{m} F_{\ell, m} \Longrightarrow-i \hbar \frac{\partial \Phi_{m}}{\partial \varphi}=\hbar m \Phi_{m} \tag{8.35}
\end{equation*}

Comparing these expressions yields a differential equation for the azimuthal function.

\begin{equation*}
-i \frac{\partial \Phi_{m}}{\partial \varphi}=m \Phi_{m} \tag{8.36}
\end{equation*}

This first-order differential equation has a straightforward solution with the physical constraint of $2\pi$ periodicity.

\begin{equation*}
\Phi_{m}=\frac{\mathrm{e}^{i m \varphi}}{\sqrt{2 \pi}} \tag{8.37}
\end{equation*}

The normalization factor ensures proper integration over the full azimuthal range.

\begin{equation*}
L^{2}=-\hbar^{2}\left[\frac{1}{\sin \theta} \frac{\partial}{\partial \theta}\left(\sin \theta \frac{\partial}{\partial \theta}\right)+\frac{1}{\sin ^{2} \theta} \frac{\partial^{2}}{\partial \varphi^{2}}\right] \tag{8.38}
\end{equation*}

The total angular momentum operator combines contributions from both angular coordinates.

\begin{equation*}
\frac{\partial^{2} \Phi_{m}}{\partial \varphi^{2}}=-m^{2} \Phi_{m} \tag{8.39}
\end{equation*}

The second derivative of the azimuthal function reveals how $m$ quantifies the rate of phase change around the azimuthal direction.

\[
\begin{array}{r}
-\hbar^{2}\left[\frac{1}{\sin \theta} \frac{\partial}{\partial \theta}\left(\sin \theta \frac{\partial}{\partial \theta}\right)+\frac{1}{\sin ^{2} \theta} \frac{\partial^{2}}{\partial \varphi^{2}}\right] \Phi_{m} F_{\ell, m}=  \tag{8.40}\\
-\hbar^{2}\left[\frac{1}{\sin \theta} \frac{\partial}{\partial \theta}\left(\sin \theta \frac{\partial}{\partial \theta}\right)-\frac{m^{2}}{\sin ^{2} \theta}\right] \Phi_{m} F_{\ell, m}
\end{array}
\]

Substituting the azimuthal solution transforms the total angular momentum equation into one involving only the polar angle.

\begin{equation*}
L^{2} Y_{\ell, m}=\hbar^{2} \ell(\ell+1) Y_{\ell, m} \tag{8.41}
\end{equation*}

The eigenvalue of $L^2$ characterizes the total angular momentum of the state.

\begin{equation*}
-\hbar^{\not 2}\left[\frac{1}{\sin \theta} \frac{\partial}{\partial \theta}\left(\sin \theta \frac{\partial}{\partial \theta}\right)-\frac{m^{2}}{\sin ^{2} \theta}\right] F_{\ell, m}=\hbar^{\not 2} \ell(\ell+1) F_{\ell, m} \tag{8.42}
\end{equation*}

Comparing with the eigenvalue equation yields a differential equation for the polar component.

\begin{equation*}
-\left[\frac{1}{\sin \theta} \frac{\partial}{\partial \theta}\left(\sin \theta \frac{\partial}{\partial \theta}\right)-\frac{m^{2}}{\sin ^{2} \theta}\right] F_{\ell, m}=\ell(\ell+1) F_{\ell, m} \tag{8.43}
\end{equation*}

This is a form of the associated Legendre differential equation, whose solutions are well-established in mathematical physics.

\begin{equation*}
F_{\ell, m}=P_{\ell, m}(\cos \theta) \tag{8.44}
\end{equation*}

The polar component is expressed in terms of associated Legendre functions of the variable $\cos\theta$.

\begin{align*}
& \text { Let } \xi=\cos \theta \\
& P_{\ell, m}(\xi)=\left(1-\xi^{2}\right)^{m / 2} \frac{\partial^{m}}{\partial \xi^{m}} P_{l}(\xi)  \tag{8.45}\\
& P_{\ell}(\xi)=\frac{1}{2^{\ell} \ell!} \frac{\partial^{\ell}}{\partial \xi^{\ell}}\left(\xi^{2}-1\right)^{\ell}
\end{align*}

```
The terms $P_{l}(\xi)$ are the Legendre polynomials. They are defined for $-1 \leq \xi \leq 1$.\\
From the definition of $\Phi(\varphi)$ we can understand that the choice of $\ell$ as an half integer must be excluded, because this would make $m$ be a half integer aswell, and so we have:

\begin{align*}
& \text { Let } m=\frac{r}{2} \quad \text { with } r \in \mathbb{Z} \\
& \Phi_{m}(\varphi+2 \pi)=\frac{\mathrm{e}^{i m(\varphi+2 \pi)}}{\sqrt{2 \pi}}=\frac{\mathrm{e}^{i m \varphi}}{\sqrt{2 \pi}} \mathrm{e}^{i r \varphi}=(-1)^{r} \Phi_{m}(\varphi) \tag{8.46}
\end{align*}

This periodicity constraint demonstrates why $m$ must be an integer, as the wavefunction must return to its original value after a complete rotation of $2\pi$.

\section*{Property 1.}

\begin{equation*}
P_{\ell, m}=(-1)^{m} \frac{(\ell+m)!}{(\ell-m)!} P_{\ell,|m|} \tag{8.47}
\end{equation*}

This relation connects associated Legendre functions with negative and positive values of $m$, providing a complete characterization for all integer values of $m$.

\section*{Property 2.}

\begin{equation*}
\int_{-1}^{1} \mathrm{~d} \xi P_{\ell, m} P_{n, m}=\frac{2}{2 l+1} \frac{(\ell+m)!}{(\ell-m)!} \delta_{l n} \tag{8.48}
\end{equation*}

This orthogonality relation is crucial for establishing the completeness of the spherical harmonic basis.

Property 3. The functions:

\begin{equation*}
Y_{\ell, m}(\theta, \varphi)=(-1)^{m} N_{\ell, m} \Phi_{m}(\varphi) P_{\ell, m}(\cos \theta) \tag{8.49}
\end{equation*}

diagonalize two Hamiltonians, so they form a basis. This is a complete orthonormal system in $L^{2}\left(\mathbb{S}^{2}\right)$ defined on the sphere $\mathbb{S}^{2}$, thus:

\begin{equation*}
\left(Y_{h, n}, Y_{\ell, m}\right)=\delta_{h \ell} \delta_{n m} \tag{8.50}
\end{equation*}

The spherical harmonics form a complete orthonormal basis for functions on the sphere, making them ideal for representing angular wavefunctions.

Proof. To prove this we explicit the integral:

\begin{equation*}
\int_{0}^{\pi} \mathrm{d} \theta \sin \theta \int_{0}^{2 \pi} \mathrm{~d} \varphi Y_{h, n}^{*}(\theta, \varphi) Y_{\ell, m}(\theta, \varphi) \tag{8.51}
\end{equation*}

This integral represents the inner product of two spherical harmonics over the entire spherical surface.

Separating the two contributions:

\begin{equation*}
N_{\ell, m} N_{h, n}(-1)^{n}(-1)^{m} \int_{0}^{\pi} \mathrm{d} \theta \sin \theta P_{h, n}(\cos \theta) P_{\ell, m}(\cos \theta) \int_{0}^{2 \pi} \mathrm{~d} \varphi \Phi_{n}^{*}(\varphi) \Phi_{m}(\varphi) \tag{8.52}
\end{equation*}

The integral separates into independent azimuthal and polar components.

\begin{equation*}
\int_{0}^{2 \pi} \mathrm{~d} \varphi \Phi_{n}^{*}(\varphi) \Phi_{m}(\varphi)=\int_{0}^{2 \pi} \mathrm{~d} \varphi \frac{\mathrm{e}^{-i n \varphi} \mathrm{e}^{i m \varphi}}{2 \pi}=\delta_{n m} \tag{8.53}
\end{equation*}

The orthogonality of complex exponentials ensures that different azimuthal modes do not overlap.

\begin{equation*}
(-1)^{n}(-1)^{m}=(-1)^{2 n}=1 \tag{8.54}
\end{equation*}

When $m=n$, the phase factors combine to yield unity.

\begin{equation*}
\int_{0}^{\pi} \mathrm{d} \theta \sin \theta P_{h, m}(\cos \theta) P_{\ell, m}(\cos \theta)=\int_{-1}^{1} \mathrm{~d} \xi P_{h, m}(\xi) P_{\ell, m}(\xi) \tag{8.55}
\end{equation*}

A change of variable transforms the polar integral into the standard form for associated Legendre functions.

\begin{equation*}
\frac{2}{2 l+1} \frac{(\ell+m)!}{(\ell-m)!} \delta_{\ell h} \tag{8.56}
\end{equation*}

```
Also if $h=\ell$ :

\begin{equation*}
N_{\ell, m}^{2} \frac{2}{2 \ell+1} \frac{(\ell+m)!}{(\ell-m)!}=1 \Longrightarrow N_{\ell, m}=\sqrt{\frac{2 l+1}{2} \frac{(\ell-m)!}{(\ell+m)!}} \tag{8.57}
\end{equation*}

This determines the normalization constant needed to ensure that spherical harmonics form an orthonormal set.

\begin{equation*}
\left(Y_{h, n}, Y_{\ell, m}\right)=\delta_{h \ell} \delta_{n m} \tag{8.58}
\end{equation*}

The completed proof confirms that spherical harmonics satisfy the orthonormality condition, making them ideal basis functions for angular problems.

\subsection*{8.4 Probability density distributions}
The probability density associated with spherical harmonics reveals the spatial distribution of quantum states:

\begin{equation*}
\left|Y_{\ell, m}(\theta, \varphi)\right|^{2}=N_{\ell, m}^{2} P_{\ell, m}^{2}(\cos \theta) \Phi_{m}^{2}(\varphi) \tag{8.59}
\end{equation*}

For the ground state ($\ell=0, m=0$), we have a perfectly spherical distribution:

\begin{equation*}
\left|Y_{0,0}(\theta, \varphi)\right|^{2}=N_{0,0}^{2} P_{0,0}^{2}(\cos \theta) \Phi_{0}^{2}(\varphi)=\frac{1}{4 \pi} \tag{8.60}
\end{equation*}

For $\ell=1$ states, we observe the characteristic p-orbital patterns:

\[
\begin{array}{r}
\left|Y_{1,0}(\theta, \varphi)\right|^{2}=N_{1,0}^{2} P_{1,0}^{2}(\cos \theta) \Phi_{0}^{2}(\varphi)=\frac{3}{4 \pi} \cos ^{2} \theta \\
\left|Y_{1, \pm 1}(\theta, \varphi)\right|^{2}=N_{1, \pm 1}^{2} P_{1, \pm 1}^{2}(\cos \theta) \Phi_{ \pm 1}^{2}(\varphi)=\frac{3}{8 \pi} \sin ^{2} \theta \tag{8.61}
\end{array}
\]

The $Y_{1,0}$ state has zero probability in the equatorial plane where $\theta=\pi/2$, creating the characteristic dumbbell shape.

For $\ell=2$ (d-orbital) states:

\begin{align*}
\left|Y_{2,0}(\theta, \varphi)\right|^{2} & =N_{2,0}^{2} P_{2,0}^{2}(\cos \theta) \Phi_{0}^{2}(\varphi)=\frac{5}{16 \pi}\left(3 \cos ^{2} \theta-1\right)^{2} \\
\left|Y_{2, \pm 1}(\theta, \varphi)\right|^{2} & =N_{2,1}^{2} P_{2,1}^{2}(\cos \theta) \Phi_{ \pm 1}^{2}(\varphi)=\frac{15}{8 \pi} \sin ^{2} \theta \cos ^{2} \theta  \tag{8.62}\\
\left|Y_{2, \pm 2}(\theta, \varphi)\right|^{2} & =N_{2,2}^{2} P_{2,2}^{2}(\cos \theta) \Phi_{ \pm 2}^{2}(\varphi)=\frac{15}{32 \pi} \sin ^{4} \theta
\end{align*}

These more complex angular distributions correspond to the characteristic shapes of d-orbitals in atomic physics, demonstrating how quantum numbers determine orbital geometry.

\subsection*{8.5 Application to diatomic molecules}
We now apply our understanding of angular momentum to analyze diatomic molecules. In the classical approach, such a system consists of:

\begin{itemize}
  \item 2 atoms with the same mass $m_{1}=m_{2}=m$
  \item The two atoms are rotating with respect to the centre of mass
  \item The centre of mass is translating in space
\end{itemize}

The classical Hamiltonian takes the form:

\begin{equation*}
\mathcal{H}=\frac{\vec{P}^{2}}{2 M}+\frac{\vec{L}^{2}}{2 I} \tag{8.63}
\end{equation*}

This separates the translational motion of the center of mass from the rotational motion around it.

\[
\begin{array}{r}
\hat{u}(\phi, \theta)=(\cos \phi \sin \theta, \sin \theta \sin \phi, \cos \theta)  \tag{8.64}\\
\vec{r}_{i}=r_{i} \hat{u}(\phi, \theta)
\end{array}
\]

The unit vector $\hat{u}$ specifies the orientation of the molecular axis in spherical coordinates.

\begin{equation*}
\frac{\mathrm{d} \hat{u}}{\mathrm{~d} t}=\dot{\phi} \sin \theta(-\sin \phi, \cos \phi, 0)+\dot{\theta}(\cos \phi \cos \theta, \sin \phi \cos \theta,-\sin \theta) \tag{8.65}
\end{equation*}

The time derivative of this unit vector captures the rotational motion of the molecule.

\begin{equation*}
E_{R}=\frac{1}{2} m \sum_{i} \vec{v}_{i}^{2}=\frac{1}{2} m \sum_{i}\left[\frac{\mathrm{d}}{\mathrm{~d} t}\left(r_{i} \hat{u}\right)\right]^{2} \tag{8.66}
\end{equation*}

The rotational kinetic energy depends on the velocities of both atoms relative to the center of mass.

\begin{equation*}
E=\frac{1}{2} m \sum_{i}\left[\frac{\mathrm{~d}}{\mathrm{~d} t}\left(r_{i} \hat{u}\right)\right]^{2}=\frac{1}{2} m \sum_{i} r_{i}^{2}\left(\frac{\mathrm{~d} \hat{u}}{\mathrm{~d} t}\right)^{2} \tag{8.67}
\end{equation*}

For a rigid molecule (no vibrations), the interatomic distances remain constant, simplifying the rotational energy expression.

```
The derivative term is thus:

\begin{align*}
\left(\frac{\mathrm{d} \hat{u}}{\mathrm{~d} t}\right)^{2} & =\dot{\phi}^{2} \sin ^{2} \theta \underbrace{(-\sin \phi, \cos \phi, 0) \cdot(-\sin \phi, \cos \phi, 0)}_{=1}+ \\
& +\dot{\theta}^{2} \underbrace{(\cos \phi \cos \theta, \sin \phi \cos \theta,-\sin \theta) \cdot(\cos \phi \cos \theta, \sin \phi \cos \theta,-\sin \theta)}_{=1}+ \\
& +2 \dot{\theta} \dot{\phi} \underbrace{(-\sin \phi, \cos \phi, 0) \cdot(\cos \phi \cos \theta, \sin \phi \cos \theta,-\sin \theta)}_{=0}= \\
& =\dot{\phi}^{2} \sin ^{2} \theta+\dot{\theta}^{2} \tag{8.68}
\end{align*}

Expanding the squared derivative reveals the rotational kinetic energy's dependence on the angular velocities in spherical coordinates.

\begin{equation*}
E_{R}=m r^{2}\left(\dot{\phi}^{2} \sin ^{2} \theta+\dot{\theta}^{2}\right)=\frac{I}{2}\left(\dot{\phi}^{2} \sin ^{2} \theta+\dot{\theta}^{2}\right) \tag{8.69}
\end{equation*}

For identical atoms at equal distances, the moment of inertia $I=2mr^2$ simplifies the rotational energy expression.

\begin{equation*}
\vec{L}=\sum_{i} \vec{r}_{i} \wedge m \vec{v}_{i}=m \sum_{i} r_{i}^{2} \hat{u} \wedge \frac{\mathrm{d} \hat{u}}{\mathrm{~d} t}=2 m r^{2} \hat{u} \wedge \frac{\mathrm{d} \hat{u}}{\mathrm{~d} t}=I \hat{u} \wedge \frac{\mathrm{d} \hat{u}}{\mathrm{~d} t} \tag{8.70}
\end{equation*}

The angular momentum vector emerges naturally from the cross product of position and velocity vectors.

\begin{equation*}
\vec{L}^{2}=I^{2}\left(\hat{u} \wedge \frac{\mathrm{~d} \hat{u}}{\mathrm{~d} t}\right) \cdot\left(\hat{u} \wedge \frac{\mathrm{~d} \hat{u}}{\mathrm{~d} t}\right) \tag{8.71}
\end{equation*}

The square of angular momentum determines the rotational energy of the molecule.

\begin{equation*}
\vec{L}^{2}=I^{2}\left[\frac{\mathrm{~d} \hat{u}}{\mathrm{~d} t} \wedge\left(\hat{u} \wedge \frac{\mathrm{~d} \hat{u}}{\mathrm{~d} t}\right)\right] \cdot \hat{u}=I^{2}\left(\frac{\mathrm{~d} \hat{u}}{\mathrm{~d} t}\right)^{2} \tag{8.72}
\end{equation*}

Through vector identities, we establish the relationship between angular momentum and the angular velocity terms.

\begin{equation*}
E_{R}=\frac{\vec{L}^{2}}{2 I} \tag{8.73}
\end{equation*}

This elegant expression connects the rotational energy directly to the angular momentum and moment of inertia.

\begin{equation*}
\mathcal{H}=\frac{\vec{P}^{2}}{2 M}+\frac{\vec{L}^{2}}{2 I} \tag{8.74}
\end{equation*}

The complete classical Hamiltonian combines translational and rotational contributions.

\begin{equation*}
H=\frac{P^{2}}{2 M}+\frac{L^{2}}{2 I} \tag{8.75}
\end{equation*}

Transitioning to quantum mechanics involves replacing classical variables with corresponding operators while maintaining the same structural form.

\begin{equation*}
\Psi=\Phi_{\vec{K}}(X) Y_{\ell, m}(\varphi, \theta) \tag{8.76}
\end{equation*}

The separability of the Hamiltonian allows us to factorize the wavefunction into translational and rotational components.

\begin{equation*}
H \Psi=Y_{\ell, m}(\varphi, \theta) \frac{P^{2}}{2 M} \Phi_{\vec{K}}(X)+\Phi_{\vec{K}}(X) \frac{L^{2}}{2 I} Y_{\ell, m}(\varphi, \theta)=\left(\frac{\hbar^{2} K^{2}}{2 M}+\frac{\hbar^{2} \ell(\ell+1)}{2 I}\right) \Psi \tag{8.77}
\end{equation*}

Applying the Hamiltonian to our factorized wavefunction reveals the energy eigenvalues for the combined system.

\begin{equation*}
E=\frac{\hbar^{2} K^{2}}{2 M}+\frac{\hbar^{2} \ell(\ell+1)}{2 I} \tag{8.78}
\end{equation*}

The total energy consists of translational kinetic energy and quantized rotational energy levels characterized by the angular momentum quantum number $\ell$.

```
From this equation we can notice two things:

\begin{itemize}
  \item The energy does not depend on $m$ (degenerate states)
  \item $\vec{K}$ has continuous values, while $\ell$ has discrete values
\end{itemize}

These observations highlight key quantum mechanical features: the $(2\ell+1)$-fold degeneracy of rotational states and the discrete nature of angular momentum.

\begin{equation*}
\Psi=\sum_{m=-\ell}^{\ell} c_{m} \Psi_{\vec{K}, \ell, m}=\Psi_{\vec{K}, \ell} \tag{8.79}
\end{equation*}

Since energy is independent of $m$, the general solution is a superposition of all possible $m$ states for a given $\ell$.

\begin{equation*}
H=\ldots+\gamma L_{3} \tag{8.80}
\end{equation*}

Adding a term proportional to $L_3$ would break the rotational symmetry of the system.

\begin{equation*}
\gamma L_{3} \Psi=\gamma \hbar m \Psi \tag{8.81}
\end{equation*}

Such a term would split the energy levels according to the magnetic quantum number.

\begin{equation*}
E=\frac{\hbar^{2} K^{2}}{2 M}+\frac{\hbar^{2} \ell(\ell+1)}{2 I}+\gamma \hbar m \tag{8.82}
\end{equation*}

This energy expression describes Zeeman splitting, where degenerate levels separate in the presence of a magnetic field.

\section*{9 Hydrogen atom}
\subsection*{9.1 Problem formulation}
The hydrogen atom represents one of the most fundamental quantum mechanical systems. Classically, it consists of an electron orbiting a positively charged nucleus:

\begin{equation*}
\mathcal{H}=\frac{p^{2}}{2 m}+V(r)=\frac{p^{2}}{2 m}-\frac{Z e^{2}}{r^{2}} \tag{9.1}
\end{equation*}

The Coulomb potential creates a central force problem with spherical symmetry.

\begin{align*}
& \left\{\vec{L}_{3}, \mathcal{H}\right\}=0 \\
& \left\{\vec{L}^{2}, \mathcal{H}\right\}=0 \tag{9.2}
\end{align*}

These Poisson brackets (which become commutators in quantum mechanics) indicate that both the angular momentum magnitude and its z-component are conserved quantities.

\begin{equation*}
\mathcal{H}=\frac{\vec{L}^{2}}{2 m r^{2}}+\frac{\vec{p}_{r}^{2}}{2 m}+V(r) \tag{9.3}
\end{equation*}

This reformulation of the Hamiltonian separates the radial and angular contributions to the energy.

\begin{equation*}
\mathcal{H}=\frac{1}{2} m \sum_{i} \vec{v}_{i}^{2}=\frac{1}{2} m \vec{v}^{2}=\frac{1}{2} m\left(\frac{\mathrm{~d} \vec{r}}{\mathrm{~d} t}\right)^{2} \tag{9.4}
\end{equation*}

Starting from the kinetic energy expression, we need to transform to spherical coordinates.

\begin{equation*}
\frac{\mathrm{~d} \vec{r}_{i}}{\mathrm{~d} t}=\frac{\mathrm{~d}}{\mathrm{~d} t}(r \hat{u})=\dot{r} \hat{u}+r \dot{\hat{u}} \tag{9.5}
\end{equation*}

Unlike the rigid diatomic molecule, the radial coordinate can now vary, adding a radial velocity component.

\begin{align*}
\left(\frac{\mathrm{~d} \vec{r}}{\mathrm{~d} t}\right)^{2} & =(\dot{r} \hat{u}+r \dot{\hat{u}}) \cdot(\dot{r} \hat{u}+r \dot{\hat{u}})= \\
& =\dot{r}^{2} \hat{u} \cdot \hat{u}+2 \dot{r} r \hat{u} \cdot \hat{\hat{u}}+r^{2} \dot{\hat{u}} \cdot \dot{\hat{u}}=  \tag{9.6}\\
& =\dot{r}^{2}+r^{2}(\dot{\hat{u}})^{2}=\dot{r}^{2}+r^{2}\left(\dot{\theta}^{2}+\dot{\varphi}^{2} \sin ^{2} \varphi\right)
\end{align*}

The velocity squared decomposes into radial and angular components in spherical coordinates.

\begin{equation*}
\vec{L}^{2}=m^{2} r^{4}\left(\frac{\mathrm{~d} \hat{u}}{\mathrm{~d} t}\right)^{2}=m^{2} r^{4}\left(\dot{\theta}^{2}+\dot{\varphi}^{2} \sin ^{2} \varphi\right) \tag{9.7}
\end{equation*}

The angular momentum squared relates directly to the angular velocity components.

```
And so the total kinetic energy can be rewritten as:

\begin{equation*}
T=\frac{1}{2 m} m^{2} \dot{r}^{2}+\frac{1}{2} \frac{m r^{2}}{m^{2} r^{4}} \underbrace{m^{2} r^{4}\left(\dot{\theta}^{2}+\dot{\varphi}^{2} \sin ^{2} \varphi\right)}_{\vec{L}^{2}}=\frac{m^{2} \dot{r}^{2}}{2 m}+\frac{\vec{L}^{2}}{2 m r^{2}} \tag{9.8}
\end{equation*}

The kinetic energy separates into radial and angular components, with the angular part expressed in terms of angular momentum.

\begin{equation*}
p_{r}=\frac{\partial \mathcal{L}}{\partial \dot{r}} \tag{9.9}
\end{equation*}

To verify the radial momentum expression, we need to apply the principles of Lagrangian mechanics.

\begin{equation*}
\mathcal{L}=T-V(r) \tag{9.10}
\end{equation*}

The Lagrangian connects to the Hamiltonian through a Legendre transformation.

\begin{equation*}
p_{r}=\frac{\partial \mathcal{L}}{\partial \dot{r}}=\frac{\partial T}{\partial \dot{r}}-\frac{\partial V(\not \supset)}{\partial \dot{r}} \tag{9.11}
\end{equation*}

Since the potential energy is independent of velocity, it doesn't contribute to momentum.

\begin{equation*}
\frac{\partial T}{\partial \dot{r}}=\frac{\partial}{\partial \dot{r}}\left(\frac{m^{2} \dot{r}^{2}}{2 m}+\frac{\vec{L}^{2}}{2 m r^{2}}\right)=\frac{2 m^{2} \dot{r}}{2 m}=m \dot{r} \tag{9.12}
\end{equation*}

This confirms that $p_r = m\dot{r}$, consistent with our expectations for radial momentum.

\begin{equation*}
\mathcal{H}=\frac{p_{r}^{2}}{2 m}+\frac{\vec{L}^{2}}{2 m r^{2}}+V(r) \tag{9.13}
\end{equation*}

The classical Hamiltonian now has a form that clearly separates radial and angular contributions.

\begin{equation*}
\left\{r, p_{r}\right\}=1 \tag{9.14}
\end{equation*}

The classical Poisson bracket relation between position and momentum variables.

\begin{equation*}
\left[r, \hat{p}_{r}\right]=i \hbar \tag{9.15}
\end{equation*}

In quantum mechanics, this becomes a commutation relation between operators.

\begin{equation*}
\hat{p}_{r}=-i \hbar \frac{\partial}{\partial r}-\frac{i \hbar}{r} \tag{9.16}
\end{equation*}

The radial momentum operator takes this specific form to ensure hermiticity in spherical coordinates.

\begin{equation*}
\left(\phi, \hat{p}_{r} \varphi\right)^{*}=\left(\varphi, \hat{p}_{r} \phi\right) \tag{9.17}
\end{equation*}

To verify hermiticity, we must prove this inner product relation.

\begin{align*}
\left(\phi, \hat{p}_{r} \varphi\right) & =-i \hbar \int_{\mathbb{R}^{3}} \mathrm{~d}^{3} x \phi^{*}\left(\frac{\partial \varphi}{\partial r}+\varphi \frac{1}{r}\right)=  \tag{9.18}\\
& =-i \hbar \int_{\mathbb{R}^{3}} \mathrm{~d}^{3} x \phi^{*} \frac{\partial \varphi}{\partial r}-i \hbar \int_{\mathbb{R}^{3}} \mathrm{~d}^{3} x \phi^{*} \varphi \frac{1}{r}
\end{align*}

Expanding the inner product in Cartesian coordinates.

\begin{equation*}
-i \hbar \iint_{\mathbb{R}^{2}} \mathrm{~d} \Omega \int_{0}^{\infty} \mathrm{d} r \phi^{*} \frac{\partial \varphi}{\partial r} r^{2}-i \hbar \iint_{\mathbb{R}^{2}} \mathrm{~d} \Omega \int_{0}^{\infty} \mathrm{d} r \phi^{*} \varphi r \tag{9.19}
\end{equation*}

Converting to spherical coordinates introduces volume elements appropriate for this geometry.

\begin{equation*}
\frac{\partial}{\partial r}\left(\phi^{*} \varphi r\right)=\frac{\partial \phi^{*}}{\partial r} \varphi r+\phi^{*} \frac{\partial(\varphi r)}{\partial r}=\frac{\partial \phi^{*}}{\partial r} \varphi r+\phi^{*} \frac{\partial \varphi}{\partial r} r+\phi^{*} \varphi \tag{9.20}
\end{equation*}

Using the product rule to manipulate terms in preparation for integration by parts.

```
So:

\begin{equation*}
\phi^{*} \frac{\partial \varphi}{\partial r} r+\phi^{*} \varphi=\frac{\partial}{\partial r}\left(\phi^{*} \varphi r\right)-\frac{\partial \phi^{*}}{\partial r} \varphi r \tag{9.21}
\end{equation*}

Rearranging terms to prepare for integration by parts.

\begin{equation*}
-i \hbar \iint_{\mathbb{R}^{2}} \mathrm{~d} \Omega \int_{0}^{\infty} \mathrm{d} r \frac{\partial}{\partial r}\left(\phi^{*} \varphi r\right) r+i \hbar \iint_{\mathbb{R}^{2}} \mathrm{~d} \Omega \int_{0}^{\infty} \mathrm{d} r \frac{\partial \phi^{*}}{\partial r} \varphi r^{2} \tag{9.22}
\end{equation*}

Substituting the rearranged terms into the original integral.

\begin{align*}
& -i \hbar \iint_{\mathbb{R}^{2}} \mathrm{~d} \Omega \int_{0}^{\infty} \mathrm{d} r \frac{\partial}{\partial r}\left(\phi^{*} \varphi r\right) r \\
& =\left.i \hbar \iint_{\mathbb{R}^{2}} \mathrm{~d} \Omega \phi^{*} \varphi r^{2}\right|_{0} ^{\infty}+i \hbar \iint_{\mathbb{R}^{2}} \mathrm{~d} \Omega \int_{0}^{\infty} \mathrm{d} r \phi^{*} \varphi r \tag{9.23}
\end{align*}

Integrating by parts, with boundary terms that vanish due to the behavior of well-behaved wavefunctions.

\begin{align*}
\left(\phi, \hat{p}_{r} \varphi\right) & =i \hbar \iint_{\mathbb{R}^{2}} \mathrm{~d} \Omega \int_{0}^{\infty} \mathrm{d} r \phi^{*} \varphi r+i \hbar \iint_{\mathbb{R}^{2}} \mathrm{~d} \Omega \int_{0}^{\infty} \mathrm{d} r \frac{\partial \phi^{*}}{\partial r} \varphi r^{2}=  \tag{9.24}\\
& =i \hbar \int_{\mathbb{R}^{3}} \mathrm{~d}^{3} x\left(\frac{\partial \phi^{*}}{\partial r}+\phi^{*} \frac{1}{r}\right) \varphi
\end{align*}

Recombining terms and converting back to a volume integral.

\begin{equation*}
\left(\phi, \hat{p}_{r} \varphi\right)^{*}=-i \hbar \int_{\mathbb{R}^{3}} \mathrm{~d}^{3} x\left(\frac{\partial \phi}{\partial r}+\phi \frac{1}{r}\right) \varphi^{*} \tag{9.25}
\end{equation*}

Taking the complex conjugate of the inner product.

\begin{equation*}
\left(\phi, \hat{p}_{r} \varphi\right)^{*}=\left(\varphi, \hat{p}_{r} \phi\right) \tag{9.26}
\end{equation*}

This confirms that the radial momentum operator is indeed Hermitian, validating our choice of operator form.

\subsection*{9.2 Schrödinger equation solution}
Now we can evaluate the square of the radial momentum operator:

\begin{align*}
\hat{p}_{r}^{2} \Psi & =-\hbar^{2}\left(\frac{\partial}{\partial r}+\frac{1}{r}\right)\left(\frac{\partial \Psi}{\partial r}+\frac{\Psi}{r}\right)= \\
& =-\hbar^{2}\left(\frac{\partial^{2} \Psi}{\partial r^{2}}+\frac{1}{r} \frac{\partial \Psi}{\partial r}-\frac{\Psi}{\not r^{2}}+\frac{1}{r} \frac{\partial \Psi}{\partial r}+\frac{\Psi}{\not r^{2}}\right)=  \tag{9.27}\\
& =-\hbar^{2}\left(\frac{\partial^{2} \Psi}{\partial r^{2}}+\frac{2}{r} \frac{\partial \Psi}{\partial r}\right)
\end{align*}

The cancellation of terms leads to a simpler form than might initially be expected.

\begin{equation*}
\hat{p}_{r}^{2}=-\hbar^{2}\left(\frac{\partial^{2}}{\partial r^{2}}+\frac{2}{r} \frac{\partial}{\partial r}\right) \tag{9.28}
\end{equation*}

This is the radial part of the Laplacian operator in spherical coordinates.

\begin{equation*}
H=-\frac{\hbar^{2}}{2 m}\left(\frac{\partial^{2}}{\partial r^{2}}+\frac{2}{r} \frac{\partial}{\partial r}\right)+\frac{L^{2}}{2 m r^{2}}+V(r) \tag{9.29}
\end{equation*}

The complete Hamiltonian combines the radial kinetic energy, angular momentum contribution, and potential energy.

\begin{equation*}
\Psi=R(r) Y_{\ell, m}(\varphi, \theta) \tag{9.30}
\end{equation*}

The separation of variables approach allows us to focus on solving for the radial function.

\begin{align*}
& {\left[-\frac{\hbar^{2}}{2 m}\left(\frac{\partial^{2}}{\partial r^{2}}+\frac{2}{r} \frac{\partial}{\partial r}\right)+\frac{L^{2}}{2 m r^{2}}+V\right] R Y_{\ell, m}=} \\
& =\left[-Y_{\ell, m} \frac{\hbar^{2}}{2 m}\left(\frac{\partial^{2}}{\partial r^{2}}+\frac{2}{r} \frac{\partial}{\partial r}\right) R+R \frac{L^{2}}{2 m r^{2}} Y_{\ell, m}+V R Y_{\ell, m}\right]=  \tag{9.31}\\
& =\left[-Y_{\ell, m} \frac{\hbar^{2}}{2 m}\left(\frac{\partial^{2} R}{\partial r^{2}}+\frac{2}{r} \frac{\partial R}{\partial r}\right)+R Y_{\ell, m} \frac{\hbar^{2} \ell(\ell+1)}{2 m r^{2}}+V R Y_{\ell, m}\right]
\end{align*}

```
This must be equal to:

\begin{equation*}
Y_{\ell, m}\left[-\frac{\hbar^{2}}{2 m}\left(\frac{\partial^{2} R}{\partial r^{2}}+\frac{2}{r} \frac{\partial R}{\partial r}\right)+R \frac{\hbar^{2} \ell(\ell+1)}{2 m r^{2}}+V R\right]=E R Y_{\ell, m} \tag{9.32}
\end{equation*}

Setting this equal to the energy eigenvalue equation.

\begin{equation*}
-\frac{\hbar^{2}}{2 m}\left(\frac{\partial^{2} R}{\partial r^{2}}+\frac{2}{r} \frac{\partial R}{\partial r}\right)+R \frac{\hbar^{2} \ell(\ell+1)}{2 m r^{2}}+V R=E R \tag{9.33}
\end{equation*}

The resulting radial differential equation contains the centrifugal potential term that depends on angular momentum.

\begin{align*}
R(r) & =\frac{u(r)}{r} \\
\frac{\partial R}{\partial r} & =\frac{\partial}{\partial r}\left(\frac{u}{r}\right)=\frac{1}{r} \frac{\partial u}{\partial r}-\frac{u}{r^{2}} \\
\frac{\partial^{2} R}{\partial r^{2}} & =\frac{\partial}{\partial r}\left(\frac{1}{r} \frac{\partial u}{\partial r}-\frac{u}{r^{2}}\right)  \tag{9.34}\\
& =\frac{1}{r} \frac{\partial^{2} u}{\partial r^{2}}-\frac{1}{r^{2}} \frac{\partial u}{\partial r}-\frac{1}{r^{2}} \frac{\partial u}{\partial r}+\frac{2 u}{r^{3}}= \\
& =\frac{1}{r} \frac{\partial^{2} u}{\partial r^{2}}-\frac{2}{r^{2}} \frac{\partial u}{\partial r}+\frac{2 u}{r^{3}}
\end{align*}

The substitution $R(r) = u(r)/r$ simplifies the differential equation by eliminating the first derivative term.

\begin{align*}
\frac{\partial^{2} R}{\partial r^{2}}+\frac{2}{r} \frac{\partial R}{\partial r} & =\frac{1}{r} \frac{\partial^{2} u}{\partial r^{2}}-\frac{2}{r^{2}} \frac{\partial u}{\partial r}+\frac{2 u}{r^{3}}+\frac{2}{r}\left(\frac{1}{r} \frac{\partial u}{\partial r}-\frac{u}{r^{2}}\right)= \\
& =\frac{1}{r} \frac{\partial^{2} u}{\partial r^{2}}-\frac{2}{y^{2}} \frac{\partial u}{\partial r}+\frac{2 u}{\not r^{3}}+\frac{2}{y^{2}} \frac{\partial u}{\partial r}-\frac{2 u}{\not r^{3}}=  \tag{9.35}\\
& =\frac{1}{r} \frac{\partial^{2} u}{\partial r^{2}}
\end{align*}

The terms cancel elegantly, resulting in a much simpler form of the radial equation.

\begin{align*}
- & \frac{1}{r} \frac{\hbar^{2}}{2 m} \frac{\partial^{2} u}{\partial r^{2}}+\frac{\hbar^{2} \ell(\ell+1)}{2 m r^{2}} \frac{u}{r}+V \frac{u}{r}=E \frac{u}{r}  \tag{9.36}\\
- & \frac{\hbar^{2}}{2 m} \frac{\partial^{2} u}{\partial r^{2}}+\frac{\hbar^{2} \ell(\ell+1)}{2 m r^{2}} u+V u=E u
\end{align*}

The transformed equation resembles a one-dimensional Schrödinger equation with an effective potential that includes the centrifugal term.

\begin{equation*}
\int \mathrm{d}^{3} x|\Psi|^{2}=1 \tag{9.37}
\end{equation*}

The normalization condition constrains the acceptable solutions.

\begin{align*}
\underbrace{\iint_{4 \pi} \mathrm{~d} \Omega\left|Y_{\ell, m}\right|^{2}}_{=1} \int_{0}^{+\infty} \mathrm{d} r |u|^{2} = \int_{0}^{+\infty} \mathrm{d} r|u|^{2} \tag{9.38}
\end{align*}

The normalization condition for the radial function simplifies due to the orthonormality of spherical harmonics.

\begin{equation*}
u(r) \sim \frac{1}{r^{\varepsilon+1 / 2}} \quad \text { for } \varepsilon>0 \tag{9.39}
\end{equation*}

The asymptotic behavior at large distances ensures normalizability.

\begin{equation*}
-r^{2} \frac{\hbar^{2}}{2 m} \frac{\partial^{2} u}{\partial r^{2}}+\frac{\hbar^{2} \ell(\ell+1)}{2 m} u+r^{2}(V-E) u=0 \tag{9.40}
\end{equation*}

Examining the behavior near the origin by multiplying by $r^2$.

\begin{equation*}
\frac{\hbar^{2}}{2 m} \frac{\partial^{2} u}{\partial r^{2}}=\frac{\hbar^{2} \ell(\ell+1)}{2 m r^{2}} u \tag{9.41}
\end{equation*}

As $r \to 0$, the centrifugal term dominates the behavior of the solution.

\begin{equation*}
\frac{\hbar^{2}}{2 m} \alpha(\alpha-1) r^{\alpha-2}=\frac{\hbar^{2} \ell(\ell+1)}{2 m} r^{\alpha-2} \tag{9.42}
\end{equation*}

Assuming a power law solution near the origin reveals the characteristic exponent.

\begin{equation*}
\alpha(\alpha-1)=\ell(\ell+1) \tag{9.43}
\end{equation*}

This quadratic equation determines the behavior of the wavefunction near the origin.

```
So either $\alpha=-\ell$ or $\alpha=\ell+1$, but we have:

\begin{equation*}
u(r)=A r^{\ell+1}+\frac{B}{r^{\ell}} \rightarrow A r^{\ell+1} \tag{9.44}
\end{equation*}

Near the origin, we select only the solution that remains finite ($\alpha=\ell+1$), rejecting the divergent solution ($\alpha=-\ell$).

\begin{equation*}
-\frac{\hbar^{2}}{2 m} \frac{\partial^{2} u}{\partial r^{2}}=E u \tag{9.45}
\end{equation*}

At large distances, the potential becomes negligible, and the equation approaches that of a free particle with negative energy.

\begin{equation*}
u(r)=D \mathrm{e}^{k r}+C \mathrm{e}^{-k r} \rightarrow C \mathrm{e}^{-k r} \tag{9.46}
\end{equation*}

For normalizability, we must select the decaying exponential solution, rejecting the growing one.

\begin{equation*}
E=-\frac{\hbar^{2} k^{2}}{2 m} \Longrightarrow k=\sqrt{\frac{2 m|E|}{\hbar^{2}}} \tag{9.47}
\end{equation*}

The negative energy indicates bound states, consistent with the physical expectation for the hydrogen atom.

\begin{equation*}
-\frac{\hbar^{2}}{2 m} \frac{\partial^{2} u}{\partial r^{2}}+\frac{\hbar^{2} \ell(\ell+1)}{2 m r^{2}} u-\frac{Z e^{2}}{r} u=E u \tag{9.48}
\end{equation*}

The complete radial equation with the Coulomb potential explicitly included.

\begin{equation*}
\frac{\partial^{2} u}{\partial r^{2}}-\frac{\ell(\ell+1)}{r^{2}} u+\frac{2 m Z e^{2}}{\hbar^{2} r} u=-\frac{2 m E}{\hbar^{2}} u \tag{9.49}
\end{equation*}

Rearranging to isolate the second derivative term.

\begin{align*}
\frac{1}{r_{0}^{2}} \frac{\partial^{2} u}{\partial x^{2}}-\frac{\ell(\ell+1)}{r_{0}^{2} x^{2}} u+\frac{2 m Z e^{2}}{\hbar^{2} r_{0} x} u & =\frac{2 m|E|}{\hbar^{2}} u \\
\frac{1}{r_{0}^{2}}\left[\frac{\partial^{2} u}{\partial x^{2}}-\frac{\ell(\ell+1)}{x^{2}} u+\frac{2 m Z e^{2} r_{0}}{\hbar^{2} x} u\right] & =\frac{2 m|E|}{\hbar^{2}} u \tag{9.50}
\end{align*}

Introducing the dimensionless variable $x = r/r_0$ simplifies the equation.

\begin{align*}
& r_{0}=\sqrt{\frac{\hbar^{2}}{2 m|E|}}=\frac{1}{k}  \tag{9.51}\\
& x_{0}=\frac{2 m Z e^{2} r_{0}}{\hbar^{2}}
\end{align*}

Defining characteristic length scales for the problem.

\begin{equation*}
\left[\frac{\partial^{2}}{\partial x^{2}}-\frac{\ell(\ell+1)}{x^{2}}+\frac{x_{0}}{x}-1\right] u=0 \tag{9.52}
\end{equation*}

The dimensionless form of the radial equation.

\begin{equation*}
u\left(r_{0} x\right)=x^{\ell+1} \mathrm{e}^{-x} W(x) \tag{9.53}
\end{equation*}

Proposing a solution form that satisfies both boundary conditions through the factor $W(x)$.

\begin{align*}
\frac{\partial u}{\partial x} & =(\ell+1) x^{\ell} \mathrm{e}^{-x} W-x^{\ell+1} \mathrm{e}^{-x} W+x^{\ell+1} \mathrm{e}^{-x} \frac{\partial W}{\partial x} \\
\frac{\partial^{2} u}{\partial x^{2}} & =\frac{\partial}{\partial x}\left((\ell+1) x^{\ell} \mathrm{e}^{-x} W-x^{\ell+1} \mathrm{e}^{-x} W+x^{\ell+1} \mathrm{e}^{-x} \frac{\partial W}{\partial x}\right)= \\
& =\ell(\ell+1) x^{\ell-1} \mathrm{e}^{-x} W-(\ell+1) x^{\ell} \mathrm{e}^{-x} W+(\ell+1) x^{\ell} \mathrm{e}^{-x} \frac{\partial W}{\partial x}-(\ell+1) x^{\ell} \mathrm{e}^{-x} W+ \\
& +x^{\ell+1} \mathrm{e}^{-x} W-x^{\ell+1} \mathrm{e}^{-x} \frac{\partial W}{\partial x}+(\ell+1) x^{\ell} \mathrm{e}^{-x} \frac{\partial W}{\partial x}-x^{\ell+1} \mathrm{e}^{-x} \frac{\partial W}{\partial x}+x^{\ell+1} \mathrm{e}^{-x} \frac{\partial^{2} W}{\partial x^{2}}= \\
& =\left(\frac{\ell(\ell+1)}{x^{2}}-\frac{2(\ell+1)}{x}+1\right) x^{\ell+1} \mathrm{e}^{-x} W+2\left(\frac{\ell+1}{x}-1\right) x^{\ell+1} \mathrm{e}^{-x} \frac{\partial W}{\partial x}+x^{\ell+1} \mathrm{e}^{-x} \frac{\partial^{2} W}{\partial x^{2}} \tag{9.54}
\end{align*}

Calculating the derivatives needed to substitute into the differential equation.

```
And so (9.52) becomes:

\begin{align*}
& \left(\frac{\ell(\ell+1)}{x^{2}}-\frac{2(\ell+1)}{x}+1\right) x^{\ell+1} e^{-x} W+2\left(\frac{\ell+1}{x}-1\right) x^{\ell+1} e^{-x} \frac{\partial W}{\partial x}+x^{\ell+1} e^{-x} \frac{\partial^{2} W}{\partial x^{2}}- \\
& -\frac{\ell(\ell+1)}{x^{2}} x^{\ell+1} e^{-x} W+\frac{x_{0}}{x} x^{\ell+1} e^{-x} W-x^{\ell+1} e^{-x} W=0 \\
& \left.\left(\frac{\ell(\ell+1)}{x^{2}}-\frac{2(\ell+1)}{x}+\not\right)\right) W+2\left(\frac{\ell+1}{x}-1\right) \frac{\partial W}{\partial x}+\frac{\partial^{2} W}{\partial x^{2}}- \\
& -\frac{\ell(\ell+1)}{x^{2}} W(x)+\frac{x_{0}}{x} W- W = 0 \\
& -\frac{2(\ell+1)}{x} W+2\left(\frac{\ell+1}{x}-1\right) \frac{\partial W}{\partial x}+\frac{\partial^{2} W}{\partial x^{2}}+\frac{x_{0}}{x} W=0 \\
& 2(\ell+1-x) \frac{\partial W}{\partial x}+x \frac{\partial^{2} W}{\partial x^{2}}+\left(x_{0}-2(\ell+1)\right) W=0 \tag{9.55}
\end{align*}

Substituting the trial solution and simplifying by canceling common factors.

\begin{equation*}
x \frac{\partial^{2} W}{\partial x^{2}}+2(\ell+1-x) \frac{\partial W}{\partial x}+\left(x_{0}-2(\ell+1)\right) W=0 \tag{9.56}
\end{equation*}

This is the associated Laguerre differential equation, a well-known equation in mathematical physics.

\begin{equation*}
W=\sum_{k=0}^{\infty} a_{k} x^{k} \tag{9.57}
\end{equation*}

Attempting a power series solution to find the allowed energy values.

\begin{align*}
\frac{\partial W}{\partial x} & =\sum_{k=0}^{\infty} k a_{k} x^{k-1} \\
x \frac{\partial^{2} W}{\partial x^{2}} & =x \sum_{k=0}^{\infty} k(k-1) a_{k} x^{k-2}=\sum_{k=0}^{\infty} k(k-1) a_{k} x^{k-1} \tag{9.58}
\end{align*}

Computing the derivatives of the power series.

\begin{equation*}
\sum_{k=0}^{\infty} a_{k}\left[k(k-1) x^{k-1}+2(\ell+1) k x^{k-1}-2 k x^{k}+\left(x_{0}-2(\ell+1)\right) x^{k}\right]=0 \tag{9.59}
\end{equation*}

Substituting into the differential equation and collecting terms.

\begin{align*}
& \sum_{k=0}^{\infty} a_{k}[k(k-1)+2(\ell+1) k] x^{k-1}+\sum_{k=0}^{\infty} a_{k}\left(x_{0}-2(\ell+1)-2 k\right) x^{k} \\
& \sum_{k=0}^{\infty} a_{k+1}[k(k+1)+2(\ell+1)(k+1)] x^{k}+\sum_{k=0}^{\infty} a_{k}\left(x_{0}-2(\ell+1)-2 k\right) x^{k} \tag{9.60}
\end{align*}

Shifting indices to match powers of x.

\begin{align*}
& a_{k+1}[k(k+1)+2(\ell+1)(k+1)]+a_{k}\left(x_{0}-2(\ell+1)-2 k\right)=0 \\
& a_{k+1}=-\frac{x_{0}-2(\ell+1)-2 k}{k(k+1)+2(\ell+1)(k+1)} a_{k}  \tag{9.61}\\
& a_{k+1}=-\frac{x_{0}-2(\ell+1+k)}{(k+1)(k+2 l+2)} a_{k}
\end{align*}

The recurrence relation between successive coefficients in the power series.

\begin{align*}
& a_{N+2}=(\ldots) a_{N+1}=0 \\
& a_{N+3}=(\ldots) a_{N+2}=0  \tag{9.62}\\
& \ldots
\end{align*}

For physical solutions, the series must terminate, creating a polynomial.

\begin{equation*}
a_{N+1}=0 \Longrightarrow-\frac{x_{0}-2(\ell+1+N)}{(k+1)(k+2 l+2)} a_{N}=0 \Longrightarrow x_{0}=2(\ell+1+N) \tag{9.63}
\end{equation*}

The condition for the series to terminate yields a quantization condition.

\begin{equation*}
\frac{2 m Z e^{2} r_{0}}{\hbar^{2}}=2 n \tag{9.64}
\end{equation*}

Substituting the definition of $x_0$ and introducing the principal quantum number $n = \ell+1+N$.

\begin{align*}
& \frac{Z e^{2}}{\hbar} \sqrt{\frac{2 m}{|E|}}=2 n \\
& \frac{Z^{2} e^{4}}{\hbar^{2}} \frac{2 m}{|E|}=4 n^{2}  \tag{9.65}\\
& |E|=\frac{m Z^{2} e^{4}}{2 n^{2} \hbar^{2}}
\end{align*}

This is the famous Bohr formula for the energy levels of the hydrogen atom, derived through quantum mechanics.

```
The number $n$ is the principal quantum number and is an integer. Since the energy depends on $n$ we call it $E_{n}$. Also remember that we used the absolute value but $E_{n}$ is negative so:

\begin{equation*}
E_{n}=-\frac{m Z^{2} e^{4}}{2 n^{2} \hbar^{2}} \tag{9.66}
\end{equation*}

This negative energy formula matches the Bohr model results but is derived rigorously from quantum mechanics.

\begin{equation*}
N=n-\ell-1 \geq 0 \Longrightarrow \ell \leq n-1 \tag{9.67}
\end{equation*}

This constraint establishes the allowed values of angular momentum for a given principal quantum number.

\begin{equation*}
\Psi=R_{n}(r) Y_{\ell, m}=\Psi_{n \ell m} \tag{9.68}
\end{equation*}

The complete wavefunction depends on all three quantum numbers, which determine its spatial properties.

\begin{itemize}
  \item 2 quantum numbers associated to the Legendre polynomials
  \item 1 quantum number associated with the Laguerre polynomials
\end{itemize}

The three quantum numbers arise naturally from the separation of variables in spherical coordinates.

\begin{align*}
& n=0,1,2, \ldots \\
& \ell=0,1,2, \ldots, n-1  \tag{9.69}\\
& m=-\ell, \ldots, 0, \ldots, \ell
\end{align*}

These are the ranges for the quantum numbers that characterize the hydrogen atom states.

\begin{equation*}
\sum_{\ell=0}^{n-1}(2 l+1)=2 \sum_{\ell=0}^{n-1} \ell+\sum_{\ell=0}^{n-1} 1=2 \frac{n(n-1)}{2}+n=n^{2} \tag{9.70}
\end{equation*}

The degeneracy of the energy level $E_n$ is $n^2$, accounting for all possible angular momentum states.

\begin{align*}
& 4 x \frac{\partial^{2} W}{\partial(2 x)^{2}}+4(\ell+1-x) \frac{\partial W}{\partial 2 x}+2 N W=0 \\
& 2 y \frac{\partial^{2} W}{\partial y^{2}}+4\left(\ell+1-\frac{y}{2}\right) \frac{\partial W}{\partial y}+2 N W=0  \tag{9.71}\\
& y \frac{\partial^{2} W}{\partial y^{2}}+2\left(\ell+1-\frac{y}{2}\right) \frac{\partial W}{\partial y}+N W=0 \\
& y \frac{\partial^{2} W}{\partial y^{2}}+(s+1-y) \frac{\partial W}{\partial y}+N W=0
\end{align*}

Transforming the differential equation into the standard form for associated Laguerre polynomials.

\begin{equation*}
W(y)=L_{N}^{s}(y) \tag{9.72}
\end{equation*}

The solution is expressed in terms of associated Laguerre polynomials.

\begin{equation*}
L_{N}^{s}(y)=\sum_{k=0}^{N} \frac{(-1)^{k}(N+s)!}{k!(k+s)!(N-k)!} y^{k} \tag{9.73}
\end{equation*}

The explicit form of the associated Laguerre polynomials as a finite series.

\begin{equation*}
L_{N}^{s}(y)=(-1)^{s} \frac{\mathrm{~d}^{s}}{\mathrm{~d} y^{s}} L_{N+s} \quad \text { with } \quad L_{r}=\mathrm{e}^{y} \frac{\mathrm{~d}^{r}}{\mathrm{~d} y^{r}} \mathrm{e}^{-y} y^{r} \tag{9.74}
\end{equation*}

An alternative definition relating associated Laguerre polynomials to derivatives of standard Laguerre polynomials.

\subsection*{9.3 Properties of the Laguerre polynomials}
\section*{Property 1.}

\begin{equation*}
L_{N}^{s}(y)=(-1)^{s} \frac{\mathrm{~d}^{s}}{\mathrm{~d} y^{s}} L_{N+s} \quad \text { with } \quad L_{r}=\mathrm{e}^{y} \frac{\mathrm{~d}^{r}}{\mathrm{~d} y^{r}} \mathrm{e}^{-y} y^{r} \tag{9.75}
\end{equation*}

This property connects associated Laguerre polynomials to derivatives of standard Laguerre polynomials.

Property 2. Recursive formula:

\begin{equation*}
(N+1) L_{N+1}^{s}(y)-(2 N+s+1-y) L_{N}^{s}(y)+(N+s) L_{N-1}^{s}(y)=0 \tag{9.76}
\end{equation*}

The recurrence relation allows efficient computation of higher-order polynomials from lower ones.

```
\section*{Property 3.}

\begin{equation*}
\int_{0}^{+\infty} \mathrm{d} y \mathrm{e}^{-y} y^{s} L_{r}^{s} L_{q}^{s}=\delta_{r q} \frac{(r+s)!}{r!} \tag{9.77}
\end{equation*}

The orthogonality relation for associated Laguerre polynomials with respect to the weight function $e^{-y}y^s$.

\section*{Property 4.}

\begin{equation*}
\int_{0}^{+\infty} \mathrm{d} y \mathrm{e}^{-y} y^{s+1}\left(L_{N}^{s}\right)^{2}=(2 N+s+1) \frac{(N+s)!}{N!} \tag{9.78}
\end{equation*}

A specific weighted integral of squared associated Laguerre polynomials.

\begin{equation*}
\Psi_{n \ell m}=D_{n \ell} Y_{l m}\left(2 k_{n} r\right)^{\ell} \mathrm{e}^{-k_{n} r} L_{n-\ell-1}^{2 l+1}\left(2 k_{n} r\right) \tag{9.79}
\end{equation*}

The complete hydrogen atom wavefunction with normalization constant $D_{n\ell}$.

\begin{align*}
& \left(\Psi_{n \ell m}, \Psi_{n^{\prime} \ell^{\prime} m^{\prime}}\right)=\underbrace{\iint_{4 \pi} \mathrm{~d} \Omega Y_{l m}^{*} Y_{\ell^{\prime} m^{\prime}}}_{\delta_{l l^{\prime}} \delta_{m m^{\prime}}} \\
& \int_{0}^{+\infty} \mathrm{d} r r^{2} D_{n l}^{*} D_{n^{\prime} \ell^{\prime}}\left(2 k_{n} r\right)^{\ell}\left(2 k_{n^{\prime}} r\right)^{\ell^{\prime}} \mathrm{e}^{-k_{n} r} \mathrm{e}^{-k_{n^{\prime}} r} L_{n-\ell-1}^{2 l+1}\left(2 k_{n} r\right) L_{n^{\prime}-\ell^{\prime}-1}^{2 l^{\prime}+1}\left(2 k_{n^{\prime}} r\right) \tag{9.80}
\end{align*}

Computing the inner product between two hydrogen atom wavefunctions.

\begin{align*}
\left(\Psi_{n \ell m}, \Psi_{n^{\prime} \ell^{\prime} m^{\prime}}\right) & =\int_{0}^{+\infty} \mathrm{d} r r^{2} D_{n l}^{*} D_{n^{\prime} \ell}\left(2 k_{n} r\right)^{\ell}\left(2 k_{n^{\prime}} r\right)^{\ell} \mathrm{e}^{-k_{n} r} \mathrm{e}^{-k_{n^{\prime}} r}\left[L_{n-\ell-1}^{2 l+1}\left(2 k_{n} r\right)\right]\left[L_{n^{\prime}-\ell-1}^{2 l+1}\left(2 k_{n^{\prime}} r\right)\right]= \\
& =\int_{0}^{+\infty} \mathrm{d} r D_{n l}^{*} D_{n^{\prime} \ell}\left(2 k_{n}\right)^{\ell}\left(2 k_{n}\right)^{\ell^{\prime}} r^{(2 l+1)+1} \mathrm{e}^{-r\left(k_{n}+k_{n^{\prime}}\right)}\left[L_{n-\ell-1}^{2 l+1}\left(2 k_{n} r\right)\right]\left[L_{n^{\prime}-\ell-1}^{2 l+1}\left(2 k_{n^{\prime}} r\right)\right]= \tag{9.81}
\end{align*}

Due to orthogonality of spherical harmonics, the integral is non-zero only when $\ell=\ell'$ and $m=m'$.

\begin{align*}
& \int_{0}^{+\infty} \mathrm{d}\left(2 k_{n} r\right)\left|D_{n \ell}\right|^{2}\left(2 k_{n}\right)^{2 l-1} r^{(2 l+1)+1} \mathrm{e}^{-2 r k_{n}}\left[L_{n-\ell-1}^{2 l+1}\left(2 k_{n} r\right)\right]^{2}= \\
& =\int_{0}^{+\infty} \mathrm{d}\left(2 k_{n} r\right)\left|D_{n \ell}\right|^{2}\left(2 k_{n}\right)^{-3}\left(2 k_{n} r\right)^{(2 l+1)+1} \mathrm{e}^{-2 r k_{n}}\left[L_{n-\ell-1}^{2 l+1}\left(2 k_{n} r\right)\right]^{2} \tag{9.82}
\end{align*}

For the case where $n=n'$, the integral simplifies and can be evaluated using Property 4.

\begin{align*}
& \frac{\left|D_{n \ell}\right|^{2}}{\left(2 k_{n}\right)^{3}}(2(n-\ell-1)+2 l+1+1) \frac{(n-\ell-1+2 l+1)!}{(n-\ell-1)!}=  \tag{9.83}\\
& =\frac{\left|D_{n \ell}\right|^{2} 2 n}{\left(2 k_{n}\right)^{3}} \frac{(n+\ell)!}{(n-\ell-1)!}
\end{align*}

Applying Property 4 and simplifying the expression.

\begin{equation*}
D_{n \ell}=\sqrt{\frac{\left(2 k_{n}\right)^{3}(n-\ell-1)!}{2 n(n+\ell)!}} \tag{9.84}
\end{equation*}

The normalization constant determined to ensure that the wavefunction has unit norm.

\subsection*{9.4 Energy spectrum}
The energy of a Hydrogen atom $E_{n}$ can be written in terms of a quantity called the Bohr radius $a$ :

\begin{equation*}
a=\frac{\hbar^{2}}{m_{e} e^{2}} \approx 0.529 \times 10^{-10} \mathrm{~m} \tag{9.85}
\end{equation*}

The Bohr radius represents the most probable distance of the electron from the nucleus in the ground state.

\begin{equation*}
E_{n}=-\frac{m_{e} Z^{2} e^{4}}{2 \hbar^{2} n^{2}}=-\frac{(Z e)^{2}}{2 a n^{2}} \tag{9.86}
\end{equation*}

The energy levels expressed in terms of the Bohr radius.

\begin{equation*}
\alpha=\frac{r^{2}}{\hbar c}=\frac{1}{137} \tag{9.87}
\end{equation*}

The fine structure constant, a fundamental physical constant that characterizes the strength of the electromagnetic interaction.

```
Which gives:

\begin{equation*}
E_{n}=-\frac{m_{e} c^{2}}{2} \alpha^{2} \frac{Z^{2}}{n^{2}} \tag{9.88}
\end{equation*}

The energy levels expressed in terms of the fine structure constant and the rest energy of the electron.

\begin{equation*}
\left|E_{n}-E_{m}\right|=\hbar \omega=\frac{h c}{\lambda} \tag{9.89}
\end{equation*}

When an electron transitions between energy levels, the energy difference corresponds to the energy of an absorbed or emitted photon.

\begin{equation*}
E_{1}=-\frac{m_{e} Z^{2} e^{4}}{2 \hbar^{2}} \tag{9.90}
\end{equation*}

The ground state energy of the hydrogen atom.

\begin{equation*}
E_{n}=\frac{E_{1}}{n^{2}} \tag{9.91}
\end{equation*}

All energy levels can be expressed in terms of the ground state energy.

\begin{equation*}
\left|E_{n}-E_{m}\right|=\left|\frac{E_{1}}{n^{2}}-\frac{E_{1}}{m^{2}}\right|=\left|E_{1}\right|\left|\frac{1}{n^{2}}-\frac{1}{m^{2}}\right| \tag{9.92}
\end{equation*}

The energy difference between levels expressed in terms of principal quantum numbers.

\begin{equation*}
\left|E_{1}\right|\left|\frac{1}{n^{2}}-\frac{1}{m^{2}}\right|=\frac{h c}{\lambda} \Longrightarrow \frac{1}{\lambda}=\frac{\left|E_{1}\right|}{h c}\left|\frac{1}{n^{2}}-\frac{1}{m^{2}}\right| \tag{9.93}
\end{equation*}

Relating the wavelength of emitted or absorbed light to energy level differences.

\begin{equation*}
\frac{1}{\lambda}=\frac{\left|E_{1}\right|}{h c}\left(\frac{1}{n^{2}}-\frac{1}{m^{2}}\right)=R\left(\frac{1}{n^{2}}-\frac{1}{m^{2}}\right) \tag{9.94}
\end{equation*}

This is the Rydberg formula, which predicts the wavelengths of spectral lines with remarkable accuracy.

\subsection*{9.5 Radial probability density}

\begin{equation*}
\Psi_{n \ell m}=R_{n l}(r) Y_{\ell, m}(\varphi, \theta) \tag{9.95}
\end{equation*}

The complete wavefunction separates into radial and angular components.

\begin{equation*}
\left|\Psi_{n \ell m}\right|^{2}=\left|R_{n l} Y_{\ell, m}\right|^{2}=R_{n l}^{2}\left|Y_{\ell, m}\right|^{2} \tag{9.96}
\end{equation*}

The probability density is the square of the wavefunction's magnitude.

\begin{equation*}
\left(\Psi_{n \ell m}, \Psi_{n \ell m}\right)=1 \tag{9.97}
\end{equation*}

The normalization condition ensures that the total probability is 1.

\begin{equation*}
\underbrace{\iint_{4 \pi} \mathrm{~d} \Omega\left|Y_{\ell, m}\right|^{2}}_{=1} \int_{0}^{+\infty} \mathrm{d} r r^{2} R_{n l}^{2}=\int_{0}^{+\infty} \mathrm{d} r r^{2} R_{n l}^{2} \tag{9.98}
\end{equation*}

The spherical symmetry allows separation of the radial and angular parts of the normalization integral.

\begin{equation*}
\mathrm{d} P=P_{n l} \mathrm{~d} r \tag{9.99}
\end{equation*}

The radial probability density $P_{nl} = r^2 R_{nl}^2$ gives the probability of finding the electron in a spherical shell of radius $r$ and thickness $dr$. This resolves the apparent paradox for $\ell=0$ states, as the probability density properly approaches zero at the nucleus due to the $r^2$ factor, even though $R_{nl}$ remains finite there.

```
This is the probability of finding an electron in a shell of $r$ radius and $\mathrm{d} r$ thickness.

\subsection*{9.6 Radial position expectation values}
By using the recursion formula of the Laguerre polynomials we can show that:

\begin{equation*}
\langle r\rangle_{n, \ell}=\frac{a}{2 Z}\left(3 n^{2}-\ell(\ell+1)\right) \tag{9.100}
\end{equation*}

The expectation value of the radial position depends on both quantum numbers.

\begin{equation*}
\langle r\rangle_{n, n-1}=\frac{a}{2 Z}\left(3 n^{2}-n(n-1)\right)=\frac{a}{2 Z}\left(3 n^{2}-n^{2}+n\right)=\frac{a}{Z} n\left(n+\frac{1}{2}\right) \tag{9.101}
\end{equation*}

For the maximum angular momentum states, the expectation value simplifies.

\begin{equation*}
\left\langle r^{2}\right\rangle_{n, n-1}=\frac{a^{2}}{Z^{2}} n^{2}(n+1)\left(n+\frac{1}{2}\right) \tag{9.102}
\end{equation*}

The expectation value of $r^2$ for maximum angular momentum states.

\begin{align*}
\Delta r & =\sqrt{\left\langle r^{2}\right\rangle_{n, n-1}-\langle r\rangle_{n, n-1}^{2}}=\sqrt{\frac{a^{2}}{Z^{2}} n^{2}(n+1)\left(n+\frac{1}{2}\right)-\frac{a^{2}}{Z^{2}} n^{2}\left(n+\frac{1}{2}\right)^{2}}= \\
& =\sqrt{\frac{a^{2}}{Z^{2}} n^{2}\left(n+\frac{1}{2}\right)\left(n+1-n-\frac{1}{2}\right)}=\frac{a n}{Z \sqrt{2}} \sqrt{n+\frac{1}{2}} \tag{9.103}
\end{align*}

The uncertainty in radial position calculated from the variance.

\begin{equation*}
\frac{\Delta r}{\langle r\rangle_{n, n-1}}=\frac{a \pi}{\not Z \sqrt{2}} \sqrt{n+\frac{1}{2}} \frac{Z}{a n} \frac{1}{\left(n+\frac{1}{2}\right)}=\frac{1}{\sqrt{2 n+1}} \tag{9.104}
\end{equation*}

The relative uncertainty decreases with increasing quantum number, approaching classical behavior for large $n$.

\begin{equation*}
P_{n, n-1}=C r^{2 n} \mathrm{e}^{-2 k_{n} r} \Longrightarrow \frac{\mathrm{~d} P_{n, n-1}}{\mathrm{~d} r}=C r^{2 n} \mathrm{e}^{-2 k_{n} r}\left(\frac{2 n}{r}-2 k_{n}\right) \tag{9.105}
\end{equation*}

The radial probability density and its derivative for maximum angular momentum states.

\begin{equation*}
r=\frac{n}{k_{n}}=\frac{a n^{2}}{Z} \tag{9.106}
\end{equation*}

The radius at which the probability density reaches its maximum.

\begin{equation*}
\langle r\rangle_{n, n-1}=\frac{a}{Z} n\left(n+\frac{1}{2}\right) \sim \frac{a n^{2}}{Z} \tag{9.107}
\end{equation*}

For large $n$, the most probable radius approaches the expectation value, consistent with classical orbits.

\subsection*{9.7 Two body problem}
The system given by an electron orbiting a proton is essentially a two body problem. In the discussion of the Hydrogen atom we have only discussed the orbital problem at rest, but the proton is, in general, moving.

The centre of mass is essentially the proton since $m_{p} \gg m_{e}$, instead the reduced mass will be approximately the one of the electron $\mu \approx m_{e}$.

If we analyze the problem in classical mechanics we find that the kinetic energy can be divided into the contribution of the centre of mass and the contribution of a free particle of mass $\mu$ :

\begin{equation*}
T=\frac{\vec{P}^{2}}{2 M}+\frac{\vec{p}^{2}}{2 \mu} \tag{9.108}
\end{equation*}

The kinetic energy separates into center of mass motion and relative motion terms.

\begin{equation*}
\mathcal{H}=\frac{\vec{P}^{2}}{2 M}+\frac{\vec{p}^{2}}{2 \mu}+V(r) \tag{9.109}
\end{equation*}

The Hamiltonian includes both kinetic energy terms and the potential energy.

\begin{align*}
& \left\{X_{i}, P_{j}\right\}=\delta_{i j} \\
& \left\{x_{i}, p_{j}\right\}=\delta_{i j} \\
& \left\{X_{i}, x_{j}\right\}=0 \\
& \left\{X_{i}, p_{j}\right\}=0  \tag{9.110}\\
& \left\{x_{i}, P_{j}\right\}=0 \\
& \left\{P_{i}, p_{j}\right\}=0
\end{align*}

The Poisson brackets demonstrate that the center of mass and relative coordinates form independent canonical pairs.

\begin{align*}
& {\left[\hat{X}_{i}, \hat{P}_{j}\right]=i \hbar \delta_{i j}} \\
& {\left[\hat{x}_{i}, \hat{p}_{j}\right]=i \hbar \delta_{i j}} \\
& {\left[\hat{X}_{i}, \hat{x}_{j}\right]=0} \\
& {\left[\hat{X}_{i}, \hat{p}_{j}\right]=0}  \tag{9.111}\\
& {\left[\hat{x}_{i}, \hat{P}_{j}\right]=0} \\
& {\left[\hat{P}_{i}, \hat{p}_{j}\right]=0}
\end{align*}

The commutation relation $[\hat{P}, H]=0$ leads to a separable Hamiltonian form:

\begin{equation*}
H=H_{CM}+H_{rel}=\frac{\hat{P}^{2}}{2 M}+\frac{\hat{p}^{2}}{2 m_{e}}+V(r) \tag{9.112}
\end{equation*}

Where $H_{CM}$ describes center-of-mass motion and $H_{rel}$ represents the internal hydrogen dynamics. This separation allows us to factorize the wavefunction:

\begin{equation*}
\Psi=\phi_{\vec{K}} \Psi_{n \ell m}=\Psi_{\vec{K} n \ell m} \tag{9.113}
\end{equation*}

The total energy eigenvalue becomes:

\begin{align*}
H \Psi_{\vec{K} n \ell m} & =\left(\frac{\hat{P}^{2}}{2 M}+H_{rel}\right) \phi_{\vec{K}} \Psi_{n \ell m}=\Psi_{n \ell m} \frac{\hat{P}^{2}}{2 M} \phi_{\vec{K}}+\phi_{\vec{K}} H_{rel} \Psi_{n \ell m}=  \tag{9.114}\\
& =\Psi_{n \ell m} \phi_{\vec{K}} \frac{\hbar^{2} K^{2}}{2 M}+\phi_{\vec{K}} \Psi_{n \ell m} E_{n}=\left(\frac{\hbar^{2} K^{2}}{2 M}+E_{n}\right) \Psi_{n \ell m} \phi_{\vec{K}}
\end{align*}

\section*{10 Equivalent formulations of quantum theory}
\subsection*{10.1 Dirac's vector notation}

Comparing Schrödinger's wave mechanics with Dirac's abstract vector formalism:

\begin{center}
\begin{tabular}{|l|l|l|}
\hline
Schrödinger &  & Dirac \\
\hline
Wavefunctions $\Psi_{n}$ & $\Longleftrightarrow$ & Vectors $|n\rangle$ \\
\hline
Fourier series decomposition: & $\Longleftrightarrow$ & Vector components decomposition: \\
\hline
$\Psi_{v}=\sum_{i} v_{i} \phi_{i}$ &  & $|v\rangle=\sum_{i} v_{i} \vec{i}$ \\
\hline
Operators $\hat{A}$ & $\Longleftrightarrow$ & Matrices $\tilde{A}$ \\
\hline
Probability density $\rho=|\Psi|^{2}$ &  & ? \\
\hline
\end{tabular}
\end{center}

The probabilistic interpretation remains intact in Dirac's formalism through the inner product:

\begin{equation*}
\left(\Psi_{u}, \Psi_{v}\right)=\langle u \mid v\rangle=\sum_{i} u_{i}^{*} v_{i} \tag{10.1}
\end{equation*}

For the special case where $u=v$:

\begin{equation*}
\left(\Psi_{v}, \Psi_{v}\right)=\langle v \mid v\rangle=\sum_{i}\left|v_{i}\right|^{2} \tag{10.2}
\end{equation*}

Here $\left|v_{i}\right|^{2}$ represents the probability of finding the system in state $\phi_{i}$ or $|i\rangle$.

Operators in Schrödinger's formulation transform into matrices in Dirac's approach. This transformation occurs through the basis representation:

\begin{equation*}
\mathcal{B}=\left\{\phi_{i}: i \in[0,+\infty]\right\} \tag{10.3}
\end{equation*}

The matrix elements are defined as:

\begin{equation*}
A_{n m}=\left(\phi_{n}, A \phi_{m}\right) \tag{10.4}
\end{equation*}

Leading to the complete operator representation:

\begin{equation*}
\tilde{A}=\sum_{i} \sum_{j}|i\rangle A_{i j}\langle j| \tag{10.5}
\end{equation*}


Recall that $|i\rangle\langle j|$ generates a $n \times n$ matrix with all zeros and 1 in position $i j$. For example let $i=1, j=2$ with $n=3$ :

\[
|1\rangle\langle 2|=\left[\begin{array}{l}
1  \tag{10.6}\\
0 \\
0
\end{array}\right]\left[\begin{array}{lll}
0 & 1 & 0
\end{array}\right]=\left[\begin{array}{lll}
0 & 1 & 0 \\
0 & 0 & 0 \\
0 & 0 & 0
\end{array}\right]
\]

Using bra-ket notation, we can derive the matrix elements:

\begin{equation*}
\langle n| \tilde{A}|m\rangle=\langle n|\left(\sum_{i} \sum_{j}|i\rangle A_{i j}\langle j|\right)|m\rangle=\sum_{i} \sum_{j} \underbrace{\langle n \mid i\rangle}_{\delta_{n i}} A_{i j} \underbrace{\langle j \mid m\rangle}_{\delta_{j m}}=\sum_{i} \sum_{j} \delta_{n i} A_{i j} \delta_{j m}=A_{n m} \tag{10.7}
\end{equation*}

The adjoint matrix follows naturally from the Schrödinger picture definition:

\begin{equation*}
A_{m n}^{\dagger}=\left(\phi_{m}, A^{\dagger} \phi_{n}\right)=\left(A \phi_{m}, \phi_{n}\right)=\left(\phi_{n}, A \phi_{m}\right)^{*}=A_{n m}^{*} \tag{10.8}
\end{equation*}

Thus the adjoint matrix equals the transpose complex conjugate. For Hermitian operators where $A^{\dagger}=A$, we have $A_{n m}^{\dagger}=A_{n m}$.

When the basis $\mathcal{B}$ diagonalizes operator $A$, the matrix representation simplifies to:

\begin{equation*}
\tilde{A}=\sum_{n} \sum_{m}|n\rangle a_{n} \delta_{n m}\langle m|=\sum_{n} a_{n}|n\rangle\langle n| \tag{10.9}
\end{equation*}

Where $a_n$ are the eigenvalues satisfying:

\begin{equation*}
A \phi_{n}=a_{n} \phi_{n} \tag{10.10}
\end{equation*}

\subsection*{10.2 Basis transformations}
When working with a different basis $\mathcal{B}^{\prime}=\left\{\phi_{k}^{\prime}: k \in[0,+\infty]\right\}$, we need transformation rules between bases.

Lemma 10.1. Two complete orthonormal bases $\mathcal{B}$ and $\mathcal{B}^{\prime}$ are connected by a unitary transformation.

Proof. Any element of $\mathcal{B}^{\prime}$ can be expanded in terms of $\mathcal{B}$:

\begin{equation*}
\phi_{n}^{\prime}=\sum_{m} S_{m n} \phi_{m} \tag{10.11}
\end{equation*}

Where the transformation coefficients are:

\begin{equation*}
S_{m n}=\left(\phi_{m}, \phi_{n}^{\prime}\right) \tag{10.12}
\end{equation*}

For $S$ to be unitary, it must satisfy:

\begin{equation*}
S S^{\dagger}=S^{\dagger} S=\mathbb{I} \Longleftrightarrow \sum_{i} S_{m i} S_{i n}^{\dagger}=\sum_{i} S_{m i}^{\dagger} S_{i n}=\delta_{n m} \tag{10.13}
\end{equation*}

Using $S_{i n}^{\dagger}=S_{n i}^{*}$ and the completeness of basis $\mathcal{B}^{\prime}$:

\begin{align*}
\sum_{i} S_{m i} S_{i n}^{\dagger} & =\sum_{i} S_{m i} S_{n i}^{*}=\sum_{i}\left(\phi_{m}, \phi_{i}^{\prime}\right)\left(\phi_{n}, \phi_{i}^{\prime}\right)^{*}=\sum_{i}\left(\phi_{m}, \phi_{i}^{\prime}\right)\left(\phi_{i}^{\prime}, \phi_{n}\right)= \\
& =\sum_{i}\left(\int \mathrm{d} x \phi_{m}^{*}(x) \phi_{i}^{\prime}(x)\right)\left(\int \mathrm{d} y \phi_{i}^{*}(y) \phi_{n}(y)\right)= \\
& =\int \mathrm{d} x \int \mathrm{d} y \phi_{m}^{*}(x) \phi_{n}(y) \underbrace{\sum_{i}\left[\phi_{i}^{\prime}(x) \phi_{i}^{\prime}(y)\right]}_{\delta(x-y)}= \\
& =\int \mathrm{d} x \int \mathrm{d} y \phi_{m}^{*}(x) \phi_{n}(y) \delta(x-y)= \\
& =\int \mathrm{d} x \phi_{m}^{*}(x) \phi_{n}(x)=\left(\phi_{m}, \phi_{n}\right)=\delta_{m n} \tag{10.14}
\end{align*}

This completes the proof.

Corollary 10.2. Given a unitary transformation $S$ from basis $\mathcal{B}^{\prime}$ to $\mathcal{B}$, the inverse transformation is $S^{\dagger}$.

Proof.

\begin{equation*}
\sum_{n} S_{n i}^{\dagger} \phi_{n}^{\prime}=\sum_{n} S_{n i}^{\dagger} \sum_{m} S_{m n} \phi_{m}=\sum_{m} \underbrace{\left(\sum_{n} S_{m n} S_{n i}^{\dagger}\right)}_{\delta_{m i}} \phi_{m}=\sum_{m} \delta_{m i} \phi_{m}=\phi_{i} \tag{10.15}
\end{equation*}


So:

\begin{equation*}
\phi_{i}=\sum_{n} S_{n i}^{\dagger} \phi_{n}^{\prime} \tag{10.16}
\end{equation*}

The transformation also relates vector components between bases:

\begin{equation*}
v_{n}^{\prime}=\left(\phi_{n}^{\prime}, \Psi_{v}\right)=\left(\sum_{m} S_{m n} \phi_{m}, \sum_{i} v_{i} \phi_{i}\right)=\sum_{m} \sum_{i} S_{m n}^{*} v_{i} \underbrace{\left(\phi_{m}, \phi_{i}\right)}_{\delta_{m i}}=\sum_{m} \sum_{i} S_{m n}^{*} v_{i} \delta_{m i}=\sum_{m} S_{m n}^{*} v_{m} \tag{10.17}
\end{equation*}

Using $S_{m n}^{*}=S_{n m}^{\dagger}$, we can derive the inverse relation:

\begin{equation*}
\sum_{i} S_{i n} v_{n}^{\prime}=\sum_{i} S_{i n} \sum_{m} S_{m n}^{*} v_{m}=\sum_{m} \underbrace{\sum_{i} S_{i n} S_{n m}^{\dagger}}_{\delta_{i m}} v_{m}=\sum_{m} \delta_{i m} v_{m}=v_{i} \tag{10.18}
\end{equation*}

Corollary 10.3. Unitary transformations preserve normalization.

Proof. Starting with $\sum_{i}\left|v_{i}\right|^{2}=\left(\Psi_{v}, \Psi_{v}\right)=1$, we can show:

\begin{align*}
\left(\Psi_{v}, \Psi_{v}\right) & =\langle v \mid v\rangle=\sum_{m} v_{m}^{*} v_{m}=\sum_{m}\left(\sum_{i} S_{m i} v_{i}^{\prime}\right)^{*}\left(\sum_{j} S_{m j} v_{j}^{\prime}\right)= \\
& =\sum_{m} \sum_{i} \sum_{j} S_{i m}^{\dagger} S_{m j} v_{i}^{\prime*} v_{j}^{\prime}=\sum_{i} \sum_{j} \underbrace{\left(\sum_{m} S_{i m}^{\dagger} S_{m j}\right)}_{\delta_{i j}} v_{i}^{\prime*} v_{j}^{\prime}=  \tag{10.19}\\
& =\sum_{i} \sum_{j} \delta_{i j} v_{i}^{\prime*} v_{j}^{\prime}=\sum_{i}\left|v_{i}^{\prime}\right|^{2}
\end{align*}

\subsection*{10.3 Practical implementations}
The Dirac formulation provides elegant representations of quantum dynamics. For the Schrödinger equation:

\begin{equation*}
i \hbar \partial_{t} \Psi(x, t)=\hat{H} \Psi(x, t) \tag{10.20}
\end{equation*}

We can express the state vector as:

\begin{equation*}
|\Psi\rangle=\sum_{m} c_{m}|m\rangle \tag{10.21}
\end{equation*}

Where $c_{m}=c_{m}(t)$ are time-dependent coefficients. This gives:

\[
\begin{array}{r}
i \hbar \partial_{t}|\Psi\rangle=\hat{H}|\Psi\rangle \\
i \hbar \partial_{t} \sum_{m} c_{m}|m\rangle=\hat{H} \sum_{m} c_{m}|m\rangle \tag{10.22}
\end{array}
\]

When the basis states are energy eigenstates with $\hat{H}|m\rangle=E_{m}|m\rangle$, the Hamiltonian becomes diagonal:

\begin{equation*}
H_{n m}=\langle n| \hat{H}|m\rangle=E_{m}\langle n \mid m\rangle=E_{m} \delta_{n m} \tag{10.23}
\end{equation*}

Resulting in the matrix representation:

\[
H=\left[\begin{array}{llll}
E_{1} & & &  \tag{10.24}\\
& E_{2} & & \\
& & \ddots & \\
& & & E_{n}
\end{array}\right]
\]

This leads to decoupled differential equations for each coefficient:

\begin{equation*}
i \hbar \partial_{t} c_{n}=E_{n} c_{n} \tag{10.25}
\end{equation*}

With solutions:

\begin{equation*}
c_{n}(t)=c_{n}(0) \exp \left(-i \frac{t}{\hbar} E_{n}\right) \tag{10.26}
\end{equation*}


And so:

\begin{equation*}
|\Psi\rangle=\sum_{m} c_{m}(0) \exp \left(-i \frac{t}{\hbar} E_{m}\right)|m\rangle \tag{10.27}
\end{equation*}

\subsection*{10.4 The Heisenberg representation}
In Heisenberg's alternative formulation, we focus on the time evolution of observables rather than states. The expectation value of an operator is:

\begin{equation*}
\langle\hat{A}\rangle_{t}=(\Psi(x, t), \hat{A} \Psi(x, t))=\langle\Psi| \tilde{A}|\Psi\rangle \tag{10.28}
\end{equation*}

Introducing the time evolution operator:

\begin{equation*}
\hat{U}_{t}=\exp \left(-i \frac{t}{\hbar} \hat{H}\right) \Longrightarrow \Psi(x, t)=\hat{U}_{t} \Psi(x, 0) \tag{10.29}
\end{equation*}

We can express the expectation value in terms of the initial state:

\begin{equation*}
(\Psi(x, t), \hat{A} \Psi(x, t))=\left(\hat{U}_{t} \Psi(x, 0), \hat{A} \hat{U}_{t} \Psi(x, 0)\right)=\left(\Psi(x, 0), \hat{U}_{t}^{\dagger} \hat{A} \hat{U}_{t} \Psi(x, 0)\right) \tag{10.30}
\end{equation*}

For notational clarity:

\begin{align*}
& \hat{A}=A \\
& \hat{U}_{t}=U  \tag{10.31}\\
& \Psi(x, 0)=\Psi_{0}
\end{align*}

\subsection*{10.5 Statistical description with density matrices}
To calculate expectation values, we expand the wavefunction in the energy eigenbasis:

\begin{equation*}
\Psi(x, t)=\sum_{r} c_{r} \phi_{r}(x) \tag{10.32}
\end{equation*}

Where $H \phi_{r}(x)=E_{r} \phi_{r}(x)$. The expectation value becomes:

\begin{equation*}
\langle A\rangle_{t}=\left(\sum_{r} c_{r} \phi_{r}, A \sum_{s} c_{s} \phi_{s}\right)=\sum_{r} \sum_{s} c_{r}^{*} c_{s}\left(\phi_{r}, A \phi_{s}\right) \tag{10.33}
\end{equation*}

With matrix elements $A_{rs} = (\phi_r, A\phi_s)$ and time-dependent coefficients:

\begin{equation*}
c_{r}(t)=c_{r}(0) \exp \left(-i \frac{t}{\hbar} E_{r}\right) \tag{10.34}
\end{equation*}

This yields:

\begin{align*}
\langle A\rangle_{t} & =\sum_{r} \sum_{s} c_{r}^{*}(0) \exp \left(i \frac{t}{\hbar} E_{r}\right) c_{s}(0) \exp \left(-i \frac{t}{\hbar} E_{s}\right) A_{r s}= \\
& =\sum_{r} \sum_{s} c_{r}^{*}(0) c_{s}(0) \exp \left(i \frac{t}{\hbar}\left(E_{r}-E_{s}\right)\right) A_{r s} \tag{10.35}
\end{align*}

For sufficiently large times, rapidly oscillating terms with $r \neq s$ average to zero:

\begin{equation*}
\langle A\rangle_{t} \simeq \sum_{s}\left|c_{s}(0)\right|^{2} A_{s s} \tag{10.36}
\end{equation*}

In statistical mechanics, the probability distribution follows Boltzmann statistics:

\begin{equation*}
\langle A\rangle_{t} \simeq \sum_{s} \frac{\mathrm{e}^{-\beta E_{s}}}{Z} A_{s s} \tag{10.37}
\end{equation*}

These expressions can be written compactly as $\operatorname{Tr}(\rho A)$, where $\rho$ is the density matrix.

\subsection*{10.6 Dynamics of observables}
For time-dependent operators in the Heisenberg picture, $A_t = U^\dagger A U$, the equation of motion is:

\begin{equation*}
\frac{\mathrm{d} A_{t}}{\mathrm{~d} t}=\frac{1}{i \hbar}\left[A_{t}, H\right]+\left(\partial_{t} A\right)_{t} \tag{10.38}
\end{equation*}


Proof. Starting from:

\begin{equation*}
\frac{\mathrm{d} A_{t}}{\mathrm{~d} t}=\frac{\mathrm{d}}{\mathrm{~d} t}\left(U^{\dagger} A U\right) \tag{10.39}
\end{equation*}

The time derivatives of the evolution operators are:

\begin{align*}
\frac{\mathrm{d} U}{\mathrm{~d} t} & =-\frac{i}{\hbar} H U \\
\frac{\mathrm{d} U^{\dagger}}{\mathrm{~d} t} & =\frac{i}{\hbar} U^{\dagger} H \tag{10.40}
\end{align*}

Applying the product rule:

\begin{align*}
\frac{\mathrm{d}}{\mathrm{~d} t}\left(U^{\dagger} A U\right) & =\frac{\mathrm{d}}{\mathrm{~d} t}\left(U^{\dagger}\right) A U+U^{\dagger} \frac{\mathrm{d}}{\mathrm{~d} t}(A) U+U^{\dagger} A \frac{\mathrm{~d}}{\mathrm{~d} t}(U)=  \tag{10.41}\\
& =\frac{i}{\hbar} U^{\dagger} H A U+U^{\dagger} \frac{\mathrm{~d}}{\mathrm{~d} t}(A) U-\frac{i}{\hbar} U^{\dagger} A H U
\end{align*}

Since $H$ commutes with $U$ (which can be shown using Taylor expansion), we have:

\begin{align*}
\frac{\mathrm{d}}{\mathrm{~d} t}\left(U^{\dagger} A U\right) & =\frac{i}{\hbar} \underbrace{\left(U^{\dagger} H A U-U^{\dagger} A H U\right)}_{\left[U^{\dagger}HU, U^{\dagger}AU\right]}+U^{\dagger} \frac{\mathrm{~d}}{\mathrm{~d} t}(A) U= \\
& =\frac{i}{\hbar} U^{\dagger}[H,A]U + U^{\dagger} \frac{\mathrm{d}}{\mathrm{~d} t}(A) U= \\
& =\frac{1}{i \hbar}\left[A_{t}, H\right]+\left(\partial_{t} A\right)_{t} \tag{10.42}
\end{align*}

This equation closely resembles Ehrenfest's theorem and provides the dynamics of operators in the Heisenberg picture.

\subsection*{10.7 Harmonic oscillator example}
For the harmonic oscillator, we define time-evolving operators:

\begin{align*}
& x_{t}=U^{\dagger} x U \\
& p_{t}=U^{\dagger} p U \tag{10.43}
\end{align*}

With the Hamiltonian:

\begin{equation*}
H\left(p_{t}, x_{t}\right)=\frac{p_{t}^{2}}{2 m}+\frac{1}{2} m \omega^{2} x_{t}^{2} \tag{10.44}
\end{equation*}

Substituting the definitions:

\begin{equation*}
H\left(p_{t}, x_{t}\right)=\frac{1}{2 m}\left(U^{\dagger} p U U^{\dagger} p U\right) + \frac{1}{2} m \omega^{2}\left(U^{\dagger} x U U^{\dagger} x U\right) \tag{10.45}
\end{equation*}

Using the unitary property $U U^{\dagger}=\mathbb{I}$:

\begin{equation*}
H\left(p_{t}, x_{t}\right)=U^{\dagger}\left(\frac{p^{2}}{2 m}+\frac{1}{2} m \omega^{2} x^{2}\right) U=U^{\dagger} H U=H \tag{10.46}
\end{equation*}

Thus, the Hamiltonian remains time-independent in the Heisenberg picture.

\subsection*{10.8 Important unitary transformations}
Several unitary transformations play key roles in quantum mechanics.

Spatial translation operator:

\begin{equation*}
T(\lambda)=\exp \left(i \frac{\lambda}{\hbar} p\right)=\exp \left(\lambda \frac{\partial}{\partial x}\right) \tag{10.47}
\end{equation*}

Applying this to a wavefunction yields:

\begin{equation*}
T(\lambda) \Psi(x)=\sum_{n} \frac{\lambda^{n}}{n!} \frac{\partial^{n} \Psi(x)}{\partial x^{n}}=\Psi(x+\lambda) \tag{10.48}
\end{equation*}

\section*{Rotational transformations}

The angular momentum operator for rotation around the z-axis:

\begin{equation*}
L_{3}=-i \hbar\left(x_{1} \frac{\partial}{\partial x_{2}}-x_{2} \frac{\partial}{\partial x_{1}}\right)=-i \hbar \frac{\partial}{\partial \phi} \tag{10.49}
\end{equation*}


The expression in spherical coordinates represents rotation around the z-axis:

\begin{equation*}
R(\alpha)=\exp \left(i \frac{\alpha}{\hbar} L_{3}\right)=\exp \left(\alpha \frac{\partial}{\partial \phi}\right) \tag{10.50}
\end{equation*}

This operator rotates functions by angle $\alpha$. For a wavefunction expressed in spherical coordinates:

\begin{equation*}
\Psi=\Psi(r \sin (\theta) \cos (\phi), r \sin \theta \sin (\phi), r \cos (\theta)) \tag{10.51}
\end{equation*}

The rotation transforms it to:

\begin{equation*}
R(\alpha) \Psi=\Psi(r \sin (\theta) \cos (\phi+\alpha), r \sin \theta \sin (\phi+\alpha), r \cos (\theta)) \tag{10.52}
\end{equation*}

Using trigonometric addition formulas:

\begin{align*}
& r \sin (\theta) \cos (\phi+\alpha)=r \sin (\theta)(\cos (\alpha) \cos (\phi)-\sin (\alpha) \sin (\phi))= \\
& =\underbrace{r \sin (\theta) \cos (\phi)}_{x_{1}} \cos (\alpha)-\underbrace{r \sin (\theta) \sin (\phi)}_{x_{2}} \sin (\alpha)  \tag{10.53}\\
& r \sin \theta \sin (\phi+\alpha)=r \sin (\theta)(\sin (\alpha) \cos (\phi)+\cos (\alpha) \sin (\phi))= \\
& =\underbrace{r \sin (\theta) \cos (\phi)}_{x_{1}} \sin (\alpha)+\underbrace{r \sin (\theta) \sin (\phi)}_{x_{2}} \cos (\alpha) \tag{10.54}
\end{align*}

This corresponds to the matrix representation:

\[
R(\alpha)=\left[\begin{array}{ccc}
\cos \alpha & -\sin \alpha & 0  \tag{10.55}\\
\sin \alpha & \cos \alpha & 0 \\
0 & 0 & 1
\end{array}\right]
\]

For time translation, the evolution operator is:

\begin{equation*}
U(t+\Delta t)=\exp \left(\frac{-i}{\hbar}(t+\Delta t) H\right) \tag{10.56}
\end{equation*}

Applying this to a wavefunction:

\begin{equation*}
U(t+\Delta t) \Psi(x, 0)=\exp \left(\frac{-i}{\hbar} \Delta t H\right) \exp \left(\frac{-i}{\hbar} t H\right) \Psi(x, 0)=\exp \left(\frac{-i}{\hbar} \Delta t H\right) \Psi(x, t)=\Psi(x, t+\Delta t) \tag{10.57}
\end{equation*}

For small time increments $\Delta t \ll t$, using first-order Taylor expansion:

\begin{equation*}
U(t+\Delta t) \Psi(x, 0) \sim\left(\mathbb{I}-i \Delta t \frac{H}{\hbar}+\ldots\right) \Psi(x, t)=\Psi(x, t+\Delta t) \tag{10.58}
\end{equation*}

This gives:

\begin{align*}
\Psi(x, t+\Delta t) & \sim U(\Delta t) \Psi(x, t)=\mathbb{I} \Psi(x, t)-\Delta t \frac{i}{\hbar} H \Psi(x, t)  \tag{10.59}\\
\Psi(x, t+\Delta t) & \sim \Psi(x, t)+\frac{\partial \Psi(x, t)}{\partial t} \Delta t \tag{10.60}
\end{align*}

Comparing these expressions:

\begin{align*}
\Psi(x, t)-\Delta t \frac{i}{\hbar} H \Psi(x, t) & =\Psi(x, t)+\frac{\partial \Psi(x, t)}{\partial t} \Delta t \\
-\frac{i}{\hbar} H \Psi(x, t) & =\frac{\partial \Psi(x, t)}{\partial t}  \tag{10.61}\\
H \Psi(x, t) & =i \hbar \frac{\partial \Psi(x, t)}{\partial t}
\end{align*}

Thus recovering the Schrödinger equation.

\section*{11 Quantum spin}
\subsection*{11.1 The Stern-Gerlach experiment}
The Stern-Gerlach experiment (1922) demonstrated that a beam of silver atoms, despite having zero orbital angular momentum, splits into two distinct beams when passing through a non-uniform magnetic field $\vec{B}$. Silver atoms have this property because, except for a single external electron, their electronic configuration has spherical symmetry with $\vec{L}_{\text{TOT}}=0$. The external electron is in the $5s$ state ($\ell=0$), also contributing zero orbital angular momentum.


\begin{equation*}
H=\frac{\vec{p}^{2}}{2 m}-\vec{\mu} \cdot \vec{B} \tag{11.1}
\end{equation*}

Where $\vec{\mu}=-\frac{e}{2 m_{e} c} \vec{L}$ represents the magnetic moment of a hydrogen-like atom with a single electron. Silver atoms closely match this model. The unexpected splitting observed in the experiment arises from the force due to the magnetic moment interacting with a non-uniform magnetic field:

\begin{equation*}
\vec{F}=-\vec{\nabla}(-\vec{\mu} \cdot \vec{B})=\mu_{3} \frac{\partial B_{3}}{\partial x_{3}} \hat{u}_{3} \tag{11.2}
\end{equation*}

According to conventional quantum theory, since the outer electron has $\ell=0$, its magnetic moment should be:

\begin{equation*}
\mu_{3}=-\frac{e}{2 m_{e} c} L_{3} \propto \hbar m=0 \tag{11.3}
\end{equation*}

With $m=0$ being the only possible eigenvalue for $L_3$ in s-orbital states. The observed splitting revealed that electrons possess an intrinsic angular momentum called spin $\vec{S}$, which contributes to the total magnetic moment. This spin has two possible states corresponding to eigenvalues $\pm\hbar/2$ of $S_3$, explaining the two-beam splitting pattern.

Numerous other experiments, including the anomalous Zeeman effect and helium spectroscopy, confirmed that spin is fundamental to atomic structure. This led to the spin-statistics theorem, which classifies elementary particles into fermions (half-integer spin particles like electrons, protons, and neutrons) and bosons (integer spin particles like photons or composite particles with even numbers of fermions).

\subsection*{11.2 Mathematical formulation of spin}
While spin operators $S$ and angular momentum operators $L$ share similar algebraic properties, they differ fundamentally in their definitions:

\begin{itemize}
  \item $L$ is expressed through differential operators
  \item $S$ is defined using matrices
\end{itemize}

The spin operators are constructed from the Pauli matrices:

\[
\sigma_{1}=\left[\begin{array}{ll}
0 & 1  \tag{11.4}\\
1 & 0
\end{array}\right], \quad \sigma_{2}=\left[\begin{array}{cc}
0 & -i \\
i & 0
\end{array}\right], \quad \sigma_{3}=\left[\begin{array}{cc}
1 & 0 \\
0 & -1
\end{array}\right]
\]

The spin operators are defined as:

\begin{align*}
S_{1} & =\frac{\hbar}{2} \sigma_{1} \\
S_{2} & =\frac{\hbar}{2} \sigma_{2}  \tag{11.5}\\
S_{3} & =\frac{\hbar}{2} \sigma_{3}
\end{align*}

The Pauli matrices have the property:

\begin{equation*}
\sigma_{1}^{2}=\mathbb{I}, \quad \sigma_{2}^{2}=\mathbb{I}, \quad \sigma_{3}^{2}=\mathbb{I} \tag{11.6}
\end{equation*}

We can define raising and lowering operators for spin:

\begin{align*}
& S_{+}=S_{1}+i S_{2}=\hbar\left[\begin{array}{ll}
0 & 1 \\
0 & 0
\end{array}\right] \\
& S_{-}=S_{1}-i S_{2}=\hbar\left[\begin{array}{ll}
0 & 0 \\
1 & 0
\end{array}\right] \tag{11.7}
\end{align*}

Examining their commutation relations:

\begin{align*}
S_1 S_2 - S_2 S_1 &= \frac{\hbar}{2}\left[\begin{array}{ll}
0 & 1 \\
1 & 0
\end{array}\right] \frac{\hbar}{2}\left[\begin{array}{cc}
0 & -i \\
i & 0
\end{array}\right] - \frac{\hbar}{2}\left[\begin{array}{cc}
0 & -i \\
i & 0
\end{array}\right] \frac{\hbar}{2}\left[\begin{array}{ll}
0 & 1 \\
1 & 0
\end{array}\right] \\
&= i\frac{\hbar^{2}}{4}\left[\begin{array}{cc}
1 & 0 \\
0 & -1
\end{array}\right] = i\hbar S_3 \tag{11.8}
\end{align*}


Indeed, the commutation relations for spin operators mirror those of angular momentum:

\begin{equation*}
\left[S_{1}, S_{2}\right]=i \hbar S_{3} \tag{11.10}
\end{equation*}

And by cyclic permutation:

\begin{align*}
& {\left[S_{2}, S_{3}\right]=i \hbar S_{1}} \\
& {\left[S_{3}, S_{1}\right]=i \hbar S_{2}} \\
& {\left[S_{+}, S_{-}\right]=2 \hbar S_{3}}  \tag{11.11}\\
& {\left[S_{3}, S_{\pm}\right]=\pm \hbar S_{\pm}}
\end{align*}

The total spin squared operator is:
\begin{equation*}
S^{2}=S_{1}^{2}+S_{2}^{2}+S_{3}^{2}=\frac{3}{4} \hbar^{2} \mathbb{I}
\end{equation*}

The standard basis consists of eigenstates of $S_3$, satisfying:

\begin{equation*}
S_{3}\left|s, m_{s}\right\rangle=\hbar m_{s}\left|s, m_{s}\right\rangle \quad \text{with} \quad m_{s}=\pm\frac{1}{2} \tag{11.12}
\end{equation*}

These eigenstates have the explicit matrix representation:

\begin{align*}
& \left|s,+\frac{1}{2}\right\rangle=\left[\begin{array}{l}
1 \\
0
\end{array}\right]  \tag{11.13}\\
& \left|s,-\frac{1}{2}\right\rangle=\left[\begin{array}{l}
0 \\
1
\end{array}\right]
\end{align*}

The eigenvalue equation for $S^2$ is:

\begin{equation*}
S^{2}\left|s, m_{s}\right\rangle=\hbar^{2} s(s+1)\left|s, m_{s}\right\rangle \tag{11.14}
\end{equation*}

From our matrix representation, we know:

\begin{equation*}
S^{2}\left|s, m_{s}\right\rangle=\frac{3}{4} \hbar^{2} \mathbb{I}\left|s, m_{s}\right\rangle=\frac{3}{4} \hbar^{2}\left|s, m_{s}\right\rangle \tag{11.15}
\end{equation*}

This gives us the value of $s$:

\begin{equation*}
s(s+1)=\frac{3}{4} \Longrightarrow s=\frac{1}{2} \tag{11.16}
\end{equation*}

We select the positive solution by analogy with angular momentum. Since $s$ has only one value, we can simplify the notation:

\begin{equation*}
|s, \pm 1/2\rangle=|\pm 1/2\rangle \tag{11.17}
\end{equation*}

The action of the raising and lowering operators on these basis states is:

\[
S_{+}|-1/2\rangle=\hbar\left[\begin{array}{ll}
0 & 1  \tag{11.18}\\
0 & 0
\end{array}\right]\left[\begin{array}{l}
0 \\
1
\end{array}\right]=\hbar\left[\begin{array}{l}
1 \\
0
\end{array}\right]=\hbar|+1/2\rangle
\]

And similarly:

\[
S_{-}|+1/2\rangle=\hbar\left[\begin{array}{ll}
0 & 0  \tag{11.19}\\
1 & 0
\end{array}\right]\left[\begin{array}{l}
1 \\
0
\end{array}\right]=\hbar\left[\begin{array}{l}
0 \\
1
\end{array}\right]=\hbar|-1/2\rangle
\]


The remaining operations yield zero results:
\begin{align*}
S_+|+1/2\rangle = 0\\
S_-|-1/2\rangle = 0
\end{align*}

When dealing with spin-dependent Hamiltonians, we work with spinor wavefunctions:

\[
|\Psi\rangle=\sum_{m_{s}= \pm 1/2} \psi_{m_{s}}\left|m_{s}\right\rangle=\psi_{+}|+1/2\rangle+\psi_{-}|-1/2\rangle=\left[\begin{array}{l}
\psi_{+}  \tag{11.20}\\
\psi_{-}
\end{array}\right]
\]

For a Hamiltonian of the form:

\begin{equation*}
\mathcal{H}=\mathcal{H}_{0}+\vec{\alpha} \cdot(\vec{L}+g \vec{S}) \tag{11.21}
\end{equation*}

The corresponding quantum mechanical operator is:

\begin{equation*}
H=H_{0}+\vec{\alpha} \cdot(\vec{L}+g \vec{S}) \tag{11.22}
\end{equation*}

Where:

\[
\vec{L}=\left[\begin{array}{l}
L_{1}  \tag{11.23}\\
L_{2} \\
L_{3}
\end{array}\right], \quad \vec{S}=\left[\begin{array}{l}
S_{1} \\
S_{2} \\
S_{3}
\end{array}\right]
\]

Here $\vec{L}$ and $\vec{S}$ represent "vectors of operators" - a notation for operations like scalar products where each component operator acts on the corresponding component of another vector.

Applying this Hamiltonian to a spinor state:

\begin{align*}
& \left(H_{0}+\vec{\alpha} \cdot(\vec{L}+g \vec{S})\right)\left[\begin{array}{l}
\psi_{+} \\
\psi_{-}
\end{array}\right]= \\
& =\left[\begin{array}{l}
H_{0} \psi_{+} \\
H_{0} \psi_{-}
\end{array}\right]+\left[\begin{array}{l}
\vec{\alpha} \cdot \vec{L} \psi_{+} \\
\vec{\alpha} \cdot \vec{L} \psi_{-}
\end{array}\right]+\frac{g \hbar}{2}\left[\begin{array}{cc}
\alpha_{3} & \alpha_{1}-i \alpha_{2} \\
\alpha_{1}+i \alpha_{2} & -\alpha_{3}
\end{array}\right]\left[\begin{array}{l}
\psi_{+} \\
\psi_{-}
\end{array}\right]=  \tag{11.24}\\
& =\left[\begin{array}{l}
H_{0} \psi_{+} \\
H_{0} \psi_{-}
\end{array}\right]+\left[\begin{array}{l}
\vec{\alpha} \cdot \vec{L} \psi_{+} \\
\vec{\alpha} \cdot \vec{L} \psi_{-}
\end{array}\right]+\frac{g \hbar}{2}\left[\begin{array}{l}
\alpha_{3} \psi_{+}+\left(\alpha_{1}-i \alpha_{2}\right) \psi_{-} \\
\left(\alpha_{1}+i \alpha_{2}\right) \psi_{+}-\alpha_{3} \psi_{-}
\end{array}\right]
\end{align*}

This leads to the coupled Schrödinger equations for the spinor components:

\[
-i \hbar \frac{\partial}{\partial t}\left[\begin{array}{l}
\psi_{+}  \tag{11.25}\\
\psi_{-}
\end{array}\right]=\left[\begin{array}{l}
\left(H_{0}+\vec{\alpha} \cdot \vec{L}+\frac{g\hbar}{2}\alpha_{3}\right) \psi_{+}+\frac{g\hbar}{2}\left(\alpha_{1}-i \alpha_{2}\right) \psi_{-} \\
\left(H_{0}+\vec{\alpha} \cdot \vec{L}-\frac{g\hbar}{2}\alpha_{3}\right) \psi_{-}+\frac{g\hbar}{2}\left(\alpha_{1}+i \alpha_{2}\right) \psi_{+}
\end{array}\right]
\]


These equations form a coupled system of differential equations.

Setting $\vec{\alpha}=e \vec{B} / 2 m_{e} c$, we obtain the Pauli equation, which represents the non-relativistic limit of the Dirac equation:

\[
-i \hbar \frac{\partial}{\partial t}\left[\begin{array}{l}
\psi_{+}  \tag{11.26}\\
\psi_{-}
\end{array}\right]=\left(-\frac{\hbar^{2}}{2 m_{e}} \nabla^{2}+V(\vec{x})+\frac{e}{2 m_{e} c} \vec{B} \cdot(\vec{L}+g \vec{S})\right)\left[\begin{array}{l}
\psi_{+} \\
\psi_{-}
\end{array}\right]
\]

\subsection*{11.3 Spin-1/2 free particle in a magnetic field}
For a point-like particle with spin 1/2 and mass $m$ in a magnetic field, the Hamiltonian simplifies to:

\begin{equation*}
H=H_{k}+H_{s}=\frac{p^{2}}{2 m}+k \vec{B} \cdot \vec{S} \tag{11.27}
\end{equation*}

Using spherical coordinates for the magnetic field:

\begin{equation*}
B_{1}=B \sin \theta \cos \phi, \quad B_{2}=B \sin \theta \sin \phi, \quad B_{3}=B \cos \theta \tag{11.28}
\end{equation*}

The spin-dependent term becomes:

\begin{align*}
k \vec{B} \cdot \vec{S} &= k \frac{\hbar}{2}\left[B_{1}\left(\begin{array}{ll}
0 & 1 \\
1 & 0
\end{array}\right)+B_{2}\left(\begin{array}{cc}
0 & -i \\
i & 0
\end{array}\right)+B_{3}\left(\begin{array}{cc}
1 & 0 \\
0 & -1
\end{array}\right)\right] \\
&= \frac{\hbar}{2} k B\left[\begin{array}{cc}
\cos \theta & e^{-i \phi} \sin \theta \\
e^{i \phi} \sin \theta & -\cos \theta
\end{array}\right] \tag{11.29}
\end{align*}

This model describes a neutral particle (like a neutron) in a time-dependent magnetic field $\vec{B}$ where $\theta=\theta(t)$ and $\phi=\phi(t)$. Such systems exhibit the Berry phase - an additional phase in the wavefunction arising from the time-dependence of $\vec{B}$.

The energy eigenvalue problem is:

\[
\left[\frac{p^{2}}{2 m}+k \vec{B} \cdot \vec{S}\right]\left[\begin{array}{l}
\psi_{+}  \tag{11.30}\\
\psi_{-}
\end{array}\right]=E\left[\begin{array}{l}
\psi_{+} \\
\psi_{-}
\end{array}\right]
\]

The spinor wavefunction can be factorized into a plane wave and a spin vector:

\[
|\psi\rangle=\left[\begin{array}{l}
\psi_{+}  \tag{11.31}\\
\psi_{-}
\end{array}\right]=e^{i \vec{k} \cdot \vec{x}}|v\rangle=e^{i \vec{k} \cdot \vec{x}}\left[\begin{array}{l}
v_{+} \\
v_{-}
\end{array}\right], \quad \psi_{\pm}=v_{\pm} e^{i \vec{k} \cdot \vec{x}}
\]

Since the kinetic energy operator commutes with the spin operators, the eigenvectors of $k \vec{B} \cdot \vec{S}$ are:

\[
|v(+1)\rangle=\left[\begin{array}{c}
\cos (\theta/2) e^{-i \phi/2}  \tag{11.32}\\
\sin (\theta/2) e^{i \phi/2}
\end{array}\right], \quad|v(-1)\rangle=\left[\begin{array}{c}
-\sin (\theta/2) e^{-i \phi/2} \\
\cos (\theta/2) e^{i \phi/2}
\end{array}\right]
\]

These satisfy the eigenvalue equation:

\[
\frac{\hbar}{2} k B\left[\begin{array}{cc}
\cos \theta & e^{-i \phi} \sin \theta  \tag{11.33}\\
e^{i \phi} \sin \theta & -\cos \theta
\end{array}\right]|v(\mu)\rangle=\mu \frac{\hbar}{2} k B|v(\mu)\rangle
\]

where $\mu = \pm 1$ corresponds to spin aligned or anti-aligned with the magnetic field.


From these eigenvectors, we can determine the energy eigenvalues:

\begin{equation*}
\left[\frac{p^{2}}{2 m}+k \vec{B} \cdot \vec{S}\right] e^{i \vec{k} \cdot \vec{x}}|v(\mu)\rangle=|v(\mu)\rangle \frac{p^{2}}{2 m} e^{i \vec{k} \cdot \vec{x}}+e^{i \vec{k} \cdot \vec{x}} k \vec{B} \cdot \vec{S}|v(\mu)\rangle=\left(\frac{\hbar^{2} k^{2}}{2 m}+\mu \frac{\hbar}{2} k B\right) e^{i \vec{k} \cdot \vec{x}}|v(\mu)\rangle \tag{11.34}
\end{equation*}

Thus, the energy spectrum is:

\begin{equation*}
E(\vec{k}, \mu)=\frac{\hbar^{2} k^{2}}{2 m}+\mu \frac{\hbar}{2} k B \tag{11.35}
\end{equation*}

This shows how the energy splits into two levels due to the magnetic field, with the separation proportional to the field strength.

\subsection*{11.4 Addition of angular momenta}
When two angular momenta interact in a quantum system, we need to understand how to combine them. Classically, we can add vectors $\vec{J}=\vec{L}+\vec{S}$ or $\vec{J}=\vec{S}_1+\vec{S}_2$. In quantum mechanics, this corresponds to the operator $\vec{J}=\vec{L}+\vec{S}$, which satisfies:

\begin{align*}
\left[J_{n}, J_{m}\right] &= \left[L_{n}+S_{n}, L_{m}+S_{m}\right]\\
&= \left[L_{n}, L_{m}\right]+\left[L_{n}, S_{m}\right]+\left[S_{n}, L_{m}\right]+\left[S_{n}, S_{m}\right]\\
&= i \hbar \epsilon_{n m k} L_{k}+i \hbar \epsilon_{n m k} S_{k}=i \hbar \epsilon_{n m k} J_{k} \tag{11.36}
\end{align*}

This shows that $\vec{J}$ satisfies the same commutation relations as $\vec{L}$ and $\vec{S}$, confirming that $\left[J^{2}, J_{3}\right]=0$. Therefore, the quantum numbers $j$ and $m_j$ provide a good basis.

Classically, the magnitudes of these vectors are:

\[
\begin{array}{r}
|\vec{L}|=\hbar \ell \\
|\vec{S}|=\hbar s  \tag{11.37}\\
\hbar(\ell-s) \leq|\vec{J}| \leq \hbar(\ell+s)
\end{array}
\]

In quantum mechanics, the operators have eigenvalues $\hbar^{2}\ell(\ell+1)$ and $\hbar^{2}s(s+1)$, giving "quantum lengths":

\begin{align*}
|L| &= \sqrt{\hbar^{2} \ell(\ell+1)} \sim \hbar \ell  \tag{11.38}\\
|S| &= \sqrt{\hbar^{2} s(s+1)} \sim \hbar s
\end{align*}

The quantum number $j$ must satisfy:

\begin{equation*}
-j \leq m_{j} \leq j, \quad \ell-s \leq j \leq \ell+s \tag{11.39}
\end{equation*}

Examining the commutators between $J^2$ and the individual components $L_3$ or $S_3$:

\begin{align*}
\left[J^{2}, S_{3}\right] &= \left[(L+S)^{2}, S_{3}\right]\\
&= \left[L^{2}+S^{2}+2 \vec{L} \cdot \vec{S}, S_{3}\right]\\
&= \left[L^{2}, S_{3}\right]+\left[S^{2}, S_{3}\right]+2\left[\vec{L} \cdot \vec{S}, S_{3}\right]\\
&= 2 \sum_{i}\left[L_{i} S_{i}, S_{3}\right]\\
&= 2 \sum_{i}\left(L_{i}\left[S_{i}, S_{3}\right]+\left[L_{i}, S_{3}\right] S_{i}\right)\\
&= 2 \sum_{i} L_{i}\left[S_{i}, S_{3}\right] \neq 0 \tag{11.40}
\end{align*}

Similarly:

\begin{align*}
\left[J^{2}, L_{3}\right] &= \left[(L+S)^{2}, L_{3}\right]\\
&= 2 \sum_{i}\left[L_{i}, L_{3}\right] S_{i} \neq 0 \tag{11.41}
\end{align*}

However, the sum of these commutators is zero:

\begin{align*}
&2 \sum_{i}\left[L_{i}, L_{3}\right] S_{i}+2 \sum_{i} L_{i}\left[S_{i}, S_{3}\right]\\
&= 2\left(\left[L_{1}, L_{3}\right] S_{1}+\left[L_{2}, L_{3}\right] S_{2}+L_{1}\left[S_{1}, S_{3}\right]+L_{2}\left[S_{2}, S_{3}\right]\right)\\
&= 2\left(-i \hbar L_{2} S_{1}+i \hbar L_{1} S_{2}-i \hbar L_{1} S_{2}+i \hbar L_{2} S_{1}\right)=0 \tag{11.42}
\end{align*}

This confirms that $\left[J^{2}, J_{k}\right]=0$. Other useful commutators include:

\begin{align*}
&\left[J^{2}, L^{2}\right]=0 \\
&\left[L^{2}, J_{3}\right]=0  \tag{11.43}\\
&\left[J^{2}, S^{2}\right]=0 \\
&\left[S^{2}, J_{3}\right]=0
\end{align*}

These commutation relations allow us to define a complete set of commuting observables:

\begin{equation*}
S^{2}, L^{2}, J^{2}, J_{3} \quad \Longleftrightarrow \quad s, \ell, j, m_{j} \tag{11.44}
\end{equation*}

The states with $m_j = m + m_s$ can be formed from different combinations of $m$ and $m_s$ that sum to the same value. This degeneracy means that states in the coupled basis $|j, m_j, \ell, s\rangle$ are superpositions of the uncoupled basis states:

\begin{equation*}
\left|j, m_{j}, \ell, s\right\rangle=\sum_{m, m_{s}}^{*} C\left(m, m_{s}\right)|\ell, m\rangle\left|s, m_{s}\right\rangle \tag{11.45}
\end{equation*}


The symbol * implies that we keep $m+m_{s}$ constant. This state satisfies:

\begin{align*}
L^{2}\left|j, m_{j}, \ell, s\right\rangle &= \sum_{m, m_{s}}^{*} C\left(m, m_{s}\right) L^{2}|\ell, m\rangle\left|s, m_{s}\right\rangle \\
&= \sum_{m, m_{s}}^{*} C\left(m, m_{s}\right)\left(\hbar^{2} \ell(\ell+1)\right)|\ell, m\rangle\left|s, m_{s}\right\rangle \\
&= \hbar^{2} \ell(\ell+1) \sum_{m, m_{s}}^{*} C\left(m, m_{s}\right)|\ell, m\rangle\left|s, m_{s}\right\rangle=\hbar^{2} \ell(\ell+1)\left|j, m_{j}, \ell, s\right\rangle  \tag{11.46}
\end{align*}

Similarly for the spin operator:

\begin{align*}
S^{2}\left|j, m_{j}, \ell, s\right\rangle &= \sum_{m, m_{s}}^{*} C\left(m, m_{s}\right)|\ell, m\rangle S^{2}\left|s, m_{s}\right\rangle \\
&= \sum_{m, m_{s}}^{*} C\left(m, m_{s}\right)\left(\hbar^{2} s(s+1)\right)|\ell, m\rangle\left|s, m_{s}\right\rangle \\
&= \hbar^{2} s(s+1) \sum_{m, m_{s}}^{*} C\left(m, m_{s}\right)|\ell, m\rangle\left|s, m_{s}\right\rangle=\hbar^{2} s(s+1)\left|j, m_{j}, \ell, s\right\rangle \tag{11.47}
\end{align*}

For the z-component of total angular momentum:

\begin{align*}
J_{3}\left|j, m_{j}, \ell, s\right\rangle &= \sum_{m, m_{s}}^{*} C\left(m, m_{s}\right)\left(L_{3}|\ell, m\rangle\left|s, m_{s}\right\rangle+|\ell, m\rangle S_{3}\left|s, m_{s}\right\rangle\right) \\
&= \sum_{m, m_{s}}^{*} C\left(m, m_{s}\right) \hbar\left(m+m_{s}\right)|\ell, m\rangle\left|s, m_{s}\right\rangle \\
&= \hbar m_{j} \sum_{m, m_{s}}^{*} C\left(m, m_{s}\right)|\ell, m\rangle\left|s, m_{s}\right\rangle=\hbar m_{j}\left|j, m_{j}, \ell, s\right\rangle \tag{11.48}
\end{align*}

The coupled basis states for $j=\ell\pm\frac{1}{2}$ can be expressed as:

\[
\left\{\begin{array}{l}
\left|\ell-\frac{1}{2}, m_{j}, \ell, s\right\rangle=\alpha_{-}\left|\ell, m_{j}-\frac{1}{2}\right\rangle\left|s,+\frac{1}{2}\right\rangle+\beta_{-}\left|\ell, m_{j}+\frac{1}{2}\right\rangle\left|s,-\frac{1}{2}\right\rangle  \tag{11.49}\\
\left|\ell+\frac{1}{2}, m_{j}, \ell, s\right\rangle=\alpha_{+}\left|\ell, m_{j}-\frac{1}{2}\right\rangle\left|s,+\frac{1}{2}\right\rangle+\beta_{+}\left|\ell, m_{j}+\frac{1}{2}\right\rangle\left|s,-\frac{1}{2}\right\rangle
\end{array}\right.
\]

The coefficients are given by:

\begin{equation*}
\alpha_{\pm}= \pm \sqrt{\frac{\ell \pm m_{j}+\frac{1}{2}}{2\ell+1}}= \pm \beta_{\mp} \tag{11.50}
\end{equation*}

These coefficients satisfy the normalization condition $\left|\alpha_{\pm}\right|^{2}+\left|\beta_{\pm}\right|^{2}=1$. For instance:

\begin{align*}
\alpha_{+} &= \sqrt{\frac{\ell+m_{j}+\frac{1}{2}}{2\ell+1}}  \tag{11.51}\\
\beta_{+} &= -\sqrt{\frac{\ell-m_{j}+\frac{1}{2}}{2\ell+1}}
\end{align*}

Verifying normalization:

\begin{equation*}
\left|\alpha_{+}\right|^{2}+\left|\beta_{+}\right|^{2}=\frac{\ell+m_{j}+\frac{1}{2}}{2\ell+1}+\frac{\ell-m_{j}+\frac{1}{2}}{2\ell+1}=\frac{2\ell+1}{2\ell+1}=1 \tag{11.52}
\end{equation*}

\subsection*{11.5 Coupling arbitrary angular momenta}
For two general angular momentum operators $J_{a}$ and $J_{b}$ with eigenstates:

\begin{align*}
J_{a3}\left|j_{a}, m_{a}\right\rangle &= \hbar m_{a}\left|j_{a}, m_{a}\right\rangle \\
J_{b3}\left|j_{b}, m_{b}\right\rangle &= \hbar m_{b}\left|j_{b}, m_{b}\right\rangle \\
J_{a}^{2}\left|j_{a}, m_{a}\right\rangle &= \hbar^2 j_{a}\left(j_{a}+1\right)\left|j_{a}, m_{a}\right\rangle  \tag{11.53}\\
J_{b}^{2}\left|j_{b}, m_{b}\right\rangle &= \hbar^{2} j_{b}\left(j_{b}+1\right)\left|j_{b}, m_{b}\right\rangle
\end{align*}

The coupled states are formed as:

\begin{equation*}
\left|j, m_{j}, j_{a}, j_{b}\right\rangle=\sum_{m_{a}, m_{b}}^{*} C\left(m_{a}, m_{b}\right)\left|j_{a}, m_{a}\right\rangle\left|j_{b}, m_{b}\right\rangle \tag{11.54}
\end{equation*}

These states diagonalize the total angular momentum operators:

\begin{align*}
J_{3}\left|j, m_{j}, j_{a}, j_{b}\right\rangle &= \hbar m_{j}\left|j, m_{j}, j_{a}, j_{b}\right\rangle \\
J^{2}\left|j, m_{j}, j_{a}, j_{b}\right\rangle &= \hbar^2 j(j+1)\left|j, m_{j}, j_{a}, j_{b}\right\rangle \tag{11.55}
\end{align*}

\subsection*{11.6 Spin coupling illustration}
As an illustrative example, consider coupling two spin-$\frac{1}{2}$ systems $S_{a}$ and $S_{b}$. Here $j_{a}=j_{b}=\frac{1}{2}$, so the possible values of total angular momentum $j$ are:

\begin{equation*}
j=0,1 \tag{11.56}
\end{equation*}

And the corresponding $m_j$ values are:

\begin{equation*}
m_{j}=-1,0,1 \tag{11.57}
\end{equation*}


\subsection*{11.6.1 Analysis of coupled spin states}
For the case $j=0$, we have $m_j=0$. This state can be formed from two combinations: $m_a=\frac{1}{2}, m_b=-\frac{1}{2}$ or $m_a=-\frac{1}{2}, m_b=\frac{1}{2}$. This configuration yields the singlet state:

\begin{equation*}
|0,0,\frac{1}{2},\frac{1}{2}\rangle=\frac{1}{\sqrt{2}}(|\frac{1}{2},-\frac{1}{2}\rangle|\frac{1}{2},+\frac{1}{2}\rangle+\sigma|\frac{1}{2},+\frac{1}{2}\rangle|\frac{1}{2},-\frac{1}{2}\rangle) \tag{11.58}
\end{equation*}

While both values $\sigma=\pm 1$ satisfy normalization, we must determine which gives the correct eigenvalue for $J^2$. We require $J^2|0,0,\frac{1}{2},\frac{1}{2}\rangle=0$. First, we express $J^2$ in terms of ladder operators:

\begin{equation*}
J^{2}=J_{3}^{2}+J_{1}^{2}+J_{2}^{2}=J_{3}^{2}+\frac{1}{2}(J_{+}J_{-}+J_{-}J_{+})=J_{3}^{2}+\frac{1}{2}(2J_{+}J_{-}-2\hbar J_{3}) \tag{11.59}
\end{equation*}

Since $J_3|0,0,\frac{1}{2},\frac{1}{2}\rangle=0$, the terms involving $J_3$ vanish, leaving:

\begin{equation*}
J^{2}=J_{+}J_{-}=(S_{a+}+S_{b+})(S_{a-}+S_{b-}) \tag{11.60}
\end{equation*}

Applying this to our state:

\begin{align*}
J^{2}|0,0,\frac{1}{2},\frac{1}{2}\rangle &= (S_{a+}+S_{b+})(S_{a-}+S_{b-}) \frac{1}{\sqrt{2}}(|-\frac{1}{2}\rangle|+\frac{1}{2}\rangle+\sigma|+\frac{1}{2}\rangle|-\frac{1}{2}\rangle) \\
&= (S_{a+}+S_{b+}) \frac{1}{\sqrt{2}}(S_{a-}|-\frac{1}{2}\rangle|+\frac{1}{2}\rangle+\sigma S_{a-}|+\frac{1}{2}\rangle|-\frac{1}{2}\rangle+|-\frac{1}{2}\rangle S_{b-}|+\frac{1}{2}\rangle+\sigma|+\frac{1}{2}\rangle S_{b-}|-\frac{1}{2}\rangle) \\
&= (S_{a+}+S_{b+}) \frac{\sigma+1}{\sqrt{2}}|-\frac{1}{2}\rangle|-\frac{1}{2}\rangle \\
&= \frac{\sigma+1}{\sqrt{2}}(S_{a+}|-\frac{1}{2}\rangle|-\frac{1}{2}\rangle+|-\frac{1}{2}\rangle S_{b+}|-\frac{1}{2}\rangle) \\
&= \frac{\sigma+1}{\sqrt{2}}(|+\frac{1}{2}\rangle|-\frac{1}{2}\rangle+|-\frac{1}{2}\rangle|+\frac{1}{2}\rangle) \tag{11.61}
\end{align*}

For this expression to equal zero, we must have $\sigma=-1$. Therefore, the singlet state is:

\begin{equation*}
|0,0,\frac{1}{2},\frac{1}{2}\rangle=\frac{1}{\sqrt{2}}(|\frac{1}{2},-\frac{1}{2}\rangle|\frac{1}{2},+\frac{1}{2}\rangle-|\frac{1}{2},+\frac{1}{2}\rangle|\frac{1}{2},-\frac{1}{2}\rangle) \tag{11.62}
\end{equation*}

This state is antisymmetric under particle exchange:

\begin{equation*}
|0,0,j_a,j_b\rangle=-|0,0,j_b,j_a\rangle \tag{11.63}
\end{equation*}

For the case $j=1$, we have three possible values: $m_j=-1,0,1$. For $m_j=\pm 1$, there is only one possible configuration:

\[
\left\{\begin{array}{l}
|1,1,j_a,j_b\rangle=|j_a,\frac{1}{2}\rangle|j_b,\frac{1}{2}\rangle=|1,1\rangle  \tag{11.64}\\
|1,-1,j_a,j_b\rangle=|j_a,-\frac{1}{2}\rangle|j_b,-\frac{1}{2}\rangle=|1,-1\rangle
\end{array}\right.
\]

For $m_j=0$, we again have two possibilities: $m_a=\frac{1}{2}, m_b=-\frac{1}{2}$ or $m_a=-\frac{1}{2}, m_b=\frac{1}{2}$. The state has the form:

\begin{equation*}
|1,0,\frac{1}{2},\frac{1}{2}\rangle=\frac{1}{\sqrt{2}}(|j_a,+\frac{1}{2}\rangle|j_b,-\frac{1}{2}\rangle+\nu|j_a,-\frac{1}{2}\rangle|j_b,+\frac{1}{2}\rangle) \tag{11.65}
\end{equation*}

To determine $\nu$, we can generate this state by applying $J_+$ to the $m_j=-1$ state:

\begin{align*}
\sqrt{2}|1,0,+\frac{1}{2},+\frac{1}{2}\rangle &= J_{+}|1,-1,+\frac{1}{2},+\frac{1}{2}\rangle=(S_{a+}+S_{b+})|j_a,-\frac{1}{2}\rangle|j_b,-\frac{1}{2}\rangle \\
&= S_{a+}|j_a,-\frac{1}{2}\rangle|j_b,-\frac{1}{2}\rangle+|j_a,-\frac{1}{2}\rangle S_{b+}|j_b,-\frac{1}{2}\rangle \\
&= |j_a,+\frac{1}{2}\rangle|j_b,-\frac{1}{2}\rangle+|j_a,-\frac{1}{2}\rangle|j_b,+\frac{1}{2}\rangle \tag{11.66}
\end{align*}

This calculation shows that $\nu=1$. We can verify this by applying $J_+$ to the $m_j=0$ state:

\begin{align*}
\sqrt{2}|1,1,+\frac{1}{2},+\frac{1}{2}\rangle &= J_{+}|1,0,+\frac{1}{2},+\frac{1}{2}\rangle=(S_{a+}+S_{b+})\frac{1}{\sqrt{2}}(|j_a,+\frac{1}{2}\rangle|j_b,-\frac{1}{2}\rangle+|j_a,-\frac{1}{2}\rangle|j_b,+\frac{1}{2}\rangle) \\
&= \frac{1}{\sqrt{2}}(S_{a+}|j_a,+\frac{1}{2}\rangle|j_b,-\frac{1}{2}\rangle+S_{a+}|j_a,-\frac{1}{2}\rangle|j_b,+\frac{1}{2}\rangle \\
&\quad +|j_a,+\frac{1}{2}\rangle S_{b+}|j_b,-\frac{1}{2}\rangle+|j_a,-\frac{1}{2}\rangle S_{b+}|j_b,+\frac{1}{2}\rangle) \\
&= \frac{1}{\sqrt{2}}(|j_a,+\frac{1}{2}\rangle|j_b,+\frac{1}{2}\rangle+|j_a,+\frac{1}{2}\rangle|j_b,+\frac{1}{2}\rangle) \\
&= \sqrt{2}|j_a,+\frac{1}{2}\rangle|j_b,+\frac{1}{2}\rangle \tag{11.67}
\end{align*}

This confirms our choice of $\nu=1$ and reproduces the $m_j=1$ state correctly.

The complete set of triplet states is:

\[
\left\{\begin{array}{l}
|1,1,j_a,j_b\rangle=|j_a,\frac{1}{2}\rangle|j_b,\frac{1}{2}\rangle=|1,1\rangle  \tag{11.68}\\
|1,0,j_a,j_b\rangle=\frac{1}{\sqrt{2}}(|j_a,+\frac{1}{2}\rangle|j_b,-\frac{1}{2}\rangle+|j_a,-\frac{1}{2}\rangle|j_b,+\frac{1}{2}\rangle) \\
|1,-1,j_a,j_b\rangle=|j_a,-\frac{1}{2}\rangle|j_b,-\frac{1}{2}\rangle=|1,-1\rangle
\end{array}\right.
\]


\subsection*{11.7 Matrix formulation for angular momentum operators}
The spectrum-generating algebra approach naturally leads to matrix representations of angular momentum operators. The dimension of these matrices depends on the representation index $j$, requiring $(2j+1)\times(2j+1)$ matrices. For spin-$\frac{1}{2}$ systems, this yields $2\times2$ matrices (the familiar Pauli matrices).

For the case $j=1$, we need $3\times3$ matrices to represent the operators, corresponding to the three basis states with $m_j=-1,0,1$. These basis vectors can be written as:

\[
|1,+1\rangle=\left[\begin{array}{l}
1  \tag{11.69}\\
0 \\
0
\end{array}\right], \quad|1,0\rangle=\left[\begin{array}{l}
0 \\
1 \\
0
\end{array}\right], \quad|1,-1\rangle=\left[\begin{array}{l}
0 \\
0 \\
1
\end{array}\right]
\]

The matrix representation of $J_3$ in this basis is diagonal:

\[
J_{3}=\hbar\left[\begin{array}{ccc}
1 & 0 & 0  \tag{11.70}\\
0 & 0 & 0 \\
0 & 0 & -1
\end{array}\right]
\]

We can verify this acts correctly on the basis states, for example:

\[
J_{3}|1,-1\rangle=\hbar\left[\begin{array}{ccc}
1 & 0 & 0  \tag{11.71}\\
0 & 0 & 0 \\
0 & 0 & -1
\end{array}\right]\left[\begin{array}{l}
0 \\
0 \\
1
\end{array}\right]=-\hbar\left[\begin{array}{l}
0 \\
0 \\
1
\end{array}\right]
\]

To construct the ladder operators $J_+$ and $J_-$, we use the general formulas:

\begin{align*}
J_{+}|j,m_j\rangle &= \hbar\sqrt{j(j+1)-m_j(m_j+1)}|j,m_j+1\rangle  \tag{11.72}\\
J_{-}|j,m_j\rangle &= \hbar\sqrt{j(j+1)-m_j(m_j-1)}|j,m_j-1\rangle
\end{align*}

Applying these to our $j=1$ basis states:

\begin{align*}
J_{+}|1,+1\rangle &= \hbar\sqrt{1(1+1)-1(1+1)}|1,+2\rangle=0 \\
J_{+}|1,0\rangle &= \hbar\sqrt{1(1+1)-0(0+1)}|1,+1\rangle=\hbar\sqrt{2}|1,+1\rangle \\
J_{+}|1,-1\rangle &= \hbar\sqrt{1(1+1)-(-1)(0)}|1,0\rangle=\hbar\sqrt{2}|1,0\rangle \\
J_{-}|1,+1\rangle &= \hbar\sqrt{1(1+1)-1(0)}|1,0\rangle=\hbar\sqrt{2}|1,0\rangle  \tag{11.73}\\
J_{-}|1,0\rangle &= \hbar\sqrt{1(1+1)-0(-1)}|1,-1\rangle=\hbar\sqrt{2}|1,-1\rangle \\
J_{-}|1,-1\rangle &= \hbar\sqrt{1(1+1)-(-1)(-2)}|1,-2\rangle=0
\end{align*}

To determine the matrix elements, we start with a general form for $J_+$:

\begin{gather*}
J_{+}=\hbar\left[\begin{array}{lll}
a & b & c \\
d & e & f \\
g & h & i
\end{array}\right]  \tag{11.74}
\end{gather*}

Then we apply this to each basis vector. For example:

\begin{gather*}
J_{+}|1,+1\rangle=\hbar\left[\begin{array}{lll}
a & b & c \\
d & e & f \\
g & h & i
\end{array}\right]\left[\begin{array}{l}
1 \\
0 \\
0
\end{array}\right]=\hbar\left[\begin{array}{l}
a \\
d \\
g
\end{array}\right] \tag{11.75}
\end{gather*}


From these conditions, we determine that $a=0, d=0, g=0$. Completing the calculations for $J_+$ and $J_-$:

\[
J_{+}=\hbar \sqrt{2}\left[\begin{array}{lll}
0 & 1 & 0  \tag{11.76}\\
0 & 0 & 1 \\
0 & 0 & 0
\end{array}\right], \quad J_{-}=\hbar \sqrt{2}\left[\begin{array}{lll}
0 & 0 & 0 \\
1 & 0 & 0 \\
0 & 1 & 0
\end{array}\right]
\]

This reveals the origin of the $\sqrt{2}$ factor encountered in our triplet state calculations. For the spin-$\frac{1}{2}$ case, the corresponding factor would be:

\begin{equation*}
\hbar \sqrt{\frac{1}{2}(\frac{1}{2}+1)-\frac{1}{2}(\frac{1}{2}-1)}=\hbar \tag{11.77}
\end{equation*}

This explains the structure of the spin raising and lowering operators.

\section*{12 Quantum dynamics in electromagnetic fields}
\subsection*{12.1 Particle motion in uniform magnetic fields}
The classical motion of a charged particle in an electromagnetic field is governed by the Lorentz force:

\begin{equation*}
m \ddot{\vec{x}}=q(\vec{E}+\frac{\vec{v}}{c} \times \vec{B}) \tag{12.1}
\end{equation*}

For a constant electric field $\vec{E}=-\vec{\nabla} \phi$ and uniform magnetic field $\vec{B}=(0,0,B)$, we can express the vector potential as:

\begin{equation*}
\vec{A}=\frac{1}{2} \vec{B} \times \vec{x} \Rightarrow \vec{\nabla} \times(\frac{1}{2} \vec{B} \times \vec{x})=\vec{B} \tag{12.2}
\end{equation*}

The Lagrangian formulation includes the minimal coupling term:

\begin{equation*}
\mathcal{L}=\frac{1}{2} m \dot{\vec{x}}^{2}-q \phi+\frac{q}{c} \dot{\vec{x}} \cdot \vec{A} \tag{12.3}
\end{equation*}

Expanding this expression:

\begin{align*}
\mathcal{L} &= \frac{1}{2} m \dot{\vec{x}}^{2}-q \phi+\frac{q}{c} \frac{1}{2} (\vec{B} \times \vec{x}) \cdot \dot{\vec{x}}\\
&= \frac{1}{2} m(\dot{x}_{1}^{2}+\dot{x}_{2}^{2}+\dot{x}_{3}^{2})-q \phi+\frac{q}{c} \frac{B}{2}(x_{1} \dot{x}_{2}-x_{2}\dot{x}_{1}) \tag{12.4}
\end{align*}

To verify this formulation reproduces the correct equations of motion, we apply the Euler-Lagrange equations:

\begin{align*}
\frac{\mathrm{d}}{\mathrm{d} t}(\frac{\partial \mathcal{L}}{\partial \dot{x}_{1}})-\frac{\partial \mathcal{L}}{\partial x_{1}} &= 0 \\
\frac{\mathrm{d}}{\mathrm{d} t}(m \dot{x}_{1}-\frac{q B}{2 c} x_{2})+q \frac{\partial \phi}{\partial x_{1}}-\frac{q B}{2 c} \dot{x}_{2} &= \\
&= m \ddot{x}_{1}-\frac{q B}{2 c} \dot{x}_{2}+q \frac{\partial \phi}{\partial x_{1}}-\frac{q B}{2 c} \dot{x}_{2}\\
&= m \ddot{x}_{1}-\frac{q B}{c} \dot{x}_{2}+q \frac{\partial \phi}{\partial x_{1}}=0 \tag{12.5}
\end{align*}

This corresponds to the $x_1$ component of the Lorentz force. Similarly for $x_2$:

\begin{align*}
\frac{\mathrm{d}}{\mathrm{d} t}(\frac{\partial \mathcal{L}}{\partial \dot{x}_{2}})-\frac{\partial \mathcal{L}}{\partial x_{2}} &= \\
\frac{\mathrm{d}}{\mathrm{d} t}(m \dot{x}_{2}+\frac{q B}{2 c} x_{1})+q \frac{\partial \phi}{\partial x_{2}}+\frac{q B}{2 c} \dot{x}_{1} &= \\
&= m \ddot{x}_{2}+\frac{q B}{2 c} \dot{x}_{1}+q \frac{\partial \phi}{\partial x_{2}}+\frac{q B}{2 c} \dot{x}_{1}\\
&= m \ddot{x}_{2}+\frac{q B}{c} \dot{x}_{1}+q \frac{\partial \phi}{\partial x_{2}}=0 \tag{12.6}
\end{align*}

The canonical momenta are therefore:

\[
\begin{array}{r}
p_{1}=m \dot{x}_{1}-\frac{q B}{2 c} x_{2} \\
p_{2}=m \dot{x}_{2}+\frac{q B}{2 c} x_{1}  \tag{12.7}\\
p_{3}=m \dot{x}_{3}
\end{array}
\]

The Hamiltonian is obtained through the Legendre transformation:

\begin{align*}
\mathcal{H} &= \sum p_{j} \dot{x}_{j}-\mathcal{L}\\
&= \dot{x}_{1}(m \dot{x}_{1}-\frac{q B}{2 c} x_{2})+\dot{x}_{2}(m \dot{x}_{2}+\frac{q B}{2 c} x_{1})+m \dot{x}_{3}^{2}\\
&\quad -\frac{1}{2} m(\dot{x}_{1}^{2}+\dot{x}_{2}^{2}+\dot{x}_{3}^{2})+q \phi-\frac{q B}{2 c}(x_{1} \dot{x}_{2}-x_{2} \dot{x}_{1})\\
&= \frac{1}{2} m(\dot{x}_{1}^{2}+\dot{x}_{2}^{2}+\dot{x}_{3}^{2})+q \phi \tag{12.8}
\end{align*}


To express the Hamiltonian in terms of canonical variables, we substitute the inverse relations for velocities:

\begin{align*}
\frac{(m\dot{x}_{1})^{2}}{m} &= \frac{1}{m}(p_{1}+\frac{q B}{2 c} x_{2})^{2}=\frac{p_{1}^{2}}{m}+\frac{q B}{m c} p_{1}x_{2}+\frac{q^{2}B^{2}}{4mc^{2}}x_{2}^{2} \\
\frac{(m\dot{x}_{2})^{2}}{m} &= \frac{1}{m}(p_{2}-\frac{q B}{2 c} x_{1})^{2}=\frac{p_{2}^{2}}{m}-\frac{q B}{m c}p_{2}x_{1}+\frac{q^{2}B^{2}}{4mc^{2}}x_{1}^{2}  \tag{12.9}\\
\frac{(m\dot{x}_{3})^{2}}{m} &= \frac{p_{3}^{2}}{m}
\end{align*}

The Hamiltonian becomes:

\begin{equation*}
\mathcal{H}=\frac{p_{1}^{2}+p_{2}^{2}+p_{3}^{2}}{2m}+q\phi+\frac{qB}{2mc}(p_{1}x_{2}-p_{2}x_{1})+\frac{q^{2}B^{2}}{8mc^{2}}(x_{1}^{2}+x_{2}^{2}) \tag{12.10}
\end{equation*}

An interesting observation emerges when examining the Poisson brackets of velocity components perpendicular to $\vec{B}$. Defining mechanical momenta:

\begin{equation*}
P_{i}=m\dot{x}_{i}=p_{i}+\frac{q}{c}A_{i} \tag{12.11}
\end{equation*}

We calculate:

\begin{align*}
\{m\dot{x}_{1},m\dot{x}_{2}\} &= \{P_{1},P_{2}\}=\{p_{1}+\frac{q}{c}A_{1},p_{2}+\frac{q}{c}A_{2}\}\\
&= \{p_{1},p_{2}\}+\frac{q}{c}\{p_{1},A_{2}\}+\frac{q}{c}\{A_{1},p_{2}\}+\frac{q^{2}}{c^{2}}\{A_{1},A_{2}\}\\
&= \frac{q}{c}(\frac{\partial A_{2}}{\partial x_{1}}-\frac{\partial A_{1}}{\partial x_{2}})=\frac{q}{c}(\frac{B}{2}+\frac{B}{2})=\frac{qB}{c}\neq 0 \tag{12.12}
\end{align*}

Applying canonical quantization, we obtain:

\begin{equation*}
[P_{1},P_{2}]=i\hbar\frac{qB}{c} \tag{12.13}
\end{equation*}

For an electron ($q=-e$, $m=m_e$), the quantum Hamiltonian becomes:

\begin{equation*}
H_{B}\approx\frac{p^{2}}{2m_e}+q\phi+\frac{eB}{2m_e c}(p_{1}x_{2}-p_{2}x_{1}) \tag{12.14}
\end{equation*}

Recognizing $p_{1}x_{2}-p_{2}x_{1}$ as the $z$-component of angular momentum $L_3$, we have:

\begin{equation*}
H_{B}=\underbrace{\frac{p^{2}}{2m_e}+q\phi}_{H_0}+\frac{eB}{2m_e c}L_3 \tag{12.15}
\end{equation*}

This simplified form reveals that:

\begin{equation*}
H_{B}=H_0+\frac{eB}{2m_e c}L_3 \tag{12.16}
\end{equation*}

The operators $L^2$ and $L_3$ still diagonalize this Hamiltonian, as they commute with both terms. The eigenvalues are:

\begin{equation*}
H\Psi=H_0\Psi+\frac{eB}{2m_e c}L_3\Psi=\left(-\frac{Z^2e^4m_e}{\hbar^2n^2}+\frac{eB\hbar m}{2m_e c}\right)\Psi \tag{12.17}
\end{equation*}

The presence of the magnetic quantum number $m$ lifts the degeneracy of the hydrogen atom energy levels.

We can express the magnetic interaction term using the Bohr magneton:

\begin{equation*}
\mu_B=\frac{e\hbar}{2m_e c} \tag{12.18}
\end{equation*}

Giving:

\begin{equation*}
H_\mu\Psi=\mu_B Bm\Psi \tag{12.19}
\end{equation*}

The "weak field" approximation (neglecting $B^2$ terms) is valid when $B\leq 10^5$ G. For the hydrogen atom with Bohr radius $a=0.5$ Å:

\begin{equation*}
\frac{q^2B^2}{8mc^2}(x_1^2+x_2^2\approx a^2)\approx 10^{-27} \text{ J} \tag{12.20}
\end{equation*}


The Bohr magneton value is $\mu_{B} \approx 0.927 \times 10^{-20} \frac{\text{erg}}{\text{gauss}}$, where $1 \text{erg}=10^{-7} \text{J}$. For typical laboratory fields:

\begin{equation*}
\mu_{B}B \approx 10^{-22} \text{J} \tag{12.21}
\end{equation*}

Since the hydrogen atom ground state energy is approximately $E_n \approx 10^{-18} \text{J}$, this confirms that the $B^2$ term is indeed negligible compared to the magnetic moment interaction.

The magnetic field term in the Hamiltonian can be written in the more familiar form:

\begin{equation*}
H_{\mu}=\underbrace{\frac{e}{2m_e c}L_3}_{-\mu_3}B=-\vec{\mu}\cdot\vec{B} \tag{12.22}
\end{equation*}

Where $\vec{\mu}$ represents the orbital magnetic dipole moment:

\begin{equation*}
\vec{\mu}=-\frac{e}{2m_e c}\vec{L} \tag{12.23}
\end{equation*}

This quantum mechanical result aligns with classical electrodynamics as described by Ampère's law:

\begin{equation*}
\vec{\mu}=-\frac{e}{2m_e c}\vec{L}=-\frac{e}{2m_e c}m_e R^2\dot{\phi}\hat{u}_3 \tag{12.24}
\end{equation*}

For uniform circular motion, $\dot{\phi}=\frac{2\pi}{T}$, giving:

\begin{equation*}
\vec{\mu}=\frac{1}{c}\underbrace{\left(-\frac{e}{T}\right)}_i \underbrace{\pi R^2}_S \hat{u}_3=\frac{1}{c}iS\hat{u}_3 \tag{12.25}
\end{equation*}

This matches the classical formula for a current loop's magnetic moment.

\subsection*{12.2 Zeeman effect energy spectrum}
Considering the allowed quantum numbers:

$$
\begin{array}{llll}
\text{For} & n=1 & \ell=0 & m=0 \\
\text{For} & n=2 & \ell=0,1 & m=-1,0,1 \\
\text{For} & n=3 & \ell=0,1,2 & m=-2,-1,0,1,2 \\
\ldots & & &
\end{array}
$$

The energy spectrum with magnetic field splitting becomes:

\begin{gather*}
\text{For } n=1 \quad E_1 \\
\text{For } n=2\left\{\begin{array}{l}
E_2-\mu_B B \\
E_2 \\
E_2+\mu_B B
\end{array}\right. \\
\text{For } n=3\left\{\begin{array}{l}
E_3-2\mu_B B \\
E_3-\mu_B B \\
E_3 \\
E_3+\mu_B B \\
E_3+2\mu_B B
\end{array}\right. \tag{12.26}\\
\cdots
\end{gather*}

This splitting pattern, known as the normal Zeeman effect, does not fully match experimental observations. The discrepancy arises because we have not yet accounted for electron spin.

\subsection*{12.3 Transition selection rules}
The allowed transitions between energy levels follow specific selection rules:

\begin{align*}
\Delta m &= m'-m=0,\pm 1 \\
\Delta\ell &= \ell'-\ell=\pm 1 \tag{12.27}\\
\Delta m_s &= m_s'-m_s=0
\end{align*}

These transitions correspond to changes in charge distribution. In classical electrodynamics, this is analogous to the radiation from an oscillating dipole (antenna effect):

\begin{equation*}
I=\frac{2e^2}{3c^3}\left(\frac{d^2\vec{r}}{dt^2}\right)^2 \tag{12.28}
\end{equation*}


In quantum mechanics, the radiation intensity is given by:

\begin{equation*}
I_{\nu\nu'}=\frac{2e^2}{3c^3}\left|\frac{d^2}{dt^2}\langle t,\nu'|\vec{r}|\nu,t\rangle\right|^2 \tag{12.29}
\end{equation*}

Where the time-dependent states are:

\begin{equation*}
|\nu,t\rangle=\exp\left(-i\frac{t}{\hbar}E_n\right)|n,\ell,m\rangle \tag{12.30}
\end{equation*}

Evaluating the time derivatives:

\begin{align*}
I_{\nu\nu'} &=\frac{2e^2}{3c^3}\left|\frac{d^2}{dt^2}\exp\left(-i\frac{t}{\hbar}(E_n-E_{n'})\right)\langle n',\ell',m'|\vec{r}|n,\ell,m\rangle\right|^2\\
&=\frac{2e^2}{3c^3}\omega_{n'n}^4\left|\exp\left(-i\frac{t}{\hbar}(E_n-E_{n'})\right)\langle n',\ell',m'|\vec{r}|n,\ell,m\rangle\right|^2 \tag{12.31}\\
&=\frac{2e^2}{3c^3}\omega_{n'n}^4\left|\langle n',\ell',m'|\vec{r}|n,\ell,m\rangle\right|^2
\end{align*}

The selection rules determine when the matrix element $\langle n',\ell',m'|\vec{r}|n,\ell,m\rangle$ is non-zero, thus identifying allowed transitions.

\subsection*{12.4 Spin-orbit coupling and anomalous Zeeman effect}
To account for electron spin, we introduce an additional magnetic moment:

\begin{equation*}
H_{\mu}=\frac{eB}{2m_e c}L_3 \tag{12.32}
\end{equation*}

By analogy with the orbital magnetic moment:

\begin{equation*}
\vec{\mu}_L=-\frac{e}{2m_e c}\vec{L} \tag{12.33}
\end{equation*}

We define the spin magnetic moment:

\begin{equation*}
\vec{\mu}_s=-\frac{e}{2m_e c}g\vec{S} \tag{12.34}
\end{equation*}

Where $g$ is the gyromagnetic ratio, which equals 2 for electrons. The complete magnetic interaction term becomes:

\begin{equation*}
H_B=\frac{eB}{2m_e c}(L_3+gS_3) \tag{12.35}
\end{equation*}

A more comprehensive Hamiltonian would include relativistic corrections:

\begin{equation*}
H=H_0+H_B+H_{SL}+H_R+\ldots \tag{12.36}
\end{equation*}

The spin-orbit coupling term $H_{SL}$ arises from the interaction between the electron's spin and its orbital motion:

\begin{equation*}
H_{SL}=\frac{1}{2m_e^2c^2}\frac{1}{r}\frac{dU}{dr}\vec{S}\cdot\vec{L}=\frac{Ze^2}{2m_e^2c^2r^3}\vec{S}\cdot\vec{L}=\chi\vec{S}\cdot\vec{L} \tag{12.37}
\end{equation*}

The relativistic kinetic energy correction $H_R$ is:

\begin{equation*}
-\frac{(p^2)^2}{8m_e^3c^2}+\ldots \tag{12.38}
\end{equation*}

This term originates from the relativistic energy-momentum relation:

\begin{align*}
E &=\sqrt{p^2c^2+m_e^2c^4}=m_ec^2\sqrt{1+\frac{p^2}{m_e^2c^2}}\\
&\approx m_ec^2\left(1+\frac{p^2}{2m_e^2c^2}-\frac{(p^2)^2}{8m_e^4c^4}\right)\\
&=m_ec^2+\frac{p^2}{2m_e}-\frac{(p^2)^2}{8m_e^3c^2} \tag{12.39}
\end{align*}


To estimate the magnitude of the spin-orbit coupling, we use:

\begin{align*}
\vec{S}\cdot\vec{L} &\sim |S||L| \sim \hbar^2\sqrt{\ell(\ell+1)s(s+1)} \approx \hbar^2 \\
r &\approx a=\frac{\hbar^2}{m_e e^2} \\
\alpha &= \frac{e^2}{\hbar c} \approx \frac{1}{137} \tag{12.40}\\
\alpha a &= \frac{\hbar}{m_e c} \\
k &\approx \frac{1}{a}
\end{align*}

The energy eigenvalue of the spin-orbit term is therefore:

\begin{equation*}
E_{SL} \approx \frac{Ze^2\hbar^2}{2m_e^2c^2r^3} \approx \frac{Ze^2\hbar^2}{2m_e^2c^2a^3}=Z\alpha^2\frac{e^2}{2a}=Z\alpha^2|E_1| \tag{12.41}
\end{equation*}

For the relativistic correction term:

\begin{align*}
|E_R| &= \frac{p^4}{8m_e^3c^2} \approx \frac{(\hbar k)^4}{8m_e^3c^2}=\frac{\hbar^4}{8m_e^3c^2a^4}=\frac{\hbar^2}{8m_e}\frac{(\alpha a)^2}{a^4}\\
&= \frac{1}{4}\underbrace{\frac{\hbar^2}{2m_e}\frac{2}{e^2}}_a\frac{\alpha^2}{a}\underbrace{\frac{e^2}{2a}}_{|E_1|}=\frac{\alpha^2}{4}|E_1| \tag{12.42}
\end{align*}

\subsection*{12.5 Anomalous Zeeman effect and Paschen-Back effect}
As a first approximation, we consider only:

\begin{equation*}
H=H_0+H_B \tag{12.43}
\end{equation*}

The wavefunction now includes spin:

\begin{equation*}
\Psi_{n\ell mm_s}=\psi_{n\ell m}|s,m_s\rangle \tag{12.44}
\end{equation*}

Applying the Hamiltonian:

\begin{align*}
H\Psi_{n\ell mm_s} &= (H_0+\frac{eB}{2m_e c}(L_3+gS_3))\psi_{n\ell m}|s,m_s\rangle\\
&= H_0\psi_{n\ell m}|s,m_s\rangle+\frac{eB}{2m_e c}L_3\psi_{n\ell m}|s,m_s\rangle+\psi_{n\ell m}\frac{eB}{2m_e c}gS_3|s,m_s\rangle\\
&= (E_n+\frac{eB}{2m_e c}\hbar m+\frac{eB}{2m_e c}g\hbar m_s)\psi_{n\ell m}|s,m_s\rangle\\
&= (E_n+\frac{e\hbar B}{2m_e c}(m+gm_s))\Psi_{n\ell mm_s}\\
&= (E_n+\mu_B B(m+gm_s))\Psi_{n\ell mm_s} \tag{12.45}
\end{align*}

This gives the energy spectrum:

\begin{align*}
\text{For } n=2&\left\{\begin{array}{l}
E_2+\mu_B B(-1\pm 1)\\
E_2\pm\mu_B B\\
E_2+\mu_B B(1\pm 1)
\end{array}\right.\\
\text{For } n=3&\left\{\begin{array}{l}
E_3+\mu_B B(-2\pm 1)\\
E_3+\mu_B B(-1\pm 1)\\
E_3\pm\mu_B B\\
E_3+\mu_B B(1\pm 1)\\
E_3+\mu_B B(2\pm 1)
\end{array}\right.\tag{12.46}\\
&\ldots
\end{align*}

However, this approach does not fully explain the observed spectrum because we've neglected the spin-orbit coupling. When we include $H_{SL}$, we can no longer use $S_3$ and $L_3$ as good quantum numbers since:

\begin{align*}
[\vec{S}\cdot\vec{L}, L_3] &\neq 0 \tag{12.47}\\
[\vec{S}\cdot\vec{L}, S_3] &\neq 0
\end{align*}

Instead, we must use the total angular momentum $\vec{J}=\vec{S}+\vec{L}$, which provides a set of commuting operators:

\begin{align*}
[S^2, J^2] &= 0\\
[L^2, J^2] &= 0\\
[\vec{S}\cdot\vec{L}, J^2] &= 0 \tag{12.48}\\
[\vec{S}\cdot\vec{L}, J_3] &= 0
\end{align*}

The Hamiltonian is then diagonalized by the set $\{L^2, S^2, J^2, J_3\}$, giving:

\begin{equation*}
H=H_0+\frac{eB}{2m_e c}(J_3+S_3) \tag{12.49}
\end{equation*}


Although the relativistic terms are numerically small, they guide us to the correct quantum numbers for our problem:

\begin{equation*}
n, j, m_j, \ell, s \tag{12.50}
\end{equation*}

The corresponding spinor states are:

\begin{equation*}
|n, j, m_j, \ell, s\rangle \tag{12.51}
\end{equation*}

We separate the Hamiltonian into unperturbed and perturbation terms:

\begin{equation*}
H=H_0+\frac{eB}{2m_e c}J_3+\frac{eB}{2m_e c}S_3=H_{un}+H_p \tag{12.52}
\end{equation*}

The unperturbed Hamiltonian gives:

\begin{equation*}
H_{un}|n, j, m_j, \ell, s\rangle=(E_n+\mu_B Bm_j)|n, j, m_j, \ell, s\rangle \tag{12.53}
\end{equation*}

The first-order correction from the $S_3$ term is:

\begin{equation*}
E \approx E_n+\mu_B Bm_j+\frac{eB}{2m_e c}\langle s,\ell,m_j,j|S_3|j,m_j,\ell,s\rangle \tag{12.54}
\end{equation*}

For $j=\ell-\frac{1}{2}$, we calculate:

\begin{align*}
&\langle s,\ell,m_j,\ell-\frac{1}{2}|S_3|\ell-\frac{1}{2},m_j,\ell,s\rangle\\
&=(\langle-\frac{1}{2},s|\langle\ell,m_j+\frac{1}{2}|\beta_-^*+|+\frac{1}{2},s\rangle|\ell,m_j-\frac{1}{2}\rangle\alpha_-^*)\\
&\times S_3(\alpha_-|\ell,m_j-\frac{1}{2}\rangle|s,+\frac{1}{2}\rangle+\beta_-|\ell,m_j+\frac{1}{2}\rangle|s,-\frac{1}{2}\rangle)\\
&=|\alpha_-|^2\langle m_j-\frac{1}{2},\ell|\ell,m_j-\frac{1}{2}\rangle\langle+\frac{1}{2},s|S_3|s,+\frac{1}{2}\rangle\\
&+|\beta_-|^2\langle m_j+\frac{1}{2},\ell|\ell,m_j+\frac{1}{2}\rangle\langle-\frac{1}{2},s|S_3|s,-\frac{1}{2}\rangle \tag{12.55}\\
&=|\alpha_-|^2\langle+\frac{1}{2},s|S_3|s,+\frac{1}{2}\rangle+|\beta_-|^2\langle-\frac{1}{2},s|S_3|s,-\frac{1}{2}\rangle\\
&=|\alpha_-|^2\frac{\hbar}{2}+|\beta_-|^2(-\frac{\hbar}{2})=\frac{\hbar}{2}\left(\frac{\ell-m_j+\frac{1}{2}}{2\ell+1}-\frac{\ell+m_j+\frac{1}{2}}{2\ell+1}\right)\\
&=-\frac{\hbar m_j}{2\ell+1}
\end{align*}

Similarly, for $j=\ell+\frac{1}{2}$:

\begin{equation*}
\frac{\hbar m_j}{2\ell+1} \tag{12.56}
\end{equation*}

Therefore, the energy spectrum becomes:

\begin{equation*}
E=E_n+\frac{eB\hbar}{2m_e c}m_j\left(1\pm\frac{1}{2\ell+1}\right)=E_n+\frac{eB\hbar m_j}{2m_e c}\frac{2\ell+1\pm 1}{2\ell+1} \tag{12.57}
\end{equation*}

This gives the experimentally observed spectrum:

\begin{equation*}
E_n(\ell\pm 1/2,m_j)=E_n+\mu_B Bm_j\frac{2\ell+1\pm 1}{2\ell+1} \tag{12.58}
\end{equation*}

Which, according to the selection rules, yields:

\[
\text{For } n=2\left\{\begin{array}{l}
E_2(1-1/2,\pm 1/2)=E_2+\mu_B B(\pm 1/2)\frac{2}{3} \tag{12.59}\\
E_2(1+1/2,\pm 1/2)=E_2+\mu_B B(\pm 1/2)\frac{4}{3}\\
E_2(1+1/2,\pm 3/2)=E_2+\mu_B B(\pm 3/2)\frac{4}{3}
\end{array}\right.
\]

$\cdots$

\section*{13 Perturbation theory}
\subsection*{13.1 General methodology}
In quantum mechanics, we often solve problems by finding solutions to simpler systems and then determining how these solutions change when the system is modified. When the modification is small, we call it a perturbation.


The perturbation approach begins by expressing the Hamiltonian as:

\begin{equation*}
H=H_0+\tau V \tag{13.1}
\end{equation*}

Where $H_0$ is the unperturbed Hamiltonian with known eigenvalues and eigenstates:

\begin{equation*}
H_0|n\rangle=E_n^0|n\rangle \tag{13.2}
\end{equation*}

The perturbation is considered small when:

\begin{equation*}
\langle n|H_0|n\rangle=E_n^0 \gg \tau\langle n|V|n\rangle \tag{13.3}
\end{equation*}

Our goal is to solve the perturbed eigenvalue problem:

\begin{equation*}
H|E_n\rangle=E_n|E_n\rangle \tag{13.4}
\end{equation*}

We assume that both eigenstates and eigenvalues can be expanded in powers of $\tau$:

\[
\left\{\begin{array}{l}
|E_n\rangle=|a_n\rangle+\tau|b_n\rangle+\tau^2|c_n\rangle+\ldots \tag{13.5}\\
E_n=E_n^0+\tau E_n^1+\tau^2 E_n^2+\ldots
\end{array}\right.
\]

With $|a_n\rangle=|n\rangle$ so that when $\tau\rightarrow 0$ we recover the unperturbed solution. Substituting these expansions into the eigenvalue equation:

\[
\begin{array}{r}
H|E_n\rangle=(H_0+\tau V)(|n\rangle+\tau|b_n\rangle+\tau^2|c_n\rangle+\ldots)=\\
=H_0|n\rangle+\tau(H_0|b_n\rangle+V|n\rangle)+\tau^2(H_0|c_n\rangle+V|b_n\rangle)+\ldots \tag{13.6}
\end{array}
\]

For the right side of the eigenvalue equation:

\begin{gather*}
E_n|E_n\rangle=(E_n^0+\tau E_n^1+\tau^2 E_n^2+\ldots)(|n\rangle+\tau|b_n\rangle+\tau^2|c_n\rangle+\ldots)=\\
=E_n^0|n\rangle+\tau(E_n^0|b_n\rangle+E_n^1|n\rangle)+\tau^2(E_n^0|c_n\rangle+E_n^1|b_n\rangle+E_n^2|n\rangle)+\ldots \tag{13.7}
\end{gather*}

Equating coefficients of like powers of $\tau$:

\begin{align*}
&H_0|n\rangle=E_n^0|n\rangle\\
&H_0|b_n\rangle+V|n\rangle=E_n^0|b_n\rangle+E_n^1|n\rangle \tag{13.8}\\
&H_0|c_n\rangle+V|b_n\rangle=E_n^0|c_n\rangle+E_n^1|b_n\rangle+E_n^2|n\rangle
\end{align*}

We expand the correction states in terms of the unperturbed eigenbasis, excluding the state $|n\rangle$:

\begin{equation*}
|b_n\rangle=\sum_{i\neq n}B_i(n)|i\rangle \tag{13.9}
\end{equation*}

This exclusion is necessary because any component along $|n\rangle$ can be absorbed into the eigenvalue:

\begin{equation*}
|b_n\rangle'=\sum_i B_i'(n)|i\rangle=B_n'(n)|n\rangle+\sum_{i\neq n}B_i(n)|i\rangle=B_n'(n)|n\rangle+|b_n\rangle \tag{13.10}
\end{equation*}

This principle applies similarly to higher-order corrections. If we include such terms, the eigenvalue equation becomes:

\begin{align*}
&H(|E_n\rangle+|n\rangle+\tau B_n'(n)|n\rangle+\tau^2 C_n'(n)|n\rangle+\ldots)=\\
&E_n(|E_n\rangle+|n\rangle+\tau B_n'(n)|n\rangle+\tau^2 C_n'(n)|n\rangle+\ldots)\\
&H|E_n\rangle+H(|n\rangle+\tau B_n'(n)|n\rangle+\tau^2 C_n'(n)|n\rangle+\ldots)=\\
&E_n|E_n\rangle+E_n(|n\rangle+\tau B_n'(n)|n\rangle+\tau^2 C_n'(n)|n\rangle+\ldots) \tag{13.11}
\end{align*}

The action of $H$ on these additional terms gives:

\begin{align*}
&H(|n\rangle+\tau B_n'(n)|n\rangle+\tau^2 C_n'(n)|n\rangle+\ldots)=\\
&(H|n\rangle+\tau B_n'(n)H|n\rangle+\tau^2 C_n'(n)H|n\rangle+\ldots)=\\
&(E_n|n\rangle+\tau B_n'(n)E_n|n\rangle+\tau^2 C_n'(n)E_n|n\rangle+\ldots) \tag{13.12}
\end{align*}


These additional terms indeed cancel out.

For the first-order energy correction, we multiply the first-order equation by $\langle n|$:

\begin{equation*}
\langle n|H_0|b_n\rangle+\langle n|V|n\rangle=\langle n|E_n^0|b_n\rangle+\langle n|E_n^1|n\rangle \tag{13.13}
\end{equation*}

Since $|b_n\rangle$ is constructed to have no component along $|n\rangle$:

\begin{equation*}
\langle n|b_n\rangle=\langle n|\sum_{i\neq n}B_i(n)|i\rangle=\sum_{i\neq n}B_i(n)\langle n|i\rangle=0 \tag{13.14}
\end{equation*}

The first-order correction simplifies to:

\begin{align*}
&\langle n|H_0|b_n\rangle+\langle n|V|n\rangle=\langle n|E_n^0|b_n\rangle+\langle n|E_n^1|n\rangle\\
&\langle n|\sum_{i\neq n}B_i(n)H_0|i\rangle+\langle n|V|n\rangle=E_n^0\langle n|b_n\rangle+E_n^1\underbrace{\langle n|n\rangle}_{=1}\\
&\sum_{i\neq n}E_i^0B_i(n)\langle n|i\rangle+\langle n|V|n\rangle=E_n^1 \tag{13.15}\\
&\langle n|V|n\rangle=E_n^1
\end{align*}

For the second-order energy correction:

\begin{align*}
&\langle n|H_0|c_n\rangle+\langle n|V|b_n\rangle=\langle n|E_n^0|c_n\rangle+\langle n|E_n^1|b_n\rangle+\langle n|E_n^2|n\rangle\\
&\langle n|V|b_n\rangle=E_n^2 \tag{13.16}
\end{align*}

To find the corrections to the eigenstates, we project the first-order equation onto $\langle k|$ where $k\neq n$:

\begin{align*}
&\langle k|H_0|b_n\rangle+\langle k|V|n\rangle=\langle k|E_n^0|b_n\rangle+\langle k|E_n^1|n\rangle\\
&\langle k|\sum_{i\neq n}B_i(n)H_0|i\rangle+\langle k|V|n\rangle=E_n^0\sum_{i\neq n}B_i(n)\underbrace{\langle k|i\rangle}_{\delta_{ik}}+E_n^1\langle k|n\rangle\\
&\sum_{i\neq n}E_i^0B_i(n)\underbrace{\langle k|i\rangle}_{\delta_{ik}}+\langle k|V|n\rangle=E_n^0B_k(n) \tag{13.17}\\
&(E_k^0-E_n^0)B_k(n)+\langle k|V|n\rangle=0\\
&B_k(n)=-\frac{\langle k|V|n\rangle}{E_k^0-E_n^0}
\end{align*}

Similarly for higher-order corrections:

\begin{align*}
C_k(n)&=-\frac{\langle k|V|b_n\rangle}{E_k^0-E_n^0}+E_n^1\frac{\langle k|b_n\rangle}{E_k^0-E_n^0} \tag{13.18}\\
D_k(n)&=-\frac{\langle k|V|c_n\rangle}{E_k^0-E_n^0}+E_n^1\frac{\langle k|c_n\rangle}{E_k^0-E_n^0}+E_n^2\frac{\langle k|b_n\rangle}{E_k^0-E_n^0}
\end{align*}

\subsection*{13.2 Application to the harmonic oscillator}
Consider a harmonic oscillator with Hamiltonian:

\begin{equation*}
H=\frac{p^2}{2m}+\frac{k}{2}x^2 \tag{13.19}
\end{equation*}

If the spring constant undergoes a small change:

\begin{equation*}
k\rightarrow k_0(1+\tau)\Rightarrow H=H_0+\tau\frac{k_0}{2}x^2 \tag{13.20}
\end{equation*}

The first-order energy correction is:

\begin{align*}
E_n^1&=\langle n|V|n\rangle=\frac{k_0}{2}\langle n|x^2|n\rangle=\frac{k_0}{2}\frac{\hbar}{2m\omega_0}\langle n|(a^2+(a^\dagger)^2+2n+1)|n\rangle\\
&=\frac{m\omega_0^2}{2}\frac{\hbar}{m\omega_0}(n+\frac{1}{2})=\frac{\hbar\omega_0}{2}(n+\frac{1}{2}) \tag{13.21}
\end{align*}

For the second-order correction, we need:

\begin{align*}
&|b_n\rangle=\sum_{k\neq n}B_k(n)|k\rangle \tag{13.22}\\
&B_k(n)=-\frac{\langle k|V|n\rangle}{E_k^0-E_n^0} \tag{13.23}
\end{align*}


Substituting the expression for the first-order state correction:

\begin{equation*}
|b_n\rangle=-\sum_{k\neq n}\frac{\langle k|V|n\rangle}{E_k^0-E_n^0}|k\rangle \tag{13.24}
\end{equation*}

The second-order energy correction is:

\begin{equation*}
E_n^2=\langle n|V|b_n\rangle \tag{13.25}
\end{equation*}

Recognizing that $\langle k|V|n\rangle$ is generally complex and using $\langle n|V|k\rangle=\langle k|V|n\rangle^*$:

\begin{align*}
E_n^2&=-\sum_{k\neq n}\langle n|V|\frac{\langle k|V|n\rangle}{E_k^0-E_n^0}k\rangle\\
&=-\sum_{k\neq n}\frac{\langle n|V|k\rangle\langle k|V|n\rangle}{E_k^0-E_n^0}\\
&=-\sum_{k\neq n}\frac{\langle k|V|n\rangle^*\langle k|V|n\rangle}{E_k^0-E_n^0} \tag{13.26}\\
&=-\sum_{k\neq n}\frac{|\langle k|V|n\rangle|^2}{E_k^0-E_n^0}
\end{align*}

For the harmonic oscillator with perturbation $V=\frac{k_0}{2}x^2$, only states $n\pm 2$ contribute due to the selection rules of the ladder operators:

\begin{align*}
E_n^2&=-\frac{|\langle n+2|V|n\rangle|^2}{E_{n+2}^0-E_n^0}-\frac{|\langle n-2|V|n\rangle|^2}{E_{n-2}^0-E_n^0}\\
&=-\frac{\hbar^2}{4m^2\omega_0^2}\frac{m^2\omega_0^4}{4}\frac{|\langle n+2|a^2|n\rangle|^2}{E_{n+2}^0-E_n^0}-\frac{\hbar^2}{4m^2\omega_0^2}\frac{m^2\omega_0^4}{4}\frac{|\langle n-2|(a^\dagger)^2|n\rangle|^2}{E_{n-2}^0-E_n^0}\\
&=-\frac{\hbar^2\omega_0^2}{16}\frac{(n+1)(n+2)|\langle n+2|n+2\rangle|^2}{E_{n+2}^0-E_n^0}-\frac{\hbar^2\omega_0^2}{16}\frac{n(n-1)|\langle n-2|n-2\rangle|^2}{E_{n-2}^0-E_n^0}\\
&=-\frac{\hbar^2\omega_0^2}{16}\frac{(n+1)(n+2)}{2\hbar\omega_0}-\frac{\hbar^2\omega_0^2}{16}\frac{n(n-1)}{-2\hbar\omega_0}\\
&=-\frac{\hbar\omega_0}{32}(n^2+3n+2-n^2+n)=-\frac{\hbar\omega_0}{8}(n+\frac{1}{2}) \tag{13.27}
\end{align*}

Continuing this process for higher-order corrections, we obtain:

\begin{equation*}
E_n=\hbar\omega_0(n+\frac{1}{2})(1+\frac{\tau}{2}-\frac{\tau^2}{8}+\ldots) \tag{13.28}
\end{equation*}

This series is precisely the Taylor expansion of:

\begin{equation*}
\sqrt{1+\tau}\sim 1+\frac{\tau}{2}-\frac{\tau^2}{8}+\ldots \tag{13.29}
\end{equation*}

The exact result should be $\hbar\omega=\hbar\frac{\sqrt{k}}{m}$, and with the perturbed spring constant:

\begin{equation*}
\hbar\omega=\hbar\frac{\sqrt{k_0(1+\tau)}}{m}=\hbar\omega_0\sqrt{1+\tau} \tag{13.30}
\end{equation*}

This confirms that our perturbation approach correctly reproduces the exact solution.

\section*{14 Identical particles and Spin-Statistics Theorem}
\subsection*{14.1 Many-particle systems}
In classical physics, we can theoretically track each particle's position and momentum at all times. In quantum mechanics, however, the uncertainty principle creates an effective "error box" around each particle. When particles are close enough that their error boxes overlap, we cannot distinguish them individually.

Quantum mechanics reveals that only two types of particle statistics are possible:
\begin{itemize}
  \item Fermions with antisymmetric wavefunctions
  \item Bosons with symmetric wavefunctions
\end{itemize}

The Hamiltonian for a multi-particle system has the form:

\begin{equation*}
H=H(\vec{x}_1,\vec{x}_2,\ldots,\vec{x}_N) \tag{14.1}
\end{equation*}

Which might be explicitly written as:

\begin{equation*}
H=\sum_{j}^{N}(\frac{p_j^2}{2m_j}+U(x_j))+\frac{1}{2}\sum_{j}^{N}\sum_{l\neq j}^{N}U(|x_j-x_l|) \tag{14.2}
\end{equation*}

This Hamiltonian is symmetric under particle exchange:

\begin{equation*}
H(\vec{x}_1,\vec{x}_2,\ldots,\vec{x}_i,\ldots\vec{x}_j,\ldots,\vec{x}_N)=H(\vec{x}_1,\vec{x}_2,\ldots,\vec{x}_j,\ldots\vec{x}_i,\ldots,\vec{x}_N) \tag{14.3}
\end{equation*}


A more general Hamiltonian could incorporate various interaction terms:

\begin{equation*}
H=H_0+U+H_{SO}+H_B+H_{SS}+\ldots \tag{14.4}
\end{equation*}

Where:
\begin{itemize}
  \item $H_{SO}$ is the spin-orbit coupling: $H_{SO}=\sum_{j=1}^{N}\chi L_j\cdot S_j$ where $\chi=\frac{1}{2m_e^2c^2r}\frac{dU}{dr}$
  \item $H_B$ is the magnetic field coupling: $H_B=\sum_{j=1}^{N}\gamma B(L_j+gS_j)$
  \item $H_{SS}$ is the spin-spin coupling: $H_{SS}=\frac{1}{2}\sum_{j=1}^{N}\sum_{i\neq j}^{N}\eta_{ji}S_j\cdot S_i$
\end{itemize}

\subsection*{14.2 Construction of symmetrized states}
We introduce the permutation operator $P$ that exchanges particle labels:

\begin{equation*}
P_{ji}\Psi(\vec{x}_1,\vec{x}_2,\ldots,\vec{x}_i,\ldots\vec{x}_j,\ldots,\vec{x}_N)=\Psi(\vec{x}_1,\vec{x}_2,\ldots,\vec{x}_j,\ldots\vec{x}_i,\ldots,\vec{x}_N) \tag{14.5}
\end{equation*}

For a system of $N$ particles, there are $N!$ possible permutations. Using these operators, we can construct fully symmetric states. For example, with $N=3$:

\begin{align*}
\phi_s(1,2,3)&=\frac{1}{\sqrt{3!}}(1+P_{13}P_{12}+P_{23}P_{12}+P_{12}+P_{23}+P_{13})\Psi(1,2,3)\\
&=\frac{1}{\sqrt{6}}(\Psi(1,2,3)+\Psi(2,3,1)+\Psi(3,1,2)+\Psi(2,1,3)+\Psi(1,3,2)+\Psi(3,2,1)) \tag{14.6}
\end{align*}

A key property of symmetric states is that they remain unchanged under particle exchange:

\begin{equation*}
\phi_s(1,3,2)=\phi_s(1,2,3) \tag{14.7}
\end{equation*}

Similarly, we can construct fully antisymmetric wavefunctions:

\begin{align*}
\phi_a(1,2,3)&=\frac{1}{\sqrt{3!}}(1+P_{13}P_{12}+P_{23}P_{12}-P_{12}-P_{23}-P_{13})\Psi(1,2,3)\\
&=\frac{1}{\sqrt{6}}(\Psi(1,2,3)+\Psi(2,3,1)+\Psi(3,1,2)-\Psi(2,1,3)-\Psi(1,3,2)-\Psi(3,2,1)) \tag{14.8}
\end{align*}

Antisymmetric wavefunctions change sign under odd permutations:

\begin{equation*}
\phi_a(1,3,2)=-\phi_a(1,2,3) \tag{14.9}
\end{equation*}

The general properties of these wavefunctions are:

\begin{align*}
&\phi_s(1,3,2)=\phi_s(\text{any permutation})\\
&\phi_a(1,3,2)=\phi_a(\text{even permutations}) \tag{14.10}\\
&\phi_a(1,3,2)=-\phi_a(\text{odd permutations})
\end{align*}

The general construction formulas are:

\begin{gather*}
\phi_s=\sum_P P\Psi(1,2,\ldots,N) \tag{14.11}\\
\phi_a=\sum_P(-1)^{\eta_P}P\Psi(1,2,\ldots,N) \tag{14.12}
\end{gather*}

Where $\eta_P$ is defined as:
\[
\eta_P= \begin{cases}
0 & \text{for even elementary permutations} \tag{14.13}\\
1 & \text{for odd elementary permutations}
\end{cases}
\]

An important consequence is that if $\vec{x}_i=\vec{x}_j$ for any two particles in an antisymmetric state, then $\phi_a=0$. This means there is zero probability of finding two fermions at the same position—a precursor to the Pauli exclusion principle.

Conversely, $\phi_s$ reaches its maximum when all particles are in the same position: $\vec{x}_1=\vec{x}_2=\ldots=\vec{x}_N$.

Since the Hamiltonian is symmetric under particle exchange, the permutation operator commutes with it:

\begin{gather*}
PH\Psi=PE\Psi\\
H(P\Psi)=E(P\Psi) \tag{14.14}
\end{gather*}

Therefore, both symmetric and antisymmetric states are eigenstates of the Hamiltonian:

\begin{align*}
H\phi_s&=H\sum_P P\Psi(1,2,\ldots,N)=\sum_P PH\Psi(1,2,\ldots,N)\\
&=\sum_P PE\Psi(1,2,\ldots,N)=E\phi_s \tag{14.15}\\
H\phi_a&=H\sum_P(-1)^{\eta_P}P\Psi(1,2,\ldots,N)=\sum_P(-1)^{\eta_P}PH\Psi(1,2,\ldots,N)\\
&=\sum_P(-1)^{\eta_P}PE\Psi(1,2,\ldots,N)=E\phi_a \tag{14.16}
\end{align*}


Both $\phi_s$ and $\phi_a$ are indeed eigenstates of the Hamiltonian, and their symmetry properties are preserved during time evolution:

\begin{align*}
&P\exp(-i\frac{t}{\hbar}H)\Psi(1,2,\ldots,N)=\sum_{k}^{\infty}(\frac{-i}{\hbar})^k t^k PH^k\Psi(1,2,\ldots,N)\\
&=\sum_{k}^{\infty}(\frac{-i}{\hbar})^k t^k H^kP\Psi(1,2,\ldots,N)=\exp(-i\frac{t}{\hbar}H)P\Psi(1,2,\ldots,N) \tag{14.17}
\end{align*}

\subsection*{14.3 Bosons, Fermions, Pauli's principle and Spin-Statistics Theorem}
For many systems, the Hamiltonian can be separated into single-particle terms:

\begin{equation*}
H=\sum_i H_i \tag{14.18}
\end{equation*}

Where each single-particle Hamiltonian might include various interactions:

\begin{equation*}
H_j=\frac{p_j^2}{2m_j}+\chi L_j\cdot S_j+\gamma B(L_j+gS_j) \tag{14.19}
\end{equation*}

When these Hamiltonians commute ($[H_j,H_i]=0$ for $i\neq j$), the wavefunction can be factorized:

\begin{equation*}
\Psi(\vec{x}_1,\vec{x}_2,\ldots,\vec{x}_N)=\prod_{j=1}^{N}\Psi_{\alpha_j}(\vec{x}_j) \tag{14.20}
\end{equation*}

Where $\alpha_j$ represents the quantum numbers for the $j$-th particle.

\subsection*{14.3.1 Symmetric wavefunctions for a separable Hamiltonian}
For symmetric states, we can distinguish three cases:

Case 1: All quantum numbers are different ($\alpha_i\neq\alpha_j$ for all $i\neq j$). The symmetric state is:

\begin{equation*}
\phi_s=\frac{1}{\sqrt{N!}}\sum_P P\Psi_{\alpha_1}(\vec{x}_1)\Psi_{\alpha_2}(\vec{x}_2)\ldots\Psi_{\alpha_N}(\vec{x}_N) \tag{14.21}
\end{equation*}

Case 2: All quantum numbers are identical ($\alpha_1=\alpha_2=\ldots=\alpha_N=\alpha$). The state is already symmetric:

\begin{equation*}
\phi_s=\Psi_{\alpha}(\vec{x}_1)\Psi_{\alpha}(\vec{x}_2)\ldots\Psi_{\alpha}(\vec{x}_N) \tag{14.22}
\end{equation*}

Case 3: The general case with $r<N$ distinct quantum numbers. If there are $N_1$ particles with quantum number $\alpha_1$, $N_2$ with $\alpha_2$, etc., the symmetric state is:

\begin{equation*}
\phi_s=\frac{1}{\sqrt{Q}}\sum_P^* P\left[\prod_{k_1=1}^{N_1}\Psi_{\alpha_1}(\vec{x}_{k_1})\prod_{k_2=N_1+1}^{N_1+N_2}\Psi_{\alpha_2}(\vec{x}_{k_2})\ldots\prod_{k_r=N_1+N_2+\ldots N_{r-1}+1}^{N_1+N_2+\ldots N_r}\Psi_{\alpha_r}(\vec{x}_{k_r})\right] \tag{14.23}
\end{equation*}

The normalization factor $Q$ is a multinomial coefficient:

\begin{equation*}
Q=\binom{N}{N_1,N_2,\ldots,N_r}=\frac{N!}{N_1!N_2!\ldots N_r!} \tag{14.24}
\end{equation*}

\subsection*{14.3.2 Antisymmetric wavefunctions}
For the simplest case of $N=2$, the antisymmetric wavefunction is:

\begin{equation*}
\phi_a(\vec{x}_1,\vec{x}_2)=\frac{1}{\sqrt{2!}}(\Psi_{\alpha_1}(\vec{x}_1)\Psi_{\alpha_2}(\vec{x}_2)-\Psi_{\alpha_1}(\vec{x}_2)\Psi_{\alpha_2}(\vec{x}_1)) \tag{14.25}
\end{equation*}

This can be expressed as a determinant:

\[
\phi_a=\frac{1}{\sqrt{2!}}\left|\begin{array}{ll}
\Psi_{\alpha_1}(\vec{x}_1) & \Psi_{\alpha_1}(\vec{x}_2) \tag{14.26}\\
\Psi_{\alpha_2}(\vec{x}_1) & \Psi_{\alpha_2}(\vec{x}_2)
\end{array}\right|
\]

For the general case of $N$ particles, the antisymmetric wavefunction is the Slater determinant:

\[
\phi_a=\frac{1}{\sqrt{N!}}\left|\begin{array}{cccc}
\Psi_{\alpha_1}(\vec{x}_1) & \Psi_{\alpha_1}(\vec{x}_2) & \ldots & \Psi_{\alpha_1}(\vec{x}_N) \tag{14.27}\\
\Psi_{\alpha_2}(\vec{x}_1) & \Psi_{\alpha_2}(\vec{x}_2) & \ldots & \Psi_{\alpha_2}(\vec{x}_N)\\
\vdots & \vdots & \ddots & \vdots\\
\Psi_{\alpha_N}(\vec{x}_1) & \Psi_{\alpha_N}(\vec{x}_2) & \ldots & \Psi_{\alpha_N}(\vec{x}_N)
\end{array}\right|
\]


From the Slater determinant representation of antisymmetric wavefunctions, we can immediately identify several key properties:

\begin{itemize}
  \item If any two quantum numbers are equal ($\alpha_i=\alpha_j$ for any $i\neq j$), the determinant contains two identical rows and therefore equals zero
  \item If any two position coordinates are equal ($\vec{x}_i=\vec{x}_j$), the determinant contains two identical columns and equals zero
  \item Exchanging any two positions ($\vec{x}_i \leftrightarrow \vec{x}_j$) is equivalent to swapping columns in the determinant, which multiplies the result by $-1$
\end{itemize}

\subsection*{14.3.3 Spin-Statistics Theorem}
The Spin-Statistics Theorem (also called the symmetrization principle) establishes a profound connection between a particle's intrinsic spin and its quantum statistical behavior:

Theorem 14.1 (Spin-Statistics Theorem): A system of identical particles with integer spin is characterized by symmetric wavefunctions. These particles are called bosons and obey Bose-Einstein statistics.

A system of identical particles with half-integer spin is characterized by antisymmetric wavefunctions. These particles are called fermions and obey Fermi-Dirac statistics.

Examples include:
\begin{itemize}
  \item Spin 1: photon, Z boson, W boson, gluon (all bosons)
  \item Spin $1/2$: quarks up, charm, top ($q=2/3$) (all fermions)
  \item Spin $1/2$: quarks down, strange, bottom ($q=-1/3$) (all fermions)
  \item Spin $1/2$: Leptons (electrons, muons, tau) (all fermions)
\end{itemize}

Even composite particles exhibit bosonic or fermionic behavior based on their total spin. For example, $^{87}$Rb has 37 protons and 50 neutrons. The total spin contribution is:

\begin{equation*}
\text{Spin}=\frac{1}{2}(37+37+50)=\text{integer} \tag{14.28}
\end{equation*}

Therefore, $^{87}$Rb is a boson.

Similarly, for $^{41}$K with 19 protons:

\begin{equation*}
\text{Spin}=\frac{1}{2}(19+19+22)=\text{integer} \tag{14.29}
\end{equation*}

Making it a boson. However, $^{40}$K, also with 19 protons:

\begin{equation*}
\text{Spin}=\frac{1}{2}(19+19+21)=\text{half-integer} \tag{14.30}
\end{equation*}

Is therefore a fermion.

\subsection*{14.4 Helium atom}
For the helium atom, the Hamiltonian can be written as:

\begin{equation*}
H=\sum_j(\frac{p_j^2}{2m_e}+V(r_j)) \tag{14.31}
\end{equation*}

Where $V(r_j)$ is the Coulomb potential. We must also include the electron-electron interaction term:

\begin{equation*}
U(r_1,r_2)=\frac{e^2}{|\vec{r}_1-\vec{r}_2|} \tag{14.32}
\end{equation*}

This interaction can be treated as a perturbation. Although there is no explicit spin interaction term, we must account for the fermionic nature of electrons. We begin by finding the eigenstates of the unperturbed Hamiltonian:

\begin{equation*}
H=\sum_j(\frac{p_j^2}{2m_e}+\frac{2e^2}{r_j})=2H_0 \tag{14.33}
\end{equation*}

This represents two hydrogen-like atoms with $Z=2$. A first approach might be to define:

\begin{align*}
&u(r_j)=\Psi_{n\ell m} \tag{14.34}\\
&v(r_j)=\Psi_{n'\ell'm'}
\end{align*}

And construct symmetric and antisymmetric spatial wavefunctions:

\begin{align*}
&\Psi_s=\frac{1}{\sqrt{2}}(u(\vec{r}_1)v(\vec{r}_2)+v(\vec{r}_1)u(\vec{r}_2)) \tag{14.35}\\
&\Psi_a=\frac{1}{\sqrt{2}}(u(\vec{r}_1)v(\vec{r}_2)-v(\vec{r}_1)u(\vec{r}_2))
\end{align*}

However, this approach is incomplete because we need to account for electron spin. The complete wavefunction must include spin terms:

\begin{equation*}
\phi=\chi_{j,m_j}\Psi \tag{14.36}
\end{equation*}


The complete wavefunction must include the spin part, which comes from combining the spins of the two electrons. For total spin $j=0$, only $m_j=0$ is possible:

\begin{equation*}
\chi_{0,0}=\frac{1}{\sqrt{2}}(|s_1,+1/2\rangle|s_2,-1/2\rangle-|s_2,+1/2\rangle|s_1,-1/2\rangle) \tag{14.37}
\end{equation*}

This spin state is antisymmetric under particle exchange. For total spin $j=1$, we have three possible states with $m_j=-1,0,1$:

\begin{align*}
&\chi_{1,1}=|s_1,+1/2\rangle|s_2,+1/2\rangle\\
&\chi_{1,0}=\frac{1}{\sqrt{2}}(|s_1,+1/2\rangle|s_2,-1/2\rangle+|s_2,+1/2\rangle|s_1,-1/2\rangle) \tag{14.38}\\
&\chi_{1,-1}=|s_1,-1/2\rangle|s_2,-1/2\rangle
\end{align*}

All these $j=1$ states are symmetric under particle exchange. We can now construct two sets of wavefunctions:

\[
\left\{\begin{array}{l}
\phi_1=\chi_{1,\mu}\Psi_a \tag{14.39}\\
\phi_2=\chi_{0,0}\Psi_s
\end{array}\right.
\quad
\left\{\begin{array}{l}
\phi_3=\chi_{1,\mu}\Psi_s\\
\phi_4=\chi_{0,0}\Psi_a
\end{array}\right.
\]

The first set consists of antisymmetric total wavefunctions (required for fermions), formed by combining antisymmetric spatial functions with symmetric spin functions, or symmetric spatial functions with antisymmetric spin functions.

The action of the unperturbed Hamiltonian on these states gives:

\begin{align*}
H\phi_2&=(H_1+H_2)\chi_{0,0}\Psi_s=\chi_{0,0}(H_1+H_2)\frac{1}{\sqrt{2}}(u(\vec{r}_1)v(\vec{r}_2)+v(\vec{r}_1)u(\vec{r}_2))\\
&=\chi_{0,0}\frac{1}{\sqrt{2}}(H_1u(\vec{r}_1)v(\vec{r}_2)+v(\vec{r}_1)H_1u(\vec{r}_2)+u(\vec{r}_1)H_2v(\vec{r}_2)+H_2v(\vec{r}_1)u(\vec{r}_2))\\
&=\chi_{0,0}\frac{1}{\sqrt{2}}(E_nu(\vec{r}_1)v(\vec{r}_2)+E_nv(\vec{r}_1)u(\vec{r}_2)+E_{n'}u(\vec{r}_1)v(\vec{r}_2)+E_{n'}v(\vec{r}_1)u(\vec{r}_2))\\
&=\chi_{0,0}(E_n\Psi_s+E_{n'}\Psi_s)=(E_n+E_{n'})\phi_2 \tag{14.40}
\end{align*}

A similar result holds for $\phi_1$. Now we apply perturbation theory to find the energy corrections due to electron-electron interactions. We denote the perturbed states as:

\[
\begin{array}{lll}
E_{OH} & \text{orthohelium} & |\uparrow\rangle|\uparrow\rangle \\
E_{PH} & \text{parahelium} & |\uparrow\rangle|\downarrow\rangle \tag{14.41}
\end{array}
\]

The first-order energy corrections are:

\begin{align*}
&E_{OH}=E_n+E_{n'}+(\phi_1,U\phi_1)\\
&E_{PH}=E_n+E_{n'}+(\phi_2,U\phi_2) \tag{14.42}
\end{align*}

Calculating the perturbation term for $\phi_1$:

\begin{align*}
(\phi_1,U\phi_1)&=\underbrace{(\chi_{1,\mu},\chi_{1,\mu})}_{=1}(\Psi_a,U\Psi_a)=\int d^3x_1\int d^3x_2\Psi_a^*U\Psi_a\\
&=\int d^3x_1\int d^3x_2\frac{1}{\sqrt{2}}(u^*(\vec{r}_1)v^*(\vec{r}_2)-v^*(\vec{r}_1)u^*(\vec{r}_2))U(\vec{r}_1,\vec{r}_2)\frac{1}{\sqrt{2}}(u(\vec{r}_1)v(\vec{r}_2)-v(\vec{r}_1)u(\vec{r}_2))\\
&=\frac{1}{2}\int d^3x_1\int d^3x_2(u^*(\vec{r}_1)v^*(\vec{r}_2)u(\vec{r}_1)v(\vec{r}_2)-u^*(\vec{r}_1)v^*(\vec{r}_2)v(\vec{r}_1)u(\vec{r}_2)\\
&-v^*(\vec{r}_1)u^*(\vec{r}_2)u(\vec{r}_1)v(\vec{r}_2)+v^*(\vec{r}_1)u^*(\vec{r}_2)v(\vec{r}_1)u(\vec{r}_2))U(\vec{r}_1,\vec{r}_2)\\
&=\frac{1}{2}\int d^3x_1\int d^3x_2(|u(\vec{r}_1)|^2|v(\vec{r}_2)|^2-u^*(\vec{r}_1)u(\vec{r}_2)v^*(\vec{r}_2)v(\vec{r}_1)\\
&-u^*(\vec{r}_2)u(\vec{r}_1)v^*(\vec{r}_1)v(\vec{r}_2)+|v(\vec{r}_1)|^2|u(\vec{r}_2)|^2)U(\vec{r}_1,\vec{r}_2) \tag{14.43}
\end{align*}

This can be written as:

\begin{equation*}
(\phi_1,U\phi_1)=\int d^3x_1\int d^3x_2(|u(\vec{r}_1)|^2|v(\vec{r}_2)|^2)U(\vec{r}_1,\vec{r}_2)-\int d^3x_1\int d^3x_2(u^*(\vec{r}_1)u(\vec{r}_2)v^*(\vec{r}_2)v(\vec{r}_1))U(\vec{r}_1,\vec{r}_2) \tag{14.44}
\end{equation*}

We define these integrals as:

\begin{align*}
&I_C=\int d^3x_1\int d^3x_2(|u(\vec{r}_1)|^2|v(\vec{r}_2)|^2)U(\vec{r}_1,\vec{r}_2)\\
&I_Q=\int d^3x_1\int d^3x_2(u^*(\vec{r}_1)u(\vec{r}_2)v^*(\vec{r}_2)v(\vec{r}_1))U(\vec{r}_1,\vec{r}_2) \tag{14.45}
\end{align*}

The first integral $I_C$ represents the classical Coulomb interaction, while the second integral $I_Q$ is a purely quantum mechanical exchange term. The energy corrections are:

\begin{align*}
&E_{OH}=E_n+E_{n'}+I_C-I_Q\\
&E_{PH}=E_n+E_{n'}+I_C+I_Q \tag{14.46}
\end{align*}


For the ground state of helium, we might initially expect the orthohelium state to have lower energy. However, when $n=n'$ (both electrons in the same orbital), the antisymmetric spatial wavefunction $\Psi_a=0$, meaning $I_C=I_Q=0$ for $\phi_1$.

In contrast, for $\phi_2$, which has a symmetric spatial wavefunction $\Psi_s$ and antisymmetric spin function, we have $I_C=I_Q\neq 0$. Therefore, at the ground state:

\begin{equation*}
E_0=E_{PH}=2E_n+2I_C \tag{14.47}
\end{equation*}

This confirms that the helium atom ground state has both electrons in the same orbital with opposite spins (parahelium), consistent with experimental observations.

\section*{15 Quantization of the electromagnetic field}
For electromagnetic waves in vacuum ($\rho=0$ and $\vec{J}=0$), we have:

\begin{align*}
\vec{E}&=\frac{\vec{D}}{\varepsilon_0} \tag{15.1}\\
\vec{B}&=\mu_0\vec{H}
\end{align*}

In the Coulomb gauge, we set $\vec{\nabla}\cdot\vec{A}=0$, which gives:

\begin{align*}
\vec{E}&=-\partial_t\vec{A}\\
\vec{B}&=\vec{\nabla}\times\vec{A} \tag{15.2}
\end{align*}

From Maxwell's fourth equation, we derive:

\begin{align*}
&\vec{\nabla}\times\vec{B}-\frac{1}{c^2}\partial_t\vec{E}=0\\
&\vec{\nabla}\times\vec{\nabla}\times\vec{A}+\frac{1}{c^2}\partial_t^2\vec{A}=0\\
&\vec{\nabla}(\vec{\nabla}\cdot\vec{A})-\nabla^2\vec{A}+\frac{1}{c^2}\partial_t^2\vec{A}=0 \tag{15.3}\\
&\nabla^2\vec{A}-\frac{1}{c^2}\partial_t^2\vec{A}=0
\end{align*}

This confirms that $\vec{A}$ satisfies the wave equation and can be expressed as a superposition of plane waves:

\begin{equation*}
\vec{A}=\sum_{\vec{k}}F_{\vec{k}}(t)e^{i\vec{k}\cdot\vec{x}}\vec{u}_{\vec{k}} \tag{15.4}
\end{equation*}

Each term must satisfy the wave equation:

\[
\begin{array}{r}
\nabla^2\vec{A}=-\sum_{\vec{k}}F_{\vec{k}}(t)\vec{k}^2 e^{i\vec{k}\cdot\vec{x}}\vec{u}_{\vec{k}}\\
\partial_t^2\vec{A}=\sum_{\vec{k}}\ddot{F}_{\vec{k}}(t)e^{i\vec{k}\cdot\vec{x}}\vec{u}_{\vec{k}} \tag{15.5}
\end{array}
\]

Therefore, each component must satisfy:

\begin{align*}
-F_k k^2-\frac{1}{c^2}\ddot{F}_k&=0 \tag{15.6}\\
\ddot{F}_k+c^2k^2 F_k&=0
\end{align*}

This is the equation of a harmonic oscillator with solution:

\begin{equation*}
F_k=Be^{i\omega_k t}+Ce^{-i\omega_k t} \tag{15.7}
\end{equation*}

Where $\omega_k=c|\vec{k}|$. Accounting for polarization, the complete vector potential is:

\begin{equation*}
\vec{A}=\sum_{\vec{k}}\sum_s F_{s\vec{k}}(t)e^{i\vec{k}\cdot\vec{x}}\vec{u}_{s\vec{k}} \tag{15.8}
\end{equation*}


This is a Fourier series representation. Since $\vec{A}$ must be real, we have $\vec{A}=\vec{A}^*$, which gives us:

\begin{align*}
&\sum_{\vec{k}}\sum_s(B_{sk}e^{i\omega_k t}+C_{sk}e^{-i\omega_k t})e^{i\vec{k}\cdot\vec{x}}\vec{u}_{s\vec{k}}=\\
&=\sum_{\vec{k}}\sum_s(B_{sk}^*e^{-i\omega_k t}+C_{sk}^*e^{i\omega_k t})e^{-i\vec{k}\cdot\vec{x}}\vec{u}_{s\vec{k}} \tag{15.9}
\end{align*}

This leads to the following conditions:

\begin{align*}
&B_{s,-\vec{k}}=C_{s,\vec{k}}^* \tag{15.10}\\
&C_{s,-\vec{k}}=B_{s,\vec{k}}^*
\end{align*}

The electromagnetic field Hamiltonian is given by:

\begin{equation*}
\mathcal{H}=\int d^3r\frac{1}{2}\left(\varepsilon_0\vec{E}^2+\frac{1}{\mu_0}\vec{B}^2\right) \tag{15.11}
\end{equation*}

Substituting $\vec{E}=-\partial_t\vec{A}$ and $\vec{B}=\vec{\nabla}\times\vec{A}$ and using our expression for $\vec{A}$, we can simplify this to:

\begin{equation*}
\sum_s\sum_{\vec{k}}\hbar\omega_k|b_{s\vec{k}}|^2 \tag{15.12}
\end{equation*}

Where $b_{s\vec{k}}=d_kB_{sk}$ and $d_k=\sqrt{2\varepsilon_0\omega_kV/\hbar}$. The Poisson brackets are:

\begin{equation*}
\{b_{s\vec{k}},b_{r\vec{q}}^*\}=\delta_{sr}\delta_{\vec{k}\vec{q}}\frac{1}{i\hbar} \tag{15.13}
\end{equation*}

We can define a set of canonical Poisson brackets:

\begin{equation*}
\{A,B\}=\sum_s\sum_{\vec{k}}\frac{\partial A}{\partial b_{s\vec{k}}}\frac{\partial B}{\partial b_{s\vec{k}}^*}-\frac{\partial B}{\partial b_{s\vec{k}}}\frac{\partial A}{\partial b_{s\vec{k}}^*} \tag{15.14}
\end{equation*}

Verifying these brackets:

\begin{align*}
\dot{b}_{s\vec{k}}&=\{b_{s\vec{k}},H\}=\sum_r\sum_{\vec{q}}\hbar\omega_q\{b_{s\vec{k}},b_{r\vec{q}}^*b_{r\vec{q}}\}\\
&=\sum_r\sum_{\vec{q}}\hbar\omega_q(b_{r\vec{q}}^*\{b_{s\vec{k}},b_{r\vec{q}}\}+b_{r\vec{q}}\underbrace{\{b_{s\vec{k}},b_{r\vec{q}}^*\}}_{\delta_{sr}\delta_{\vec{k}\vec{q}}\frac{1}{i\hbar}}) \tag{15.15}\\
&=-i\omega_kb_{s\vec{k}}
\end{align*}

The second derivative is:

\begin{equation*}
\ddot{b}_{s\vec{k}}=\frac{d\dot{b}_{s\vec{k}}}{dt}=\frac{d}{dt}(-i\omega_kb_{s\vec{k}})=-i\omega_k\dot{b}_{s\vec{k}}=-\omega_k^2b_{s\vec{k}} \tag{15.16}
\end{equation*}

Which means:

\begin{equation*}
\ddot{b}_{s\vec{k}}+\omega_k^2b_{s\vec{k}}=0 \tag{15.17}
\end{equation*}

\subsection*{15.1 Field quantization and Fock states}
Now we apply the canonical quantization rules (CQR) and obtain:

\begin{equation*}
[b_{sk},b_{rq}^\dagger]=\delta_{rs}\delta_{qk} \tag{15.18}
\end{equation*}

We recognize that the operators $b_{sk}, b_{sk}^\dagger$ form a set similar to the ladder operators $a, a^\dagger$ for the harmonic oscillator. We can write $\vec{A}$ as an operator:

\begin{equation*}
\vec{A}=\sum_{\vec{k}}\sum_s(B_{sk}e^{i\omega_k t}+B_{sk}^\dagger e^{-i\omega_k t})e^{i\vec{k}\cdot\vec{x}}\vec{u}_{s\vec{k}} \tag{15.19}
\end{equation*}

Where the operators $B_{sk}$ and $B_{sk}^\dagger$ are related to $b_{sk}, b_{sk}^\dagger$. The Hamiltonian becomes:

\begin{equation*}
H=\sum_s\sum_{\vec{k}}\hbar\omega_k\frac{1}{2}(b_{sk}^\dagger b_{sk}+b_{sk}b_{sk}^\dagger)=\sum_s\sum_{\vec{k}}\hbar\omega_k(b_{sk}^\dagger b_{sk}+\frac{1}{2}) \tag{15.20}
\end{equation*}


This Hamiltonian is essentially a sum of independent harmonic oscillators. Since it is separable, we can factorize the wavefunction into what are called Fock states:

\begin{equation*}
|n\rangle=\prod_s\prod_k|n_{sk}\rangle \tag{15.21}
\end{equation*}

Where we define the number operator:

\begin{equation*}
b_{sk}^\dagger b_{sk}=\hat{n}_{sk} \tag{15.22}
\end{equation*}

Acting on the Fock states:

\begin{equation*}
\hat{n}_{sk}|n_{sk}\rangle=n_{sk}|n_{sk}\rangle \tag{15.23}
\end{equation*}

The creation and annihilation operators act as:

\begin{align*}
&b_{sk}^\dagger|n_{sk}\rangle=\sqrt{n_{sk}+1}|n_{sk}+1\rangle \tag{15.24}\\
&b_{sk}|n_{sk}\rangle=\sqrt{n_{sk}}|n_{sk}-1\rangle
\end{align*}

The Hamiltonian applied to a Fock state gives:

\begin{equation*}
H|n\rangle=\sum_s\sum_{\vec{k}}\hbar\omega_k(\hat{n}_{sk}+\frac{1}{2})\prod_r\prod_q|n_{rq}\rangle \tag{15.25}
\end{equation*}

Since $\hat{n}_{sk}$ only acts on its corresponding state:

\begin{align*}
H|n\rangle&=\sum_s\sum_{\vec{k}}(\prod_{r\neq s}\prod_{q\neq k}|n_{rq}\rangle)\hbar\omega_k(n_{sk}+\frac{1}{2})|n_{sk}\rangle\\
&=(\prod_s\prod_k|n_{sk}\rangle)\underbrace{\sum_s\sum_{\vec{k}}\hbar\omega_k(n_{sk}+\frac{1}{2})}_E \tag{15.26}
\end{align*}

The quantum number $n_{sk}$ represents the number of photons with wavevector $k$ and polarization $s$. Importantly, even with zero photons ($n_{sk}=0$ for all $s,k$), the energy $E\neq 0$ due to the vacuum energy contribution $\frac{1}{2}\hbar\omega_k$ for each mode.

We can define the total photon number operator:

\begin{equation*}
\hat{N}=\sum_s\sum_{\vec{k}}\hat{n}_{sk} \tag{15.27}
\end{equation*}

Which gives:

\begin{equation*}
\hat{N}|n\rangle=N|n\rangle \tag{15.28}
\end{equation*}

Where $N$ is the total number of photons. We can verify that:

\begin{equation*}
[\hat{N},H]=0 \tag{15.29}
\end{equation*}

This commutation relation confirms that the total number of photons is conserved in free electromagnetic field evolution.

\backmatter
\end{document}


\end{document}
