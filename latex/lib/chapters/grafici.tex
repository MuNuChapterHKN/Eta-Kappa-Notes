\chapter{Grafici}
Qui vi mostro semplicemente come importare i grafici in formato .svg, aggiungerci una didascalia e dimensionarli correttamente.

\begin{figure}[!h]
	\centering
	% Importa l'immagine SVG, applicando il ritaglio e il ridimensionamento
	\includesvg[width=0.8\textwidth,inkscapelatex=false, ]{res/svg/esempio_geogebra}
	\caption{Questo è il grafico di esempio che ho creato con Geogebra, con una didascalia e una label associata. Per poter importare i grafici svg, dovete installare inkscape, aggiungerlo al path di sistema (occhio al path, non funziona niente se non lo fai, di solito è \texttt{C:/Program Files/Inkscape/bin}), modificare il flag del compilatore aggiungendo \texttt{-shell-escape}. Se sei su overleaf non serve, lo fa in automatico (beati voi che non avete perso 1h a capire ste cose, per ripicca scrivo le cose più importanti nelle didascalie xD).}
	\label{graph:esempiografico}
\end{figure}

\hfill

\begin{figure}[!h]
	\centering
	% Importa l'immagine SVG, applicando il ritaglio e il ridimensionamento
	\includesvg[width=0.8\textwidth,inkscapelatex=false, ]{res/svg/2d_example}
	\caption{Questo è il grafico 2d che ho creato con matplotlib, vedi script.}
	\label{graph:esempiograficopy}
\end{figure}

\hfill

\begin{figure}[!h]
	\centering
	% Importa l'immagine SVG, applicando il ritaglio e il ridimensionamento
	\includesvg[width=0.8\textwidth,inkscapelatex=false, ]{res/svg/3d_continuous_example}
	\caption{Questo è il grafico 3d continuo che ho creato con matplotlib, vedi script.}
	\label{graph:esempiograficopy2}
\end{figure}

\hfill

\begin{figure}[!h]
	\centering
	% Importa l'immagine SVG, applicando il ritaglio e il ridimensionamento
	\includesvg[width=0.8\linewidth,inkscapelatex=false]{res/svg/3d_wireframe_example}
	\caption{Questo è il grafico 3d continuo che ho creato con matplotlib, vedi script.}
	\label{graph:esempiograficopy3}
\end{figure}

\hfill
\begin{figure}[!h]
	\centering
	% Importa l'immagine SVG, applicando il ritaglio e il ridimensionamento
	\includesvg[width=0.8\linewidth,inkscapelatex=false]{src/drawio/esempioMappaConcettuale.drawio}
	\caption{Una mappa concettuale dummy creata con draw.io online editor.}
	\label{graph:esempioMappaConcettuale}
\end{figure}

\hfill
\begin{figure}[!h]
	\centering
	% Importa l'immagine SVG, applicando il ritaglio e il ridimensionamento
	\includesvg[width=0.8\linewidth,inkscapelatex=false, ]{src/drawio/esempioCircuito.drawio}
	\caption{Un circuito basilare creato con draw.io online editor .}
	\label{graph:esempioCircuito}
\end{figure}

\hfill

Se aprite il .tex osserverete che ho inserito l'istruzione \texttt{\textbackslash href} tra un grafico e l'altro. Questo per evitare che \LaTeX ottimizzi lo spazio frammentando l'elenco puntato qui sotto tra un grafico e l'altro. Non mi credete? fate una prova, e vedrete che casino fa.

Le label possono essere associate a tantissimi oggetti in matlab, fatene buon uso! Soprattutto quando nel vostro testo dovete citare formule, tabelle o grafici. Eccovi tutti i modi in cui è possibile citare un oggetto con una etichetta associata:
\begin{enumerate}
	\item \texttt{\textbackslash ref\{graph:esempiografico\}}: Per fare riferimento al numero associato all'etichetta, come una figura, tabella, sezione, ecc. (esempio: Figura~\ref{graph:esempiografico}).
	\item \texttt{\textbackslash pageref\{graph:esempiografico\}}: Per fare riferimento al numero di pagina in cui appare l'oggetto etichettato (esempio: Vedi pagina~\pageref{graph:esempiografico}).
	\item \texttt{\textbackslash hyperref\{graph:esempiografico\}}: Se usi il pacchetto \texttt{hyperref}, puoi fare clic sul riferimento per andare direttamente alla figura, tabella, ecc.
	\item \texttt{\textbackslash eqref\{graph:esempiografico\}}: Per fare riferimento a un'equazione, se l'etichetta è stata applicata a un ambiente di equazione (esempio: come mostrato nell'equazione~\eqref{graph:esempiografico}).
\end{enumerate}