\chapter{Introduzione}

In questo file, che ha la stessa struttura degli appunti che dovrete scrivere, vi mostro l'utilizzo base di alcuni pacchetti e vi spiego la struttura del progetto.

\section{Struttura del progetto}

Nel progetto ci sono diverse cartelle:
\begin{itemize}
	\item \texttt{chapters}, dove inserirete per ogni capitolo un file .tex con lo stesso nome (potete anche nominarli capitolo 1, 2, ma fidatevi che se li nominate in modo sensato vi verrà molto più utile riordinarli in seguito);
	\item \texttt{src}, questa cartella conterrà gli eventuali file sorgente di tipo \texttt{.gbb} se provenienti da Geogebra e  \texttt{.py} se invece volete usare python per la creazione dei grafici. Ogni volta che create un grafico, dovrete salvare il file sorgente nella cartella corrispondente. Verranno aggiunte altre cartelle in base ai futuri strumenti scelti per la generazione di grafici o altre risorse. 
	\begin{itemize}
		\item \texttt{gbb}, contiene il file sorgente di Geogebra. Deve essere salvato per garantire in futuro di poter modificare il grafico corrispondente.
		\item \texttt{py}, vi ho preparato alcuni script .py per valutare la potenzialità dell'utilizzo di matplotlib per la creazione dei vostri grafici. All'interno, troverete anche un file nominato \texttt{run\_all\_scripts.py} il quale vi permette di runnare tutti gli altri file .py nella cartella e quindi generare automaticamente tutti i grafici. Notare che questa soluzione vi evita il problema di salvare i grafici, infatti i vari esempi che potrete prendere come dei template di base per i vostri grafici salvano automaticamente nella cartella \texttt{./res/svg} il grafico, con lo stesso nome del file .py e in formato svg.
		\item \texttt{draw.io}:draw.io è un potentissimo strumento che vi permette di creare in modo intuitivo e grafico schemi e circuiteria elettrica varia. In questo caso specifico, non serve che salviate sia il sorgente che l'svg, ma solo il file in formato \texttt{.drawio.svg}, perché usabile sia da Latex come svg che da draw.io per future modifiche.
	\end{itemize}
	\item \texttt{res}, qui è dove verranno messe tutte le risorse che utilizzate all'interno del vostro blocco appunti, come eventuali svg ottenuti da Geogebra o Python, png (tenetela come ultima risorsa). Potete creare una cartella per ogni formato file.
	\item \texttt{config}, in questa cartella verranno salvati alcuni file .tex e forse .sty per la formattazione del documento, l'importazione dei pacchetti e la definizione di comandi personalizzati. Vi invito a dare una occhiata al pacchetto settings.tex, che contiene i comandi personalizzati e i pacchetti.
	\item \texttt{template.tex}, qui è dove avviene la magia, di base voi dovete occuparvi solo dell'ordinare i capitoli nel modo in cui più vi aggrada, eventualmente dovrete modificare altre piccole cose che vi verranno segnalate man mano che il progetto prende forma, ovvero l'inserimento delle vostre informazioni qualora vorreste essere accreditati e una breve spiegazione delle modifiche apportate al documento.
	\item altri file di cui non dovete preoccuparvi, sono generati automaticamente dal compilatore. Tra questi di rilevante ci sarà solo \texttt{main.pdf} o come avete voluto chiamare il main, che sarà il file .pdf compilato.
\end{itemize}