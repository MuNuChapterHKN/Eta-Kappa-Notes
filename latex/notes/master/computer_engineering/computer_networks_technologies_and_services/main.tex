% Packages
\usepackage{listings}                    % Code highlighting
\usepackage{xcolor}                      % Custom colors
\usepackage{longtable}                   % Breakable tables
\usepackage{ulem}                        % Underline
\usepackage{contour}                     % Border around text
\usepackage{tcolorbox}                   % Custom boxes

% Primary (Accent) Colors
% Primary (Accent) Colors
\definecolor{accentYellow}{RGB}{254, 196, 41}  % #FEC421
\definecolor{accentRed}{RGB}{236, 45, 36}      % #EC2D24

% Secondary Colors
\definecolor{supportOrange}{RGB}{242, 183, 5}  % #F2B705
\definecolor{supportDarkBlue}{RGB}{55, 81, 113} % #375171

% Background Colors
\definecolor{backgroundLight}{RGB}{242, 242, 242} % #F2F2F2

% Text & Border Colors
\definecolor{textGrayBlue}{RGB}{100, 117, 140}   % #64758C
\definecolor{textGrayMedium}{RGB}{146, 154, 166}  % #929AA6
\definecolor{textGrayLight}{RGB}{184, 187, 191}   % #B8BBBF



% Listings style
\lstdefinestyle{hkn}{
  basicstyle=\ttfamily\small\color{textGrayBlue},                         % Base style (size and font)
  keywordstyle=\bfseries\color{accentRed},           % Keywords in red (important, eye-catching)
  identifierstyle=\color{supportDarkBlue},               % Identifiers in blue (clear distinction)
  commentstyle=\color{textGrayMedium},                 % Comments in gray-blue (less prominent)
  stringstyle=\color{supportOrange},                 % Strings in orange (warm and readable)
  numberstyle=\ttfamily\scriptsize\color{textGrayMedium}, % Line numbers in gray (non-intrusive)
  backgroundcolor=\color{backgroundLight},           % Light background for contrast
  rulecolor=\color{textGrayLight},                   % Soft gray border for structure
  frame=single,                          % Border around code (single, double, shadowbox, none)
  framerule=0.8pt,                       % Border thickness
  frameround=tttt,                       % Round all corners
  framesep=5pt,                          % Distance between border and code
  rulesep=2pt,                           % Distance between border and code line
  numbers=left,                          % Line number position (left, right, none)
  stepnumber=1,                          % Line number interval
  numbersep=10pt,                        % Distance between line numbers and code
  xleftmargin=30pt,                      % Left margin
  xrightmargin=30pt,                     % Right margin
  resetmargins=true,                     % Reset margins
  numberblanklines=false,                % Number blank lines
  firstnumber=auto,                      % Initial line number
  columns=fixed,                         % Fixed column width
  showstringspaces=false,                % Show spaces in strings
  tabsize=2,                             % Tab size
  breaklines=true,                       % Automatic line break for long lines
  breakatwhitespace=true,                % Line break at whitespace
  breakautoindent=true,                  % Automatic indentation after line break
  escapeinside={(*@}{@*)}                % LaTeX commands in code
}

% Underline settings
\renewcommand{\ULdepth}{1.8pt} % Underline depth
\contourlength{0.8pt}

% Custom underline command
\newcommand{\myuline}[1]{%
\uline{\phantom{#1}}%
\llap{\contour{white}{#1}}%
}

% tcolorbox color settings
\definecolor{tcolorboxLeftColor}{RGB}{2, 65, 191}
\definecolor{tcolorboxBackTitleColor}{RGB}{119, 152, 255}
\definecolor{tcolorboxBackColor}{RGB}{210, 226, 255}

% Custom boxes
\newtcolorbox[auto counter, number within=chapter]{definition}[1]{
  title={\iflanguage{italian}{Definizione}{Definition}\par~\arabic{\tcbcounter}.~#1},
  boxrule=0mm,                       % Bordo principale (disabilitato)
  leftrule=1mm,                    % Bordo sinistro principale
  arc=2mm,
  colframe=accentRed,       % Colore bordo
  colbacktitle=textGrayMedium,
  colback=backgroundLight,        % Colore sfondo
  fonttitle=\bfseries,
  rounded corners=all,               % Bordi arrotondati
  }

\newtcolorbox[auto counter, number within=chapter]{theorem}[1]{
  title={\iflanguage{italian}{Teorema}{Theorem}~\arabic{\tcbcounter}.~#1},
  boxrule=0mm,                       % Bordo principale (disabilitato)
  leftrule=1mm,                    % Bordo sinistro principale
  arc=2mm,
  colframe=accentYellow,       % Colore bordo
  colbacktitle=textGrayMedium,
  colback=backgroundLight,        % Colore sfondo
  fonttitle=\bfseries,
  rounded corners=all,               % Bordi arrotondati
}

\newtcolorbox[auto counter, number within=chapter]{corollary}[1]{
  title={\iflanguage{italian}{Corollario}{Corollary}~\arabic{\tcbcounter}.~#1},
  boxrule=0mm,                       % Bordo principale (disabilitato)
  leftrule=1mm,                    % Bordo sinistro principale
  arc=2mm,
  colframe=supportOrange,       % Colore bordo
  colbacktitle=textGrayMedium,
  colback=backgroundLight,        % Colore sfondo
  fonttitle=\bfseries,
  rounded corners=all,               % Bordi arrotondati
}

\newtcolorbox[auto counter, number within=chapter]{exercise}[1]{
  title={\iflanguage{italian}{Esercizio}{Exercise}~\arabic{\tcbcounter}.~#1},
  boxrule=0mm,                       % Bordo principale (disabilitato)
  leftrule=1mm,                    % Bordo sinistro principale
  arc=2mm,
  colframe=supportDarkBlue,       % Colore bordo
  colbacktitle=textGrayMedium,
  colback=backgroundLight,        % Colore sfondo
  fonttitle=\bfseries,
  rounded corners=all,               % Bordi arrotondati
}

\newtcolorbox[auto counter, number within=chapter]{observation}[1]{
  title={\iflanguage{italian}{Osservazione}{Observation}~\arabic{\tcbcounter}.~#1},
  boxrule=0mm,                       % Bordo principale (disabilitato)
  leftrule=1mm,                    % Bordo sinistro principale
  arc=2mm,
  colframe=textGrayBlue,       % Colore bordo
  colbacktitle=textGrayMedium,
  colback=backgroundLight,        % Colore sfondo
  fonttitle=\bfseries,
  rounded corners=all,               % Bordi arrotondati
  }


\title{Computer Networks Technologies and Services}

\maketitle
\tableofcontents

\section{IPv4 Addressing and Routing}

\subsection{IPv4 Addressing}

IP addresses have the goal to be "nominally unique", that means that they should be unique in the network they are in.

IP addresses are 32 bit identifiers for the \hl{interface}, not the device. The device can have multiple interfaces, and each interface has its own IP address. For example a smartphone can have an IP address for the Wi-Fi interface and another one for the cellular interface.

\subsection{Notation}

\begin{figure}[htb]
  \centering
  \begin{subfigure}[b]{0.45\textwidth}
    \centering
    \includesvg[width=\textwidth,inkscapelatex=false]{res/svg/datagram.drawio}
    \caption{IP Datagram}
  \end{subfigure}
  \hfill
  \begin{subfigure}[b]{0.45\textwidth}
    \centering
    \includesvg[width=\textwidth,inkscapelatex=false]{res/svg/router.drawio}
    \caption{Router - Layer 3}
  \end{subfigure}

  \vspace{1em}

  \begin{subfigure}[b]{0.45\textwidth}
    \centering
    \includesvg[width=\textwidth,inkscapelatex=false]{res/svg/switch.drawio}
    \caption{Switch - Layer 2}
  \end{subfigure}
  \hfill
  \begin{subfigure}[b]{0.45\textwidth}
    \centering
    \includesvg[width=\textwidth,inkscapelatex=false]{res/svg/access_point.drawio}
    \caption{Access Point - Layer 2}
  \end{subfigure}
\end{figure}

\begin{figure}
  \centering
  \includesvg[width=0.5\textwidth,inkscapelatex=false]{res/svg/ip_address}
  \caption{IP Address}
\end{figure}

\begin{itemize}
  \item An IP Network is a set of IP devices whose interfaces have
  \begin{itemize}
    \item same network ID
    \item common physical connection (link-layer network)
  \end{itemize}
\end{itemize}

\subsection{Special Addresses}

\begin{figure}[h]
  \begin{minipage}{0.45\textwidth}
    \begin{bytefield}[bitwidth=\linewidth]{32}
      \bitbox{20}{Some value} & \bitbox{12}{All zeros}
    \end{bytefield}
    \caption*{The (sub)network ID}
  \end{minipage}
  \hfill
  \begin{minipage}{0.45\textwidth}
    \begin{bytefield}[bitwidth=\linewidth]{32}
      \bitbox{32}{All ones}
    \end{bytefield}
    \caption*{\hl{Limited}: Broadcast packet becomes a L2 packet}
  \end{minipage}

  \begin{minipage}{0.45\textwidth}
    \begin{bytefield}[bitwidth=\linewidth]{32}
      \bitbox{20}{Some value} & \bitbox{12}{All ones}
    \end{bytefield}
    \caption*{Directed broadcast for network (usually restricted)}
  \end{minipage}
  \hfill
  \begin{minipage}{0.45\textwidth}
    \begin{bytefield}[bitwidth=\linewidth]{32}
      \bitbox{8}{127} & \bitbox{24}{Anything (often ones)}
    \end{bytefield}
    \caption*{Loopback}
  \end{minipage}
  \end{figure}

  \section{Addressing Classes}

  \begin{figure}[h]
    \begin{bytefield}{32}
      \bitheader{0-31}
      \bitbox{1}{\colorbox{yellow!20}{0}} & \bitbox{7}{Network} & \bitbox{24}{Host}
    \end{bytefield}
    \caption{Class A - 128 networks, 16M hosts per network}
  \end{figure}

  \begin{figure}[h]
    \begin{bytefield}{32}
      \bitheader{0-31}
      \bitbox{2}{\colorbox{yellow!20}{10}} & \bitbox{14}{Network} & \bitbox{16}{Host}
    \end{bytefield}
    \caption{Class B - 16K networks, 64K hosts per network}
  \end{figure}

  \begin{figure}[h]
    \begin{bytefield}{32}
      \bitheader{0-31}
      \bitbox{3}{\colorbox{yellow!20}{110}} & \bitbox{21}{Network} & \bitbox{8}{Host}
    \end{bytefield}
    \caption{Class C - 2M networks, 254 hosts per network}
  \end{figure}

  \section{CIDR | Classless Inter-Domain Routing}

  Enables the use of arbitrary length network prefixes. The network prefix is written as \texttt{a.b.c.d/x}, where \texttt{x} is the number of bits in the network prefix.

  The network prefix is also referred to as the \colorbox{yellow!20}{subnet mask} or \colorbox{yellow!20}{netmask}. The netmask is a 32 bit value that has \texttt{x} bits set to 1 and the rest set to 0.

  \begin{itemize}
    \item \texttt{200.23.16.0 / 23}: prefix length notation
    \item \texttt{200.23.16.0   255.255.254.0}: netmask notation
  \end{itemize}

  \subsection{Table of Valid Netmask}

  \begin{table}[h]
    \centering
    \begin{tabular}{lll}
      \hline
      Netmask & Prefix & Number of Usable IDs \\
      \hline
      \texttt{255.255.255.0} & \texttt{/24} & 254 \\
      \texttt{255.255.255.128} & \texttt{/25} & 126 \\
      \texttt{255.255.255.192} & \texttt{/26} & 62 \\
      \texttt{255.255.255.224} & \texttt{/27} & 30 \\
      \texttt{255.255.255.240} & \texttt{/28} & 14 \\
      \texttt{255.255.255.248} & \texttt{/29} & 6 \\
      \texttt{255.255.255.252} & \texttt{/30} & 2 \colorbox{green!20}{Smallest usable netmask} \\
      \texttt{255.255.255.254} & \texttt{/31} & \colorbox{red!20}{Useless} \\
      \texttt{255.255.255.255} & \texttt{/32} & \colorbox{yellow!20}{Represents the single device} \\
      \hline
    \end{tabular}
    \caption{Valid Netmask Table}
  \end{table}

\end{document}

