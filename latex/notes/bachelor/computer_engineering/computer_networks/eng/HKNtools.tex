%% HKNtools.sty

%% PACCHETTI
\usepackage[italian]{babel}              % Lingua italiana
\usepackage{listings}                    % Per formattare il codice
\usepackage{xcolor}                      % Supporto per i colori
\usepackage{graphicx}                    % Per includere immagini
\usepackage{longtable}                   % Per tabelle su più pagine
\usepackage{amsmath, amssymb}            % Per le espressioni matematiche
\usepackage{svg}                         % Per includere grafici SVG

%% IMPOSTAZIONI DEI COLORI
\definecolor{sqlBackground}{RGB}{240, 240, 240}     % Sfondo chiaro
\definecolor{sqlText}{RGB}{0, 0, 0}                 % Testo principale nero
\definecolor{sqlKeyword}{RGB}{0, 0, 150}            % Parole chiave in blu
\definecolor{sqlComment}{RGB}{0, 128, 0}            % Commenti in verde
\definecolor{sqlString}{RGB}{200, 0, 0}             % Stringhe in rosso
\definecolor{sqlNumber}{RGB}{100, 100, 100}         % Numeri di riga in grigio
\definecolor{sqlIdentifier}{RGB}{0, 0, 100}         % Identificatori in blu scuro

%% COMANDI PERSONALIZZATI
% Definizione dello stile per SQL
\lstdefinestyle{sqlstyle}{
	language=SQL,
	backgroundcolor=\color{sqlBackground},      % Sfondo chiaro
	basicstyle=\ttfamily\small\color{sqlText},  % Stile di base (testo nero)
	keywordstyle=\color{sqlKeyword}\bfseries,   % Stile delle parole chiave (blu)
	commentstyle=\color{sqlComment}\itshape,    % Stile dei commenti (verde)
	stringstyle=\color{sqlString},              % Stile delle stringhe (rosso)
	numberstyle=\tiny\color{sqlNumber},         % Stile dei numeri di riga (grigio)
	identifierstyle=\color{sqlIdentifier},      % Stile degli identificatori (blu scuro)
	breaklines=true,                            % Wrapping delle righe
	breakautoindent,							% per l'indentazione automatica in caso di a capo
	frame=single,                               % Bordo attorno al codice
	rulecolor=\color{gray},                     % Colore del bordo
	tabsize=4,                                  % Dimensione del tab
	showstringspaces=false,                     % Nasconde gli spazi nelle stringhe
	numbers=left,                               % Numeri di riga a sinistra
	numbersep=5pt,                              % Distanza tra i numeri e il codice
}

% Altri comandi personalizzati (puoi aggiungerli qui)
% \newcommand{\insertlicense}{\texttt{Licenza...}}


%TODO: aggiungere altri stili per altri codici

%TODO: creare il comando per l'inserimento automatico della licenza

%TODO: 