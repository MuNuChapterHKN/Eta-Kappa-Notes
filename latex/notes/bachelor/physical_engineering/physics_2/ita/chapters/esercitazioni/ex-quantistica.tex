\chapter{Meccanica quantistica -- esercitazioni}

\section{Riassunto della teoria}

Relazioni di de Broglie:
\begin{equation}
    \begin{cases}
        E = h \nu = \hbar \omega \\
        p = \frac{h}{\lambda} = \hbar k
    \end{cases}
\end{equation}

Funzione d'onda di Schrödinger:
\begin{equation}
    \Psi = A \exp{\im (\vt{k} \cdot \p - \omega t)}
    = A \exp{\frac{\im (\vt{p} \cdot \p - Et)}{\hbar}}
\end{equation}

Equazione di Schrödinger completa:
\begin{equation}
    \im \hbar \parder{t} \Psi(\p, t) = - \frac{\hbar^2}{2 m} \lapl \Psi(\p, t) + V(\p, t) \Psi(\p, t)
\end{equation}
Vale con le ipotesi di
\begin{itemize}
    \item particella puntiforme
    \item $v \ll c$
    \item forze conservative
\end{itemize}

Separazione delle variabili:
\begin{equation}
    \Psi(\p, t) = R(\p) T(t)
\end{equation}

$T$ è armonica:
\begin{equation}
    T(t) = \exp{-\im E t/\hbar}
\end{equation}

L'equzione di Schrödinger stazionaria è:
\begin{equation}
\label{eq:schrstaz}
    \hat{H} \Psi(\p) = E \Psi(\p)
\end{equation}

$\hat{H}$ è l'hamiltoniana ed è un operatore lineare, quindi l'\cref{eq:schrstaz} è un'equazione agli autovalori.

È possibile scrivere decomporre $\Psi$ nella base degli autovettori di $\hat{H}$:
\begin{equation}
    \Psi = \sum_i a_i \Psi_i
\end{equation}

Svolgendo la misura, lo stato $\Psi	$ collassa in uno degli autostati:
\begin{gather}
    \hat{H} \Psi_i = E_i \Psi_i \\
    \abs{a_i}^2 = \mathrm{P}(\text{la particella sarà misurata nello stato $\Psi_i$})
\end{gather}

\section{Buca di potenziale}

È un caso di confinamento della particella in una regione di spazio.

In una dimensione,
\begin{equation}
    V(x) = \begin{cases}
        0 & 0 < x < L \\
        +\infty & \text{altrimenti}
    \end{cases}
\end{equation}
Ovvero, la particella è confinata nell'intervallo $(0, L)$.
Poiché non può uscire, $\Psi = 0$ al di fuori dell'intervallo e la rinormalizzazione interesserà solo quest'ultimo.

L'hamiltoniana classica è:
\begin{equation}
    H = \frac{p^2}{2 m} + V(x)
\end{equation}

Per ottenere l'hamiltoniana quantistica, si sostituisce $p$ con $\hat{p} = -\im \hbar \parder{x}$:

\begin{equation}
    \hat{H} = -\frac{\hbar^2}{2 m} \parder[][2]{x} + V(x)
\end{equation}

L'equazione di Schrödinger stazionaria risulta:
\begin{gather}
    -\frac{\hbar^2}{2 m} \parder[][2]{x} \Psi = E \Psi \\
    \parder[][2]{x} \Psi = -\frac{2 m E}{\hbar^2} \Psi
\end{gather}

Solitamente si definisce il fattore $\alpha$:
\begin{equation}
    \alpha \coloneq \frac{\sqrt{2 m E}}{\hbar}
\end{equation}

L'equazione diventa
\begin{equation}
    \parder[][2]{x} \Psi + \alpha^2 \Psi = 0
\end{equation}

$\Psi = A \exp{\lambda x}$, quindi l'equazione caratteristica è:
\begin{equation}
    \lambda^2 + \alpha^2 = 0
    \implies
    \lambda = \pm \im \alpha
\end{equation}

La soluzione generale è
\begin{equation}
    \Psi(x) = A \exp{\im \alpha x} + B \exp{-\im \alpha x}
\end{equation}
ovvero, la somma di un'onda progressiva e una regressiva.

I valori di $A$ e $B$ si determinano imponendo le condizioni al contorno:
\begin{gather}
    \begin{cases}
        \Psi(0) = 0 \implies B = -A \\
        \Psi(L) = 0 \implies A \exp{\im \alpha L} + B \exp{-\im \alpha L} = 0
    \end{cases} \\
    \implies \exp{\im \alpha L} - \exp{-\im \alpha L} = 0 \\
    \implies 2 \im \sin(\alpha L) = 0
    \implies \alpha L = n \pi
\end{gather}

Questo mostra che i valori di energia possibili sono quantizzati:
\begin{equation}
    \boxed{E_n = \frac{\hbar^2 \alpha^2}{2 m} = \frac{\pi^2 \hbar^2}{2 m L^2} n^2, \quad
    n \in \Z^+}
\end{equation}
Si esclude $n = 0$ \textit{a posteriori} poiché l'\cref{eq:autostati_scatola} non può essere la funzione nulla.

Trovati gli autovalori, si determinano gli autostati:
\begin{equation}
    \hat{H} \Psi_n = E_n \Psi_n
\end{equation}

Abbiamo già mostrato che le soluzioni sono nella forma:
\begin{equation}
    \Psi(x) = A \exp{\im \alpha x} - A \exp{-\im \alpha x}
    = 2 \im A \sin(\alpha x)
\end{equation}
Occorre imporre la normalizzazione:
\begin{gather}
    \int_0^L \abs{\Psi(\p)}^2 \de x
    = \int_0^L \abs{2 A \im \sin(\alpha x)}^2 \de x
    = 4 A^2 \int_0^L \sin^2(\alpha x) \de x
    = 2 A^2 L = 1 \\
    \implies
    A = \sqrt{\frac{1}{2 L}}
\end{gather}
È stato usato il fatto che $\sin(\alpha L) = 0$, e quindi $L$ è multiplo del periodo di $\sin^2(\alpha x)$.

Gli autostati, quindi, sono
\begin{equation}
\label{eq:autostati_scatola}
    \Psi_n(x) = \im \sqrt{\frac{2}{L}} \sin\pts{\frac{n \pi}{L} x}
\end{equation}

La densità di probabilità di trovare la particella in $x$ è
\begin{equation}
    \abs{\Psi_n(x)}^2 = \frac{2}{L} \sin^2\pts{\frac{n \pi}{L} x}
\end{equation}

In tre dimensioni, il numero quantico diventa un vettore $\vt{n} = (n_x, n_y, n_z)$ e gli autostati sono
\begin{equation}
    \Psi_n(\p) = 2 A \im \sin\pts{\frac{n_x \pi}{L} x} \sin\pts{\frac{n_y \pi}{L} y} \sin\pts{\frac{n_z \pi}{L} z}
\end{equation}

Livelli energetici:
\begin{equation}
    E_\vt{n} = \frac{\hbar^2}{2 m L^2} \pts{n_x^2 + n_y^2 + n_z^2}
\end{equation}
Sono detti \important{degeneri}, poiché più stati possono corrispondere allo stesso livello energetico.


\subsection{Molecola come buca di potenziale}

Modellizziamo la molecola \ce{H2} come due protoni di carica $e$ e massa $m_p = \qty{1.67e-27}{\kilo\gram}$ e un elettrone di carica $-e$ e massa $m_e = \qty{9.11e-31}{\kilo\gram}$.

L'equazione di Schrödinger stazionaria è nuovamente
\begin{gather}
    \parder[][2]{x} \Psi + \alpha^2 \Psi = 0
    \implies
    \Psi(x) = A \exp{\im \alpha x} + B \exp{-\im \alpha x} \\
    E_n = \frac{\pi^2 \hbar^2}{2 m L^2} n^2
\end{gather}

Se vogliamo trovare le velocità:
\begin{equation}
    E_n = \frac{1}{2} m v_n^2 \implies v_n
    = \frac{\pi \hbar}{m L} n \propto \frac{1}{L}
\end{equation}
Cioè, un confinamento maggiore porta a energie e velocità maggiori.

Con $L = \qty{1}{\meter}$, le velocità dei protoni saranno dell'ordine \qty{e-7}{\metre\per\second}, quelle degli elettroni dell'ordine \qty{e-4}{\metre\per\second}.

Con una buca di potenziale adatta alla descrizione di un atomo, come \qtyrange{0.5}{5}{\angstrom}, le velocità saranno dell'ordine \qty{e4}{\metre\per\second}.

\subsection{Transizioni energetiche}

Una particella può assorbire energia tramite un fotone e aumentare il suo livello energetico, cioè passare a uno stato eccitato.

Similmente, può emettere un fotone e diminuire il proprio livello energetico.

Esercizio: un ettrone in un atomo assorbe un fotone.
Rappresentiamo l'atomo come una buca di potenziale di lato $L = \qty{5}{\angstrom}$.
Qual è la lunghezza d'onda $\lambda$ del fotone?
\begin{equation}
    E_f = E_2 - E_1 = \frac{\hbar^2 \pi^2}{2 m L^2} (2^2 - 1^2), \quad
    \lambda = \frac{h c}{E_f} = \qty{275}{\nano\metre}
\end{equation}

Esercizio: qual è la dimensione della buca di potenziale per cui la diseccitazione di un elettrone da $n_2$ a $n_1$ corrisponde a un fotone con lunghezza d'onda $\lambda$?
\begin{equation}
    E_f = \frac{h c}{\lambda} = E_2 - E_1 = \frac{\hbar^2 \pi^2}{2 m L^2} (n_2^2 - n_1^2)
    \implies
    L = \sqrt{\frac{\lambda h (n_2^2 - n_1^2)}{8 m c}}
\end{equation}

Esercizio: all'interno di un nucleo ($L = \qty{e-15}{\metre}$), un protone assorbe un fotone e passa da $n_1 = 1$ a $n_2 = 2$.
Qual è la lunghezza d'onda del fotone?
\begin{equation}
    E_f = E_2 - E_1 = \frac{\hbar^2 \pi^2}{2 m L^2} (2^2 - 1^2), \quad
    \lambda = \frac{h c}{E_f} = \qty{2e-15}{\metre}
\end{equation}
I fotoni scambiati dai nuclei sono raggi gamma, molto energetici.

\section{Gradino di potenziale}

\begin{equation}
    V(x) = \begin{cases}
        0 & x < 0 \text{ (regione I)} \\
        V_0 & x \ge 0 \text{ (regione II)}
    \end{cases}
\end{equation}
con $V_0 > E$.
\begin{subequations}
\begin{gather}
    \hat{H}_\mathrm{I} = -\dfrac{\hbar^2}{2 m} \lapl \\
    \hat{H}_\mathrm{II} = -\dfrac{\hbar^2}{2 m} \lapl + V_0
\end{gather}
\end{subequations}

Definiamo i seguenti parametri
\begin{equation}
    \alpha = \sqrt{\frac{2 m E}{\hbar^2}}, \quad
    \beta = \sqrt{\frac{2 m (E - V_0)}{\hbar^2}} \eqcolon \im \gamma
\end{equation}
in modo che $\alpha, \gamma \in \R$, mentre $\beta$ è immaginario.
\begin{subequations}
\begin{gather}
    \Psi_\mathrm{I}(x) = A \exp{\im \alpha x} + B \exp{-\im \alpha x} \\
    \Psi_\mathrm{II}(x) = C \exp{-\gamma x} + D \exp{\gamma x}
\end{gather}
\end{subequations}

Imponiamo le seguenti condizioni:
\begin{itemize}
    \item $\int_{-\infty}^{+\infty} \abs{\Psi_\mathrm{I} + \Psi_\mathrm{II}}^2 \de x = 1 \implies D = 0$
    \item $\Psi \text{ continua} \implies \Psi_\mathrm{I}(0) = \Psi_\mathrm{II}(0) \implies A + B = C$
    \item $\Psi \text{ derivabile} \implies \Psi_\mathrm{I}'(0) = \Psi_\mathrm{II}'(0) \implies \im \alpha (A - B) = - \gamma C$
\end{itemize}

Interessano in particolare i rapporti $C/A$ e $B/A$: sono legati alle probabilità che l'onda sia trasmessa o riflessa, rispettivamente.

\begin{subequations}
\begin{align}
    B = C - A
    \implies \im \alpha A - \im \alpha (C - A) = - \gamma C
    \implies C = \frac{2 \im \alpha}{\im \alpha - \gamma} A \\
    C = A - B
    \implies \im \alpha A - \im \alpha B = - \gamma (A + B)
    \implies B = \frac{\im \alpha + \gamma}{\im \alpha - \gamma} A
\end{align}
\end{subequations}

Si definisce il coefficiente di riflessione:
\begin{equation}
    R = \abs{\frac{B}{A}}^2 = \frac{\alpha^2 + \gamma^2}{\alpha^2 + \gamma^2} = 1
\end{equation}
Cioè, la particella viene sempre riflessa.

Allo stesso tempo, la probabilità di trovare la particella nella regione II è
\begin{equation}
    \abs{\Psi_\mathrm{II}}^2 = \abs{C}^2 \exp{-2 \gamma x} \ne 0
\end{equation}
Non è contraddittorio poiché la particella, effettivamente, non viene trasmessa.

Questo è l'\important{effetto tunnel}.

\section{Buca di potenziale a pareti finite}

\begin{equation}
    V(x) = V_0 \, [x < 0 \lor x > L]
\end{equation}

A differenza del caso con pareti infinite, la probabilità di trovare la particella al di là delle pareti è non nulla, proprio a causa dell'effetto tunnel.

\section{Barriera di potenziale}

\begin{equation}
    V(x) = V_0 \, [0 < x < L]
\end{equation}
con $V_0 > E$.

Definiamo le zone I ($x < 0$), II ($0 < x < L$) e III ($x > L$).

Zone I e III:
\begin{equation}
    \parder[][2]{x} \Psi + \frac{2 m E}{\hbar^2} \Psi
    = \parder[][2]{x} \Psi + \alpha^2 \Psi
    = 0
\end{equation}
Zona II:
\begin{equation}
    \parder[][2]{x} \Psi + \frac{2 m (E - V_0)}{\hbar^2} \Psi
    = \parder[][2]{x} \Psi - \gamma^2 \Psi
    = 0
\end{equation}
avendo definito
\begin{subequations}
\begin{gather}
    \alpha = \sqrt{\frac{2 m E}{\hbar^2}} \\
    \gamma = \sqrt{\frac{2 m (V_0 - E)}{\hbar^2}}
\end{gather}
\end{subequations}

Soluzioni:
\begin{itemize}
    \item Regione I:
    \begin{equation}
        \Psi_\mathrm{I}(x) = A \exp{\im \alpha x} + B \exp{-\im \alpha x}
    \end{equation}
    \item Regione II:
    \begin{equation}
        \Psi_\mathrm{II}(x) = F \exp{-\gamma x} + G \exp{\gamma x}
    \end{equation}
    \item Regione III:
    \begin{equation}
        \Psi_\mathrm{III}(x) = C \exp{\im \alpha x}
    \end{equation}
\end{itemize}

Imponiamo continuità e derivabilità in $0$ e $L$:
\begin{gather}
    \Psi_\mathrm{I}(0) = \Psi_\mathrm{II}(0)
    \implies A + B = F + G \\
    \Psi_\mathrm{II}(L) = \Psi_\mathrm{III}(L)
    \implies F \exp{-\gamma L} + G \exp{\gamma L} = C \exp{\im \alpha L} \\
    \Psi_\mathrm{I}'(0) = \Psi_\mathrm{II}'(0) \implies
    \im \alpha A - \im \alpha B = -\gamma F + \gamma G \\
    \Psi_\mathrm{II}'(L) = \Psi_\mathrm{III}'(L) \implies
    -\gamma F \exp{-\gamma L} + \gamma G \exp{\gamma L} = \im \alpha C \exp{\im \alpha L}
\end{gather}

\redtext{
Poiché
\begin{equation}
    \int_{-\infty}^{+\infty} \abs{\Psi_\mathrm{I} + \Psi_\mathrm{II} + \Psi_\mathrm{III}}^2 \de x = 1
\end{equation}
si deve avere $G = 0$.
}

Si ottiene
\begin{equation}
    \frac{C}{A} = \frac{4 \im \alpha \gamma \exp{(\gamma - \im \alpha) L}}{(\gamma^2 - \alpha^2) (1 - \exp{2 \gamma L}) + 2 \im \alpha \gamma (1 + \exp{2 \gamma L})}
\end{equation}
e la probabilità di trasmissione è
\begin{gather}
    T = \abs{\frac{C}{A}}^2 \sim \exp{-2 \gamma L}
\end{gather}

Se le grandezze fisiche hanno dimensioni classiche, quindi, questa probabilità è praticamente nulla.

Nel caso micorscopico, invece, ad esempio con un elettrone con energia $E = \qty{5}{\electronvolt}$, $V_0 = \qty{6}{\electronvolt}$ e $L = \qty{0.5}{\angstrom}$, si ottiene $T = \qty{20}{\percent}$.

Per una particella che proviene da sinistra, la funzione d'onda sarà sinusoidale nella regione I, un esponenziale decrescente nella regione II e nuovamente una sinusoide nella regione III, di ampiezza pari alla precedente.

\section{Rotore rigido}

È un modello che si usa per rappresentare sistemi quantistici in cui è presente una rotazione.

Esempio: massa $m$ che orbita a distanza $r$ fissa intorno a un asse.

Il momento angolare classico è $\vt{L} = \vt{r} \times \vt{p}$.

In meccanica quantistica, il momento angolare è un operatore:
\begin{equation}
    \hat{\vt{L}} = \hat{\vt{r}} \times \hat{\vt{p}}
    = -\im \hbar \vt{r} \times \grad
\end{equation}

L'energia cinetica classica è $E = L^2 / (2 I)$, con $I = m r^2$ momento di inerzia.
Quindi,
\begin{equation}
    \hat{H} = \frac{\hat{L}^2}{2 I}
\end{equation}

Equazione di Schrödinger stazionaria:
\begin{equation}
    \hat{L}^2 \Psi = \underbrace{2 I E}_\lambda \Psi
\end{equation}

Si osserva anche che, $\forall i, j \in \{x, y, z\}, i \ne j$:
\begin{equation}
    [\hat{L}_i, \hat{L}_j] \ne 0
\end{equation}
Quindi, non è possibile conoscere due componenti del momento angolare simultaneamente (se due operatori non commutano, vale la disuguaglianza di Heisenberg).

Ad esempio:
\begin{equation}
    [\hat{\vt{x}}, \hat{\vt{p}}] = \im \hbar \ne 0
    \implies
    \Delta x \Delta p \ge \frac{\hbar}{2}
\end{equation}

Però vale sempre che
\begin{equation}
    [\hat{L}^2, \hat{L}_i] = 0
\end{equation}
Quindi posso conoscere una componente e il modulo.

Si può quindi associare all'equazione di Schrödinger l'equzione stazionaria per la componente $z$:
\begin{equation}
    \begin{cases}
        \hat{L}^2 \Psi = \lambda \Psi \\
        \hat{L}_z \Psi = M_z \Psi
    \end{cases}
\end{equation}

Gli autovalori sono:
\begin{gather}
    \lambda = l (l + 1) \hbar^2, \quad l \in \N \\
    M_z = m \hbar, \quad m \in \{-l, -l + 1, \ldots, l - 1, l\}
\end{gather}

Si introducono quindi i numeri quantici $l$ e $m$.

Ricordando che $\lambda = 2 I E$,
\begin{equation}
    E = \frac{\hbar^2}{2 I} l (l + 1)
\end{equation}

Quindi, $E$ è identificato da $l$ e si possono avere $2l + 1$ stati quantici diversi (identificati da $m$) associati allo stesso livello energetico.

L'energia del fotone associato alla transizione da $E_l$ a $E_{l-1}$ è
\begin{equation}
    h \nu = \frac{\hbar^2}{2 I} \pts{l (l + 1) - (l - 1) (l - 1 + 1)} = \frac{\hbar^2}{I} l
\end{equation}

Alla fine, abbiamo i seguenti \important{numeri quantici}:
\begin{table}[!h]
    \centering
    \begin{tabular}{|c|c|c|c|}
        \hline
        Nome & Simbolo & Operatore & Autovalori \\
        \hline
        Numero quantico principale & $n$ & $\hat{H}$ & $E_n$ \\
        Numero quantico azimutale & $l$ & $\hat{L}^2$ & $\hbar^2 l (l + 1)$ \\
        Numero quantico magnetico & $m$ & $\hat{L}_z$ & $m \hbar$ \\
        \hline
    \end{tabular}
    \caption{Numeri quantici e relativi operatori e autovalori}
    \label{tab:numeri_quantici}
\end{table}

\section{Oscillatore armonico}

\begin{equation}
    V(x) = \frac{1}{2} m \omega^2 x^2
\end{equation}

Operatore hamiltoniano:
\begin{equation}
    \hat{H} = -\frac{\hbar^2}{2 m} \parder[][2]{x} + \frac{1}{2} m \omega^2 x^2
\end{equation}

Equazione stazionaria di Schrödinger:
\begin{equation}
    \parder[][2]{x} \Psi + \pts{- \frac{m^2 \omega^2}{\hbar^2} x^2 + \frac{2 m E}{\hbar^2}} \Psi = 0
\end{equation}

Autovalori:
\begin{equation}
    E_n = \hbar \omega \pts{n + \frac{1}{2}}
\end{equation}

Autostati:
\begin{equation}
    \Psi(\xi) = H_n(\xi) \exp{-\frac{\xi^2}{2}}, \quad
    \xi = \sqrt{\frac{m \omega}{\hbar}} x
\end{equation}
usando i polinomi di Hermite:
\begin{equation}
    H_n(x) = (-1)^n \exp{x^2} \parder[][n]{x} \exp{-x^2}
\end{equation}

\redtext{Grafico di $V(x)$ con livelli energetici $E_n$}

\redtext{Grafici di $\Psi$ per $n = 0, 1, 2$}

Nota che $E_n \ne 0$ sempre, cioè non è possibile che l'oscillatore sia fermo, o sarebbe violato il principio di indeterminazione di Heisenberg per posizione e quantità di moto.

\section{Atomo monoelettronico}

Atomo con $Z$ protoni e un elettrone a distanza $r$.
\begin{gather}
    V(r) = -\frac{Z e^2}{4 \pi \eps_0 r} \\
    H = \frac{p^2}{2 m} - \frac{Z e^2}{4 \pi \eps_0 r} \\
    \hat{H} \Psi(\p) = \pts{-\frac{\hbar^2}{2 m} \lapl - \frac{Z e^2}{4 \pi \eps_0 r}} \Psi(\p) = E \Psi(\p)
\end{gather}

La soluzione in coordinate sferiche è
\begin{equation}
    \Psi_{n, l, m}(r, \theta, \phi) = \Phi_{n, l}(r) Y_l^m(\theta, \phi) \eqcolon \ket{n, l, m}
\end{equation}
le funzioni $Y_l^m$ sono le armoniche sferiche.

Le tre equazioni stazionarie sono:
\begin{equation}
    \begin{cases}
        \hat{H} \Psi_{n,l,m} = E_n \Psi_{n,l,m} \\
        \hat{L}^2 \Psi_{n,l,m} = \hbar^2 l (l + 1) \Psi_{n,l,m} \\
        \hat{L}_z \Psi_{n,l,m} = \hbar m \Psi_{n,l,m}
    \end{cases}
\end{equation}

Ogni livello energetico $E_n$ ha $n$ possibili valori di $l$ e ogni sottolivello (individuato da $n$ ed $l$) ha $2 l + 1$ possibili valori di $m$:
\begin{subequations}
\begin{gather}
    n \in \Z^+ \\
    l \in \{0, 1, \ldots, n - 1\} \\
    m \in \{-l, -l + 1, \ldots, l - 1, l\}
\end{gather}
\end{subequations}

Gli autovalori $E_n$ sono
\begin{equation}
    E_n = -\frac{m Z^2 e^4}{2 \pts{4\pi \eps_0}^2 \hbar^2 n^2} \propto -\frac{Z^2}{n^2}
\end{equation}
È la stessa espressione dell'\cref{eq:energia_bohr}, l'energia di un elettrone secondo il modello di Bohr (lì, con $Z = 1$).


Ogni possibile stato $\ket{n, l, m}$ corrisponde a un orbitale.

Si indicano con numeri i livelli di energetici e con $s, p, d, f$ i valori di $l = 0, 1, 2, 3$ (le lettere proseguono poi in ordine alfabetico da $g$ in avanti).
Ad esempio, $3p$ è il sottolivello con $n = 3$, $l = 1$ e $m = -1, 0, 1$.

La forma degli orbitali dipende dalle armoniche sferiche:
\begin{table}[!h]
    \centering
    \begin{tabular}{|c|c|c|c|}
        \hline
        $l$ & Orbitali & Forma & Immagine \\
        \hline
        0 & 1 orbitale $s$ & sferica & \redtext{immagine}
        \\
        1 & 3 orbitali $p$ & bilobata & \redtext{immagine}
        \\
        \hline
    \end{tabular}
    \caption{Orbitali atomici per diversi valori di $l$}
    \label{tab:orbitali}
\end{table}

\subsection{Esperimento di Stern-Gerlach e spin}

Pauli introdusse un quarto numero quantico legato a una proprietà (lo \important{spin}) che esprime il momento angolare intrinseco di una particella.
Questo numero quantico è detto \important{numero quantico di spin} e si indica con $m_s$.

Consideriamo un fascio di elettroni che attraversa una regione con campo magnetico $\B(\p) = B(z) \uz$ e colpisce uno schermo.
Classicamente, mi aspetterei una striscia continua di elettroni sullo schermo:
\begin{gather}
    \parder[B_z]{z} \ne 0 \\
    U = - \vt{\mu} \cdot \B \\
    \force = - \grad U = \mu_z \parder[\B]{z}
\end{gather}
Invece, si riconoscono due fasci.
Anche lo spin, quindi, è discreto e può assumere solo due valori opposti.

Questo fenomeno avviene lungo ogni direzione e non è possibile conoscere simultaneamente le componenti di spin lungo due direzioni diverse (come nel caso del momento angolare, non commutano).

Lo spin (questo momento angolare intrinseco) si comporta come un momento angolare orbitale.
\begin{gather}
    \hat{s}^2 \Psi = \hbar^2 s (s + 1) \Psi \\
    \hat{s}_z \Psi = \hbar m_s \Psi
\end{gather}
Inoltre, per gli elettroni, vale sempre
\begin{equation}
    s = \frac{1}{2} \implies m_s = \pm \frac{1}{2}
\end{equation}

Lo stato di un elettrone, quindi, si scrive come $\ket{n, l, m, m_s}$.
I primi tre numeri quantici individuano l'orbitale, l'ultimo caratterizza lo spin della particella.

Il principio di esclusione di Pauli afferma (in particolare) che due elettroni non possono avere lo stesso stato quantico.

Esempio: i due elettroni dell'atomo di elio (che ha configurazione elettronica $1s^2$) avranno stati $\ket{1, 0, 0, \pm 1/2}$.

È anche possibile distinguere due tipi di particelle:
\begin{itemize}
    \item \important{Fermioni}, con spin semintero, che obbediscono al principio di esclusione di Pauli
    \item \important{Bosoni}, con spin intero, che \important{non} obbediscono al principio di esclusione di Pauli

    Più bosoni possono quindi occupare lo stesso stato quantico, anche in gran numero.
\end{itemize}

\section{Nucleo atomico}

Un nucleo $_Z^A X$ è identificato dal numero di protoni $Z$, dal numero di neutroni $N$ e dal numero di massa $A = Z + N$.

Il nucleo è descritto da questo potenziale:
\begin{equation}
    V(r) = \begin{cases}
        -E_\text{legame} & r < R \\
        \dfrac{Z e^2}{4 \pi \eps_0 r} & r > R \\
    \end{cases}
\end{equation}

Definiamo anche $C$, il valore per $r = R^+$:
\begin{equation}
    C \eqcolon \frac{Z e^2}{4 \pi \eps_0 R}
\end{equation}

L'energia di legame si esprime con la formula:
\begin{equation}
    E_\text{legame} = \Delta m c^2
\end{equation}
dove $\Delta m$ è la differenza di massa tra i nucleoni liberi e quelli legati nel nucleo.

Infatti, quando delle particelle si legano a formare un nucleo, si ha una perdita di massa (\important{difetto di massa}), che si trasforma in energia di legame.

\subsection{Decadimento alfa}

La particella $\alpha$ è un nucleo di elio, cioè due protoni e due neutroni.
\begin{equation}
    \alpha = \ce{^4_2He}
\end{equation}

Decadimento alfa:
\begin{equation}
    \ce{^{A}_{Z}X -> ^{A-4}_{Z-2}Y + \alpha}
\end{equation}

Esempio:
\begin{equation}
    \ce{^{226}_{88}Ra -> ^{222}_{86}Rn + \alpha}
\end{equation}

Il radio ha $C_{\ce{Ra}} = \qty{50}{\mega\electronvolt}$ e $R_{\ce{Ra}} = \qty{e-15}{\metre}$.
Si osserva che la particella $\alpha$ emessa ha una energia cinetica di circa \qty{10}{\mega\electronvolt}.

Il problema è stato descritto da Gamow.
\begin{gather}
    V(R_B) = \frac{(Z - 2)e \cdot 2 e}{4 \pi \eps_0 R_B} = \qty{10}{\mega\electronvolt} \\
    \abs{\frac{C}{A}}^2 = \num{e-31} \\
    f_\text{urti} = \frac{v_\alpha}{2 R_{\ce{Ra}}}
\end{gather}

$v_\alpha$ si stima con il principio di indeterminazione:
\begin{gather}
    \Delta x \Delta p \approx \frac{\hbar}{2}
    \implies
    \Delta (2 R_{\ce{Ra}}) (m_\alpha \Delta v_\alpha) \approx \frac{\hbar}{2} \\
    \implies
    v_\alpha \sim \qty{e5}{\metre\per\second}
    \implies
    f_\text{urti} \sim \qty{e20}{urti\per\second}
\end{gather}

Esercizio: quanti decadimenti al secondo avvengono in un campione di \qty{1}{\gram} di radio?
\begin{gather}
    N = \frac{m}{{M\!M}_{\ce{Ra}}} N_A = \qty{e21}{atomi} \\
    f_\text{urti} \cdot N \cdot \abs{\frac{C}{A}}^2 = \qty{e10}{decadimenti\per\second}
\end{gather}

\subsection{Fusione nucleare}

\redtext{Grafico di $V(r)$ per la fusione nucleare}

Fusione di due protoni in un nucleo di deuterio:
\begin{equation}
    \ce{^1_1H + ^1_1H -> ^2_1H}
\end{equation}

L'energia potenziale massima è \qty{14}{\mega\electronvolt}, quella che un protone deve avere per svolgere la fusione è \qty{1.9}{\kilo\electronvolt}, quindi il fenomeno deve avvenire per effetto tunnel.

\begin{equation}
    E_k = \frac{3}{2} k_B T = \frac{e^2}{4 \pi \eps_0 r} \implies r \sim \qty{e-13}{\metre}
\end{equation}

\begin{equation}
    \abs{\frac{C}{A}}^2 = \num{e-36}
\end{equation}

Successivamente alla fusione di due protoni, si ha la fusione di un protone con il deuterio:
\begin{gather}
    \ce{^1_1H + ^1_1H -> ^2_1H} \\
    \ce{^2_1H + ^1_1H -> ^3_2He} \\
    \ce{^3_2He + ^1_1H -> ^4_2He}
\end{gather}

\subsection{Fissione nucleare}

\begin{equation}
    \ce{^{235}_{92}U + n -> ^{141}_{56}Ba + ^{92}_{36}Kr + 3 n + \qty{200}{\mega\electronvolt}}
\end{equation}

I tre neutroni emessi colpiscono altri nuclei di uranio e il processo si autoalimenta in una reazione a catena.

Serve però un campione di uranio arricchito, cioè con una percentuale maggiore del normale di isotopi fissili.

\section{Esercizi vari}

Esercizio: trovare i possibili valori di energia per una particella di massa $m = \qty{e-30}{\kilo\gram}$ e carica $Q = \qty{e-19}{\coulomb}$ in moto a distanza $R = \qty{1}{\nano\metre}$ da un punto.
\begin{gather}
    \hat{H} \Psi = E \Psi \implies \frac{\hat{L}^2}{2 I} \Psi = E \Psi
    \implies
    \hat{L}^2 \Psi = 2 I E \Psi \\
    2 I E = \hbar^2 l (l + 1) \implies E = \frac{\hbar^2}{2 m R^2} l (l + 1)
\end{gather}

Calcolare la frequenza del fotone che permette il passaggio dal secondo livello al livello fondamentale.
\begin{equation}
    \nu = \frac{E_2 - E_0}{h} = \frac{\hbar^2}{2 m R^2 h} \pts{2 (2 + 1) - 0 (0 + 1)}
    = \frac{3 \hbar}{2 \pi m R^2}
    = \qty{5e13}{\hertz}
\end{equation}

Esercizio:
calcolare $[x, p_y]$.
\begin{gather}
    \Psi(\vt{x}) = \exp{\im \vt{k} \cdot \vt{x}} = \exp{\frac{\im \vt{p} \cdot \vt{x}}{\hbar}} \\
    \grad \Psi(\vt{x}) = \im \frac{\vt{p}}{\hbar} \Psi(\vt{x})
    \implies \vt{p} \Psi(\vt{x}) = -\im \hbar \grad \Psi(\vt{x})
    \implies \hat{\vt{p}} = -\im \hbar \grad \\
    [\hat{x}, \hat{p}_y] = \hat{x} \hat{p}_y - \hat{p}_y \hat{x} = -\im \hbar x \parder{y} + \im \hbar \parder{y} x = 0
\end{gather}

Esercizio:
determinare la massa $m$ di un corpo vincolato a una molla di costante elastica $k = \qty{4e-8}{\newton\per\metre}$ se, transendo tra due livelli contigui, la lunghezza d'onda del fotone associato è $\lambda = \qty{400}{\nano\metre}$.
\begin{gather}
    \hat{H} = -\frac{\hbar^2}{2 m} \parder[][2]{x} + \frac{1}{2} m \omega^2 x^2 \\
    \hat{H} \Psi = E \Psi \implies
    \parder[][2]{x} \Psi = \pts{\frac{m^2 \omega^2}{\hbar^2} x^2 - \frac{2 m E}{\hbar^2}} \Psi \\
    E_n = \pts{n + \frac{1}{2}} \hbar \omega \\
    E_{n+1} - E_n = \hbar \omega = \hbar \sqrt{\frac{k}{m}} = \frac{h c}{\lambda} \\
    m = \frac{k \lambda^2}{4 \pi^2 c^2} = \qty{1.8e-39}{\kilo\gram}
\end{gather}

Esercizio:
un elettrone è in una buca di potenziale con barriere infinite di lato $L = \qty{0.1}{\nano\metre}$.
Trovare l'energia del secondo livello eccitato e la lunghezza d'onda del fotone che va assorbito per saltare dal livello fondamentale al secondo livello eccitato e scrivere la funzione d'onda per questi due livelli.
\begin{gather}
    \hat{H} = -\frac{\hbar^2}{2 m} \lapl, \quad \text{ per } x \in [0, L] \\
    \bigg(\parder[][2]{x} + \underbrace{\frac{2 m E}{\hbar^2}}_{\alpha^2}\bigg) \Psi = 0 \\
    \Psi(x) = A \exp{\lambda x} \implies \lambda^2 + \alpha^2 = 0 \implies \lambda = \pm \im \alpha \\
    \Psi(x) = A \exp{\im \alpha x} + B \exp{-\im \alpha x} \\
    \Psi(0) = 0 \implies A + B = 0 \\
    \Psi(L) = 0 \implies \exp{\im \alpha L} - \exp{-\im \alpha L} = 0 \implies \sin(\alpha L) = 0 \\
    \implies \alpha = \frac{n \pi}{L}, \, n \in \N \\
    E_n = \frac{\hbar^2 \pi^2}{2 m L^2} n^2 \\
    \int_0^L \abs{\Psi}^2 \de x = 1 \implies 4 A^2 \int_0^L \sin^2(\alpha x) \de x = 1 \implies A = \frac{1}{\sqrt{2 L}} \\
    \Psi_n(x) = 2 \im \frac{1}{\sqrt{2 L}} \sin\pts{\frac{n \pi x}{L}} \\
    \implies n \ge 1
\end{gather}
Il secondo livello eccitato, quindi, è quello per $n = 3$.
\begin{gather}
    E_3 = \frac{\hbar^2 \pi^2}{2 m L^2} 3^2 = \qty{338}{\electronvolt} \\
    \lambda = \frac{h c}{E_3 - E_1} = \frac{2 h c m L^2}{\hbar^2 \pi^2 \cdot (9 - 1)} = \qty{4.13}{\nano\metre} \\
    \Psi_1(x) = \im \sqrt{\frac{2}{L}} \sin\pts{\frac{\pi x}{L}} \\
    \Psi_3(x) = \im \sqrt{\frac{2}{L}} \sin\pts{\frac{3 \pi x}{L}}
\end{gather}

Esercizio:
molecola di azoto \ce{N2} con atomi a distanza fissa $d$.
Trovare le energie dello stato fondamentale e dei primi due stati eccitati.
Si consideri solo la rotazione.

Sia $m$ la massa di un nucleone e $M = 14 m$ la massa di un nucleo di azoto.
\begin{gather}
    I = 2 M \pts{\frac{d}{2}}^2 \\
    \hat{H} = \frac{\hat{L}^2}{2 I} \\
    \hat{L}^2 \Psi = 2 I E \Psi \\
    2 I E = \hbar^2 l (l + 1) \implies E
    = \frac{\hbar^2}{2 I} l (l + 1)
    = \frac{\hbar^2}{M d^2} l (l + 1)
    = \frac{\hbar^2}{14 m d^2} l (l + 1) \\
    E_0 = 0 \\
    E_1 = \frac{\hbar^2}{7 m d^2} \\
    E_2 = \frac{3 \hbar^2}{7 m d^2}
\end{gather}

Esercizio:
quanto vale $E_1 - E_0$ per una molecola di ossigeno \ce{O2} con atomi a distanza fissa $d = \qty{1.2}{\angstrom}$ e massa $M = \qty{2.2e-26}{\kilo\gram}$?
\begin{gather}
    I = 2 M \pts{\frac{d}{2}}^2 \\
    2 I E = \hbar^2 l (l + 1) \implies E_l = \frac{\hbar^2}{2 I} l (l + 1) \\
    E_1 - E_0 = \frac{\hbar^2}{2 I} \pts{1 (1 + 1) - 0 (0 + 1)}
    = \frac{2 \hbar^2}{M d^2}
    = \qty{439}{\micro\electronvolt}
\end{gather}

Esercizio:
trovare la massa degli atomi a distanza $d = \qty{2}{\nano\metre}$ in una molecola, sapendo che la lunghezza d'onda del fotone associato al salto tra i livelli 0 e 1 è $\lambda = \qty{500}{\nano\metre}$.
\begin{gather}
    E_1 - E_0 = \frac{\hbar^2}{2 I} \pts{1 (1 + 1) - 0 (0 + 1)} = \frac{h c}{\lambda} \\
    I = \frac{\hbar^2 \lambda}{hc} = \frac{\hbar \lambda}{2 \pi c} = 2 M \pts{\frac{d}{2}}^2 \\
    M = \frac{\hbar \lambda}{\pi c d^2} = \qty{1.4e-31}{\kilo\gram}
\end{gather}
