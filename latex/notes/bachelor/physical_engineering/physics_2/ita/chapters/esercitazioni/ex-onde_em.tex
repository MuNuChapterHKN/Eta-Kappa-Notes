\chapter{Onde elettromagnetiche -- esercitazioni}

\section{Onda sferica}

Un'onda sferica di genera quando una sorgente puntiforme emette onde in un mezzo isotropo ed è caratterizzata da $\vt{k} \parallel \p$ (ponendo la sorgente nell'origine).
\begin{gather}
    \E(\p, t) = \E_0 \sin(k r - \omega t) \\
    \vt{S} = \frac{E_0^2}{\mu_0 c} \sin^2(kr - \omega t) \ver{r}
\end{gather}

Il flusso di $\vt{S}$ attraverso una superficie sferica $\Sigma$ di raggio $r$ è la seguente potenza:
\begin{equation}
\begin{gathered}
    P(r) = \int_\Sigma \vt{S} \sde = \int_{\Sigma} \frac{E_0^2}{\mu_0 c} \sin^2(kr - \omega t) \ver{r} \sde = \\
    = \int_{\phi = 0}^{2\pi} \int_{\theta = 0}^{\pi} \frac{E_0^2}{\mu_0 c} \sin^2(kr - \omega t) \, r^2 \sin\theta \, \de\theta \de\phi
    = \frac{E_0^2}{\mu_0 c} \sin^2(kr - \omega t) \, 4 \pi r^2
\end{gathered}
\end{equation}

La potenza media in un periodo vale
\begin{equation}
    \avg{P} = \frac{1}{T} \int_0^T P(t) \de t
    = \frac{E_0^2}{\mu_0 c} 4 \pi r^2 \frac{1}{T} \int_0^T \sin^2(kr - \omega t) \de t
    = \frac{E_0^2}{2\mu_0 c} 4 \pi r^2
    = 4 \pi r^2 I
\end{equation}

La potenza deve essere indipendente dal raggio, e questo vale se e solo se
\begin{equation}
    E_0 \propto \frac{1}{r}
\end{equation}

Esercizio: il campo elettrico raccolto da un ricevitore radio a una distanza $d = \qty{500}{\metre}$ dalla sorgente ha un'ampiezza massima $E_0 = \qty{0.1}{\volt\per\metre}$. Calcolare l'intensità dell'onda presso il ricevitore e la potenza emessa dalla sorgente.
\begin{gather}
    I = \frac{E_0^2}{2 \mu_0 c} = \qty{1.33e-5}{\watt\per\metre\squared} \\
    \avg{P} = 4 \pi d^2 I = \qty{41.7}{\watt}
\end{gather}
