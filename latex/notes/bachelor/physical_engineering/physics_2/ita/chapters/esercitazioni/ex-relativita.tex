\chapter{Relatività -- esercitazioni}
\label{sec:relativita}

\section{Esperimento di Michelson e Morley}

La differenza di fase tra due onde coerenti in un certo punto a distanza $r_1$ dalla prima sorgente e a $r_2$ dalla seconda è
\begin{equation}
    \Delta \phi = (k r_2 - \omega t + \phi_2) - (k r_2 - \omega t + \phi_1) = k (r_2 - r_1) + \phi_2 - \phi_1
\end{equation}

\redtext{Immagine interferometro}

La lastra D serve a compensare la dispersione del raggio verso l'alto dovuta alla riflessione presso la lastra B.

La direzione orizzontale è coerente con la velocità della Terra, quindi l'interferenza risultante doveva dipendere dalla velocità della terra, che determina la differenza dei cammini.

Riguardo il raggio $L_2$:
\begin{gather}
    (c t_3)^2 = L_2^2 + (u t_3)^2 \implies t_3 = \frac{L}{\sqrt{c^2 - u^2}} \\
    t_4 = t_3 \\
    \implies \Delta t_2 = t_3 + t_4  = \frac{2 L}{c} \frac{1}{\sqrt{1 - \frac{u^2}{c^2}}}
\end{gather}

Riguardo il raggio $L_1$
\begin{gather}
    c t_1 = L_1 + u t_1 \implies t_1 = \frac{L}{c - u} \\
    c t_2 = L_1 - u t_2 \implies t_2 = \frac{L}{c + u} \\
    \implies \Delta t_1 = t_1 + t_2 = \frac{2 L}{c} \frac{1}{1 - \frac{u^2}{c^2}}
\end{gather}

Sperimentalmente, però, non si osservò nessuno sfasamento.

Questo perché la composizione delle velocità non avviene come previsto dalle trasformazioni di Galileo.

Si definisce il \important{fattore di Lorentz}:
\begin{gather}
    \gamma(v) \coloneq \frac{1}{\sqrt{1 - \frac{v^2}{c^2}}}
    = 1 + \frac{1}{2} \frac{v^2}{c^2} + o\pts{\frac{v^2}{c^2}}, \quad \text{per } \frac{v}{c} \to 0 \\
    \abs{v} \ll c \implies \gamma \approx 1
\end{gather}

Ma se $c$ è invariante rispetto ai sistemi di riferimento, allora non può esserlo lo spazio.
Ipotizzando la seguente contrazione delle lunghezze per i sistemi in moto a velocità $v$,
\begin{equation}
    L' = \frac{L}{\gamma}
\end{equation}
Effettivamente si ottiene l'uguaglianza tra $\Delta t_1$ e $\Delta t_2$:
\begin{equation}
    \implies \Delta t_1 = \frac{2 L'}{c} \frac{1}{1 - \frac{u^2}{c^2}}
    = \frac{2 L}{c} \frac{\sqrt{1 - \frac{u^2}{c^2}}}{1 - \frac{u^2}{c^2}}
    = \frac{2 L}{c} \frac{1}{\sqrt{1 - \frac{u^2}{c^2}}}
    = \Delta t_2
\end{equation}
Ne segue anche una dilatazione dei tempi:
\begin{equation}
    \Delta t = \gamma \Delta t'
\end{equation}

\section{Trasformazioni di Galileo}

Consideriamo due sistemi di riferimento $S_1: x, y, z$ e $S_2: x', y', z'$.
$S_2$ è in moto con velocità $\vt{v}$ rispetto a $S_1$.
In fisica classica valgono le trasformazioni di Galileo:
\begin{itemize}
    \item Osservatore 1:
        \begin{equation}
            \begin{cases}
                x = x' + v_x t \\
                y = y' + v_y t \\
                z = z' + v_z t \\
                t = t'
            \end{cases}
        \end{equation}
    \item Osservatore 2:
        \begin{equation}
            \begin{cases}
                x' = x - v_x t \\
                y' = y - v_y t \\
                z' = z - v_z t \\
                t' = t
            \end{cases}
        \end{equation}
\end{itemize}

Le equazioni di Maxwell non sono compatibili con queste trasformazioni.

Consideriamo un filo rettilineo con densità di carica $\lambda$ e, a distanza $r$, una carica $q$.
La forza di Coulomb è
\begin{equation}
    F = \frac{q \lambda}{2\pi \eps_0 r}
\end{equation}
Tuttavia, da un sistema di riferimento in moto a velocità $v$, vedo il filo carico in moto, e dunque una corrente elettrica $\lambda v$ e una forza magnetica attrattiva dovuta a un campo magnetico $\mu_0 \lambda v / (2\pi r)$:
\begin{equation}
\label{eq:frel}
    F = \frac{q \lambda}{2\pi \eps_0 r} - q v \frac{\mu_0 \lambda v}{2\pi r}
    = \frac{q \lambda}{2\pi r \eps_0} \pts{1 - \frac{v^2}{c^2}}
    = \frac{q \lambda}{2\pi r \eps_0} \gamma^{-2}
\end{equation}

Poiché le equazioni di Maxwell non sono valide in ogni sistema di riferimento inerziale, vorrebbe dire che ne esiste uno privilegiato.

In realtà, risulta che le trasformazioni di Galileo valgono solo a velocità $\abs{v} \ll c$ e sono un caso limite delle trasformazioni di Lorentz.
Con $\vt{v} = v \ux$,
\begin{itemize}
    \item Osservatore 1:
    \begin{equation}
        \begin{cases}
            x = \gamma (x' + v t) \\
            t = \gamma \pts{t' + \dfrac{v}{c^2} x'}
        \end{cases}
    \end{equation}
    \item Osservatore 2:
    \begin{equation}
        \begin{cases}
            x' = \gamma (x - v t) \\
            t' = \gamma \pts{t - \dfrac{v}{c^2} x}
        \end{cases}
    \end{equation}
\end{itemize}

Con queste trasformazioni, $c$ è la stessa in tutti i sistemi di riferimento.

Immaginiamo due specchi paralleli a distanza $d$ con un raggio di luce che va da uno all'altro dei due.

In un sistema di riferimento in cui il raggio è verticale, il raggio percorre una distanza $d$ in un tempo $\Delta t$.

In uno in cui il raggio ha un angolo, questo percorre una distanza $L > d$ in un tempo $\Delta t'$.

Il cateto parallelo agli specchi si misura nel sistema di riferimento in moto.

Inoltre,
\begin{gather}
    L^2 = d^2 + (v \Delta t')^2 \\
    \implies c^2 \Delta t'^2 = c^2 \Delta t^2 + v^2 \Delta t'^2 \\
    \implies \Delta t'^2 = \Delta t^2 + \frac{v^2}{c^2} \Delta t'^2 \\
    \implies \Delta t' = \gamma \Delta t > \Delta t
\end{gather}

Si ha una \important{dilatazione dei tempi} e si perde il concetto di simultaneità degli eventi: lo stesso evento può avvenire in momenti diversi a seconda dell'osservatore.

Analogamente, vale la \important{contrazione delle lunghezze}:
\begin{equation}
    c = \frac{L}{\Delta t} = \frac{L'}{\Delta t'}
    \implies L' = \frac{L}{\gamma} < L
\end{equation}

È solo per $v > c/2$ circa che $\gamma$ diventa percettibilmente maggiore di 1.
La relatività, quindi, vale solo per velocità elevatissime.

Per quanto riguarda l'esempio precedente, a dover essere corrette sono
\begin{itemize}
    \item la densità di carica, poiché è diversa la lunghezza:
    \begin{equation}
        \lambda' = \gamma \lambda
    \end{equation}
    \item la forza, che è derivata della quantità di moto rispetto al tempo:
    \begin{equation}
        \Delta p' = \frac{q \lambda'}{2\pi \eps_0 r} \gamma^{-2} \Delta t'
        = \frac{q \lambda \gamma}{2\pi \eps_0 r} \gamma^{-2} \gamma \Delta t
        = \frac{q \lambda}{2\pi \eps_0 r} \Delta t
        = \Delta p
    \end{equation}
\end{itemize}

In conclusione, il comportamento del sistema non cambia tra i due sistemi di riferimento.

\section{Equivalenza massa-energia}

Nel sistema $O$, due onde elettromagnetiche colpiscono una particella di massa $m$, propagandosi l'una contro l'altra in verticale.
In un sistema $O'$ in moto $v$ orizzontale, i raggi sono a un angolo $\alpha$ con la verticale.

In $O$, la particella non cambia la sua quantità di moto.

In $O'$, le quantità di moto iniziale e finale sono
\begin{gather}
    p_i = m v_i + 2 \frac{E}{c} \sin\alpha \\
    p_f = m v_f
\end{gather}

Poiché in $O$ la particella non cambia velocità, non deve farlo neanche in $O'$.

Del resto, come in $O$, anche in $O'$ deve valere $p_i = p_f$.

Si ottiene l'assurdo $E = 0$.

L'errore risiede nell'assumere che la massa sia la stessa.
Invece,
\begin{equation}
    p_f = p_i, \, v_f = v_i \eqcolon v
    \implies m_f v = m_i v + 2 \frac{E}{c} \sin\alpha
\end{equation}

Usando anche $\sin \alpha = v/c$,
\begin{equation}
    m_f = m_i + 2 \frac{E}{c^2}
    \implies
    2E = (m_f - m_i) c^2
\end{equation}
La massa della particella aumenta di una quantità equivalente all'energia delle due onde.

In generale, la massa finale dopo un trasferimento di energia che porta il corpo a velocità $v$ per una particella di massa a riposo $m_0$ è
\begin{equation}
    m = \gamma m_0
\end{equation}
Nel limite $v \ll c$,
\begin{equation}
    m = m_0 + \frac{1}{c^2} \frac{1}{2} m_0 v^2 + o\pts{\frac{v^2}{c^2}}
\end{equation}
cioè si ottiene l'espressione dell'energia cinetica

\section{Paradosso dei gemelli}

Di due gemelli di 20 anni sulla Terra uno parte per un lontano pianeta, viaggiando a velocità $v = 0.8 c$ per un tempo $\Delta t = \qty{10}{anni}$.

Considerando andata e ritorno, al ritorno il gemello rimasto sulla terra avrà 40 anni.
Il gemello che torna, invece, avrà 32 anni.

Tuttavia, nel sistema di riferimento dell'astronave, la situazione dovrebbe essere opposta: è il gemello sulla terra a essere invecchiato meno, poiché è la Terra a essere in moto rispetto all'astronave.

Quindi, dovrei essere in grado di scegliere un sistema di riferimento corretto, mentre l'altro sarà sbagliato (perché uno dei due gemelli è effettivamente invecchiato di più).

Il problema sta nel presupporre il concetto di simultaneità.

Siano $T: (x, t)$ il sistema di riferimento della Terra, $A: (x', t')$ quello dell'andata e $R: (x'', t'')$ quello del ritorno.

Il tempo in cui il gemello che parte arriva sul pianeta è diverso in $T$ e in $A$ e quello in cui il gemello arriva e inizia a tornare è diverso tra $A$ e $R$.
