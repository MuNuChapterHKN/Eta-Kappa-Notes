\chapter{Interferenza e diffrazione -- esercitazioni}

% Definizione di intensità:
% \begin{equation}
%     I = \avg{\norm{\vt{S}}} = \frac{E^2}{2 \mu_0 c}
% \end{equation}

% Quando un onda piana incide uno schermo con $N$ fenditure puntiformi, tutte queste fenditure si comportano come sorgenti coerenti.

% Su un secondo schermo, si vede luce là dove le $N$ sorgenti hanno interferenza costruttiva.

% Ci poniamo nella condizione per cui lo schermo è a grande distanza $L$ dalle fenditure.

% Definendo $I_0$ il fattore di proporzionalità,
% \begin{equation}
% \label{eq:intensitynsource}
%     I = I_0 \frac{\sin^2\frac{N \pi d \sin\theta}{\lambda}}{\sin^2\frac{\pi d \sin\theta}{\lambda}}
% \end{equation}
% dove $d$ è la distanza tra le fenditure e $\theta$ è l'angolo rispetto all'orizzontale sullo schermo, l'unico parametro per determinare la posizione del punto di osservazione, per cui
% \begin{equation}
%     - \frac{\pi}{2} < \theta < \frac{\pi}{2}
% \end{equation}

% Definendo $x = \pi d \sin \theta / \lambda$, la funzione
% \begin{equation}
%     f(x) = \frac{\sin^2 N x}{\sin^2 x}
% \end{equation}
% determina il profilo dell'intensità.

% $f$ ha dei massimi quando il numeratore è massimo.

% Di questi, uno ogni $N$ è un massimo principale con valore $N^2 I_0$, che annulla anche il denominatore.

% Tra due massimi principali, ci sono $N - 1$ minimi in cui la funzione si annulla (là dove si annulla il numeratore ma non il denominatore).

% Massimi principali:
% \begin{equation}
%     d \sin \theta_M = m \lambda, \quad m \in \Z
% \end{equation}
% $m$ è detto \important{ordine del massimo}.

% Minimi ($N - 1$ tra ogni coppia di massimi principali):
% \begin{equation}
%     N d \sin \theta_0 = m' \lambda, \quad
%     m' \in \Z, \quad N \nmid m'
% \end{equation}

% Massimi secondari ($N - 2$ tra ogni coppia di massimi principali):
% \begin{equation}
%     N d \sin \theta_m = \pts{m'' + \frac{1}{2}} \lambda, \quad m'' \in \Z, \quad N, N-1 \nmid m''
% \end{equation}

Se sono richieste le posizioni $x$ sullo schermo, queste si trovano come
\begin{equation}
    x = L \tan \theta
\end{equation}

Se $L \gg \abs{x}$,
\begin{equation}
    x \approx L \sin\theta
\end{equation}
È per questo che i massimi e minimi delle figure di interferenza e diffrazione risultano equispaziati.

\section{Inteferenza tra due sorgenti}

Sorgenti isofrequenziali:
\begin{subequations}
\begin{gather}
    \E_1 = \E_{01} \sin(\vt{k}_1 \cdot \p_1 - \omega t + \phi_1) \\
    \E_2 = \E_{02} \sin(\vt{k}_2 \cdot \p_2 - \omega t + \phi_2)
\end{gather}
\end{subequations}

Differenza di fase:
\begin{equation}
    \phi \coloneq \vt{k}_1 \cdot \p_1 + \phi_1 - \vt{k}_2 \cdot \p_2 - \phi_2
\end{equation}

Intensità risultante:
\begin{equation}
    I_\text{tot} = I_1 + I_2 + 2 \sqrt{I_1 I_2} \cos \phi
\end{equation}

Se le sorgenti sono identiche,
\begin{gather}
    I_1 = I_2 \eqcolon I_0 \\
    \implies I_\text{tot} = 4 I_0 \frac{1 + \cos \phi}{2}
    = 4 I_0 \cos^2 \frac{\phi}{2}
\end{gather}

\section{Interferenza}

Esercizio: reticolo di diffrazione con $N = 100$ fenditure e una distanza tra le fenditure di $d = \qty{50}{\micro\metre}$, luce rossa con $\lambda_r = \qty{700}{\nano\metre}$ e luce blu con $\lambda_b = \qty{500}{\nano\metre}$, schermo posto a distanza $L = \qty{2}{\metre}$.
Quanto distano i due massimi di ordine $m = 1$ per i due colori?
\begin{equation}
\begin{gathered}
    \Delta x = x_r - x_b = L \pts{\tan \arcsin \theta_r - \tan \arcsin \theta_b} \approx \\
    \approx L \pts{\arcsin \theta_r - \arcsin \theta_b}
    = L \frac{\lambda_r}{d} m - L \frac{\lambda_b}{d} m
    = \qty{8}{\milli\metre}
\end{gathered}
\end{equation}

Esercizio: doppia fenditura (esperimento di Young), $d = \qty{1.5e-4}{\metre}$, $L = \qty{1.4}{\metre}$, $\lambda = \qty{643}{\nano\metre}$.
Nel punto sullo schermo in $x = \qty{1.8e-2}{\metre}$ si ha un massimo o un minimo?

$x \ll L$, quindi $\sin \theta \approx \tan \theta = x/L$:
\begin{equation}
    \cos^2\pts{\frac{\phi}{2}}
    = \cos^2\pts{\pi \frac{d}{\lambda} \sin\theta}
    \approx \cos^2\pts{\pi \frac{d}{\lambda} \frac{x}{L}} = 1
\end{equation}
L'interferenza è costruttiva.

Esercizio: doppia fenditura (esperimento di Young), $d = \qty{e-5}{\metre}$, $L = \qty{1}{\metre}$, $\lambda = \qty{500}{\nano\metre}$.
In che posizione $x$ si trova il terzo minimo di interferenza?

Con $m = 3$,
\begin{equation}
    x = L \tan \theta
    = L \tan \arcsin \pts{\frac{\lambda}{d} \pts{m - \frac{1}{2}}}
    = \qty{12.6}{\centi\metre}
\end{equation}
\textit{A posteriori}, si vede che si sarebbe anche potuta usare l'approssimazione per angoli piccoli.

Esercizio: reticolo di diffrazione con $4000$ tratti al centimetro.
Il massimo di secondo ordine ($m = 2$) è deviato di $\theta = \qty{34}{\degree}$.
Quanto vale $\lambda$?
\begin{gather}
    d = \pts{\qty{4000}{\per\centi\meter}}^{-1} = \qty{2.5}{\micro\metre} \\
    \lambda = \frac{d \sin{\theta}}{m} = \qty{699}{\nano\metre}
\end{gather}

Esercizio: quattro antenne in fila a una distanza l'una dall'altra di $d = \qty{10}{\metre}$ irradiano (in un mezzo isotropo) una potenza $P = \qty{100}{\kilo\watt}$ a frequenza $\nu = \qty{30}{\mega\hertz}$.
Come varia l'intensità a una distanza $L = \qty{10}{\kilo\metre}$?

Si usa l'\cref{eq:reticolo}.

Poiché il mezzo è isotropo, le sorgenti irradiano onde sferiche.
Allora, l'intensità dovuta a una singola sorgente è
\begin{equation}
    I_0 = \frac{P}{4\pi L^2} = \qty{7.96e-5}{\watt\per\meter\squared}
\end{equation}

I massimi principali sono ad angoli
\begin{equation}
    \theta = \arcsin \frac{m \lambda}{d}
    = \arcsin \frac{m c}{\nu d}
    = \arcsin m
\end{equation}
Poiché $-\pi/2 \le \theta \le \pi/2$, $m \in \{-1, 0, 1\}$.
Gli unici quattro massimi principali sono nelle quattro direzioni rispetto all'allineamento delle antenne.

I minimi saranno $4 - 1 = 3$ per quadrante, in angoli
\begin{gather}
    \theta
    = \arcsin \frac{m' c}{4 \nu d}
    = \arcsin \frac{m'}{4} \\
    \theta_1 = \qty{14.5}{\degree}, \quad
    \theta_2 = \qty{30.0}{\degree}, \quad
    \theta_3 = \qty{48.6}{\degree}
\end{gather}

I massimi secondari saranno due per quadrante:
\begin{gather}
    \theta
    = \arcsin \frac{(m'' + 0.5) c}{4 \nu d}
    = \arcsin \frac{2m'' + 1}{8} \\
    \theta_1 = \qty{22.0}{\degree}, \quad
    \theta_2 = \qty{38.7}{\degree}
\end{gather}

\addfigure[Pattern di radiazione.][0.6]{book/quattro_antenne}

Esercizio: reticolo di diffrazione con $1000$ fenditure al centimetro. Due onde incidenti, $\lambda_1 = \qty{650}{\nano\metre}$ e $\lambda_2 = \qty{640}{\nano\metre}$.
Qual è la separazione angolare tra i due massimi del primo ordine?
\begin{gather}
    d = \pts{\qty{1000}{\per\centi\meter}}^{-1} = \qty{e-5}{\metre} \\
    \Delta \theta = \theta_1 - \theta_2
    \approx \frac{m \lambda_1}{d} - \frac{m \lambda_2}{d}
    = \frac{m}{d}(\lambda_1 - \lambda_2)
    = \qty{e-3}{\radian}
\end{gather}

È possibile osservare la separazione tra i due picchi?

\section{Risoluzione}

\important{Criterio di Rayleigh}: è possibile distinguere due massimi se il secondo cade al di là dei minimi che circondano il primo, o viceversa.
\begin{gather}
    \theta_{\text{max}1} = \frac{\lambda_1}{d}
    \qquad
    \theta_{\text{max}2} = \frac{\lambda_2}{d} \\
    \theta_{\text{min}1^+} = \frac{N+1}{N} \frac{\lambda_1}{d}
    \qquad
    \theta_{\text{min}2^+} = \frac{N+1}{N} \frac{\lambda_2}{d} \\
    \theta_{\text{min}1^-} = \frac{N-1}{N} \frac{\lambda_1}{d}
    \qquad
    \theta_{\text{min}2^-} = \frac{N-1}{N} \frac{\lambda_2}{d}
\end{gather}

Quindi, non è indifferente quale onda scelgo per i minimi.

Sia $\lambda_1 < \lambda_2$.
Le due opzioni sono:
\begin{itemize}
    \item massimo di 2 sul minimo di destra di 1:
    \begin{equation}
        \frac{\lambda_2}{d} = \frac{N+1}{N} \frac{\lambda_1}{d}
        = \frac{\lambda_1}{d} + \frac{\lambda_1}{N_{L1} d}
    \end{equation}
    \item massimo di 1 sul minimo di sinistra di 2:
    \begin{equation}
        \frac{\lambda_1}{d} = \frac{N-1}{N} \frac{\lambda_2}{d}
        = \frac{\lambda_2}{d} - \frac{\lambda_2}{N_{L2} d}
    \end{equation}
\end{itemize}
$N_{L1}$ e $N_{L2}$ sono gli $N$ limite per soddisfare le due condizioni.

Risulta
\begin{subequations}
\begin{gather}
    N_{L1} = \frac{\lambda_1}{\lambda_2 - \lambda_1} \\
    N_{L2} = \frac{\lambda_2}{\lambda_2 - \lambda_1}
\end{gather}
\end{subequations}

Si sceglie di considerare
\begin{equation}
    N_L = \frac{N_{L1} + N_{L2}}{2} = \frac{\frac{\lambda_1 + \lambda_2}{2}}{\lambda_2 - \lambda_1}
    = \frac{\lambda}{\lambda_2 - \lambda_1}
\end{equation}

Esercizio: reticolo con $N = 50$ fenditure e $\lambda_1 = \qty{400}{\nano\metre}$.
Quali valori può assumere $\lambda_2$ affinché non si possano distinguere i massimi di secondo ordine?
\begin{equation}
    \lambda_2 \in \pts{\frac{N m - 1}{N m} \lambda_1, \, \frac{N m + 1}{N m} \lambda_1}
    = \pts{\qty{396}{\nano\metre}, \qty{404}{\nano\metre}}
\end{equation}

\section{Diffrazione}

% Si parla di diffrazione quando si ha una singola fenditura di ampiezza $a > \lambda$.

% Nell'ipotesi che $a \ll L$, la distribuzione dell'intensità sarà
% \begin{equation}
%     I_\text{tot} = I_0 \pts{\frac{\sin \frac{\pi a \sin\theta}{\lambda}}{\frac{\pi a \sin\theta}{\lambda}}}^2
% \end{equation}

% I massimi diversi da quello centrale sono subito molto flebili.

% Minimi:
% \begin{equation}
%     a \sin\theta = m \lambda, \quad
%     m \in \Z, \, m \ne 0
% \end{equation}
% Se $a < \lambda$, non ci sono minimi e non si parla di diffrazione.

% Massimi secondari:
% \begin{equation}
%     a \sin\theta = \pts{m' + \frac{1}{2}} \lambda, \quad
%     m' \in \Z, \, m' \ne 0
% \end{equation}

Esercizio: $a = \qty{6e-4}{\metre}$, $\lambda = \qty{400}{\nano\metre}$, $L = \qty{1.5}{\metre}$.
A che angolo e in che posizione si trova il minimo di ordine 2?
\begin{gather}
    \theta = \arcsin \frac{m \lambda}{a} = \qty{1.33e-3}{\radian} \\
    x = L \tan \theta \approx L \sin \theta = \qty{2}{\milli\metre}
\end{gather}

L'intensità nel caso del reticolo di diffrazione con $a > \lambda$ va descritta considerando sia la diffrazione che l'interferenza:
\begin{equation}
    I_\text{tot} = I_0
    \underbrace{\pts{\frac{\sin\pts{\pi \frac{a}{\lambda} \sin\theta}}{\pi \frac{a}{\lambda} \sin\theta}}^2}_\text{diffrazione}
    \underbrace{\frac{\sin^2 \pts{N \pi \frac{d}{\lambda} \sin \theta}}{\sin^2 \pts{\pi \frac{d}{\lambda} \sin \theta}}}_\text{interferenza}
\end{equation}

\addfigure[][0.8]{book/figura_interferenza_e_diffrazione}

Esercizio: doppia fenditura con $d = \qty{30}{\micro\metre}$, $a = \qty{3}{\micro\metre}$, $\lambda = \qty{500}{\nano\metre}$.
A occhio nudo si vedono solo le regioni con $I > \qty{5}{\percent} I_\text{max}$.
Quanti picchi si vedono?

Capiamo quali massimi di diffrazioni rispettano la condizione richiesta.
Per $m = 1$,
\begin{equation}
    \frac{I}{N^2 I_0} = \pts{\frac{\sin\pts{\pi \frac{a}{\lambda} \sin\theta}}{\pi \frac{a}{\lambda} \sin\theta}}^2
    = \pts{\frac{\sin \frac{3}{2} \pi}{\frac{3}{2} \pi}}^2
    = \qty{4.5}{\percent} < \qty{5}{\percent}
\end{equation}
Quindi, si studia solo il massimo centrale di diffrazione.

Il primo minimo di diffrazione è in $\theta$ tale che $\sin \theta = \lambda / a$.

L'ordine $m$ maggiore del massimo di interferenza che sta entro il minimo di diffrazione si trova imponendo
\begin{equation}
    \frac{\lambda}{a} > m \frac{\lambda}{d}
    \implies
    m_\text{max} = \left\lceil \frac{d}{a} \right\rceil - 1 = 9
\end{equation}
Quindi, i massimi candidati sono $1 + 2 \cdot 9 = 19$.

Cercando caso per caso, si osserva che il massimo di ordine 8 dà $\qty{5.5}{\percent}$ e quello di ordine 9 dà $\qty{1.2}{\percent}$.
Quindi, si vedono $1 + 2 \cdot 8 = 17$ massimi.
