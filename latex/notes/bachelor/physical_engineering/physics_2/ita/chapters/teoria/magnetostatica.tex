\chapter{Magnetostatica}

\section{Principi della magnetostatica}

\subsection{Campo magnetico}

L'evidenza più basilare è costituita da barrette di magnetite che interagiscono reciprocamente.

Queste, quando affiancate, si comportano esattamente come un dipolo elettrico:
\begin{itemize}
    \item In preseza di un campo elettrico, il dipolo si orienta fino a diventare parallelo.
    \item Due dipoli affiancati si dispongono in direzione opposta.
    \item Due dipoli adiacenti si dispongono in direzione concorde.
\end{itemize}
Tuttavia, mentre un dipolo elettrico genera un campo elettrico, le barrette di magnetite no (non sono cariche nemmeno localmente).

Immaginiamo di riempire lo spazio con molti dipoli piccoli: si allineeranno sulle linee di campo.

La limatura di ferro si dispone, attorno alle barrette, secondo linee di campo simili a quelle di un dipolo elettrico.

C'è, quindi, un'altra proprietà della materia: il magnetismo

Si parla (provvisoriamente) di cariche magnetiche nord e sud e di campo magnetico $\B$ (tangente a queste linee di campo).

Il campo magnetico imprime una forza a cariche in movimento.

\important{Forza di Lorentz}:
\begin{equation}
    \force = q \vt{v} \times \B
\end{equation}

Grazie alla forza di Lorentz si definisce il campo magnetico (unità di misura: tesla $\unit{\tesla} = \unit{\newton\second\per\coulomb\per\metre}$).

Tuttavia, questa legge non definisce la sorgente del campo.
Inoltre, non si può parlare di carica magnetica poiché sperimentalmente non si riescono a isolare le ``cariche'' nord e sud, vanno sempre a coppie.

Poiché risultato di un prodotto vettoriale, $\force \perp \vt{v} \implies P = \force \cdot \vt{v} = 0$: il campo magnetico non svolge lavoro.

Quindi, non varia l'energia cinetica e neanche il modulo della velocità.

I tipi di moto possibili per una particella carica di massa $m$ in presenza di un campo magnetico sono i seguenti:
\begin{itemize}
    \item Se $\vt{v} \perp \B$, il moto è circolare uniforme di raggio $R$:
        \begin{equation}
            F = qvB = m \frac{v^2}{R} \implies R = \frac{mv}{qB}
        \end{equation}
    \item Se $\vt{v} \parallel \B$, il moto è rettilineo uniforme
    \item Ogni altro caso è una combinazione dei precedenti: moto a elica con $R = mv\sin{\alpha}/(qB)$
\end{itemize}

Questo rende possibile la spettrometria di massa (vedi \autoref{sec:spettrometro_massa}).

Inoltre, la Terra ha un campo magnetico e intrappola le particelle cariche che si avvicinano all'atmosfera terrestre, tranne vicino ai poli (nascono così le aurore boreali).

\subsection{Esperimento di Ampère}

La limatura di ferro si distribuisce secondo linee di campo in presenza di un filo conduttore attraverso cui passa corrente.
Se la corrente viene spenta, la limatura torna uniforme.
Quindi, la sorgente del campo magnetico è la carica in moto.

\important{Legge elementare di Ampère}: il campo magnetico generato da una carica $q$ in posizione $\p'$ e in moto con velocità istantanea $\vt{v}$ è
\begin{equation}
    \B(\p) = \frac{\mu_0}{4\pi} \frac{q \vt{v}}{\norm{\p - \p'}^2} \times \ver{\p'}
\end{equation}
con $\ver{\p'} = (\p - \p')/\norm{\p - \p'}$.
$\mu_0$ è detta \important{permeabilità magnetica del vuoto}.

Osserviamo che
\begin{itemize}
    \item La forma di quest'equazione è molto simile a quella della legge di Coulomb.
    \item $\B$ ha smmetria cilindrica rispetto alla retta individuata da $\vt{v}$.
    \item Le linee di campo sono tutte chiuse.
    \item vale la seguente relazione:
        \begin{equation}
            \B = \mu_0 \eps_0 \vt{v} \times \E
        \end{equation}
\end{itemize}

\subsection{Riassunto dei principi}

\begin{itemize}
    \item Forza di Lorentz; è il principio con cui si definisce il campo magnetico e l'equivalente della forza di Coulomb.
    \item Legge elementare di Ampère; il campo magnetico, che agisce su cariche in moto, è generato da cariche in moto.
    \item Principio di sovrapposizione degli effetti.
\end{itemize}

La magnetostatica ha due diverse equazioni su
\begin{itemize}
    \item come il campo è generato, sorgenti (legge elementare di Ampère)
    \item come il campo agisce (legge di Lorentz)
\end{itemize}

Per l'elettrostatica, invece, l'equazione è unica (la legge di Coulomb definisce sia il campo elettrico sia la carica; sia la sorgente del campo sia il suo effetto).


\section{Campo magnetico e fili conduttori}

Detta $n$ la densità volumica di cariche, la densità di carica sarà $\rho = n q$.
Ricordando $\dcurr = \rho \vt{v}$ e considerando che un tratto infinitesimo di filo ha volume $S \, \de l$,
\begin{equation}
\label{eq:corrente_infinitesima}
    \de q \, \vt{v} = \rho \, \de V \, \vt{v} = \dcurr \, \de V =  \pts{\int_S \dcurr \sde} \de \vt{l} = I \de \vt{l}
\end{equation}

\subsection{Campo magnetico generato da una corrente}

A partire dalla legge elementare di Ampère e usando la \eqref{eq:corrente_infinitesima}, si ottiene la \important{Legge di Ampère-Laplace}:
\begin{equation}
\label{eq:ampere_laplace}
    \de\B(\p) = \frac{\mu_0}{4\pi} \frac{I \de \vt{l} \times \ver{\p'}}{\norm{\p -\p'}^2}
    = \frac{\mu_0}{4\pi} \frac{\dcurr \times \ver{\p'}}{\norm{\p -\p'}^2} \de V
\end{equation}

Formato integrale:
\begin{equation}
\label{eq:ampere_laplace_integrale}
    \B(\p) = \frac{\mu_0 I}{4\pi} \!\int_\text{filo} \frac{\de \vt{l}' \times \ver{\p'}}{\norm{\p -\p'}^2}
    = \frac{\mu_0}{4\pi} \!\int_\text{filo} \frac{\dcurr(\p') \times \ver{\p'}}{\norm{\p -\p'}^2} \de V'
\end{equation}
Correla la geometria del filo al campo magnetico generato.

\subsection{Campo magnetico esercitato su un filo}

Riutilizzando la \eqref{eq:corrente_infinitesima},
\begin{equation}
    \de\force = I \de\vt{l} \times \B
\end{equation}

Formato integrale:
\begin{equation}
    \force = I\!\int_\text{filo} \de\vt{l}' \times \B(\p')
\end{equation}

\subsection{Forza esercitata reciprocamente}
\label{sec:def_ampere}

\important{Legge di Biot-Savart}: considerando un filo rettilineo infinito con corrente $I$, il campo magnetico a una distanza $R$ è
\begin{equation}
\label{eq:biot_savart}
    \B = \frac{\mu_0 I}{2\pi R} \ver{\theta}
\end{equation}
Di nuovo, ``infinito'' significa che $R \ll$ lunghezza del filo.

A distanza $R$ si colloca parallelo un altro filo rettilineo di lunghezza $L'$ e con corrente $I'$.

Per il principio di azione e reazione, la forza che i due fili esercitano l'uno sull'altro è la stessa.

Quella esercitata sul secondo filo è
\begin{equation}
    \force = I'\!\int_\text{filo 2} \de\vt{l} \times \B = I' L' \norm{\B} \ver{R} = \frac{\mu_0 I I' L'}{2\pi R} \ver{R}
\end{equation}
con $\B$ generato dal primo filo e $\ver{R}$ rivolto dal secondo filo verso il primo (la forza è attrattiva).

Poiché è molto semplice avere fili con corrente e invece molto difficile avere cariche nette stabili, storicamente è stato definito prima l'ampere e poi il coulomb.

\qty{1}{\ampere} è, per definizione, la corrente che percorre due fili a distanza di \qty{1}{\metre} e che si attirano con una forza di \qty{2e-7}{\newton} per ogni metro di lunghezza dei fili.

Per cui, $\mu_0 = \qty{4\pi e-7}{\newton\per\coulomb\squared\second\squared}$, e da questo si definiscono $\eps_0$ e il coulomb.

Definendo una tra carica e corrente, si definiscono sia $\mu_0$ sia $\eps_0$, che non sono indipendenti.
(In particolare, risulterà $\mu_0\eps_0 c^2 = 1$.)

\section{Legge di Gauss per il magnetismo}

Partendo dalla \eqref{eq:ampere_laplace_integrale} scritta con la densità di corrente, prendiamo la divergenza (rispetto alla variabile $\p$)
\begin{equation}
    \diver \B(\p) = \frac{\mu_0}{4\pi} \!\int_V \diver \frac{\dcurr(\p') \times \ver{\p'}}{\norm{\p - \p'}^2} \de V'
\end{equation}
La divergenza può entrare nell'integrale poiché è in $\p$, mentre l'integrale è in $\p'$.

Si usa la formula
\begin{equation}
    \diver (\vt{A} \times \vt{B}) = \vt{B} \cdot (\curl \vt{A}) - \vt{A} \cdot (\curl \vt{B})
\end{equation}

\begin{equation}
    \diver \B(\p) = \frac{\mu_0}{4\pi} \!\int_V \pts{\frac{\ver{\p'}}{\norm{\p - \p'}^2} \cdot \pts{\curl \dcurr(\p')} - \dcurr(\p') \cdot \pts{\curl \frac{\ver{\p'}}{\norm{\p - \p'}^2}}} \de V'
\end{equation}

Il primo termine è nullo perché $\dcurr$ non è funzione di $\p$, il secondo è anch'esso nullo perché è un campo centrale del tipo $1/r^2$, dunque è conservativo.

Risulta
\begin{equation}
    \diver \B = 0
\end{equation}
Ovvero, $\B$ è \important{solenoidale} e tutte le linee di campo sono chiuse.

Formato integrale tramite il teorema di Gauss:
\begin{equation}
    \oint_S \B \sde = 0
\end{equation}

\subsection{Potenziale vettore}

Il fatto che il campo magnetico sia solenoidale permette di introdurre il potenziale vettore $\vpot$:
\begin{equation}
    \B = \curl \vpot
\end{equation}

È il corrispettivo del potenziale scalare $\spot$, che esprime $\curl \E = \vt{0}$ così come il potenziale vettore esprime $\diver \B = 0$.

$\vpot$ non è unico, ma varia a meno del gradiente di un campo scalare.
Se $\vpot$ è un possibile potenziale vettore, andrebbe bene anche $\vpot + \grad \phi$ per un campo scalare $\phi$ qualsiasi, poiché
\begin{equation}
    \curl (\vpot + \grad \phi) = \curl \vpot = \B
\end{equation}

Vogliamo trovarne un'espressione.

Tramite la seguente relazione (gradiente in $\p$)
\begin{equation}
    - \grad\pts{\frac{1}{\norm{\p - \p'}}} = \frac{\ver{\p'}}{\norm{\p - \p'}^2}
\end{equation}
riscriviamo come segue la legge di Ampère-Laplace:
\begin{equation}
    \B(\p)
    = \frac{\mu_0}{4\pi} \int_V \frac{\dcurr(\p') \times \ver{\p'}}{\norm{\p - \p'}^2} \de V'
    = \frac{\mu_0}{4\pi} \int_V \grad\pts{\frac{1}{\norm{\p - \p'}}} \times \dcurr(\p') \, \de V'
\end{equation}
La perdita del segno meno è compensata dall'aver scambiato i fattori del prodotto vettoriale.

Si usa la seguente indentità del calcolo vettoriale:
\begin{equation}
    \grad f \times \dcurr = \curl (f \dcurr)  - f \, \curl \dcurr
\end{equation}
con $f = 1/{\norm{\p - \p'}}$.
\begin{gather}
    \B(\p) =
    \frac{\mu_0}{4\pi} \int_V \curl \pts{\frac{\dcurr(\p')}{\norm{\p - \p'}}} \de V' -
    \frac{\mu_0}{4\pi} \int_V \frac{1}{\norm{\p - \p'}} \nabla \times \dcurr(\p') \de V' = \\
    = \frac{\mu_0}{4\pi} \curl \int_V \frac{\dcurr(\p')}{\norm{\p - \p'}} \de V'
\end{gather}

Abbiamo espresso $\B$ come rotore di un campo vettoriale, che quindi è $\vpot$:
\begin{equation}
\label{eq:potenziale_vettore}
    \vpot(\p) = \frac{\mu_0}{4\pi} \int_V \frac{\dcurr(\p')}{\norm{\p - \p'}} \de V'
    % = \mu_0 \slp \dcurr(\p)
\end{equation}

Si ricorda l'analoga espressione per il potenziale scalare:
\begin{equation}
    \spot(r) = \frac{1}{4\pi \eps_0} \int_V \frac{\rho(\p')}{\norm{\p - \p'}} \de V'
    % = \frac{1}{\eps_0} \slp \rho(\p)
\end{equation}
Per analogia, quindi, anche per ogni componente del potenziale vettore vale l'equazione di Poisson:
\begin{gather}
    \vlapl \vpot(\p) = - \mu_0 \dcurr(\p) \\
    \lapl \spot(\p) = - \frac{1}{\eps_0} \rho(\p)
\end{gather}
Si tratta complessivamente di quattro equazioni differenziali scalari.

\section{Legge di Ampère}

Calcoliamo il rotore di $\B$:
\begin{equation}
    \curl \B = \curl \curl \vpot = - \vlapl \vpot + \cancel{\grad (\diver \vpot)} = \mu_0 \dcurr
\end{equation}
Infatti, definire $\vpot$ come in \eqref{eq:potenziale_vettore} implica che $\vpot$ è solenoidale.

Formato integrale (tramite il teorema di Stokes):
\begin{equation}
\label{eq:legge_di_ampere}
    \oint_\gamma \B \lde = \mu_0 I
\end{equation}
dove $I$ è la \textbf{corrente concatenata a $\gamma$}, ovvero la corrente attraverso una qualunque superficie $S$ che abbia bordo $\gamma$, orientata secondo il verso di percorrenza di $\gamma$.

Simmetria tra campi e sorgenti:
\begin{itemize}
    \item divergenze: $\rho$ sorgente, quindi è nulla quella di $\B$
    \item rotori: $\dcurr$ sorgente, quindi nullo quello di $\E$
    \item la sorgente minima del campo elettrico è la carica
    \item la sorgente minima del campo magnetico è un dipolo $\iff$ non esistono monopoli magnetici (è un'evidenza sperimentale, nella teoria sarebbero possibili)
\end{itemize}

\section{Esempi di calcolo}

\subsection{Filo rettilineo}

Per simmetria le linee di campo sono circonferenze centrate nel filo; lo si vede anche dal prodotto vettoriale nella legge di Ampère-Laplace.

Si ha simmetria cilindrica e $B = \norm{\B}$ è costante se il punto dista $R$ dal filo

\begin{equation}
    \oint_\text{circ. $R$} \B \lde
    = \oint_\text{circ. $R$} B \de l
    = 2\pi R B
    = \mu_0 I \implies B
    = \frac{\mu_0 I}{2\pi R}
\end{equation}

È una conferma della legge di Biot-Savart \eqref{eq:biot_savart}.

\subsection{Solenoide}

= filo ``arrotolato'' attorno a un cilindro.

Consideriamolo di lunghezza $L$ e formato da $N$ giri del filo.

Se $L \gg$ del diametro, il campo magnetico è nullo all'esterno.

Si applica la legge di Ampère su un a sezione rettangolare che interseca la superficie del cilindro.

La curva è $\gamma$ e il lato parallelo all'asse del solenoide è $h \ll L$.

\begin{equation}
    \oint_\gamma \B \lde = B(R) h = \mu_0 \pts{h \frac{N}{L} I} \implies B = \mu_0 \frac{N}{L} I
\end{equation}

I contributi dei lati che attraversano il cilindro si elidono per simmetria e il lato esterno dà contributo zero.

\section{Riassunto: elettrostatica e magnetostatica}

Dagli esperimenti alle leggi:
\begin{itemize}
    \item evidenze sperimentali
    \item principi
    \item legge in formato integrale (valgono complessivamente rispetto a un volume o superficie), sono quelle che si ottengono con esperimenti
    \item legge in formato differenziale (valgono punto per punto), molto più semplici da trattare
\end{itemize}

Leggi sul campo elettrico:
\begin{itemize}
    \item legge di Gauss: $\diver \E = \rho/\eps_0$
    \item conservatività: $\curl \E = \vt{0}$
    \begin{itemize}
        \item potenziale scalare: $\E = -\grad \spot$
    \end{itemize}
\end{itemize}
Riassunte nell'equazione di Poisson: $\lapl \spot = -\rho/\eps_0$.

Leggi sul campo magnetico:
\begin{itemize}
    \item legge di Gauss: $\diver \B = 0$
    \begin{itemize}
        \item potenziale vettore: $\B = \curl \vpot$
    \end{itemize}
    \item legge di Ampère: $\curl \B = \mu_0 \dcurr$
\end{itemize}
Riassunte nell'equazione $\vlapl \vpot = -\mu_0 \dcurr$.

Parlando della sorgente del campo magnetico, abbiamo trascurato il campo elettrico generato dalle cariche in moto poiché i fili conduttori sono complessivamente neutri.

Un'altra osservazione da sospendere riguarda le contraddizioni rispetto alla relatività galileiana: il campo magnetico misurato in un sistema in quiete con una carica è nullo, ma non lo è in un sistema (sempre inerziale) in moto rispetto al primo. Risolveremo queste contraddizioni nel \autoref{sec:relativita}.
