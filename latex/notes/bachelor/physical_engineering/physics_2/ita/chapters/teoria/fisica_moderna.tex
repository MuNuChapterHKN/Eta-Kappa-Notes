\chapter{Inadeguatezza della fisica classica}

Fine Ottocento, nel contesto della seconda Rivoluzione industriale e dopo aver definito le equazioni di Maxwell.
Fino ad allora, le teorie fisiche (di Newton e Maxwell) non avevano fallito.

La meccanica classica descrive oggetti materiali che possono essere descritti come scomponibili e portano proprietà che possono essere descritte come onde.

L'elettromagnetismo introduce la carica, una proprietà ulteriore.
Le interazioni si risolvono comunque in forze, che quindi possono essere meccaniche o elettromagnetiche.

Essendo un onda, il campo elettromagnetico lo si interpreta come perturbazione di un mezzo chiamato ``etere''.

L'esperimento di Michelson e Morley, tuttavia, mostra che questo tessuto connettivo non esiste.

Inoltre, nelle equazioni di Maxwell esiste un parametro velocità ($c$) che non cambia mai.
Un dipolo in moto e uno in quiete generano campi elettromagnetici che si muovono alla stessa velocità: la velocità della luce non si compone secondo la relatività galileiana.

Einstein derivò le trasformazioni di Lorentz partendo dalla meccanica e non dall'elettromagnetismo, principalmente perché la meccanica era ritenuta vera da tre secoli mentre l'elettromagnetismo era una teoria  più nuova.
Il ragionamento dalla meccanica, però, è più complicato.

Caveat: l'applicazione di una teoria (che è assoluta, matematica) presuppone la modellizzazione del sistema fisico.
I fisici, per questo, pensavano che non fossero sbagliate le teorie, ma i modelli della materia.


\section{Corpo nero}

Tutti i corpi a temperatura finita ($T > 0$) emettono un campo elettromagnetico, poiché le cariche nella materia si muovono di moto accelerato (con oscillazioni meccaniche).

Quando questa radiazione diventa visibile si ha incandescenza e all'aumentare della temperatura il colore passa dal rosso al giallo, perché aumenta la frequenza delle onde emesse.

Maxwell definì il corpo nero per modellizzare una lampadina a incandescenza.

\important{Corpo nero}: cavità di conduttore ideale (pareti perfettamente riflettenti) a temperatura fissa.

Il campo elettromagnetico emesso dalle pareti è continuamente riassorbito e riemesso (cioè, riflesso), quindi \important{è in equilibrio termodinamico con le pareti} stesse.

Per misurare l'energia emessa, il corpo nero è dotato di un piccolo foro (sufficientemente piccolo da non perturbare l'equilibrio).

L'energia emessa viene misurata per ogni frequenza.
L'energia per unità di volume a una certa frequenza si scrive come
\begin{equation}
\label{eq:densita_frequenza}
    I(\nu) \de \nu = g(\nu) \avg{\energy} \de \nu
\end{equation}
dove $g(\nu) \de \nu$ indica il numero di onde a frequenza $\nu$ per unità di volume e $\avg{\energy}$ è l'energia media di un onda a frequenza $\nu$.

Assumiamo che il corpo nero sia un cubo di lato $L$.
Lungo ogni direzione si avranno un'onda progressiva e una regressiva con stessa frequenza e polarizzazione.

Per il principio di sovrapposizione degli effetti, nella direzione $x$,
\begin{equation}
    E(x, t) = E_p \exp{\im (k x - \omega t)} + E_r \exp{-\im (k x + \omega t)}
\end{equation}

Presso il conduttore ci sarebbe dissipazione di energia per effetto Joule, a meno che
\begin{equation}
    E(0, t) = E(L, t) \equiv 0
\end{equation}

Imponendo queste condizioni al contorno, risulta
\begin{gather}
    E_p + E_r = 0 \\
    \exp{\im k L} = \exp{-\im k L} \implies \frac{k L}{\pi} \in \Z
\end{gather}

Riassumendo, lungo la direzione $x$,
\begin{equation}
    k_x = \frac{\pi}{L} n_x, \quad n_x \in \Z^+
\end{equation}
Ci limitiamo a $n_x > 0$ poiché $n_x = 0$ non ha senso e i valori negativi corrispondono a scambiare le onde progressive e regressive. Inoltre, onde che avessero $k$ diverso si eliderebbero per interferenza distruttiva.

Per contare le onde, si ragiona nello \important{spazio reciproco}, con i numeri d'onda presso gli assi $k_x$, $k_y$, $k_z$.

Ogni punto dello spazio reciproco corrisponde a un vettore d'onda.

Le onde ammesse sono quelle con vettore d'onda $\vt{k} = (n_x, n_y, n_z) \pi/L$, nell'ottante positivo. A ogni onda corrisponde un punto, a cui corrisponde (asintoticamente) un cubetto.

A ogni frequenza corrisponde un modulo di $\vt{k}$.
Il numero di onde di frequenza compresa tra $0$ e $\nu$ è il numero di cubetti di lato $\pi / L$ in un'ottante di sfera di raggio $k = \norm{\vt{k}} = 2\pi \nu / c$.
In realtà, queste sono solo le onde polarizzate in una certa direzione.
Serve almeno un'altra direzione di polarizzazione per ottenere tutte le onde, quindi occorre moltiplicare per due.
\begin{equation}
    N = 2 \cdot \frac{1}{8} \cdot \frac{\frac{4}{3} \pi k^3}{\pts{\frac{\pi}{L}}^3}
    % = \frac{1}{3 \pi^2 c^3} \omega^3 L^3
    = \frac{8 \pi}{3 c^3} \nu^3 L^3
\end{equation}

La densità, rispetto alla frequenza, di onde per unità di volume è:
\begin{equation}
    g(\nu) = \der{\nu} \pts{\frac{N}{L^3}}
    = \frac{8 \pi}{c^3} \nu^2
\end{equation}
% \begin{equation}
%     g(\omega) = \der{\omega} \pts{\frac{N}{L^3}}
%     = \frac{1}{\pi^2 c^3} \omega^2
% \end{equation}

La \eqref{eq:densita_frequenza} ora è
\begin{equation}
    I(\nu) = \frac{8 \pi}{c^3} \nu^2 \avg{\energy}
\end{equation}

Poiché l'onda viene generata da un'oscillazione armonica di cariche, la sua energia deve corrispondere all'energia meccanica (potenziale e cinetica) media di un oscillatore armonico a temperatura $T$.

La frequenza $\omega$ degli oscillatori è un parametro puramente meccanico e dipende dalla costante elastica e dalla massa delle cariche ($k = m \omega^2$).

L'energia media del sistema termodinamico è
\begin{equation}
\label{eq:energia_media}
    \avg{\energy} = \int_0^{+\infty} \energy \, F(\energy) \, \de \energy
\end{equation}
dove $F(\energy) \, \de \energy$ è la frazione di particelle che hanno energia in $[\energy, \, \energy + \de \energy]$.

La densità $F$ è funzione della velocità, poiché lo è l'energia: $F = F(v_x, v_y, v_z)$.
Queste tre componenti sono statisticamente indipendenti, per cui esistono densità $f$ per cui
\begin{equation}
    F(v_x, v_y, v_z) = f_x(v_x) f_y(v_z) f_z(v_z)
\end{equation}
Poiché lo spazio è isotropo, $f_x = f_y = f_z \eqcolon f$.

Per la stessa ragione, poiché non può dipendere dalla scelta delle coordinate, $F$ deve dipendere solo dal modulo di $\vt{v}$ o, equivalentemente, dal suo quadrato, cioè $v_x^2 + v_y^2 + v_z^2$:
\begin{equation}
    F(v_x^2 + v_y^2 + v_z^2) = f(v_x) f(v_z) f(v_z)
\end{equation}
L'unica funzione che soddisfa è l'esponenziale:
\begin{equation}
    F(\vt{v}) = A \exp{B \pts{v_x^2 + v_y^2 + v_z^2}}
\end{equation}
con $\dimension{B} = \mathsf{T}^2 \mathsf{L}^{-2}$.

Nel caso del gas ideale,
\begin{gather}
    \int_0^{+\infty} \frac{1}{2} m \vt{v}^2 F(\vt{v}) \, \de \vt{v} = \frac{3}{2} k_B T \\
    \implies F(\vt{v}) = A \exp{-\frac{m}{2 k_B T} \vt{v}^2}
\end{gather}
È la \important{distribuzione di Maxwell}.
In termini di energia ($A$ è diversa), si chiama \important{distribuzione di Boltzmann} ed è soddisfatta da qualunque sistema termodinamico all'equilibrio:
\begin{gather}
    F(\energy) = A \exp{-\frac{\energy}{k_B T}} \\
    \int_0^{+\infty} F(\energy) \de \energy = 1
    \implies A = \frac{1}{\int_0^{+\infty} \exp{-\frac{\energy}{k_B T}} \de \energy}
\end{gather}

Segue che
\begin{equation}
    \avg{\energy} = \int_0^{+\infty} A \energy \exp{-\frac{\energy}{k_B T}} \de \energy
    = \frac{\int_0^{+\infty} \energy \exp{-\frac{\energy}{k_B T}} \de \energy}{\int_0^{+\infty} \exp{-\frac{\energy}{k_B T}} \de \energy}
    = k_B T
\end{equation}

Infatti, classicamente a ogni grado di libertà (cioè a ogni termine quadratico nell'espressione dell'energia meccanica totale) corrisponde un'energia $\frac{1}{2} k_B T$, e l'oscillatore armonico monodimensionale ha due gradi di libertà: energia cinetica e potenziale.

La \eqref{eq:densita_frequenza} ora è
\begin{equation}
    I(\nu) = \frac{8\pi}{c^3} k_B T \nu^2
\end{equation}

Sperimentalmente, $I(\nu)$ ha una forma a campana.
Questa legge, invece, indica una parabola ed è adeguata solo per la regione a frequenze basse (fin quasi al massimo della curva sperimentale).

Nelle curve sperimentali si osserva anche che il massimo di $I(\nu)$ è direttamente proporzionale alla temperatura.

Le frequenze in cui si trova il massimo se la temperatura è circa quella della superficie solare sono quelle della luce visibile.

Planck, per far tornare i risultati teorici con i dati sperimentali, ipotizza la seguente distribuzione di energia:
\begin{equation}
    \energy = j h \nu, \quad j \in \N
\end{equation}

Sostituendo nella \eqref{eq:energia_media} e passando al discreto:
\begin{equation}
    \avg{\energy} = \sum_{j = 0}^{\infty} j h \nu A \exp{-\frac{j h \nu}{k_B T}}
\end{equation}
Si impone la condizione di normalizzazione sulla distribuzione $F$:
\begin{equation}
    \sum_{j = 0}^{\infty} A \exp{-\frac{j h \nu}{k_B T}} = 1
    \implies A = \frac{1}{\sum_j \exp{-\frac{j h \nu}{k_B T}}}
\end{equation}
Dunque, definendo $\beta \coloneq 1/(k_B T)$:
\begin{equation}
    \avg{\energy} = \frac{\sum_j j h \nu \exp{-\beta j h \nu}}{\sum_j \exp{-\beta j h \nu}}
\end{equation}

Osserviamo che la derivata rispetto a $\beta$ del denominatore è
\begin{equation}
    \parder{\beta} \sum_{j = 0}^\infty \exp{-\beta j h \nu}
    = - \sum_{j = 0}^\infty j h \nu \exp{-\beta j h \nu} \implies \avg{\energy}
    = -\frac{\parder{\beta} \sum_j \exp{-\beta j h \nu}}{\sum_j \exp{-\beta j h \nu}}
\end{equation}

La somma è una serie geometrica e vale:
\begin{gather}
    \sum_{j = 0}^\infty \exp{-\beta j h \nu} = \frac{1}{1 - \exp{-\beta h \nu}} \\
    \implies
    \avg{\energy} = -\frac{\displaystyle\parder{\beta} \dfrac{1}{1 - \exp{-\beta h \nu}}}{\dfrac{1}{1 - \exp{-\beta h \nu}}}
    = \frac{h \nu}{\exp{\dfrac{h \nu}{k_B T}} - 1}
\end{gather}

Aggiorniamo $I(\nu)$:
\begin{equation}
    I(\nu) = \frac{8\pi h}{c^3} \frac{\nu^3}{\exp{\dfrac{h \nu}{k_B T}} - 1}
\end{equation}

Fittando $h$ ai dati sperimentali, si ottiene lo stesso valore per tutte le curve (cioè per ogni temperatura):
\begin{equation}
    h = 2\pi \cdot \qty{1.05e-34}{\joule\second}
\end{equation}

La distribuzione corretta arriva al prezzo di aver violato la continuità della fisica classica, aver introdotto un oscillatore che assume solo valori discreti di energia.



\section{Effetto fotoelettrico}

È l'effetto per cui, quando un'onda elettromagnetica colpisce una superficie metallica, quest'ultima emette elettroni.
È dovuto al fatto che questi sentono la forza di Lorentz dovuta al campo elettrico e possono essere strappati se il campo è sufficientemente forte.

L'onda è caratterizzata da
\begin{itemize}
    \item ampiezza (quindi, intensità $I$)
    \item frequenza $\nu$
\end{itemize}

L'effetto è caratterizzato da
\begin{itemize}
    \item energia cinetica $E_k$ degli elettroni emessi
    \item numero di elettroni emessi $N_e$
    \item tempo di uscita (o di attesa) $t$
\end{itemize}

Classicamente, ci si aspetta:
\begin{itemize}
    \item $E_k$ in uscita che cresce all'aumentare dell'intensità dell'onda.
    \item $t$ che diminuisce all'aumentare dell'intensità dell'onda.
    \item $N_e$ legato all'intensità (in modo anche complicato).
    \item che $\nu$ sia ininfluente.
\end{itemize}

Invece, Hertz trovò che:
\begin{itemize}
    \item Esiste una \important{frequenza di soglia}...
    \begin{itemize}
        \item al di sotto della quale gli elettroni non sono mai emessi, per ogni intensità
        \item al di sopra della quale sono sempre emessi, per ogni intensità.
    \end{itemize}
    \item L'energia cinetica degli elettroni
    \begin{itemize}
        \item dipende linearmente dalla frequenza
        \item non dipende dall'intensità
    \end{itemize}
        \begin{equation}
        \label{eq:fotoelettrico}
            E_k = h \nu - \phi
        \end{equation}
    \item Il tempo di uscita è sempre nullo:
        \begin{equation}
            t = 0
        \end{equation}
    \item Il numero di elettroni emessi ha il seguente andamento:
        \begin{equation}
            N_e \propto \frac{I}{h \nu}
        \end{equation}
\end{itemize}

A trovare che la costante di proporzionalità era proprio $h$ fu Einstein.

Fu chiaro che $h$ era una costante fondamentale, e venne detta \important{costante di Planck}.

Osservando la \eqref{eq:fotoelettrico}, sembra un urto elastico con una particella in cui l'elettrone guadagna energia $h \nu$.

In un certo senso, è il campo elettromagnetico (e non solo l'oscillatore armonico) a essere discreto.

L'entità che discretizza l'onda elettromagnetica si chiama \important{fotone}, immaginabile come la particella di energia $h \nu$ che urta gli elettroni.

\section{Quantità di moto di un'onda elettromagnetica}

Richiamiamo l'equazione per la densità di quantità di moto di un'onda elettromagnetica:
\begin{equation}
    p\EM  = \frac{w\EM}{c}
\end{equation}

La quantità di moto, classicamente, è associata a una massa.

Ora è possibile immaginare che il campo elettromagnetico sia costituito da fotoni, particelle di energia $h \nu$.
Quando queste urtano il materiale, ``si attaccano'' all'elettrone e gli trasferiscono energia e quantità di moto.

La massa dell'elettrone non varia nel processo, quindi il fotone non ha massa.

Detta $n_f$ la densità volumica di fotoni, la quantità di moto del singolo fotone è
\begin{equation}
\label{eq:momento_fotone}
    p = \frac{p\EM}{n_f} = \frac{w\EM}{n_f c} = \frac{h \nu}{c} = \frac{h}{\lambda} = \hbar k
\end{equation}

\subsection{Momento angolare di un'onda elettromagnetica}

Consideriamo un disco con sfere cariche sul bordo e un solenoide presso l'asse collegato a un interruttore tramite un timer.

Quando il circuito si accende, il flusso del campo magnetico genera un campo elettrico che agisce sulle cariche, mettendo il disco in rotazione con una certa velocità angolare.
Quindi, poiché il disco ha momento d'inerzia non nullo, il sistema acquisisce un momento angolare anche se ce l'aveva originariamente apparentemente nullo.

Quindi, il campo elettromagnetico deve avere anche un momento angolare.

Da queste considerazioni seguirà che:
\begin{itemize}
    \item Massa ed energia sono equivalenti.
    \item Un fotone non può che muoversi a velocità $c$.
\end{itemize}


\section{Modello atomico}

\subsection{Esperimento di Rutherford}

I chimici erano convinti che la materia fosse discreta, mentre le teorie fisiche la descrivevano come continua.

Vigeva il \important{modello di Thomson}: gran parte della massa della materia deve essere costituita di una componente positiva, in cui erano immersi gli elettroni di massa molto trascurabile.

Rutherford svolse un esperimento con una lamina d'oro sottilissima ($\sim \unit{\micro\metre}$).
Bombardò la lamina con particelle $\alpha$, positive: sarebbero dovute passare attraverso senza grandi deviazioni, invece una piccola frazione (dell'ordine di una su $10^4$) veniva riflessa.
Rutherford concluse che la carica positiva era raggruppata in nuclei di grande massa.

Il rapporto tra le dimensioni di queste particelle positive rispetto a quelle degli elettroni doveva essere paragonabile al rapporto tra le particelle $\alpha$ riflesse.

La distanza tra due nuclei risultava dell'ordine di \qty{e-10}{\metre}, le dimensioni dei nuclei \qty{e-15}{\metre}.

La materia è essenzialmente vuota, il fatto che sembri ``piena'' deriva dai campi elettromagnetici.

Viene sviluppato un \important{modello planetario} per gli atomi, in moto circolare uniforme dovuto alla legge di Coulomb.

Tuttavia, nel suo modo accelerato, l'elettrone dovrebbe generare campo elettromagnetico e perdere energia, infine collassando sul nucleo.


\subsection{Spettri di emissione e assorbimento}

In un gas scaldato gli urti tra le molecole che hanno energia differente trasferiscono energia da un atomo all'altro.
Questa energia viene emessa o assorbita dagli elettroni degli atomi, che emettono o assorbono onde elettromagnetiche.

Rydberg identificò una legge empirica per le righe spettrali dell'idrogeno:
\begin{equation}
\label{eq:rydberg}
    \frac{1}{\lambda} = R \pts{\frac{1}{m^2} - \frac{1}{n^2}}, \quad n > m, \quad n, m \in \Z^+
\end{equation}
$R$ è detta \important{costante di Rydberg}:
\begin{equation}
    R = \qty{1.09e7}{\per\metre}
\end{equation}
Quindi, gli spettri di emissione sono righe discrete, mentre quelli di assorbimento sono i loro complementari.

\subsection{Modello atomico di Bohr}

Il modello di Bohr vuole spiegare:
\begin{itemize}
    \item Le dimensioni dell'atomo
    \item La stabilità dell'atomo
    \item L'energia di ionizzazione e l'effetto fotoelettrico
    \item Il carattere discreto degli spettri di emissione e di assorbimento
\end{itemize}

Consideriamo l'atomo di idrogeno, costituito da un protone di carica $e > 0$ e massa $m_p$ e un'elettrone di carica $-e$ e massa $m \ll m_p$ che orbita attorno al protone in moto circolare uniforme a distanza $r$ e con velocità $v$, grazie a una forza centripeta elettrostatica (la forza di Coulomb):
\begin{equation}
    F = m \frac{v^2}{r} = \frac{e^2}{4 \pi \eps_0 r^2}
\end{equation}
Momento angolare:
\begin{equation}
    L = r m v
\end{equation}
Energia meccanica totale ($H$ per Hamilton):
\begin{equation}
    H = \frac{1}{2} m v^2 + (-e) \frac{e}{4 \pi \eps_0 r} = - \frac{e^2}{8 \pi \eps_0 r}
\end{equation}


Gli spettri di emissione suggerivano una discretizzazione.
Bohr impose la discretizzazione del momento angolare dell'elettrone:
\begin{equation}
    L = n \hbar, \quad n \in \Z
\end{equation}
Si introduce qui la \important{costante di Planck ridotta}:
\begin{equation}
    \hbar \coloneq \frac{h}{2\pi}
\end{equation}

Si ricava la velocità
\begin{gather}
    r m v = n \hbar
    \implies
    m v^2 = \frac{e^2}{4 \pi \eps_0} \frac{m v}{n \hbar}
    \implies v = \frac{e^2}{4 \pi \eps_0 \hbar} \frac{1}{n}
\end{gather}

Si ricava il raggio dell'orbita:
\begin{gather}
    r = \frac{n \hbar}{m v} = \frac{n \hbar}{m} \frac{4 \pi \eps_0 \hbar}{e^2} n = \frac{4\pi \eps_0 \hbar^2}{m e^2} n^2 = r_1 n^2
\end{gather}

Si ricava l'energia meccanica:
\begin{equation}
\label{eq:energia_bohr}
    H = - \frac{e^2}{8 \pi \eps_0} \frac{m e^2}{4\pi \eps_0 \hbar^2} \frac{1}{n^2} = - \frac{m e^4}{2 (4\pi \eps_0)^2 \hbar^2} \frac{1}{n^2} = -\frac{H_1}{n^2}
\end{equation}

Quindi dobbiamo avere $n \ne 0$.
Inoltre, il segno influisce solo sul verso della velocità.
Consideriamo quindi $n$ intero positivo:
\begin{equation}
    n \in \Z^+
\end{equation}

Per $n \to +\infty$, $H \to 0^-$.
Per $n = 1$, l'energia è minima e l'elettrone ruota con la velocità massima e lungo l'orbita minore, il cui raggio (\important{raggio di Bohr}) è $r_1 = \qty{5.3e-11}{\metre}$.

Non è possibile che l'elettrone abbia energia minore di $-H_1$: l'ipotesi \textit{ad hoc} di Bohr implica la discretizzazione anche dell'energia e il fatto che gli elettroni non possono perdere energia indefinitamente.
$H_1$ vale
\begin{equation}
    H_1 = \qty{13.6}{\electronvolt}
\end{equation}
e corrisponde all'energia necessaria per portare l'elettrone dallo stato fondamentale a un punto lontano dall'atomo ($H = 0$).
È l'energia di ionizzazione per l'idrogeno allo stato fondamentale.

L'unità di misura \important{elettronvolt} (\unit{\electronvolt}) è l'energia di un elettrone che attraversa la differenza di potenziale di \qty{1}{\volt}.
\begin{equation}
    \qty{1}{\electronvolt}
    = e \cdot \qty{1}{\volt}
    = (\qty{1.6e-19}{\coulomb}) (\qty{1}{\volt})
\end{equation}

La formula di Rydberg, nel caso dell'emissione, corrisponde alla variazione di energia dal livello $n$ al livello $m$:
\begin{equation}
    H_n - H_m = H_1 \pts{\frac{1}{m^2} - \frac{1}{n^2}}, \quad 1 \le m < n
\end{equation}
Questa differenza di energia diventa radiazione elettromagnetica, quindi viene emessa come fotone:
\begin{equation}
    h \nu = \frac{h c}{\lambda} = H_1 \pts{\frac{1}{m^2} - \frac{1}{n^2}}
\end{equation}
La costante di Rydberg, calcolata empiricamente, risulta correttamente prevista dalla teoria.
Confrontando con la \eqref{eq:rydberg}, infatti, risulta
\begin{equation}
    R = \frac{H_1}{h c}
\end{equation}


\section{Calore specifico}

La capacità termica $C$ di un corpo alla temperatura $T$ è il rapporto tra il calore $\delta Q$ fornito al corpo e la variazione di temperatura $\de T$ che ne risulta:
\begin{equation}
    C = \frac{\delta Q}{\de T} = \frac{\de U + \delta L}{\de T}
\end{equation}
Il calore specifico è la capacità termica per unità di massa, o quantità di sostanza $n$, o numero di particelle $N$.

Si definisce il calore specifico isocoro molare per le trasformazioni isocore (per un solido o liquido, quasi sempre):
\begin{equation}
    c_V = \frac{1}{n} \der[U]{T}
\end{equation}

\subsection{Molecola biatomica}

Nel caso della molecola di idrogeno \ce{H2} allineata lungo l'asse $x$, l'energia totale è
\begin{equation}
    U = \frac{1}{2} M v^2
    + \frac{1}{2} I_y \omega_y^2
    + \frac{1}{2} I_z \omega_z^2
    + \frac{1}{2} M v_r^2
    + \frac{1}{2} k_e x^2
\end{equation}
$M$ è la massa di entrambi i nuclei, mentre il momento d'inerzia $I_x$ è trascurabile. Il sistema, quindi, ha sette gradi di libertà.

Quindi,
\begin{equation}
    U = n N_A \frac{7}{2} k_B T \implies c_V = \frac{7}{2} R
\end{equation}
$R = N_A k_B$ qui è la costante universale dei gas perfetti.

La fisica classica prevede questo valore sempre.
Tuttavia, i due gradi di libertà corrispondenti alle oscillazioni si misurano solo per temperatura molto alte (purché non sufficientemente alte da rompere le molecole).
A temperatura ambiente $c_V = 5/2 \, R$. A temperature ancora più basse si perdono anche i gradi di libertà della rotazione: $c_V = 3/2 \, R$.

\redtext{Grafico $c_V(T)$}

Si svolge un'ipotesi \textit{ad hoc}, sostituendo al momento angolare e all'energia dell'oscillatore armonico le espressioni con le ipotesi di Planck e Bohr:
\begin{equation}
    U = \frac{1}{2} M v^2 + \frac{(n \hbar)^2}{2 I} + j \hbar \omega
\end{equation}

Consideriamo un termostato a temperatura $T = T_0 + \de T$, mentre la molecola di idrogeno ha temperatura $T_0$.

Dopo l'interazione con il termostato, la molecola avrà temperatura $T$.
Al massimo, il termostato potrà trasferire energia $3/2 \, k_B T$ (i gas monoatomici non mostrano problemi, quindi usiamo un termostato classico a gas monoatomico).
\begin{equation}
    \Delta U \le \frac{3}{2} k_B T
\end{equation}
Quest'energia deve distribuirsi equamente su tutti i gradi di libertà della molecola.

Ma questo è contraddetto dagli esperimenti, come detto sopra.

Il modo di far quadrare i conti prevede che
\begin{itemize}
    \item l'energia di traslazione vari nel continuo
    \item l'energia di rotazione assuma valori discreti $(n \hbar)^2 / (2 I)$
    \item l'energia di vibrazione assuma valori discreti $j \hbar \omega$
\end{itemize}
Ma allora, la ragione per il fenomeno sopra è che l'energia, per poter andare distribuita sulla rotazione o vibrazione, deve superare una soglia minima dovuta ai ``salti''.
È anche possibile calcolare le temperature di transizione.

\subsection{Solido cristallino}

Consideriamo un cristallo atomico con struttura cubica.
Ogni atomo è circondato da sei atomi.

Se un atomo riceve energia termica, inizia a oscillare lungo una direzione, respinto dalle coppie di atomi nella direzione lungo cui oscilla.
L'energia media di oscillazione vale il triplo di quella di un singolo oscillatore, poiché corrisponde a tre oscillatori arminici (uno per ogni direzione):
\begin{equation}
    \avg{\energy} = \frac{3 h \nu}{\exp{\dfrac{h \nu}{k_B T}} - 1}
\end{equation}
Ma quindi, il calore specifico isocoro molare vale
\begin{equation}
    c_V = \der{T} \frac{3 N_A h \nu}{\exp{\dfrac{h \nu}{k_B T}} - 1}
\end{equation}

\redtext{Grafico $c_V(T)$}

In particolare,
\begin{gather}
    c_V
    % \propto \frac{1}{T^2} \exp{- \frac{h \nu}{k_B T}}
    \to 0, \quad \text{per } T \to 0^+ \\
    c_V \to 3 R, \quad \text{per } T \to +\infty
\end{gather}
Il modello classico darebbe invece $c_V = 3 R$ per ogni temperatura.


\section{Diffrazione nei cristalli}

\subsection{Diffrazione di raggi X}

Consideriamo un cristallo con struttura cubica e passo reticolare $d \sim \qty{e0}{\angstrom}$.
Mandando un'onda elettromagnetica a una angolo $\theta$ dal piano, si osserva diffrazione su uno schermo lontano perpendicolare all'onda riflessa.

Similmente al caso di diffrazione da doppia fenditura (vedi \autoref{sec:doubleslit}),
\begin{equation}
    I = 4 I_0 \cos^2 \frac{k (r_2 - r_1)}{2}
\end{equation}
La differenza dei cammini stavolta è
\begin{gather}
    r_2 - r_1 = 2 d \sin \theta \\
    \implies I = 4 I_0 \cos^2 \pts{\pi \frac{2 d \sin \theta}{\lambda}}
\end{gather}

La condizione per i massimi di interferenza è nota come \important{legge di Bragg}:
\begin{equation}
    2 d \sin \theta_M= \lambda n, \quad n \in \Z
\end{equation}

\subsection{Diffrazione di elettroni}

Si tentò di replicare la diffrazione nei cristalli con gli elettroni.
Era difficile poiché occorreva mandare un elettrone alla volta e creare il vuoto per distanze relativamente lunghe.

Gli elettroni vengono accelerati con un condensatore a facce piane parallele (vedi \autoref{sec:acceleratore_carica}) fino ad avere una certa energia cinetica $E_k$.

Si osserva una figura di interferenza da cui è possibile calcolare la lunghezza d'onda tramite la legge di Bragg, come se si trattasse di un'onda elettromagnetica.
Graficando la quantità di moto degli elettroni $p = \sqrt{2 m E_k}$ rispetto a $1 / \lambda$ si ottiene una retta di pendenza $h$ e quindi la seguente ralazione:
\begin{equation}
    p = \frac{h}{\lambda}
\end{equation}

Questo risultato è coerente con la \eqref{eq:momento_fotone} per i fotoni e indica che anche l'elettrone esibisce dualismo onda-particella: a seconda del constesto, può essere descritto come un'onda o come una particella.

Le stesse conclusioni si potrebbero svolgere in generale, affermando che ogni particella (caratterizzata da $E$ e $p$) può anche essere descritta come un'onda (caratterizzata da $\nu$ e $\lambda$) e viceversa e che valgano le \important{relazioni di de Broglie}
\begin{subequations}
\begin{gather}
    E = h \nu \\
    p = \frac{h}{\lambda}
\end{gather}
\end{subequations}

Ma allora, è opportuno introdurre una \important{funzione d'onda} $\Psi(\p, t)$ che descriva la dinamica dell'elettrone in quanto onda, così come la legge oraria $\p(t)$ ne descrive la dinamica in quanto particella.


\section{Conclusione}

Caratteristiche comuni alle osservazioni discusse:
\begin{itemize}
    \item Gli effetti avvengono per dimensioni su scala atomica o subatomica.
    \item Nuova costante universale $h$, dal valore numerico piccolissimo.
    \item Grandezze fisiche quantizzate (cioè discretizzate):
    \begin{itemize}
        \item energia di un oscillatore armonico
        \item momento angolare
        \item energia del campo elettromagnetico
        \item quantità di moto del campo elettromagnetico
    \end{itemize}
    \item Dualismo onda-particella:
    \begin{itemize}
        \item onde descritte come particelle (effetto fotoelettrico e quantità di moto dei fotoni)
        \item particelle descritte come onde (diffrazione di elettroni)
    \end{itemize}
\end{itemize}

Discretizzando il momento angolare e l'energia dell'oscillatore armonico si spiegano tutti questi fenomeni.

Rimane da spiegare da cosa derivano queste discretizzazioni e che significato fisico abbia la funzione d'onda (cosa si propaga?).

La discretizzazione significa, ad esempio, che non ha neanche senso \textit{immaginare} un elettrone che disti dal nucleo una certa distanza $r$ qualsiasi.
Nel caso dell'oscillatore, significa che una massa che oscilla non può assumere determinate ampiezze e determinate velocità massime.
