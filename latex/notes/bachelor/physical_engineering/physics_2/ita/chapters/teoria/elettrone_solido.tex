\chapter{Fisica dello stato solido}

Abbiamo risolto l'equazione di Schrödinger nel caso di una particella che
\begin{itemize}
    \item trasla (conferma la discretizzazione)
    \item oscilla (conferma l'ipotesi \textit{ad hoc} di Planck, giusta: $\Delta E = \hbar \omega n$)
    \item ruota (conferma l'ipotesi \textit{ad hoc} di Bohr, sbagliata: $L = \hbar \sqrt{l (l + 1)}$, e non $L = \hbar l$).
\end{itemize}
Nel caso dell'effetto tunnel non si ha alcuna discretizzazione perché, dal momento che la particella può essere in tutto $\R$, non ci sono condizioni al contorno.

Studiamo ora il comportamento degli elettroni nei solidi.

\section{Modellizzazione}

Consideriamo un cristallo (struttura ordinata e periodica) a struttura cubica (ogni atomo è circondato da sei atomi, secondo i vertici di un cubo).

Non è possibile risolvere analiticamente il problema descrivendo tutti gli elettroni del sistema, poiché sono in numero dell'ordine di grandezza del numero di Avogadro.

Per non perdere di vista il messaggio fisico, si svolgono cinque ipotesi semplificative.
\begin{enumerate}
    \item $V$ non è l'energia potenziale coulombiana, ma una schiera di buche di potenziale.

    $V$ dovrebbe essere:
    \begin{equation}
        V(\p) = \sum_i -\frac{Z e^2}{4 \pi \eps_0 \norm{\p - \p_i}}
    \end{equation}
    Tuttavia, l'elettrone si trova in uno stato fondamentale molto profondo (tanto più quanto maggiore è $Z$), quindi vede una barriera praticamente verticale che può superare solo in caso di ionizzazione.

    L'energia potenziale si può quindi approssimare con una buca di potenziale a pareti finite per ogni nucleo.

    \item Il problema è monodimensionale (si può scegliere una direzione arbitraria poiché lo spazio è isotropo).
    \item Il solido ha dimensione infinita, per garantire la periodicità e limitare le definizioni a tratti
    \item Il solido contiene un solo elettrone, oltre ai nuclei positivi.
    \item Il solido è a temperatura $T = \qty{0}{\kelvin}$, così che i nuclei siano fermi.
\end{enumerate}

Descriveremo la soluzione nel caso di questo ``solido'', e poi elimineremo queste semplificazioni per avvicinarci al comportamento di un solido reale.

Questo solido è quindi rappresentabile tramite una disposizione di buche di potenziale profonde $V_0$, di ampiezza $a$ e a distanza $b$.
\begin{equation}
    % V(x) = V_0 \sum_{k = -\infty}^{\infty} [(a + b)k + a < x < (a + b)k + a + b] \\
    V(x) = \begin{cases}
        0 & x \in [0, a] \\
        V_0 & x \in [a, a + b]
    \end{cases}
\end{equation}
estesa per periodicità a tutto $\R$.
Qui, $V_0$ è l'\important{energia di ionizzazione} e $a + b \approx a$ è il \important{passo reticolare}.

% Si potrà rappresentare un cristallo a struttura cubica definendo un vettore $\vt{a}$ in modo che tutti i nuclei siano in posizioni $n \vt{a}$.
% Le tre componenti di $\vt{a}$ sono i passi reticolari nelle tre direzioni.

Condizioni:
\begin{subequations}
\label{eq:condcristallo}
    \begin{gather}
        V_0 \gg E \\
        b \ll a
    \end{gather}
\end{subequations}
con $E$ energia dell'elettrone.
Inoltre, $V$ è periodico di periodo $a + b$.

\section{Soluzione dell'equazione di Schrödinger}

Siano ``regione I'' le barriere e ``regione II'' le buche.

Equazioni di Schrödinger nelle regioni 1 e 2:
\begin{subequations}
\label{eq:schrcristallo}
    \begin{gather}
        -\frac{\hbar^2}{2 m} \parder[][2]{x} \Psi_\mathrm{I} = E \Psi_\mathrm{I}
        \implies
        \parder[][2]{x} \Psi_\mathrm{I} = - \underbrace{\frac{2 m E}{\hbar^2}}_{\alpha^2} \Psi_\mathrm{I} \\
        -\frac{\hbar^2}{2 m} \parder[][2]{x} \Psi_\mathrm{II} + V_0 \Psi_\mathrm{II} = E \Psi_\mathrm{II}
        \implies
        \parder[][2]{x} \Psi_\mathrm{II} = \underbrace{\frac{2 m (V_0 - E)}{\hbar^2}}_{\gamma^2} \Psi_\mathrm{II}
    \end{gather}
\end{subequations}

Un elettrone che si trova inizialmente in una buca, dopo un certo tempo ha probabilità finita di trovarsi ovunque.

Quindi, mi aspetto una collettività di elettroni che interagiscono con una collettività di nuclei.

Bloch dimostrò (\important{teorema di Bloch}) che quando l'equazione di Schrödinger ha un potenziale periodico, la soluzione è del tipo
\begin{equation}
\label{eq:bloch}
    \Psi(x) = u(x) \exp{\im k x}
\end{equation}
dove $k \in \R$ è un parametro di adimensionamento con dimensioni $\dimension{k} = \mathsf{L}^{-1}$ e $u$ ha la stessa periodicità del potenziale:
\begin{equation}
    u(x + a + b) = u(x) \quad \forall x \in \R
\end{equation}

Sostituendo nelle due soluzioni $\Psi_\mathrm{I}$ e $\Psi_\mathrm{II}$ nella \eqref{eq:schrcristallo}, si ottiengono due equazioni differenziali in $u_\mathrm{I}$ e $u_\mathrm{II}$
\begin{subequations}
    \begin{gather}
        \parder[][2]{x} u_\mathrm{I} + 2\im k \parder{x} u_\mathrm{I} - (k^2 - \alpha^2) u_\mathrm{I} = 0 \\
        \parder[][2]{x} u_\mathrm{II} + 2\im k \parder{x} u_\mathrm{II} - (k^2 + \gamma^2) u_\mathrm{II} = 0
    \end{gather}
\end{subequations}
con soluzioni
\begin{subequations}
    \begin{gather}
        u_\mathrm{I}(x) = \exp{-\im k x} \pts{A \exp{\im \alpha x} + B \exp{-\im \alpha x}} \\
        u_\mathrm{II}(x) = \exp{-\im k x} \pts{C \exp{\gamma x} + D \exp{-\gamma x}}
    \end{gather}
\end{subequations}

Condizioni al contorno: continuità di $\Psi$ e della sua derivata prima all'inizio e alla fine di ogni barriera.
\begin{subequations}
    \begin{gather}
        u_\mathrm{I}(0) = u_\mathrm{II}(0) \qquad
        \evalat{\der[u_\mathrm{I}]{x}}{x = 0} = \evalat{\der[u_\mathrm{II}]{x}}{x = 0} \\
    \label{eq:boundaryuii}
        u_\mathrm{I}(a) = u_\mathrm{II}(-b) \qquad
        \evalat{\der[u_\mathrm{I}]{x}}{x = a} = \evalat{\der[u_\mathrm{II}]{x}}{x = -b}
    \end{gather}
\end{subequations}

Sostituire $a$ con $-b$ per $u_\mathrm{II}$ nella \eqref{eq:boundaryuii} permette di tenere automaticamente in considerazione il fatto che $u$ è periodica e che le condizioni al contorno vanno imposte per ogni buca.

Risulta un sistema lineare omogeneo di quattro equazioni in quattro incognite:
\begin{equation}
    \begin{bmatrix}
        a_{11} & a_{22} & a_{13} & a_{14} \\
        a_{21} & a_{22} & a_{23} & a_{24} \\
        a_{31} & a_{32} & a_{33} & a_{34} \\
        a_{41} & a_{42} & a_{43} & a_{44}
    \end{bmatrix}
    \begin{bmatrix}
        A \\ B \\ C \\ D
    \end{bmatrix}
    = \begin{bmatrix}
        0 \\ 0 \\ 0 \\ 0
    \end{bmatrix}
\end{equation}
I coefficienti dipendono da $k$, $\alpha$ e $\gamma$.

Perché il sistema abbia soluzioni non banali è necessario che la matrice dei coefficienti non abbia rango massimo.
Porre il determinante uguale a zero corrisponde alla seguente equazione:
\begin{equation}
    \frac{\gamma^2 - \alpha^2}{2 \alpha \gamma} \sinh(\gamma b) \sin(\alpha a) + \cosh(\gamma b) \cos(\alpha a) = \cos\pts{k(a + b)}
\end{equation}
Si usano le seguenti approssimazioni derivate dalle condizioni nella \eqref{eq:condcristallo}:
\begin{subequations}
    \begin{gather}
        b \ll a \implies \cosh(\gamma b) \approx 1, \ \sinh(\gamma b) \approx \gamma b \\
        E \ll V_0 \implies \alpha \ll \gamma
    \end{gather}
\end{subequations}
che risultano in
\begin{equation}
\label{eq:bande}
    \begin{gathered}
        \frac{\gamma^2 - \cancel{\alpha^2}}{2 \alpha \cancel{\gamma}} \cancel{\gamma} b \sin(\alpha a) + \cancel{\cosh(\gamma b)} \cos(\alpha a) = \cos\pts{k(a + \cancel{b})} \\
        \implies \frac{\gamma^2 b}{2 \alpha} \sin(\alpha a) + \cos(\alpha a) = \underbrace{\frac{m a b V_0}{\hbar^2}}_P \frac{\sin(\alpha a)}{\alpha a} + \cos(\alpha a) = \cos(ka) \\
        \implies
        \boxed{P \, \frac{\sin(\alpha a)}{\alpha a} + \cos(\alpha a) = \cos(ka)}
    \end{gathered}
\end{equation}
Questa è la condizione che l'elettrone deve rispettare per esistere nel solido e indica quanto devono valere il parametro di Bloch $k$ e l'energia $E$ dell'elettrone (implicita in $\alpha$).

Se $V_0 = 0$, l'elettrone non vedrebbe alcun potenziale e sarebbe libero.
Varrebbe $P = 0$ e la \eqref{eq:bande} diventerebbe
\begin{gather}
    \cos(\alpha a) = \cos(k a)
    \implies
    k^2 = \alpha^2 = \frac{2 m E}{\hbar^2} = \frac{2 m}{\hbar^2} \frac{p^2}{2 m}
    \implies
    p = \hbar k
\end{gather}
$k$ ha quindi un significato fisico legato alla quantità di moto dell'elettrone.

\section{Schema a bande}

In generale, la \eqref{eq:bande} relaziona $E$ e $k$ (ma non è si può esplicitare rispetto a nessuna delle due variabili).
Non è quindi possibile che l'elettrone abbia energia e quantità di moto arbitrarie.
Il luogo dei punti $(k, E)$ che soddisfano la \eqref{eq:bande} (non nel caso di elettrone libero) è:

\redtext{Figura}

Il periodo delle curve è $2 \pi / a$, quindi i punti di massimo e minimo di $E$ rispetto a $k$ sono per
\begin{equation}
    k = \frac{\pi}{a} n, \quad n \in \N
\end{equation}

Lo stesso grafico sarebbe da interpretare in tre dimensioni, con $\vt{k} = (k_x, k_y, k_z)$.
Ogni elettrone sarà caratterizzato da una quaterna $(E, \vt{k})$.

Sono le informazioni contenute della funzione d'onda.
Questo significa che, di un elettrone in un solido, posso misurare solamente energia e quantità di moto senza indeterminazione.

In realtà manca lo spin, che non è considerato dall'equazione di Schrödinger, poiché questa tratta solo masse puntiformi e questo sarebbe contraddittorio con il concetto di momento angolare intrinseco.

Quindi, lo stato è caratterizzato da
\begin{equation}
    (E, \vt{k}, s) \longleftrightarrow \ket{\Psi, \uparrow}
\end{equation}
A destra è rappresentata la notazione di Dirac, che evidenzia come $\Psi$ e $\uparrow$ siamo elementi di spazi vettoriali diversi.
\begin{table}[!h]
    \centering
    \begin{tabular}{|c|c|c|}
        \hline
        Elettrone in un... & Caratteristiche fisiche \\
        \hline
        atomo & 4: $(n, l, m, m_s)$ \\
        solido & 5: $(E, k_x, k_y, k_z, s)$ \\
        \hline
    \end{tabular}
    \caption{Caratteristiche fisiche degli elettroni in un atomo e in un solido}
    \label{tab:caratteristiche}
\end{table}

Un elettrone in un solido, ora, è propriamente un onda (lo si vede dalla forma della \eqref{eq:bloch}).

La ``posizione'' di un onda è meglio descritta nello \important{spazio reciproco}, in cui ``vive'' il numero d'onda $k$, anziché nello spazio reale dove  ``vive'' la posizione.

La periodicità in $k$ vuol dire che
\begin{equation}
    \Psi(x; k) = \Psi\pts{x; k + \frac{2\pi}{a}}
\end{equation}
$k$ è un parametro, non l'argomento della funzione.
Questo vuol dire che al crescere della quantità di moto l'energia non necessariamente aumenta, ma si ripete.
La funzione d'onda, inoltre, si ripeterà anch'essa.

Per questa ragione, la quantità sull'asse delle ascisse viene talvolta chiamata ``pseudo-quantità di moto''.

È quindi possibile studiare $\Psi(x)$ in un unica oscillazione fondamentale: per $k \in [-\pi / a, \pi / a]$, regione detta \important{prima zona di Brillouin}.

Il grafico $E$ vs. $k$ nella prima zona di Brillouin è detto \important{schema a bande}:

\redtext{Figura}

Ogni curva rappresenta una \important{banda di energia} ``nel senso della fisica'': ci sono zone di energia in cui l'elettrone può esistere (bande di energia \important{permesse}) e zone in cui non può (bande di energia \important{proibite}).

Le proiezioni delle bande sull'asse $E$ costituiscono le bande ``nel senso dell'elettronica''.

Ogni banda, inoltre, corrisponde fisicamente da un livello energetico nel senso degli orbitali (la prima banda all'$1s$, la seconda a $2s$ e $2p$ ecc.).

\section{Solido finito}

Ora rimuoviamo l'ipotesi di estensione infinita del solido.

Un solido unidimensionale infinito, poiché periodico, può essere descritto come un'infinità di solidi uguali di lunghezza $L = N a$, che è un multiplo intero del passo reticolare $a$:
\begin{gather}
    \Psi(x) = \Psi(x + L) = \Psi(x + N a) \\
    \implies u(x) \exp{\im k x} = u(x + L) \exp{\im k (x + L)} \\
    \implies 1 = \exp{\im k L} \implies 2\pi \mid k L \\
    \implies k = \frac{2\pi}{L} n, \quad n \in \Z
\end{gather}
Poiché siamo in una dimensione,
\begin{equation}
    k_x = \frac{2\pi}{L} n_x
\end{equation}

In un intervallo $[-\pi / a, \pi / a]$ ci sono quindi solo alcuni valori discreti di $k$ possibili, separati a distanza $2\pi / L$.
I valori ammessi sono in numero
\begin{equation}
    N = \frac{\frac{2\pi}{a}}{\frac{2\pi}{L}} = \frac{L}{a}
\end{equation}
Ovvero, il rapporto tra la lunghezza del cristallo e il passo reticolare.
Questo numero è tanto alto che il carattere discreto delle bande non è osservabile sperimentalmente.

\section{Più elettroni}

Rimuoviamo l'ipotesi di un solo elettrone.

Un elettrone solo, nello schema a bande, si collocherà nel punto di minima energia, in $(E = 0, \, \vt{k} = \vt{0}, \, \uparrow)$.
Un secondo elettrone si collocherà nella stessa posizione ma, per il principio di Pauli, con spin opposto: $(E = 0, \, \vt{k} = \vt{0}, \, \downarrow)$.
Nota che ogni elettrone introdotto vede la periodicità del potenziale, quindi non ha senso pensare che possa scegliere un altro dei minimi periodici.

Come si aggiungono altri elettroni?
Il terzo si dovrà collocare in un punto contiguo (le bande sono discrete se il solido è finito) e avrà energia non nulla.
Questo vuol dire che allo zero assoluto possiamo immaginare i nuclei fermi ma gli elettroni avranno comunque energia e quantità di moto non nulle.

Se la banda ha un massimo di energia al centro anziché un minimo, allora quattro elettroni potranno posizionarsi a energia minima negli estremi della banda.

In tutto, ogni banda può ospitare $2 N$ elettroni.

Esempio: un atomo di sodio \ce{Na} ha 11 elettroni.
Se ne posizionano $2 N$ in ciascuna delle prime cinque bande e $N$ nell'ultima.
L'ultima banda sarà piena a metà e l'energia massima occupata si chiama \important{energia di Fermi} $E_F$.
I materiali cristallini che si riempiono come il sodio sono i \important{conduttori}.

Altro esempio: silicio \ce{Si}, 14 elettroni.
$2 N$ in ciascuna delle prime sette bande.
L'ottava è del tutto vuota.
I materiali cristallini che si riempiono come il silicio sono gli \important{isolanti}.
L'ultima banda piena si chiama banda di valenza (BV), la prima banda vuota si chiama banda di conduzione (BC).
La distanza tra il minimo della banda di conduzione ($E_c$) e il massimo della banda di valenza ($E_v$) si chiama \important{energy gap} o banda proibita ($E_g = E_c - E_v$).

Un semiconduttore è un materiale cristallino di ``tipo isolante'' che ha basso energy gap e può essere drogato (cioè, le sue cariche possono essere modulate tramite la sostituzione di alcuni atomi con altri elementi).

\subsection{Approssimazione parabolica}

Ora, svolgiamo lo sviluppo in serie della funzione $E(k)$ per una certa banda:
\begin{equation}
    E(k) = E(0) + \evalat{\der[E]{k}}{k = 0} k + \frac{1}{2} \evalat{\der[E][2]{k}}{k = 0} k^2 + o(k^2), \quad k \to 0
\end{equation}

In un minimo la derivata prima è nulla, quindi se la banda ha un minimo in $k = 0$ e si sceglie lì il riferimento per l'energia, si ha
\begin{equation}
\label{eq:approssparabolica}
    \boxed{
    E(k) \approx \frac{1}{2} \evalat{\der[E][2]{k}}{0} k^2
    }
\end{equation}
Questa approssimazione è detta \important{approssimazione parabolica} ed è buona per ogni punto della banda tranne la zona in cui la curva flette, agli estremi.
Si può riscrivere come
\begin{equation}
    E(k) \approx p^2 \pts{\frac{1}{2 \hbar^2} \evalat{\der[E][2]{k}}{0}} \\
    \implies \frac{1}{2 m} \approx \frac{1}{2 m^*} \coloneq \frac{1}{2 \hbar^2} \evalat{\der[E][2]{k}}{0}
\end{equation}
$m^*$ si chiama \important{massa efficace} ed è una massa surrettizia definita in modo che la particella si possa descrivere come libera.
\begin{equation}
    m^* = \frac{\hbar^2}{\evalat{\der[E][2]{k}}{0}}
\end{equation}

Normalmente la massa efficace è dello stesso ordine di grandezza della massa dell'elettrone, ma può essere anche molto diversa, soprattutto con semiconduttori particolari.

\subsection{Densità degli stati}

Definiamo ora il numero di stati energetici disponibili in un internallo $\Delta E$ intorno all'energia $E$.
Al $\Delta E$ corrisponde un $\Delta k_x$ con $2 \cdot \Delta k_x / (2\pi / L)$ stati (il fattore due è dovuto ai due possibili valori di spin).
Per considerare tutte le direzioni, immaginiamo un cubetto nello spazio reciproco che ha per vertici otto valori di $\vt{k}$ possibili.

Il numero di stati $N(k)$ che hanno quantità di moto minore di $k = \norm{\vt{k}}$, ovvero in $[0, \, k]$, è il numero di tali cubetti in una sfera di raggio $k$:
\begin{equation}
    N(k) = 2 \cdot \frac{\frac{4}{3} \pi k^3}{\pts{\frac{2\pi}{L}}^3}
\end{equation}

A energia $E(k)$,
\begin{gather}
    E(k) = \frac{\hbar^2 k^2}{2 m^*}
    \implies k = \sqrt{\frac{2 m^* E}{\hbar^2}} \\
    \implies N(k) = 2 \cdot \frac{\frac{4}{3} \pi \pts{\frac{2 m^* E}{\hbar^2}}^{3/2}}{\pts{\frac{2\pi}{L}}^3}
    = \frac{L^3}{3 \pi^2} \pts{\frac{2 m^* E}{\hbar^2}}^{3/2} \\
    \implies g(E) \coloneq \der[N]{E} = \frac{L^3}{2 \pi^2} \pts{\frac{2 m^*}{\hbar}}^{3/2} \sqrt{E} \propto \sqrt{E}
\end{gather}
$g(E)$ è la densità degli stati elettronici all'energia $E$.

\redtext{Grafico $E$ vs. $g(E)$}

\section{Distribuzione di Fermi-Dirac}

Rimuoviamo ora l'ultima ipotesi, quella della temperatura allo zero assoluto.

Consideriamo l'ultima banda di un conduttore.

Suddividiamo l'intervallo di energia della banda in intervallini $\Delta E_i$ alle energie $E_i$.
Il numero di stati in ogni intervallino è $g_i \coloneq g(E_i) \Delta E_i$.

Se $T > \qty{0}{\kelvin}$, i nuclei iniziano a muoversi e urtano gli elettroni.
Allo zero assoluto gli elettroni erano in equilibrio termodinamico con i nuclei, pur avendo energia non nulla.
Adesso, invece, deve avvernire un trasferimento di energia dai nuclei agli elettroni.
Con l'energia ricevuta dagli urti, alcuni elettroni saltano oltre il livello di Fermi.

Ogni intervallino è caratterizzato da un energia per elettrone $E_i$, numero di stati $g_i$ e numero di elettroni $n_i$.
Usiamo gli unici tre vincoli fisici indiscutibili in questa situazione:
\begin{itemize}
    \item Principio di Pauli (quantistico)

    Il numero di elettroni in un intervallino di energia non può eccedere il numero di stati disponibili:
    \begin{equation}
        n_i \leq g_i
    \end{equation}
    \item Conservazione del numero di elettroni (classico)
    \begin{equation}
        N = \sum_i n_i \text{ costante}
    \end{equation}
    \item Conservazione dell'energia (classico)

    L'energia totale deve essere pari alla somma delle energie di ogni livello:
    \begin{equation}
        U = \sum_i E_i n_i \text{ costante}
    \end{equation}
\end{itemize}

Se non si raggiungesse un equilibrio dinamico, osserveremmo i materiali variare la loro disposizione di elettroni (e quindi anche la loro chimica e le loro proprietà) nel tempo, ma così non è.
Fermi, allora, postula l'esistenza di una distribuzione (oggi detta distribuzione di Fermi-Dirac) degli elettroni negli stati disponibili che si può realizzare in un numero grandissimo di combinazioni, in modo da trascurare le altre:
\begin{equation}
    f(E_i; T) = \frac{n_i}{g_i}
\end{equation}

Il ragionamento è analogo al caso classico della teoria cinetica dei gas: la distribuzione uniforme nello spazio è la più probabile;
non è impossibile che, ad esempio, metà contenitore sia vuoto, ma questo si realizza in un modo solo.
% La situazione che osserviamo sempre è quella immensamente più probabile.

Per ricavare la distribuzione, aggiungeremo un quarto vincolo non classico: l'indistinguibilità delle particelle.

Il numero di modi in cui posso disporre $n_i$ elettroni \important{indistinguibili} in $g_i$ stati è
\begin{equation}
    \binom{g_i}{n_i} = \frac{g_i!}{n_i! (g_i - n_i)!}
\end{equation}

Il numero totale di modi $W$ di disporre gli elettroni in tutti i livelli è una funzione delle variabili $\{n_i\}$:
\begin{equation}
    % W(\{n_i\}) = \prod_i \binom{g_i}{n_i}
    W(\ldots, n_i, \ldots) = \prod_i \binom{g_i}{n_i}
\end{equation}
Questa funzione va massimizzata sotto i vincoli dati prima.

Usiamo il primo principio della termodinamica e la definizione di variazione di entropia:
\begin{equation}
    \de U = \delta Q = T \de S
    \implies \de U - T \de S = \de (U - T S) = 0
\end{equation}
L'entropia è una misura del disordine di un sistema:
\begin{equation}
    S = k_B \ln W
\end{equation}
Definiamo l'\important{energia libera} $F$:
\begin{equation}
\begin{gathered}
    F \coloneq U - T S = \sum_i n_i E_i - k_B T \ln \prod_i \binom{g_i}{n_i} = \\
    = \sum_i n_i E_i - k_B T \sum_i \pts{\ln g_i! - \ln n_i! - \ln{(g_i - n_i)!}}
\end{gathered}
\end{equation}
L'energia libera deve essere costante:
\begin{equation}
    \de F = \sum_i \parder[F]{n_i} \delta n_i = 0
\end{equation}
$\delta n_i$, dal punto di vista fisico, rappresenta lo spostamento degli elettroni da uno stato all'altro, quindi la loro somma deve essere nulla (poiché il numero totale di elettroni è costante):
\begin{equation}
    \sum_i \delta n_i = 0
\end{equation}
Fermi intuì che, allora, la derivata di $F$ rispetto a ciascuna variabile $n_i$ deve essere una costante, il \important{potenziale chimico} $\mu$:
\begin{gather}
    \mu = \parder[F]{n_i} \quad \text{costante } \forall i \\
    \implies \de F = \sum_i \parder[F]{n_i} \delta n_i = \mu \sum \delta n_i = 0
\end{gather}


Tenendo presente l'equivalenza asintotica per $n \to+\infty$ nota come approssimazione di Stirling,
\begin{gather}
    n! \sim \sqrt{2 \pi n} \pts{\frac{n}{e}}^n \\
    \implies \ln n! \sim \frac{1}{2} \ln(2 \pi n) +  n \ln n - n \sim n \ln n
\end{gather}
imponiamo che la derivata di $F$ rispetto a $n_i$ sia $\mu$:
\begin{gather}
\begin{gathered}
    \mu = \parder[F]{n_i}
    = E_i + k_B T \parder{n_i} \pts{\ln n_i! + \ln{(g_i - n_i)!}} \approx \\
    \approx E_i + k_B T \parder{n_i} \pts{n_i \ln n_i + (g_i - n_i) \ln(g_i - n_i)} = \\
    = E_i + k_B T \ln \frac{n_i}{g_i - n_i}
\end{gathered} \\
    \implies n_i = \frac{g_i}{1 + \exp{\dfrac{E_i - \mu}{k_B T}}}
\end{gather}
Per cui la distribuzione di Fermi-Dirac cercata è
\begin{equation}
    f(E_i; T) = \frac{1}{1 + \exp{\dfrac{E_i - \mu}{k_B T}}}
\end{equation}
Tornando a valori di energia continui:
\begin{equation}
    \boxed{
    f(E; T) = \frac{1}{1 + \exp{\dfrac{E - \mu}{k_B T}}}
    }
\end{equation}
È una sigmoide decrescente centrata in $\mu$:

\redtext{Grafico $f(E; T)$ vs. $E$ per $E \ge 0$ e segnati $\mu$ in ascissa e $1/2$ e $1$ in ordinata}

Per $T \to 0$, la distribuzione diventa una funzione a gradino centrata in $\mu$.
Quindi, $\mu$ è proprio l'energia di Fermi.

Consideriamo l'ultima banda (quella semipiena) per un cristallo conduttore.
Possiamo ora determinare la densità di elettroni rispetto all'energia, moltiplicando la densità degli stati e la distrinuzione di Fermi-Dirac:

\redtext{Grafici $E$ vs. $g(E)$, $f(E)$ e $g(E) f(E)$}

La regione in cui $f(E)$ passa da $1$ a $0$ è ampia solo circa $2 k_B T$ attorno a $E_F$.
A $T = \qty{300}{\kelvin}$, questo valore corrisponde a circa \qty{0.05}{\electronvolt}.

\section{Calcolo del livello di Fermi}

Supponiamo di essere a $T = \qty{0}{\kelvin}$.

Detto $n_\text{el}$ il numero di elettroni nell'ultima banda per ogni atomo, il numero di elettroni totali nell'ultima banda è
\begin{gather}
    N = n_\text{el} N_\text{atomi} = \int_0^{E_F} g(E) \de E = \\
    = \frac{L^3}{2 \pi^2} \pts{\frac{2 m^*}{\hbar}}^{3/2} \int_0^{E_F} \sqrt{E} \de E
    = \frac{L^3}{2 \pi^2} \pts{\frac{2 m^* E_F}{\hbar}}^{3/2}
\end{gather}
Qui, lo zero dell'energia corrisponde all'inizio dell'ultima banda.
Si può quindi trovare il livello di Fermi:
\begin{equation}
    \boxed{
    E_F = \frac{\hbar^2}{2 m^*} \pts{3 \pi^2 n_\text{el} \frac{N_\text{atomi}}{L^3}}^{2/3}
    }
\end{equation}

% Nel caso del sodio, ad esempio, $n_\text{el} = 1$ e risulta $E_F = \qty{4.7}{\electronvolt}$.

Si osserva anche che, in questa scala e nel caso di un conduttore, la distribuzione di Fermi-Dirac è approssimabile con una funzione a gradino centrata in $E_F$.

\section{Conduttore in un campo elettrico}

Accendiamo un campo elettrico $\vt{C}$ nel verso $-\uy$.
La forza elettrica determina una variazione di quantità di moto di ogni elettrone:
\begin{equation}
\label{eq:campocristallo}
    e C = \frac{\Delta p_x}{\Delta t}
    \implies
    \frac{e C}{\hbar} \Delta t = \frac{\Delta p_x}{\hbar} = \Delta k_x
\end{equation}
Le quantità di moto aumentano tutte nel verso positivo di $k_x$.

Si sposta prima l'elettrone con $k_x$ maggiore liberando il suo stato, poi quello successivo, e così via.
Gli stati con $k_x$ più negativi vengono lasciati liberi.
Lo schema a bande, alla fine, è riempito in modo asimmetrico.
Gli elettroni che hanno stati appaiati con uno stato con $k_x$ opposto non contribuiscono a quantità di moto netta.
Gli elettroni spaiati, invece, danno origine a una corrente elettrica netta (in direzione opposta al ``moto'' degli elettroni, che hanno carica negativa).

Il $\Delta t$ nella \eqref{eq:campocristallo} è un tempo di urto $\tau$: il tempo medio prima di urtare contro un nucleo.
Dopo ogni urto, gli elettroni vengono accelerati nuovamente.
Per questa ragione, le quantità di moto variano ralativamente poco e la velocità di deriva è circa quella che corrisponde al livello di Fermi (interpretato come energia cinetica):
\begin{equation}
    v_F = \sqrt{\frac{2 E_F}{m}} \sim \qty{e5}{\metre\per\second}
\end{equation}
Da quest'ordine di grandezza, considerando la distanza tra i nuclei, risulta $\tau \sim \qty{e-12}{\second}$.

È anche possibile stimare il tempo di urto in modo non classico.
Infatti, la frequenza di urto $1/\tau$ è proporzionale alla densità dei nuclei.
I nuclei, che sono in vibrazione, corrispondono a onde elastiche.
Si introducono quindi delle particelle, dette \important{fononi}, che rappresentano le onde con cui gli elettroni interagiscono (cioè, contro cui ``urtano'').

L'energia media di oscillazione di un atomo in un solido è
\begin{equation}
    E_\text{osc} = \frac{3}{\exp{\frac{h \nu}{k_B T}} - 1} h \nu
\end{equation}
Se ho $N$ nuclei, il numero di fononi può essere considerato
\begin{equation}
    N_\text{fon} = \frac{3 N}{\exp{\frac{h \nu}{k_B T}} - 1}
\end{equation}
ognuno con energia $h \nu$.

$\tau$ si stima quindi come
\begin{equation}
    \tau \propto \frac{1}{N_\text{fon}}
\end{equation}
