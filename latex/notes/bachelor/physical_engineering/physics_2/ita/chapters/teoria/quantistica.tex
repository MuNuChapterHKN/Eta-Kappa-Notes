\chapter{Meccanica quantistica}

\section{Equazione di Schrödinger}

Occore ora indicare un'equazione differenziale che definisca la funzione d'onda $\Psi(\p, t)$, analogamente a come la legge di Newton definisce la legge oraria $\p(t)$.

% Descriviamo la diffrazione di elettroni in termini di $\Psi$:
% \begin{subequations}
% \begin{gather}
%     \Psi_1(r_1, t) = A \exp{\im (k r_1 - \omega t)} \\
%     \Psi_2(r_2, t) = A \exp{\im (k r_2 - \omega t)}
% \end{gather}
% \end{subequations}
% \begin{gather}
%     \phi = k r_1 - \omega t - k r_2 + \omega t = k (r_1 - r_2) \\
%     I \propto \norm{\Psi_1 + \Psi_2}^2 = 4 A^2 \cos^2 \frac{k (r_1 - r_2)}{2}
% \end{gather}

Bisogna partire dal presupposto che l'elettrone sia un'onda, quindi deve valere l'equazione di d'Alembert:
\begin{equation}
    \lapl \Psi = \frac{1}{v_f^2} \parder[\Psi][2]{t}
\end{equation}
$v_f$ è la velocità di fase dell'onda, non la velocità della particella $v_p$.

Presupponiamo per $\Psi$ la seguente forma:
\begin{equation}
    \Psi(\p, t) = A \exp{\im (k r - \omega t)}
\end{equation}
Bisogna ora usare il principio di de Broglie.

Si calcola la derivata seconda rispetto al tempo di $\Psi$:
\begin{equation}
    \parder[\Psi][2]{t} = - \omega^2 \Psi
    \implies \lapl \Psi = - \frac{\omega^2}{v_f^2} \Psi
\end{equation}

Per il principio di de Broglie,
\begin{gather}
    p = k \hbar \implies
    \frac{\omega^2}{v_f^2} = k^2 = \frac{p^2}{\hbar^2} \\
    \lapl \Psi = - \frac{p^2}{\hbar^2} \Psi
\end{gather}

Sia $E = T + V$ l'energia meccanica totale della particella, con $T$ energia cinetica e $V$ energia potenziale.

Se $v_p \ll c$ allora non è necessario usare le relazioni relativistiche per energia cinetica e quantità di moto.
Se la particella è puntiforme, vale
\begin{equation}
    p = m v_p, \quad
    T = \frac{1}{2} m v_p^2 = \frac{p^2}{2 m}
\end{equation}

Quindi,
\begin{gather}
    \lapl \Psi = - \frac{2 m T}{\hbar^2} \Psi = - \frac{2 m}{\hbar^2} (E - V) \Psi \\
    \implies - \frac{\hbar}{2 m} \lapl \Psi + V \Psi = E \Psi
\end{gather}

Inoltre, usando nuovamente il principio di de Broglie nella forma $E = \hbar \omega$,
\begin{equation}
    \parder[\Psi]{t}
    = - \im \omega \Psi
    = - \im \frac{E}{\hbar} \Psi
    \implies
    E \Psi = \im \hbar \parder[\Psi]{t}
\end{equation}

Sostituendo, si ottiene l'\important{equazione di Schrödinger}:
\begin{equation}
    \boxed{
    -\frac{\hbar^2}{2 m} \lapl \Psi(\p, t) + V(\p, t) \Psi(\p, t) = \im \hbar \parder{t} \Psi(\p, t)
    }
\end{equation}
valida per
\begin{itemize}
    \item masse puntiformi
    \item velocità non relativistiche
\end{itemize}

Le ``forze'' (o meglio, l'ambiente) sono codificate nel termine dell'energia potenziale $V$, quindi è necessario che siano tutte conservative (ma tutte le interazioni fondamentali sono conservative, quindi su scala microscopica questa richiesta è sempre rispettata).

L'equazione di Schrödinger sostituisce l'equazione di Newton:
\begin{itemize}
    \item Secondo Newton, un sistema fisico evolve in quanto i punti materiali che lo costituiscono si muovono lungo curve $\p(t)$ nello spazio euclideo $\R^3$.
    \item Secondo l'equazione di Schrödinger, il sistema fisico evolve secondo la funzione d'onda $\Psi(\p, t)$.
\end{itemize}

Heisenberg aveva contemporaneamente prodotto una descrizione analoga ma differente, un diverso formalismo basato sull'algebra lineare che forniva le stesse previsioni.

\section{Interpretazione di Copenhagen}

Sorge la necessità di interpretare $\Psi$: a cosa corrisponde?

La prima cosa che viene in mente è che $\Psi$ rappresenti comunque il ``corpo'' della particella, come se la sua massa fosse dispersa in un volume maggiore, ma questo porterebbe a paradossi.

Born fornì una migliore interpretazione confrontando la diffrazione della luce e quella degli elettroni.

Nello scenario della doppia fenditura (\autoref{sec:doubleslit}), il campo elettrico in $P$ è
\begin{equation}
    \E(P) = \E_1 + \E_2
\end{equation}

Sullo schermo si vede l'intensità dell'onda:
\begin{equation}
    I(P) \propto \norm{\E}^2 = \norm{\E_1 + \E_2}^2
\end{equation}

La sorgente emette potenza $W$.
Quindi, $I(P)/W$ è la frazione di potenza per unità di superficie che arriva in $P$:
\begin{equation}
\label{eq:probphot}
    \frac{I(P)}{W}
    = \frac{N_f(P) h \nu}{N_0 h \nu}
    = \boxed{\frac{N_f(P)}{N_0}
    \propto \norm{\E_1 + \E_2}^2}
\end{equation}
dove $N_f(P)$ è il numero di fotoni per unità di superficie in $P$ e $N_0$ è il numero di fotoni emessi dalla sorgente.

$N_f(P)/N_0$ può anche essere interpretata come densità di probabilità (rispetto alla superficie sullo schermo) che un fotone emesso dalla sorgente arrivi in $P$.

Questa è un'interpretazione ``termodinamica'': la si può svolgere poiché i fotoni emessi dalla sorgente sono moltissimi, è una considerazione statistica.

Se al posto dei fotoni consideriamo gli elettroni, la \eqref{eq:probphot} diventa:
\begin{equation}
    \frac{N_e(P)}{N_0} = \abs{\Psi_1 + \Psi_2}^2
\end{equation}

Quindi, il modulo quadro di $\Psi$ rappresenta (almeno) la densità di probabilità in senso frequentistico (cioè, la frazione delle osservazioni sui casi totali) di trovare elettroni in $P$.

Ovvero,
\begin{equation}
    \mathrm{P}(\text{particella} \in \de\p) = \abs{\Psi(\p, t)}^2 \de\p
\end{equation}
$\de\p$ rappresenta un elemento di volume.

Questa interpretazione, ad oggi, è l'unica che è coerente con gli esperimenti.

Ogni altro tentativo di dare un significato fisico a $\Psi$ risulta contraddittorio logicamente o sperimentalmente.

In particolare, non è vero che la particella ha, ad esempio, una traiettoria che non si può conoscere per motivi pratici (come nel caso della termodinamica).
Il concetto di traiettoria risulta inadeguato in un senso fondamentale.
Non è in alcun modo possibile studiare con certezza la dinamica della particella: la statistica, in questo caso, non entra in gioco per circoscrivere i limiti di chi sta studiando il sistema, ma quelli del sistema stesso.

L'interpretazione di Born, poiché adottata anche da Heisenberg e discussa a Copenhagen, viene detta \important{interpretazione di Copenhagen}.


\section{Vettore di stato}

Nella meccanica classica, tutte le grandezze fisiche sono funzione della posizione e della quantità di moto, cioè le si ricava dalla legge oraria.

Come è possibile, allora, estrarre da $\Psi$ le grandezze fisiche?
A questa domanda risposero Heisenberg e Dirac.

Matematicamente, $\Psi$ è non nulla nella regione in cui è possibile trovare l'elettrone e deve essere derivabile rispetto a $\p$ e a $t$.

Se è possibile che l'elettrone sia solo nella regione $\Omega$, allora
\begin{equation}
    \p \notin \Omega \lor \p \in \partial \Omega \implies \Psi(\p, t) = 0
\end{equation}

Poiché la probabilità deve essere complessivamente $1$,
\begin{equation}
    \int_\Omega \abs{\Psi}^2 \de\p = 1
\end{equation}

In particolare, $\Psi$ deve essere una funzione \important{a quadrato sommabile} in $\Omega$.
Si scrive anche $\Psi \in L^2(\Omega)$.

Mentre il precedente integrale era reale, il seguente è in generale un numero complesso:
\begin{equation}
    \inner{\Psi_1}{\Psi_2} \coloneqq \int_\Omega \conj{\Psi_1} \Psi_2 \de\p \in \C
\end{equation}
Questa espressione costituisce un prodotto scalare.
Le funzioni d'onda, quindi, sono elementi dello spazio di Hilbert $L^2(\Omega)$.

Il formalismo di Heisenberg si basava sulla versione in vettori colonna $\C^{n,1}$ di questo spazio vettoriale.
Le formulazioni sono equivalenti poiché l'algebra lineare garantisce che due spazi vettoriali con la stessa dimensione siano isomorfi.

Dirac indica gli elementi di questo spazio vettoriale come \textit{ket} $\ket{\Psi}$, in modo che giustapposti ai \textit{bra} $\bra{\Psi}$ diano i prodotti scalari $\braket{\Psi_1}{\Psi_2}$.

Comunque sia interpretato (come funzione di uno spazio di Hilbert, vettore colonna complesso o ket), $\Psi$ è elemento di uno spazio vettoriale dotato di prodotto scalare e identifica lo stato del sistema: è un \important{vettore di stato}.

\section{Informazioni fisiche}

Le particelle sono descritte da uno vettore di stato che ``vive'' in uno spazio vettoriale e l'evoluzione di un sistema fisico corrisponde a una traiettoria in questo spazio vettoriale.
È la proiezione di questi stati a dare le informazioni fisiche nello spazio fisico $\R^3$.

Questa teoria fisica non è mai stata falsificata sperimentalmente (da fine anni '30 del Novecento).

Ogni misura della posizione della particella fornisce una posizione $\p_i$.
Prese $N$ misure, posso definire la posizione media
\begin{equation}
    \avg{\p} = \frac{1}{N} \sum_{j = 1}^N \p_j
\end{equation}

Il numero di volte in cui si trova l'elettrone in $\p_i$ è $N \abs{\Psi(\p_i)}^2$, per cui
\begin{equation}
    \avg{\p} = \frac{1}{N} \sum_{j = 1}^N \p_j N \abs{\Psi(\p_i)}^2 = \sum_{i = 1}^N \p_i \abs{\Psi(\p_i)}^2
\end{equation}
Nel continuo:
\begin{equation}
    \avg{\p} = \int_\Omega \p \abs{\Psi(\p)}^2 \de\p
    = \int_\Omega \conj{\Psi(\p)} \pts{\p \Psi(\p)} \de\p
    = \inner{\Psi}{\p \Psi}
\end{equation}
Questa è un'espressione per estrarre da $\Psi$ il valor medio della posizione.

Per quanto riguarda la quantità di moto:
\begin{equation}
    \avg{\vt{p}} = m \der{t} \inner{\Psi}{\p \Psi}
\end{equation}
In una dimensione, $\Omega = [a, b] \subset \R$ e:
\begin{equation}
    \avg{p_x} = m \der{t} \inner{\Psi}{x \Psi}
    = m \int_a^b x \parder{t}\pts{\conj{\Psi} \Psi} \de x
\end{equation}

Ora, consideriamo l'equazione di Schrödinger, moltiplichiamo per $\conj{\Psi}$ e prendiamo il complesso coniugato:
\begin{gather}
    \im \hbar \parder{t} \Psi = -\frac{\hbar^2}{2 m} \lapl \Psi + V \Psi \\
\label{eq:conj1}
    \implies \im \hbar \conj{\Psi} \parder{t} \Psi = -\frac{\hbar^2}{2 m} \conj{\Psi} \lapl \Psi + \conj{\Psi} V \Psi \\
\label{eq:conj2}
    \implies -\im \hbar \Psi \parder{t} \conj{\Psi} = -\frac{\hbar^2}{2 m} \Psi \lapl \conj{\Psi} + \Psi V \conj{\Psi}
\end{gather}
Sottraendo membro a membro la \eqref{eq:conj2} dalla \eqref{eq:conj1},
\begin{gather}
    \im \hbar \pts{\conj{\Psi} \parder{t} \Psi + \Psi \parder{t} \conj{\Psi}}
    = -\frac{\hbar^2}{2 m} \pts{\conj{\Psi} \lapl \Psi - \Psi \lapl \conj{\Psi}} \\
    \implies \im \hbar \parder{t} \pts{\conj{\Psi} \Psi}
    = -\frac{\hbar^2}{2 m} \pts{\conj{\Psi} \lapl \Psi - \Psi \lapl \conj{\Psi}} \\
    \implies  \parder{t} \pts{\conj{\Psi} \Psi} = \frac{\im \hbar}{2 m} \pts{\conj{\Psi} \lapl \Psi - \Psi \lapl \conj{\Psi}}
\end{gather}

Per cui,
\begin{gather}
    \avg{p_x}
    = m \int_a^b x \parder{t}\pts{\conj{\Psi} \Psi} \de x = \\
    = \frac{\im \hbar}{2} \int_a^b x \pts{\conj{\Psi} \parder[][2]{x} \Psi - \Psi \parder[][2]{x} \conj{\Psi}} \de x = \\
    = \frac{\im \hbar}{2} \int_a^b x \, \parder{x} \pts{\conj{\Psi} \parder{x} \Psi - \Psi \parder{x} \conj{\Psi}} \de x = \\
    = \frac{\im \hbar}{2} \pts{
        \left[
            x \pts{
                \conj{\Psi} \parder{x} \Psi
                - \Psi \parder{x} \conj{\Psi}
            }
        \right]_{x = a}^b
        - \int_a^b \pts{
            \conj{\Psi} \parder{x} \Psi
            - \Psi \parder{x} \conj{\Psi}
        } \de x
    } = \\
    = -\frac{\im \hbar}{2} \int_a^b \pts{
        \conj{\Psi} \parder{x} \Psi
        - \Psi \parder{x} \conj{\Psi}
    } \de x = \\
    = -\frac{\im \hbar}{2} \pts{
        \int_a^b \conj{\Psi} \parder{x} \Psi \,\de x
        - \left[\Psi \conj{\Psi}\right]_{x = a}^b
        + \int_a^b \conj{\Psi} \parder{x} \Psi \,\de x
    } = \\
    = - \im \hbar \int_a^b \conj{\Psi} \parder{x} \Psi \,\de x
    = \inner{\Psi}{-\im \hbar \parder{x} \Psi}
\end{gather}
I termini da valutare nelle integrazioni per parti sono nulli poiché $\Psi$ è nulla agli estremi dell'intervallo.

In tre dimensioni,
\begin{equation}
    \avg{\vt{p}} = \inner{\Psi}{-\im \hbar \grad \Psi}
\end{equation}

In generale, i valori medi delle grandezze fisiche si estraggono applicando degli operatori lineari a $\Psi$ e poi prendendo il prodotto scalare con $\conj{\Psi}$.

Risulta che è sempre questo il caso, anche per altre grandezze fisiche.
Infatti, ogni grandezza fisica è funzione di $\p$ e $\vt{p}$.
Quindi basta sostituire gli operatori per $\p$ e $\vt{p}$, che abbiamo visto essere $\p$ stessa e $-\im \hbar \grad$.
\begin{align}
    \p & \to \hat{\p} = \p \\
    \vt{p} & \to \hat{\vt{p}} = -\im \hbar \grad \\
    F(\p, \vt{p}) & \to \hat{F}(\p, -\im \hbar \grad)
\end{align}

Inoltre, i valori medi delle grandezze fisiche devono essere reali.
Cioè, per una grandezza fisica $F$ associata a un operatore $\hat{F}$,
\begin{gather}
    \conj{\avg{F}} = \avg{F}
    \implies
    \inner{\Psi}{\hat{F} \Psi} = \conj{\inner{\Psi}{\hat{F} \Psi}} = \inner{\hat{F} \Psi}{\Psi}
\end{gather}
Poiché, detto $\herm{\hat{F}}$ l'aggiunto di $\hat{F}$, vale in generale
\begin{equation}
    \inner{\Psi}{\hat{F} \Psi} = \inner{\herm{\hat{F}} \Psi}{\Psi}
\end{equation}
allora per gli operatori che corrispondono a grandezze fisiche deve valere $\hat{F} = \herm{\hat{F}}$, ovvero devono essere \textbf{hermitiani} (si dice anche ``\textbf{autoaggiunti}'').

Operatori hermitiani hanno autovalori reali.

In conclusione,
\begin{itemize}
    \item Le grandezze fisiche classiche diventano operatori lineari hermitiani.
    \item Il valore medio delle misure di una grandezza fisica $F$ è
        \begin{equation}
            \avg{F} = \inner{\Psi}{\hat{F} \Psi}
            = \int_\Omega \conj{\Psi} \hat{F} \Psi \de\p
            \in \R
        \end{equation}
\end{itemize}


\section{Equazione di Schrödinger stazionaria}

Otteniamo l'operatore associato all'energia totale (hamiltoniana) di un sistema:
\begin{gather}
    H = \frac{p^2}{2 m} + V(\p, t) \\
    \hat{H} = \frac{(-\im \hbar \grad)^2}{2 m} + V(\p, t)
    = -\frac{\hbar^2}{2 m} \lapl + V(\p, t)
\end{gather}
Sostituendo nell'equazione di Schrödinger,
\begin{equation}
\label{eq:schr_hamiltoniana}
    \hat{H} \Psi(\p, t) = \im \hbar \parder{t} \Psi(\p, t)
\end{equation}
L'energia di un sistema, quindi, è legata all'evoluzione del sistema nel tempo.

$\hat{H}$ dipende dal tempo, ma solo in quanto $V$ dipende dal tempo.
Un esempio di energia potenziale che dipende dal tempo è quella dovuta alla presenza di un campo elettromagnetico variabile nel tempo.
Occupiamoci ora di potenziali statici, che esprimono il fatto che l'ambiente è statico.
Possiamo supporre che la \eqref{eq:schr_hamiltoniana} abbia soluzione a variabili separabili:
\begin{gather}
    \Psi(\p, t) = T(t) R(\p) \\
    T \hat{H} R = \im \hbar R \der[T]{t} \\
\label{eq:variabili_separate}
    \frac{\im \hbar}{T} \der[T]{t} = \frac{\hat{H} R}{R}
\end{gather}
Poiché nella \eqref{eq:variabili_separate} i due membri sono funzioni di variabili diverse ma sempre costanti, allora ciascuno dei due è uguale alla stessa costante $A$:
\begin{equation}
    \begin{cases}
        \frac{1}{T} \der[T]{t} = -\im \frac{A}{\hbar} \\
        \hat{H} R = A R
    \end{cases}
\end{equation}
Si risolve la prima equazione differenziale:
\begin{equation}
    T(t) = B \exp{-\im \frac{A}{\hbar} t}
\end{equation}
Al variare di $V(\p)$, quindi, la componente temporale in $\Psi$ è sempre la stessa.

La seconda equazione, invece, è un'equazione agli autovalori per l'operatore $\hat{H}$.
Le soluzioni $R$ sono gli sutostati di $\hat{H}$.

Calcoliamo il valor medio dell'energia:
\begin{equation}
\begin{gathered}
    \avg{E} = \int_\Omega \conj{\Psi} \hat{H} \Psi \de\p
    = \int_\Omega \conj{\Psi} T(t) \hat{H} R(\p) \de\p = \\
    = \int_\Omega \conj{\Psi} T(t) A R(\p) \de\p
    = A \int_\Omega \conj{\Psi} \Psi \de\p
    = A
\end{gathered}
\end{equation}
$A$, quindi, è l'energia media totale del sistema;
rinominiamola $E$.

Nel caso in cui l'energia potenziale $V$ sia statica, quindi, si risolve l'\important{equazione di Schrödinger stazionaria}:
\begin{equation}
    \hat{H} \Psi(\p) = E \Psi(\p)
\end{equation}
che ha per autovalori le possibili energie del sistema e in modo che le funzioni d'onda che risolvono l'equazione di Schrödinger completa siano:
\begin{equation}
    \Psi(\p, t) = \exp{-\im \frac{E}{\hbar} t} \Psi(\p)
\end{equation}


\section{Misura fisica}

Immaginiamo di voler misurare una grandezza $F$.
Il valore medio previsto dalla teoria è
\begin{equation}
    \avg{F} = \inner{\Psi}{\hat{F}\Psi}
\end{equation}

Il valore medio ottenuto sperimentalmente dopo aver svolto $N$ misure è
\begin{equation}
    \avg{F}_s = \frac{1}{N} \sum_{j = 1}^N n_j f_j
\end{equation}
Dove $\{f_j\}$ sono gli esiti delle misure e $\{n_j\}$ sono le frequenze con cui si ottengono gli esiti.
Analogamente, $n_j / N$ è la probabilità (frequentistica) di ottenere l'esito $f_j$.

Tra i due risultati ci sarà un errore $\Delta F$:
\begin{equation}
    \avg{F} = \avg{F}_s \pm \Delta F
\end{equation}
Trattiamo $\Delta F$ come deviazione standard:
\begin{equation}
    \pts{\Delta F}^2 = \frac{1}{N} \sum_{j = 1}^N \pts{f_j - \avg{F}_s}^2
\end{equation}

Oltre all'indeterminazione sperimentale (casuale, strumentale e sistematica) c'è un'indeterminazione intrinseca alla meccanica quantistica, che non è epistemica ma un elemento del sistema fisico.
\begin{equation}
    \pts{\Delta F}^2 = \inner{\Psi}{\pts{\hat{F} - \avg{F}}^2 \Psi}
\end{equation}

L'operatore sopra deve essere hermitiano:
\begin{equation}
    \inner{\Psi}{\pts{\hat{F} - \avg{F}}^2 \Psi}
    = \inner{\pts{\hat{F} - \avg{F}} \Psi}{\pts{\hat{F} - \avg{F}} \Psi}
    = \abs{\pts{\hat{F} - \avg{F}} \Psi}^2
\end{equation}
Quindi, l'indeterminazione quantistica è nulla se e solo se è nullo l'argomento del modulo:
\begin{equation}
\label{eq:indeterminazione_nulla}
\begin{gathered}
    \Delta F = 0
    \iff \hat{F} \Psi = \avg{F} \Psi \iff \\
    \iff \avg{F} \text{ è un autovalore di } \hat{F}
    \iff \Psi \text{ è un autostato di } \hat{F}
\end{gathered}
\end{equation}

Poiché $\hat{F}$ è hermitiano, i suoi autostati $\{\Psi_i\}$ possono essere scelti ortonormali rispetto al prodotto scalare $\inner{\cdot}{\cdot}$ e costituiscono quindi una base per i vettori di stato possibili.
Ogni soluzione $\Psi$ si può quindi scrivere come combinazione lineare degli autostati:
\begin{gather}
    \Psi = \sum_i a_i \Psi_i \\
\begin{gathered}
    \avg{F}
    = \inner{\Psi}{\hat{F} \Psi}
    = \inner{\sum_i a_i \Psi_i}{\hat{F} \sum_i a_i \Psi_i}
    = \inner{\sum_i a_i \Psi_i}{\sum_i a_i \hat{F} \Psi_i} = \\
    = \sum_{i,j} \conj{a_j} a_i \inner{\Psi_j}{\hat{F} \Psi_i}
    = \sum_{i,j} \conj{a_j} a_i f_i \inner{\Psi_j}{\Psi_i}
    = \sum_{i,j} \conj{a_j} a_i f_i \delta_{j,i}
    = \sum_i \abs{a_i}^2 f_i
\end{gathered}
\end{gather}
$\inner{\Psi_j}{\Psi_i} = \delta_{j,i}$ poiché ortonormali.

Abbiamo, ora,
\begin{subequations}
\begin{gather}
    \avg{F} = \sum_i \abs{a_i}^2 f_i \\
    \avg{F}_s = \sum_i \frac{n_i}{N} f_i
\end{gather}
\end{subequations}
\important{Ogni misura fisica è un autovalore}, anche perché una misura determina esattamente lo stato e deve valere la \eqref{eq:indeterminazione_nulla}.

Dopo aver effettuato una misura, quindi, \important{il sistema è collassato in un autostato}.

Prima della misura, invece, lo stato del sistema non era necessariamente un autostato e si avevano solo dei pesi probabilistici $\{\abs{a_i}^2\}$ sull'esito della misura.
La grandezza fisica, quindi, non era definita.

\section{Indeterminazione}

Se una misura altera lo stato del sistema, è possibile misurare due grandezze fisiche $F$ e $G$ contemporaneamente e con precisione arbitraria?
Poiché lo stato del sistema è unico, deve valere:
\begin{equation}
    \begin{cases}
        \hat{F} \Psi = f \Psi \\
        \hat{G} \Psi = g \Psi
    \end{cases}
\end{equation}
Le grandezze devono quindi avere autostati comuni.

Inoltre,
\begin{gather}
    \begin{cases}
        \hat{F} \hat{G} \Psi_i = \hat{F} g_i \Psi_i = g_i \hat{F} \Psi_i = f_i g_i \Psi_i \\
        \hat{G} \hat{F} \Psi_i = \hat{G} f_i \Psi_i = f_i \hat{G} \Psi_i = f_i g_i \Psi_i
    \end{cases} \\
    \implies
    \hat{F} \hat{G} \Psi_i = \hat{G} \hat{F} \Psi_i
    \implies \pts{\hat{F} \hat{G} - \hat{G} \hat{F}} \Psi_i = 0
\end{gather}
Questo vale per ogni autostato, quindi l'operatore deve essere identicamente nullo.

Definiamo le \important{parentesi di commutazione} $[\cdot, \cdot]$:
\begin{equation}
    [\hat{F}, \hat{G}] \coloneq \hat{F} \hat{G} - \hat{G} \hat{F}
\end{equation}

Si conclude che deve valere:
\begin{equation}
    [\hat{F}, \hat{G}] = 0
\end{equation}
Due grandezze fisiche possono essere misurate contemporaneamente con precisione arbitraria se e solo se gli operatori associati commutano.

Ad esempio:
\begin{gather}
    [\hat{x}, \hat{p}_x] \Psi = - x \im \hbar \parder{x} \Psi - \pts{-\im \hbar \parder{x} \pts{x \Psi}} = \im \hbar \Psi
    \implies [\hat{x}, \hat{p}_x] = \im \hbar
\end{gather}
Quindi, posizione e quantità di moto lungo la stessa componente non commutano.

Vale, in particolare, il \important{principio di indeterminazione di Heisenberg}:
\begin{equation}
    \Delta x \Delta p_x \ge \frac{\hbar}{2}
\end{equation}

% \begin{equation}
%     \Delta F \Delta G \geq \frac{1}{2} \abs{\avg{[\hat{F}, \hat{G}]}}
% \end{equation}


\section{Principi della meccanica quantistica}

\begin{itemize}
    \item Principio di corrispondenza: la meccanica quantistica deve tendere alla meccanica classica per sistemi macroscopici.
    \item Assioma dell'osservabile: i sistemi fisici sono descritti da un vettore di stato (nell'equazione di Schrödinger, è una funzione a valori complessi; nella formulazione di Dirac, un \textit{ket}; Heisenberg ha usato un altro formalismo).
    \begin{itemize}
        \item Interpretazione di Copenhagen: il vettore di stato è connesso alla probabilità in senso frequentistico, cioè la frazione di volte in cui una misura dà un certo risultato.
    \end{itemize}
    \item Assioma dell'operatore: le grandezze fisiche sono associate a operatori lineari hermitiani.
    Questi, come le grandezze classiche, sono funzione degli operatori posizione e quantità di moto.

    Il valore medio di una grandezza fisica si estrae applicando l'operatore e prendendo il prodotto scalare con il vettore di stato.
    \item Principio del collasso: una misura fa collassare il vettore di stato in un autostato dell'operatore associato alla grandezza misurata.

    Ne seguono:
    \begin{itemize}
        \item Principio di indeterminazione
        \item Principio di complementarietà: esistono diversi massimi insiemi di informazioni che posso conoscere di un sistema fisico, poiché serve che il sistema collassi in uno stato che sia autostato di ogni grandezza misurata.
    \end{itemize}
    \item Principio di Pauli: due fermioni (particelle che hanno spin semintero) non possono occupare lo stesso stato quantico.
    Si basa interamente su evidenza sperimentale, non può essere dedotto teoricamente.
    \item Principio di indistinguibilità delle particelle: due particelle identiche non possono essere distinte, ``etichettate''.
    Si pensi alle onde: non ha senso chiedersi se due onde che collidono si sovrappongono o si riflettono.

    È per questo che non vale più la distribuzione di Maxwell-Boltzmann: si basa sul fatto che ogni particella abbia un'``identità''.
\end{itemize}


Fondazione assiomatica della meccanica quantistica:
\begin{itemize}
    \item Principio dell'osservabile
    \item Principio dell'operatore
    \item Principio dell'evoluzione temporale: lo stato fisico evolve in quanto evolve il vettore di stato, purché inizialmente si conosca l'informazione massima sul sistema.
\end{itemize}
