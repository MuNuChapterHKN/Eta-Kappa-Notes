%! TEX root = analisi_2.tex

\documentclass[italian,12pt,toc=sections]{HKNdocument}
% Packages
\usepackage{listings}                    % Code highlighting
\usepackage{xcolor}                      % Custom colors
\usepackage{longtable}                   % Breakable tables
\usepackage{ulem}                        % Underline
\usepackage{contour}                     % Border around text
\usepackage{tcolorbox}                   % Custom boxes

% Primary (Accent) Colors
% Primary (Accent) Colors
\definecolor{accentYellow}{RGB}{254, 196, 41}  % #FEC421
\definecolor{accentRed}{RGB}{236, 45, 36}      % #EC2D24

% Secondary Colors
\definecolor{supportOrange}{RGB}{242, 183, 5}  % #F2B705
\definecolor{supportDarkBlue}{RGB}{55, 81, 113} % #375171

% Background Colors
\definecolor{backgroundLight}{RGB}{242, 242, 242} % #F2F2F2

% Text & Border Colors
\definecolor{textGrayBlue}{RGB}{100, 117, 140}   % #64758C
\definecolor{textGrayMedium}{RGB}{146, 154, 166}  % #929AA6
\definecolor{textGrayLight}{RGB}{184, 187, 191}   % #B8BBBF



% Listings style
\lstdefinestyle{hkn}{
  basicstyle=\ttfamily\small\color{textGrayBlue},                         % Base style (size and font)
  keywordstyle=\bfseries\color{accentRed},           % Keywords in red (important, eye-catching)
  identifierstyle=\color{supportDarkBlue},               % Identifiers in blue (clear distinction)
  commentstyle=\color{textGrayMedium},                 % Comments in gray-blue (less prominent)
  stringstyle=\color{supportOrange},                 % Strings in orange (warm and readable)
  numberstyle=\ttfamily\scriptsize\color{textGrayMedium}, % Line numbers in gray (non-intrusive)
  backgroundcolor=\color{backgroundLight},           % Light background for contrast
  rulecolor=\color{textGrayLight},                   % Soft gray border for structure
  frame=single,                          % Border around code (single, double, shadowbox, none)
  framerule=0.8pt,                       % Border thickness
  frameround=tttt,                       % Round all corners
  framesep=5pt,                          % Distance between border and code
  rulesep=2pt,                           % Distance between border and code line
  numbers=left,                          % Line number position (left, right, none)
  stepnumber=1,                          % Line number interval
  numbersep=10pt,                        % Distance between line numbers and code
  xleftmargin=30pt,                      % Left margin
  xrightmargin=30pt,                     % Right margin
  resetmargins=true,                     % Reset margins
  numberblanklines=false,                % Number blank lines
  firstnumber=auto,                      % Initial line number
  columns=fixed,                         % Fixed column width
  showstringspaces=false,                % Show spaces in strings
  tabsize=2,                             % Tab size
  breaklines=true,                       % Automatic line break for long lines
  breakatwhitespace=true,                % Line break at whitespace
  breakautoindent=true,                  % Automatic indentation after line break
  escapeinside={(*@}{@*)}                % LaTeX commands in code
}

% Underline settings
\renewcommand{\ULdepth}{1.8pt} % Underline depth
\contourlength{0.8pt}

% Custom underline command
\newcommand{\myuline}[1]{%
\uline{\phantom{#1}}%
\llap{\contour{white}{#1}}%
}

% tcolorbox color settings
\definecolor{tcolorboxLeftColor}{RGB}{2, 65, 191}
\definecolor{tcolorboxBackTitleColor}{RGB}{119, 152, 255}
\definecolor{tcolorboxBackColor}{RGB}{210, 226, 255}

% Custom boxes
\newtcolorbox[auto counter, number within=chapter]{definition}[1]{
  title={\iflanguage{italian}{Definizione}{Definition}~\arabic{\tcbcounter}.~#1},
  boxrule=0mm,                       % Bordo principale (disabilitato)
  leftrule=1mm,                    % Bordo sinistro principale
  arc=2mm,
  colframe=accentRed,       % Colore bordo
  colbacktitle=textGrayMedium,
  colback=backgroundLight,        % Colore sfondo
  fonttitle=\bfseries,
  rounded corners=all,               % Bordi arrotondati
  }

\newtcolorbox[auto counter, number within=chapter]{theorem}[1]{
  title={\iflanguage{italian}{Teorema}{Theorem}~\arabic{\tcbcounter}.~#1},
  boxrule=0mm,                       % Bordo principale (disabilitato)
  leftrule=1mm,                    % Bordo sinistro principale
  arc=2mm,
  colframe=accentYellow,       % Colore bordo
  colbacktitle=textGrayMedium,
  colback=backgroundLight,        % Colore sfondo
  fonttitle=\bfseries,
  rounded corners=all,               % Bordi arrotondati
}

\newtcolorbox[auto counter, number within=chapter]{corollary}[1]{
  title={\iflanguage{italian}{Corollario}{Corollary}~\arabic{\tcbcounter}.~#1},
  boxrule=0mm,                       % Bordo principale (disabilitato)
  leftrule=1mm,                    % Bordo sinistro principale
  arc=2mm,
  colframe=supportOrange,       % Colore bordo
  colbacktitle=textGrayMedium,
  colback=backgroundLight,        % Colore sfondo
  fonttitle=\bfseries,
  rounded corners=all,               % Bordi arrotondati
}

\newtcolorbox[auto counter, number within=chapter]{exercise}[1]{
  title={\iflanguage{italian}{Esercizio}{Exercise}~\arabic{\tcbcounter}.~#1},
  boxrule=0mm,                       % Bordo principale (disabilitato)
  leftrule=1mm,                    % Bordo sinistro principale
  arc=2mm,
  colframe=supportDarkBlue,       % Colore bordo
  colbacktitle=textGrayMedium,
  colback=backgroundLight,        % Colore sfondo
  fonttitle=\bfseries,
  rounded corners=all,               % Bordi arrotondati
}

\newtcolorbox[auto counter, number within=chapter]{observation}[1]{
  title={\iflanguage{italian}{Osservazione}{Observation}~\arabic{\tcbcounter}.~#1},
  boxrule=0mm,                       % Bordo principale (disabilitato)
  leftrule=1mm,                    % Bordo sinistro principale
  arc=2mm,
  colframe=textGrayBlue,       % Colore bordo
  colbacktitle=textGrayMedium,
  colback=backgroundLight,        % Colore sfondo
  fonttitle=\bfseries,
  rounded corners=all,               % Bordi arrotondati
  }




%adeguare formattazione

\newcommand{\R}{\mathbb{R}}
\newcommand{\inc}{\subseteq}


\usepackage{verbatim}
\usepackage{hyperref}

%Ho creato questi comandi personalizzati perchè nel frattempo sono cambiati, questa
% sintassi permette di usarli in italiano senza cambiare tutte le occorrenze nel file
\newenvironment{definizione}{\begin{definition}}{\end{definition}}
\newenvironment{teorema}{\begin{theorem}}{\end{theorem}}
\newenvironment{osservazione}{\begin{observation}}{\end{observation}}
\newenvironment{esercizio}{\begin{exercise}}{\end{exercise}}
\newenvironment{corollario}{\begin{corollary}}{\end{corollary}}

\begin{document}

\title{Analisi 2}
\shorttitle{Analisi 2}
\date{\today}
\author{Tommaso Pignatelli}

% You can include multiple \editor commands to list all editors in the frontpage.
\editor{Paolo Gialli}
\editor{Sofia Blu}
\editor{Carlo Grigi}
\editor{Elisa Marroni}
\editor{Marco Violetti}

\organization{IEEE-HKN Mu Nu Chapter}
\docdate{Primo semestre 2024/2025}
\docversion{1.1}
\doclogo{hkn_logo_blu.png}
\doclicense{CC_BYNCND.png}

\frontmatter
% Create the title page
\maketitle

% Table of Contents after the title page
\clearpage
\cclicense
\clearpage
\tableofcontents
\clearpage

\mainmatter
%\chapter{Introduzione}

I seguenti appunti sono relativi al corso di Analisi 2 tenuto dal professor Paolo Tilli presso il Politecnico di Torino.
Il programma di analisi 2 a cui si fa riferimento riguarda principalmente 2 argomenti: funzioni di più variabili e serie.





\chapter{Calcolo differenziale}


\section{Definizioni}
	\begin{definizione}{Gradiente di $f$}
	Sia $f:\Omega \inc \R^n \rightarrow \R$ con $\Omega$ aperto e sia $\vec x_0 \in \Omega$, si dice \textbf{gradiente} di $f$ in $\vec x_0$ il vettore:
	$$\nabla f(\vec x_0)=\left( \frac{\partial f}{\partial x_1}(\vec x_0), \dots, \frac{\partial f}{\partial x_n}(\vec x_0)\right)$$
  \end{definizione}
	\begin{osservazione}{}
	La \textbf{derivata direzionale}  di $f$ lungo la direzione individuata da un vettore $\vec u$ si ottiene derivando per il vettore normalizzato $\vec v=\frac{\vec u}{|\vec u|}.$
  \end{osservazione}
	\begin{definizione}{Matrice Jacobiana}
	Sia $f:\Omega \inc \R^n \rightarrow \R^m$ con $\Omega$ aperto e sia $f(\vec x)=\left(f_1(\vec x), \dots, f_m(\vec x)\right)$, si dice \textbf{matrice Jacobiana} di $f$ in $\vec x_0$ la matrice:
	$$Jf(\vec x_0)=
	\begin{Bmatrix}
		\cdots & \nabla f_1(\vec x_0) & \cdots \\
		\cdots & \vdots  & \cdots \\
		\cdots & \nabla f_n(\vec x_0) & \cdots \\
	\end{Bmatrix} $$
\end{definizione}
	\begin{definizione}{Matrice Hessiana}
	Sia $f:\Omega \inc \R^n \rightarrow \R^m$ con $\Omega$ aperto, se $\exists \ \nabla f(\vec x)\ \forall \vec x \in B_{r_0}(\vec x_0)$ allora si dice \textbf{matrice Hessiana} di $f$ in $\vec x_0$ la matrice:
	$$J(\nabla f))(\vec x_0)=
	\begin{Bmatrix}
		\cdots & \nabla (\frac{\partial f}{\partial x_1})(\vec x_0) & \cdots \\
		\cdots & \vdots  & \cdots \\
		\cdots & \nabla (\frac{\partial f}{\partial x_n})(\vec x_0) & \cdots \\
	\end{Bmatrix} $$
  \end{definizione}
	\begin{definizione}{Classe $C^1$}
	Sia $f:\Omega \inc \R^n \rightarrow R^m$ con $\Omega$ aperto e con $f(\vec x)=\left(f_1(\vec x),\dots,f_m(\vec x)\right)$, $f$ si dice di \textbf{classe} $C^1$ in $\Omega$ se $\exists \ \frac{\partial f_i}{\partial x_j}(\vec x) \quad \forall \vec x \in \Omega$ e se queste sono continue.
  \end{definizione}
	\begin{definizione}{Classe $C^2$}
	Sia $f:\Omega \inc \R^n \rightarrow R^m$ con $\Omega$ aperto, $f$ si dice di \textbf{classe} $C^2$ in $\Omega$ se $\exists \ \frac{\partial^2 f_i}{\partial x_j x_i}(\vec x) \quad \forall \vec x \in \Omega$ e se queste sono continue.
  \end{definizione}
	\begin{definizione}{Differenziabilità}
	Sia $f:\Omega \inc \R^n \rightarrow R^m$ con $\Omega$ aperto, $f$ si dice \textbf{differenziabile} in $\vec x_0  \in \Omega$ se $\exists \ \nabla f(\vec x_0)=\left(\frac{\partial f}{\partial x_1}(\vec x_0), \dots , \frac{\partial f}{\partial x_n}(\vec x_0)\right)$ e se ammette uno sviluppo di Taylor di ordine 1 in $\vec x_0$ con resto di Peano, che significa che:
	$$f(\vec x)=f(\vec x_0)+\nabla f(\vec x_0)\cdot (\vec x - \vec x_0)+ o(|\vec x-\vec x_0|) \quad \text{per }\vec x \rightarrow \vec x_0$$
  \end{definizione}


	\section{Formule del piano tangente, gradiente e Taylor}


	\begin{definizione}{Piano tangente}
	Sia $f:\Omega \inc \R^n \rightarrow R^m$ differenziabile in $(x_0,y_0) \in \Omega$, si dice \textbf{piano tangente} al grafico di $f$ in $(x_0,y_0,f(x_0,y_0))$ il piano:
	$$z=f(x_0,y_0)+\frac{\partial f}{\partial x}(x_0,y_0)(x-x_0)+\frac{\partial f}{\partial y}(x_0,y_0)(y-y_0)$$
  \end{definizione}
	\begin{teorema}{Formula del gradiente}
	Sia $f:\Omega \inc \R^n \rightarrow \R$, con $\Omega$ aperto, differenziabile in $\vec x_0 \in \Omega$, sia poi $\vec v \in \R^n, v \ne 0$, allora:
	$$\frac{\partial f}{\partial \vec v}(\vec x_0) = \nabla f(\vec x_0) \cdot \vec v$$
  \end{teorema}
	\begin{teorema}
	Se $f:\Omega \inc \R^n \rightarrow \R$ è differenziabile in $(x_0, y_0)$, allora $\nabla f(x_0, y_0)$ individua la direzione di massima pendenza del grafico di $f$ in quel punto.
  \end{teorema}
	\begin{teorema}{Formula di Taylor di ordine 2  con resto di Peano}
	Sia $f:\Omega \inc \R^n \rightarrow R$ di classe $C^2$ e sia, allora:
	$$f(x)=f(x_0,y_0)+\frac{\partial f}{\partial x}(x_0,y_0)(x-x_0)+\frac{\partial f}{\partial y}(x_0,y_0)(y-y_0)+$$
	$$+\frac 1 2 \frac{\partial^2 f}{\partial x^2}(x_0,y_0)(x-x_0)^2+\frac{\partial^2 f}{\partial x \partial y}(x_0,y_0)(x-x_0)(y-y_0)+\frac 1 2\frac{\partial^2 f}{\partial y^2}(x_0,y_0)(y-y_0)^2$$
  \end{teorema}


	\section{Massimi e minimi di una funzione}


	\begin{definizione}{Punto stazionario}
	Se $\nabla f (\vec x_0)=\vec 0$ allora $\vec x_0$ viene chiamato \textbf{punto stazionario}, o punto critico, di f.
  \end{definizione}
	\begin{definizione}{Massimi e minimi}
	Dato il $det$ della matrice Hessiana:
	$$detHf(x,y)=\begin{Bmatrix}
		\frac{\partial^2 f}{\partial x^2} & \frac{\partial^2 f}{\partial x \partial y}\\
		\frac{\partial^2 f}{\partial x \partial y} & \frac{\partial^2 f}{\partial y^2}
	\end{Bmatrix}=\left( \frac{\partial^2 f}{\partial x^2} \frac{\partial^2 f}{\partial y^2} \right)- \left(\frac{\partial^2 f}{\partial x \partial y}\right)^2$$
	semplificando la procedura possiamo dire che:
	$$det Hf(\vec x_0)>0 \wedge \frac{\partial^2 f}{\partial x^2}>0 \quad \Rightarrow \quad \vec x_0 \text{ punto di minimo}$$
	$$det Hf(\vec x_0)>0 \wedge \frac{\partial^2 f}{\partial x^2}<0 \quad \Rightarrow \quad \vec x_0 \text{ punto di massimo}$$
	$$det Hf(\vec x_0)<0 \quad \Rightarrow \quad \vec x_0 \text{ punto di sella}$$
  \end{definizione}

\chapter{Integrali doppi}

\section{Introduzione}
In questo capitolo ci occuperemo dello studio degli integrali doppi, ovvero integrali di funzioni di due variabili reali.\\

Gli ingredienti necessari per un integrale doppio sono:
\begin{itemize}
\item Un insieme $\Omega \in \R^2$, aperto e limitato su cui integrare la funzione.
\item Una funzione $f:\Omega \rightarrow \R$ continua e limitata.
\end{itemize}

L'obiettivo è definire l'integrale doppio di $f$ su $\Omega$
\[
\iint_{\Omega} f(x,y) \, dx \, dy
\]
e capirne il significato.\\

Per arrivare all'integrale doppio effettuiamo le seguente costruzione in più passaggi:
\begin{enumerate}
  \item Scelgo $\epsilon>0$ e divido $\R^2$ in quadrati di lato $\epsilon$.
  \item Chiamo $Q_{\epsilon}$ la famiglia di quadrati interamente contenuti in $\Omega$. (Saranno in numero finito poichè $\Omega$ è limitato).
  \item Campiono $f(x,y)$ su ogni quadratino della famiglia, cioè prendo un punto $P_i$ in ogni quadratino $q_i\in Q_{\epsilon}$ e calcolo la somma: $S(\epsilon)= \sum_{q_i \in Q_{\epsilon}} f(P_i) \cdot \text{area}(q_i)$.
  \item Si dimostra che esiste finito il limite: $\lim_{\epsilon \to 0} S(\epsilon)$.
\end{enumerate}

\begin{definizione}{Integrale doppio}
  Seguendo quanto detto detto prima, si chiama \textbf{integrale doppio} di $f$ su $\Omega$ il valore:
  \[
  \lim_{\epsilon \to 0} \sum_{q_i \in Q_{\epsilon}} f(P_i) \cdot \text{area}(q_i) = \iint_{\Omega} f(x,y) \, dx \, dy
  \]
\end{definizione}

\section{Significati dell'integrale doppio}
Diamo ora alcuni possibili significati e interpretazioni dell'integrale doppio di una funzione $f(x,y)$ su un insieme $\Omega$:

\begin{itemize}
  \item Se $f(x,y)=1$ allora $\iint_{\Omega} f(x,y) \, dx \, dy$ è l'area di $\Omega$.
  \item La quantità $\frac{\iint_{\Omega} f(x,y) \, dx \, dy}{\text{area}(\Omega)}$ è il valore medio pesato di $f$ su $\Omega$.
  \item Se $f(x,y)=x$ allora $\iint_{\Omega} f(x,y) \, dx \, dy$ è la coordinata $x$ del baricentro di $\Omega$.
  \item Se $f(x,y)=y$ allora $\iint_{\Omega} f(x,y) \, dx \, dy$ è la  coordinata $y$ del baricentro di $\Omega$.
  \item Se $f(x,y)$ è una densità di massa allora $\iint_{\Omega} f(x,y) \, dx \, dy$ è la massa di $\Omega$.
  \item Se $f(x,y)=x^2+y^2$ allora $\iint_{\Omega} f(x,y) \, dx \, dy$ è il momento di inerzia di $\Omega$ rispetto all'origine.
\end{itemize}

\section{Calcolo dell'integrale doppio}
In questa sezione vedremo alcuni metodi per il calcolo degli integrali doppi.\\

\begin{definizione}{Semplice per fili verticali}
  $\Omega$ si dice \textbf{semplice per fili verticali} se è del tipo: $\Omega = \{ (x,y) \in \R^2 : a \leq x \leq b, g(x) \leq y \leq h(x) \}$, dove $g,h:[a,b] \rightarrow \R$ sono continue e $g(x) \leq h(x) \ \forall x \in [a,b]$.
\end{definizione}

Se $\Omega$ è semplice per fili verticali allora:
\[
\iint_{\Omega} f(x,y) \, dx \, dy = \int_{a}^{b} \left( \int_{g(x)}^{h(x)} f(x,y) \, dy \right) \, dx
\]

\begin{osservazione}{}
  Nell'integrale più interno la $x$ ha il ruolo di un parametro.
\end{osservazione}

\begin{definizione}{Semplice per fili orizzontali}
  $\Omega$ si dice \textbf{semplice per fili orizzontali} se è del tipo: $\Omega = \{ (x,y) \in \R^2 : c \leq y \leq d, p(y) \leq x \leq q(y) \}$, dove $p,q:[c,d] \rightarrow \R$ sono continue e $p(y) \leq q(y) \ \forall y \in [c,d]$.
\end{definizione}

Se $\Omega$ è semplice per fili orizzontali allora:
\[
\iint_{\Omega} f(x,y) \, dx \, dy = \int_{c}^{d} \left( \int_{p(y)}^{q(y)} f(x,y) \, dx \right) \, dy
\]

Tuttavia, per quanto i due metodi sembrino molto simili, ci sono casi in cui l'asimmetria del problema rende uno dei due inutile e l'altro molto comodo.\\


Ad esempio, onsideriamo il triangolo $\Omega$ con vertici $(0,0)$, $(2,0)$ e $(2,4)$. Vogliamo calcolare l'integrale doppio $\iint_{\Omega} e^{x^2} \, dx \, dy$.

Proviamo prima con il metodo dei fili orizzontali.\\
Dobbiamo esprimere $\Omega$ come:
\[
\Omega = \{ (x,y) \in \R^2 : 0 \leq y \leq 4, p(y) \leq x \leq q(y) \}
\]
Nel nostro caso, $p(y) = 0$ e $q(y) = 2$ per $0 \leq y \leq 4$. Quindi:
\[
\iint_{\Omega} e^{x^2} \, dx \, dy = \int_{0}^{4} \left( \int_{0}^{2} e^{x^2} \, dx \right) \, dy
\]
Tuttavia, l'integrale interno $\int_{0}^{2} e^{x^2} \, dx$ non è elementare, quindi questo metodo non è conveniente.\\

Proviamo invece con il metodo dei fili verticali.\\
In questo caso, dobbiamo esprimere $\Omega$ come:
\[
\Omega = \{ (x,y) \in \R^2 : 0 \leq x \leq 2, g(x) \leq y \leq h(x) \}
\]
Nel nostro caso, $g(x) = 0$ e $h(x) = 2x$ per $0 \leq x \leq 2$. Quindi:
\[
\iint_{\Omega} e^{x^2} \, dx \, dy = \int_{0}^{2} \left( \int_{0}^{2x} e^{x^2} \, dy \right) \, dx
\]
Poiché $e^{x^2}$ è costante rispetto a $y$, possiamo semplificare l'integrale interno:
\[
\int_{0}^{2x} e^{x^2} \, dy = e^{x^2} \int_{0}^{2x} \, dy = e^{x^2} \cdot 2x
\]
Quindi l'integrale doppio diventa:
\[
\iint_{\Omega} e^{x^2} \, dx \, dy = \int_{0}^{2} 2x e^{x^2} \, dx
\]
Facciamo il cambio di variabile $u = x^2$, quindi $du = 2x \, dx$:
\[
\int_{0}^{2} 2x e^{x^2} \, dx = \int_{0}^{4} e^u \, du = e^u \bigg|_{0}^{4} = e^4 - e^0 = e^4 - 1
\]
Quindi:
\[
\iint_{\Omega} e^{x^2} \, dx \, dy = e^4 - 1
\]
Abbiamo visto dunque che il metodo dei fili verticali è molto più conveniente in questo caso.

\section{Cambiamento di variabile}
In questa sezione ci occupiamo di definire i cambiamenti di variabili in due variabili.\\
Vorremmo descrivere un integrale doppio $\iint_{\Omega} f(x,y) \, dx \, dy$ tramite due nuove variabili $u$ e $v$.\\
Date $x = g_1(u,v)$ e $y = g_2(u,v)$, $\varphi(u,v)=(g_1(u,v)$,$y = g_2(u,v))$ è una funzione invertibili e di classe $C^1$.

\begin{teorema}{Teorema di cambiamento di variabile}
  Nelle ipotesi precedenti si ha: $\iint_{\Omega} f(x,y) \, dx \, dy = \iint_{\Omega'} f(g_1(u,v),g_2(u,v)) \cdot |J_{\varphi}(u,v)| \, du \, dv$, dove $J_{\varphi}(u,v)$ è il determinante della matrice jacobiana di $\varphi$.
\end{teorema}

\begin{definizione}{Matrice Jacobiana}
  La \textbf{matrice Jacobiana} di una funzione $\varphi: \R^m \rightarrow \R^n$ è la matrice $n \times m$ delle derivate parziali:
  \[
  J_{\varphi}(u_1, u_2, \ldots, u_m) = \begin{pmatrix}
  \frac{\partial g_1}{\partial u_1} & \frac{\partial g_1}{\partial u_2} & \cdots & \frac{\partial g_1}{\partial u_m} \\
  \frac{\partial g_2}{\partial u_1} & \frac{\partial g_2}{\partial u_2} & \cdots & \frac{\partial g_2}{\partial u_m} \\
  \vdots & \vdots & \ddots & \vdots \\
  \frac{\partial g_n}{\partial u_1} & \frac{\partial g_n}{\partial u_2} & \cdots & \frac{\partial g_n}{\partial u_m}
  \end{pmatrix}
  \]
\end{definizione}

\begin{osservazione}{}
  $|J_{\varphi}(u,v)|$ si può vedere geometricamente come il fattore di scala del cambiamento di variabile.
\end{osservazione}

\subsection{Coordinate polari}
Consideriamo il passaggio alle coordinate polari $(r, \theta)$, dove $x = r \cos \theta$ e $y = r \sin \theta$.\\
Il determinante della matrice Jacobiana in questo caso è:
\[
J_{\varphi}(r,\theta) = \begin{vmatrix}
\frac{\partial x}{\partial r} & \frac{\partial x}{\partial \theta} \\
\frac{\partial y}{\partial r} & \frac{\partial y}{\partial \theta}
\end{vmatrix} = \begin{vmatrix}
\cos \theta & -r \sin \theta \\
\sin \theta & r \cos \theta
\end{vmatrix} = r (\cos^2 \theta + \sin^2 \theta) = r
\]

Quindi, l'integrale doppio in coordinate polari diventa:
\[
\iint_{\Omega} f(x,y) \, dx \, dy = \iint_{\Omega'} f(r \cos \theta, r \sin \theta) \cdot r \, dr \, d\theta
\]

Ad esempio, calcoliamo l'integrale doppio $\iint_{\Omega} (x^2 + y^2) \, dx \, dy$ dove $\Omega$ è il cerchio di raggio $R$ centrato nell'origine.\\
In coordinate polari, abbiamo $x^2 + y^2 = r^2$ e $\Omega' = \{ (r, \theta) : 0 \leq r \leq R, 0 \leq \theta \leq 2\pi \}$. Quindi:
\[
\iint_{\Omega} (x^2 + y^2) \, dx \, dy = \iint_{\Omega'} r^2 \cdot r \, dr \, d\theta = \int_{0}^{2\pi} \int_{0}^{R} r^3 \, dr \, d\theta
\]

Calcoliamo l'integrale interno:
\[
\int_{0}^{R} r^3 \, dr = \frac{r^4}{4} \bigg|_{0}^{R} = \frac{R^4}{4}
\]

Quindi l'integrale doppio diventa:
\[
\iint_{\Omega} (x^2 + y^2) \, dx \, dy = \int_{0}^{2\pi} \frac{R^4}{4} \, d\theta = \frac{R^4}{4} \cdot 2\pi = \frac{\pi R^4}{2}
\]

\subsection{Coordinate ellittiche}
Consideriamo il passaggio alle coordinate ellittiche $(r, \theta)$, dove $x = a r \cos \theta$ e $y = b r \cos \theta$.\\
Il determinante della matrice Jacobiana in questo caso è:
\[
J_{\varphi}(r,\theta) = \begin{vmatrix}
\frac{\partial x}{\partial r} & \frac{\partial x}{\partial \theta} \\
\frac{\partial y}{\partial r} & \frac{\partial y}{\partial \theta}
\end{vmatrix} = \begin{vmatrix}
a \cos \theta & -a r \sin \theta \\
b \cos \theta & -b r \sin \theta
\end{vmatrix} = ab r (\cos^2 \theta + \sin^2 \theta) = abr
\]

Utilizzando l'identità trigonometrica $\cos^2 \theta + \sin^2 \theta = 1$, possiamo semplificare il determinante:
\[
J_{\varphi}(r,\theta) = abr
\]

Quindi, l'integrale doppio in coordinate ellittiche diventa:
\[
\iint_{\Omega} f(x,y) \, dx \, dy = \iint_{\Omega'} f(a r \cos \theta, b r \cos \theta) \cdot abr \, dr \, d\theta
\]





\chapter{Integrali tripli}

\section{Introduzione}
La costruzione è la stessa degli integrali doppi, ma stavolta siamo in $\R^3$ e non in $\R^2$.\\
Ingredienti: un dominio $\Omega \subseteq \R^3$ e una funzione $f:\Omega \rightarrow \R$ continua e limitata.\\
L'obiettivo è definire l'integrale triplo di $f$ su $\Omega$:
\[
\iiint_{\Omega} f(x,y,z) \, dx \, dy \, dz
\]
e capirne il significato.\\
Per arrivare all'integrale triplo effettuiamo le seguente costruzione in più passaggi:
\begin{enumerate}
  \item Scelgo $\epsilon>0$ e divido $\R^3$ in cubetti di lato $\epsilon$.
  \item Chiamo $C_{\epsilon}$ la famiglia di cubetti interamente contenuti in $\Omega$. (Saranno in numero finito poichè $\Omega$ è limitato).
  \item Campiono $f(x,y,z)$ su ogni cubetto della famiglia, cioè prendo un punto $P_i$ in ogni cubetto $c_i\in C_{\epsilon}$ e calcolo la somma: $S(\epsilon)= \sum_{c_i \in C_{\epsilon}} f(P_i) \cdot \text{volume}(c_i)$.
  \item Si dimostra che esiste finito il limite: $\lim_{\epsilon \to 0} S(\epsilon)$.
\end{enumerate}

\begin{definizione}{Integrale triplo}
  Seguendo quanto detto detto prima, si chiama integrale triplo di $f$ su $\Omega$ il valore:
  \[
  \lim_{\epsilon \to 0} \sum_{c_i \in C_{\epsilon}} f(P_i) \cdot \text{volume}(c_i) = \iiint_{\Omega} f(x,y,z) \, dx \, dy \, dz
  \]
\end{definizione}


\section{Significati degli integrali tripli}
Gli integrali tripli hanno diversi significati e interpretazioni, vediamone alcuni:
\begin{itemize}
  \item Se $f(x,y,z)=1$ allora $\iiint_{\Omega} f(x,y,z) \, dx \, dy \, dz$ è il volume di $\Omega$.
  \item La quantità $\frac{\iiint_{\Omega} f(x,y,z) \, dx \, dy \, dz}{\text{volume}(\Omega)}$ è il valore medio pesato di $f$ su $\Omega$.
  \item Se $f(x,y,z)=x$ allora $\iiint_{\Omega} f(x,y,z) \, dx \, dy \, dz$ è il baricentro di $\Omega$ rispetto all'asse $x$.
  \item Se $f(x,y,z)=y$ allora $\iiint_{\Omega} f(x,y,z) \, dx \, dy \, dz$ è il baricentro di $\Omega$ rispetto all'asse $y$.
  \item Se $f(x,y,z)=z$ allora $\iiint_{\Omega} f(x,y,z) \, dx \, dy \, dz$ è il baricentro di $\Omega$ rispetto all'asse $z$.
  \item Se $f(x,y,z)$ è una densità di massa allora $\iiint_{\Omega} f(x,y,z) \, dx \, dy \, dz$ è la massa di $\Omega$.
  \item Se $f(x,y,z)$ è una densità di massa allora $\iiint_{\Omega} (x^2 + y^2) f(x,y,z) \, dx \, dy \, dz$ è il momento d'inerzia di $\Omega$ rispetto all'asse $z$.
\end{itemize}


\section{Cambiamento di variabili negli integrali tripli}
Vediamo ora come si comportano gli integrali tripli sotto cambiamento di variabili.\\
In 3 variabili consideriamo il seguente cambiamento di coordinate:
\[
  x = x(u,v,w)\\
  y = y(u,v,w)\\
  z = z(u,v,w)
\]
con $(u,v,w) \in \Omega \subseteq \R^3$ e $(x,y,z) \in \Omega' \subseteq \R^3$.\\
L'integrale triplo diventa:
\[
\iiint_{\Omega'} f(x,y,z) \, dx \, dy \, dz = \iiint_{\Omega} f(x(u,v,w),y(u,v,w),z(u,v,w)) \cdot |J(u,v,w)| \, du \, dv \, dw
\]
dove $J(u,v,w)$ è il determinante della matrice jacobiana del cambio di variabili:
\[
J(u,v,w) = \begin{vmatrix}
  \frac{\partial x}{\partial u} & \frac{\partial x}{\partial v} & \frac{\partial x}{\partial w}\\
  \frac{\partial y}{\partial u} & \frac{\partial y}{\partial v} & \frac{\partial y}{\partial w}\\
  \frac{\partial z}{\partial u} & \frac{\partial z}{\partial v} & \frac{\partial z}{\partial w}
\end{vmatrix}
\]

\subsection{Coordinate cilindriche}
Consideriamo il cambiamento di coordinate cilindriche:
\[
\begin{cases}
  x = \rho \cos \theta \\
  y = \rho \sin \theta \\
  z = z
\end{cases}
\]
con $\rho \geq 0$, $0 \leq \theta < 2\pi$ e $-\infty < z < \infty$.\\
Il determinante della matrice jacobiana è:
\[
J(\rho, \theta, z) = \begin{vmatrix}
  \frac{\partial x}{\partial \rho} & \frac{\partial x}{\partial \theta} & \frac{\partial x}{\partial z} \\
  \frac{\partial y}{\partial \rho} & \frac{\partial y}{\partial \theta} & \frac{\partial y}{\partial z} \\
  \frac{\partial z}{\partial \rho} & \frac{\partial z}{\partial \theta} & \frac{\partial z}{\partial z}
\end{vmatrix} = \begin{vmatrix}
  \cos \theta & -\rho \sin \theta & 0 \\
  \sin \theta & \rho \cos \theta & 0 \\
  0 & 0 & 1
\end{vmatrix} = \rho
\]
Quindi l'integrale triplo in coordinate cilindriche diventa:
\[
\iiint_{\Omega'} f(x,y,z) \, dx \, dy \, dz = \iiint_{\Omega} f(\rho \cos \theta, \rho \sin \theta, z) \cdot \rho \, d\rho \, d\theta \, dz
\]

\subsection{Coordinate sferiche}
Consideriamo il cambiamento di coordinate sferiche:
\[
\begin{cases}
  x = \rho \sin \phi \cos \theta \\
  y = \rho \sin \phi \sin \theta \\
  z = \rho \cos \phi
\end{cases}
\]
con $\rho \geq 0$, $0 \leq \theta < 2\pi$ e $0 \leq \phi \leq \pi$.\\
Il determinante della matrice jacobiana è:
\[
J(\rho, \theta, \phi) = \begin{vmatrix}
  \frac{\partial x}{\partial \rho} & \frac{\partial x}{\partial \theta} & \frac{\partial x}{\partial \phi} \\
  \frac{\partial y}{\partial \rho} & \frac{\partial y}{\partial \theta} & \frac{\partial y}{\partial \phi} \\
  \frac{\partial z}{\partial \rho} & \frac{\partial z}{\partial \theta} & \frac{\partial z}{\partial \phi}
\end{vmatrix} = \begin{vmatrix}
  \sin \phi \cos \theta & -\rho \sin \phi \sin \theta & \rho \cos \phi \cos \theta \\
  \sin \phi \sin \theta & \rho \sin \phi \cos \theta & \rho \cos \phi \sin \theta \\
  \cos \phi & 0 & -\rho \sin \phi
\end{vmatrix} = \rho^2 \sin \phi
\]
Quindi l'integrale triplo in coordinate sferiche diventa:
\[
\iiint_{\Omega'} f(x,y,z) \, dx \, dy \, dz = \iiint_{\Omega} f(\rho \sin \phi \cos \theta, \rho \sin \phi \sin \theta, \rho \cos \phi) \cdot \rho^2 \sin \phi \, d\rho \, d\theta \, d\phi
\]

\section{Calcolo degli integrali tripli}
Per calcolare gli integrali tripli possiamo procedere in due modi: per fili e per strati.
Per fili significa pensare $\Omega$ come un insieme di fili paralleli ad un asse.\\
Se esistono due funzioni $g_1(x,y,z)$ e $g_2(x,y,z)$ tali che $\Omega = {(x,y,z)\in \R^3|(x,y)\in D, g_1(x,y,z) \leq z \leq g_2(x,y,z)}$ allora:
\[
\iiint_{\Omega} f(x,y,z) \, dx \, dy \, dz = \iint_{D} \left( \int_{g_1(x,y)}^{g_2(x,y)} f(x,y,z) \, dz \right) \, dx \, dy
\]

Per strati invece significa pensare $\Omega$ come un insieme di strati paralleli ad un piano, cioè di sezioni orizontali.\\
Dato $\Omega$ definiamo le sezioni a quota $z$: $S(z) = \{(x,y)\in \R^2 | (x,y,z)\in \Omega\}$.\\
Possiamo dunque ricavare la seguente formula di integrazione per strati:
\[
\iiint_{\Omega} f(x,y,z) \, dx \, dy \, dz = \int_{-\infty}^{+\infty} \left( \iint_{S(z)} f(x,y,z) \, dx \, dy \right) \, dz
\]
ma di fatto gli estremi di integrazione su $z$ sono quelli che definiscono l'altezza di $\Omega$ e si riducono perciò a quelli per cui $S(z)\neq 0$ (esistono di fatto uno $Z_{\text{min}}$ e uno $Z_{\text{max}}$).

\chapter{Integrali curvilinei}

\section{Introduzione}
In questo capitolo ci occuperemo di integrali curvilinei, ovvero integrali definiti su curve.\\
Nel paragrafo ~\ref{sec:derivata-lungo-una-curva} abbiamo definito le curve e le derivate lungo esse.\\
Vediamo ora un utile teorema che ci servirà per ricavare gli integrali curvilinei.

\begin{teorema}{Derivata della funzione composta}
  SIano $f(x_1, \dots, x_n): \R^n \rightarrow \R^k$ e $g(y_1, \dots, y_m): \R^m \rightarrow \R^n$ funzioni differenziabili. Allora la funzione composta $f \circ g: \R^m \rightarrow \R^k$ è differenziabile e la sua matrice Jacobiana è data da:
  $$J(f \circ g)(y) = Jf(g(y)) \cdot Jg(y)$$
\end{teorema}

Gli integrali curvilinei possono essere di due tipi: di prima e di seconda specie.\\
Gli integrali di prima specie sono integrali di una funzione scalare $f(x_1, \dots, x_n)$ lungo una curva $\gamma: [a, b] \rightarrow \R^n$.\\
Gli integrali di seconda specie sono integrali di un campo vettoriale $\vec F(x_1, \dots, x_n)$ lungo una curva $\gamma: [a, b] \rightarrow \R^n$.\\

\begin{osservazione}{}
  Supporremo sempre che $\gamma$ sia $C^1$ (o $C^1$ a tratti) e che $f$ e $\vec F$ siano continue.
\end{osservazione}

\section{Integrali curvilinei di prima specie}
\begin{definizione}{Integrale curvilineo di prima specie}
  Sia $f: \R^n \rightarrow \R$ una funzione continua e $\gamma: [a, b] \rightarrow \R^n$ una curva $C^1$ (o $C^1$ a tratti).\\
  L'\textbf{integrale curvilineo di prima specie} di $f$ lungo $\gamma$ è definito come:
  $$\int_\gamma f ds = \int_a^b f(\gamma(t)) \cdot ||\gamma'(t)|| dt$$
\end{definizione}

Capiamo meglio il significato. Posso pensare di dividere la curva in piccoli tratti di diametro $< \epsilon$. Campiono $f$ in ogni punto e moltiplico per la lunghezza del tratto. Sommo tutti i tratti e faccio tendere $\epsilon$ a 0. Ovvero:
$$\int_\gamma f ds = \lim_{\epsilon \rightarrow 0} \sum_{i=1}^n f(\gamma(t_i)) \cdot ||\gamma(t_i) - \gamma(t_{i-1})||$$.\\
Qui $\epsilon$ limita il valore di $||\gamma(t_i) - \gamma(t_{i-1})||$.\\

\section{Integrali curvilinei di seconda specie}\label{sec:integrali-curvilinei-di-seconda-specie}
\begin{definizione}{Integrale curvilineo di seconda specie}
  Sia $\vec F: \R^n \rightarrow \R^n$ un campo vettoriale continuo e $\gamma: [a, b] \rightarrow \R^n$ una curva $C^1$ (o $C^1$ a tratti).\\
  L'\textbf{integrale curvilineo di seconda specie} di $\vec F$ lungo $\gamma$ è definito come:
  $$\int_\gamma \vec F \cdot d\vec r = \int_a^b \vec F(\gamma(t)) \cdot \gamma'(t) dt$$
\end{definizione}

Nuovamente, il significato è quello di dividere la curva in piccoli tratti di diametro $< \epsilon$. Campiono $\vec F$ in ogni punto e moltiplico per il vettore tangente al tratto. Sommo tutti i tratti e faccio tendere $\epsilon$ a 0. Ovvero:
$$\int_\gamma \vec F \cdot d\vec r = \lim_{\epsilon \rightarrow 0} \sum_{i=1}^n \vec F(\gamma(t_i)) \cdot (\gamma(t_i) - \gamma(t_{i-1}))$$.\\
Questo integrale esprime il lavoro compiuto dal campo $\vec F$ lungo la curva $\gamma$. Non dipende da come parametrizzo $\gamma$ se non per il verso di percorrenza.

\chapter{Campi conservativi}

\section{Introduzione}\label{sec:introduzione}
Vediamo come prima cosa un importante esempio di integrale curvilineo che ci servirà nella spiegazione dei campi conservativi.\\
Consideriamo il campo (magnetico) $\vec F (x,y) = (\frac{-y}{x^2+y^2}, \frac{x}{x^2+y^2})$ e il cammino $\gamma$ che è la circonferenza di raggio $1$ centrata nell'origine.\\
Come spiegato nel paragrafo ~\ref{sec:integrali-curvilinei-di-seconda-specie}, considerata la parametrizzazione di $\gamma$ data da $\gamma(t) = (\cos t, \sin t)$ con $t \in [0, 2\pi]$, possiamo calcolare l'integrale curvilineo di $\vec F$ lungo $\gamma$:
$$\int_\gamma \vec F \cdot d\vec r = \int_0^{2\pi} \vec F(\gamma(t)) \cdot \gamma'(t) dt =$$\\
$$\int_0^{2\pi} \left( \frac{-\sin t}{\cos^2 t + \sin^2 t}, \frac{\cos t}{\cos^2 t + \sin^2 t} \right) \cdot (-\sin t, \cos t) dt = \int_0^{2\pi} 1 dt =2\pi$$.

\begin{osservazione}{}
  Il valore dell'integrale vale $2\pi$ indipendentemente dal raggio della circonferenza.\\
\end{osservazione}

In qualche caso tuttavia $\int_\gamma \vec F \cdot d\vec r$ è calcolabile in altro modo (senza integrale).\\

\section{Campi conservativi}
Vediamo la seguente proposizione:

\begin{teorema}{}
  Sia $\Omega$ un aperto in $\R^n$ e $\gamma: [a,b] \rightarrow \R^n$ una curva $C^1$ a tratti contenuta in $\Omega$. Sia $\vec F: \Omega \rightarrow \R^n$ un campo vettoriale di classe $C^1$ e sia $f: \Omega \rightarrow \R$ tale che $\vec F = \nabla f$. Allora:
  $$\int_\gamma \vec F \cdot d\vec r = f(\gamma(b)) - f(\gamma(a))$$.
\end{teorema}

Questo motiva la seguente definizione:
\begin{definizione}{Campo conservativo}
  Nelle ipotesi precedenti, un campo vettoriale $\vec F: \Omega \rightarrow \R^n$ è detto \textbf{conservativo} se esiste una funzione $f: \Omega \rightarrow \R$ tale che $\vec F = \nabla f$. \\ In tal caso la funzione $f$ si chiama potenziale di $\vec F$ in $\Omega$.
\end{definizione}

\begin{osservazione}{}
  Dire che un campo è o non è conservativo senza specificare dove è definito non ha significato.\\
\end{osservazione}

Consideriamo ad esempio il campo visto nel paragrafo precedente $\vec F(x,y) = (\frac{-y}{x^2+y^2}, \frac{x}{x^2+y^2})$. Questa volta però consideriamo come $\Omega_1$ il primo quadrante.\\
Vediamo che $\vec F$ è conservativo in $\Omega_1$ in quanto $\vec F = \nabla \arctan \frac{y}{x}$.\\
Tuttavia abbiamo già calcolato che l'integrale di questo campo lungo una circonferenza di raggio generico centrata nell'origine è $2\pi$. Perciò possiamo concludere che il campo non è conservativo in qualunque aperto $\Omega_2 \in \R^2$ che contenga una circonferenza centrata nell'origine.\\

\begin{osservazione}{}
  Se $\vec F$ è conservativo in $\Omega$ e $\gamma$ è una curva chiusa in $\Omega$ allora $\int_\gamma \vec F \cdot d\vec r = 0$.\\
  Se $\gamma$ è chiusa l'integrale si chiama circuitazione di $\vec F$ lungo $\gamma$.
\end{osservazione}

Capiamo allora cosa manca al campo precendente per essere conservativo in $\R^2 \setminus \{0\}$.\\

\begin{definizione}{Connessione per archi}
  Un aperto $\Omega \in \R^n$ si dice connesso per archi se per ogni coppia di punti $P,Q \in \Omega$ esiste una curva $C^1$ a tratti $\gamma: [a,b] \rightarrow \Omega$ tale che $\gamma(a) = P$ e $\gamma(b) = Q$.
\end{definizione}

Enunciamo ora il seguente teorema:
\begin{teorema}{}
  Sia $\Omega \in \R^n$ un aperto connesso per archi e $\vec F: \Omega \rightarrow \R^n$ un campo vettoriale di classe $C^1$. Allora le seguenti affermazioni sono equivalenti:
  \begin{enumerate}
    \item $\vec F$ è conservativo in $\Omega$.
    \item Per ogni curva chiusa $\gamma$ in $\Omega$ si ha $\int_\gamma \vec F \cdot d\vec r = 0$.
    \item Se $\gamma_1$ e $\gamma_2$ sono curve con lo stesso punto iniziale e finale in $\Omega$ allora $\int_{\gamma_1} \vec F \cdot d\vec r = \int_{\gamma_2} \vec F \cdot d\vec r$.
  \end{enumerate}
\end{teorema}

Enunciamo ora condizioni necessarie e sufficienti per la conservatività:


\begin{definizione}{Irrotazionale}
  Un campo vettoriale $\vec F: \Omega \rightarrow \R^n$ si dice \textbf{irrotazionale} se $\nabla \times \vec F = 0$. \\
\end{definizione}
In due variabili è equivalente a:
$$\frac{\partial F_2}{\partial x} - \frac{\partial F_1}{\partial y}=0$$.
In tre variabili questo significa:
$$\nabla \times \vec F = \left( \frac{\partial F_3}{\partial y} - \frac{\partial F_2}{\partial z}, \frac{\partial F_1}{\partial z} - \frac{\partial F_3}{\partial x}, \frac{\partial F_2}{\partial x} - \frac{\partial F_1}{\partial y} \right) = (0,0,0)$$.

\begin{teorema}{Condizione necessaria per la conservatività}
  Sia $\Omega \in \R^n$ un aperto connesso per archi e $\vec F: \Omega \rightarrow \R^n$ un campo vettoriale di classe $C^1$. Se $\vec F$ è conservativo in $\Omega$ allora $\frac{\partial F_i}{\partial x_j} = \frac{\partial F_j}{\partial x_i}$ per ogni $i,j = 1, \ldots, n$. \\
  Ciò significa che la matrice jacobiana di $\vec F$ è simmetrica e che il campo $\vec F$ è irrotazionale.
\end{teorema}

La seguente definizione sarà utile per trovare una condizione sufficiente di conservatività:

\begin{definizione}{Semplicemente connesso}
  Un aperto $\Omega \in \R^n$ si dice semplicemente connesso se per ogni curva chiusa $\gamma$ in $\Omega$ si ha che $\gamma$ può essere deformata con continuità fino a stringerla ottenendo un singolo punto senza mai uscire da $\Omega$.
\end{definizione}

Nel piano la condizione è equivalente a dire che $\Omega$ non ha buchi.\\
In $\R^3$ la condizione è più complessa da verificare.\\

\begin{teorema}{Condizione sufficiente per la conservatività}
  Sia $\Omega \in \R^n$ un aperto semplicemente connesso e $\vec F: \Omega \rightarrow \R^n$ un campo vettoriale di classe $C^1$. Se $\vec F$ è irrotazionale in $\Omega$ allora $\vec F$ è conservativo in $\Omega$.
\end{teorema}

\section{Campi centrali}
Vediamo un importante tipo di campi:

\begin{definizione}{Campo centrale}
  Un campo vettoriale $\vec F: \R^n \setminus \{0\} \rightarrow \R^n$ si dice \textbf{centrale} se esiste una funzione $f: \R^n \setminus \{0\} \rightarrow \R$ tale che $\vec F(\vec{x}) = f(\|\vec{x}\|) \frac{\vec{x}}{\|\vec{x}\|}$, dove $\|\vec{x}\|$ è la norma euclidea di $\vec{x}$.
\end{definizione}


Esiste allora un teorema che ci permette di risolvere molte patologie:
\begin{teorema}{Conservatività dei campi centrali}
  Un campo centrale $\vec F: \R^n \setminus \{0\} \rightarrow \R^n$ è conservativo.
\end{teorema}

\begin{osservazione}{}
  Da quanto detto riguardo alla condizione necessaria di conservatività segue che campo centrale $\vec F: \R^n \setminus \{0\} \rightarrow \R^n$ è irrotazionale.
\end{osservazione}

\section{Potenziale di un campo}

Dato un aperto $\Omega \inc \R^n$ e un campo vettoriale $\vec F: \Omega \rightarrow \R^n$ conservativo (oppure almeno irrotazionale) in $\Omega$, vediamo come si può ricostruire (quando esiste) un potenziale $f$ in $\Omega$.

\begin{definizione}{Potenziale}
  Nelle condizioni precedenti una funzione $f: \Omega \rightarrow \R$ tale che $\vec F = \nabla f$ si chiama \textbf{potenziale} di $\vec F$ in $\Omega$.
\end{definizione}

In due variabili $\Omega \in \R^2$ abbiamo che $\vec F = (F_1, F_2)$ e vogliamo\\
$$F_1(x,y) = \frac{\partial f(x,y)}{\partial{x}} $$
$$F_2(x,y) = \frac{\partial f(x,y)}{\partial{y}} $$.\\
Lo schema è il seguente:
\begin{enumerate}
\item Calcoliamo $f(x,y) = \int F_1(x,y)dx + g(y)$.
\item Calcoliamo $\frac{\partial f(x,y)}{\partial{y}}$.
\item Imponiamo $F_2(x,y)=\frac{\partial f(x,y)}{\partial{y}}$ e troviamo $g(y)$.
\end{enumerate}

In 3 variabili il processo è analogo.\\
\subsection{Esempio di calcolo del potenziale}

Consideriamo il campo vettoriale $\vec F(x,y) = (2xy, x^2 + 1)$. Vogliamo trovare un potenziale $f(x,y)$ tale che $\vec F = \nabla f$.

\begin{enumerate}
\item Calcoliamo $f(x,y)$ integrando $F_1$ rispetto a $x$:
$$ f(x,y) = \int 2xy \, dx = x^2 y + g(y) $$
dove $g(y)$ è una funzione da determinare.

\item Calcoliamo la derivata parziale di $f(x,y)$ rispetto a $y$:
$$ \frac{\partial f(x,y)}{\partial y} = x^2 + g'(y) $$

\item Imponiamo che questa derivata sia uguale a $F_2(x,y)$:
$$ x^2 + g'(y) = x^2 + 1 $$
da cui otteniamo:
$$ g'(y) = 1 $$

Integrando rispetto a $y$, troviamo:
$$ g(y) = y + C $$

Quindi il potenziale è:
$$ f(x,y) = x^2 y + y + C $$
\end{enumerate}

Abbiamo quindi trovato che il potenziale del campo $\vec F(x,y) = (2xy, x^2 + 1)$ è $f(x,y) = x^2 y + y + C$.\\





\chapter{Green}
Questo capitolo sarà interamente dedicato al teorema di Green nel piano. Esso è un caso particolare del teorema del rotore (o teorema di Stokes) che verrà affrontato prossimamente.\\

Il teorema di Green consente di caclolare la circuitazione di un campo riducendola a un integrale doppio.

\section{Teorema di Green}
\begin{teorema}{Teorema di Green}
Sia $\Omega \inc \R^2$ aperto limitato tale che la sua frontiera è l'unione di $N$ curve chiuse disgiunte $\gamma_1, \dots, \gamma_N$  e sia $\vec F:\Omega\rightarrow \R^2$ di classe $C^1$, allora:
$$ \int_\Omega \left(\frac{\partial F_2}{\partial x}-\frac{\partial F_1}{\partial y}\right) dxdy = \sum_{i=1}^{N} \int_{\gamma_i} \vec{F} \cdot d\vec{r} $$,\\dove ogni $\gamma_i$ è orientata in modo da avere $\Omega$ alla sua sinistra.
\end{teorema}

Il teorema si può applicare in vari modi ad esempio:
\begin{itemize}
\item Calcolare la circuitazione di un campo lungo una curva chiusa.
\item Confrontare tra loro due circuitazioni (specialmente quando $\vec F$ è irrotazionale ma non conservativo).
\end{itemize}

Inoltre grazie al teorema si ricava un utile fatto generale:
\begin{corollario}{}
  Se ho due curve chiuse $\gamma_1$ e $\gamma_2$ tali che $\gamma_1$ è interna a $\gamma_2$ allora: $$\int_{\gamma_1} \vec F \cdot d\vec r = \int_{\gamma_2} \vec F \cdot d\vec r$$\\
  per ogni campo $\vec F$ irrotazionale (nella sezione di piano che contiene le curve).
\end{corollario}

Applichiamo ora il teorema al calcolo dell'area racchiusa tra curve chiuse.\\
Se scelgo un campo $\vec F$ tale che $\frac{\partial F_2}{\partial x}-\frac{\partial F_1}{\partial y}=1$ (ad esempio $\vec F = (0,x)$)allora ottengo l'area di $\Omega$ come circuitazione di $\vec F$ lungo il bordo di $\Omega$.\\
Riassumendo:
\begin{teorema}{}
  Se $\gamma(t)= (x(t), y(t)), t \in [a,b]$ è una curva chiusa semplice e regolare, allora l'area racchiusa da $\gamma$ è data da:
  $$\text{Area}(\Omega) = \int_{a}^{b} x(t)\cdot y'(t) dt$$.\\
  Sono possibili anche altre scelte per $\vec F$ che generano formule analogo per il calcolo dell'area di $\Omega$.
\end{teorema}

\subsection{Esempio: area del cardiode}

Consideriamo il campo vettoriale $\vec{F} = (0, x)$. Utilizziamo il teorema di Green per calcolare l'area racchiusa dal cardioide. Secondo il teorema di Green, abbiamo:

$$ \text{Area}(\Omega) = \oint_{\gamma} \vec{F} \cdot d\vec{r} $$

Dove $\gamma$ è la curva chiusa che descrive il bordo del cardioide. In coordinate polari, il cardioide è dato da:

$$ r = 1 + \cos\theta $$

Le coordinate cartesiane sono:

$$ x = r \cos\theta = (1 + \cos\theta) \cos\theta $$
$$ y = r \sin\theta = (1 + \cos\theta) \sin\theta $$

Il differenziale di posizione è:

$$ d\vec{r} = \left( \frac{dx}{d\theta}, \frac{dy}{d\theta} \right) d\theta $$

Calcoliamo le derivate:

$$ \frac{dx}{d\theta} = \frac{d}{d\theta} \left( (1 + \cos\theta) \cos\theta \right) = -\cos\theta \sin\theta + \cos^2\theta - \sin^2\theta $$
$$ \frac{dy}{d\theta} = \frac{d}{d\theta} \left( (1 + \cos\theta) \sin\theta \right) = \cos\theta \sin\theta + \sin^2\theta + \cos\theta \cos\theta $$

Quindi:

$$ d\vec{r} = \left( -\cos\theta \sin\theta + \cos^2\theta - \sin^2\theta, \cos\theta \sin\theta + \sin^2\theta + \cos^2\theta \right) d\theta $$

Il campo $\vec{F}$ in coordinate polari è:

$$ \vec{F} = (0, x) = (0, (1 + \cos\theta) \cos\theta) $$

Il prodotto scalare $\vec{F} \cdot d\vec{r}$ è:

$$ \vec{F} \cdot d\vec{r} = 0 \cdot \left( -\cos\theta \sin\theta + \cos^2\theta - \sin^2\theta \right) + (1 + \cos\theta) \cos\theta \cdot \left( \cos\theta \sin\theta + \sin^2\theta + \cos^2\theta \right) $$

Semplificando:

$$ \vec{F} \cdot d\vec{r} = (1 + \cos\theta) \cos\theta \left( \cos\theta \sin\theta + \sin^2\theta + \cos^2\theta \right) $$

Integrando lungo $\theta$ da $0$ a $2\pi$:

$$ \text{Area}(\Omega) = \int_{0}^{2\pi} (1 + \cos\theta) \cos\theta \left( \cos\theta \sin\theta + \sin^2\theta + \cos^2\theta \right) d\theta $$

Semplificando ulteriormente e risolvendo l'integrale, otteniamo:

$$ \text{Area}(\Omega) = \frac{3\pi}{2} $$

Pertanto, l'area racchiusa dal cardioide è:

$$ \text{Area} = \frac{3\pi}{2} $$


\chapter{Integrali curvilinei e di superficie}
	\section{Definizione delle curve}
	\deff
	Una funzione $\gamma : [ a, b ] \subseteq \R \Rightarrow \R^m$ si dice \textbf{curva} in $\R^m$ se è continua.\\
	\\
	L'insieme $\gamma ([a,b])= \{\gamma(t)\in \R^m : t \in [a,b] \}$ si dice sostegno di $\gamma$, cioè il sostegno è l'immagine di $\gamma$
	\deff
	$\gamma:[a,b]\Rightarrow \R^m$, curva, si dice \textbf{semplice} se vale l'implicazione:
	$$t \in [a, b],\ s \in (a, b)\ \text{e} \ s\ne t \Rightarrow \gamma(t)\ne \gamma(s)$$
	$\gamma:[a,b]\Rightarrow \R^m$, curva, si dice \textbf{chiusa} se vale l'implicazione:
	$$\gamma(a)=\gamma(b)$$
	Se $\gamma:[a,b]\Rightarrow \R^m$ è una curva chiusa e semplice si dice \textbf{curva di Jordan}
	\deff
	Sia $\gamma : [a, b] \Rightarrow \R^m$ curva con $\gamma(t) = (\gamma_1(t),\dots, \gamma_m(t))$, $\gamma$ si dice \textbf{regolare} se $\gamma \in C^1([a, b])$ e se $\gamma'(t)\ne \vec{0}, \ \forall t \in [a, b]$.\\
	\\
	$\gamma'(t)$ si dice \textbf{vettore tangente} a $\gamma$ in $\gamma(t)$, oppure vettore derivata o vettore velocità.\\
	\\
	Se $\gamma'(t)\ne \vec 0 \  \forall t$ significa che $\gamma(t)$ non si fermerà mai.
	\deff
	$\gamma:[a,b]\rightarrow \R^m$, curva, si dice \textbf{regolare a tratti} se $\exists a=t_0<t_1<\dots<t_{n-1}<t_{n}=b$ tale che $\gamma \in C^1([t_{k-1}, t_k]) \ \forall k \in {1, \dots, n}$ e $\gamma'(t)\ne \vec 0 \ \forall t \ne t_k$
	\section{Integrali curvilinei di prima specie (integrali di funzione su curve)}
	\deff
	Sia $\gamma:[a, b] \rightarrow \R^m$ curva regolare a tratti, sia $f:D\subseteq \R^m \rightarrow \R$ tale che $\gamma([a,b])\subseteq D$, si pone:
	$$\int_\gamma f ds = \int_\gamma \gamma f = \int_a^b f(\gamma(t))|\gamma'(t)|dt$$
	se l'integrale esiste.\\
	\\
	Se $\gamma$ è di Jordan si scrive $\oint_\gamma f ds = \int_\gamma fds$.\\
	\\
	Si pone $L(\gamma)=\int_\gamma ds=\int_{a}^{b}|\gamma'(t)|dt$ lunghezza di $\gamma$.
	\prop
	Sia $\gamma:[a, b] \rightarrow \R^m$ curva regolare a tratti e sia $\phi : [c, d] \rightarrow [a, b] \ C^1$ strettamente monotone e suriettiva e sia $\tilde{\gamma} : [c, d]\rightarrow \R^m$ definita come $\tilde{\gamma}=\gamma ( \phi (r) )$, ovvero  $\phi$ è un riscalamento del parametro t e $\tilde{\gamma}$ ha la stessa traiettoria di $\gamma$ ma la percorre in modo diverso, allora:
	$$\int_{\tilde{\gamma}} fds=\int_\gamma fds$$
	Quindi gli integrali di prima specie non dipendo né dalla parametrizzazione $\gamma$, né dal verso di percorrenza, ma dipendono solo dal sostegno $\Gamma=\gamma([a,b])$
	\deff
	Sia $\Gamma \subseteq \R^m$ che si sostegno si una curva $\gammadef$ regolare a tratti semplice, allora si pone:
	$$\int_\Gamma fds= \int_\gamma fds$$
	Allo stesso modo $L(\Gamma)=L(\gamma)$
	\deff
	Sia $\gamma :  [a,b]\rightarrow \R ^3$ curva regolare a tratti semplice, alla quale è associata una \textbf{funzione di densità} (di massa) $\mu:  \Gamma =\gamma([a,b])\subseteq \R^3 \rightarrow \R$, si dice \textbf{massa} di $\gamma$ (o di $\Gamma$) il numero:
	$$M(\Gamma)=m(\Gamma)=\int_\Gamma \mu \ d\sigma$$
	Il \textbf{baricentro} di $\Gamma$ è il punto $(x_G, y_G, z_G)$ tale che:
	$$x_G=\frac{1}{M(\Gamma)}\int_\Gamma x \mu (x,y,z) d \sigma$$
	$$y_G=\frac{1}{M(\Gamma)}\int_\Gamma y \mu (x,y,z) d \sigma$$
	$$z_G=\frac{1}{M(\Gamma)}\int_\Gamma z \mu (x,y,z) d \sigma$$
	\section{Integrali curvilinei di seconda specie (lavori)}
	Sia $\gammadef$ curva regolare a tratti e  sia $F:D\inc \R^m \rightarrow \R^m$ con $\gamma([a,b])\inc D$, si dice \textbf{lavoro di} $\vec F$ \textbf{lungo} $\gamma$ (o integrale di seconda specie) il numero:
	$$\int_\gamma F \cdot dl = \int_a^bF(\gamma(t))\cdot \gamma'(t) dt$$
	Se $\gamma$ è di Jordan si scrive $\oint_\gamma F \cdot dl = \int_\gamma F \cdot dl$ detto \textbf{circuitazione di $F$ lungo $\gamma$}.\\
	\\
	Sia $\vec r(t)= \frac{\gamma'(t)}{|\gamma'(t)|}\ \forall t \in [a,b]$ detto \textbf{versore tangente a} $\gamma$, si ha:
	$$\int_\gamma F \cdot dl = \int_\gamma (F\cdot \vec r) d\sigma$$
	\prop
	Sia $\gamma:[a, b] \rightarrow \R^m$ curva regolare a tratti e sia $\phi : [c, d] \rightarrow [a, b] \ C^1$ strettamente crescente e suriettiva e sia $\tilde{\gamma} : [c, d]\rightarrow \R^m$ definita come $\tilde{\gamma}=\gamma ( \phi (r) )$, ovvero $\tilde{\gamma}$ percorre il sostegno di $\gamma$ lo stesso numero di volte nello stesso verso, allora:
	$$\int_{\tilde{\gamma}} F \cdot dl = \int_\gamma F \cdot dl$$
	Nel caso in cui $\hat \gamma$, definita come $\tilde{\gamma}$, sia strettamente decrescente, e quindi percorra il sostegno di $\gamma$ lo stesso numero di volte ma in verso opposto, allora:
	$$\int_{\hat{\gamma}} F \cdot dl = - \int_\gamma F \cdot dl$$
	\section{Integrali di superficie}
	\deffname{Superfici parametrizzate}
	Una funzione $\sigma:\bar{A} \inc \R^2 \rightarrow \R^3$ con $A \inc \R^2$ aperto e con $\sigma(u,v)=(\sigma_1(u,v),\sigma_2(u,v),\sigma_3(u,v))$ si dice \textbf{superficie parametrizzata} se $\sigma \in C^1$, $\sigma$ è iniettiva in $A$ e la matrice Jacobiana ha rango 2, allora:
	$$\Sigma=\sigma(\bar{A})$$
	ed è detto \textbf{sostegno} di $\sigma$.
	\deff
	Le derivate parziali di $\sigma$ in $u$ e $v$ generano in piano in $\R^3$ ,detto \textbf{piano tangente} a $\Sigma$ in $\sigma(u_0,v0)$, che ha equazione:
	$$\Pi(u,v)=\sigma(u_0,v_0)+\frac{\partial \sigma}{\partial u}(u_0,v_0)(u-u_0)+\frac{\partial \sigma}{\partial v}(u_0,v_0)(v-v_0)$$
	Definiamo il \textbf{vettore normale} a $\Sigma$ in $\sigma(u_0,v0)$:
	$$\vec N(u_0, v_0)=\frac{\partial \sigma}{\partial u} \times \frac{\partial \sigma}{\partial v}$$
	da cui possiamo ricavare il versore $\vec n (u_0,v_0) = \frac{\vec N (u_0,v_0)}{|\vec N (u_0,v_0)|}$.
	\deffname{Superfici cartesiane}
	Sia $g:\bar A \inc \R^2 \rightarrow \R$, con $A$ aperto e $g \in C^1(A)$, data una superficie $\sigma(u,v)=(u,v,g(u,v))$ si ha che $\Sigma = \sigma (\bar A)$ è il grafico di $g$. Sappiamo quindi che:
	$$\Sigma  = \sigma(\bar A)=Gr(g)=\{(x,y,z)\in \R^3:(x,y)\in \bar A , z = g(x,y)\}$$
	$$\vec N(x, y)=\frac{\partial \sigma}{\partial u} \times \frac{\partial \sigma}{\partial v}=\left(-\frac{\partial g}{\partial x}, -\frac{\partial g}{\partial y}, 1 \right)$$
	$$|\vec N (x,y)|=\sqrt{1+|\nabla g|^2}$$
	\deffname{Integrali di superficie di prima specie}
	Sia $\sigma:\bar{A} \inc \R^2 \rightarrow \R^3$ superficie con $A$ misurabile, poniamo:
	$$\int_\sigma f d\sigma = \int_\sigma f(x,y,z)d\sigma =\int_{\bar A} f(x,y,g(x,y))|\vec N (x,y)|dxdy$$
	\prop
	L'integrale è indipendente dalla parametrizzazione, infatti se due superfici sono diverse ma hanno stesso sostegno l'integrale non cambia.
	\section{Flusso di un campo vettoriale}
	\deffname{Superfici orientabili}
	Una superficie $\sigma:\bar{A} \inc \R^2 \rightarrow \R^3$ si dice \textbf{orientabile} se la funzione $\Sigma \rightarrow \R^3$ è continua.
	\oss
	Se due superfici orientabili con lo stesso sostegno allora i versori normali possono essere solo uguali od opposti tra loro.
	\prop
	Se $\Sigma = \sigma (\bar A)$ è il sostegno di $\sigma:\bar{A} \inc \R^2 \rightarrow \R^3$ e se anche $\Sigma=\partial \Omega$ con $\Omega$ aperto, connesso e limitato, allora $\Sigma$ è orientabile. \\
	\\
	In particola re $\vec n$ punta verso l'interno di $\Omega$ allora viene detto \textbf{entrante}, altrimenti, se punta verso l'esterno, viene detto \textbf{uscente}. Il verso positivo del vettore è quello uscente dalla superficie.
	\deffname{Flusso di un campo vettoriale}
	Sia $\sigma:\bar{A} \inc \R^2 \rightarrow \R^3$ superficie orientabile e sia $F:D\inc \R^3 \rightarrow \R^3$ campo vettoriale continuo con $\Sigma = \sigma (\bar A)\inc D$, allora il \textbf{flusso} di $\vec F$ attraverso $\sigma$ nella direzione $\vec n$ è:
	$$\int_\sigma (\vec F \cdot \vec n) d\sigma = \int_{\bar A}F(\sigma(u,v))\cdot \frac{\vec N(u,v)}{|\vec N(u,v)|}|\vec N(u,v)|dudv=\int_D F(x,y,g(x,y))\cdot \vec N(x,y)dxdy$$
	\deff
	Sia $\Sigma$ una superficie orientabile, se $\Sigma=\partial \Omega$ con $\Omega$ aperto, connesso, limitato e misurabile allora:
	$$\int_{\partial \Omega} \vec F \cdot \\vec n$$
	per convenzione denota il \textbf{flusso uscente} da $\Omega$ , cioè il flusso di $F$ lungo $\vec n$ uscente.
	\thhname{Th. della divergenza di Gauss}
	Sia $\Omega \inc \R^3$ connesso, limitato e misurabile, tale che sia $\Sigma=\partial \Omega$ una superficie orientata con $\vec n$ uscente, sia $\vec F:D \inc \R^3 \rightarrow \R^3$, allora:
	$$\int_{\partial \Omega}(\vec F \cdot \vec n)d\sigma=\int_\Omega div \vec F dxdydz$$
	\deffname{Aperto con bordo}
	Sia $D \inc \R^2$ aperto, connesso e limitato. $D$ si dice \textbf{aperto con bordo} se $\partial D$ è l'unione di un numero finito di sostegni di curve di Jordan regolari a tratti a due a due disgiunti. Su $\partial D$ si definisce come orientazione positiva quella per cui percorrendo $\partial D$ vedo $D$ a sinistra.
	\thhname{Formula di Green nel piano}
	Sia $\vec F:E\ inc \R^2 \rightarrow \R^2$ di classe $C^1$ con $E$ aperto, sia $D\inc \R^2$ aperto con bordo con $\bar D \inc E$, allora:
	$$\int_{\partial D} \vec F \cdot dl = \int_D \left(\frac{\partial F_2}{\partial x}-\frac{\partial F_1}{\partial y}\right)dxdy$$
	\thhname{Th. del rotore di Stokes}
	Sia $D \inc \R^2$ aperto con bordo e sia $\sigma : \bar D \inc \R^2 \rightarrow \R^3$ superficie orientabile iniettiva si $\bar D$, chiamiamo $\partial \sigma = \sigma (\partial D)$, l'immagine di $\partial D$ tramite $\sigma$, \textbf{frontiera della superficie} $\sigma$. Orientiamo $\partial \sigma$ con l'orientazione indotta dall'orientazione positiva di $\partial D$, ovvero quando il versore normale $\vec n$ percorre $\partial \sigma$ vede $\sigma$ a sinistra. Allora:
	$$\int_{\partial\sigma} \vec F \cdot dl = \int_\sigma \left(rot \vec F \cdot \vec n\right)d\sigma$$
	ovvero il lavoro di $F$ lungo $\partial \sigma$ è uguale al flusso del rotore di $\vec F$ attraverso $\sigma$.
	\section{Campi conservativi}
	\deffname{Campi conservativi}
	Un campo $\vec F: \Omega \inc \R^n \rightarrow \R^n$ continuo con $\Omega$ aperto si dice \textbf{conservativo} se $\exists \ \Phi : \Omega \inc \R^n \rightarrow \R$ tale che $\nabla \Phi = \vec F$, in tal caso $\Phi$ si dice \textbf{potenziale} di $\vec F$, oppure primitiva di $\vec F$.
	\prop
	Sia $\gamma:[a,b]\rightarrow \Omega \inc \R^n$ curva regolare a tratti, se $F$ è \textbf{conservativo} si ha che il lavoro lungo $\gamma$ è:
	$$\int_\gamma \vec F \cdot dl = \int_a^b \vec F(\gamma (t))\cdot \gamma'(t)dt=\int_{a}^{b} \nabla \Phi (\gamma(t))\cdot \gamma'(t)dt=\Phi(\gamma(b))-\Phi(\gamma(a))$$
	ovvero se $F$ è conservativo ($F\nabla \Phi$) il lavoro lungo una curva $\gamma$ dipende sola dal punto iniziale e da quello finale.
	\thh
	Sia $\vec F: \Omega \inc \R^n \rightarrow \R^n$ conservativo ($F\nabla \Phi$), allora per ogni curva $\gamma:[a,b]\rightarrow \Omega \inc \R^3$ si ha:
	$$\int_\gamma \vec F \cdot dl =\Phi(\gamma(b))-\Phi(\gamma(a))$$
	Se $\gamma_1$ e $\gamma_2$ sono due curve con stesso punto iniziale e finale:
	$$\int_{\gamma_2} \vec F \cdot dl = \int_{\gamma_1} \vec F \cdot dl$$
	Se $gamma$ è chiusa allora:
	$$\int_\gamma \vec F \cdot dl =0$$
	\thh
	Sia $\vec F: \Omega \inc \R^n \rightarrow \R^n$ continuo con $\Omega$ connesso, allora le tre affermazioni sono equivalenti:
	$$\begin{Bmatrix}
		F \text{ è conservativo}\\
		  \Updownarrow\\
		 \int_{\gamma_2} \vec F \cdot dl = \int_{\gamma_1} \vec F \cdot dl\\
		 (\text{con } \gamma_1,\gamma_2 \text{ curve con stesso inizio e fine})\\
		 \Updownarrow\\
		 \int_\gamma \vec F \cdot dl =0\\
		 (\text{con } \gamma \text{ chiusa})
	\end{Bmatrix}$$
	\thh
	Sia $F: \Omega \subseteq \mathbb{R}^n \rightarrow \mathbb{R}^n$ di classe $C^1$ ($\Sigma$ aperto), se $F$ è conservativo allora:
	$$\frac{\partial F_i}{\partial x_j} = \frac{\partial F_j}{\partial x_i} \qquad \forall i, j \in \{ 1,\dots, n\} \text{ in } \Omega$$
	\deff
	Se $rotF=0$, $F$ si dice \textbf{irrotazionale}.
	\prop
	Se $F$ è conservativo e di classe $C^1$ $\Rightarrow$ $F$ è irrotazionale.\\
	\\
	In generale la $\Rightarrow$ non si può invertire.
	\deff
	Un aperto connesso $\Omega$ si dice \textbf{semplicemente connesso} se ogni curva $\gamma$ di Jordan può essere deformata con continuità fino a contrarsi ad un punto, rimanendo sempre dentro $\Omega$.\\
	\\
	In $\mathbb{R}^2$ un insieme semplicemente connesso è un insieme "privo di buchi". Sono esempi di insiemi semplicemente connessi: tutti gli aperti convessi, tutti gli aperti limitati con frontiera costituita da un'unica curva e il piano privato di una semiretta. Sono esempi di insiemi non semplicemente connessi: tutti gli aperti privati di un punto, le corone circolari e il piano privato di una retta (perché non è connesso).\\
	\\
	In $\mathbb{R}^3$ sono esempi di insiemi semplicemente connessi: tutti gli aperti convessi, lo spazio privato di un punto e le corone sferiche. Sono esempi di insiemi non semplicemente connessi: lo spazio privato di una retta o di un piano.
	\thh
	Sia $\Omega \subseteq \mathbb{R}^3$ un aperto semplicemente connesso, e sia $F$ un campo
	vettoriale di classe $C^1 $, se $rotF=0$ (irrotazionale) allora $F$ è conservativo.

\chapter{Gauss-Green}
\section{Teorema di Gauss-Green}
Questo breve capitolo è interamente dedicato al teorema di Gauss-Green anche noto come teorema della divergenza.\\
Diamo intanto la definizione di divergenza di un campo vettoriale.
\begin{definizione}{Divergenza di un campo vettoriale}
  Sia $\vec{F} \in C^1(\Omega, \R^3)$ un campo vettoriale definito su un insieme aperto $\Omega \inc \R^3$. La \textbf{divergenza} di $\vec{F}$ è la funzione $\nabla \cdot \vec{F} : \Omega \to \R$ definita come:
  $$
    \nabla \cdot \vec{F} = \frac{\partial F_1}{\partial x} + \frac{\partial F_2}{\partial y} + \frac{\partial F_3}{\partial z}
  $$
\end{definizione}
Ora possiamo enunciare il teorema di Gauss-Green.
\begin{teorema}{Teorema di Gauss-Green}
 Sia $\Omega \inc \R^3$ un solido (aperto e limitato) la cui frontiera $\partial \Omega$ è una superficie regolare orientata con la scelta della normale uscente. Sia $\vec{F} \in C^1(\Omega, \R^3)$ un campo vettoriale. Allora vale la seguente formula:
  $$
    \int_{\Omega} \nabla \cdot \vec{F} \, dV = \int_{\partial \Omega} \vec{F} \cdot \vec{n} \, dS$$
  dove $\nabla \cdot \vec{F}$ è la divergenza di $\vec{F}$ e $\vec{n}$ è il vettore normale alla superficie $\partial \Omega$.
\end{teorema}

Vediamone alcune applicazioni.
\begin{itemize}
\item Se ad esempio $\vec{F} = (0, 0, z)$, allora $\nabla \cdot \vec{F} = 1$ e il teorema di Gauss-Green diventa:
  $$
    \int_{\Omega} 1 \, dV = \int_{\partial \Omega} z \, dS
  $$
  cioè l'integrale del volume di $\Omega$ è uguale all'integrale della funzione $z$ sulla superficie $\partial \Omega$.
\item Possiamo usarlo per calcolare il volume di una sfera considerando un campo vettoriale $F(x,y,z)=(x,y,z)$: la divergenza di $F$ è $3$ e quindi:
  $$
    \int_{\Omega} 3 \, dV = 3 \cdot \text{Volume}(\Omega) = \int_{\partial \Omega} \vec{F} \cdot \vec{n} \, dS = \int_{\partial \Omega} (x,y,z) \cdot \vec{n} \, dS
  $$
  ma $(x,y,z) \cdot \vec{n} = r$ e quindi:
  $$
    3 \cdot \text{Volume}(\Omega) = \int_{\partial \Omega} r \, dS
  $$
  e quindi:
  $$
    \text{Volume}(\Omega) = \frac{1}{3} \int_{\partial \Omega} r \, dS
  $$
  Per una sfera di raggio $R$, la superficie $\partial \Omega$ è una sfera di raggio $R$ e quindi:
  $$
    \text{Volume}(\Omega) = \frac{1}{3} \int_{\partial \Omega} r \, dS = \frac{1}{3} \int_{\partial \Omega} R \, dS = \frac{R}{3} \int_{\partial \Omega} dS
  $$
  L'integrale $\int_{\partial \Omega} dS$ è semplicemente l'area della superficie della sfera, che è $4 \pi R^2$. Quindi:
  $$
    \text{Volume}(\Omega) = \frac{R}{3} \cdot 4 \pi R^2 = \frac{4 \pi R^3}{3}
  $$
  che è la ben nota formula per il volume di una sfera di raggio $R$.
\end{itemize}

In generale nei casi in cui la divergenza del campo ha un'espressione semplice, il teorema di Gauss-Green può essere molto utile per calcolare integrali di volume.



\chapter{Stokes}
\section{Teorema di Stokes}
In quest'altro breve capitolo vedremo il teorema di Stokes, che è un'estensione del teorema di Green a campi vettoriali in tre dimensioni.

Il teorema di Stokes permette di legare tra loro il flusso del rotore di un campo vettoriale attraverso una superficie e la circuitazione del campo stesso lungo il bordo della superficie.

Vediamo innanzitutto una definizione preliminare:\\


\begin{definizione}{Compatibilità delle orientazioni}
  Dati una superficie orientata $ S $ e una curva $\gamma$ orientata lungo il suo bordo (dove in questo caso non va inteso come frontiera della superficie in $\R^3$ ma più come "orlo" della stessa) , diciamo che le due orientazioni sono \textbf{compatibili} se percorrendo $\gamma$ dal lato di $N$, $S$ risulta essere a sinistra.
\end{definizione}

Ora possiamo enunciare il teorema di Stokes:
\begin{teorema}{Teorema di Stokes}
    Sia $ S $ una superficie orientata con una curva $\gamma$ come bordo orientata compatibilmente. Sia $ \vec{F} $ un campo vettoriale di classe $ C^1 $ in un intorno aperto di $ S $. Allora vale la seguente formula:
    \[
        \int_S \nabla \times \vec{F} \cdot \hat{N} \, dS = \int_{\gamma} \vec{F} \cdot d\vec{r}
    \]
\end{teorema}

Un'importante applicazione del teorema di Stokes è la seguente:\\
Se $S_1$ e $S_2$ sono due superfici orientate con bordo $\partial S_1=\partial S_2 =\gamma$ orientati compatibilmente, allora:
\[
    \int_{\gamma} \vec{F} \cdot d\vec{r}=
\int_{S_1} \nabla \times \vec{F} \cdot \hat{N} \, dS + \int_{S_2} \nabla \times \vec{F} \cdot \hat{N} \, dS
\]

\subsection{Esempio di applicazione del teorema di Stokes}

Consideriamo il campo vettoriale $\vec{F} = (y, -x, z)$ e la superficie $S$ definita dal paraboloide $z = 1 - x^2 - y^2$ con $z \geq 0$. Calcoliamo il flusso del rotore di $\vec{F}$ attraverso $S$ e la circuitazione di $\vec{F}$ lungo il bordo di $S$.

Innanzitutto, calcoliamo il rotore di $\vec{F}$:
\[
\nabla \times \vec{F} = \left( \frac{\partial z}{\partial y} - \frac{\partial (-x)}{\partial z}, \frac{\partial y}{\partial z} - \frac{\partial z}{\partial x}, \frac{\partial (-x)}{\partial x} - \frac{\partial y}{\partial y} \right) = (0, 0, -2)
\]



Sfruttando il fatto che $z$ è grafico e ricordando quanto visto in \ref{sec:flusso-superficie}, il flusso del rotore di $\vec{F}$ attraverso $S$ è quindi:
\[
\int_S (0, 0, -2) \cdot (-2x, -2y, -1) dx dy
\]

Passiamo alle coordinate polari, dove $x = r \cos \theta$ e $y = r \sin \theta$. La superficie $S$ è definita da $z = 1 - r^2$ con $0 \leq r \leq 1$ e $0 \leq \theta < 2\pi$. Otteniamo:
\[
\int_0^{2\pi} \int_0^1 2r \, dr \, d\theta
\]

Separiamo gli integrali:
\[
\int_0^{2\pi} d\theta \int_0^1 2r \, dr = 2 \int_0^{2\pi} d\theta \int_0^1 r \, dr = 2 \cdot 2\pi \cdot \left[ \frac{r^2}{2} \right]_0^1 = 2 \cdot 2\pi \cdot \frac{1}{2} = 2\pi
\]

Abbiamo quindi:
\[
\int_S \nabla \times \vec{F} \cdot \hat{N} \, dS = 2\pi
\]

Calcoliamo ora la circuitazione di $\vec{F}$ lungo il bordo di $S$. Il bordo di $S$ è la circonferenza $x^2 + y^2 = 1$ nel piano $z = 0$. Parametrizziamo il bordo come $\vec{r}(t) = (\cos t, \sin t, 0)$ con $t \in [0, 2\pi]$. La circuitazione di $\vec{F}$ lungo il bordo è:
\[
\int_{\partial S} \vec{F} \cdot d\vec{r} = \int_0^{2\pi} \vec{F}(\cos t, \sin t, 0) \cdot \frac{d\vec{r}}{dt} \, dt
\]
\\
\[= \int_0^{2\pi} (\sin t, -\cos t, 0) \cdot (-\sin t, \cos t, 0) \, dt = \int_0^{2\pi} (\sin^2 t + \cos^2 t) \, dt = 2\pi
\]

Abbiamo quindi verificato il teorema di Stokes per questo esempio.



\chapter{Serie numeriche}

\begin{definizione}{Serie numerica}
  Data una succesione di numeri reali $\{a_n\}_{n=1}^{\infty}$, si definisce \textbf{serie}degli $a_n$ il simbolo: $\sum_{n=1}^{\infty} a_n$.
\end{definizione}

\begin{definizione}{Somma parziale}
  Si definisce \textbf{somma parziale} di una serie numerica la somma dei primi $n$ termini della serie: $S_N = \sum_{n=1}^{N} a_n$.
\end{definizione}

Si studia il $\lim_{N \to \infty} S_N$ e si valutano i diversi casi:
\begin{itemize}
  \item Se il limite esiste finito, la serie è convergente.
  \item Se il limite è infinito, la serie è divergente.
  \item Se il limite non esiste, la serie è indeterminata.
\end{itemize}

\begin{osservazione}{}
  La definizione precedente si adatta anche al caso in cui la serie non parta da $n=1$.\\
  Il comportamento di una serie numerica non cambia se modifico un numero finito di termini. Infatti il comportamento dipende soltanto dalle "code".
\end{osservazione}

Ecco un'altra osservazione importante:
\begin{osservazione}{}
  Se le due serie $\sum_{n=1}^{\infty} a_n$ e $\sum_{n=1}^{\infty} b_n$ sono convergenti, allora la serie $\sum_{n=1}^{\infty} (a_n + b_n)$ è convergente.\\ Se invece una delle due serie è divergente, allora la serie $\sum_{n=1}^{\infty} (a_n + b_n)$ è divergente.\\
  Infine si ha che se $\sum_{n=1}^{\infty} a_n$ è convergente, allora per ogni $\lambda \in \R$, la serie $\sum_{n=1}^{\infty} \lambda a_n$ è convergente e vale $\sum_{n=1}^{\infty} \lambda a_n = \lambda \sum_{n=1}^{\infty} a_n$.
\end{osservazione}

Vediamo un caso particolare in cui è facile determinare oltre al comportamento anche il valore della somma della serie:
\begin{definizione}{Serie geometrica}
  Una serie del tipo $\sum_{n=0}^{\infty} q^n$ è detta \textbf{serie geometrica}.\\
\end{definizione}
\begin{teorema}{}
  Si ha che:
  \begin{itemize}
    \item Se $|q| < 1$, allora la serie è convergente e vale $\sum_{n=0}^{\infty} q^n = \frac{1}{1-q}$.
    \item Se $|q| \geq 1$, allora la serie è divergente.
  \end{itemize}
\end{teorema}
Vediamo come generalizzarlo per una serie che non parte da $n=0$:
\begin{osservazione}{}
  Se $|q| < 1$, allora la serie $\sum_{n=k}^{\infty} q^n$ è convergente e vale $\sum_{n=k}^{\infty} q^n = \frac{q^k}{1-q}$.
\end{osservazione}

\section{Criteri di convergenza}
Vediamo ora alcuni importanti criteri che ci permettono di determinare il comportamento di una serie numerica.
\begin{teorema}{Condizione necessaria di convergenza}
  Se la serie $\sum_{n=1}^{\infty} a_n$ è convergente, allora $\lim_{n \to \infty} a_n = 0$.
\end{teorema}

In generale è più facile studiare serie con tutti i termini positivi:
\begin{definizione}{Serie a termini positivi}
  Una serie $\sum_{n=1}^{\infty} a_n$ si dice \textbf{a termini positivi} se $a_n \geq 0$ per ogni $n$.
\end{definizione}

Vediamo subito un teorema che risolve molte patologie:
\begin{teorema}{}
  Una serie a termini positivi può essere convergente, divergente ma non indeterminata.\\
\end{teorema}
Inoltre sulle serie a termini positivi valgono i seguenti criteri:
\begin{teorema}{Criterio del confronto integrale}
  Sia $f:[1, +\infty) \to \R$ una funzione continua e a valori positivi. Se $a_n = f(n)$, allora la serie $\sum_{n=1}^{\infty} a_n$ è convergente se e solo se l'integrale $\int_{1}^{+\infty} f(x) \, dx$ è convergente.
\end{teorema}
Vediamo un esempio.\\

Studiamo la serie $\sum_{n=1}^{\infty} \frac{1}{n}$.\\
Applichiamo il criterio del confronto integrale con $f(x) = \frac{1}{x}$, ottenendo che la serie è divergente.\\

Vediamo ora un altro criterio:
\begin{teorema}{Criterio del confronto asintotico}
  \begin{itemize}
  \item  Siano $\{a_n\}_{n=1}^{\infty}$ e $\{b_n\}_{n=1}^{\infty}$ due successioni di numeri reali tali che $0 \leq a_n \leq b_n$ per ogni $n$. Se la serie $\sum_{n=1}^{\infty} b_n$ è convergente, allora la serie $\sum_{n=1}^{\infty} a_n$ è convergente.\\
  Se la serie $\sum_{n=1}^{\infty} a_n$ è divergente, allora la serie $\sum_{n=1}^{\infty} b_n$ è divergente.
  \item Siano $\{a_n\}_{n=1}^{\infty}$ e $\{b_n\}_{n=1}^{\infty}$ due successioni di numeri reali tali che $\lim_{n \to \infty} \frac{a_n}{b_n} = L$ con $0 < L < \infty$. Allora le due serie $\sum_{n=1}^{\infty} a_n$ e $\sum_{n=1}^{\infty} b_n$ sono entrambe convergenti o entrambe divergenti.
  \item Siano $\{a_n\}_{n=1}^{\infty}$ e $\{b_n\}_{n=1}^{\infty}$ due successioni di numeri reali tali che $a_n = o(b_n)$ per $n \to \infty$. Se la serie $\sum_{n=1}^{\infty} b_n$ è convergente, allora la serie $\sum_{n=1}^{\infty} a_n$ è convergente. Se la serie $\sum_{n=1}^{\infty} a_n$ è divergente, allora la serie $\sum_{n=1}^{\infty} b_n$ è divergente.
  \end{itemize}
\end{teorema}

Esistono inoltre altri due importanti criteri:
\begin{teorema}{Criterio della radice}
  Sia $\{a_n\}_{n=1}^{\infty}$ una successione di numeri reali. Se esiste il limite $\lim_{n \to \infty} \sqrt[n]{|a_n|} = L$, allora:
  \begin{itemize}
    \item Se $L < 1$, la serie $\sum_{n=1}^{\infty} a_n$ è convergente.
    \item Se $L > 1$, la serie $\sum_{n=1}^{\infty} a_n$ è divergente.
    \item Se $L = 1$, il criterio non fornisce informazioni.
  \end{itemize}
\end{teorema}
\begin{teorema}{Criterio del rapporto}
  Sia $\{a_n\}_{n=1}^{\infty}$ una successione di numeri reali. Se esiste il limite $\lim_{n \to \infty} \left| \frac{a_{n+1}}{a_n} \right| = L$, allora:
  \begin{itemize}
    \item Se $L < 1$, la serie $\sum_{n=1}^{\infty} a_n$ è convergente.
    \item Se $L > 1$, la serie $\sum_{n=1}^{\infty} a_n$ è divergente.
    \item Se $L = 1$, il criterio non fornisce informazioni.
  \end{itemize}
\end{teorema}

Inoltre ecco la seguente osservazione:
\begin{osservazione}{}
  Si può dimostrare che se esiste il limite $\lim_{n \to \infty} \left| \frac{a_{n+1}}{a_n} \right| = L$ esiste anche il limite $\lim_{n \to \infty} \sqrt[n]{|a_n|} = L$.
\end{osservazione}

In alcuni casi può essere utile la formula di Stirling per il fattoriale:
\begin{teorema}{Teorema di Stirling}
  Si ha che $\lim_{n \to \infty} \frac{n!}{(\frac{n}{e})^n \sqrt{2\pi n}} = 1$.
\end{teorema}
Vediamo un esempio in cui risulta utile.\\
Studiamo la serie $\sum_{n=1}^{\infty} \frac{n!}{n^n}$.

Applichiamo il criterio della radice:
\[
\lim_{n \to \infty} \sqrt[n]{\left| \frac{n!}{n^n} \right|} = \lim_{n \to \infty} \frac{\sqrt[n]{n!}}{n}.
\]

Utilizziamo la formula di Stirling per approssimare $n!$:
\[
n! \sim \sqrt{2\pi n} \left( \frac{n}{e} \right)^n.
\]

Quindi:
\[
\sqrt[n]{n!} \sim \sqrt[n]{ \sqrt{2\pi n} \left( \frac{n}{e} \right)^n} = \cdot \sqrt[n]{\sqrt{2\pi n}} \cdot \frac{n}{e}.
\]

Poiché $\sqrt[n]{\sqrt{2\pi n}} \to 1$ per $n \to \infty$, abbiamo:
\[
\lim_{n \to \infty} \frac{\sqrt[n]{n!}}{n} = \lim_{n \to \infty} \frac{ 1 \cdot \frac{n}{e}}{n} = \frac{1}{e}.
\]

Poiché $\frac{1}{e} < 1$, la serie $\sum_{n=1}^{\infty} \frac{n!}{n^n}$ è convergente.

\section{Serie con termini di segno variabile}
Se $a_n$ cambia segno infinite volte, i criteri visti finora non sono applicabili.\\
Un caso importante sono le serie a segni alterni:
\begin{definizione}{Serie a segni alterni}
  Una serie $\sum_{n=1}^{\infty} (-1)^n a_n$ è detta serie \textbf{a segni alterni}.
\end{definizione}
Vediamo in questo caso un utile criterio:
\begin{teorema}{Criterio di Leibniz}
  Se $\{a_n\}_{n=1}^{\infty}$ è una successione a termini positivi e decrescenti con $\lim_{n \to \infty} a_n = 0$, allora la serie $\sum_{n=1}^{\infty} (-1)^n a_n$ è convergente.
\end{teorema}
\begin{osservazione}{}
  In particolare è necessario che la serie sia infinitesima. Il fatto che sia decrescente è sufficiente ma non necessario.
\end{osservazione}
Vediamo un esempio di applicazione del criterio di Leibniz:\\

Consideriamo la serie $\sum_{n=1}^{\infty} \frac{(-1)^n}{n}$.\\
Applichiamo il criterio di Leibniz:
\begin{itemize}
  \item La successione $a_n = \frac{1}{n}$ è a termini positivi.
  \item La successione $a_n = \frac{1}{n}$ è decrescente.
  \item $\lim_{n \to \infty} \frac{1}{n} = 0$.
\end{itemize}
Quindi, per il criterio di Leibniz, la serie $\sum_{n=1}^{\infty} \frac{(-1)^n}{n}$ è convergente.\\

In un caso più generale però i segni non sono necessariamente alternati.\\
Vediamo un criterio che ci permette di studiare serie con termini di segno qualunque:
\begin{teorema}{Criterio di convergenza assoluta}
  Se la serie $\sum_{n=1}^{\infty} |a_n|$ è convergente, allora la serie $\sum_{n=1}^{\infty} a_n$ è convergente.
\end{teorema}

Vediamo un esempio di applicazione del criterio di convergenza assoluta:\\

Consideriamo la serie $\sum_{n=1}^{\infty} \frac{\sin(n)}{n^2}$.\\
Studiamo la serie $\sum_{n=1}^{\infty} \left| \frac{\sin(n)}{n^2} \right|$.\\

Poiché $|\sin(n)| \leq 1$ per ogni $n$, abbiamo:
\[
\left| \frac{\sin(n)}{n^2} \right| \leq \frac{1}{n^2}.
\]

La serie $\sum_{n=1}^{\infty} \frac{1}{n^2}$ è una serie a termini positivi e sappiamo che è convergente (serie p con $p = 2 > 1$).\\

Quindi, per il criterio di convergenza assoluta, la serie $\sum_{n=1}^{\infty} \frac{\sin(n)}{n^2}$ è convergente.






\chapter{Serie di potenze}
\section{Introduzione}
In questo capitolo vedremo le serie di potenze.
\begin{definizione}{Serie di potenze}
    Una \textbf{serie di potenze} è una serie della forma
    \[
        \sum_{n=0}^{+\infty} a_n (x-x_0)^n
    \]
    dove $a_n \in \R$ e $x,x_0 \in \R$.\\
    $x_0$ è detto il centro della serie.\\
    I numeri $a_n$ sono detti coefficienti della serie.\\
    Per ogni $x$ fissato, la serie è una serie numerica.
\end{definizione}

Può essere utile sapere per quali valori di $x$ la serie converge.\\

\begin{definizione}{Insieme di convergenza}
    Sia data una serie di potenze $\sum_{n=0}^{+\infty} a_n (x-x_0)^n$.\\
    L'\textbf{insieme di convergenza} della serie è l'insieme
    \[
        D=\left\{ x \in \R \mid \sum_{n=0}^{+\infty} a_n (x-x_0)^n \text{ converge} \right\}
    \]
\end{definizione}

\begin{osservazione}{}
    L'insieme di convergenza $D$ di una serie di potenze non è mai vuoto, in quanto la serie converge sempre per $x=x_0$.
\end{osservazione}


La teoria si basa inizialmente sul seguente teorema:
\begin{teorema}{}
  Data una serie di potenze $\sum_{n=0}^{+\infty} a_n (x-x_0)^n$, se esiste $\bar{x} \in D, \bar{x}\neq x_0$, allora:
  La serie converge assolutamente per ogni $x$ tale che $\lvert x - x_0 \rvert < \lvert \bar{x} - x_0 \rvert$.\\
\end{teorema}

\begin{proof}
  Consideriamo la serie di potenze $\sum_{n=0}^{+\infty} a_n (x-x_0)^n$ con raggio di convergenza $R > 0$. Per dimostrare il teorema, sfruttiamo la serie geometrica.

  La serie geometrica $\sum_{n=0}^{+\infty} r^n$ converge per $\lvert r \rvert < 1$ e la sua somma è data da
  \[
      \sum_{n=0}^{+\infty} r^n = \frac{1}{1-r}, \quad \text{per } \lvert r \rvert < 1.
  \]

  Ora, consideriamo la funzione somma $f(x)$ associata alla serie di potenze:
  \[
      f(x) = \sum_{n=0}^{+\infty} a_n (x-x_0)^n, \quad \text{per } x \in (x_0-R, x_0+R).
  \]

  Per ogni $x$ tale che $\lvert x - x_0 \rvert < R$, possiamo riscrivere la serie come
  \[
      f(x) = \sum_{n=0}^{+\infty} a_n (x-x_0)^n = \sum_{n=0}^{+\infty} a_n \left( \frac{x-x_0}{R} \cdot R \right)^n.
  \]

  Poniamo $r = \frac{x-x_0}{R}$, con $\lvert r \rvert < 1$. La serie diventa
  \[
      f(x) = \sum_{n=0}^{+\infty} \left( a_n R^n \right) r^n.
  \]

  Questa è una serie geometrica in $r$, con coefficienti $a_n R^n$. Poiché $\lvert r \rvert < 1$, possiamo applicare la formula della somma della serie geometrica:
  \[
      f(x) = \frac{1}{1-r}, \quad \text{dove } r = \frac{x-x_0}{R}.
  \]

  Tornando alla variabile $x$, otteniamo
  \[
      f(x) = \frac{1}{1 - \frac{x-x_0}{R}} = \frac{R}{R - (x-x_0)}.
  \]

  Questo dimostra che la funzione somma $f(x)$ è ben definita e derivabile infinite volte in $(x_0-R, x_0+R)$, e che la serie converge assolutamente per $\lvert x - x_0 \rvert < R$.
\end{proof}
\begin{definizione}{Raggio di convergenza}
  Il \textbf{raggio di convergenza} $R$ di una serie di potenze è il numero
  \[
      R = \sup \left\{ r \in \R \mid \sum_{n=0}^{+\infty} a_n r^n \text{ converge} \right\}
  \]
  Se $R = +\infty$, si dice che la serie ha raggio di convergenza infinito.\\
  Se $R = 0$, si dice che la serie ha raggio di convergenza nullo.\\
  Se $0 < R < +\infty$, si dice che la serie ha raggio di convergenza finito.
\end{definizione}

Da questo, si ricava il seguente corollario:
\begin{corollario}{}
  L'insieme di convergenza D di una serie di potenze è sempre un intervallo.\\
  Più precisamente sono possibili 3 casi (a seconda del raggio di convergenza $R$):
  \begin{itemize}
  \item $D = \R$ se $R = +\infty$
  \item $D = [x_0,x_0]$ se $R = 0$
  \item Se invece $R \inc (0,+\infty)$, allora ci sono altri quattro possibili casi:
  \begin{itemize}
  \item $D = (x_0-R,x_0+R)$
  \item $D = [x_0-R,x_0+R)$
  \item $D = (x_0-R,x_0+R]$
  \item $D = [x_0-R,x_0+R]$
  \end{itemize}
  \end{itemize}
\end{corollario}

\section{Calcolo del raggio di convergenza}
In questa sezione vedremo delle tecniche per il calcolo del raggio di convergenze per le serie di potenze.

Esistono due criteri principali: il criterio della radice e del rapporto.

\begin{teorema}{Criterio della radice}
  Sia data una serie di potenze $\sum_{n=0}^{+\infty} a_n (x-x_0)^n$.\\
  Se esiste il limite
  \[
      L = \lim_{n \to +\infty} \sqrt[n]{\lvert a_n \rvert}
  \]
  allora il raggio di convergenza $R$ della serie è
  \[
      R = \frac{1}{L}
  \]
\end{teorema}
\begin{proof}
  Consideriamo la serie numerica $\sum_{n=0}^{+\infty} \lvert a_n \rvert r^n$, dove $r = \lvert x - x_0 \rvert$.\\
  Applichiamo il criterio della radice per determinare il raggio di convergenza.\\
  Calcoliamo il limite
  \[
      L = \lim_{n \to +\infty} \sqrt[n]{\lvert a_n \rvert r^n} = \lim_{n \to +\infty} \left( \sqrt[n]{\lvert a_n \rvert} \cdot \sqrt[n]{r^n} \right) = \lim_{n \to +\infty} \sqrt[n]{\lvert a_n \rvert} \cdot r
  \]
  Poiché $\sqrt[n]{r^n} = r$, il limite diventa
  \[
      L = r \cdot \lim_{n \to +\infty} \sqrt[n]{\lvert a_n \rvert}.
  \]
  Indichiamo con $L_a = \lim_{n \to +\infty} \sqrt[n]{\lvert a_n \rvert}$.\\
  La serie converge se e solo se $L < 1$, cioè
  \[
      r \cdot L_a < 1 \quad \implies \quad r < \frac{1}{L_a}.
  \]
  Pertanto, il raggio di convergenza è
  \[
      R = \frac{1}{L_a}.
  \]
  Se $L_a = 0$, allora $R = +\infty$, e la serie converge per ogni $x \in \R$.\\
  Se $L_a = +\infty$, allora $R = 0$, e la serie converge solo per $x = x_0$.\\
  Questo conclude la dimostrazione.
\end{proof}
\begin{teorema}{Criterio del rapporto}
  Sia data una serie di potenze $\sum_{n=0}^{+\infty} a_n (x-x_0)^n$.\\
  Se esiste il limite
  \[
      L = \lim_{n \to +\infty} \left\lvert \frac{a_{n+1}}{a_n} \right\rvert
  \]
  allora il raggio di convergenza $R$ della serie è
  \[
      R = \frac{1}{L}
  \]
\end{teorema}
La dimostrazione è analoga a quella appena vista per il criterio della radice.\\
Vediamo un esempio di applicazione del criterio del rapporto.\\
  Consideriamo la serie di potenze
  \[
      \sum_{n=0}^{+\infty} \frac{x^n}{2^n}
  \]
  Applichiamo il criterio del rapporto per determinare il raggio di convergenza.\\
  Calcoliamo il limite
  \[
      L = \lim_{n \to +\infty} \left\lvert \frac{a_{n+1}}{a_n} \right\rvert = \lim_{n \to +\infty} \left\lvert \frac{\frac{1}{2^{n+1}}}{\frac{1}{2^n}} \right\rvert = \lim_{n \to +\infty} \frac{1}{2} = \frac{1}{2}
  \]
  Poiché $L = \frac{1}{2}$, il raggio di convergenza $R$ è
  \[
      R = \frac{1}{L} = 2
  \]
  Quindi, la serie converge per ogni $x$ tale che $\lvert x \rvert < 2$.

  Verifichiamo il comportamento agli estremi dell'intervallo di convergenza.\\
  Per $x = 2$, la serie diventa
  \[
      \sum_{n=0}^{+\infty} \frac{2^n}{2^n} = \sum_{n=0}^{+\infty} 1
  \]
  che è divergente.\\
  Per $x = -2$, la serie diventa
  \[
      \sum_{n=0}^{+\infty} \frac{(-2)^n}{2^n} = \sum_{n=0}^{+\infty} (-1)^n
  \]
  che è anch'essa divergente.\\
  Pertanto, la serie converge per $\lvert x \rvert < 2$ e diverge per $x = \pm 2$.\\

  \section{Studio della funzione somma}
  In questa sezione vedremo come studiare la funzione somma di una serie di potenze.\\
  Ad una serie di potenze $\sum_{n=0}^{+\infty} a_n (x-x_0)^n$ è associata una funzione somma $f(x): (x_0-R,x_0+R) \to \R$ definita come segue: per ogni $x \in (x_0-R,x_0+R)$, si ha
  \[
      f(x) = \sum_{n=0}^{+\infty} a_n (x-x_0)^n
  \]
  Vediamo ora un importante teorema che ci dà diverse informazioni sullo studio della funzione somma:

  \begin{teorema}{}
    Sia data una serie di potenze $\sum_{n=0}^{+\infty} a_n (x-x_0)^n$ con raggio di convergenza $R>0$.\\
    Allora la funzione somma $f$ è derivabile infinite volte in $(x_0-R,x_0+R)$ e:
    \begin{itemize}
    \item $f'(x) = \sum_{n=1}^{+\infty} n a_n (x-x_0)^{n-1}$ in $(x_0-R,x_0+R)$, e ogni serie di potenze così ottenuta ha lo stesso raggio di convergenza $R$ della serie originale. Inoltre $f^{(n)}(x_0)=n!a_n$ per ogni $n$ naturale.
    \item Si può integrare termine a termine centrandosi in $x_0$: $\int f(x) \, dx = \sum_{n=0}^{+\infty} \frac{a_n}{n+1} (x-x_0)^{n+1}$ in $(x_0-R,x_0+R)$, e ogni serie di potenze così ottenuta ha lo stesso raggio di convergenza $R$ della serie originale.
    \end{itemize}
  \end{teorema}




\chapter{Serie di Fourier}

\section{Serie di Fourier}
In questo capitolo ci occuperemo delle serie di Fourier. Esse sono utilizzate per approssimare (in vari sensi che vedremo) una funzione $f:\R \to \R$ T-periodica, cioè tale che $f(T+x)=f(x)$ per un certo $T>0$ detto periodo, tramite una somma di sinusoidi di periodo $T, 2T, 3T, \dots$.

\begin{definizione}{Sinusoide}
  Si chiama sinusoide una funzione del tipo $f(x) = M\cdot \cos(\omega x +\phi)$.
  \begin{itemize}
    \item $M$ è l'ampiezza, ovvero il valore massimo della funzione.
    \item $\omega$ è la pulsazione, definita come $\omega = \frac{2\pi}{T}$, dove $T$ è il periodo.
    \item $\phi$ è la fase, che rappresenta uno spostamento orizzontale della funzione.
    \item $\nu$ è la frequenza, definita come $\nu = \frac{1}{T}$, ovvero il numero di cicli per unità di tempo.
    \item $T$ è il periodo, il tempo necessario affinché la funzione completi un ciclo completo.
  \end{itemize}
\end{definizione}
Una formulazione equivalente ma spesso più utile per la sinusoide è:\\
$f(x) = A\cos(\omega x) + B\sin(\omega x)$.\\

L'idea di base dovuta ha Fourier è questa: è possibile scrivere una generica funzione come somma di sinusoidi con periodo diverso?
\\
Per formalizzarla occorre la seguente definizione:
\begin{definizione}{Polinomio trigonometrico}
  Si chiama polinomio trigonometrico di grado $N>0$ e periodo T una funzione del tipo:
  $P_N(x) = a_0 + \sum_{n=1}^{N} \left( a_n \cos\left(\frac{2\pi nx}{T}\right) + b_n \sin\left(\frac{2\pi nx}{T}\right) \right)$
\end{definizione}
Qui il coefficiente $a_0$ si può immaginare come sinusoide con frequenza nulla.

\begin{osservazione}{}
  Valgono le relazioni di ortogonalità:
  \begin{itemize}
    \item $\int_{0}^{T} \cos\left(\frac{2\pi nx}{T}\right) \cos\left(\frac{2\pi mx}{T}\right) \, dx =$
    \begin{itemize}
      \item $T$ se $n = m = 0$
      \item $\frac{T}{2}$ se $n = m \neq 0$
      \item $0$ se $n \neq m$
    \end{itemize}
    \item $\int_{0}^{T} \sin\left(\frac{2\pi nx}{T}\right) \sin\left(\frac{2\pi mx}{T}\right) \, dx =$
    \begin{itemize}
      \item $0$ se $n = 0$ o $m = 0$
      \item $\frac{T}{2}$ se $n = m \neq 0$
      \item $0$ se $n \neq m$
    \end{itemize}
    \item $\int_{0}^{T} \cos\left(\frac{2\pi nx}{T}\right) \sin\left(\frac{2\pi mx}{T}\right) \, dx = 0$ per ogni $n, m$
  \end{itemize}
\end{osservazione}
Ora siamo pronti a dare un risultato cruciale per le serie di Fourier.

\begin{teorema}{Teorema di miglior approssimazione quadratica}

  Sia $f$ una funzione integrabile su $[0, T]$. Allora esiste un unico polinomio trigonometrico $P_N$ di grado $N$ tale che
  $$
  \int_{0}^{T} \left| f(x) - P_N(x) \right|^2 \, dx
  $$
  è minimo tra tutti i polinomi trigonometrici di grado $N$ e periodo $T$.
  I coefficienti $a_n$ e $b_n$ che minimizzano l'integrale sono dati da:
  $$
  a_n = \frac{2}{T} \int_{0}^{T} f(x) \cos\left(\frac{2\pi nx}{T}\right) \, dx
  $$
  e
  $$
  b_n = \frac{2}{T} \int_{0}^{T} f(x) \sin\left(\frac{2\pi nx}{T}\right) \, dx
  $$
  Il coefficiente $a_0$ è dato da:
  $$
  a_0 = \frac{1}{T} \int_{0}^{T} f(x) \, dx
  $$

\end{teorema}

\section{Convergenza nelle serie di Fourier}
Un primo modo di vedere che qualunque $f$ periodica è sovrapposizione di sinusoidi è il seguente:
\begin{teorema}{Teorema della convergenza in media quadratica}
  Sia $f$ una funzione integrabile su $[0, T]$. Allora la serie di Fourier di $f$ converge a $f$ in media quadratica, cioè:
  $$
  \lim_{N \to \infty} \int_{0}^{T} \left| f(x) - S_N(f)(x) \right|^2 \, dx = 0
  $$
  dove $S_N(f)(x)$ è la somma parziale della serie di Fourier di $f$ di grado $N$:
  $$
  S_N(f)(x) = a_0 + \sum_{n=1}^{N} \left( a_n \cos\left(\frac{2\pi nx}{T}\right) + b_n \sin\left(\frac{2\pi nx}{T}\right) \right)
  $$
\end{teorema}
Vale inoltre un'utile identità detta identità di Parseval:
\begin{osservazione}{Identità di Parseval}
  L'identità di Parseval afferma che:
  $$
  \int_{0}^{T} \left| f(x) \right|^2 \, dx = T \left( a_0^2 + \frac{1}{2} \sum_{n=1}^{\infty} \left( a_n^2 + b_n^2 \right) \right)
  $$
\end{osservazione}
Oltre alla convergenza in media quadratica, esistono altri due tipi di convergenza: puntuale e uniforme. Per capire meglio cosa significano occorre introdurre il concetto di funzione regolarizzata.
\begin{definizione}{Funzione regolarizzata}
  Si chiama funzione regolarizzata di $f$ una funzione $\tilde{f}$ ottenuta come media del limite in ogni punto. Formalmente, per ogni $x \in \R$, la funzione regolarizzata $\tilde{f}$ è definita come:
  $$
  \tilde{f}(x) = \frac{1}{2}\cdot (\lim_{t \to x^+} f(t)+\lim_{t \to x^-} f(t))
  $$

\end{definizione}
In particolare nei punti in cui $f$ è continua si ha $\tilde{f}=f$, negli altri punti $\tilde{f}$ assume un valore intermedio dove c'è un salto.\\
Vediamo un importante teorema:
\begin{teorema}{Teorema di convergena puntuale}
  Se $f$ è una funzione T-periodica, continua a tratti e $C^1$ a tratti, allora la serie di Fourier di $f$ converge puntualmente alla regolarizzata $\tilde{f}$, cioè per ogni $x \inc \R$ vale: $\lim_{N \to \infty} P_N(x) = \tilde{f}(x)$.
\end{teorema}

L'ipotesi $C^1$ a tratti può essere sostituita con monotona a tratti.
\begin{osservazione}{}
  Le ipotesi $C^1$ a tratti o monotona a tratti sono sufficienti ma non necessarie
\end{osservazione}

Vediamo ora cosa significa invece convergenza uniforme:
\begin{definizione}{Convergenza uniforme}
  Una serie di funzioni $\sum_{n=0}^{\infty} f_n(x)$ converge uniformemente ad una funzione $f(x)$ su un intervallo $I$ se, per ogni $\epsilon > 0$, esiste un intero $N$ tale che per ogni $n \geq N$ e per ogni $x \in I$ si ha:
  $$
  \left| \sum_{k=0}^{n} f_k(x) - f(x) \right| < \epsilon
  $$
  In altre parole, la convergenza uniforme richiede che la serie converga a $f(x)$ in modo uniforme su tutto l'intervallo $I$, indipendentemente dal punto $x$ scelto.
\end{definizione}

Diamo ora un teorema sulla convergenza uniforme:
\begin{teorema}{Teorema di convergenza uniforme}
  Se $f$ è T-periodica, continua su $\R$ (non solamente a tratti) e $C^1$ a tratti, allora la serie di Fourier di $f$ converge uniformemente a $f$, cioè vale:\\
  $ \lim_{n \to \infty} \max_{x \inc \R} |P_N(x)-f(x)|=0$
\end{teorema}

Vediamo ora un ultimo risultato che ci permette di dire qualcosa sulla convergenza della serie di Fourier di $f$ senza conoscere esplicitamente la funzione ma solo attraverso i coefficienti della serie:
\begin{teorema}{}
  Se ${a_n,b_n}$ sono i coefficienti della serie di Fourier di una funzione $f$ T-periodica (continua a tratti) e se la serie numerica:\\
  $\sum_{n=1}^{\infty} |a_n|+|b_n|$\\
  è convergente, allora la $\tilde{f}$ da cui la serie di Fourier proviene è continua e la serie di Fourier converge uniformemente.
\end{teorema}


\backmatter

\input{_postamble.tex}
\end{document}
