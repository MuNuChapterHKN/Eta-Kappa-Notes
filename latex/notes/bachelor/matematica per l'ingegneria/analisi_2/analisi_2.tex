%! TEX root = analisi_2.tex

% Packages
\usepackage{listings}                    % Code highlighting
\usepackage{xcolor}                      % Custom colors
\usepackage{longtable}                   % Breakable tables
\usepackage{ulem}                        % Underline
\usepackage{contour}                     % Border around text
\usepackage{tcolorbox}                   % Custom boxes

% Primary (Accent) Colors
% Primary (Accent) Colors
\definecolor{accentYellow}{RGB}{254, 196, 41}  % #FEC421
\definecolor{accentRed}{RGB}{236, 45, 36}      % #EC2D24

% Secondary Colors
\definecolor{supportOrange}{RGB}{242, 183, 5}  % #F2B705
\definecolor{supportDarkBlue}{RGB}{55, 81, 113} % #375171

% Background Colors
\definecolor{backgroundLight}{RGB}{242, 242, 242} % #F2F2F2

% Text & Border Colors
\definecolor{textGrayBlue}{RGB}{100, 117, 140}   % #64758C
\definecolor{textGrayMedium}{RGB}{146, 154, 166}  % #929AA6
\definecolor{textGrayLight}{RGB}{184, 187, 191}   % #B8BBBF



% Listings style
\lstdefinestyle{hkn}{
  basicstyle=\ttfamily\small\color{textGrayBlue},                         % Base style (size and font)
  keywordstyle=\bfseries\color{accentRed},           % Keywords in red (important, eye-catching)
  identifierstyle=\color{supportDarkBlue},               % Identifiers in blue (clear distinction)
  commentstyle=\color{textGrayMedium},                 % Comments in gray-blue (less prominent)
  stringstyle=\color{supportOrange},                 % Strings in orange (warm and readable)
  numberstyle=\ttfamily\scriptsize\color{textGrayMedium}, % Line numbers in gray (non-intrusive)
  backgroundcolor=\color{backgroundLight},           % Light background for contrast
  rulecolor=\color{textGrayLight},                   % Soft gray border for structure
  frame=single,                          % Border around code (single, double, shadowbox, none)
  framerule=0.8pt,                       % Border thickness
  frameround=tttt,                       % Round all corners
  framesep=5pt,                          % Distance between border and code
  rulesep=2pt,                           % Distance between border and code line
  numbers=left,                          % Line number position (left, right, none)
  stepnumber=1,                          % Line number interval
  numbersep=10pt,                        % Distance between line numbers and code
  xleftmargin=30pt,                      % Left margin
  xrightmargin=30pt,                     % Right margin
  resetmargins=true,                     % Reset margins
  numberblanklines=false,                % Number blank lines
  firstnumber=auto,                      % Initial line number
  columns=fixed,                         % Fixed column width
  showstringspaces=false,                % Show spaces in strings
  tabsize=2,                             % Tab size
  breaklines=true,                       % Automatic line break for long lines
  breakatwhitespace=true,                % Line break at whitespace
  breakautoindent=true,                  % Automatic indentation after line break
  escapeinside={(*@}{@*)}                % LaTeX commands in code
}

% Underline settings
\renewcommand{\ULdepth}{1.8pt} % Underline depth
\contourlength{0.8pt}

% Custom underline command
\newcommand{\myuline}[1]{%
\uline{\phantom{#1}}%
\llap{\contour{white}{#1}}%
}

% tcolorbox color settings
\definecolor{tcolorboxLeftColor}{RGB}{2, 65, 191}
\definecolor{tcolorboxBackTitleColor}{RGB}{119, 152, 255}
\definecolor{tcolorboxBackColor}{RGB}{210, 226, 255}

% Custom boxes
\newtcolorbox[auto counter, number within=chapter]{definition}[1]{
  title={\iflanguage{italian}{Definizione}{Definition}\par~\arabic{\tcbcounter}.~#1},
  boxrule=0mm,                       % Bordo principale (disabilitato)
  leftrule=1mm,                    % Bordo sinistro principale
  arc=2mm,
  colframe=accentRed,       % Colore bordo
  colbacktitle=textGrayMedium,
  colback=backgroundLight,        % Colore sfondo
  fonttitle=\bfseries,
  rounded corners=all,               % Bordi arrotondati
  }

\newtcolorbox[auto counter, number within=chapter]{theorem}[1]{
  title={\iflanguage{italian}{Teorema}{Theorem}~\arabic{\tcbcounter}.~#1},
  boxrule=0mm,                       % Bordo principale (disabilitato)
  leftrule=1mm,                    % Bordo sinistro principale
  arc=2mm,
  colframe=accentYellow,       % Colore bordo
  colbacktitle=textGrayMedium,
  colback=backgroundLight,        % Colore sfondo
  fonttitle=\bfseries,
  rounded corners=all,               % Bordi arrotondati
}

\newtcolorbox[auto counter, number within=chapter]{corollary}[1]{
  title={\iflanguage{italian}{Corollario}{Corollary}~\arabic{\tcbcounter}.~#1},
  boxrule=0mm,                       % Bordo principale (disabilitato)
  leftrule=1mm,                    % Bordo sinistro principale
  arc=2mm,
  colframe=supportOrange,       % Colore bordo
  colbacktitle=textGrayMedium,
  colback=backgroundLight,        % Colore sfondo
  fonttitle=\bfseries,
  rounded corners=all,               % Bordi arrotondati
}

\newtcolorbox[auto counter, number within=chapter]{exercise}[1]{
  title={\iflanguage{italian}{Esercizio}{Exercise}~\arabic{\tcbcounter}.~#1},
  boxrule=0mm,                       % Bordo principale (disabilitato)
  leftrule=1mm,                    % Bordo sinistro principale
  arc=2mm,
  colframe=supportDarkBlue,       % Colore bordo
  colbacktitle=textGrayMedium,
  colback=backgroundLight,        % Colore sfondo
  fonttitle=\bfseries,
  rounded corners=all,               % Bordi arrotondati
}

\newtcolorbox[auto counter, number within=chapter]{observation}[1]{
  title={\iflanguage{italian}{Osservazione}{Observation}~\arabic{\tcbcounter}.~#1},
  boxrule=0mm,                       % Bordo principale (disabilitato)
  leftrule=1mm,                    % Bordo sinistro principale
  arc=2mm,
  colframe=textGrayBlue,       % Colore bordo
  colbacktitle=textGrayMedium,
  colback=backgroundLight,        % Colore sfondo
  fonttitle=\bfseries,
  rounded corners=all,               % Bordi arrotondati
  }


%adeguare formattazione
\newcommand{\deff}{\subsubsection{Definizione}}
\newcommand{\deffname}[1]{\subsubsection{Definizione ($\textit{#1}$)}}
\newcommand{\prop}{\subsubsection{Proposizione}}
\newcommand{\propname}[1]{\subsubsection{Proposizione ($\textit{#1}$)}}
\newcommand{\thh}{\subsubsection{Teorema}}
\newcommand{\thhname}[1]{\subsubsection{Teorema ($\textit{#1}$)}}
\newcommand{\oss}{\subsubsection{Osservazione}}
\newcommand{\R}{\mathbb{R}}
\newcommand{\C}{\mathbb{C}}
\newcommand{\Ll}{\mathcal{L}}
\newcommand{\gammadef}{\gamma:[a, b] \rightarrow \R^m}
\newcommand{\suminf}[1]{\sum_{n=#1}^{\infty}}
\newcommand{\suminfan}{\sum_{n=0}^{\infty}a_n}
\newcommand{\liminff}{\lim\limits_{n->\infty}}
\newcommand{\inc}{\subseteq}

\begin{document}

\title{Analisi 2}
\date{\today}
\author{Tommaso Pignatelli}

\frontmatter
\maketitle
\tableofcontents

\mainmatter
\chapter{Introduzione}

Scriviamo qualcosa.





\chapter{Calcolo differenziale}

\section{Introduzione}
\begin{definizione}{Funzione di più variabili}
  Una \textbf{funzione di più variabili} è una applicazione $f:\Omega \subset \R^n \rightarrow \R$, con $n\geq 1$. $\Omega$ è detto dominio della funzione.
  \end{definizione}

Come si visualizza una funzione di più variabili?\\
Ci sono due strumenti principali: il grafico e le curve di livello.\\

\begin{definizione}{Curve di livello}
Data $f:\Omega \inc \R^2 \rightarrow \R$, si definisce \textbf{curva di livello} $\lambda$ di $f$ il luogo dei punti $L_\lambda=\{(x,y)\in \Omega : f(x,y)=\lambda\}$.\\
\end{definizione}
Può capitare che $L_\lambda$ sia vuoto, es: $L_{-1}$ per $f(x,y)= x^2+y^2$.\\
Tipicamente $L_\lambda$ è una curva nel piano o l'unione di più curve.\\


\begin{definizione}{Grafico}
Data $f:\Omega \inc \R^2 \rightarrow \R$, si definisce \textbf{grafico} di $f$ il luogo dei punti $G_f=\{(x,y,z)\in \R^3 : z=f(x,y), (x,y)\in \Omega\}$.\\
\end{definizione}
In generale $\Omega \inc \R^n$ e $f:\Omega \rightarrow \R$ si definisce $G(f)=\{(x_1, \dots, x_n, z)\in \R^{n+1} : z=f(x_1, \dots, x_n), (x_1, \dots, x_n)\in \Omega\}$.\\

\subsection{Esempio di grafici da sapere}
Sono esempi di grafici da sapere i seguenti:
\begin{itemize}
  \item $f(x,y)=x^2+y^2$ è il grafico di un paraboloide
  \item $f(x,y)=x^2-y^2$ è il grafico di un iperbolide
  \item $f(x,y)=x^2$ è il grafico di un cilindro
  \item $f(x,y)=\sqrt{1-x^2-y^2}$ è il grafico di una semisfera
  \item $f(x,y)=\sqrt{x^2+y^2}$ è il grafico di un cono
\end{itemize}

\section{Cenni di limiti e continuità}

\begin{definizione}{Intorno di un punto}
Sia $\vec x_0 \in \R^n$ e sia $r>0$, si dice \textbf{intorno} di $\vec x_0$ la palla aperta $B_r(\vec x_0)=\{\vec x \in \R^n : \|\vec x - \vec x_0\|<r\}$, dove $\|\cdot\|$ denota la norma euclidea.\\
\end{definizione}

\begin{definizione}{Limite}
Sia $f:\Omega \inc \R^n \rightarrow \R$ e sia $p_0$ un punto di accumulazione per $\Omega$. Si dice che $f$ tende a $L \in \R$ per $p$ che tende a $p_0$ (cioè che il \textbf{limite} per $p$ che tende a $p_0$ di $f(p)$ è $L$), e si scrive
$$\lim_{p \to p_0} f(p) = L$$
se per ogni $\epsilon > 0$ esiste $\delta > 0$ tale che, per ogni $p \in \Omega$, $0 < \|p - p_0\| < \delta$ implica $|f(p) - L| < \epsilon$.
\end{definizione}

\begin{definizione}{Continuità}
Sia $f:\Omega \inc \R^n \rightarrow \R$ e sia $p_0 \in \Omega$. Si dice che $f$ è \textbf{continua} in $p_0$ se
$$\lim_{p \to p_0} f(p) = f(p_0).$$
\end{definizione}

Data $f$ come faccio a capire in quali punti è continua?\\
Si usano i seguenti fatti:
\begin{itemize}
  \item Una funzione di una sola variabile è continua se non presenta discontinuità nel suo dominio.
  \item Se $f$ e $g$ sono funzioni continue, allora le seguenti funzioni sono continue:
  \begin{itemize}
    \item La somma: $f + g$
    \item Il prodotto: $f \cdot g$
    \item Il rapporto: $\frac{f}{g}$, purché $g \neq 0$
    \item La composizione: $f \circ g$
  \end{itemize}
\end{itemize}

\subsection{Esempio di limite lungo semirette}
Consideriamo la funzione $f(x,y)= \frac{2xy}{x^2+y^2}$. Vogliamo studiare il limite di $f$ quando $(x,y)$ tende a $(0,0)$ lungo diverse semirette.

Osserviamo prima il comportamento lungo gli assi $x$ e $y$:
\[
f(x,0) = \frac{2x \cdot 0}{x^2 + 0^2} = 0
\]
\[
f(0,y) = \frac{2 \cdot 0 \cdot y}{0^2 + y^2} = 0
\]
Quindi, lungo gli assi il limite è $0$.

Consideriamo ora il limite lungo la semiretta $y=mx$ con $m$ costante:
\[
f(x,mx) = \frac{2x(mx)}{x^2+(mx)^2} = \frac{2mx^2}{x^2(1+m^2)} = \frac{2m}{1+m^2}
\]
Quindi, il limite dipende da $m$ e vale $\frac{2m}{1+m^2}$.

Osserviamo che il limite di $f(x,y)$ quando $(x,y)$ tende a $(0,0)$ dipende dalla direzione lungo la quale ci avviciniamo al punto $(0,0)$. Pertanto, possiamo concludere che il limite non esiste in senso assoluto.

\subsection{Esempio di limite lungo parabole}
Consideriamo la funzione $f(x,y)= \frac{x^2y}{y^2+x^4}$. Vogliamo studiare il limite di $f$ quando $(x,y)$ tende a $(0,0)$ lungo diverse curve.

Lungo la semiretta $y=mx$ con $m$ costante:
\[
f(x,mx) = \frac{x^2(mx)}{(mx)^2+x^4} = \frac{mx^3}{m^2x^2+x^4} = \frac{mx^3}{x^2(m^2+x^2)} = \frac{mx}{m^2+x^2}
\]
Quando $x \to 0$, il limite di $f(x,mx)$ è $0$.

Consideriamo la parabola $y=mx^2$ con $m$ costante:
\[
f(x,mx^2) = \frac{x^2(mx^2)}{(mx^2)^2+x^4} = \frac{mx^4}{m^2x^4+x^4} = \frac{mx^4}{x^4(m^2+1)} = \frac{m}{m^2+1}
\]
Quando $x \to 0$, il limite di $f(x,mx^2)$ è $0$ solo se $m=0$. Per $m \neq 0$, il limite non è $0$.

Osserviamo che il limite di $f(x,y)$ quando $(x,y)$ tende a $(0,0)$ dipende dalla curva lungo la quale ci avviciniamo al punto $(0,0)$. Pertanto, possiamo concludere che il limite non esiste in senso assoluto.

\section{Derivate parziali}
\begin{definizione}{Derivata in una variabile}
Sia $f: \R \rightarrow \R$ una funzione e sia $x_0 \in \R$. La \textbf{derivata} di $f$ in $x_0$, indicata con $f'(x_0)$, è definita come il limite del rapporto incrementale, se esiste:
$$f'(x_0) = \lim_{h \to 0} \frac{f(x_0 + h) - f(x_0)}{h}.$$
Geometricamente, la derivata rappresenta il coefficiente angolare della retta tangente al grafico di $f$ nel punto $(x_0, f(x_0))$.
\end{definizione}

\begin{definizione}{Derivata parziale}
  Sia ora $P_0 \in \R^n $ e $f(x_1, \dots, x_n)$ una funzione definita in un intorno di $P_0$. La \textbf{derivata parziale} di $f$ rispetto alla variabile $x_i$ in $P_0$ è definita come il limite del rapporto incrementale, se esiste: $
\frac{\partial f}{\partial x_i}(P_0) = \lim_{h \to 0} \frac{f(P_0+e_i\cdot h) - f(P_0)}{h}$, dove $e_i$ indica il versore fondamentale lungo la componente $i$.
\end{definizione}
Questo equivale a derivare $f(x_1,\dots, x_n)$, rispetto alla variabile $x_i$, lasciando fissi i vettori delle altre $n-1$ componenti.

In particolare, per una funzione di due variabili $f(x,y)$, esistono due derivate parziali: la derivata parziale rispetto a $x$ e la derivata parziale rispetto a $y$. Queste sono indicate rispettivamente con $\frac{\partial f}{\partial x}$ e $\frac{\partial f}{\partial y}$. \\
Consideriamo la funzione $f(x,y) = x^2 + y^2$. Le derivate parziali di $f$ sono:
\[
\frac{\partial f}{\partial x} = \frac{\partial}{\partial x}(x^2 + y^2) = 2x
\]
\[
\frac{\partial f}{\partial y} = \frac{\partial}{\partial y}(x^2 + y^2) = 2y
\]
Quindi, la derivata parziale rispetto a $x$ è $2x$ e la derivata parziale rispetto a $y$ è $2y$.\\

\section{Piano tangente}
Il piano tangente è un concetto fondamentale nel calcolo differenziale e rappresenta il piano che "tocca" una superficie in un punto dato, approssimando la superficie stessa vicino a quel punto.

\begin{definizione}{Piano tangente}
Sia $f:\Omega \inc \R^2 \rightarrow \R$ differenziabile in $(x_0,y_0) \in \Omega$, si dice \textbf{piano tangente} al grafico di $f$ in $(x_0,y_0,f(x_0,y_0))$ il piano:
$$z=f(x_0,y_0)+\frac{\partial f}{\partial x}(x_0,y_0)(x-x_0)+\frac{\partial f}{\partial y}(x_0,y_0)(y-y_0)$$
\end{definizione}

In altre parole, il piano tangente al grafico di una funzione $f(x,y)$ in un punto $(x_0, y_0)$ è dato dall'equazione lineare che approssima $f$ vicino a quel punto. Le derivate parziali $\frac{\partial f}{\partial x}$ e $\frac{\partial f}{\partial y}$ rappresentano le pendenze del piano tangente nelle direzioni $x$ e $y$, rispettivamente.
\\
Consideriamo la funzione $f(x,y) = x^2 + y^2$. Troviamo il piano tangente al grafico di $f$ nel punto $(1,1)$.

Calcoliamo le derivate parziali:
\[
\frac{\partial f}{\partial x} = 2x \quad \text{e} \quad \frac{\partial f}{\partial y} = 2y
\]

Valutiamo le derivate parziali nel punto $(1,1)$:
\[
\frac{\partial f}{\partial x}(1,1) = 2 \quad \text{e} \quad \frac{\partial f}{\partial y}(1,1) = 2
\]

L'equazione del piano tangente è quindi:
\[
z = f(1,1) + \frac{\partial f}{\partial x}(1,1)(x-1) + \frac{\partial f}{\partial y}(1,1)(y-1)
\]
\[
z = 1^2 + 1^2 + 2(x-1) + 2(y-1)
\]
\[
z = 2 + 2(x-1) + 2(y-1)
\]
\[
z = 2x + 2y - 2
\]

Quindi, il piano tangente al grafico di $f$ nel punto $(1,1)$ è dato dall'equazione $z = 2x + 2y - 2$.\\


La differenza rispetto al caso in una variabile è che in due (o più) variabili può capitare che esistanto tutte le derivate parziali ma $f$ potrebbe non solo non avere piano tangente al grafico ma addirittura essere discontinua in $P_0$ (si veda ad esempio la funzione nulla sugli assi, e che vale $1$ altrove).

\section{Differenziabilità}
\begin{definizione}{Differenziabilità in due variabili}
Sia $f:\Omega \inc \R^2 \rightarrow \R$ e sia $P_0 = (x_0,y_0) \in \Omega$. Si dice che $f$ è \textbf{differenziabile} in $P_0$ se esistono le derivate parziali di $f$ in $P_0$ rispetto a $x$ e $y$ e se vale lo sviluppo di Taylor di ordine 1 in $P_0$ con resto di Peano, cioè: $f(x,y)=f(x_0,y_0)+\frac{\partial f}{\partial x}(x_0,y_0)(x-x_0)+\frac{\partial f}{\partial y}(x_0,y_0)(y-y_0)+o(\sqrt{(x-x_0)^2+(y-y_0)^2}) \quad \text{per }(x,y) \rightarrow (x_0,y_0)$.
\end{definizione}

\begin{osservazione}{}
Se $f$ è differenziabile in $(x_0,y_0)$, allora il piano di equazione $z=f(x_0,y_0)+\frac{\partial f}{\partial x}(x_0,y_0)(x-x_0)+\frac{\partial f}{\partial y}(x_0,y_0)(y-y_0)$ è il piano tangente al grafico di $f$ in $(x_0,y_0,f(x_0,y_0))$.
\end{osservazione}
\begin{osservazione}{}
  La differenziabilità esprime il fatto che f(x,y) è approssibile con una funzione lineare con un errore che è $o(\sqrt{(x-x_0)^2+(y-y_0)^2}) \quad \text{per }(x,y) \rightarrow (x_0,y_0)$.
\end{osservazione}

\begin{definizione}{Gradiente}
Sia $f:\Omega \inc \R^n \rightarrow \R$ e sia $P_0 \in \Omega$. Si dice \textbf{gradiente} di $f$ in $P_0$ il vettore: $\nabla f(P_0)=\left(\frac{\partial f}{\partial x_1}(P_0), \dots, \frac{\partial f}{\partial x_n}(P_0)\right)$.
\end{definizione}

\begin{definizione}{Differenziabilità in $n$ variabili}
Sia $f:\Omega \inc \R^n \rightarrow \R$ e sia $P_0 \in \Omega$. Si dice che $f$ è \textbf{differenziabile} in $P_0$ se esistono le derivate parziali di $f$ in $P_0$ rispetto a tutte le variabili e se vale lo sviluppo di Taylor di ordine 1 in $P_0$ con resto di Peano, cioè: $x_{n+1}=f(P_0)+\nabla f(P_0)\cdot (P - P_0)+o(|P-P_0|) \quad \text{per } P \rightarrow P_0$.
\end{definizione}

Per analogia il "piano" (iperpiano) tangente al grafico di $f$ in $P_0$ è dato dall'equazione $z=f(P_0)+\nabla f(P_0)\cdot (P - P_0)$. In ogni caso resta vera la proprietà di approssimazione lineare con errore $o(|P-P_0|) \quad \text{per } P \rightarrow P_0$.\\

Per garantire che $f$ sia effettivamente differenziabile è utile il seguente teorema:
\begin{teorema}{Criterio di differenziabilità}
Se esistono le derivate parziali di $f$ in $P_0$ e sono continue in un intorno di $P_0$, allora $f$ è differenziabile in $P_0$.
\end{teorema}

\subsection{Esempio di verifica della differenziabilità e calcolo del piano tangente}
Consideriamo la funzione $f(x,y) = \frac{e^{2x+y}}{x^2+y^2}$. Vogliamo verificare il teorema di differenziabilità e calcolare il piano tangente al grafico di $f$ nel punto $(1,1)$.

Calcoliamo le derivate parziali di $f$ rispetto a $x$ e $y$:
\[
\frac{\partial f}{\partial x} = \frac{(2e^{2x+y})(x^2+y^2) - e^{2x+y}(2x)}{(x^2+y^2)^2} = \frac{2e^{2x+y}(x^2+y^2) - 2xe^{2x+y}}{(x^2+y^2)^2}
\]
\[
\frac{\partial f}{\partial y} = \frac{(e^{2x+y})(x^2+y^2) - e^{2x+y}(2y)}{(x^2+y^2)^2} = \frac{e^{2x+y}(x^2+y^2) - 2ye^{2x+y}}{(x^2+y^2)^2}
\]
Osserviamo che le derivate parziali sono continue in un intorno di $(1,1)$, quindi possiamo concludere che $f$ è differenziabile in $(1,1)$.\\
Valutiamo le derivate parziali nel punto $(1,1)$:
\[
\frac{\partial f}{\partial x}(1,1) = \frac{2e^{3}(1^2+1^2) - 2e^{3}}{(1^2+1^2)^2} = \frac{4e^{3} - 2e^{3}}{4} = \frac{2e^{3}}{4} = \frac{e^{3}}{2}
\]
\[
\frac{\partial f}{\partial y}(1,1) = \frac{e^{3}(1^2+1^2) - 2e^{3}}{(1^2+1^2)^2} = \frac{2e^{3} - 2e^{3}}{4} = 0
\]

L'equazione del piano tangente è quindi:
\[
z = f(1,1) + \frac{\partial f}{\partial x}(1,1)(x-1) + \frac{\partial f}{\partial y}(1,1)(y-1)
\]
\[
z = \frac{e^{3}}{2} + \frac{e^{3}}{2}(x-1) + 0(y-1)
\]
\[
z = \frac{e^{3}}{2} + \frac{e^{3}}{2}(x-1)
\]
\[
z = \frac{e^{3}}{2}(1 + x - 1)
\]
\[
z = \frac{e^{3}}{2}x
\]

Quindi, il piano tangente al grafico di $f$ nel punto $(1,1)$ è dato dall'equazione $z = \frac{e^{3}}{2}x$.\\

Dalla diiferenziabilità possiamo dunque trarre importanti conseguenze. Le più rilevanti sono riassunte nel seguente teorema:

\begin{teorema}{Conseguenze della differenziabilità}
  Sia $f$ differenziabile in $P_0 \in \R^n$. Allora:
  \begin{itemize}
    \item $f$ è continua in $P_0$
    \item Il vettore $\nabla f(P_0)$ indica nel dominio di $f$ e partendo dal punto $P_0$ la direzione di massima crescita di $f$, cioè quella lungo cui $f$ cresce più velocemente.
    \item In $P_0$ il grafico ha pendenza $\alpha$ rispetto alla direzione di $\nabla f(P_0)$, dove $\alpha =  \|\nabla f(P_0)\| = \sqrt{\left(\frac{\partial f}{\partial x_1}(P_0)\right)^2 + \cdots + \left(\frac{\partial f}{\partial x_n}(P_0)\right)^2}$.
    \item Se $n=2$ e $\nabla f(P_0)\neq (0,0)$, allora il gradiente $\nabla f(P_0)$ è ortogonale alla curva di livello di $f$ passante per $P_0$
\end{itemize}
\end{teorema}

\section{Derivata direzionale}
\begin{definizione}{Derivata direzionale}
Sia $f:\Omega \inc \R^n \rightarrow \R$ e sia $P_0 \in \Omega$. La \textbf{derivata direzionale} di $f$ in $P_0$ lungo il vettore $\vec v$ (di norma unitaria) è definita come il limite del rapporto incrementale, se esiste: $D_{\vec v}f(P_0) = \frac{\partial f}{\partial \vec v}(P_0) = \lim_{h \to 0} \frac{f(P_0 + h\vec v) - f(P_0)}{h}$.
\end{definizione}

\begin{osservazione}{}
La derivata direzionale di $f$ in $P_0$ lungo il vettore $\vec v$ rappresenta la pendenza di $f$ in $P_0$ nella direzione di $\vec v$. \\
Inoltre, definito il gradiente come in precedenza, se $f$ è differenziabile, si ha: $\frac{\partial f}{\partial \vec v}(P_0) = \nabla f(P_0) \cdot \vec v $ con $\vec v$ vettore unitario e con la notazione precedente che indica il prodotto scalare.
\end{osservazione}

\section{Derivata lungo una curva}\label{sec:derivata-lungo-una-curva}

\begin{definizione}{Curva in $\R^n$}
Una \textbf{curva} in $\R^n$ è una funzione $\gamma: I \rightarrow \R^n$ con $I$ intervallo di $\R$.
\end{definizione}

Tipicamente considereremo curve dove l'intervallo $I$ è un intervallo chiuso e limitato, cioè $I=[a,b]$.\\
$\gamma(a)$ e $\gamma(b)$ sono detti rispettivamente punto iniziale e punto finale della curva.\\
Concretamente quindi, per ogni $t\in I$ si ha $\gamma(t)=(x_1(t), \dots, x_n(t))$. Gli $x_i(t)$ sono le componenti della curva. Ciascuna componente è una funzione dell'analisi 1.\\
In seguito supporremo sempre che le componenti della curva siano funzioni di classe $C^1$.

\begin{definizione}{Derivata lungo una curva}
  La \textbf{derivata lungo la curva} $\gamma$ in $t$ è definita come il vettore $\gamma'(t)=(x_1'(t), \dots, x_n'(t))$.\\
  Se esiste la derivata seconda di $\gamma$ in $t$ allora si definisce $\gamma''(t)=(x_1''(t), \dots, x_n''(t))$ e così via.
\end{definizione}
Ora data una funzione $f:\Omega \inc \R^n \rightarrow \R$ e una curva $\gamma: [a,b] \rightarrow \R^n$ possiamo definire la derivata di $f$ lungo $\gamma$, grazie al seguente teorema:

\begin{teorema}{}
  Sia $f:\Omega \inc \R^n \rightarrow \R$ e $\gamma: [a,b] \rightarrow \R^n$ una curva. Se $f$ è differenziabile in $\gamma(t)$ per ogni $t \in [a,b]$ e se $\gamma$ è derivabile in $t$, allora la funzione composta $f \circ \gamma$ è derivabile in $t$ e si ha:
  $\frac{d}{dt}(f \circ \gamma)(t) = \nabla f(\gamma(t)) \cdot \gamma'(t)$.
\end{teorema}

\section{Derivate successive}
\begin{definizione}{Derivate successive}
  Se $f$ è differenziabile in un intorno di $P_0$ allora si definiscono le \textbf{derivate seconde} di $f$ (nei punti in cui esistono) come le derivate parziali delle derivate parziali: $\frac{\partial^2 f}{\partial x_i \partial x_j}(P_0) = \frac{\partial}{\partial x_i} \left(\frac{\partial f}{\partial x_j}\right)(P_0)$.  \\
  Quando le derivate seconde esistono si può provare ad andare a avanti a derivare fino a ottenere le derivate terze e così via.
\end{definizione}
\begin{osservazione}{}
  In linea di princio ci sono al più $n$ derivate prime, $n^2$ derivate seconde, $n^k$ derivate k-esime.
\end{osservazione}

Consideriamo ad esempio la funzione $f(x,y) = (x^2 + y^2)\sin(2y)$. Calcoliamo le derivate parziali prime e seconde di $f$.


\[
\frac{\partial f}{\partial x} = \frac{\partial}{\partial x} \left( (x^2 + y^2)\sin(2y) \right) = 2x\sin(2y)
\]
\[
\frac{\partial f}{\partial y} = \frac{\partial}{\partial y} \left( (x^2 + y^2)\sin(2y) \right) = 2y\sin(2y) + (x^2 + y^2)2\cos(2y)
\]

\[
\frac{\partial^2 f}{\partial x^2} = \frac{\partial}{\partial x} \left( 2x\sin(2y) \right) = 2\sin(2y)
\]
\[
\frac{\partial^2 f}{\partial y^2} = \frac{\partial}{\partial y} \left( 2y\sin(2y) + (x^2 + y^2)2\cos(2y) \right)
\]
\[
= 2\sin(2y) + 2y \cdot 2\cos(2y) + 2\cos(2y) \cdot 2y + (x^2 + y^2) \cdot 2(-2\sin(2y))
\]
\[
= 2\sin(2y) + 4y\cos(2y) + 4y\cos(2y) - 4(x^2 + y^2)\sin(2y)
\]
\[
= 2\sin(2y) + 8y\cos(2y) - 4(x^2 + y^2)\sin(2y)
\]
\[
= 2\sin(2y)(1 - 2(x^2 + y^2)) + 8y\cos(2y)
\]

\[
\frac{\partial^2 f}{\partial x \partial y} = \frac{\partial}{\partial y} \left( 2x\sin(2y) \right) = 2x \cdot 2\cos(2y) = 4x\cos(2y)
\]
\[
\frac{\partial^2 f}{\partial y \partial x} = \frac{\partial}{\partial x} \left( 2y\sin(2y) + (x^2 + y^2)2\cos(2y) \right) = 4x\cos(2y)
\]

Osserviamo che in questo esempio le derivate miste ($\frac{\partial^2 f}{\partial y \partial x}$,$\frac{\partial^2 f}{\partial x \partial y}$) sono uguali. In generale questo non è un caso.

\begin{teorema}{Teorema di Schwarz}
  Se le derivate miste $\frac{\partial^2 f}{\partial x \partial y}$ e $\frac{\partial^2 f}{\partial y \partial x}$ esistono e sono continue in un intorno di $P_0$ allora sono uguali.
\end{teorema}

\begin{corollario}{}
  Se $f$ è di classe $C^k$ allora le derivate miste di ordine $k$ sono indipendenti dall'ordine di derivazione.
\end{corollario}

Vediamo ora come si calcolano le derivate direzionali di ordine 2.
\begin{definizione}{Matrice Hessiana}
  Se $f$ è $C^2$ allora si definisce \textbf{matrice Hessiana} di $f$ in $P_0$ la matrice:
  $Hf(P_0)=\begin{Bmatrix}
    \frac{\partial^2 f}{\partial x_1^2}(P_0) & \frac{\partial^2 f}{\partial x_1 \partial x_2}(P_0) & \cdots & \frac{\partial^2 f}{\partial x_1 \partial x_n}(P_0)\\
    \frac{\partial^2 f}{\partial x_2 \partial x_1}(P_0) & \frac{\partial^2 f}{\partial x_2^2}(P_0) & \cdots & \frac{\partial^2 f}{\partial x_2 \partial x_n}(P_0)\\
    \vdots & \vdots & \ddots & \vdots\\
    \frac{\partial^2 f}{\partial x_n \partial x_1}(P_0) & \frac{\partial^2 f}{\partial x_n \partial x_2}(P_0) & \cdots & \frac{\partial^2 f}{\partial x_n^2}(P_0)
  \end{Bmatrix}$
\end{definizione}
\begin{osservazione}{}
  La matrice Hessiana è simmetrica.
\end{osservazione}

\begin{definizione}{Derivata direzionale di ordine 2}
  Se $f$ è $C^2$ allora la \textbf{derivata direzionale di ordine 2} di $f$ in $P_0$ lungo il vettore $\vec v$ è data da:
  $D^2_{\vec v}f(P_0)=\vec v^T Hf(P_0) \vec v$
\end{definizione}

\begin{teorema}{Sviluppo di Taylor di ordine 2}
  Se $f$ è $C^2$ allora si ha lo sviluppo di Taylor di ordine 2 di $f$ in $P_0$:
  $f(P) = f(P_0) + \nabla f(P_0) \cdot (P - P_0) + \frac{1}{2} (P - P_0)^T Hf(P_0) (P - P_0) + o(|P - P_0|^2)$
\end{teorema}

\subsection{Esempio di calcolo dello sviluppo di Taylor di ordine 2}
Consideriamo la funzione $f(x,y) = e^{x+y}$. Vogliamo trovare lo sviluppo di Taylor di ordine 2 di $f$ attorno al punto $(0,0)$.

Calcoliamo le derivate parziali prime di $f$:
\[
\frac{\partial f}{\partial x} = e^{x+y}, \quad \frac{\partial f}{\partial y} = e^{x+y}
\]

Calcoliamo le derivate parziali seconde di $f$:
\[
\frac{\partial^2 f}{\partial x^2} = e^{x+y}, \quad \frac{\partial^2 f}{\partial y^2} = e^{x+y}, \quad \frac{\partial^2 f}{\partial x \partial y} = e^{x+y}
\]

Valutiamo le derivate parziali nel punto $(0,0)$:
\[
\frac{\partial f}{\partial x}(0,0) = 1, \quad \frac{\partial f}{\partial y}(0,0) = 1
\]
\[
\frac{\partial^2 f}{\partial x^2}(0,0) = 1, \quad \frac{\partial^2 f}{\partial y^2}(0,0) = 1, \quad \frac{\partial^2 f}{\partial x \partial y}(0,0) = 1
\]

Lo sviluppo di Taylor di ordine 2 di $f$ attorno al punto $(0,0)$ è dato da:
\[
f(x,y) \approx f(0,0) + \frac{\partial f}{\partial x}(0,0)x + \frac{\partial f}{\partial y}(0,0)y + \frac{1}{2}\left( \frac{\partial^2 f}{\partial x^2}(0,0)x^2 + 2\frac{\partial^2 f}{\partial x \partial y}(0,0)xy + \frac{\partial^2 f}{\partial y^2}(0,0)y^2 \right)
\]
\[
= 1 + x + y + \frac{1}{2}(x^2 + 2xy + y^2)
\]
\[
= 1 + x + y + \frac{1}{2}(x^2 + 2xy + y^2)
\]
\[
= 1 + x + y + \frac{1}{2}x^2 + xy + \frac{1}{2}y^2
\]

Quindi, lo sviluppo di Taylor di ordine 2 di $f(x,y) = e^{x+y}$ attorno al punto $(0,0)$ è:
\[
f(x,y) \approx 1 + x + y + \frac{1}{2}x^2 + xy + \frac{1}{2}y^2
\]

\section{Massimi e minimi di una funzione}

Questa sezione è dedicata allo studio dei punti di massimo e minimo delle funzioni di più variabili e alla loro classificazione.

\begin{definizione}{Punto di minimo}
  Sia $f:\Omega \inc \R^n \rightarrow \R$ e sia $P_0 \in \Omega$. Si dice che $f$ ha un \textbf{punto di minimo} locale in $P_0$ se esiste un intorno di $P_0$ tale che $f(P) \geq f(P_0)$ per ogni $P$ nell'intorno.
\end{definizione}
Analogamente si definisce il punto di massimo locale.
\begin{teorema}{}
  Se $f$ ha un punto di minimo o massimo locale in $P_0$ e $f$ è differenziabile in $P_0$ allora $\nabla f(P_0) = 0$.
\end{teorema}


\begin{osservazione}{}
  Questa condizione corrisponde a un sistema di equazioni:
  \[
  \frac{\partial f}{\partial x_1}(P_0) = 0, \quad \frac{\partial f}{\partial x_2}(P_0) = 0, \quad \dots, \quad \frac{\partial f}{\partial x_n}(P_0) = 0
  \]
\end{osservazione}

\begin{definizione}{Punti stazionari}
  I punti $P_0$ tali che $\nabla f(P_0) = 0$ sono detti \textbf{punti stazionari}.
\end{definizione}

Facciamo un esempio.
Consideriamo la funzione $f(x,y) = x^3 - xy^2 + 2xy$. Vogliamo trovare i punti critici di $f$.

Calcoliamo le derivate parziali prime di $f$:
\[
\frac{\partial f}{\partial x} = 3x^2 - y^2 + 2y
\]
\[
\frac{\partial f}{\partial y} = -2xy + 2x
\]

Poniamo le derivate parziali uguali a zero per trovare i punti critici:
\[
3x^2 - y^2 + 2y = 0
\]
\[
-2xy + 2x = 0
\]

Dalla seconda equazione, possiamo fattorizzare:
\[
2x(-y + 1) = 0
\]

Quindi, abbiamo due casi:
1. $2x = 0 \implies x = 0$\\
2. $-y + 1 = 0 \implies y = 1$

Caso 1: $x = 0$
\[
3(0)^2 - y^2 + 2y = 0 \implies -y^2 + 2y = 0 \implies y(y - 2) = 0
\]
Quindi, $y = 0$ o $y = 2$. I punti critici sono $(0,0)$ e $(0,2)$.

Caso 2: $y = 1$
\[
3x^2 - (1)^2 + 2(1) = 0 \implies 3x^2 - 1 + 2 = 0 \implies 3x^2 + 1 = 0
\]
Questa equazione non ha soluzioni reali.

Quindi, i punti critici della funzione $f(x,y) = x^3 - xy^2 + 2xy$ sono $(0,0)$ e $(0,2)$.\\

Come possiamo capire la natura di un punto critico?
Se $P_0$ è un punto critico di $f$ e $f$ è $C^2$ in un intorno di $P_0$ allora si può studiare lo sviluppo al secondo ordine di $f$ in $P_0$ per capire la natura del punto critico. Lo sviluppo si riduce a: $f(P) - f(P_0) = \frac{1}{2} (P - P_0)^T Hf(P_0) (P - P_0) + o(|P - P_0|^2)$.\\
Si tratta dunque di capire che segno ha il membro destro in un intorno di $P_0$.
\begin{itemize}
\item $P_0$ è un punto di massimo se e solo se il membro destro è sempre negativo in un intorno di $P_0$.
\item $P_0$ è un punto di minimo se e solo se il membro destro è sempre positivo in un intorno di $P_0$.
\item $P_0$ è un punto di sella se e solo se il membro destro cambia segno in un intorno di $P_0$.
\end{itemize}
\begin{osservazione}{}
  Il segno del membro destro dipende solo da segno di $\frac{1}{2} (P - P_0)^T Hf(P_0) (P - P_0)$, che è uguale al segno di $\frac{1}{2} \frac{(P - P_0)^T}{\|P - P_0\|} Hf(P_0) \frac{P-P_0}{\|P - P_0\|}$, e avendo $ \frac{P - P_0}{\|P - P_0\|}$ norma 1, il segno non dipende dal raggio dell'intorno di $P_0$ considerato.
\end{osservazione}

Ricordiamo che una matrice simmetrica $A$ è definita positiva se per ogni vettore $v \neq 0$ si ha $v^T A v > 0$. Analogamente si definiscono le matrici definite negative quelle per cui $v^T A v < 0$ per ogni $v \neq 0$ e indefinite quelle per cui $v^T A v$ può assumere segni diversi a seconda di $v$.

\begin{teorema}{Classificazione dei punti critici}
  Sia $f$ una funzione $C^2$ e sia $P_0$ un punto critico di $f$. Allora:
  \begin{itemize}
    \item Se $Hf(P_0)$ è definita positiva allora $P_0$ è un punto di minimo locale.
    \item Se $Hf(P_0)$ è definita negativa allora $P_0$ è un punto di massimo locale.
    \item Se $Hf(P_0)$ è indefinita allora $P_0$ è un punto di sella.
  \end{itemize}
\end{teorema}

\subsection{Esempio di classificazione dei punti critici}
Consideriamo la funzione $f(x,y) = x^3 - xy^2 + 2xy$. Abbiamo già trovato i punti critici $(0,0)$ e $(0,2)$.

Calcoliamo le derivate parziali seconde di $f$:
\[
\frac{\partial^2 f}{\partial x^2} = 6x, \quad \frac{\partial^2 f}{\partial y^2} = -2x, \quad \frac{\partial^2 f}{\partial x \partial y} = -2y + 2, \quad \frac{\partial^2 f}{\partial y \partial x} = -2y + 2
\]

Valutiamo le derivate parziali seconde nei punti critici.

Per il punto $(0,0)$:
\[
\frac{\partial^2 f}{\partial x^2}(0,0) = 0, \quad \frac{\partial^2 f}{\partial y^2}(0,0) = 0, \quad \frac{\partial^2 f}{\partial x \partial y}(0,0) = 2, \quad \frac{\partial^2 f}{\partial y \partial x}(0,0) = 2
\]

La matrice Hessiana in $(0,0)$ è:
\[
Hf(0,0) = \begin{pmatrix}
0 & 2 \\
2 & 0
\end{pmatrix}
\]

Il determinante della matrice Hessiana è:
\[
\det(Hf(0,0)) = (0)(0) - (2)(2) = -4
\]

Poiché il determinante è negativo, la matrice Hessiana è indefinita. Quindi, il punto $(0,0)$ è un punto di sella.

Per il punto $(0,2)$:
\[
\frac{\partial^2 f}{\partial x^2}(0,2) = 0, \quad \frac{\partial^2 f}{\partial y^2}(0,2) = 0, \quad \frac{\partial^2 f}{\partial x \partial y}(0,2) = -2, \quad \frac{\partial^2 f}{\partial y \partial x}(0,2) = -2
\]

La matrice Hessiana in $(0,2)$ è:
\[
Hf(0,2) = \begin{pmatrix}
0 & -2 \\
-2 & 0
\end{pmatrix}
\]

Il determinante della matrice Hessiana è:
\[
\det(Hf(0,2)) = (0)(0) - (-2)(-2) = -4
\]

Poiché il determinante è negativo, la matrice Hessiana è indefinita. Quindi, il punto $(0,2)$ è un punto di sella.

In conclusione, i punti critici della funzione $f(x,y) = x^3 - xy^2 + 2xy$ sono entrambi punti di sella.

\section{Massimi e minimi vincolati}
In questa sezione ci occuperemo di trovare i massimi e i minimi di una funzione $f$ di più variabili vincolata da una o più equazioni.\\
Esempio: vogliamo trovare i punti critici per una certa $f$ vincolati a stare su una circonferenza di equazione $x^2 + y^2 = 16$.

\begin{osservazione}{}
  Trovare i punti critici per $f$ qui non è significativo, perché potrebbero non essere sulla circonferenza.
\end{osservazione}

\begin{teorema}{Moltiplicatori di Lagrange}
  Sia $f$ una funzione $C^1$ su $V\in \R^n$, $V={(x_1, \dots, x_n)\in \R^n | g(x_1, \dots, x_n) = 0}$, con $g$ una funzione $C^1$ su $V$ e tale che $\nabla g(p) \neq (0, 0, \dots, 0)$ per ogni $p \in V$. Sia $P_0$ un punto critico di $f$ ristretta a $V$. Allora esiste un $\lambda \in \R$ tale che $\nabla f(P_0) = \lambda \nabla g(P_0)$.
\end{teorema}



\chapter{Calcolo integrale}
	\section{Integrali doppi}
	\deffname{integrale doppio}
	Sia $f:E \inc \R^2 \rightarrow R$ con $E$ misurabile, poniamo:
	$$\int_{\genfrac{}{}{0pt}{}{E}{*}} f(x,y)  dxdy=sup\{\in ts(x,y)dxdy: s\le f, s \text{ a scalino}\} \quad \text{Integrale inferiore di }f$$
	$$\int_E^* f(x,y)  dxdy=sup\{\int s(x,y)dxdy: f\le s, s \text{ a scalino}\} \quad \text{Integrale superiore di }f$$
	$f$ si dice \textbf{integrabile} secondo Riemann se:
	$$\int_{\genfrac{}{}{0pt}{}{E}{*}} f(x,y)  dxdy = \int_E^* f(x,y)  dxdy$$
	e $\int_E f(x,y)  dxdy$ è detto \textbf{integrale} di Riemann di $f$ su $E$.
	\thhname{Riduzione di Fubini}
	Se $f:[a,b]\times[c,d]\rightarrow \R$ è integrabile, allora:
	$$\int_E f(x,y)  dxdy = \int_{c}^{d} \left(\int_{a}^{b} f(x,y)dx\right) dy$$
	se $\int_a^b f(x,y)dx$ esiste finito, oppure:
	$$\int_E f(x,y)  dxdy = \int_{a}^{b} \left(\int_{c}^{d} f(x,y)dx\right) dy$$
	se $\int_c^d f(x,y)dx$ esiste finito.
	\deffname{Verticalmente convesso}
	$E\inc \R^2$ si dice \textbf{normale}, o semplice, rispetto all'asse $y$, oppure \textbf{verticalmente convesso} se è nella forma:
	$$E=\{(x,y)\in\R^2:a\le x \le b, h(x) \le y \le g(x)\}$$
	\prop
	Sia $E\{(x,y)\in\R^2:a\le x \le b, h(x) \le y \le g(x)\}$, allora:
	$$\int_E f(x,y) dxdy=\int_a^b\left(\int_{h(x)}^{g(x)}f(x,y)dy\right)dx$$
	segue da Fubini.
	\deffname{Orizzontalmente convesso}
	$E\inc \R^2$ si dice \textbf{normale}, o semplice, rispetto all'asse $x$, oppure \textbf{orizzontalmente convesso} se è nella forma:
	$$E=\{(x,y)\in\R^2:a\le y \le b, h(x) \le x \le g(x)\}$$
	\prop
	Sia $E\{(x,y)\in\R^2:c\le y \le d, \tilde h(x) \le x \le \tilde g(x)\}$, allora:
	$$\int_E f(x,y) dxdy=\int_c^d\left(\int_{\tilde h(x)}^{\tilde g(x)}f(x,y)dx\right)dy$$
	segue da Fubini.
	\section{Integrali tripli}
	\deffname{Integrale triplo}
	Analogamente a $\R^2$, in $\R^3$, con $E \inc \R^3$ misurabile, si definisce:
	$$\int_E f(x,y,z)  dxdydz$$
	\propname{Integrazione per fili}
	Se $E=\{(x,y,z)\in\R^3:(x,y)\in D \inc \R^2, h(x,y) \le z \le g(x,y)\}$, allora:
	$$\int_E f(x,y,z) dxdydz=\int_D\left(\int_{h(x)}^{g(x)}f(x,y,z)dz\right)dxdy$$
	\propname{Integrazione per strati}
	Se $E=\{(x,y,z)\in\R^3:a\le z\le b(x,y)\in A_z\}$, allora:
	$$\int_E f(x,y,z) dxdydz=\int_a^b\left(\int_{A_z}f(x,y,z)dxdy\right)dz$$
	\thhname{Th. di Pappo-Guldino}
	Sia $S$ contenuto nel piano $xz$ e sia $E$ il ruotato di $S$ di un angolo $\theta_0$ attorno all'asse $z$, cioè:
	$$E=\{(x,y,z)\in\R^3:x=\rho \cos \theta, y=\rho \sin \theta, (\rho,z)\in S \text{ e } 0 \le \theta \le \theta_0\}$$
	allora:
	$$Vol(E)=\int_e dxdydx=\int_0^{\theta_0}\int_s \rho d\rho dz d\theta=Area(s)(X_g \theta_0)$$
	\section{Cambiamento di variabile}
	\thhname{Cambiamento di variabile negli integrali doppi}
	Sia $\Phi:E \inc \R^2 \rightarrow F \inc \R^2$ con $E$, $F$ aperti misurabili, supponendo $\Phi$ invertibile e sia $\Phi$ che $\Phi^{-1}$ di classe $C^1$, se $\Phi(u,v)=\left(\Phi_1(u,v),\Phi_2(u,v)\right)\in F \ \forall (u,v)\in E$, allora:
	$$\int_E f(x,y) dxdy=\int_{F=\Phi^{-1}(E)}f\left(\Phi_1(u,v),\Phi_2(u,v)\right)|detJ\Phi(u,v)|dudv$$
	\propname{Coordinate polari}
	$$\int_D f(x,y)dxdy \Rightarrow \begin{cases} x = \rho \cos \theta \\ y = \rho \sin \theta \\ detJ\Phi(\rho,\theta)=\rho \end{cases} \Rightarrow \int_{\theta_1}^{\theta_2}\int_{r_1}^{r_2}f(\rho \cos\theta, \rho \sin \theta)\ detJ\Phi\  d\rho d\theta$$
	\propname{Coordinate polari traslate}
	$$\int_D f(x,y)dxdy \Rightarrow \begin{cases} x = x_0 + \rho \cos \theta \\ y =y_0 + \rho \sin \theta \\ detJ\Phi(\rho,\theta)=\rho \end{cases} \Rightarrow \int_{\theta_1}^{\theta_2}\int_{r_1}^{r_2}f(\rho \cos\theta, \rho \sin \theta)\ detJ\Phi\  d\rho d\theta$$
	\propname{Coordinate polari asimmetriche}
	$$\int_D f(x,y)dxdy \Rightarrow \begin{cases} x = a \rho \cos \theta \\ y = b \rho \sin \theta \\ detJ\Phi(\rho,\theta)=ab\rho \end{cases} \Rightarrow \int_{\theta_1}^{\theta_2}\int_{r_1}^{r_2}f(\rho \cos\theta, \rho \sin \theta)\ detJ\Phi\  d\rho d\theta$$
	\propname{Coordinate polari asimmetriche variate}
	$$\int_D f(x,y)dxdy \Rightarrow \begin{cases} x = x_0 + a \rho \cos \theta \\ y = y_0 + b \rho \sin \theta \\ detJ\Phi(\rho,\theta)=ab\rho \end{cases} \Rightarrow \int_{\theta_1}^{\theta_2}\int_{r_1}^{r_2}f(\rho \cos\theta, \rho \sin \theta)\ detJ\Phi\  d\rho d\theta$$
	\propname{Coordinate cilindriche}
	$$\int_D f(x,y,z)dxdydz \Rightarrow \begin{cases} x = \rho \cos \theta \\ y = \rho \sin \theta \\z=z\\ detJ\Phi(\rho,\theta)=\rho \end{cases} \Rightarrow \int_{\theta_1}^{\theta_2}\int_{r_1}^{r_2}\int_{z_1}^{z_2}f(\rho \cos\theta, \rho \sin \theta, z)\ detJ\Phi\  dzd\rho d\theta$$
	\propname{Coordinate sferiche}
	$$\int_D f(x,y,z)dxdydz \Rightarrow \begin{cases} x = \rho \cos \theta \sin \varphi \\ y = \rho \sin \theta \sin \varphi \\z=\rho \cos \varphi\\ detJ\Phi(\rho,\theta)=\rho^2 \sin \varphi \end{cases}$$
	$$\Rightarrow \int_{\varphi_1}^{\varphi_2}\int_{\theta_1}^{\theta_2}\int_{r_1}^{r_2}f(\rho \cos \theta \sin \varphi,\rho \sin \theta \sin \varphi, \rho \cos \varphi)\ detJ\Phi\  d\rho d\theta d\varphi$$

\chapter{Integrali curvilinei e di superficie}
	\section{Definizione delle curve}
	\deff
	Una funzione $\gamma : [ a, b ] \subseteq \R \Rightarrow \R^m$ si dice \textbf{curva} in $\R^m$ se è continua.\\
	\\
	L'insieme $\gamma ([a,b])= \{\gamma(t)\in \R^m : t \in [a,b] \}$ si dice sostegno di $\gamma$, cioè il sostegno è l'immagine di $\gamma$
	\deff
	$\gamma:[a,b]\Rightarrow \R^m$, curva, si dice \textbf{semplice} se vale l'implicazione:
	$$t \in [a, b],\ s \in (a, b)\ \text{e} \ s\ne t \Rightarrow \gamma(t)\ne \gamma(s)$$
	$\gamma:[a,b]\Rightarrow \R^m$, curva, si dice \textbf{chiusa} se vale l'implicazione:
	$$\gamma(a)=\gamma(b)$$
	Se $\gamma:[a,b]\Rightarrow \R^m$ è una curva chiusa e semplice si dice \textbf{curva di Jordan}
	\deff
	Sia $\gamma : [a, b] \Rightarrow \R^m$ curva con $\gamma(t) = (\gamma_1(t),\dots, \gamma_m(t))$, $\gamma$ si dice \textbf{regolare} se $\gamma \in C^1([a, b])$ e se $\gamma'(t)\ne \vec{0}, \ \forall t \in [a, b]$.\\
	\\
	$\gamma'(t)$ si dice \textbf{vettore tangente} a $\gamma$ in $\gamma(t)$, oppure vettore derivata o vettore velocità.\\
	\\
	Se $\gamma'(t)\ne \vec 0 \  \forall t$ significa che $\gamma(t)$ non si fermerà mai.
	\deff
	$\gamma:[a,b]\rightarrow \R^m$, curva, si dice \textbf{regolare a tratti} se $\exists a=t_0<t_1<\dots<t_{n-1}<t_{n}=b$ tale che $\gamma \in C^1([t_{k-1}, t_k]) \ \forall k \in {1, \dots, n}$ e $\gamma'(t)\ne \vec 0 \ \forall t \ne t_k$
	\section{Integrali curvilinei di prima specie (integrali di funzione su curve)}
	\deff
	Sia $\gamma:[a, b] \rightarrow \R^m$ curva regolare a tratti, sia $f:D\subseteq \R^m \rightarrow \R$ tale che $\gamma([a,b])\subseteq D$, si pone:
	$$\int_\gamma f ds = \int_\gamma \gamma f = \int_a^b f(\gamma(t))|\gamma'(t)|dt$$
	se l'integrale esiste.\\
	\\
	Se $\gamma$ è di Jordan si scrive $\oint_\gamma f ds = \int_\gamma fds$.\\
	\\
	Si pone $L(\gamma)=\int_\gamma ds=\int_{a}^{b}|\gamma'(t)|dt$ lunghezza di $\gamma$.
	\prop
	Sia $\gamma:[a, b] \rightarrow \R^m$ curva regolare a tratti e sia $\phi : [c, d] \rightarrow [a, b] \ C^1$ strettamente monotone e suriettiva e sia $\tilde{\gamma} : [c, d]\rightarrow \R^m$ definita come $\tilde{\gamma}=\gamma ( \phi (r) )$, ovvero  $\phi$ è un riscalamento del parametro t e $\tilde{\gamma}$ ha la stessa traiettoria di $\gamma$ ma la percorre in modo diverso, allora:
	$$\int_{\tilde{\gamma}} fds=\int_\gamma fds$$
	Quindi gli integrali di prima specie non dipendo né dalla parametrizzazione $\gamma$, né dal verso di percorrenza, ma dipendono solo dal sostegno $\Gamma=\gamma([a,b])$
	\deff
	Sia $\Gamma \subseteq \R^m$ che si sostegno si una curva $\gammadef$ regolare a tratti semplice, allora si pone:
	$$\int_\Gamma fds= \int_\gamma fds$$
	Allo stesso modo $L(\Gamma)=L(\gamma)$
	\deff
	Sia $\gamma :  [a,b]\rightarrow \R ^3$ curva regolare a tratti semplice, alla quale è associata una \textbf{funzione di densità} (di massa) $\mu:  \Gamma =\gamma([a,b])\subseteq \R^3 \rightarrow \R$, si dice \textbf{massa} di $\gamma$ (o di $\Gamma$) il numero:
	$$M(\Gamma)=m(\Gamma)=\int_\Gamma \mu \ d\sigma$$
	Il \textbf{baricentro} di $\Gamma$ è il punto $(x_G, y_G, z_G)$ tale che:
	$$x_G=\frac{1}{M(\Gamma)}\int_\Gamma x \mu (x,y,z) d \sigma$$
	$$y_G=\frac{1}{M(\Gamma)}\int_\Gamma y \mu (x,y,z) d \sigma$$
	$$z_G=\frac{1}{M(\Gamma)}\int_\Gamma z \mu (x,y,z) d \sigma$$
	\section{Integrali curvilinei di seconda specie (lavori)}
	Sia $\gammadef$ curva regolare a tratti e  sia $F:D\inc \R^m \rightarrow \R^m$ con $\gamma([a,b])\inc D$, si dice \textbf{lavoro di} $\vec F$ \textbf{lungo} $\gamma$ (o integrale di seconda specie) il numero:
	$$\int_\gamma F \cdot dl = \int_a^bF(\gamma(t))\cdot \gamma'(t) dt$$
	Se $\gamma$ è di Jordan si scrive $\oint_\gamma F \cdot dl = \int_\gamma F \cdot dl$ detto \textbf{circuitazione di $F$ lungo $\gamma$}.\\
	\\
	Sia $\vec r(t)= \frac{\gamma'(t)}{|\gamma'(t)|}\ \forall t \in [a,b]$ detto \textbf{versore tangente a} $\gamma$, si ha:
	$$\int_\gamma F \cdot dl = \int_\gamma (F\cdot \vec r) d\sigma$$
	\prop
	Sia $\gamma:[a, b] \rightarrow \R^m$ curva regolare a tratti e sia $\phi : [c, d] \rightarrow [a, b] \ C^1$ strettamente crescente e suriettiva e sia $\tilde{\gamma} : [c, d]\rightarrow \R^m$ definita come $\tilde{\gamma}=\gamma ( \phi (r) )$, ovvero $\tilde{\gamma}$ percorre il sostegno di $\gamma$ lo stesso numero di volte nello stesso verso, allora:
	$$\int_{\tilde{\gamma}} F \cdot dl = \int_\gamma F \cdot dl$$
	Nel caso in cui $\hat \gamma$, definita come $\tilde{\gamma}$, sia strettamente decrescente, e quindi percorra il sostegno di $\gamma$ lo stesso numero di volte ma in verso opposto, allora:
	$$\int_{\hat{\gamma}} F \cdot dl = - \int_\gamma F \cdot dl$$
	\section{Integrali di superficie}
	\deffname{Superfici parametrizzate}
	Una funzione $\sigma:\bar{A} \inc \R^2 \rightarrow \R^3$ con $A \inc \R^2$ aperto e con $\sigma(u,v)=(\sigma_1(u,v),\sigma_2(u,v),\sigma_3(u,v))$ si dice \textbf{superficie parametrizzata} se $\sigma \in C^1$, $\sigma$ è iniettiva in $A$ e la matrice Jacobiana ha rango 2, allora:
	$$\Sigma=\sigma(\bar{A})$$
	ed è detto \textbf{sostegno} di $\sigma$.
	\deff
	Le derivate parziali di $\sigma$ in $u$ e $v$ generano in piano in $\R^3$ ,detto \textbf{piano tangente} a $\Sigma$ in $\sigma(u_0,v0)$, che ha equazione:
	$$\Pi(u,v)=\sigma(u_0,v_0)+\frac{\partial \sigma}{\partial u}(u_0,v_0)(u-u_0)+\frac{\partial \sigma}{\partial v}(u_0,v_0)(v-v_0)$$
	Definiamo il \textbf{vettore normale} a $\Sigma$ in $\sigma(u_0,v0)$:
	$$\vec N(u_0, v_0)=\frac{\partial \sigma}{\partial u} \times \frac{\partial \sigma}{\partial v}$$
	da cui possiamo ricavare il versore $\vec n (u_0,v_0) = \frac{\vec N (u_0,v_0)}{|\vec N (u_0,v_0)|}$.
	\deffname{Superfici cartesiane}
	Sia $g:\bar A \inc \R^2 \rightarrow \R$, con $A$ aperto e $g \in C^1(A)$, data una superficie $\sigma(u,v)=(u,v,g(u,v))$ si ha che $\Sigma = \sigma (\bar A)$ è il grafico di $g$. Sappiamo quindi che:
	$$\Sigma  = \sigma(\bar A)=Gr(g)=\{(x,y,z)\in \R^3:(x,y)\in \bar A , z = g(x,y)\}$$
	$$\vec N(x, y)=\frac{\partial \sigma}{\partial u} \times \frac{\partial \sigma}{\partial v}=\left(-\frac{\partial g}{\partial x}, -\frac{\partial g}{\partial y}, 1 \right)$$
	$$|\vec N (x,y)|=\sqrt{1+|\nabla g|^2}$$
	\deffname{Integrali di superficie di prima specie}
	Sia $\sigma:\bar{A} \inc \R^2 \rightarrow \R^3$ superficie con $A$ misurabile, poniamo:
	$$\int_\sigma f d\sigma = \int_\sigma f(x,y,z)d\sigma =\int_{\bar A} f(x,y,g(x,y))|\vec N (x,y)|dxdy$$
	\prop
	L'integrale è indipendente dalla parametrizzazione, infatti se due superfici sono diverse ma hanno stesso sostegno l'integrale non cambia.
	\section{Flusso di un campo vettoriale}
	\deffname{Superfici orientabili}
	Una superficie $\sigma:\bar{A} \inc \R^2 \rightarrow \R^3$ si dice \textbf{orientabile} se la funzione $\Sigma \rightarrow \R^3$ è continua.
	\oss
	Se due superfici orientabili con lo stesso sostegno allora i versori normali possono essere solo uguali od opposti tra loro.
	\prop
	Se $\Sigma = \sigma (\bar A)$ è il sostegno di $\sigma:\bar{A} \inc \R^2 \rightarrow \R^3$ e se anche $\Sigma=\partial \Omega$ con $\Omega$ aperto, connesso e limitato, allora $\Sigma$ è orientabile. \\
	\\
	In particola re $\vec n$ punta verso l'interno di $\Omega$ allora viene detto \textbf{entrante}, altrimenti, se punta verso l'esterno, viene detto \textbf{uscente}. Il verso positivo del vettore è quello uscente dalla superficie.
	\deffname{Flusso di un campo vettoriale}
	Sia $\sigma:\bar{A} \inc \R^2 \rightarrow \R^3$ superficie orientabile e sia $F:D\inc \R^3 \rightarrow \R^3$ campo vettoriale continuo con $\Sigma = \sigma (\bar A)\inc D$, allora il \textbf{flusso} di $\vec F$ attraverso $\sigma$ nella direzione $\vec n$ è:
	$$\int_\sigma (\vec F \cdot \vec n) d\sigma = \int_{\bar A}F(\sigma(u,v))\cdot \frac{\vec N(u,v)}{|\vec N(u,v)|}|\vec N(u,v)|dudv=\int_D F(x,y,g(x,y))\cdot \vec N(x,y)dxdy$$
	\deff
	Sia $\Sigma$ una superficie orientabile, se $\Sigma=\partial \Omega$ con $\Omega$ aperto, connesso, limitato e misurabile allora:
	$$\int_{\partial \Omega} \vec F \cdot \\vec n$$
	per convenzione denota il \textbf{flusso uscente} da $\Omega$ , cioè il flusso di $F$ lungo $\vec n$ uscente.
	\thhname{Th. della divergenza di Gauss}
	Sia $\Omega \inc \R^3$ connesso, limitato e misurabile, tale che sia $\Sigma=\partial \Omega$ una superficie orientata con $\vec n$ uscente, sia $\vec F:D \inc \R^3 \rightarrow \R^3$, allora:
	$$\int_{\partial \Omega}(\vec F \cdot \vec n)d\sigma=\int_\Omega div \vec F dxdydz$$
	\deffname{Aperto con bordo}
	Sia $D \inc \R^2$ aperto, connesso e limitato. $D$ si dice \textbf{aperto con bordo} se $\partial D$ è l'unione di un numero finito di sostegni di curve di Jordan regolari a tratti a due a due disgiunti. Su $\partial D$ si definisce come orientazione positiva quella per cui percorrendo $\partial D$ vedo $D$ a sinistra.
	\thhname{Formula di Green nel piano}
	Sia $\vec F:E\ inc \R^2 \rightarrow \R^2$ di classe $C^1$ con $E$ aperto, sia $D\inc \R^2$ aperto con bordo con $\bar D \inc E$, allora:
	$$\int_{\partial D} \vec F \cdot dl = \int_D \left(\frac{\partial F_2}{\partial x}-\frac{\partial F_1}{\partial y}\right)dxdy$$
	\thhname{Th. del rotore di Stokes}
	Sia $D \inc \R^2$ aperto con bordo e sia $\sigma : \bar D \inc \R^2 \rightarrow \R^3$ superficie orientabile iniettiva si $\bar D$, chiamiamo $\partial \sigma = \sigma (\partial D)$, l'immagine di $\partial D$ tramite $\sigma$, \textbf{frontiera della superficie} $\sigma$. Orientiamo $\partial \sigma$ con l'orientazione indotta dall'orientazione positiva di $\partial D$, ovvero quando il versore normale $\vec n$ percorre $\partial \sigma$ vede $\sigma$ a sinistra. Allora:
	$$\int_{\partial\sigma} \vec F \cdot dl = \int_\sigma \left(rot \vec F \cdot \vec n\right)d\sigma$$
	ovvero il lavoro di $F$ lungo $\partial \sigma$ è uguale al flusso del rotore di $\vec F$ attraverso $\sigma$.
	\section{Campi conservativi}
	\deffname{Campi conservativi}
	Un campo $\vec F: \Omega \inc \R^n \rightarrow \R^n$ continuo con $\Omega$ aperto si dice \textbf{conservativo} se $\exists \ \Phi : \Omega \inc \R^n \rightarrow \R$ tale che $\nabla \Phi = \vec F$, in tal caso $\Phi$ si dice \textbf{potenziale} di $\vec F$, oppure primitiva di $\vec F$.
	\prop
	Sia $\gamma:[a,b]\rightarrow \Omega \inc \R^n$ curva regolare a tratti, se $F$ è \textbf{conservativo} si ha che il lavoro lungo $\gamma$ è:
	$$\int_\gamma \vec F \cdot dl = \int_a^b \vec F(\gamma (t))\cdot \gamma'(t)dt=\int_{a}^{b} \nabla \Phi (\gamma(t))\cdot \gamma'(t)dt=\Phi(\gamma(b))-\Phi(\gamma(a))$$
	ovvero se $F$ è conservativo ($F\nabla \Phi$) il lavoro lungo una curva $\gamma$ dipende sola dal punto iniziale e da quello finale.
	\thh
	Sia $\vec F: \Omega \inc \R^n \rightarrow \R^n$ conservativo ($F\nabla \Phi$), allora per ogni curva $\gamma:[a,b]\rightarrow \Omega \inc \R^3$ si ha:
	$$\int_\gamma \vec F \cdot dl =\Phi(\gamma(b))-\Phi(\gamma(a))$$
	Se $\gamma_1$ e $\gamma_2$ sono due curve con stesso punto iniziale e finale:
	$$\int_{\gamma_2} \vec F \cdot dl = \int_{\gamma_1} \vec F \cdot dl$$
	Se $gamma$ è chiusa allora:
	$$\int_\gamma \vec F \cdot dl =0$$
	\thh
	Sia $\vec F: \Omega \inc \R^n \rightarrow \R^n$ continuo con $\Omega$ connesso, allora le tre affermazioni sono equivalenti:
	$$\begin{Bmatrix}
		F \text{ è conservativo}\\
		  \Updownarrow\\
		 \int_{\gamma_2} \vec F \cdot dl = \int_{\gamma_1} \vec F \cdot dl\\
		 (\text{con } \gamma_1,\gamma_2 \text{ curve con stesso inizio e fine})\\
		 \Updownarrow\\
		 \int_\gamma \vec F \cdot dl =0\\
		 (\text{con } \gamma \text{ chiusa})
	\end{Bmatrix}$$
	\thh
	Sia $F: \Omega \subseteq \mathbb{R}^n \rightarrow \mathbb{R}^n$ di classe $C^1$ ($\Sigma$ aperto), se $F$ è conservativo allora:
	$$\frac{\partial F_i}{\partial x_j} = \frac{\partial F_j}{\partial x_i} \qquad \forall i, j \in \{ 1,\dots, n\} \text{ in } \Omega$$
	\deff
	Se $rotF=0$, $F$ si dice \textbf{irrotazionale}.
	\prop
	Se $F$ è conservativo e di classe $C^1$ $\Rightarrow$ $F$ è irrotazionale.\\
	\\
	In generale la $\Rightarrow$ non si può invertire.
	\deff
	Un aperto connesso $\Omega$ si dice \textbf{semplicemente connesso} se ogni curva $\gamma$ di Jordan può essere deformata con continuità fino a contrarsi ad un punto, rimanendo sempre dentro $\Omega$.\\
	\\
	In $\mathbb{R}^2$ un insieme semplicemente connesso è un insieme "privo di buchi". Sono esempi di insiemi semplicemente connessi: tutti gli aperti convessi, tutti gli aperti limitati con frontiera costituita da un'unica curva e il piano privato di una semiretta. Sono esempi di insiemi non semplicemente connessi: tutti gli aperti privati di un punto, le corone circolari e il piano privato di una retta (perché non è connesso).\\
	\\
	In $\mathbb{R}^3$ sono esempi di insiemi semplicemente connessi: tutti gli aperti convessi, lo spazio privato di un punto e le corone sferiche. Sono esempi di insiemi non semplicemente connessi: lo spazio privato di una retta o di un piano.
	\thh
	Sia $\Omega \subseteq \mathbb{R}^3$ un aperto semplicemente connesso, e sia $F$ un campo
	vettoriale di classe $C^1 $, se $rotF=0$ (irrotazionale) allora $F$ è conservativo.

\chapter{Serie numeriche}
\section{Definizione delle serie numeriche}
\deffname{Successioni in $\C$}
Si dice che che una \textbf{successione} $z_n$ in $\C$ \textbf{converge} a $z \in \C$ se:
$$\forall \epsilon >0\  \exists n_\epsilon \in \mathbb{N} \text{ tale che se } n>n_\epsilon \text{ allora } |z_n-z|<\epsilon$$
e si scrive come $\lim\limits_{n\rightarrow \infty}z_n = z$ oppure $z_n\rightarrow z \text{ per } n \rightarrow \infty$.
\deffname{Serie numerica}
Sia $a_n$ successione in $\C$, si dice \textbf{serie di termine generale} $a_n$ il simbolo:
$$\suminfan$$
Si dice \textbf{successione delle somme parziali} la successione:
$$\sum_{n=0}^{n}a_n=a_0+a_1+\dots+a_n$$
\deffname{Serie convergente}
Si dice che $\suminfan$ è \textbf{convergente} se:
$$\exists \lim\limits_{n \rightarrow \infty} s_n = s \in \C \ \ne \pm \infty$$
$s$ si dice \textbf{somma della serie} e vale:
$$\suminfan = s$$
In tal caso si pone $r_n=s-s_n$ \textbf{resto n-esimo}.
\deffname{Serie divergente}
Si dice che $\suminfan$, con $a_n \in \R$, è \textbf{divergente} a $\pm \infty$ se:
$$\exists \lim\limits_{n \rightarrow \infty} s_n =  \pm \infty$$
\deffname{Serie oscillante}
Nel caso in cui $a_n \in \R$ e in cui $s_n$ è oscillante, si dice che la serie $\suminfan$ è \textbf{oscillante}.
\prop
Se $\suminfan = s$ e $\suminf{0} b_n = t$ sono convergenti allora:
$$\suminf{0}(\alpha a_n + \beta b_n)=\alpha \suminfan + \beta \suminf{0} b_n=\alpha s + \beta t $$
\propname{Condizione necessaria di convergenza}
Se $\suminfan$ converge, allora:
$$\lim\limits_{n \rightarrow \infty} a_n = \lim\limits_{n \rightarrow \infty} |a_n| = 0$$
\deffname{Serie assolutamente convergenti}
La serie $\suminfan$ si dice \textbf{assolutamente convergente} se $\suminf{0}|a_n|$ è convergente.
\thh
Se $\suminf{0}|a_n|$ converge allora anche $\suminfan$ converge.
\deffname{Serie a segni alterni}
Si dice che una serie è \textbf{a segni alterni} se è nella forma $\suminf{0} (-1)^n b_n$.
\section{Metodi di risoluzione delle serie numeriche}
\propname{Serie geometrica di ragione $q$}
Data la serie $\suminf{0} q^n$ con $q \in \C$, allora se $|q|<1$ la serie converge, in particolare:
$$\suminf{0} q^n = \frac{1}{1-q} \quad \text{per } |q|<1$$
\oss
Sapendo che se due serie differiscono per un numero di termini finito hanno lo stesso carattere ma somme differenti, possiamo ricondurre serie del tipo $\suminf{k} q^n$ con $k>0$ a serie geometriche, per esempio:
$$\suminf{1} q^n = \suminf{0} q^n - q^0=\frac{1}{1-q}-1$$
\propname{Criterio integrale per le serie}
sia $f:[1, +\infty]\rightarrow \R$, $f(x)\ge 0 \ \forall x$ e $f$ decrescente, allora:
$$\suminf{1} f(n) \text{ converge } \Leftrightarrow \int_1^{+\infty} f(x)dx \text{ converge}$$
\propname{Serie armonica di esponete $p$}
Data la serie $\suminf{1} \frac{1}{n^p}$ con $p \in \R$ allora:
$$\suminf{1} \frac{1}{n^p} \text{ converge } \Leftrightarrow \ p>1$$
per il criterio integrale.
\thhname{Criterio del confronto}
Sia $0 \leq a_n \leq b_n$, allora:
$$\suminf{0} b_n \text{ converge } \Rightarrow \suminfan \text{ converge }$$
$$\suminfan \text{ diverge a } +\infty \Rightarrow \suminf{0} b_n \text{ diverge a } \infty$$
\thhname{Criterio del confronto asintotico}
Siano $a_n>0$ e $b_n > 0$ tale che $a_n \sim b_n$ per $n \rightarrow \infty$ ($\liminff \frac{a_n}{b_n}=1$), allora $\suminfan$ e $\suminf{0} b_n$ hanno lo stesso carattere.
\propname{Serie telescopiche}
Sia $\suminf{1} a_n = \suminf{1} (b_n-b_{n+1})$ con $b_n \rightarrow l$, allora:
$$\suminf{1} (b_n-b_{n+1})=b_1-l$$
\thhname{Criterio della radice}
Sia $a_n \in \C$ e $\exists \ \liminff \sqrt[n]{|a_n|}=l$, allora:
$$\suminfan \text{ è assolutamente convergente se } 0\le l< 1$$
$$\liminff a_n \ne 0 \text{ se } l>1 \Rightarrow \suminf{0}|a_n| \text{ non converge}$$
se $l=1$ non possiamo dire nulla a priori.
\thhname{Criterio del rapporto}
Sia $a_n \in \C$ e $\exists \ \liminff|\frac{a_{n+1}}{a_n}|=l$, allora:
$$\suminfan \text{ è assolutamente convergente se } 0\le l< 1$$
$$\liminff a_n \ne 0 \text{ se } l>1 \Rightarrow \suminf{0}|a_n| \text{ non converge}$$
se $l=1$ non possiamo dire nulla a priori.
\thhname{Criterio di Leibniz}
Se $\suminf{0} (-1)^n b_n$ è con $b_n \ge 0$, $\liminff b_n =0$ e $b_n$ decrescente, allora:
$$\suminf{0} (-1)^n b_n \text{ converge}$$
\section{Serie di potenze}
\deffname{Serie di potenze}
Sia $c_n \in \C$ e sia $z_0 \in \C$, si dice \textbf{serie di potenze} di centro $z_0$ e coefficiente $c_n$ la funzione:
$$\suminf{0} c_n (z-z_0)^n$$
\thh
Sia $\suminf{0} c_n (z-z_0)^n$ serie di potenze, allora $\exists \ \R \in [0, +\infty] $ tale che:
$$\suminf{0} c_n (z-z_0)^n  \text{ converge se } |z-z_0|<R \ (z \in B_R(z_0))$$
$$\suminf{0} c_n (z-z_0)^n  \text{ non converge se } |z-z_0|>R \ (z \in B_R(z_0))$$
Se $|z-z_0|=R$ non si può dire nulla a priori.\\
\\
$R$ si dice \textbf{raggio di convergenza}
\prop
Se almeno uno dei due limiti esiste:
$$\frac{1}{R}=\liminff \sqrt[n]{|c_n|}$$
$$\frac{1}{R}=\liminff |\frac{c_{n+1}}{c_n}|$$
\section{Serie di Fourier}
Vogliamo approssimare un segnale $2\pi$-periodico con un segnale sinusoidale del tipo:
$$S_n(x)=\sum_{k=0}^{n}a_k \cos{(kx)} + b_k \sin{(kx)}=a_0+\sum_{k=1}^{n}a_k \cos{(kx)} + b_k \sin{(kx)}$$
detto anche \textbf{polinomio trigonometrico}.
\deff
Data $f:\R \rightarrow \R$ $2\pi$-periodica, si pone:
$$\|f\|_2=\left( \int_0^{2\pi}|f(x)|^2 dx \right)^{\frac 12}$$
e viene detta \textbf{norma quadratica} di $f$ su un periodo.
\deff
Data $f,g:\R \rightarrow \R$ $2\pi$-periodica, si pone:
$$\|f-g\|_2=\left( \int_0^{2\pi}|f(x)-g(x)|^2 dx \right)^{\frac 12}$$
e viene detta \textbf{distanza quadratica} tra $f$ e $g$.
\deff
Definiamo i coefficienti di Fourier:
$$a_0=\frac{1}{2 \pi}\int_{0}^{2\pi}f(x)dx$$
$$a_n=\frac{1}{\pi}\int_{0}^{2\pi}f(x)\cos(nx)dx$$
$$b_n=\frac{1}{\pi}\int_{0}^{2\pi}f(x)\sin(nx)dx$$
\deffname{Serie di Fourier}
Sia $f:\R \rightarrow \R$ $2\pi$-periodica, integrabile su $[0,2\pi]$, allora:
$$s_f(x)=a_0+\suminf{1}\left(a_n \cos (nx)+ b_n \sin (nx)\right)$$
è detta \textbf{Serie di Fourier} di $f$.
\thh
Sia $f:\R \rightarrow \R$ $2\pi$-periodica, integrabile su $[0,2\pi]$, se:
$$\lim\limits_{n \rightarrow \infty} \|f(x)-s_n(x)\|_2=0$$
o equivalentemente:
$$\lim\limits_{n \rightarrow \infty} a_n=\lim\limits_{n \rightarrow \infty} b_n =0$$
allora $S_f(x)$ converge a$f(x)$ in norma quadratica.
\prop
$S_f(x)=f(x) \ \forall x$  dove $f$ è continua, ovvero $S_f(x)$ converge a $f(x)$ dove $f$ è continua.\\
\\
Se $f$ non è continua in $x$, allora $S_f(x)$ converge a:
$$S_f(x)=\frac{f(x^+)+f(x^-)}{2}$$
\oss
Sia $f:\R \rightarrow \R$ $2\pi$-periodica, integrabile su $[0,2\pi]$,  allora:
$$f \text{ pari } \Rightarrow b_n=0$$
$$f \text{ dispari } \Rightarrow a_0= a_n=0$$


\backmatter

\end{document}

