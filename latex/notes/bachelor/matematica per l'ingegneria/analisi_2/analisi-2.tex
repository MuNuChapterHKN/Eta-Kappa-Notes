%! TEX root = analisi-2.tex

\documentclass[italian,12pt,toc=sections]{HKNdocument}
% Packages
\usepackage{listings}                    % Code highlighting
\usepackage{xcolor}                      % Custom colors
\usepackage{longtable}                   % Breakable tables
\usepackage{ulem}                        % Underline
\usepackage{contour}                     % Border around text
\usepackage{tcolorbox}                   % Custom boxes

% Primary (Accent) Colors
% Primary (Accent) Colors
\definecolor{accentYellow}{RGB}{254, 196, 41}  % #FEC421
\definecolor{accentRed}{RGB}{236, 45, 36}      % #EC2D24

% Secondary Colors
\definecolor{supportOrange}{RGB}{242, 183, 5}  % #F2B705
\definecolor{supportDarkBlue}{RGB}{55, 81, 113} % #375171

% Background Colors
\definecolor{backgroundLight}{RGB}{242, 242, 242} % #F2F2F2

% Text & Border Colors
\definecolor{textGrayBlue}{RGB}{100, 117, 140}   % #64758C
\definecolor{textGrayMedium}{RGB}{146, 154, 166}  % #929AA6
\definecolor{textGrayLight}{RGB}{184, 187, 191}   % #B8BBBF



% Listings style
\lstdefinestyle{hkn}{
  basicstyle=\ttfamily\small\color{textGrayBlue},                         % Base style (size and font)
  keywordstyle=\bfseries\color{accentRed},           % Keywords in red (important, eye-catching)
  identifierstyle=\color{supportDarkBlue},               % Identifiers in blue (clear distinction)
  commentstyle=\color{textGrayMedium},                 % Comments in gray-blue (less prominent)
  stringstyle=\color{supportOrange},                 % Strings in orange (warm and readable)
  numberstyle=\ttfamily\scriptsize\color{textGrayMedium}, % Line numbers in gray (non-intrusive)
  backgroundcolor=\color{backgroundLight},           % Light background for contrast
  rulecolor=\color{textGrayLight},                   % Soft gray border for structure
  frame=single,                          % Border around code (single, double, shadowbox, none)
  framerule=0.8pt,                       % Border thickness
  frameround=tttt,                       % Round all corners
  framesep=5pt,                          % Distance between border and code
  rulesep=2pt,                           % Distance between border and code line
  numbers=left,                          % Line number position (left, right, none)
  stepnumber=1,                          % Line number interval
  numbersep=10pt,                        % Distance between line numbers and code
  xleftmargin=30pt,                      % Left margin
  xrightmargin=30pt,                     % Right margin
  resetmargins=true,                     % Reset margins
  numberblanklines=false,                % Number blank lines
  firstnumber=auto,                      % Initial line number
  columns=fixed,                         % Fixed column width
  showstringspaces=false,                % Show spaces in strings
  tabsize=2,                             % Tab size
  breaklines=true,                       % Automatic line break for long lines
  breakatwhitespace=true,                % Line break at whitespace
  breakautoindent=true,                  % Automatic indentation after line break
  escapeinside={(*@}{@*)}                % LaTeX commands in code
}

% Underline settings
\renewcommand{\ULdepth}{1.8pt} % Underline depth
\contourlength{0.8pt}

% Custom underline command
\newcommand{\myuline}[1]{%
\uline{\phantom{#1}}%
\llap{\contour{white}{#1}}%
}

% tcolorbox color settings
\definecolor{tcolorboxLeftColor}{RGB}{2, 65, 191}
\definecolor{tcolorboxBackTitleColor}{RGB}{119, 152, 255}
\definecolor{tcolorboxBackColor}{RGB}{210, 226, 255}

% Custom boxes
\newtcolorbox[auto counter, number within=chapter]{definition}[1]{
  title={\iflanguage{italian}{Definizione}{Definition}\par~\arabic{\tcbcounter}.~#1},
  boxrule=0mm,                       % Bordo principale (disabilitato)
  leftrule=1mm,                    % Bordo sinistro principale
  arc=2mm,
  colframe=accentRed,       % Colore bordo
  colbacktitle=textGrayMedium,
  colback=backgroundLight,        % Colore sfondo
  fonttitle=\bfseries,
  rounded corners=all,               % Bordi arrotondati
  }

\newtcolorbox[auto counter, number within=chapter]{theorem}[1]{
  title={\iflanguage{italian}{Teorema}{Theorem}~\arabic{\tcbcounter}.~#1},
  boxrule=0mm,                       % Bordo principale (disabilitato)
  leftrule=1mm,                    % Bordo sinistro principale
  arc=2mm,
  colframe=accentYellow,       % Colore bordo
  colbacktitle=textGrayMedium,
  colback=backgroundLight,        % Colore sfondo
  fonttitle=\bfseries,
  rounded corners=all,               % Bordi arrotondati
}

\newtcolorbox[auto counter, number within=chapter]{corollary}[1]{
  title={\iflanguage{italian}{Corollario}{Corollary}~\arabic{\tcbcounter}.~#1},
  boxrule=0mm,                       % Bordo principale (disabilitato)
  leftrule=1mm,                    % Bordo sinistro principale
  arc=2mm,
  colframe=supportOrange,       % Colore bordo
  colbacktitle=textGrayMedium,
  colback=backgroundLight,        % Colore sfondo
  fonttitle=\bfseries,
  rounded corners=all,               % Bordi arrotondati
}

\newtcolorbox[auto counter, number within=chapter]{exercise}[1]{
  title={\iflanguage{italian}{Esercizio}{Exercise}~\arabic{\tcbcounter}.~#1},
  boxrule=0mm,                       % Bordo principale (disabilitato)
  leftrule=1mm,                    % Bordo sinistro principale
  arc=2mm,
  colframe=supportDarkBlue,       % Colore bordo
  colbacktitle=textGrayMedium,
  colback=backgroundLight,        % Colore sfondo
  fonttitle=\bfseries,
  rounded corners=all,               % Bordi arrotondati
}

\newtcolorbox[auto counter, number within=chapter]{observation}[1]{
  title={\iflanguage{italian}{Osservazione}{Observation}~\arabic{\tcbcounter}.~#1},
  boxrule=0mm,                       % Bordo principale (disabilitato)
  leftrule=1mm,                    % Bordo sinistro principale
  arc=2mm,
  colframe=textGrayBlue,       % Colore bordo
  colbacktitle=textGrayMedium,
  colback=backgroundLight,        % Colore sfondo
  fonttitle=\bfseries,
  rounded corners=all,               % Bordi arrotondati
  }


%adeguare formattazione

\newcommand{\R}{\mathbb{R}}
\newcommand{\inc}{\subseteq}

\usepackage{verbatim}
\usepackage{hyperref}

%Ho creato questi comandi personalizzati perchè nel frattempo sono cambiati, questa
% sintassi permette di usarli in italiano senza cambiare tutte le occorrenze nel file
\newenvironment{definizione}{\begin{definition}}{\end{definition}}
\newenvironment{teorema}{\begin{theorem}}{\end{theorem}}
\newenvironment{osservazione}{\begin{observation}}{\end{observation}}
\newenvironment{esercizio}{\begin{exercise}}{\end{exercise}}
\newenvironment{corollario}{\begin{corollary}}{\end{corollary}}

\begin{document}

\title{Analisi 2}
\shorttitle{Analisi 2}
\date{\today}
\author{Tommaso Pignatelli}

\docdate{Primo semestre 2024/2025}
\docversion{1.0}

\frontmatter
% Create the title page
\maketitle
\cclicense
\tableofcontents
\clearpage

\mainmatter
\chapter{Introduzione}

Scriviamo qualcosa.





\chapter{Calcolo differenziale}

\section{Introduzione}
\begin{definizione}{Funzione di più variabili}
  Una \textbf{funzione di più variabili} è una applicazione $f:\Omega \subset \R^n \rightarrow \R$, con $n\geq 1$. $\Omega$ è detto dominio della funzione.
  \end{definizione}

Come si visualizza una funzione di più variabili?\\
Ci sono due strumenti principali: il grafico e le curve di livello.\\

\begin{definizione}{Curve di livello}
Data $f:\Omega \inc \R^2 \rightarrow \R$, si definisce \textbf{curva di livello} $\lambda$ di $f$ il luogo dei punti $L_\lambda=\{(x,y)\in \Omega : f(x,y)=\lambda\}$.\\
\end{definizione}
Può capitare che $L_\lambda$ sia vuoto, es: $L_{-1}$ per $f(x,y)= x^2+y^2$.\\
Tipicamente $L_\lambda$ è una curva nel piano o l'unione di più curve.\\


\begin{definizione}{Grafico}
Data $f:\Omega \inc \R^2 \rightarrow \R$, si definisce \textbf{grafico} di $f$ il luogo dei punti $G_f=\{(x,y,z)\in \R^3 : z=f(x,y), (x,y)\in \Omega\}$.\\
\end{definizione}
In generale $\Omega \inc \R^n$ e $f:\Omega \rightarrow \R$ si definisce $G(f)=\{(x_1, \dots, x_n, z)\in \R^{n+1} : z=f(x_1, \dots, x_n), (x_1, \dots, x_n)\in \Omega\}$.\\

\subsection{Esempio di grafici da sapere}
Sono esempi di grafici da sapere i seguenti:
\begin{itemize}
  \item $f(x,y)=x^2+y^2$ è il grafico di un paraboloide
  \item $f(x,y)=x^2-y^2$ è il grafico di un iperbolide
  \item $f(x,y)=x^2$ è il grafico di un cilindro
  \item $f(x,y)=\sqrt{1-x^2-y^2}$ è il grafico di una semisfera
  \item $f(x,y)=\sqrt{x^2+y^2}$ è il grafico di un cono
\end{itemize}

\section{Cenni di limiti e continuità}

\begin{definizione}{Intorno di un punto}
Sia $\vec x_0 \in \R^n$ e sia $r>0$, si dice \textbf{intorno} di $\vec x_0$ la palla aperta $B_r(\vec x_0)=\{\vec x \in \R^n : \|\vec x - \vec x_0\|<r\}$, dove $\|\cdot\|$ denota la norma euclidea.\\
\end{definizione}

\begin{definizione}{Limite}
Sia $f:\Omega \inc \R^n \rightarrow \R$ e sia $p_0$ un punto di accumulazione per $\Omega$. Si dice che $f$ tende a $L \in \R$ per $p$ che tende a $p_0$ (cioè che il \textbf{limite} per $p$ che tende a $p_0$ di $f(p)$ è $L$), e si scrive
$$\lim_{p \to p_0} f(p) = L$$
se per ogni $\epsilon > 0$ esiste $\delta > 0$ tale che, per ogni $p \in \Omega$, $0 < \|p - p_0\| < \delta$ implica $|f(p) - L| < \epsilon$.
\end{definizione}

\begin{definizione}{Continuità}
Sia $f:\Omega \inc \R^n \rightarrow \R$ e sia $p_0 \in \Omega$. Si dice che $f$ è \textbf{continua} in $p_0$ se
$$\lim_{p \to p_0} f(p) = f(p_0).$$
\end{definizione}

Data $f$ come faccio a capire in quali punti è continua?\\
Si usano i seguenti fatti:
\begin{itemize}
  \item Una funzione di una sola variabile è continua se non presenta discontinuità nel suo dominio.
  \item Se $f$ e $g$ sono funzioni continue, allora le seguenti funzioni sono continue:
  \begin{itemize}
    \item La somma: $f + g$
    \item Il prodotto: $f \cdot g$
    \item Il rapporto: $\frac{f}{g}$, purché $g \neq 0$
    \item La composizione: $f \circ g$
  \end{itemize}
\end{itemize}

\subsection{Esempio di limite lungo semirette}
Consideriamo la funzione $f(x,y)= \frac{2xy}{x^2+y^2}$. Vogliamo studiare il limite di $f$ quando $(x,y)$ tende a $(0,0)$ lungo diverse semirette.

Osserviamo prima il comportamento lungo gli assi $x$ e $y$:
\[
f(x,0) = \frac{2x \cdot 0}{x^2 + 0^2} = 0
\]
\[
f(0,y) = \frac{2 \cdot 0 \cdot y}{0^2 + y^2} = 0
\]
Quindi, lungo gli assi il limite è $0$.

Consideriamo ora il limite lungo la semiretta $y=mx$ con $m$ costante:
\[
f(x,mx) = \frac{2x(mx)}{x^2+(mx)^2} = \frac{2mx^2}{x^2(1+m^2)} = \frac{2m}{1+m^2}
\]
Quindi, il limite dipende da $m$ e vale $\frac{2m}{1+m^2}$.

Osserviamo che il limite di $f(x,y)$ quando $(x,y)$ tende a $(0,0)$ dipende dalla direzione lungo la quale ci avviciniamo al punto $(0,0)$. Pertanto, possiamo concludere che il limite non esiste in senso assoluto.

\subsection{Esempio di limite lungo parabole}
Consideriamo la funzione $f(x,y)= \frac{x^2y}{y^2+x^4}$. Vogliamo studiare il limite di $f$ quando $(x,y)$ tende a $(0,0)$ lungo diverse curve.

Lungo la semiretta $y=mx$ con $m$ costante:
\[
f(x,mx) = \frac{x^2(mx)}{(mx)^2+x^4} = \frac{mx^3}{m^2x^2+x^4} = \frac{mx^3}{x^2(m^2+x^2)} = \frac{mx}{m^2+x^2}
\]
Quando $x \to 0$, il limite di $f(x,mx)$ è $0$.

Consideriamo la parabola $y=mx^2$ con $m$ costante:
\[
f(x,mx^2) = \frac{x^2(mx^2)}{(mx^2)^2+x^4} = \frac{mx^4}{m^2x^4+x^4} = \frac{mx^4}{x^4(m^2+1)} = \frac{m}{m^2+1}
\]
Quando $x \to 0$, il limite di $f(x,mx^2)$ è $0$ solo se $m=0$. Per $m \neq 0$, il limite non è $0$.

Osserviamo che il limite di $f(x,y)$ quando $(x,y)$ tende a $(0,0)$ dipende dalla curva lungo la quale ci avviciniamo al punto $(0,0)$. Pertanto, possiamo concludere che il limite non esiste in senso assoluto.

\section{Derivate parziali}
\begin{definizione}{Derivata in una variabile}
Sia $f: \R \rightarrow \R$ una funzione e sia $x_0 \in \R$. La \textbf{derivata} di $f$ in $x_0$, indicata con $f'(x_0)$, è definita come il limite del rapporto incrementale, se esiste:
$$f'(x_0) = \lim_{h \to 0} \frac{f(x_0 + h) - f(x_0)}{h}.$$
Geometricamente, la derivata rappresenta il coefficiente angolare della retta tangente al grafico di $f$ nel punto $(x_0, f(x_0))$.
\end{definizione}

\begin{definizione}{Derivata parziale}
  Sia ora $P_0 \in \R^n $ e $f(x_1, \dots, x_n)$ una funzione definita in un intorno di $P_0$. La \textbf{derivata parziale} di $f$ rispetto alla variabile $x_i$ in $P_0$ è definita come il limite del rapporto incrementale, se esiste: $
\frac{\partial f}{\partial x_i}(P_0) = \lim_{h \to 0} \frac{f(P_0+e_i\cdot h) - f(P_0)}{h}$, dove $e_i$ indica il versore fondamentale lungo la componente $i$.
\end{definizione}
Questo equivale a derivare $f(x_1,\dots, x_n)$, rispetto alla variabile $x_i$, lasciando fissi i vettori delle altre $n-1$ componenti.

In particolare, per una funzione di due variabili $f(x,y)$, esistono due derivate parziali: la derivata parziale rispetto a $x$ e la derivata parziale rispetto a $y$. Queste sono indicate rispettivamente con $\frac{\partial f}{\partial x}$ e $\frac{\partial f}{\partial y}$. \\
Consideriamo la funzione $f(x,y) = x^2 + y^2$. Le derivate parziali di $f$ sono:
\[
\frac{\partial f}{\partial x} = \frac{\partial}{\partial x}(x^2 + y^2) = 2x
\]
\[
\frac{\partial f}{\partial y} = \frac{\partial}{\partial y}(x^2 + y^2) = 2y
\]
Quindi, la derivata parziale rispetto a $x$ è $2x$ e la derivata parziale rispetto a $y$ è $2y$.\\

\section{Piano tangente}
Il piano tangente è un concetto fondamentale nel calcolo differenziale e rappresenta il piano che "tocca" una superficie in un punto dato, approssimando la superficie stessa vicino a quel punto.

\begin{definizione}{Piano tangente}
Sia $f:\Omega \inc \R^2 \rightarrow \R$ differenziabile in $(x_0,y_0) \in \Omega$, si dice \textbf{piano tangente} al grafico di $f$ in $(x_0,y_0,f(x_0,y_0))$ il piano:
$$z=f(x_0,y_0)+\frac{\partial f}{\partial x}(x_0,y_0)(x-x_0)+\frac{\partial f}{\partial y}(x_0,y_0)(y-y_0)$$
\end{definizione}

In altre parole, il piano tangente al grafico di una funzione $f(x,y)$ in un punto $(x_0, y_0)$ è dato dall'equazione lineare che approssima $f$ vicino a quel punto. Le derivate parziali $\frac{\partial f}{\partial x}$ e $\frac{\partial f}{\partial y}$ rappresentano le pendenze del piano tangente nelle direzioni $x$ e $y$, rispettivamente.
\\
Consideriamo la funzione $f(x,y) = x^2 + y^2$. Troviamo il piano tangente al grafico di $f$ nel punto $(1,1)$.

Calcoliamo le derivate parziali:
\[
\frac{\partial f}{\partial x} = 2x \quad \text{e} \quad \frac{\partial f}{\partial y} = 2y
\]

Valutiamo le derivate parziali nel punto $(1,1)$:
\[
\frac{\partial f}{\partial x}(1,1) = 2 \quad \text{e} \quad \frac{\partial f}{\partial y}(1,1) = 2
\]

L'equazione del piano tangente è quindi:
\[
z = f(1,1) + \frac{\partial f}{\partial x}(1,1)(x-1) + \frac{\partial f}{\partial y}(1,1)(y-1)
\]
\[
z = 1^2 + 1^2 + 2(x-1) + 2(y-1)
\]
\[
z = 2 + 2(x-1) + 2(y-1)
\]
\[
z = 2x + 2y - 2
\]

Quindi, il piano tangente al grafico di $f$ nel punto $(1,1)$ è dato dall'equazione $z = 2x + 2y - 2$.\\


La differenza rispetto al caso in una variabile è che in due (o più) variabili può capitare che esistanto tutte le derivate parziali ma $f$ potrebbe non solo non avere piano tangente al grafico ma addirittura essere discontinua in $P_0$ (si veda ad esempio la funzione nulla sugli assi, e che vale $1$ altrove).

\section{Differenziabilità}
\begin{definizione}{Differenziabilità in due variabili}
Sia $f:\Omega \inc \R^2 \rightarrow \R$ e sia $P_0 = (x_0,y_0) \in \Omega$. Si dice che $f$ è \textbf{differenziabile} in $P_0$ se esistono le derivate parziali di $f$ in $P_0$ rispetto a $x$ e $y$ e se vale lo sviluppo di Taylor di ordine 1 in $P_0$ con resto di Peano, cioè: $f(x,y)=f(x_0,y_0)+\frac{\partial f}{\partial x}(x_0,y_0)(x-x_0)+\frac{\partial f}{\partial y}(x_0,y_0)(y-y_0)+o(\sqrt{(x-x_0)^2+(y-y_0)^2}) \quad \text{per }(x,y) \rightarrow (x_0,y_0)$.
\end{definizione}

\begin{osservazione}{}
Se $f$ è differenziabile in $(x_0,y_0)$, allora il piano di equazione $z=f(x_0,y_0)+\frac{\partial f}{\partial x}(x_0,y_0)(x-x_0)+\frac{\partial f}{\partial y}(x_0,y_0)(y-y_0)$ è il piano tangente al grafico di $f$ in $(x_0,y_0,f(x_0,y_0))$.
\end{osservazione}
\begin{osservazione}{}
  La differenziabilità esprime il fatto che f(x,y) è approssibile con una funzione lineare con un errore che è $o(\sqrt{(x-x_0)^2+(y-y_0)^2}) \quad \text{per }(x,y) \rightarrow (x_0,y_0)$.
\end{osservazione}

\begin{definizione}{Gradiente}
Sia $f:\Omega \inc \R^n \rightarrow \R$ e sia $P_0 \in \Omega$. Si dice \textbf{gradiente} di $f$ in $P_0$ il vettore: $\nabla f(P_0)=\left(\frac{\partial f}{\partial x_1}(P_0), \dots, \frac{\partial f}{\partial x_n}(P_0)\right)$.
\end{definizione}

\begin{definizione}{Differenziabilità in $n$ variabili}
Sia $f:\Omega \inc \R^n \rightarrow \R$ e sia $P_0 \in \Omega$. Si dice che $f$ è \textbf{differenziabile} in $P_0$ se esistono le derivate parziali di $f$ in $P_0$ rispetto a tutte le variabili e se vale lo sviluppo di Taylor di ordine 1 in $P_0$ con resto di Peano, cioè: $x_{n+1}=f(P_0)+\nabla f(P_0)\cdot (P - P_0)+o(|P-P_0|) \quad \text{per } P \rightarrow P_0$.
\end{definizione}

Per analogia il "piano" (iperpiano) tangente al grafico di $f$ in $P_0$ è dato dall'equazione $z=f(P_0)+\nabla f(P_0)\cdot (P - P_0)$. In ogni caso resta vera la proprietà di approssimazione lineare con errore $o(|P-P_0|) \quad \text{per } P \rightarrow P_0$.\\

Per garantire che $f$ sia effettivamente differenziabile è utile il seguente teorema:
\begin{teorema}{Criterio di differenziabilità}
Se esistono le derivate parziali di $f$ in $P_0$ e sono continue in un intorno di $P_0$, allora $f$ è differenziabile in $P_0$.
\end{teorema}

\subsection{Esempio di verifica della differenziabilità e calcolo del piano tangente}
Consideriamo la funzione $f(x,y) = \frac{e^{2x+y}}{x^2+y^2}$. Vogliamo verificare il teorema di differenziabilità e calcolare il piano tangente al grafico di $f$ nel punto $(1,1)$.

Calcoliamo le derivate parziali di $f$ rispetto a $x$ e $y$:
\[
\frac{\partial f}{\partial x} = \frac{(2e^{2x+y})(x^2+y^2) - e^{2x+y}(2x)}{(x^2+y^2)^2} = \frac{2e^{2x+y}(x^2+y^2) - 2xe^{2x+y}}{(x^2+y^2)^2}
\]
\[
\frac{\partial f}{\partial y} = \frac{(e^{2x+y})(x^2+y^2) - e^{2x+y}(2y)}{(x^2+y^2)^2} = \frac{e^{2x+y}(x^2+y^2) - 2ye^{2x+y}}{(x^2+y^2)^2}
\]
Osserviamo che le derivate parziali sono continue in un intorno di $(1,1)$, quindi possiamo concludere che $f$ è differenziabile in $(1,1)$.\\
Valutiamo le derivate parziali nel punto $(1,1)$:
\[
\frac{\partial f}{\partial x}(1,1) = \frac{2e^{3}(1^2+1^2) - 2e^{3}}{(1^2+1^2)^2} = \frac{4e^{3} - 2e^{3}}{4} = \frac{2e^{3}}{4} = \frac{e^{3}}{2}
\]
\[
\frac{\partial f}{\partial y}(1,1) = \frac{e^{3}(1^2+1^2) - 2e^{3}}{(1^2+1^2)^2} = \frac{2e^{3} - 2e^{3}}{4} = 0
\]

L'equazione del piano tangente è quindi:
\[
z = f(1,1) + \frac{\partial f}{\partial x}(1,1)(x-1) + \frac{\partial f}{\partial y}(1,1)(y-1)
\]
\[
z = \frac{e^{3}}{2} + \frac{e^{3}}{2}(x-1) + 0(y-1)
\]
\[
z = \frac{e^{3}}{2} + \frac{e^{3}}{2}(x-1)
\]
\[
z = \frac{e^{3}}{2}(1 + x - 1)
\]
\[
z = \frac{e^{3}}{2}x
\]

Quindi, il piano tangente al grafico di $f$ nel punto $(1,1)$ è dato dall'equazione $z = \frac{e^{3}}{2}x$.\\

Dalla diiferenziabilità possiamo dunque trarre importanti conseguenze. Le più rilevanti sono riassunte nel seguente teorema:

\begin{teorema}{Conseguenze della differenziabilità}
  Sia $f$ differenziabile in $P_0 \in \R^n$. Allora:
  \begin{itemize}
    \item $f$ è continua in $P_0$
    \item Il vettore $\nabla f(P_0)$ indica nel dominio di $f$ e partendo dal punto $P_0$ la direzione di massima crescita di $f$, cioè quella lungo cui $f$ cresce più velocemente.
    \item In $P_0$ il grafico ha pendenza $\alpha$ rispetto alla direzione di $\nabla f(P_0)$, dove $\alpha =  \|\nabla f(P_0)\| = \sqrt{\left(\frac{\partial f}{\partial x_1}(P_0)\right)^2 + \cdots + \left(\frac{\partial f}{\partial x_n}(P_0)\right)^2}$.
    \item Se $n=2$ e $\nabla f(P_0)\neq (0,0)$, allora il gradiente $\nabla f(P_0)$ è ortogonale alla curva di livello di $f$ passante per $P_0$
\end{itemize}
\end{teorema}

\section{Derivata direzionale}
\begin{definizione}{Derivata direzionale}
Sia $f:\Omega \inc \R^n \rightarrow \R$ e sia $P_0 \in \Omega$. La \textbf{derivata direzionale} di $f$ in $P_0$ lungo il vettore $\vec v$ (di norma unitaria) è definita come il limite del rapporto incrementale, se esiste: $D_{\vec v}f(P_0) = \frac{\partial f}{\partial \vec v}(P_0) = \lim_{h \to 0} \frac{f(P_0 + h\vec v) - f(P_0)}{h}$.
\end{definizione}

\begin{osservazione}{}
La derivata direzionale di $f$ in $P_0$ lungo il vettore $\vec v$ rappresenta la pendenza di $f$ in $P_0$ nella direzione di $\vec v$. \\
Inoltre, definito il gradiente come in precedenza, se $f$ è differenziabile, si ha: $\frac{\partial f}{\partial \vec v}(P_0) = \nabla f(P_0) \cdot \vec v $ con $\vec v$ vettore unitario e con la notazione precedente che indica il prodotto scalare.
\end{osservazione}

\section{Derivata lungo una curva}\label{sec:derivata-lungo-una-curva}

\begin{definizione}{Curva in $\R^n$}
Una \textbf{curva} in $\R^n$ è una funzione $\gamma: I \rightarrow \R^n$ con $I$ intervallo di $\R$.
\end{definizione}

Tipicamente considereremo curve dove l'intervallo $I$ è un intervallo chiuso e limitato, cioè $I=[a,b]$.\\
$\gamma(a)$ e $\gamma(b)$ sono detti rispettivamente punto iniziale e punto finale della curva.\\
Concretamente quindi, per ogni $t\in I$ si ha $\gamma(t)=(x_1(t), \dots, x_n(t))$. Gli $x_i(t)$ sono le componenti della curva. Ciascuna componente è una funzione dell'analisi 1.\\
In seguito supporremo sempre che le componenti della curva siano funzioni di classe $C^1$.

\begin{definizione}{Derivata lungo una curva}
  La \textbf{derivata lungo la curva} $\gamma$ in $t$ è definita come il vettore $\gamma'(t)=(x_1'(t), \dots, x_n'(t))$.\\
  Se esiste la derivata seconda di $\gamma$ in $t$ allora si definisce $\gamma''(t)=(x_1''(t), \dots, x_n''(t))$ e così via.
\end{definizione}
Ora data una funzione $f:\Omega \inc \R^n \rightarrow \R$ e una curva $\gamma: [a,b] \rightarrow \R^n$ possiamo definire la derivata di $f$ lungo $\gamma$, grazie al seguente teorema:

\begin{teorema}{}
  Sia $f:\Omega \inc \R^n \rightarrow \R$ e $\gamma: [a,b] \rightarrow \R^n$ una curva. Se $f$ è differenziabile in $\gamma(t)$ per ogni $t \in [a,b]$ e se $\gamma$ è derivabile in $t$, allora la funzione composta $f \circ \gamma$ è derivabile in $t$ e si ha:
  $\frac{d}{dt}(f \circ \gamma)(t) = \nabla f(\gamma(t)) \cdot \gamma'(t)$.
\end{teorema}

\section{Derivate successive}
\begin{definizione}{Derivate successive}
  Se $f$ è differenziabile in un intorno di $P_0$ allora si definiscono le \textbf{derivate seconde} di $f$ (nei punti in cui esistono) come le derivate parziali delle derivate parziali: $\frac{\partial^2 f}{\partial x_i \partial x_j}(P_0) = \frac{\partial}{\partial x_i} \left(\frac{\partial f}{\partial x_j}\right)(P_0)$.  \\
  Quando le derivate seconde esistono si può provare ad andare a avanti a derivare fino a ottenere le derivate terze e così via.
\end{definizione}
\begin{osservazione}{}
  In linea di princio ci sono al più $n$ derivate prime, $n^2$ derivate seconde, $n^k$ derivate k-esime.
\end{osservazione}

Consideriamo ad esempio la funzione $f(x,y) = (x^2 + y^2)\sin(2y)$. Calcoliamo le derivate parziali prime e seconde di $f$.


\[
\frac{\partial f}{\partial x} = \frac{\partial}{\partial x} \left( (x^2 + y^2)\sin(2y) \right) = 2x\sin(2y)
\]
\[
\frac{\partial f}{\partial y} = \frac{\partial}{\partial y} \left( (x^2 + y^2)\sin(2y) \right) = 2y\sin(2y) + (x^2 + y^2)2\cos(2y)
\]

\[
\frac{\partial^2 f}{\partial x^2} = \frac{\partial}{\partial x} \left( 2x\sin(2y) \right) = 2\sin(2y)
\]
\[
\frac{\partial^2 f}{\partial y^2} = \frac{\partial}{\partial y} \left( 2y\sin(2y) + (x^2 + y^2)2\cos(2y) \right)
\]
\[
= 2\sin(2y) + 2y \cdot 2\cos(2y) + 2\cos(2y) \cdot 2y + (x^2 + y^2) \cdot 2(-2\sin(2y))
\]
\[
= 2\sin(2y) + 4y\cos(2y) + 4y\cos(2y) - 4(x^2 + y^2)\sin(2y)
\]
\[
= 2\sin(2y) + 8y\cos(2y) - 4(x^2 + y^2)\sin(2y)
\]
\[
= 2\sin(2y)(1 - 2(x^2 + y^2)) + 8y\cos(2y)
\]

\[
\frac{\partial^2 f}{\partial x \partial y} = \frac{\partial}{\partial y} \left( 2x\sin(2y) \right) = 2x \cdot 2\cos(2y) = 4x\cos(2y)
\]
\[
\frac{\partial^2 f}{\partial y \partial x} = \frac{\partial}{\partial x} \left( 2y\sin(2y) + (x^2 + y^2)2\cos(2y) \right) = 4x\cos(2y)
\]

Osserviamo che in questo esempio le derivate miste ($\frac{\partial^2 f}{\partial y \partial x}$,$\frac{\partial^2 f}{\partial x \partial y}$) sono uguali. In generale questo non è un caso.

\begin{teorema}{Teorema di Schwarz}
  Se le derivate miste $\frac{\partial^2 f}{\partial x \partial y}$ e $\frac{\partial^2 f}{\partial y \partial x}$ esistono e sono continue in un intorno di $P_0$ allora sono uguali.
\end{teorema}

\begin{corollario}{}
  Se $f$ è di classe $C^k$ allora le derivate miste di ordine $k$ sono indipendenti dall'ordine di derivazione.
\end{corollario}

Vediamo ora come si calcolano le derivate direzionali di ordine 2.
\begin{definizione}{Matrice Hessiana}
  Se $f$ è $C^2$ allora si definisce \textbf{matrice Hessiana} di $f$ in $P_0$ la matrice:
  $Hf(P_0)=\begin{Bmatrix}
    \frac{\partial^2 f}{\partial x_1^2}(P_0) & \frac{\partial^2 f}{\partial x_1 \partial x_2}(P_0) & \cdots & \frac{\partial^2 f}{\partial x_1 \partial x_n}(P_0)\\
    \frac{\partial^2 f}{\partial x_2 \partial x_1}(P_0) & \frac{\partial^2 f}{\partial x_2^2}(P_0) & \cdots & \frac{\partial^2 f}{\partial x_2 \partial x_n}(P_0)\\
    \vdots & \vdots & \ddots & \vdots\\
    \frac{\partial^2 f}{\partial x_n \partial x_1}(P_0) & \frac{\partial^2 f}{\partial x_n \partial x_2}(P_0) & \cdots & \frac{\partial^2 f}{\partial x_n^2}(P_0)
  \end{Bmatrix}$
\end{definizione}
\begin{osservazione}{}
  La matrice Hessiana è simmetrica.
\end{osservazione}

\begin{definizione}{Derivata direzionale di ordine 2}
  Se $f$ è $C^2$ allora la \textbf{derivata direzionale di ordine 2} di $f$ in $P_0$ lungo il vettore $\vec v$ è data da:
  $D^2_{\vec v}f(P_0)=\vec v^T Hf(P_0) \vec v$
\end{definizione}

\begin{teorema}{Sviluppo di Taylor di ordine 2}
  Se $f$ è $C^2$ allora si ha lo sviluppo di Taylor di ordine 2 di $f$ in $P_0$:
  $f(P) = f(P_0) + \nabla f(P_0) \cdot (P - P_0) + \frac{1}{2} (P - P_0)^T Hf(P_0) (P - P_0) + o(|P - P_0|^2)$
\end{teorema}

\subsection{Esempio di calcolo dello sviluppo di Taylor di ordine 2}
Consideriamo la funzione $f(x,y) = e^{x+y}$. Vogliamo trovare lo sviluppo di Taylor di ordine 2 di $f$ attorno al punto $(0,0)$.

Calcoliamo le derivate parziali prime di $f$:
\[
\frac{\partial f}{\partial x} = e^{x+y}, \quad \frac{\partial f}{\partial y} = e^{x+y}
\]

Calcoliamo le derivate parziali seconde di $f$:
\[
\frac{\partial^2 f}{\partial x^2} = e^{x+y}, \quad \frac{\partial^2 f}{\partial y^2} = e^{x+y}, \quad \frac{\partial^2 f}{\partial x \partial y} = e^{x+y}
\]

Valutiamo le derivate parziali nel punto $(0,0)$:
\[
\frac{\partial f}{\partial x}(0,0) = 1, \quad \frac{\partial f}{\partial y}(0,0) = 1
\]
\[
\frac{\partial^2 f}{\partial x^2}(0,0) = 1, \quad \frac{\partial^2 f}{\partial y^2}(0,0) = 1, \quad \frac{\partial^2 f}{\partial x \partial y}(0,0) = 1
\]

Lo sviluppo di Taylor di ordine 2 di $f$ attorno al punto $(0,0)$ è dato da:
\[
f(x,y) \approx f(0,0) + \frac{\partial f}{\partial x}(0,0)x + \frac{\partial f}{\partial y}(0,0)y + \frac{1}{2}\left( \frac{\partial^2 f}{\partial x^2}(0,0)x^2 + 2\frac{\partial^2 f}{\partial x \partial y}(0,0)xy + \frac{\partial^2 f}{\partial y^2}(0,0)y^2 \right)
\]
\[
= 1 + x + y + \frac{1}{2}(x^2 + 2xy + y^2)
\]
\[
= 1 + x + y + \frac{1}{2}(x^2 + 2xy + y^2)
\]
\[
= 1 + x + y + \frac{1}{2}x^2 + xy + \frac{1}{2}y^2
\]

Quindi, lo sviluppo di Taylor di ordine 2 di $f(x,y) = e^{x+y}$ attorno al punto $(0,0)$ è:
\[
f(x,y) \approx 1 + x + y + \frac{1}{2}x^2 + xy + \frac{1}{2}y^2
\]

\section{Massimi e minimi di una funzione}

Questa sezione è dedicata allo studio dei punti di massimo e minimo delle funzioni di più variabili e alla loro classificazione.

\begin{definizione}{Punto di minimo}
  Sia $f:\Omega \inc \R^n \rightarrow \R$ e sia $P_0 \in \Omega$. Si dice che $f$ ha un \textbf{punto di minimo} locale in $P_0$ se esiste un intorno di $P_0$ tale che $f(P) \geq f(P_0)$ per ogni $P$ nell'intorno.
\end{definizione}
Analogamente si definisce il punto di massimo locale.
\begin{teorema}{}
  Se $f$ ha un punto di minimo o massimo locale in $P_0$ e $f$ è differenziabile in $P_0$ allora $\nabla f(P_0) = 0$.
\end{teorema}


\begin{osservazione}{}
  Questa condizione corrisponde a un sistema di equazioni:
  \[
  \frac{\partial f}{\partial x_1}(P_0) = 0, \quad \frac{\partial f}{\partial x_2}(P_0) = 0, \quad \dots, \quad \frac{\partial f}{\partial x_n}(P_0) = 0
  \]
\end{osservazione}

\begin{definizione}{Punti stazionari}
  I punti $P_0$ tali che $\nabla f(P_0) = 0$ sono detti \textbf{punti stazionari}.
\end{definizione}

Facciamo un esempio.
Consideriamo la funzione $f(x,y) = x^3 - xy^2 + 2xy$. Vogliamo trovare i punti critici di $f$.

Calcoliamo le derivate parziali prime di $f$:
\[
\frac{\partial f}{\partial x} = 3x^2 - y^2 + 2y
\]
\[
\frac{\partial f}{\partial y} = -2xy + 2x
\]

Poniamo le derivate parziali uguali a zero per trovare i punti critici:
\[
3x^2 - y^2 + 2y = 0
\]
\[
-2xy + 2x = 0
\]

Dalla seconda equazione, possiamo fattorizzare:
\[
2x(-y + 1) = 0
\]

Quindi, abbiamo due casi:
1. $2x = 0 \implies x = 0$\\
2. $-y + 1 = 0 \implies y = 1$

Caso 1: $x = 0$
\[
3(0)^2 - y^2 + 2y = 0 \implies -y^2 + 2y = 0 \implies y(y - 2) = 0
\]
Quindi, $y = 0$ o $y = 2$. I punti critici sono $(0,0)$ e $(0,2)$.

Caso 2: $y = 1$
\[
3x^2 - (1)^2 + 2(1) = 0 \implies 3x^2 - 1 + 2 = 0 \implies 3x^2 + 1 = 0
\]
Questa equazione non ha soluzioni reali.

Quindi, i punti critici della funzione $f(x,y) = x^3 - xy^2 + 2xy$ sono $(0,0)$ e $(0,2)$.\\

Come possiamo capire la natura di un punto critico?
Se $P_0$ è un punto critico di $f$ e $f$ è $C^2$ in un intorno di $P_0$ allora si può studiare lo sviluppo al secondo ordine di $f$ in $P_0$ per capire la natura del punto critico. Lo sviluppo si riduce a: $f(P) - f(P_0) = \frac{1}{2} (P - P_0)^T Hf(P_0) (P - P_0) + o(|P - P_0|^2)$.\\
Si tratta dunque di capire che segno ha il membro destro in un intorno di $P_0$.
\begin{itemize}
\item $P_0$ è un punto di massimo se e solo se il membro destro è sempre negativo in un intorno di $P_0$.
\item $P_0$ è un punto di minimo se e solo se il membro destro è sempre positivo in un intorno di $P_0$.
\item $P_0$ è un punto di sella se e solo se il membro destro cambia segno in un intorno di $P_0$.
\end{itemize}
\begin{osservazione}{}
  Il segno del membro destro dipende solo da segno di $\frac{1}{2} (P - P_0)^T Hf(P_0) (P - P_0)$, che è uguale al segno di $\frac{1}{2} \frac{(P - P_0)^T}{\|P - P_0\|} Hf(P_0) \frac{P-P_0}{\|P - P_0\|}$, e avendo $ \frac{P - P_0}{\|P - P_0\|}$ norma 1, il segno non dipende dal raggio dell'intorno di $P_0$ considerato.
\end{osservazione}

Ricordiamo che una matrice simmetrica $A$ è definita positiva se per ogni vettore $v \neq 0$ si ha $v^T A v > 0$. Analogamente si definiscono le matrici definite negative quelle per cui $v^T A v < 0$ per ogni $v \neq 0$ e indefinite quelle per cui $v^T A v$ può assumere segni diversi a seconda di $v$.

\begin{teorema}{Classificazione dei punti critici}
  Sia $f$ una funzione $C^2$ e sia $P_0$ un punto critico di $f$. Allora:
  \begin{itemize}
    \item Se $Hf(P_0)$ è definita positiva allora $P_0$ è un punto di minimo locale.
    \item Se $Hf(P_0)$ è definita negativa allora $P_0$ è un punto di massimo locale.
    \item Se $Hf(P_0)$ è indefinita allora $P_0$ è un punto di sella.
  \end{itemize}
\end{teorema}

\subsection{Esempio di classificazione dei punti critici}
Consideriamo la funzione $f(x,y) = x^3 - xy^2 + 2xy$. Abbiamo già trovato i punti critici $(0,0)$ e $(0,2)$.

Calcoliamo le derivate parziali seconde di $f$:
\[
\frac{\partial^2 f}{\partial x^2} = 6x, \quad \frac{\partial^2 f}{\partial y^2} = -2x, \quad \frac{\partial^2 f}{\partial x \partial y} = -2y + 2, \quad \frac{\partial^2 f}{\partial y \partial x} = -2y + 2
\]

Valutiamo le derivate parziali seconde nei punti critici.

Per il punto $(0,0)$:
\[
\frac{\partial^2 f}{\partial x^2}(0,0) = 0, \quad \frac{\partial^2 f}{\partial y^2}(0,0) = 0, \quad \frac{\partial^2 f}{\partial x \partial y}(0,0) = 2, \quad \frac{\partial^2 f}{\partial y \partial x}(0,0) = 2
\]

La matrice Hessiana in $(0,0)$ è:
\[
Hf(0,0) = \begin{pmatrix}
0 & 2 \\
2 & 0
\end{pmatrix}
\]

Il determinante della matrice Hessiana è:
\[
\det(Hf(0,0)) = (0)(0) - (2)(2) = -4
\]

Poiché il determinante è negativo, la matrice Hessiana è indefinita. Quindi, il punto $(0,0)$ è un punto di sella.

Per il punto $(0,2)$:
\[
\frac{\partial^2 f}{\partial x^2}(0,2) = 0, \quad \frac{\partial^2 f}{\partial y^2}(0,2) = 0, \quad \frac{\partial^2 f}{\partial x \partial y}(0,2) = -2, \quad \frac{\partial^2 f}{\partial y \partial x}(0,2) = -2
\]

La matrice Hessiana in $(0,2)$ è:
\[
Hf(0,2) = \begin{pmatrix}
0 & -2 \\
-2 & 0
\end{pmatrix}
\]

Il determinante della matrice Hessiana è:
\[
\det(Hf(0,2)) = (0)(0) - (-2)(-2) = -4
\]

Poiché il determinante è negativo, la matrice Hessiana è indefinita. Quindi, il punto $(0,2)$ è un punto di sella.

In conclusione, i punti critici della funzione $f(x,y) = x^3 - xy^2 + 2xy$ sono entrambi punti di sella.

\section{Massimi e minimi vincolati}
In questa sezione ci occuperemo di trovare i massimi e i minimi di una funzione $f$ di più variabili vincolata da una o più equazioni.\\
Esempio: vogliamo trovare i punti critici per una certa $f$ vincolati a stare su una circonferenza di equazione $x^2 + y^2 = 16$.

\begin{osservazione}{}
  Trovare i punti critici per $f$ qui non è significativo, perché potrebbero non essere sulla circonferenza.
\end{osservazione}

\begin{teorema}{Moltiplicatori di Lagrange}
  Sia $f$ una funzione $C^1$ su $V\in \R^n$, $V={(x_1, \dots, x_n)\in \R^n | g(x_1, \dots, x_n) = 0}$, con $g$ una funzione $C^1$ su $V$ e tale che $\nabla g(p) \neq (0, 0, \dots, 0)$ per ogni $p \in V$. Sia $P_0$ un punto critico di $f$ ristretta a $V$. Allora esiste un $\lambda \in \R$ tale che $\nabla f(P_0) = \lambda \nabla g(P_0)$.
\end{teorema}



\chapter{Integrali tripli}

\section{Introduzione}
La costruzione è la stessa degli integrali doppi, ma stavolta siamo in $\R^3$ e non in $\R^2$.\\
Ingredienti: un dominio $\Omega \subseteq \R^3$ e una funzione $f:\Omega \rightarrow \R$ continua e limitata.\\
L'obiettivo è definire l'integrale triplo di $f$ su $\Omega$:
\[
\iiint_{\Omega} f(x,y,z) \, dx \, dy \, dz
\]
e capirne il significato.\\
Per arrivare all'integrale triplo effettuiamo le seguente costruzione in più passaggi:
\begin{enumerate}
  \item Scelgo $\epsilon>0$ e divido $\R^3$ in cubetti di lato $\epsilon$.
  \item Chiamo $C_{\epsilon}$ la famiglia di cubetti interamente contenuti in $\Omega$. (Saranno in numero finito poichè $\Omega$ è limitato).
  \item Campiono $f(x,y,z)$ su ogni cubetto della famiglia, cioè prendo un punto $P_i$ in ogni cubetto $c_i\in C_{\epsilon}$ e calcolo la somma: $S(\epsilon)= \sum_{c_i \in C_{\epsilon}} f(P_i) \cdot \text{volume}(c_i)$.
  \item Si dimostra che esiste finito il limite: $\lim_{\epsilon \to 0} S(\epsilon)$.
\end{enumerate}

\begin{definizione}{Integrale triplo}
  Seguendo quanto detto detto prima, si chiama integrale triplo di $f$ su $\Omega$ il valore:
  \[
  \lim_{\epsilon \to 0} \sum_{c_i \in C_{\epsilon}} f(P_i) \cdot \text{volume}(c_i) = \iiint_{\Omega} f(x,y,z) \, dx \, dy \, dz
  \]
\end{definizione}


\section{Significati degli integrali tripli}
Gli integrali tripli hanno diversi significati e interpretazioni, vediamone alcuni:
\begin{itemize}
  \item Se $f(x,y,z)=1$ allora $\iiint_{\Omega} f(x,y,z) \, dx \, dy \, dz$ è il volume di $\Omega$.
  \item La quantità $\frac{\iiint_{\Omega} f(x,y,z) \, dx \, dy \, dz}{\text{volume}(\Omega)}$ è il valore medio pesato di $f$ su $\Omega$.
  \item Se $f(x,y,z)=x$ allora $\iiint_{\Omega} f(x,y,z) \, dx \, dy \, dz$ è il baricentro di $\Omega$ rispetto all'asse $x$.
  \item Se $f(x,y,z)=y$ allora $\iiint_{\Omega} f(x,y,z) \, dx \, dy \, dz$ è il baricentro di $\Omega$ rispetto all'asse $y$.
  \item Se $f(x,y,z)=z$ allora $\iiint_{\Omega} f(x,y,z) \, dx \, dy \, dz$ è il baricentro di $\Omega$ rispetto all'asse $z$.
  \item Se $f(x,y,z)$ è una densità di massa allora $\iiint_{\Omega} f(x,y,z) \, dx \, dy \, dz$ è la massa di $\Omega$.
  \item Se $f(x,y,z)$ è una densità di massa allora $\iiint_{\Omega} (x^2 + y^2) f(x,y,z) \, dx \, dy \, dz$ è il momento d'inerzia di $\Omega$ rispetto all'asse $z$.
\end{itemize}


\section{Cambiamento di variabili negli integrali tripli}
Vediamo ora come si comportano gli integrali tripli sotto cambiamento di variabili.\\
In 3 variabili consideriamo il seguente cambiamento di coordinate:
\[
  x = x(u,v,w)\\
  y = y(u,v,w)\\
  z = z(u,v,w)
\]
con $(u,v,w) \in \Omega \subseteq \R^3$ e $(x,y,z) \in \Omega' \subseteq \R^3$.\\
L'integrale triplo diventa:
\[
\iiint_{\Omega'} f(x,y,z) \, dx \, dy \, dz = \iiint_{\Omega} f(x(u,v,w),y(u,v,w),z(u,v,w)) \cdot |J(u,v,w)| \, du \, dv \, dw
\]
dove $J(u,v,w)$ è il determinante della matrice jacobiana del cambio di variabili:
\[
J(u,v,w) = \begin{vmatrix}
  \frac{\partial x}{\partial u} & \frac{\partial x}{\partial v} & \frac{\partial x}{\partial w}\\
  \frac{\partial y}{\partial u} & \frac{\partial y}{\partial v} & \frac{\partial y}{\partial w}\\
  \frac{\partial z}{\partial u} & \frac{\partial z}{\partial v} & \frac{\partial z}{\partial w}
\end{vmatrix}
\]

\subsection{Coordinate cilindriche}
Consideriamo il cambiamento di coordinate cilindriche:
\[
\begin{cases}
  x = \rho \cos \theta \\
  y = \rho \sin \theta \\
  z = z
\end{cases}
\]
con $\rho \geq 0$, $0 \leq \theta < 2\pi$ e $-\infty < z < \infty$.\\
Il determinante della matrice jacobiana è:
\[
J(\rho, \theta, z) = \begin{vmatrix}
  \frac{\partial x}{\partial \rho} & \frac{\partial x}{\partial \theta} & \frac{\partial x}{\partial z} \\
  \frac{\partial y}{\partial \rho} & \frac{\partial y}{\partial \theta} & \frac{\partial y}{\partial z} \\
  \frac{\partial z}{\partial \rho} & \frac{\partial z}{\partial \theta} & \frac{\partial z}{\partial z}
\end{vmatrix} = \begin{vmatrix}
  \cos \theta & -\rho \sin \theta & 0 \\
  \sin \theta & \rho \cos \theta & 0 \\
  0 & 0 & 1
\end{vmatrix} = \rho
\]
Quindi l'integrale triplo in coordinate cilindriche diventa:
\[
\iiint_{\Omega'} f(x,y,z) \, dx \, dy \, dz = \iiint_{\Omega} f(\rho \cos \theta, \rho \sin \theta, z) \cdot \rho \, d\rho \, d\theta \, dz
\]

\subsection{Coordinate sferiche}
Consideriamo il cambiamento di coordinate sferiche:
\[
\begin{cases}
  x = \rho \sin \phi \cos \theta \\
  y = \rho \sin \phi \sin \theta \\
  z = \rho \cos \phi
\end{cases}
\]
con $\rho \geq 0$, $0 \leq \theta < 2\pi$ e $0 \leq \phi \leq \pi$.\\
Il determinante della matrice jacobiana è:
\[
J(\rho, \theta, \phi) = \begin{vmatrix}
  \frac{\partial x}{\partial \rho} & \frac{\partial x}{\partial \theta} & \frac{\partial x}{\partial \phi} \\
  \frac{\partial y}{\partial \rho} & \frac{\partial y}{\partial \theta} & \frac{\partial y}{\partial \phi} \\
  \frac{\partial z}{\partial \rho} & \frac{\partial z}{\partial \theta} & \frac{\partial z}{\partial \phi}
\end{vmatrix} = \begin{vmatrix}
  \sin \phi \cos \theta & -\rho \sin \phi \sin \theta & \rho \cos \phi \cos \theta \\
  \sin \phi \sin \theta & \rho \sin \phi \cos \theta & \rho \cos \phi \sin \theta \\
  \cos \phi & 0 & -\rho \sin \phi
\end{vmatrix} = \rho^2 \sin \phi
\]
Quindi l'integrale triplo in coordinate sferiche diventa:
\[
\iiint_{\Omega'} f(x,y,z) \, dx \, dy \, dz = \iiint_{\Omega} f(\rho \sin \phi \cos \theta, \rho \sin \phi \sin \theta, \rho \cos \phi) \cdot \rho^2 \sin \phi \, d\rho \, d\theta \, d\phi
\]

\section{Calcolo degli integrali tripli}
Per calcolare gli integrali tripli possiamo procedere in due modi: per fili e per strati.
Per fili significa pensare $\Omega$ come un insieme di fili paralleli ad un asse.\\
Se esistono due funzioni $g_1(x,y,z)$ e $g_2(x,y,z)$ tali che $\Omega = {(x,y,z)\in \R^3|(x,y)\in D, g_1(x,y,z) \leq z \leq g_2(x,y,z)}$ allora:
\[
\iiint_{\Omega} f(x,y,z) \, dx \, dy \, dz = \iint_{D} \left( \int_{g_1(x,y)}^{g_2(x,y)} f(x,y,z) \, dz \right) \, dx \, dy
\]

Per strati invece significa pensare $\Omega$ come un insieme di strati paralleli ad un piano, cioè di sezioni orizontali.\\
Dato $\Omega$ definiamo le sezioni a quota $z$: $S(z) = \{(x,y)\in \R^2 | (x,y,z)\in \Omega\}$.\\
Possiamo dunque ricavare la seguente formula di integrazione per strati:
\[
\iiint_{\Omega} f(x,y,z) \, dx \, dy \, dz = \int_{-\infty}^{+\infty} \left( \iint_{S(z)} f(x,y,z) \, dx \, dy \right) \, dz
\]
ma di fatto gli estremi di integrazione su $z$ sono quelli che definiscono l'altezza di $\Omega$ e si riducono perciò a quelli per cui $S(z)\neq 0$ (esistono di fatto uno $Z_{\text{min}}$ e uno $Z_{\text{max}}$).

\chapter{Integrali doppi}

\section{Introduzione}
In questo capitolo ci occuperemo dello studio degli integrali doppi, ovvero integrali di funzioni di due variabili reali.\\

Gli ingredienti necessari per un integrale doppio sono:
\begin{itemize}
\item Un insieme $\Omega \in \R^2$, aperto e limitato su cui integrare la funzione.
\item Una funzione $f:\Omega \rightarrow \R$ continua e limitata.
\end{itemize}

L'obiettivo è definire l'integrale doppio di $f$ su $\Omega$
\[
\iint_{\Omega} f(x,y) \, dx \, dy
\]
e capirne il significato.\\

Per arrivare all'integrale doppio effettuiamo le seguente costruzione in più passaggi:
\begin{enumerate}
  \item Scelgo $\epsilon>0$ e divido $\R^2$ in quadrati di lato $\epsilon$.
  \item Chiamo $Q_{\epsilon}$ la famiglia di quadrati interamente contenuti in $\Omega$. (Saranno in numero finito poichè $\Omega$ è limitato).
  \item Campiono $f(x,y)$ su ogni quadratino della famiglia, cioè prendo un punto $P_i$ in ogni quadratino $q_i\in Q_{\epsilon}$ e calcolo la somma: $S(\epsilon)= \sum_{q_i \in Q_{\epsilon}} f(P_i) \cdot \text{area}(q_i)$.
  \item Si dimostra che esiste finito il limite: $\lim_{\epsilon \to 0} S(\epsilon)$.
\end{enumerate}

\begin{definizione}{Integrale doppio}
  Seguendo quanto detto detto prima, si chiama \textbf{integrale doppio} di $f$ su $\Omega$ il valore:
  \[
  \lim_{\epsilon \to 0} \sum_{q_i \in Q_{\epsilon}} f(P_i) \cdot \text{area}(q_i) = \iint_{\Omega} f(x,y) \, dx \, dy
  \]
\end{definizione}

\section{Significati dell'integrale doppio}
Diamo ora alcuni possibili significati e interpretazioni dell'integrale doppio di una funzione $f(x,y)$ su un insieme $\Omega$:

\begin{itemize}
  \item Se $f(x,y)=1$ allora $\iint_{\Omega} f(x,y) \, dx \, dy$ è l'area di $\Omega$.
  \item La quantità $\frac{\iint_{\Omega} f(x,y) \, dx \, dy}{\text{area}(\Omega)}$ è il valore medio pesato di $f$ su $\Omega$.
  \item Se $f(x,y)=x$ allora $\iint_{\Omega} f(x,y) \, dx \, dy$ è la coordinata $x$ del baricentro di $\Omega$.
  \item Se $f(x,y)=y$ allora $\iint_{\Omega} f(x,y) \, dx \, dy$ è la  coordinata $y$ del baricentro di $\Omega$.
  \item Se $f(x,y)$ è una densità di massa allora $\iint_{\Omega} f(x,y) \, dx \, dy$ è la massa di $\Omega$.
  \item Se $f(x,y)=x^2+y^2$ allora $\iint_{\Omega} f(x,y) \, dx \, dy$ è il momento di inerzia di $\Omega$ rispetto all'origine.
\end{itemize}

\section{Calcolo dell'integrale doppio}
In questa sezione vedremo alcuni metodi per il calcolo degli integrali doppi.\\

\begin{definizione}{Semplice per fili verticali}
  $\Omega$ si dice \textbf{semplice per fili verticali} se è del tipo: $\Omega = \{ (x,y) \in \R^2 : a \leq x \leq b, g(x) \leq y \leq h(x) \}$, dove $g,h:[a,b] \rightarrow \R$ sono continue e $g(x) \leq h(x) \ \forall x \in [a,b]$.
\end{definizione}

Se $\Omega$ è semplice per fili verticali allora:
\[
\iint_{\Omega} f(x,y) \, dx \, dy = \int_{a}^{b} \left( \int_{g(x)}^{h(x)} f(x,y) \, dy \right) \, dx
\]

\begin{osservazione}{}
  Nell'integrale più interno la $x$ ha il ruolo di un parametro.
\end{osservazione}

\begin{definizione}{Semplice per fili orizzontali}
  $\Omega$ si dice \textbf{semplice per fili orizzontali} se è del tipo: $\Omega = \{ (x,y) \in \R^2 : c \leq y \leq d, p(y) \leq x \leq q(y) \}$, dove $p,q:[c,d] \rightarrow \R$ sono continue e $p(y) \leq q(y) \ \forall y \in [c,d]$.
\end{definizione}

Se $\Omega$ è semplice per fili orizzontali allora:
\[
\iint_{\Omega} f(x,y) \, dx \, dy = \int_{c}^{d} \left( \int_{p(y)}^{q(y)} f(x,y) \, dx \right) \, dy
\]

Tuttavia, per quanto i due metodi sembrino molto simili, ci sono casi in cui l'asimmetria del problema rende uno dei due inutile e l'altro molto comodo.\\


Ad esempio, onsideriamo il triangolo $\Omega$ con vertici $(0,0)$, $(2,0)$ e $(2,4)$. Vogliamo calcolare l'integrale doppio $\iint_{\Omega} e^{x^2} \, dx \, dy$.

Proviamo prima con il metodo dei fili orizzontali.\\
Dobbiamo esprimere $\Omega$ come:
\[
\Omega = \{ (x,y) \in \R^2 : 0 \leq y \leq 4, p(y) \leq x \leq q(y) \}
\]
Nel nostro caso, $p(y) = 0$ e $q(y) = 2$ per $0 \leq y \leq 4$. Quindi:
\[
\iint_{\Omega} e^{x^2} \, dx \, dy = \int_{0}^{4} \left( \int_{0}^{2} e^{x^2} \, dx \right) \, dy
\]
Tuttavia, l'integrale interno $\int_{0}^{2} e^{x^2} \, dx$ non è elementare, quindi questo metodo non è conveniente.\\

Proviamo invece con il metodo dei fili verticali.\\
In questo caso, dobbiamo esprimere $\Omega$ come:
\[
\Omega = \{ (x,y) \in \R^2 : 0 \leq x \leq 2, g(x) \leq y \leq h(x) \}
\]
Nel nostro caso, $g(x) = 0$ e $h(x) = 2x$ per $0 \leq x \leq 2$. Quindi:
\[
\iint_{\Omega} e^{x^2} \, dx \, dy = \int_{0}^{2} \left( \int_{0}^{2x} e^{x^2} \, dy \right) \, dx
\]
Poiché $e^{x^2}$ è costante rispetto a $y$, possiamo semplificare l'integrale interno:
\[
\int_{0}^{2x} e^{x^2} \, dy = e^{x^2} \int_{0}^{2x} \, dy = e^{x^2} \cdot 2x
\]
Quindi l'integrale doppio diventa:
\[
\iint_{\Omega} e^{x^2} \, dx \, dy = \int_{0}^{2} 2x e^{x^2} \, dx
\]
Facciamo il cambio di variabile $u = x^2$, quindi $du = 2x \, dx$:
\[
\int_{0}^{2} 2x e^{x^2} \, dx = \int_{0}^{4} e^u \, du = e^u \bigg|_{0}^{4} = e^4 - e^0 = e^4 - 1
\]
Quindi:
\[
\iint_{\Omega} e^{x^2} \, dx \, dy = e^4 - 1
\]
Abbiamo visto dunque che il metodo dei fili verticali è molto più conveniente in questo caso.

\section{Cambiamento di variabile}
In questa sezione ci occupiamo di definire i cambiamenti di variabili in due variabili.\\
Vorremmo descrivere un integrale doppio $\iint_{\Omega} f(x,y) \, dx \, dy$ tramite due nuove variabili $u$ e $v$.\\
Date $x = g_1(u,v)$ e $y = g_2(u,v)$, $\varphi(u,v)=(g_1(u,v)$,$y = g_2(u,v))$ è una funzione invertibili e di classe $C^1$.

\begin{teorema}{Teorema di cambiamento di variabile}
  Nelle ipotesi precedenti si ha: $\iint_{\Omega} f(x,y) \, dx \, dy = \iint_{\Omega'} f(g_1(u,v),g_2(u,v)) \cdot |J_{\varphi}(u,v)| \, du \, dv$, dove $J_{\varphi}(u,v)$ è il determinante della matrice jacobiana di $\varphi$.
\end{teorema}

\begin{definizione}{Matrice Jacobiana}
  La \textbf{matrice Jacobiana} di una funzione $\varphi: \R^m \rightarrow \R^n$ è la matrice $n \times m$ delle derivate parziali:
  \[
  J_{\varphi}(u_1, u_2, \ldots, u_m) = \begin{pmatrix}
  \frac{\partial g_1}{\partial u_1} & \frac{\partial g_1}{\partial u_2} & \cdots & \frac{\partial g_1}{\partial u_m} \\
  \frac{\partial g_2}{\partial u_1} & \frac{\partial g_2}{\partial u_2} & \cdots & \frac{\partial g_2}{\partial u_m} \\
  \vdots & \vdots & \ddots & \vdots \\
  \frac{\partial g_n}{\partial u_1} & \frac{\partial g_n}{\partial u_2} & \cdots & \frac{\partial g_n}{\partial u_m}
  \end{pmatrix}
  \]
\end{definizione}

\begin{osservazione}{}
  $|J_{\varphi}(u,v)|$ si può vedere geometricamente come il fattore di scala del cambiamento di variabile.
\end{osservazione}

\subsection{Coordinate polari}
Consideriamo il passaggio alle coordinate polari $(r, \theta)$, dove $x = r \cos \theta$ e $y = r \sin \theta$.\\
Il determinante della matrice Jacobiana in questo caso è:
\[
J_{\varphi}(r,\theta) = \begin{vmatrix}
\frac{\partial x}{\partial r} & \frac{\partial x}{\partial \theta} \\
\frac{\partial y}{\partial r} & \frac{\partial y}{\partial \theta}
\end{vmatrix} = \begin{vmatrix}
\cos \theta & -r \sin \theta \\
\sin \theta & r \cos \theta
\end{vmatrix} = r (\cos^2 \theta + \sin^2 \theta) = r
\]

Quindi, l'integrale doppio in coordinate polari diventa:
\[
\iint_{\Omega} f(x,y) \, dx \, dy = \iint_{\Omega'} f(r \cos \theta, r \sin \theta) \cdot r \, dr \, d\theta
\]

Ad esempio, calcoliamo l'integrale doppio $\iint_{\Omega} (x^2 + y^2) \, dx \, dy$ dove $\Omega$ è il cerchio di raggio $R$ centrato nell'origine.\\
In coordinate polari, abbiamo $x^2 + y^2 = r^2$ e $\Omega' = \{ (r, \theta) : 0 \leq r \leq R, 0 \leq \theta \leq 2\pi \}$. Quindi:
\[
\iint_{\Omega} (x^2 + y^2) \, dx \, dy = \iint_{\Omega'} r^2 \cdot r \, dr \, d\theta = \int_{0}^{2\pi} \int_{0}^{R} r^3 \, dr \, d\theta
\]

Calcoliamo l'integrale interno:
\[
\int_{0}^{R} r^3 \, dr = \frac{r^4}{4} \bigg|_{0}^{R} = \frac{R^4}{4}
\]

Quindi l'integrale doppio diventa:
\[
\iint_{\Omega} (x^2 + y^2) \, dx \, dy = \int_{0}^{2\pi} \frac{R^4}{4} \, d\theta = \frac{R^4}{4} \cdot 2\pi = \frac{\pi R^4}{2}
\]

\subsection{Coordinate ellittiche}
Consideriamo il passaggio alle coordinate ellittiche $(r, \theta)$, dove $x = a r \cos \theta$ e $y = b r \cos \theta$.\\
Il determinante della matrice Jacobiana in questo caso è:
\[
J_{\varphi}(r,\theta) = \begin{vmatrix}
\frac{\partial x}{\partial r} & \frac{\partial x}{\partial \theta} \\
\frac{\partial y}{\partial r} & \frac{\partial y}{\partial \theta}
\end{vmatrix} = \begin{vmatrix}
a \cos \theta & -a r \sin \theta \\
b \cos \theta & -b r \sin \theta
\end{vmatrix} = ab r (\cos^2 \theta + \sin^2 \theta) = abr
\]

Utilizzando l'identità trigonometrica $\cos^2 \theta + \sin^2 \theta = 1$, possiamo semplificare il determinante:
\[
J_{\varphi}(r,\theta) = abr
\]

Quindi, l'integrale doppio in coordinate ellittiche diventa:
\[
\iint_{\Omega} f(x,y) \, dx \, dy = \iint_{\Omega'} f(a r \cos \theta, b r \cos \theta) \cdot abr \, dr \, d\theta
\]





\chapter{Integrali curvilinei}

\section{Introduzione}
In questo capitolo ci occuperemo di integrali curvilinei, ovvero integrali definiti su curve.\\
Nel paragrafo ~\ref{sec:derivata-lungo-una-curva} abbiamo definito le curve e le derivate lungo esse.\\
Vediamo ora un utile teorema che ci servirà per ricavare gli integrali curvilinei.

\begin{teorema}{Derivata della funzione composta}
  Siano $f(x_1, \dots, x_n): \R^n \rightarrow \R^k$ e $g(y_1, \dots, y_m): \R^m \rightarrow \R^n$ funzioni differenziabili. Allora la funzione composta $f \circ g: \R^m \rightarrow \R^k$ è differenziabile e la sua matrice Jacobiana è data da:
  $$J(f \circ g)(y) = Jf(g(y)) \cdot Jg(y)$$
\end{teorema}

\begin{proof}
Siano $f: \R^n \rightarrow \R^k$ e $g: \R^m \rightarrow \R^n$ funzioni differenziabili. Per definizione, la funzione composta $f \circ g: \R^m \rightarrow \R^k$ è data da $(f \circ g)(y) = f(g(y))$.\\
Per calcolare la derivata di $f \circ g$, consideriamo la variazione infinitesima di $y$ in $\R^m$. La variazione di $g(y)$ è data da:
$$dg(y) = Jg(y) \, dy,$$
dove $Jg(y)$ è la matrice Jacobiana di $g$ in $y$.\\
Analogamente, la variazione di $f(g(y))$ è data da:
$$df(g(y)) = Jf(g(y)) \, dg(y).$$
Sostituendo $dg(y)$ nella relazione precedente, otteniamo:
$$df(g(y)) = Jf(g(y)) \cdot Jg(y) \, dy.$$
Pertanto, la matrice Jacobiana di $f \circ g$ è:
$$J(f \circ g)(y) = Jf(g(y)) \cdot Jg(y).$$
Questo conclude la dimostrazione.
\end{proof}

Gli integrali curvilinei possono essere di due tipi: di prima e di seconda specie.\\
Gli integrali di prima specie sono integrali di una funzione scalare $f(x_1, \dots, x_n)$ lungo una curva $\gamma: [a, b] \rightarrow \R^n$.\\
Gli integrali di seconda specie sono integrali di un campo vettoriale $\vec F(x_1, \dots, x_n)$ lungo una curva $\gamma: [a, b] \rightarrow \R^n$.\\

\begin{osservazione}{}
  Supporremo sempre che $\gamma$ sia $C^1$ (o $C^1$ a tratti) e che $f$ e $\vec F$ siano continue.
\end{osservazione}

\section{Integrali curvilinei di prima specie}
\begin{definizione}{Integrale curvilineo di prima specie}
  Sia $f: \R^n \rightarrow \R$ una funzione continua e $\gamma: [a, b] \rightarrow \R^n$ una curva $C^1$ (o $C^1$ a tratti).\\
  L'\textbf{integrale curvilineo di prima specie} di $f$ lungo $\gamma$ è definito come:
  $$\int_\gamma f ds = \int_a^b f(\gamma(t)) \cdot ||\gamma'(t)|| dt$$
\end{definizione}

Capiamo meglio il significato. Posso pensare di dividere la curva in piccoli tratti di diametro $< \epsilon$. Campiono $f$ in ogni punto e moltiplico per la lunghezza del tratto. Sommo tutti i tratti e faccio tendere $\epsilon$ a 0. Ovvero:
$$\int_\gamma f ds = \lim_{\epsilon \rightarrow 0} \sum_{i=1}^n f(\gamma(t_i)) \cdot ||\gamma(t_i) - \gamma(t_{i-1})||$$.\\
Qui $\epsilon$ limita il valore di $||\gamma(t_i) - \gamma(t_{i-1})||$.\\

\section{Integrali curvilinei di seconda specie}\label{sec:integrali-curvilinei-di-seconda-specie}
\begin{definizione}{Integrale curvilineo di seconda specie}
  Sia $\vec F: \R^n \rightarrow \R^n$ un campo vettoriale continuo e $\gamma: [a, b] \rightarrow \R^n$ una curva $C^1$ (o $C^1$ a tratti).\\
  L'\textbf{integrale curvilineo di seconda specie} di $\vec F$ lungo $\gamma$ è definito come:
  $$\int_\gamma \vec F \cdot d\vec r = \int_a^b \vec F(\gamma(t)) \cdot \gamma'(t) dt$$
\end{definizione}

Nuovamente, il significato è quello di dividere la curva in piccoli tratti di diametro $< \epsilon$. Campiono $\vec F$ in ogni punto e moltiplico per il vettore tangente al tratto. Sommo tutti i tratti e faccio tendere $\epsilon$ a 0. Ovvero:
$$\int_\gamma \vec F \cdot d\vec r = \lim_{\epsilon \rightarrow 0} \sum_{i=1}^n \vec F(\gamma(t_i)) \cdot (\gamma(t_i) - \gamma(t_{i-1}))$$.\\
Questo integrale esprime il lavoro compiuto dal campo $\vec F$ lungo la curva $\gamma$. Non dipende da come parametrizzo $\gamma$ se non per il verso di percorrenza.

\chapter{Campi conservativi}

\section{Introduzione}\label{sec:introduzione}
Vediamo come prima cosa un importante esempio di integrale curvilineo che ci servirà nella spiegazione dei campi conservativi.\\
Consideriamo il campo (magnetico) $\vec F (x,y) = (\frac{-y}{x^2+y^2}, \frac{x}{x^2+y^2})$ e il cammino $\gamma$ che è la circonferenza di raggio $1$ centrata nell'origine.\\
Come spiegato nel paragrafo ~\ref{sec:integrali-curvilinei-di-seconda-specie}, considerata la parametrizzazione di $\gamma$ data da $\gamma(t) = (\cos t, \sin t)$ con $t \in [0, 2\pi]$, possiamo calcolare l'integrale curvilineo di $\vec F$ lungo $\gamma$:
$$\int_\gamma \vec F \cdot d\vec r = \int_0^{2\pi} \vec F(\gamma(t)) \cdot \gamma'(t) dt =$$\\
$$\int_0^{2\pi} \left( \frac{-\sin t}{\cos^2 t + \sin^2 t}, \frac{\cos t}{\cos^2 t + \sin^2 t} \right) \cdot (-\sin t, \cos t) dt = \int_0^{2\pi} 1 dt =2\pi$$.

\begin{osservazione}{}
  Il valore dell'integrale vale $2\pi$ indipendentemente dal raggio della circonferenza.\\
\end{osservazione}

In qualche caso tuttavia $\int_\gamma \vec F \cdot d\vec r$ è calcolabile in altro modo (senza integrale).\\

\section{Campi conservativi}
Vediamo la seguente proposizione:

\begin{teorema}{}
  Sia $\Omega$ un aperto in $\R^n$ e $\gamma: [a,b] \rightarrow \R^n$ una curva $C^1$ a tratti contenuta in $\Omega$. Sia $\vec F: \Omega \rightarrow \R^n$ un campo vettoriale di classe $C^1$ e sia $f: \Omega \rightarrow \R$ tale che $\vec F = \nabla f$. Allora:
  $$\int_\gamma \vec F \cdot d\vec r = f(\gamma(b)) - f(\gamma(a))$$.
\end{teorema}

Questo motiva la seguente definizione:
\begin{definizione}{Campo conservativo}
  Nelle ipotesi precedenti, un campo vettoriale $\vec F: \Omega \rightarrow \R^n$ è detto \textbf{conservativo} se esiste una funzione $f: \Omega \rightarrow \R$ tale che $\vec F = \nabla f$. \\ In tal caso la funzione $f$ si chiama potenziale di $\vec F$ in $\Omega$.
\end{definizione}

\begin{osservazione}{}
  Dire che un campo è o non è conservativo senza specificare dove è definito non ha significato.\\
\end{osservazione}

Consideriamo ad esempio il campo visto nel paragrafo precedente $\vec F(x,y) = (\frac{-y}{x^2+y^2}, \frac{x}{x^2+y^2})$. Questa volta però consideriamo come $\Omega_1$ il primo quadrante.\\
Vediamo che $\vec F$ è conservativo in $\Omega_1$ in quanto $\vec F = \nabla \arctan \frac{y}{x}$.\\
Tuttavia abbiamo già calcolato che l'integrale di questo campo lungo una circonferenza di raggio generico centrata nell'origine è $2\pi$. Perciò possiamo concludere che il campo non è conservativo in qualunque aperto $\Omega_2 \in \R^2$ che contenga una circonferenza centrata nell'origine.\\

\begin{osservazione}{}
  Se $\vec F$ è conservativo in $\Omega$ e $\gamma$ è una curva chiusa in $\Omega$ allora $\int_\gamma \vec F \cdot d\vec r = 0$.\\
  Se $\gamma$ è chiusa l'integrale si chiama circuitazione di $\vec F$ lungo $\gamma$.
\end{osservazione}

Capiamo allora cosa manca al campo precendente per essere conservativo in $\R^2 \setminus \{0\}$.\\

\begin{definizione}{Connessione per archi}
  Un aperto $\Omega \in \R^n$ si dice connesso per archi se per ogni coppia di punti $P,Q \in \Omega$ esiste una curva $C^1$ a tratti $\gamma: [a,b] \rightarrow \Omega$ tale che $\gamma(a) = P$ e $\gamma(b) = Q$.
\end{definizione}

Enunciamo ora il seguente teorema:
\begin{teorema}{}
  Sia $\Omega \in \R^n$ un aperto connesso per archi e $\vec F: \Omega \rightarrow \R^n$ un campo vettoriale di classe $C^1$. Allora le seguenti affermazioni sono equivalenti:
  \begin{enumerate}
    \item $\vec F$ è conservativo in $\Omega$.
    \item Per ogni curva chiusa $\gamma$ in $\Omega$ si ha $\int_\gamma \vec F \cdot d\vec r = 0$.
    \item Se $\gamma_1$ e $\gamma_2$ sono curve con lo stesso punto iniziale e finale in $\Omega$ allora $\int_{\gamma_1} \vec F \cdot d\vec r = \int_{\gamma_2} \vec F \cdot d\vec r$.
  \end{enumerate}
\end{teorema}

Enunciamo ora condizioni necessarie e sufficienti per la conservatività:


\begin{definizione}{Irrotazionale}
  Un campo vettoriale $\vec F: \Omega \rightarrow \R^n$ si dice \textbf{irrotazionale} se $\nabla \times \vec F = 0$. \\
\end{definizione}
In due variabili è equivalente a:
$$\frac{\partial F_2}{\partial x} - \frac{\partial F_1}{\partial y}=0$$.
In tre variabili questo significa:
$$\nabla \times \vec F = \left( \frac{\partial F_3}{\partial y} - \frac{\partial F_2}{\partial z}, \frac{\partial F_1}{\partial z} - \frac{\partial F_3}{\partial x}, \frac{\partial F_2}{\partial x} - \frac{\partial F_1}{\partial y} \right) = (0,0,0)$$.

\begin{teorema}{Condizione necessaria per la conservatività}
  Sia $\Omega \in \R^n$ un aperto connesso per archi e $\vec F: \Omega \rightarrow \R^n$ un campo vettoriale di classe $C^1$. Se $\vec F$ è conservativo in $\Omega$ allora $\frac{\partial F_i}{\partial x_j} = \frac{\partial F_j}{\partial x_i}$ per ogni $i,j = 1, \ldots, n$. \\
  Ciò significa che la matrice jacobiana di $\vec F$ è simmetrica e che il campo $\vec F$ è irrotazionale.
\end{teorema}

La seguente definizione sarà utile per trovare una condizione sufficiente di conservatività:

\begin{definizione}{Semplicemente connesso}
  Un aperto $\Omega \in \R^n$ si dice semplicemente connesso se per ogni curva chiusa $\gamma$ in $\Omega$ si ha che $\gamma$ può essere deformata con continuità fino a stringerla ottenendo un singolo punto senza mai uscire da $\Omega$.
\end{definizione}

Nel piano la condizione è equivalente a dire che $\Omega$ non ha buchi.\\
In $\R^3$ la condizione è più complessa da verificare.\\

\begin{teorema}{Condizione sufficiente per la conservatività}
  Sia $\Omega \in \R^n$ un aperto semplicemente connesso e $\vec F: \Omega \rightarrow \R^n$ un campo vettoriale di classe $C^1$. Se $\vec F$ è irrotazionale in $\Omega$ allora $\vec F$ è conservativo in $\Omega$.
\end{teorema}

\section{Campi centrali}
Vediamo un importante tipo di campi:

\begin{definizione}{Campo centrale}
  Un campo vettoriale $\vec F: \R^n \setminus \{0\} \rightarrow \R^n$ si dice \textbf{centrale} se esiste una funzione $f: \R^n \setminus \{0\} \rightarrow \R$ tale che $\vec F(\vec{x}) = f(\|\vec{x}\|) \frac{\vec{x}}{\|\vec{x}\|}$, dove $\|\vec{x}\|$ è la norma euclidea di $\vec{x}$.
\end{definizione}


Esiste allora un teorema che ci permette di risolvere molte patologie:
\begin{teorema}{Conservatività dei campi centrali}
  Un campo centrale $\vec F: \R^n \setminus \{0\} \rightarrow \R^n$ è conservativo.
\end{teorema}

\begin{osservazione}{}
  Da quanto detto riguardo alla condizione necessaria di conservatività segue che campo centrale $\vec F: \R^n \setminus \{0\} \rightarrow \R^n$ è irrotazionale.
\end{osservazione}

\section{Potenziale di un campo}

Dato un aperto $\Omega \inc \R^n$ e un campo vettoriale $\vec F: \Omega \rightarrow \R^n$ conservativo (oppure almeno irrotazionale) in $\Omega$, vediamo come si può ricostruire (quando esiste) un potenziale $f$ in $\Omega$.

\begin{definizione}{Potenziale}
  Nelle condizioni precedenti una funzione $f: \Omega \rightarrow \R$ tale che $\vec F = \nabla f$ si chiama \textbf{potenziale} di $\vec F$ in $\Omega$.
\end{definizione}

In due variabili $\Omega \in \R^2$ abbiamo che $\vec F = (F_1, F_2)$ e vogliamo\\
$$F_1(x,y) = \frac{\partial f(x,y)}{\partial{x}} $$
$$F_2(x,y) = \frac{\partial f(x,y)}{\partial{y}} $$.\\
Lo schema è il seguente:
\begin{enumerate}
\item Calcoliamo $f(x,y) = \int F_1(x,y)dx + g(y)$.
\item Calcoliamo $\frac{\partial f(x,y)}{\partial{y}}$.
\item Imponiamo $F_2(x,y)=\frac{\partial f(x,y)}{\partial{y}}$ e troviamo $g(y)$.
\end{enumerate}

In 3 variabili il processo è analogo.\\
\subsection{Esempio di calcolo del potenziale}

Consideriamo il campo vettoriale $\vec F(x,y) = (2xy, x^2 + 1)$. Vogliamo trovare un potenziale $f(x,y)$ tale che $\vec F = \nabla f$.

\begin{enumerate}
\item Calcoliamo $f(x,y)$ integrando $F_1$ rispetto a $x$:
$$ f(x,y) = \int 2xy \, dx = x^2 y + g(y) $$
dove $g(y)$ è una funzione da determinare.

\item Calcoliamo la derivata parziale di $f(x,y)$ rispetto a $y$:
$$ \frac{\partial f(x,y)}{\partial y} = x^2 + g'(y) $$

\item Imponiamo che questa derivata sia uguale a $F_2(x,y)$:
$$ x^2 + g'(y) = x^2 + 1 $$
da cui otteniamo:
$$ g'(y) = 1 $$

Integrando rispetto a $y$, troviamo:
$$ g(y) = y + C $$

Quindi il potenziale è:
$$ f(x,y) = x^2 y + y + C $$
\end{enumerate}

Abbiamo quindi trovato che il potenziale del campo $\vec F(x,y) = (2xy, x^2 + 1)$ è $f(x,y) = x^2 y + y + C$.\\





\chapter{Green}
Questo capitolo sarà interamente dedicato al teorema di Green nel piano. Esso è un caso particolare del teorema del rotore (o teorema di Stokes) che verrà affrontato prossimamente.\\

Il teorema di Green consente di caclolare la circuitazione di un campo riducendola a un integrale doppio.

\section{Teorema di Green}
\begin{teorema}{Teorema di Green}
Sia $\Omega \inc \R^2$ aperto limitato tale che la sua frontiera è l'unione di $N$ curve chiuse disgiunte $\gamma_1, \dots, \gamma_N$  e sia $\vec F:\Omega\rightarrow \R^2$ di classe $C^1$, allora:
$$ \int_\Omega \left(\frac{\partial F_2}{\partial x}-\frac{\partial F_1}{\partial y}\right) dxdy = \sum_{i=1}^{N} \int_{\gamma_i} \vec{F} \cdot d\vec{r} $$,\\dove ogni $\gamma_i$ è orientata in modo da avere $\Omega$ alla sua sinistra.
\end{teorema}

\begin{proof}
Dimostriamo il teorema di Green per un triangolo rettangolo arbitrario con un vertice nell'origine. Sia $\Delta$ il triangolo rettangolo con vertici $(0,0)$, $(a,0)$ e $(0,b)$.

Il bordo $\partial \Delta$ è costituito da tre segmenti:
\begin{itemize}
  \item $\gamma_1$: il segmento da $(0,0)$ a $(a,0)$.
  \item $\gamma_2$: il segmento da $(a,0)$ a $(0,b)$.
  \item $\gamma_3$: il segmento da $(0,b)$ a $(0,0)$.
\end{itemize}

Calcoliamo il lato sinistro del teorema di Green:
$$ \int_\Delta \left(\frac{\partial F_2}{\partial x} - \frac{\partial F_1}{\partial y}\right) dxdy. $$

Sia $\vec{F} = (F_1, F_2)$ un campo di classe $C^1$. All'interno del triangolo $\Delta$, possiamo parametrizzare l'integrale doppio come:
$$ \int_\Delta \left(\frac{\partial F_2}{\partial x} - \frac{\partial F_1}{\partial y}\right) dxdy = \int_0^a \int_0^{b(1 - \frac{x}{a})} \left(\frac{\partial F_2}{\partial x} - \frac{\partial F_1}{\partial y}\right) dydx. $$

Ora calcoliamo il lato destro del teorema di Green, ovvero la circuitazione lungo $\partial \Delta$:
$$ \int_{\partial \Delta} \vec{F} \cdot d\vec{r} = \int_{\gamma_1} \vec{F} \cdot d\vec{r} + \int_{\gamma_2} \vec{F} \cdot d\vec{r} + \int_{\gamma_3} \vec{F} \cdot d\vec{r}. $$

\begin{itemize}
  \item Per $\gamma_1$: $\vec{r}(t) = (t, 0)$ con $t \in [0, a]$, quindi $d\vec{r} = (1, 0) dt$. Abbiamo:
  $$ \int_{\gamma_1} \vec{F} \cdot d\vec{r} = \int_0^a F_1(t, 0) dt. $$

  \item Per $\gamma_2$: $\vec{r}(t) = (a - t, \frac{b}{a}t)$ con $t \in [0, a]$, quindi $d\vec{r} = (-1, \frac{b}{a}) dt$. Abbiamo:
  $$ \int_{\gamma_2} \vec{F} \cdot d\vec{r} = \int_0^a \left[ F_1(a - t, \frac{b}{a}t)(-1) + F_2(a - t, \frac{b}{a}t)\frac{b}{a} \right] dt. $$

  \item Per $\gamma_3$: $\vec{r}(t) = (0, b - t)$ con $t \in [0, b]$, quindi $d\vec{r} = (0, -1) dt$. Abbiamo:
  $$ \int_{\gamma_3} \vec{F} \cdot d\vec{r} = \int_0^b F_2(0, b - t)(-1) dt = -\int_0^b F_2(0, b - t) dt. $$
\end{itemize}

Sommando i contributi, otteniamo:
$$ \int_{\partial \Delta} \vec{F} \cdot d\vec{r} = \int_0^a F_1(t, 0) dt + \int_0^a \left[ -F_1(a - t, \frac{b}{a}t) + F_2(a - t, \frac{b}{a}t)\frac{b}{a} \right] dt - \int_0^b F_2(0, b - t) dt. $$

Confrontando i due lati, si verifica che il teorema di Green è soddisfatto per il triangolo rettangolo $\Delta$.

Inoltre, per un trinagolo qualsiasi, possiamo sempre tracciare un'altezza e suddisviderlo in triangoli rettangoli, su cui abbiamo appena dimostrato che vale il teorema di Green. Per un qualsiasi poligono possiamo suddividerlo in triangoli su cui vale il teorema. In generale su una qualsiasi superficie racchiusa da curve chiuse possiamo sempre suddividere in triangoli e applicare il teorema di Green.\\
\end{proof}

Il teorema si può applicare in vari modi ad esempio:
\begin{itemize}
\item Calcolare la circuitazione di un campo lungo una curva chiusa.
\item Confrontare tra loro due circuitazioni (specialmente quando $\vec F$ è irrotazionale ma non conservativo).
\end{itemize}

Inoltre grazie al teorema si ricava un utile fatto generale:
\begin{corollario}{}
  Se ho due curve chiuse $\gamma_1$ e $\gamma_2$ tali che $\gamma_1$ è interna a $\gamma_2$ allora: $$\int_{\gamma_1} \vec F \cdot d\vec r = \int_{\gamma_2} \vec F \cdot d\vec r$$\\
  per ogni campo $\vec F$ irrotazionale (nella sezione di piano che contiene le curve).
\end{corollario}

Applichiamo ora il teorema al calcolo dell'area racchiusa tra curve chiuse.\\
Se scelgo un campo $\vec F$ tale che $\frac{\partial F_2}{\partial x}-\frac{\partial F_1}{\partial y}=1$ (ad esempio $\vec F = (0,x)$)allora ottengo l'area di $\Omega$ come circuitazione di $\vec F$ lungo il bordo di $\Omega$.\\
Riassumendo:
\begin{teorema}{}
  Se $\gamma(t)= (x(t), y(t)), t \in [a,b]$ è una curva chiusa semplice e regolare, allora l'area racchiusa da $\gamma$ è data da:
  $$\text{Area}(\Omega) = \int_{a}^{b} x(t)\cdot y'(t) dt$$.\\
  Sono possibili anche altre scelte per $\vec F$ che generano formule analoghe per il calcolo dell'area di $\Omega$.
\end{teorema}

\chapter{Integrali curvilinei e di superficie}
	\section{Definizione delle curve}
	\deff
	Una funzione $\gamma : [ a, b ] \subseteq \R \Rightarrow \R^m$ si dice \textbf{curva} in $\R^m$ se è continua.\\
	\\
	L'insieme $\gamma ([a,b])= \{\gamma(t)\in \R^m : t \in [a,b] \}$ si dice sostegno di $\gamma$, cioè il sostegno è l'immagine di $\gamma$
	\deff
	$\gamma:[a,b]\Rightarrow \R^m$, curva, si dice \textbf{semplice} se vale l'implicazione:
	$$t \in [a, b],\ s \in (a, b)\ \text{e} \ s\ne t \Rightarrow \gamma(t)\ne \gamma(s)$$
	$\gamma:[a,b]\Rightarrow \R^m$, curva, si dice \textbf{chiusa} se vale l'implicazione:
	$$\gamma(a)=\gamma(b)$$
	Se $\gamma:[a,b]\Rightarrow \R^m$ è una curva chiusa e semplice si dice \textbf{curva di Jordan}
	\deff
	Sia $\gamma : [a, b] \Rightarrow \R^m$ curva con $\gamma(t) = (\gamma_1(t),\dots, \gamma_m(t))$, $\gamma$ si dice \textbf{regolare} se $\gamma \in C^1([a, b])$ e se $\gamma'(t)\ne \vec{0}, \ \forall t \in [a, b]$.\\
	\\
	$\gamma'(t)$ si dice \textbf{vettore tangente} a $\gamma$ in $\gamma(t)$, oppure vettore derivata o vettore velocità.\\
	\\
	Se $\gamma'(t)\ne \vec 0 \  \forall t$ significa che $\gamma(t)$ non si fermerà mai.
	\deff
	$\gamma:[a,b]\rightarrow \R^m$, curva, si dice \textbf{regolare a tratti} se $\exists a=t_0<t_1<\dots<t_{n-1}<t_{n}=b$ tale che $\gamma \in C^1([t_{k-1}, t_k]) \ \forall k \in {1, \dots, n}$ e $\gamma'(t)\ne \vec 0 \ \forall t \ne t_k$
	\section{Integrali curvilinei di prima specie (integrali di funzione su curve)}
	\deff
	Sia $\gamma:[a, b] \rightarrow \R^m$ curva regolare a tratti, sia $f:D\subseteq \R^m \rightarrow \R$ tale che $\gamma([a,b])\subseteq D$, si pone:
	$$\int_\gamma f ds = \int_\gamma \gamma f = \int_a^b f(\gamma(t))|\gamma'(t)|dt$$
	se l'integrale esiste.\\
	\\
	Se $\gamma$ è di Jordan si scrive $\oint_\gamma f ds = \int_\gamma fds$.\\
	\\
	Si pone $L(\gamma)=\int_\gamma ds=\int_{a}^{b}|\gamma'(t)|dt$ lunghezza di $\gamma$.
	\prop
	Sia $\gamma:[a, b] \rightarrow \R^m$ curva regolare a tratti e sia $\phi : [c, d] \rightarrow [a, b] \ C^1$ strettamente monotone e suriettiva e sia $\tilde{\gamma} : [c, d]\rightarrow \R^m$ definita come $\tilde{\gamma}=\gamma ( \phi (r) )$, ovvero  $\phi$ è un riscalamento del parametro t e $\tilde{\gamma}$ ha la stessa traiettoria di $\gamma$ ma la percorre in modo diverso, allora:
	$$\int_{\tilde{\gamma}} fds=\int_\gamma fds$$
	Quindi gli integrali di prima specie non dipendo né dalla parametrizzazione $\gamma$, né dal verso di percorrenza, ma dipendono solo dal sostegno $\Gamma=\gamma([a,b])$
	\deff
	Sia $\Gamma \subseteq \R^m$ che si sostegno si una curva $\gammadef$ regolare a tratti semplice, allora si pone:
	$$\int_\Gamma fds= \int_\gamma fds$$
	Allo stesso modo $L(\Gamma)=L(\gamma)$
	\deff
	Sia $\gamma :  [a,b]\rightarrow \R ^3$ curva regolare a tratti semplice, alla quale è associata una \textbf{funzione di densità} (di massa) $\mu:  \Gamma =\gamma([a,b])\subseteq \R^3 \rightarrow \R$, si dice \textbf{massa} di $\gamma$ (o di $\Gamma$) il numero:
	$$M(\Gamma)=m(\Gamma)=\int_\Gamma \mu \ d\sigma$$
	Il \textbf{baricentro} di $\Gamma$ è il punto $(x_G, y_G, z_G)$ tale che:
	$$x_G=\frac{1}{M(\Gamma)}\int_\Gamma x \mu (x,y,z) d \sigma$$
	$$y_G=\frac{1}{M(\Gamma)}\int_\Gamma y \mu (x,y,z) d \sigma$$
	$$z_G=\frac{1}{M(\Gamma)}\int_\Gamma z \mu (x,y,z) d \sigma$$
	\section{Integrali curvilinei di seconda specie (lavori)}
	Sia $\gammadef$ curva regolare a tratti e  sia $F:D\inc \R^m \rightarrow \R^m$ con $\gamma([a,b])\inc D$, si dice \textbf{lavoro di} $\vec F$ \textbf{lungo} $\gamma$ (o integrale di seconda specie) il numero:
	$$\int_\gamma F \cdot dl = \int_a^bF(\gamma(t))\cdot \gamma'(t) dt$$
	Se $\gamma$ è di Jordan si scrive $\oint_\gamma F \cdot dl = \int_\gamma F \cdot dl$ detto \textbf{circuitazione di $F$ lungo $\gamma$}.\\
	\\
	Sia $\vec r(t)= \frac{\gamma'(t)}{|\gamma'(t)|}\ \forall t \in [a,b]$ detto \textbf{versore tangente a} $\gamma$, si ha:
	$$\int_\gamma F \cdot dl = \int_\gamma (F\cdot \vec r) d\sigma$$
	\prop
	Sia $\gamma:[a, b] \rightarrow \R^m$ curva regolare a tratti e sia $\phi : [c, d] \rightarrow [a, b] \ C^1$ strettamente crescente e suriettiva e sia $\tilde{\gamma} : [c, d]\rightarrow \R^m$ definita come $\tilde{\gamma}=\gamma ( \phi (r) )$, ovvero $\tilde{\gamma}$ percorre il sostegno di $\gamma$ lo stesso numero di volte nello stesso verso, allora:
	$$\int_{\tilde{\gamma}} F \cdot dl = \int_\gamma F \cdot dl$$
	Nel caso in cui $\hat \gamma$, definita come $\tilde{\gamma}$, sia strettamente decrescente, e quindi percorra il sostegno di $\gamma$ lo stesso numero di volte ma in verso opposto, allora:
	$$\int_{\hat{\gamma}} F \cdot dl = - \int_\gamma F \cdot dl$$
	\section{Integrali di superficie}
	\deffname{Superfici parametrizzate}
	Una funzione $\sigma:\bar{A} \inc \R^2 \rightarrow \R^3$ con $A \inc \R^2$ aperto e con $\sigma(u,v)=(\sigma_1(u,v),\sigma_2(u,v),\sigma_3(u,v))$ si dice \textbf{superficie parametrizzata} se $\sigma \in C^1$, $\sigma$ è iniettiva in $A$ e la matrice Jacobiana ha rango 2, allora:
	$$\Sigma=\sigma(\bar{A})$$
	ed è detto \textbf{sostegno} di $\sigma$.
	\deff
	Le derivate parziali di $\sigma$ in $u$ e $v$ generano in piano in $\R^3$ ,detto \textbf{piano tangente} a $\Sigma$ in $\sigma(u_0,v0)$, che ha equazione:
	$$\Pi(u,v)=\sigma(u_0,v_0)+\frac{\partial \sigma}{\partial u}(u_0,v_0)(u-u_0)+\frac{\partial \sigma}{\partial v}(u_0,v_0)(v-v_0)$$
	Definiamo il \textbf{vettore normale} a $\Sigma$ in $\sigma(u_0,v0)$:
	$$\vec N(u_0, v_0)=\frac{\partial \sigma}{\partial u} \times \frac{\partial \sigma}{\partial v}$$
	da cui possiamo ricavare il versore $\vec n (u_0,v_0) = \frac{\vec N (u_0,v_0)}{|\vec N (u_0,v_0)|}$.
	\deffname{Superfici cartesiane}
	Sia $g:\bar A \inc \R^2 \rightarrow \R$, con $A$ aperto e $g \in C^1(A)$, data una superficie $\sigma(u,v)=(u,v,g(u,v))$ si ha che $\Sigma = \sigma (\bar A)$ è il grafico di $g$. Sappiamo quindi che:
	$$\Sigma  = \sigma(\bar A)=Gr(g)=\{(x,y,z)\in \R^3:(x,y)\in \bar A , z = g(x,y)\}$$
	$$\vec N(x, y)=\frac{\partial \sigma}{\partial u} \times \frac{\partial \sigma}{\partial v}=\left(-\frac{\partial g}{\partial x}, -\frac{\partial g}{\partial y}, 1 \right)$$
	$$|\vec N (x,y)|=\sqrt{1+|\nabla g|^2}$$
	\deffname{Integrali di superficie di prima specie}
	Sia $\sigma:\bar{A} \inc \R^2 \rightarrow \R^3$ superficie con $A$ misurabile, poniamo:
	$$\int_\sigma f d\sigma = \int_\sigma f(x,y,z)d\sigma =\int_{\bar A} f(x,y,g(x,y))|\vec N (x,y)|dxdy$$
	\prop
	L'integrale è indipendente dalla parametrizzazione, infatti se due superfici sono diverse ma hanno stesso sostegno l'integrale non cambia.
	\section{Flusso di un campo vettoriale}
	\deffname{Superfici orientabili}
	Una superficie $\sigma:\bar{A} \inc \R^2 \rightarrow \R^3$ si dice \textbf{orientabile} se la funzione $\Sigma \rightarrow \R^3$ è continua.
	\oss
	Se due superfici orientabili con lo stesso sostegno allora i versori normali possono essere solo uguali od opposti tra loro.
	\prop
	Se $\Sigma = \sigma (\bar A)$ è il sostegno di $\sigma:\bar{A} \inc \R^2 \rightarrow \R^3$ e se anche $\Sigma=\partial \Omega$ con $\Omega$ aperto, connesso e limitato, allora $\Sigma$ è orientabile. \\
	\\
	In particola re $\vec n$ punta verso l'interno di $\Omega$ allora viene detto \textbf{entrante}, altrimenti, se punta verso l'esterno, viene detto \textbf{uscente}. Il verso positivo del vettore è quello uscente dalla superficie.
	\deffname{Flusso di un campo vettoriale}
	Sia $\sigma:\bar{A} \inc \R^2 \rightarrow \R^3$ superficie orientabile e sia $F:D\inc \R^3 \rightarrow \R^3$ campo vettoriale continuo con $\Sigma = \sigma (\bar A)\inc D$, allora il \textbf{flusso} di $\vec F$ attraverso $\sigma$ nella direzione $\vec n$ è:
	$$\int_\sigma (\vec F \cdot \vec n) d\sigma = \int_{\bar A}F(\sigma(u,v))\cdot \frac{\vec N(u,v)}{|\vec N(u,v)|}|\vec N(u,v)|dudv=\int_D F(x,y,g(x,y))\cdot \vec N(x,y)dxdy$$
	\deff
	Sia $\Sigma$ una superficie orientabile, se $\Sigma=\partial \Omega$ con $\Omega$ aperto, connesso, limitato e misurabile allora:
	$$\int_{\partial \Omega} \vec F \cdot \\vec n$$
	per convenzione denota il \textbf{flusso uscente} da $\Omega$ , cioè il flusso di $F$ lungo $\vec n$ uscente.
	\thhname{Th. della divergenza di Gauss}
	Sia $\Omega \inc \R^3$ connesso, limitato e misurabile, tale che sia $\Sigma=\partial \Omega$ una superficie orientata con $\vec n$ uscente, sia $\vec F:D \inc \R^3 \rightarrow \R^3$, allora:
	$$\int_{\partial \Omega}(\vec F \cdot \vec n)d\sigma=\int_\Omega div \vec F dxdydz$$
	\deffname{Aperto con bordo}
	Sia $D \inc \R^2$ aperto, connesso e limitato. $D$ si dice \textbf{aperto con bordo} se $\partial D$ è l'unione di un numero finito di sostegni di curve di Jordan regolari a tratti a due a due disgiunti. Su $\partial D$ si definisce come orientazione positiva quella per cui percorrendo $\partial D$ vedo $D$ a sinistra.
	\thhname{Formula di Green nel piano}
	Sia $\vec F:E\ inc \R^2 \rightarrow \R^2$ di classe $C^1$ con $E$ aperto, sia $D\inc \R^2$ aperto con bordo con $\bar D \inc E$, allora:
	$$\int_{\partial D} \vec F \cdot dl = \int_D \left(\frac{\partial F_2}{\partial x}-\frac{\partial F_1}{\partial y}\right)dxdy$$
	\thhname{Th. del rotore di Stokes}
	Sia $D \inc \R^2$ aperto con bordo e sia $\sigma : \bar D \inc \R^2 \rightarrow \R^3$ superficie orientabile iniettiva si $\bar D$, chiamiamo $\partial \sigma = \sigma (\partial D)$, l'immagine di $\partial D$ tramite $\sigma$, \textbf{frontiera della superficie} $\sigma$. Orientiamo $\partial \sigma$ con l'orientazione indotta dall'orientazione positiva di $\partial D$, ovvero quando il versore normale $\vec n$ percorre $\partial \sigma$ vede $\sigma$ a sinistra. Allora:
	$$\int_{\partial\sigma} \vec F \cdot dl = \int_\sigma \left(rot \vec F \cdot \vec n\right)d\sigma$$
	ovvero il lavoro di $F$ lungo $\partial \sigma$ è uguale al flusso del rotore di $\vec F$ attraverso $\sigma$.
	\section{Campi conservativi}
	\deffname{Campi conservativi}
	Un campo $\vec F: \Omega \inc \R^n \rightarrow \R^n$ continuo con $\Omega$ aperto si dice \textbf{conservativo} se $\exists \ \Phi : \Omega \inc \R^n \rightarrow \R$ tale che $\nabla \Phi = \vec F$, in tal caso $\Phi$ si dice \textbf{potenziale} di $\vec F$, oppure primitiva di $\vec F$.
	\prop
	Sia $\gamma:[a,b]\rightarrow \Omega \inc \R^n$ curva regolare a tratti, se $F$ è \textbf{conservativo} si ha che il lavoro lungo $\gamma$ è:
	$$\int_\gamma \vec F \cdot dl = \int_a^b \vec F(\gamma (t))\cdot \gamma'(t)dt=\int_{a}^{b} \nabla \Phi (\gamma(t))\cdot \gamma'(t)dt=\Phi(\gamma(b))-\Phi(\gamma(a))$$
	ovvero se $F$ è conservativo ($F\nabla \Phi$) il lavoro lungo una curva $\gamma$ dipende sola dal punto iniziale e da quello finale.
	\thh
	Sia $\vec F: \Omega \inc \R^n \rightarrow \R^n$ conservativo ($F\nabla \Phi$), allora per ogni curva $\gamma:[a,b]\rightarrow \Omega \inc \R^3$ si ha:
	$$\int_\gamma \vec F \cdot dl =\Phi(\gamma(b))-\Phi(\gamma(a))$$
	Se $\gamma_1$ e $\gamma_2$ sono due curve con stesso punto iniziale e finale:
	$$\int_{\gamma_2} \vec F \cdot dl = \int_{\gamma_1} \vec F \cdot dl$$
	Se $gamma$ è chiusa allora:
	$$\int_\gamma \vec F \cdot dl =0$$
	\thh
	Sia $\vec F: \Omega \inc \R^n \rightarrow \R^n$ continuo con $\Omega$ connesso, allora le tre affermazioni sono equivalenti:
	$$\begin{Bmatrix}
		F \text{ è conservativo}\\
		  \Updownarrow\\
		 \int_{\gamma_2} \vec F \cdot dl = \int_{\gamma_1} \vec F \cdot dl\\
		 (\text{con } \gamma_1,\gamma_2 \text{ curve con stesso inizio e fine})\\
		 \Updownarrow\\
		 \int_\gamma \vec F \cdot dl =0\\
		 (\text{con } \gamma \text{ chiusa})
	\end{Bmatrix}$$
	\thh
	Sia $F: \Omega \subseteq \mathbb{R}^n \rightarrow \mathbb{R}^n$ di classe $C^1$ ($\Sigma$ aperto), se $F$ è conservativo allora:
	$$\frac{\partial F_i}{\partial x_j} = \frac{\partial F_j}{\partial x_i} \qquad \forall i, j \in \{ 1,\dots, n\} \text{ in } \Omega$$
	\deff
	Se $rotF=0$, $F$ si dice \textbf{irrotazionale}.
	\prop
	Se $F$ è conservativo e di classe $C^1$ $\Rightarrow$ $F$ è irrotazionale.\\
	\\
	In generale la $\Rightarrow$ non si può invertire.
	\deff
	Un aperto connesso $\Omega$ si dice \textbf{semplicemente connesso} se ogni curva $\gamma$ di Jordan può essere deformata con continuità fino a contrarsi ad un punto, rimanendo sempre dentro $\Omega$.\\
	\\
	In $\mathbb{R}^2$ un insieme semplicemente connesso è un insieme "privo di buchi". Sono esempi di insiemi semplicemente connessi: tutti gli aperti convessi, tutti gli aperti limitati con frontiera costituita da un'unica curva e il piano privato di una semiretta. Sono esempi di insiemi non semplicemente connessi: tutti gli aperti privati di un punto, le corone circolari e il piano privato di una retta (perché non è connesso).\\
	\\
	In $\mathbb{R}^3$ sono esempi di insiemi semplicemente connessi: tutti gli aperti convessi, lo spazio privato di un punto e le corone sferiche. Sono esempi di insiemi non semplicemente connessi: lo spazio privato di una retta o di un piano.
	\thh
	Sia $\Omega \subseteq \mathbb{R}^3$ un aperto semplicemente connesso, e sia $F$ un campo
	vettoriale di classe $C^1 $, se $rotF=0$ (irrotazionale) allora $F$ è conservativo.

\chapter{Gauss-Green}

Questo breve capitolo è interamente dedicato al teorema di Gauss-Green anche noto come teorema della divergenza.\\
Diamo intanto la definizione di divergenza di un campo vettoriale.
\begin{definizione}{Divergenza di un campo vettoriale}
  Sia $\vec{F} \in C^1(\Omega, \R^3)$ un campo vettoriale definito su un insieme aperto $\Omega \inc \R^3$. La divergenza di $\vec{F}$ è la funzione $\nabla \cdot \vec{F} : \Omega \to \R$ definita come:
  $$
    \nabla \cdot \vec{F} = \frac{\partial F_1}{\partial x} + \frac{\partial F_2}{\partial y} + \frac{\partial F_3}{\partial z}
  $$
\end{definizione}
Ora possiamo enunciare il teorema di Gauss-Green.
\begin{teorema}{Teorema di Gauss-Green}
 Sia $\Omega \inc \R^3$ un solido (aperto e limitato) la cui frontiera $\partial \Omega$ è una superficie regolare orientata con la scelta della normale uscente. Sia $\vec{F} \in C^1(\Omega, \R^3)$ un campo vettoriale. Allora vale la seguente formula:
  $$
    \int_{\Omega} \nabla \cdot \vec{F} \, dV = \int_{\partial \Omega} \vec{F} \cdot \vec{n} \, dS$$
  dove $\nabla \cdot \vec{F}$ è la divergenza di $\vec{F}$ e $\vec{n}$ è il vettore normale alla superficie $\partial \Omega$.
\end{teorema}

Vediamone alcune applicazioni.
\begin{itemize}
\item Se ad esempio $\vec{F} = (0, 0, z)$, allora $\nabla \cdot \vec{F} = 1$ e il teorema di Gauss-Green diventa:
  $$
    \int_{\Omega} 1 \, dV = \int_{\partial \Omega} z \, dS
  $$
  cioè l'integrale del volume di $\Omega$ è uguale all'integrale della funzione $z$ sulla superficie $\partial \Omega$.
\item Possiamo usarlo per calcolare il volume di una sfera considerando un campo vettoriale $F(x,y,z)=(x,y,z)$: la divergenza di $F$ è $3$ e quindi:
  $$
    \int_{\Omega} 3 \, dV = 3 \cdot \text{Volume}(\Omega) = \int_{\partial \Omega} \vec{F} \cdot \vec{n} \, dS = \int_{\partial \Omega} (x,y,z) \cdot \vec{n} \, dS
  $$
  ma $(x,y,z) \cdot \vec{n} = r$ e quindi:
  $$
    3 \cdot \text{Volume}(\Omega) = \int_{\partial \Omega} r \, dS
  $$
  e quindi:
  $$
    \text{Volume}(\Omega) = \frac{1}{3} \int_{\partial \Omega} r \, dS
  $$
  Per una sfera di raggio $R$, la superficie $\partial \Omega$ è una sfera di raggio $R$ e quindi:
  $$
    \text{Volume}(\Omega) = \frac{1}{3} \int_{\partial \Omega} r \, dS = \frac{1}{3} \int_{\partial \Omega} R \, dS = \frac{R}{3} \int_{\partial \Omega} dS
  $$
  L'integrale $\int_{\partial \Omega} dS$ è semplicemente l'area della superficie della sfera, che è $4 \pi R^2$. Quindi:
  $$
    \text{Volume}(\Omega) = \frac{R}{3} \cdot 4 \pi R^2 = \frac{4 \pi R^3}{3}
  $$
  che è la ben nota formula per il volume di una sfera di raggio $R$.
\end{itemize}

In generale nei casi in cui la divergenza del campo ha un'espressione semplice, il teorema di Gauss-Green può essere molto utile per calcolare integrali di volume.



\chapter{Stokes}
\section{Teorema di Stokes}
In quest'altro breve capitolo vedremo il teorema di Stokes, che è un'estensione del teorema di Green a campi vettoriali in tre dimensioni.

Il teorema di Stokes permette di legare tra loro il flusso del rotore di un campo vettoriale, attraverso una superficie, la circuitazione del campo stesso lungo il bordo della superficie.

Vediamo innanzitutto una definizione preliminare:\\


\begin{definizione}{Compatibilità delle orientazioni}
  Dati una superficie orientata $ S $ e il suo bordo orientato $\partial S $, diciamo che le due orientazioni sono compatibili se percorrendo $\gamma$ dal lato di $N$, $S$ risulta essere a sinistra.
\end{definizione}

Ora possiamo enunciare il teorema di Stokes:
\begin{teorema}{Teorema di Stokes}
    Sia $ S $ una superficie orientata con bordo $ \partial S $ orientato compatibilmente. Sia $ \vec{F} $ un campo vettoriale di classe $ C^1 $ in un intorno aperto di $ S $. Allora vale la seguente formula:
    \[
        \int_S \nabla \times \vec{F} \cdot \hat{N} \, dS = \int_{\partial S} \vec{F} \cdot d\vec{r}
    \]
\end{teorema}

Un'importante applicazione del teorema di Stokes è la seguente:\\
Se $S_1$ e $S_2$ sono due superfici orientate con bordo $\partial S_1=\partial S_2 =\gamma$ orientati compatibilmente, allora:
\[
    \int_{\gamma} \vec{F} \cdot d\vec{r}=
\int_{S_1} \nabla \times \vec{F} \cdot \hat{N} \, dS + \int_{S_2} \nabla \times \vec{F} \cdot \hat{N} \, dS
\]

\subsection{Esempio di applicazione del teorema di Stokes}

Consideriamo il campo vettoriale $\vec{F} = (y, -x, z)$ e la superficie $S$ definita dal paraboloide $z = 1 - x^2 - y^2$ con $z \geq 0$. Calcoliamo il flusso del rotore di $\vec{F}$ attraverso $S$ e la circuitazione di $\vec{F}$ lungo il bordo di $S$.

Innanzitutto, calcoliamo il rotore di $\vec{F}$:
\[
\nabla \times \vec{F} = \left( \frac{\partial z}{\partial y} - \frac{\partial (-x)}{\partial z}, \frac{\partial x}{\partial z} - \frac{\partial z}{\partial x}, \frac{\partial (-x)}{\partial y} - \frac{\partial y}{\partial x} \right) = (1, 0, -2)
\]

La normale unitaria alla superficie $S$ è data da:
\[
\hat{N} = \frac{\nabla (1 - x^2 - y^2 - z)}{|\nabla (1 - x^2 - y^2 - z)|} = \frac{(-2x, -2y, -1)}{\sqrt{4x^2 + 4y^2 + 1}}
\]

Il flusso del rotore di $\vec{F}$ attraverso $S$ è quindi:
\[
\int_S \nabla \times \vec{F} \cdot \hat{N} \, dS = \int_S (1, 0, -2) \cdot \frac{(-2x, -2y, -1)}{\sqrt{4x^2 + 4y^2 + 1}} \, dS = \int_S \frac{-2x + 2}{\sqrt{4x^2 + 4y^2 + 1}} \, dS
\]

Passiamo alle coordinate polari, dove $x = r \cos \theta$ e $y = r \sin \theta$. La superficie $S$ è definita da $z = 1 - r^2$ con $0 \leq r \leq 1$ e $0 \leq \theta < 2\pi$. Il differenziale di superficie in coordinate polari è $dS = \sqrt{1 + (\frac{\partial z}{\partial r})^2} \, r \, dr \, d\theta = \sqrt{1 + 4r^2} \, r \, dr \, d\theta$.

Quindi l'integrale diventa:
\[
\int_S \frac{-2r \cos \theta + 2}{\sqrt{4r^2 + 4r^2 + 1}} \sqrt{1 + 4r^2} \, r \, dr \, d\theta = \int_0^{2\pi} \int_0^1 \frac{-2r \cos \theta + 2}{\sqrt{4r^2 + 1}} \sqrt{4r^2 + 1} \, r \, dr \, d\theta
\]

Semplificando, otteniamo:
\[
\int_0^{2\pi} \int_0^1 (-2r^2 \cos \theta + 2r) \, dr \, d\theta
\]

Separiamo gli integrali:
\[
\int_0^{2\pi} \int_0^1 -2r^2 \cos \theta \, dr \, d\theta + \int_0^{2\pi} \int_0^1 2r \, dr \, d\theta
\]

Il primo integrale è nullo perché l'integrale di $\cos \theta`$ su $[0, 2\pi]$ è zero:
\[
\int_0^{2\pi} \cos \theta \, d\theta = 0
\]

Quindi rimane:
\[
\int_0^{2\pi} \int_0^1 2r \, dr \, d\theta = 2 \int_0^{2\pi} d\theta \int_0^1 r \, dr = 2 \cdot 2\pi \cdot \left[ \frac{r^2}{2} \right]_0^1 = 2 \cdot 2\pi \cdot \frac{1}{2} = 2\pi
\]

Abbiamo quindi:
\[
\int_S \nabla \times \vec{F} \cdot \hat{N} \, dS = 2\pi
\]

Calcoliamo ora la circuitazione di $\vec{F}$ lungo il bordo di $S$. Il bordo di $S$ è la circonferenza $x^2 + y^2 = 1$ nel piano $z = 0$. Parametrizziamo il bordo come $\vec{r}(t) = (\cos t, \sin t, 0)$ con $t \in [0, 2\pi]$. La circuitazione di $\vec{F}$ lungo il bordo è:
\[
\int_{\partial S} \vec{F} \cdot d\vec{r} = \int_0^{2\pi} \vec{F}(\cos t, \sin t, 0) \cdot \frac{d\vec{r}}{dt} \, dt
\]
\\
\[= \int_0^{2\pi} (\sin t, -\cos t, 0) \cdot (-\sin t, \cos t, 0) \, dt = \int_0^{2\pi} (\sin^2 t + \cos^2 t) \, dt = -2\pi
\]
Tuttavia per avere una parametrizzazione coerente dovremmo considerare il flusso con la normale opposta al calcolo effettuato in precedenza così da avere un riaultato corretto.
Abbiamo quindi verificato il teorema di Stokes per questo esempio.



\chapter{Serie numeriche}

\begin{definizione}{Serie numerica}
  Data una succesione di numeri reali $\{a_n\}_{n=1}^{\infty}$, si definisce serie degli $a_n$ il simbolo: $\sum_{n=1}^{\infty} a_n$.
\end{definizione}

\begin{definizione}{Somma parziale}
  Si definisce somma parziale di una serie numerica la somma dei primi $n$ termini della serie: $S_N = \sum_{n=1}^{N} a_n$.
\end{definizione}

Si studia il $\lim_{N \to \infty} S_N$ e si valutano i diversi casi:
\begin{itemize}
  \item Se il limite esiste finito, la serie è convergente.
  \item Se il limite è infinito, la serie è divergente.
  \item Se il limite non esiste, la serie è indeterminata.
\end{itemize}

\begin{osservazione}{}
  La definizione precedente si adatta anche al caso in cui la serie non parta da $n=1$.\\
  Il comportamento di una serie numerica non cambia se modifico un numero finito di termini. Infatti il comportamento dipende soltanto dalle "code".
\end{osservazione}

Ecco un'altra osservazione importante:
\begin{osservazione}{}
  Se le due serie $\sum_{n=1}^{\infty} a_n$ e $\sum_{n=1}^{\infty} b_n$ sono convergenti, allora la serie $\sum_{n=1}^{\infty} (a_n + b_n)$ è convergente.\\ Se invece una delle due serie è divergente, allora la serie $\sum_{n=1}^{\infty} (a_n + b_n)$ è divergente.\\
  Infine si ha che se $\sum_{n=1}^{\infty} a_n$ è convergente, allora per ogni $\lambda \in \R$, la serie $\sum_{n=1}^{\infty} \lambda a_n$ è convergente e vale $\sum_{n=1}^{\infty} \lambda a_n = \lambda \sum_{n=1}^{\infty} a_n$.
\end{osservazione}

Vediamo un caso particolare in cui è facile determinare oltre al comportamento anche il valore della somma della serie:
\begin{definizione}{Serie geometrica}
  Una serie del tipo $\sum_{n=0}^{\infty} q^n$ è detta serie geometrica.\\
\end{definizione}
\begin{teorema}{}
  Si ha che:
  \begin{itemize}
    \item Se $|q| < 1$, allora la serie è convergente e vale $\sum_{n=0}^{\infty} q^n = \frac{1}{1-q}$.
    \item Se $|q| \geq 1$, allora la serie è divergente.
  \end{itemize}
\end{teorema}
Vediamo come generalizzarlo per una serie che non parte da $n=0$:
\begin{osservazione}{}
  Se $|q| < 1$, allora la serie $\sum_{n=k}^{\infty} q^n$ è convergente e vale $\sum_{n=k}^{\infty} q^n = \frac{q^k}{1-q}$.
\end{osservazione}

\section{Criteri di convergenza}
Vediamo ora alcuni importanti criteri che ci permettono di determinare il comportamento di una serie numerica.
\begin{teorema}{Condizione necessaria di convergenza}
  Se la serie $\sum_{n=1}^{\infty} a_n$ è convergente, allora $\lim_{n \to \infty} a_n = 0$.
\end{teorema}

In generale è più facile studiare serie con tutti i termini positivi:
\begin{definizione}{Serie a termini positivi}
  Una serie $\sum_{n=1}^{\infty} a_n$ si dice a termini positivi se $a_n \geq 0$ per ogni $n$.
\end{definizione}

Vediamo subito un teorema che risolve molte patologie:
\begin{teorema}{}
  Una serie a termini positivi può essere convergente, divergente ma non indeterminata.\\
\end{teorema}
Inoltre sulle serie a termini positivi valgono i seguenti criteri:
\begin{teorema}{Criterio del confronto integrale}
  Sia $f:[1, +\infty) \to \R$ una funzione continua e a valori positivi. Se $a_n = f(n)$, allora la serie $\sum_{n=1}^{\infty} a_n$ è convergente se e solo se l'integrale $\int_{1}^{+\infty} f(x) \, dx$ è convergente.
\end{teorema}
Vediamo un esempio.\\

Studiamo la serie $\sum_{n=1}^{\infty} \frac{1}{n}$.\\
Applichiamo il criterio del confronto integrale con $f(x) = \frac{1}{x}$, ottenendo che la serie è divergente.\\

Vediamo ora un altro criterio:
\begin{teorema}{Criterio del confronto asintotico}
  \begin{itemize}
  \item  Siano $\{a_n\}_{n=1}^{\infty}$ e $\{b_n\}_{n=1}^{\infty}$ due successioni di numeri reali tali che $0 \leq a_n \leq b_n$ per ogni $n$. Se la serie $\sum_{n=1}^{\infty} b_n$ è convergente, allora la serie $\sum_{n=1}^{\infty} a_n$ è convergente.\\
  Se la serie $\sum_{n=1}^{\infty} a_n$ è divergente, allora la serie $\sum_{n=1}^{\infty} b_n$ è divergente.
  \item Siano $\{a_n\}_{n=1}^{\infty}$ e $\{b_n\}_{n=1}^{\infty}$ due successioni di numeri reali tali che $\lim_{n \to \infty} \frac{a_n}{b_n} = L$ con $0 < L < \infty$. Allora le due serie $\sum_{n=1}^{\infty} a_n$ e $\sum_{n=1}^{\infty} b_n$ sono entrambe convergenti o entrambe divergenti.
  \item Siano $\{a_n\}_{n=1}^{\infty}$ e $\{b_n\}_{n=1}^{\infty}$ due successioni di numeri reali tali che $a_n = o(b_n)$ per $n \to \infty$. Se la serie $\sum_{n=1}^{\infty} b_n$ è convergente, allora la serie $\sum_{n=1}^{\infty} a_n$ è convergente. Se la serie $\sum_{n=1}^{\infty} a_n$ è divergente, allora la serie $\sum_{n=1}^{\infty} b_n$ è divergente.
  \end{itemize}
\end{teorema}

Esistono inoltre altri due importanti criteri:
\begin{teorema}{Criterio della radice}
  Sia $\{a_n\}_{n=1}^{\infty}$ una successione di numeri reali. Se esiste il limite $\lim_{n \to \infty} \sqrt[n]{|a_n|} = L$, allora:
  \begin{itemize}
    \item Se $L < 1$, la serie $\sum_{n=1}^{\infty} a_n$ è convergente.
    \item Se $L > 1$, la serie $\sum_{n=1}^{\infty} a_n$ è divergente.
    \item Se $L = 1$, il criterio non fornisce informazioni.
  \end{itemize}
\end{teorema}
\begin{teorema}{Criterio del rapporto}
  Sia $\{a_n\}_{n=1}^{\infty}$ una successione di numeri reali. Se esiste il limite $\lim_{n \to \infty} \left| \frac{a_{n+1}}{a_n} \right| = L$, allora:
  \begin{itemize}
    \item Se $L < 1$, la serie $\sum_{n=1}^{\infty} a_n$ è convergente.
    \item Se $L > 1$, la serie $\sum_{n=1}^{\infty} a_n$ è divergente.
    \item Se $L = 1$, il criterio non fornisce informazioni.
  \end{itemize}
\end{teorema}

Inoltre ecco la seguente osservazione:
\begin{osservazione}{}
  Si può dimostrare che se esiste il limite $\lim_{n \to \infty} \left| \frac{a_{n+1}}{a_n} \right| = L$ esiste anche il limite $\lim_{n \to \infty} \sqrt[n]{|a_n|} = L$.
\end{osservazione}

In alcuni casi può essere utile la formula di Stirling per il fattoriale:
\begin{teorema}{Teorema di Stirling}
  Si ha che $\lim_{n \to \infty} \frac{n!}{(\frac{n}{e})^n \sqrt{2\pi n}} = 1$.
\end{teorema}
Vediamo un esempio in cui risulta utile.\\
Studiamo la serie $\sum_{n=1}^{\infty} \frac{n!}{n^n}$.

Applichiamo il criterio della radice:
\[
\lim_{n \to \infty} \sqrt[n]{\left| \frac{n!}{n^n} \right|} = \lim_{n \to \infty} \frac{\sqrt[n]{n!}}{n}.
\]

Utilizziamo la formula di Stirling per approssimare $n!$:
\[
n! \sim \sqrt{2\pi n} \left( \frac{n}{e} \right)^n.
\]

Quindi:
\[
\sqrt[n]{n!} \sim \sqrt[n]{ \sqrt{2\pi n} \left( \frac{n}{e} \right)^n} = \cdot \sqrt[n]{\sqrt{2\pi n}} \cdot \frac{n}{e}.
\]

Poiché $\sqrt[n]{\sqrt{2\pi n}} \to 1$ per $n \to \infty$, abbiamo:
\[
\lim_{n \to \infty} \frac{\sqrt[n]{n!}}{n} = \lim_{n \to \infty} \frac{ 1 \cdot \frac{n}{e}}{n} = \frac{1}{e}.
\]

Poiché $\frac{1}{e} < 1$, la serie $\sum_{n=1}^{\infty} \frac{n!}{n^n}$ è convergente.

\section{Serie con termini di segno variabile}
Se $a_n$ cambia segno infinite volte, i criteri visti finora non sono applicabili.\\
Un caso importante sono le serie a segni alterni:
\begin{definizione}{Serie a segni alterni}
  Una serie $\sum_{n=1}^{\infty} (-1)^n a_n$ è detta serie a segni alterni.
\end{definizione}
Vediamo in questo caso un utile criterio:
\begin{teorema}{Criterio di Leibniz}
  Se $\{a_n\}_{n=1}^{\infty}$ è una successione a termini positivi e decrescenti con $\lim_{n \to \infty} a_n = 0$, allora la serie $\sum_{n=1}^{\infty} (-1)^n a_n$ è convergente.
\end{teorema}
\begin{osservazione}{}
  In particolare è necessario che la serie sia infinitesima. Il fatto che sia decrescente è sufficiente ma non necessario.
\end{osservazione}
Vediamo un esempio di applicazione del criterio di Leibniz:\\

Consideriamo la serie $\sum_{n=1}^{\infty} \frac{(-1)^n}{n}$.\\
Applichiamo il criterio di Leibniz:
\begin{itemize}
  \item La successione $a_n = \frac{1}{n}$ è a termini positivi.
  \item La successione $a_n = \frac{1}{n}$ è decrescente.
  \item $\lim_{n \to \infty} \frac{1}{n} = 0$.
\end{itemize}
Quindi, per il criterio di Leibniz, la serie $\sum_{n=1}^{\infty} \frac{(-1)^n}{n}$ è convergente.\\

In un caso più generale però i segni non sono necessariamente alternati.\\
Vediamo un criterio che ci permette di studiare serie con termini di segno qualunque:
\begin{teorema}{Criterio di convergenza assoluta}
  Se la serie $\sum_{n=1}^{\infty} |a_n|$ è convergente, allora la serie $\sum_{n=1}^{\infty} a_n$ è convergente.
\end{teorema}

Vediamo un esempio di applicazione del criterio di convergenza assoluta:\\

Consideriamo la serie $\sum_{n=1}^{\infty} \frac{\sin(n)}{n^2}$.\\
Studiamo la serie $\sum_{n=1}^{\infty} \left| \frac{\sin(n)}{n^2} \right|$.\\

Poiché $|\sin(n)| \leq 1$ per ogni $n$, abbiamo:
\[
\left| \frac{\sin(n)}{n^2} \right| \leq \frac{1}{n^2}.
\]

La serie $\sum_{n=1}^{\infty} \frac{1}{n^2}$ è una serie a termini positivi e sappiamo che è convergente (serie p con $p = 2 > 1$).\\

Quindi, per il criterio di convergenza assoluta, la serie $\sum_{n=1}^{\infty} \frac{\sin(n)}{n^2}$ è convergente.





\chapter{Serie di potenze}




\chapter{Serie di Fourier}




\backmatter

\end{document}

\end{document}
