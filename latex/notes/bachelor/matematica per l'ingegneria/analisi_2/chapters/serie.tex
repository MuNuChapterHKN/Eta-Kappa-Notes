\chapter{Serie numeriche}
\section{Definizione delle serie numeriche}
\deffname{Successioni in $\C$}
Si dice che che una \textbf{successione} $z_n$ in $\C$ \textbf{converge} a $z \in \C$ se:
$$\forall \epsilon >0\  \exists n_\epsilon \in \mathbb{N} \text{ tale che se } n>n_\epsilon \text{ allora } |z_n-z|<\epsilon$$
e si scrive come $\lim\limits_{n\rightarrow \infty}z_n = z$ oppure $z_n\rightarrow z \text{ per } n \rightarrow \infty$.
\deffname{Serie numerica}
Sia $a_n$ successione in $\C$, si dice \textbf{serie di termine generale} $a_n$ il simbolo:
$$\suminfan$$
Si dice \textbf{successione delle somme parziali} la successione:
$$\sum_{n=0}^{n}a_n=a_0+a_1+\dots+a_n$$
\deffname{Serie convergente}
Si dice che $\suminfan$ è \textbf{convergente} se:
$$\exists \lim\limits_{n \rightarrow \infty} s_n = s \in \C \ \ne \pm \infty$$
$s$ si dice \textbf{somma della serie} e vale:
$$\suminfan = s$$
In tal caso si pone $r_n=s-s_n$ \textbf{resto n-esimo}.
\deffname{Serie divergente}
Si dice che $\suminfan$, con $a_n \in \R$, è \textbf{divergente} a $\pm \infty$ se:
$$\exists \lim\limits_{n \rightarrow \infty} s_n =  \pm \infty$$
\deffname{Serie oscillante}
Nel caso in cui $a_n \in \R$ e in cui $s_n$ è oscillante, si dice che la serie $\suminfan$ è \textbf{oscillante}.
\prop
Se $\suminfan = s$ e $\suminf{0} b_n = t$ sono convergenti allora:
$$\suminf{0}(\alpha a_n + \beta b_n)=\alpha \suminfan + \beta \suminf{0} b_n=\alpha s + \beta t $$
\propname{Condizione necessaria di convergenza}
Se $\suminfan$ converge, allora:
$$\lim\limits_{n \rightarrow \infty} a_n = \lim\limits_{n \rightarrow \infty} |a_n| = 0$$
\deffname{Serie assolutamente convergenti}
La serie $\suminfan$ si dice \textbf{assolutamente convergente} se $\suminf{0}|a_n|$ è convergente.
\thh
Se $\suminf{0}|a_n|$ converge allora anche $\suminfan$ converge.
\deffname{Serie a segni alterni}
Si dice che una serie è \textbf{a segni alterni} se è nella forma $\suminf{0} (-1)^n b_n$.
\section{Metodi di risoluzione delle serie numeriche}
\propname{Serie geometrica di ragione $q$}
Data la serie $\suminf{0} q^n$ con $q \in \C$, allora se $|q|<1$ la serie converge, in particolare:
$$\suminf{0} q^n = \frac{1}{1-q} \quad \text{per } |q|<1$$
\oss
Sapendo che se due serie differiscono per un numero di termini finito hanno lo stesso carattere ma somme differenti, possiamo ricondurre serie del tipo $\suminf{k} q^n$ con $k>0$ a serie geometriche, per esempio:
$$\suminf{1} q^n = \suminf{0} q^n - q^0=\frac{1}{1-q}-1$$
\propname{Criterio integrale per le serie}
sia $f:[1, +\infty]\rightarrow \R$, $f(x)\ge 0 \ \forall x$ e $f$ decrescente, allora:
$$\suminf{1} f(n) \text{ converge } \Leftrightarrow \int_1^{+\infty} f(x)dx \text{ converge}$$
\propname{Serie armonica di esponete $p$}
Data la serie $\suminf{1} \frac{1}{n^p}$ con $p \in \R$ allora:
$$\suminf{1} \frac{1}{n^p} \text{ converge } \Leftrightarrow \ p>1$$
per il criterio integrale.
\thhname{Criterio del confronto}
Sia $0 \leq a_n \leq b_n$, allora:
$$\suminf{0} b_n \text{ converge } \Rightarrow \suminfan \text{ converge }$$
$$\suminfan \text{ diverge a } +\infty \Rightarrow \suminf{0} b_n \text{ diverge a } \infty$$
\thhname{Criterio del confronto asintotico}
Siano $a_n>0$ e $b_n > 0$ tale che $a_n \sim b_n$ per $n \rightarrow \infty$ ($\liminff \frac{a_n}{b_n}=1$), allora $\suminfan$ e $\suminf{0} b_n$ hanno lo stesso carattere.
\propname{Serie telescopiche}
Sia $\suminf{1} a_n = \suminf{1} (b_n-b_{n+1})$ con $b_n \rightarrow l$, allora:
$$\suminf{1} (b_n-b_{n+1})=b_1-l$$
\thhname{Criterio della radice}
Sia $a_n \in \C$ e $\exists \ \liminff \sqrt[n]{|a_n|}=l$, allora:
$$\suminfan \text{ è assolutamente convergente se } 0\le l< 1$$
$$\liminff a_n \ne 0 \text{ se } l>1 \Rightarrow \suminf{0}|a_n| \text{ non converge}$$
se $l=1$ non possiamo dire nulla a priori.
\thhname{Criterio del rapporto}
Sia $a_n \in \C$ e $\exists \ \liminff|\frac{a_{n+1}}{a_n}|=l$, allora:
$$\suminfan \text{ è assolutamente convergente se } 0\le l< 1$$
$$\liminff a_n \ne 0 \text{ se } l>1 \Rightarrow \suminf{0}|a_n| \text{ non converge}$$
se $l=1$ non possiamo dire nulla a priori.
\thhname{Criterio di Leibniz}
Se $\suminf{0} (-1)^n b_n$ è con $b_n \ge 0$, $\liminff b_n =0$ e $b_n$ decrescente, allora:
$$\suminf{0} (-1)^n b_n \text{ converge}$$
\section{Serie di potenze}
\deffname{Serie di potenze}
Sia $c_n \in \C$ e sia $z_0 \in \C$, si dice \textbf{serie di potenze} di centro $z_0$ e coefficiente $c_n$ la funzione:
$$\suminf{0} c_n (z-z_0)^n$$
\thh
Sia $\suminf{0} c_n (z-z_0)^n$ serie di potenze, allora $\exists \ \R \in [0, +\infty] $ tale che:
$$\suminf{0} c_n (z-z_0)^n  \text{ converge se } |z-z_0|<R \ (z \in B_R(z_0))$$
$$\suminf{0} c_n (z-z_0)^n  \text{ non converge se } |z-z_0|>R \ (z \in B_R(z_0))$$
Se $|z-z_0|=R$ non si può dire nulla a priori.\\
\\
$R$ si dice \textbf{raggio di convergenza}
\prop
Se almeno uno dei due limiti esiste:
$$\frac{1}{R}=\liminff \sqrt[n]{|c_n|}$$
$$\frac{1}{R}=\liminff |\frac{c_{n+1}}{c_n}|$$
\section{Serie di Fourier}
Vogliamo approssimare un segnale $2\pi$-periodico con un segnale sinusoidale del tipo:
$$S_n(x)=\sum_{k=0}^{n}a_k \cos{(kx)} + b_k \sin{(kx)}=a_0+\sum_{k=1}^{n}a_k \cos{(kx)} + b_k \sin{(kx)}$$
detto anche \textbf{polinomio trigonometrico}.
\deff
Data $f:\R \rightarrow \R$ $2\pi$-periodica, si pone:
$$\|f\|_2=\left( \int_0^{2\pi}|f(x)|^2 dx \right)^{\frac 12}$$
e viene detta \textbf{norma quadratica} di $f$ su un periodo.
\deff
Data $f,g:\R \rightarrow \R$ $2\pi$-periodica, si pone:
$$\|f-g\|_2=\left( \int_0^{2\pi}|f(x)-g(x)|^2 dx \right)^{\frac 12}$$
e viene detta \textbf{distanza quadratica} tra $f$ e $g$.
\deff
Definiamo i coefficienti di Fourier:
$$a_0=\frac{1}{2 \pi}\int_{0}^{2\pi}f(x)dx$$
$$a_n=\frac{1}{\pi}\int_{0}^{2\pi}f(x)\cos(nx)dx$$
$$b_n=\frac{1}{\pi}\int_{0}^{2\pi}f(x)\sin(nx)dx$$
\deffname{Serie di Fourier}
Sia $f:\R \rightarrow \R$ $2\pi$-periodica, integrabile su $[0,2\pi]$, allora:
$$s_f(x)=a_0+\suminf{1}\left(a_n \cos (nx)+ b_n \sin (nx)\right)$$
è detta \textbf{Serie di Fourier} di $f$.
\thh
Sia $f:\R \rightarrow \R$ $2\pi$-periodica, integrabile su $[0,2\pi]$, se:
$$\lim\limits_{n \rightarrow \infty} \|f(x)-s_n(x)\|_2=0$$
o equivalentemente:
$$\lim\limits_{n \rightarrow \infty} a_n=\lim\limits_{n \rightarrow \infty} b_n =0$$
allora $S_f(x)$ converge a$f(x)$ in norma quadratica.
\prop
$S_f(x)=f(x) \ \forall x$  dove $f$ è continua, ovvero $S_f(x)$ converge a $f(x)$ dove $f$ è continua.\\
\\
Se $f$ non è continua in $x$, allora $S_f(x)$ converge a:
$$S_f(x)=\frac{f(x^+)+f(x^-)}{2}$$
\oss
Sia $f:\R \rightarrow \R$ $2\pi$-periodica, integrabile su $[0,2\pi]$,  allora:
$$f \text{ pari } \Rightarrow b_n=0$$
$$f \text{ dispari } \Rightarrow a_0= a_n=0$$
