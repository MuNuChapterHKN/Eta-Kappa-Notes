\chapter{Serie numeriche}

\begin{definizione}{Serie numerica}
  Data una succesione di numeri reali $\{a_n\}_{n=1}^{\infty}$, si definisce \textbf{serie}degli $a_n$ il simbolo: $\sum_{n=1}^{\infty} a_n$.
\end{definizione}

\begin{definizione}{Somma parziale}
  Si definisce \textbf{somma parziale} di una serie numerica la somma dei primi $n$ termini della serie: $S_N = \sum_{n=1}^{N} a_n$.
\end{definizione}

Si studia il $\lim_{N \to \infty} S_N$ e si valutano i diversi casi:
\begin{itemize}
  \item Se il limite esiste finito, la serie è convergente.
  \item Se il limite è infinito, la serie è divergente.
  \item Se il limite non esiste, la serie è indeterminata.
\end{itemize}

\begin{osservazione}{}
  La definizione precedente si adatta anche al caso in cui la serie non parta da $n=1$.\\
  Il comportamento di una serie numerica non cambia se modifico un numero finito di termini. Infatti il comportamento dipende soltanto dalle "code".
\end{osservazione}

Ecco un'altra osservazione importante:
\begin{osservazione}{}
  Se le due serie $\sum_{n=1}^{\infty} a_n$ e $\sum_{n=1}^{\infty} b_n$ sono convergenti, allora la serie $\sum_{n=1}^{\infty} (a_n + b_n)$ è convergente.\\ Se invece una delle due serie è divergente, allora la serie $\sum_{n=1}^{\infty} (a_n + b_n)$ è divergente.\\
  Infine si ha che se $\sum_{n=1}^{\infty} a_n$ è convergente, allora per ogni $\lambda \in \R$, la serie $\sum_{n=1}^{\infty} \lambda a_n$ è convergente e vale $\sum_{n=1}^{\infty} \lambda a_n = \lambda \sum_{n=1}^{\infty} a_n$.
\end{osservazione}

Vediamo un caso particolare in cui è facile determinare oltre al comportamento anche il valore della somma della serie:
\begin{definizione}{Serie geometrica}
  Una serie del tipo $\sum_{n=0}^{\infty} q^n$ è detta \textbf{serie geometrica}.\\
\end{definizione}
\begin{teorema}{}
  Si ha che:
  \begin{itemize}
    \item Se $|q| < 1$, allora la serie è convergente e vale $\sum_{n=0}^{\infty} q^n = \frac{1}{1-q}$.
    \item Se $|q| \geq 1$, allora la serie è divergente.
  \end{itemize}
\end{teorema}

\begin{proof}
  Consideriamo la serie $\sum_{n=k}^{\infty} q^n$. Possiamo riscriverla come:
  \[
  \sum_{n=k}^{\infty} q^n = q^k + q^{k+1} + q^{k+2} + \dots
  \]

  Fattorizziamo $q^k$:
  \[
  \sum_{n=k}^{\infty} q^n = q^k \left( 1 + q + q^2 + \dots \right).
  \]

  La serie tra parentesi è una serie geometrica con primo termine $1$ e ragione $q$. Poiché $|q| < 1$, sappiamo che questa serie converge e vale:
  \[
  \sum_{n=0}^{\infty} q^n = \frac{1}{1-q}.
  \]

  Sostituendo, otteniamo:
  \[
  \sum_{n=k}^{\infty} q^n = q^k \cdot \frac{1}{1-q} = \frac{q^k}{1-q}.
  \]

  Quindi, se $|q| < 1$, la serie $\sum_{n=k}^{\infty} q^n$ è convergente e vale $\frac{q^k}{1-q}$.
\end{proof}

Vediamo come generalizzarlo per una serie che non parte da $n=0$:
\begin{osservazione}{}
  Se $|q| < 1$, allora la serie $\sum_{n=k}^{\infty} q^n$ è convergente e vale $\sum_{n=k}^{\infty} q^n = \frac{q^k}{1-q}$.
\end{osservazione}

\section{Criteri di convergenza}
Vediamo ora alcuni importanti criteri che ci permettono di determinare il comportamento di una serie numerica.
\begin{teorema}{Condizione necessaria di convergenza}
  Se la serie $\sum_{n=1}^{\infty} a_n$ è convergente, allora $\lim_{n \to \infty} a_n = 0$.
\end{teorema}

\begin{proof}
  Sia $S_N = \sum_{n=1}^{N} a_n$ la somma parziale della serie. Poiché la serie $\sum_{n=1}^{\infty} a_n$ è convergente, esiste il limite $\lim_{N \to \infty} S_N = S$, con $S \in \R$.

  Consideriamo la differenza tra due somme parziali consecutive:
  \[
  S_{N+1} - S_N = a_{N+1}.
  \]

  Passando al limite per $N \to \infty$, otteniamo:
  \[
  \lim_{N \to \infty} (S_{N+1} - S_N) = \lim_{N \to \infty} a_{N+1}.
  \]

  Poiché $\lim_{N \to \infty} S_N = S$ e $\lim_{N \to \infty} S_{N+1} = S$, si ha:
  \[
  \lim_{N \to \infty} (S_{N+1} - S_N) = S - S = 0.
  \]

  Quindi $\lim_{N \to \infty} a_{N+1} = 0$, e pertanto $\lim_{n \to \infty} a_n = 0$.
\end{proof}

In generale è più facile studiare serie con tutti i termini positivi:
\begin{definizione}{Serie a termini positivi}
  Una serie $\sum_{n=1}^{\infty} a_n$ si dice \textbf{a termini positivi} se $a_n \geq 0$ per ogni $n$.
\end{definizione}

Vediamo subito un teorema che risolve molte patologie:
\begin{teorema}{}
  Una serie a termini positivi può essere convergente, divergente ma non indeterminata.\\
\end{teorema}

\begin{proof}
Le somme parziali sono una successione crescente, quindi $S_N$ ammette limite (finito o infinito).
\end{proof}
Inoltre sulle serie a termini positivi valgono i seguenti criteri:
\begin{teorema}{Criterio del confronto integrale}
  Sia $f:[1, +\infty) \to \R$ una funzione continua e a valori positivi. Se $a_n = f(n)$, allora la serie $\sum_{n=1}^{\infty} a_n$ è convergente se e solo se l'integrale $\int_{1}^{+\infty} f(x) \, dx$ è convergente.
\end{teorema}

\begin{proof}
  Consideriamo la funzione $f(x)$ continua, positiva e decrescente per $x \geq 1$, e confrontiamo la serie $\sum_{n=1}^{\infty} f(n)$ con l'integrale $\int_{1}^{+\infty} f(x) \, dx$.

  \begin{itemize}
    \item **Sottostima**: Per ogni $n \geq 1$, l'area del rettangolo di base $[n, n+1]$ e altezza $f(n)$ è minore dell'area sottesa dalla curva $f(x)$ sull'intervallo $[n, n+1]$. Quindi:
    \[
    f(n) \leq \int_{n}^{n+1} f(x) \, dx.
    \]

    Sommando per $n$ da $1$ a $N$, otteniamo:
    \[
    \sum_{n=1}^{N} f(n) \leq \int_{1}^{N+1} f(x) \, dx.
    \]

    \item **Sovrastima**: Analogamente, per ogni $n \geq 1$, l'area del rettangolo di base $[n-1, n]$ e altezza $f(n)$ è maggiore dell'area sottesa dalla curva $f(x)$ sull'intervallo $[n-1, n]$. Quindi:
    \[
    \int_{n-1}^{n} f(x) \, dx \leq f(n).
    \]

    Sommando per $n$ da $1$ a $N$, otteniamo:
    \[
    \int_{1}^{N} f(x) \, dx \leq \sum_{n=1}^{N} f(n).
    \]
  \end{itemize}

  Combinando le due disuguaglianze, otteniamo:
  \[
  \int_{1}^{N} f(x) \, dx \leq \sum_{n=1}^{N} f(n) \leq \int_{1}^{N+1} f(x) \, dx.
  \]

  Passando al limite per $N \to \infty$, si ha che la serie $\sum_{n=1}^{\infty} f(n)$ converge se e solo se l'integrale $\int_{1}^{+\infty} f(x) \, dx$ converge.
\end{proof}
Vediamo un esempio.\\

Studiamo la serie $\sum_{n=1}^{\infty} \frac{1}{n}$.\\
Applichiamo il criterio del confronto integrale con $f(x) = \frac{1}{x}$, ottenendo che la serie è divergente.\\

Vediamo ora un altro criterio:
\begin{teorema}{Criterio del confronto asintotico}
  \begin{itemize}
  \item  Siano $\{a_n\}_{n=1}^{\infty}$ e $\{b_n\}_{n=1}^{\infty}$ due successioni di numeri reali tali che $0 \leq a_n \leq b_n$ per ogni $n$. Se la serie $\sum_{n=1}^{\infty} b_n$ è convergente, allora la serie $\sum_{n=1}^{\infty} a_n$ è convergente.\\
  Se la serie $\sum_{n=1}^{\infty} a_n$ è divergente, allora la serie $\sum_{n=1}^{\infty} b_n$ è divergente.
  \item Siano $\{a_n\}_{n=1}^{\infty}$ e $\{b_n\}_{n=1}^{\infty}$ due successioni di numeri reali tali che $\lim_{n \to \infty} \frac{a_n}{b_n} = L$ con $0 < L < \infty$. Allora le due serie $\sum_{n=1}^{\infty} a_n$ e $\sum_{n=1}^{\infty} b_n$ sono entrambe convergenti o entrambe divergenti.
  \item Siano $\{a_n\}_{n=1}^{\infty}$ e $\{b_n\}_{n=1}^{\infty}$ due successioni di numeri reali tali che $a_n = o(b_n)$ per $n \to \infty$. Se la serie $\sum_{n=1}^{\infty} b_n$ è convergente, allora la serie $\sum_{n=1}^{\infty} a_n$ è convergente. Se la serie $\sum_{n=1}^{\infty} a_n$ è divergente, allora la serie $\sum_{n=1}^{\infty} b_n$ è divergente.
  \end{itemize}
\end{teorema}

\begin{proof}
  \begin{itemize}
    \item Modificando i primi termini delle due serie posso supporre che $a_n\leq b_n \forall n$. Allora: $S_N = a_1+\dots+a_N\leq \tilde{S_N}=b_1+\dots+b_N$. E quindi $\lim_{N\to\infty} S_N \leq \lim_{N\to\infty} \tilde{S_N}$.\\
    \item Supponiamo che $\lim_{n \to \infty} \frac{a_n}{b_n} = L$ con $0 < L < \infty$. Allora esiste un $N$ tale che per ogni $n \geq N$ si ha $\frac{1}{2}L b_n \leq a_n \leq 2L b_n$. Questo implica che le due serie $\sum_{n=1}^{\infty} a_n$ e $\sum_{n=1}^{\infty} b_n$ hanno lo stesso comportamento (entrambe convergenti o entrambe divergenti).

    \item Se $\lim_{n \to \infty} \frac{a_n}{b_n} = 0$, allora per $n$ sufficientemente grande si ha $a_n\leq b_n$ e si può applicare il primo punto.
  \end{itemize}
  \end{proof}

Esistono inoltre altri due importanti criteri:
\begin{teorema}{Criterio della radice}
  Sia $\{a_n\}_{n=1}^{\infty}$ una successione di numeri reali. Se esiste il limite $\lim_{n \to \infty} \sqrt[n]{|a_n|} = L$, allora:
  \begin{itemize}
    \item Se $L < 1$, la serie $\sum_{n=1}^{\infty} a_n$ è convergente.
    \item Se $L > 1$, la serie $\sum_{n=1}^{\infty} a_n$ è divergente.
    \item Se $L = 1$, il criterio non fornisce informazioni.
  \end{itemize}
\end{teorema}
\begin{teorema}{Criterio del rapporto}
  Sia $\{a_n\}_{n=1}^{\infty}$ una successione di numeri reali. Se esiste il limite $\lim_{n \to \infty} \left| \frac{a_{n+1}}{a_n} \right| = L$, allora:
  \begin{itemize}
    \item Se $L < 1$, la serie $\sum_{n=1}^{\infty} a_n$ è convergente.
    \item Se $L > 1$, la serie $\sum_{n=1}^{\infty} a_n$ è divergente.
    \item Se $L = 1$, il criterio non fornisce informazioni.
  \end{itemize}
\end{teorema}


Inoltre ecco la seguente osservazione:
\begin{osservazione}{}
  Si può dimostrare che se esiste il limite $\lim_{n \to \infty} \left| \frac{a_{n+1}}{a_n} \right| = L$ esiste anche il limite $\lim_{n \to \infty} \sqrt[n]{|a_n|} = L$.
\end{osservazione}

In alcuni casi può essere utile la formula di Stirling per il fattoriale:
\begin{teorema}{Teorema di Stirling}
  Si ha che $\lim_{n \to \infty} \frac{n!}{(\frac{n}{e})^n \sqrt{2\pi n}} = 1$.
\end{teorema}
Vediamo un esempio in cui risulta utile.\\
Studiamo la serie $\sum_{n=1}^{\infty} \frac{n!}{n^n}$.

Applichiamo il criterio della radice:
\[
\lim_{n \to \infty} \sqrt[n]{\left| \frac{n!}{n^n} \right|} = \lim_{n \to \infty} \frac{\sqrt[n]{n!}}{n}.
\]

Utilizziamo la formula di Stirling per approssimare $n!$:
\[
n! \sim \sqrt{2\pi n} \left( \frac{n}{e} \right)^n.
\]

Quindi:
\[
\sqrt[n]{n!} \sim \sqrt[n]{ \sqrt{2\pi n} \left( \frac{n}{e} \right)^n} = \cdot \sqrt[n]{\sqrt{2\pi n}} \cdot \frac{n}{e}.
\]

Poiché $\sqrt[n]{\sqrt{2\pi n}} \to 1$ per $n \to \infty$, abbiamo:
\[
\lim_{n \to \infty} \frac{\sqrt[n]{n!}}{n} = \lim_{n \to \infty} \frac{ 1 \cdot \frac{n}{e}}{n} = \frac{1}{e}.
\]

Poiché $\frac{1}{e} < 1$, la serie $\sum_{n=1}^{\infty} \frac{n!}{n^n}$ è convergente.

\section{Serie con termini di segno variabile}
Se $a_n$ cambia segno infinite volte, i criteri visti finora non sono applicabili.\\
Un caso importante sono le serie a segni alterni:
\begin{definizione}{Serie a segni alterni}
  Una serie $\sum_{n=1}^{\infty} (-1)^n a_n$ è detta serie \textbf{a segni alterni}.
\end{definizione}
Vediamo in questo caso un utile criterio:
\begin{teorema}{Criterio di Leibniz}
  Se $\{a_n\}_{n=1}^{\infty}$ è una successione a termini positivi e decrescenti con $\lim_{n \to \infty} a_n = 0$, allora la serie $\sum_{n=1}^{\infty} (-1)^n a_n$ è convergente.
\end{teorema}

\begin{proof}
Consideriamo la serie $\sum_{n=1}^{\infty} (-1)^n a_n$, dove $a_n$ è una successione a termini positivi, decrescenti e infinitesimi. Definiamo le somme parziali $S_N = \sum_{n=1}^{N} (-1)^n a_n$.

Separiamo i termini pari e dispari:
\[
S_{2N} = a_1 - a_2 + a_3 - \dots + a_{2N-1} - a_{2N},
\]
\[
S_{2N+1} = a_1 - a_2 + a_3 - \dots + a_{2N-1} - a_{2N} + a_{2N+1}.
\]

Osserviamo che $S_{2N+1} = S_{2N} + a_{2N+1}$. Poiché $a_n$ è decrescente e infinitesima, abbiamo $a_{2N+1} \to 0$ per $N \to \infty$. Quindi:
\[
\lim_{N \to \infty} S_{2N+1} = \lim_{N \to \infty} S_{2N}.
\]

Inoltre, poiché $a_n$ è decrescente, si ha $S_{2N} \leq S_{2N+1} \leq S_{2N} + a_{2N+1}$. Passando al limite per $N \to \infty$, otteniamo che $S_N$ converge a un valore finito $S$.

Pertanto, la serie $\sum_{n=1}^{\infty} (-1)^n a_n$ è convergente.
\end{proof}
\begin{osservazione}{}
  In particolare è necessario che la serie sia infinitesima. Il fatto che sia decrescente è sufficiente ma non necessario.
\end{osservazione}
Vediamo un esempio di applicazione del criterio di Leibniz:\\

Consideriamo la serie $\sum_{n=1}^{\infty} \frac{(-1)^n}{n}$.\\
Applichiamo il criterio di Leibniz:
\begin{itemize}
  \item La successione $a_n = \frac{1}{n}$ è a termini positivi.
  \item La successione $a_n = \frac{1}{n}$ è decrescente.
  \item $\lim_{n \to \infty} \frac{1}{n} = 0$.
\end{itemize}
Quindi, per il criterio di Leibniz, la serie $\sum_{n=1}^{\infty} \frac{(-1)^n}{n}$ è convergente.\\

In un caso più generale però i segni non sono necessariamente alternati.\\
Vediamo un criterio che ci permette di studiare serie con termini di segno qualunque:
\begin{teorema}{Criterio di convergenza assoluta}
  Se la serie $\sum_{n=1}^{\infty} |a_n|$ è convergente, allora la serie $\sum_{n=1}^{\infty} a_n$ è convergente.
\end{teorema}

Vediamo un esempio di applicazione del criterio di convergenza assoluta:\\

Consideriamo la serie $\sum_{n=1}^{\infty} \frac{\sin(n)}{n^2}$.\\
Studiamo la serie $\sum_{n=1}^{\infty} \left| \frac{\sin(n)}{n^2} \right|$.\\

Poiché $|\sin(n)| \leq 1$ per ogni $n$, abbiamo:
\[
\left| \frac{\sin(n)}{n^2} \right| \leq \frac{1}{n^2}.
\]

La serie $\sum_{n=1}^{\infty} \frac{1}{n^2}$ è una serie a termini positivi e sappiamo che è convergente (serie p con $p = 2 > 1$).\\

Quindi, per il criterio di convergenza assoluta, la serie $\sum_{n=1}^{\infty} \frac{\sin(n)}{n^2}$ è convergente.




