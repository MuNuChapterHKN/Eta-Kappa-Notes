\chapter{Calcolo differenziale}

\section{Introduzione}
\begin{definizione}{Funzione di più variabili}
  Una \textbf{funzione di più variabili} è una applicazione $f:\Omega \subset \R^n \rightarrow \R$, con $n\geq 1$. $\Omega$ è detto dominio della funzione.
  \end{definizione}

Come si visualizza una funzione di più variabili?\\
Ci sono due strumenti principali: il grafico e le curve di livello.\\

\begin{definizione}{Curve di livello}
Data $f:\Omega \inc \R^2 \rightarrow \R$, si definisce \textbf{curva di livello} $\lambda$ di $f$ il luogo dei punti $L_\lambda=\{(x,y)\in \Omega : f(x,y)=\lambda\}$.\\
\end{definizione}
Può capitare che $L_\lambda$ sia vuoto, es: $L_{-1}$ per $f(x,y)= x^2+y^2$.\\
Tipicamente $L_\lambda$ è una curva nel piano o l'unione di più curve.\\


\begin{definizione}{Grafico}
Data $f:\Omega \inc \R^2 \rightarrow \R$, si definisce \textbf{grafico} di $f$ il luogo dei punti $G_f=\{(x,y,z)\in \R^3 : z=f(x,y), (x,y)\in \Omega\}$.\\
\end{definizione}
In generale $\Omega \inc \R^n$ e $f:\Omega \rightarrow \R$ si definisce $G(f)=\{(x_1, \dots, x_n, z)\in \R^{n+1} : z=f(x_1, \dots, x_n), (x_1, \dots, x_n)\in \Omega\}$.\\

\subsection{Esempio di grafici da sapere}
Sono esempi di grafici da sapere i seguenti:
\begin{itemize}
  \item $f(x,y)=x^2+y^2$ è il grafico di un paraboloide
  \item $f(x,y)=x^2-y^2$ è il grafico di un iperbolide
  \item $f(x,y)=x^2$ è il grafico di un cilindro
  \item $f(x,y)=\sqrt{1-x^2-y^2}$ è il grafico di una semisfera
  \item $f(x,y)=\sqrt{x^2+y^2}$ è il grafico di un cono
\end{itemize}

\section{Cenni di limiti e continuità}

\begin{definizione}{Intorno di un punto}
Sia $\vec x_0 \in \R^n$ e sia $r>0$, si dice \textbf{intorno} di $\vec x_0$ la palla aperta $B_r(\vec x_0)=\{\vec x \in \R^n : \|\vec x - \vec x_0\|<r\}$, dove $\|\cdot\|$ denota la norma euclidea.\\
\end{definizione}

\begin{definizione}{Limite}
Sia $f:\Omega \inc \R^n \rightarrow \R$ e sia $p_0$ un punto di accumulazione per $\Omega$. Si dice che $f$ tende a $L \in \R$ per $p$ che tende a $p_0$ (cioè che il \textbf{limite} per $p$ che tende a $p_0$ di $f(p)$ è $L$), e si scrive
$$\lim_{p \to p_0} f(p) = L$$
se per ogni $\epsilon > 0$ esiste $\delta > 0$ tale che, per ogni $p \in \Omega$, $0 < \|p - p_0\| < \delta$ implica $|f(p) - L| < \epsilon$.
\end{definizione}

\begin{definizione}{Continuità}
Sia $f:\Omega \inc \R^n \rightarrow \R$ e sia $p_0 \in \Omega$. Si dice che $f$ è \textbf{continua} in $p_0$ se
$$\lim_{p \to p_0} f(p) = f(p_0).$$
\end{definizione}

Data $f$ come faccio a capire in quali punti è continua?\\
Si usano i seguenti fatti:
\begin{itemize}
  \item Una funzione di una sola variabile è continua se non presenta discontinuità nel suo dominio.
  \item Se $f$ e $g$ sono funzioni continue, allora le seguenti funzioni sono continue:
  \begin{itemize}
    \item La somma: $f + g$
    \item Il prodotto: $f \cdot g$
    \item Il rapporto: $\frac{f}{g}$, purché $g \neq 0$
    \item La composizione: $f \circ g$
  \end{itemize}
\end{itemize}

\subsection{Esempio di limite lungo semirette}
Consideriamo la funzione $f(x,y)= \frac{2xy}{x^2+y^2}$. Vogliamo studiare il limite di $f$ quando $(x,y)$ tende a $(0,0)$ lungo diverse semirette.

Osserviamo prima il comportamento lungo gli assi $x$ e $y$:
\[
f(x,0) = \frac{2x \cdot 0}{x^2 + 0^2} = 0
\]
\[
f(0,y) = \frac{2 \cdot 0 \cdot y}{0^2 + y^2} = 0
\]
Quindi, lungo gli assi il limite è $0$.

Consideriamo ora il limite lungo la semiretta $y=mx$ con $m$ costante:
\[
f(x,mx) = \frac{2x(mx)}{x^2+(mx)^2} = \frac{2mx^2}{x^2(1+m^2)} = \frac{2m}{1+m^2}
\]
Quindi, il limite dipende da $m$ e vale $\frac{2m}{1+m^2}$.

Osserviamo che il limite di $f(x,y)$ quando $(x,y)$ tende a $(0,0)$ dipende dalla direzione lungo la quale ci avviciniamo al punto $(0,0)$. Pertanto, possiamo concludere che il limite non esiste in senso assoluto.

\subsection{Esempio di limite lungo parabole}
Consideriamo la funzione $f(x,y)= \frac{x^2y}{y^2+x^4}$. Vogliamo studiare il limite di $f$ quando $(x,y)$ tende a $(0,0)$ lungo diverse curve.

Lungo la semiretta $y=mx$ con $m$ costante:
\[
f(x,mx) = \frac{x^2(mx)}{(mx)^2+x^4} = \frac{mx^3}{m^2x^2+x^4} = \frac{mx^3}{x^2(m^2+x^2)} = \frac{mx}{m^2+x^2}
\]
Quando $x \to 0$, il limite di $f(x,mx)$ è $0$.

Consideriamo la parabola $y=mx^2$ con $m$ costante:
\[
f(x,mx^2) = \frac{x^2(mx^2)}{(mx^2)^2+x^4} = \frac{mx^4}{m^2x^4+x^4} = \frac{mx^4}{x^4(m^2+1)} = \frac{m}{m^2+1}
\]
Quando $x \to 0$, il limite di $f(x,mx^2)$ è $0$ solo se $m=0$. Per $m \neq 0$, il limite non è $0$.

Osserviamo che il limite di $f(x,y)$ quando $(x,y)$ tende a $(0,0)$ dipende dalla curva lungo la quale ci avviciniamo al punto $(0,0)$. Pertanto, possiamo concludere che il limite non esiste in senso assoluto.

\section{Derivate parziali}
\begin{definizione}{Derivata in una variabile}
Sia $f: \R \rightarrow \R$ una funzione e sia $x_0 \in \R$. La \textbf{derivata} di $f$ in $x_0$, indicata con $f'(x_0)$, è definita come il limite del rapporto incrementale, se esiste:
$$f'(x_0) = \lim_{h \to 0} \frac{f(x_0 + h) - f(x_0)}{h}.$$
Geometricamente, la derivata rappresenta il coefficiente angolare della retta tangente al grafico di $f$ nel punto $(x_0, f(x_0))$.
\end{definizione}

\begin{definizione}{Derivata parziale}
  Sia ora $P_0 \in \R^n $ e $f(x_1, \dots, x_n)$ una funzione definita in un intorno di $P_0$. La \textbf{derivata parziale} di $f$ rispetto alla variabile $x_i$ in $P_0$ è definita come il limite del rapporto incrementale, se esiste: $
\frac{\partial f}{\partial x_i}(P_0) = \lim_{h \to 0} \frac{f(P_0+e_i\cdot h) - f(P_0)}{h}$, dove $e_i$ indica il versore fondamentale lungo la componente $i$.
\end{definizione}
Questo equivale a derivare $f(x_1,\dots, x_n)$, rispetto alla variabile $x_i$, lasciando fissi i vettori delle altre $n-1$ componenti.

In particolare, per una funzione di due variabili $f(x,y)$, esistono due derivate parziali: la derivata parziale rispetto a $x$ e la derivata parziale rispetto a $y$. Queste sono indicate rispettivamente con $\frac{\partial f}{\partial x}$ e $\frac{\partial f}{\partial y}$. \\
Consideriamo la funzione $f(x,y) = x^2 + y^2$. Le derivate parziali di $f$ sono:
\[
\frac{\partial f}{\partial x} = \frac{\partial}{\partial x}(x^2 + y^2) = 2x
\]
\[
\frac{\partial f}{\partial y} = \frac{\partial}{\partial y}(x^2 + y^2) = 2y
\]
Quindi, la derivata parziale rispetto a $x$ è $2x$ e la derivata parziale rispetto a $y$ è $2y$.\\

\section{Piano tangente}
Il piano tangente è un concetto fondamentale nel calcolo differenziale e rappresenta il piano che "tocca" una superficie in un punto dato, approssimando la superficie stessa vicino a quel punto.

\begin{definizione}{Piano tangente}
Sia $f:\Omega \inc \R^2 \rightarrow \R$ differenziabile in $(x_0,y_0) \in \Omega$, si dice \textbf{piano tangente} al grafico di $f$ in $(x_0,y_0,f(x_0,y_0))$ il piano:
$$z=f(x_0,y_0)+\frac{\partial f}{\partial x}(x_0,y_0)(x-x_0)+\frac{\partial f}{\partial y}(x_0,y_0)(y-y_0)$$
\end{definizione}

In altre parole, il piano tangente al grafico di una funzione $f(x,y)$ in un punto $(x_0, y_0)$ è dato dall'equazione lineare che approssima $f$ vicino a quel punto. Le derivate parziali $\frac{\partial f}{\partial x}$ e $\frac{\partial f}{\partial y}$ rappresentano le pendenze del piano tangente nelle direzioni $x$ e $y$, rispettivamente.
\\
Consideriamo la funzione $f(x,y) = x^2 + y^2$. Troviamo il piano tangente al grafico di $f$ nel punto $(1,1)$.

Calcoliamo le derivate parziali:
\[
\frac{\partial f}{\partial x} = 2x \quad \text{e} \quad \frac{\partial f}{\partial y} = 2y
\]

Valutiamo le derivate parziali nel punto $(1,1)$:
\[
\frac{\partial f}{\partial x}(1,1) = 2 \quad \text{e} \quad \frac{\partial f}{\partial y}(1,1) = 2
\]

L'equazione del piano tangente è quindi:
\[
z = f(1,1) + \frac{\partial f}{\partial x}(1,1)(x-1) + \frac{\partial f}{\partial y}(1,1)(y-1)
\]
\[
z = 1^2 + 1^2 + 2(x-1) + 2(y-1)
\]
\[
z = 2 + 2(x-1) + 2(y-1)
\]
\[
z = 2x + 2y - 2
\]

Quindi, il piano tangente al grafico di $f$ nel punto $(1,1)$ è dato dall'equazione $z = 2x + 2y - 2$.\\


La differenza rispetto al caso in una variabile è che in due (o più) variabili può capitare che esistanto tutte le derivate parziali ma $f$ potrebbe non solo non avere piano tangente al grafico ma addirittura essere discontinua in $P_0$ (si veda ad esempio la funzione nulla sugli assi, e che vale $1$ altrove).

\section{Differenziabilità}
\begin{definizione}{Differenziabilità in due variabili}
Sia $f:\Omega \inc \R^2 \rightarrow \R$ e sia $P_0 = (x_0,y_0) \in \Omega$. Si dice che $f$ è \textbf{differenziabile} in $P_0$ se esistono le derivate parziali di $f$ in $P_0$ rispetto a $x$ e $y$ e se vale lo sviluppo di Taylor di ordine 1 in $P_0$ con resto di Peano, cioè: $f(x,y)=f(x_0,y_0)+\frac{\partial f}{\partial x}(x_0,y_0)(x-x_0)+\frac{\partial f}{\partial y}(x_0,y_0)(y-y_0)+o(\sqrt{(x-x_0)^2+(y-y_0)^2}) \quad \text{per }(x,y) \rightarrow (x_0,y_0)$.
\end{definizione}

\begin{osservazione}{}
Se $f$ è differenziabile in $(x_0,y_0)$, allora il piano di equazione $z=f(x_0,y_0)+\frac{\partial f}{\partial x}(x_0,y_0)(x-x_0)+\frac{\partial f}{\partial y}(x_0,y_0)(y-y_0)$ è il piano tangente al grafico di $f$ in $(x_0,y_0,f(x_0,y_0))$.
\end{osservazione}
\begin{osservazione}{}
  La differenziabilità esprime il fatto che f(x,y) è approssibile con una funzione lineare con un errore che è $o(\sqrt{(x-x_0)^2+(y-y_0)^2}) \quad \text{per }(x,y) \rightarrow (x_0,y_0)$.
\end{osservazione}

\begin{definizione}{Gradiente}
Sia $f:\Omega \inc \R^n \rightarrow \R$ e sia $P_0 \in \Omega$. Si dice \textbf{gradiente} di $f$ in $P_0$ il vettore: $\nabla f(P_0)=\left(\frac{\partial f}{\partial x_1}(P_0), \dots, \frac{\partial f}{\partial x_n}(P_0)\right)$.
\end{definizione}

\begin{definizione}{Differenziabilità in $n$ variabili}
Sia $f:\Omega \inc \R^n \rightarrow \R$ e sia $P_0 \in \Omega$. Si dice che $f$ è \textbf{differenziabile} in $P_0$ se esistono le derivate parziali di $f$ in $P_0$ rispetto a tutte le variabili e se vale lo sviluppo di Taylor di ordine 1 in $P_0$ con resto di Peano, cioè: $x_{n+1}=f(P_0)+\nabla f(P_0)\cdot (P - P_0)+o(|P-P_0|) \quad \text{per } P \rightarrow P_0$.
\end{definizione}

Per analogia il "piano" (iperpiano) tangente al grafico di $f$ in $P_0$ è dato dall'equazione $z=f(P_0)+\nabla f(P_0)\cdot (P - P_0)$. In ogni caso resta vera la proprietà di approssimazione lineare con errore $o(|P-P_0|) \quad \text{per } P \rightarrow P_0$.\\

Per garantire che $f$ sia effettivamente differenziabile è utile il seguente teorema:
\begin{teorema}{Criterio di differenziabilità}
Se esistono le derivate parziali di $f$ in $P_0$ e sono continue in un intorno di $P_0$, allora $f$ è differenziabile in $P_0$.
\end{teorema}

\subsection{Esempio di verifica della differenziabilità e calcolo del piano tangente}
Consideriamo la funzione $f(x,y) = \frac{e^{2x+y}}{x^2+y^2}$. Vogliamo verificare il teorema di differenziabilità e calcolare il piano tangente al grafico di $f$ nel punto $(1,1)$.

Calcoliamo le derivate parziali di $f$ rispetto a $x$ e $y$:
\[
\frac{\partial f}{\partial x} = \frac{(2e^{2x+y})(x^2+y^2) - e^{2x+y}(2x)}{(x^2+y^2)^2} = \frac{2e^{2x+y}(x^2+y^2) - 2xe^{2x+y}}{(x^2+y^2)^2}
\]
\[
\frac{\partial f}{\partial y} = \frac{(e^{2x+y})(x^2+y^2) - e^{2x+y}(2y)}{(x^2+y^2)^2} = \frac{e^{2x+y}(x^2+y^2) - 2ye^{2x+y}}{(x^2+y^2)^2}
\]
Osserviamo che le derivate parziali sono continue in un intorno di $(1,1)$, quindi possiamo concludere che $f$ è differenziabile in $(1,1)$.\\
Valutiamo le derivate parziali nel punto $(1,1)$:
\[
\frac{\partial f}{\partial x}(1,1) = \frac{2e^{3}(1^2+1^2) - 2e^{3}}{(1^2+1^2)^2} = \frac{4e^{3} - 2e^{3}}{4} = \frac{2e^{3}}{4} = \frac{e^{3}}{2}
\]
\[
\frac{\partial f}{\partial y}(1,1) = \frac{e^{3}(1^2+1^2) - 2e^{3}}{(1^2+1^2)^2} = \frac{2e^{3} - 2e^{3}}{4} = 0
\]

L'equazione del piano tangente è quindi:
\[
z = f(1,1) + \frac{\partial f}{\partial x}(1,1)(x-1) + \frac{\partial f}{\partial y}(1,1)(y-1)
\]
\[
z = \frac{e^{3}}{2} + \frac{e^{3}}{2}(x-1) + 0(y-1)
\]
\[
z = \frac{e^{3}}{2} + \frac{e^{3}}{2}(x-1)
\]
\[
z = \frac{e^{3}}{2}(1 + x - 1)
\]
\[
z = \frac{e^{3}}{2}x
\]

Quindi, il piano tangente al grafico di $f$ nel punto $(1,1)$ è dato dall'equazione $z = \frac{e^{3}}{2}x$.\\

\section{Derivata direzionale}
Osserviamo che in due o più dimensioni non siamo vincolati a derivare lungo gli assi canonici $x$ e $y$, ma possiamo in generale calcolare la derivata lungo una direzione qualsiasi.
\begin{definizione}{Derivata direzionale}
Sia $f:\Omega \inc \R^n \rightarrow \R$ e sia $P_0 \in \Omega$. La \textbf{derivata direzionale} di $f$ in $P_0$ lungo il vettore $\vec v$ (di norma unitaria) è definita come il limite del rapporto incrementale, se esiste: $D_{\vec v}f(P_0) = \frac{\partial f}{\partial \vec v}(P_0) = \lim_{h \to 0} \frac{f(P_0 + h\vec v) - f(P_0)}{h}$.
\end{definizione}

\begin{osservazione}{}
La derivata direzionale di $f$ in $P_0$ lungo il vettore $\vec v$ rappresenta la pendenza di $f$ in $P_0$ nella direzione di $\vec v$. \\
\end{osservazione}


\section{Conseguenze della differenzaibilità}
Dalla diiferenziabilità possiamo infine trarre importanti conseguenze. Le più rilevanti sono riassunte nel seguente teorema:

\begin{teorema}{Conseguenze della differenziabilità}
  Sia $f$ differenziabile in $P_0 \in \R^n$. Allora:
  \begin{itemize}
    \item $(1) f$ è continua in $P_0$
    \item $(2)$ Il vettore $\nabla f(P_0)$ indica nel dominio di $f$ e partendo dal punto $P_0$ la direzione di massima crescita di $f$, cioè quella lungo cui $f$ cresce più velocemente.
    \item $(3)$ In $P_0$ il grafico ha pendenza $\alpha$ rispetto alla direzione di $\nabla f(P_0)$, dove $\alpha =  \|\nabla f(P_0)\| = \sqrt{\left(\frac{\partial f}{\partial x_1}(P_0)\right)^2 + \cdots + \left(\frac{\partial f}{\partial x_n}(P_0)\right)^2}$.
    \item $(4)$ Se $\vec{v}\in \R^n$ è un vettore unitario, allora la derivata direzionale di $f$ in $P_0$ lungo $\vec{v}$ è data da $D_{\vec{v}}f(P_0) = \nabla f(P_0) \cdot \vec{v}$.
    \item $(5) $Se $n=2$ e $\nabla f(P_0)\neq (0,0)$, allora il gradiente $\nabla f(P_0)$ è ortogonale alla curva di livello di $f$ passante per $P_0$
\end{itemize}
\end{teorema}



\begin{proof}
  Vediamo i vari punti.\\
  $(1)$\\
  L'ipotesi è equivalente a:
  $f(P)=f(P_0)+\nabla f(P_0)\cdot (P-P_0)+o(||P-P_0||) \quad \text{per } P \rightarrow P_0$\\
  \begin{gather*}
 |f(P) - f(P_0)|= |\nabla f(P_0)\cdot(P-P_0)+O(||P-P_0||)\leq\\||\nabla f(P_0) ||\cdot|| P-P_0||+\frac{O(||P-P_0||)}{||P-P_0|}\cdot ||(P-P_0)||
  \end{gather*}
  dove il primo termine va a zero per $P\rightarrow P_0$ e il secondo fa 0 per definizione di O. Da cui segue che $f$ è continua in $P_0$.\\
  \newline
  $(4)$
  Se scelgo $\vec{v}\in \R^n, ||v||=1$, per definizione di derivata direzionale ho: $\frac{\partial f}{\partial \vec{v}}(P_0) = \lim_{h \to 0} \frac{f(P_0 + h\vec{v}) - f(P_0)}{h}$. Uso lo sviluppo di Taylor con  $P=P_0 + h\vec{v}$:
  \begin{gather*}
  f(P_0 + h\vec{v}) = f(P_0) + \nabla f(P_0)\cdot (hv) + o(||h||)\\
  \end{gather*}
  Quindi:
  \begin{gather*}
  \frac{\partial f}{\partial \vec{v}}(P_0)=\frac {\nabla f(P_0)\cdot (hv)+ o(h)}{h}\\=\nabla f(P_0)\cdot \vec{v}
  \end{gather*}\\
  \newline
  $(2)$ Segue da $(4)$, in quanto la funzione in una direzione $v$, $||v||=1$, cresce partendo da $P_0$ con velocità data da $\frac{\delta f}{\delta v}(P_0) = \nabla f(P_0)\cdot v = ||\nabla f(P_0)||\cdot ||v||\cdot \cos{\alpha}$, che è massima per $\cos{\alpha}=1$, cioè quando $v$ ha la direzione del vettore gradiente. Questo dimostra analogamente (3)\\

\end{proof}

Per la dimostrazione del punto $(5)$, ci dotiamo del seguente teorema:

\begin{teorema}{Teorema di Dini}
  Sia $f:\Omega \inc \R^n \rightarrow \R$ e sia $P_0 \in \Omega$. Se $f$ è continua in $P_0$ e se esistono le derivate parziali di $f$ in $P_0$ e sono continue in un intorno di $P_0$, allora $f$ è differenziabile in $P_0$.
\end{teorema}

Tuttavia per concludere bene la dimostrazione serve introdurre la successiva sezione.

\section{Derivata lungo una curva}\label{sec:derivata-lungo-una-curva}

\begin{definizione}{Curva in $\R^n$}
Una \textbf{curva} in $\R^n$ è una funzione $\gamma: I \rightarrow \R^n$ con $I$ intervallo di $\R$.
\end{definizione}

Tipicamente considereremo curve dove l'intervallo $I$ è un intervallo chiuso e limitato, cioè $I=[a,b]$.\\
$\gamma(a)$ e $\gamma(b)$ sono detti rispettivamente punto iniziale e punto finale della curva.\\
Concretamente quindi, per ogni $t\in I$ si ha $\gamma(t)=(x_1(t), \dots, x_n(t))$. Gli $x_i(t)$ sono le componenti della curva. Ciascuna componente è una funzione dell'analisi 1.\\
In seguito supporremo sempre che le componenti della curva siano funzioni di classe $C^1$.

\begin{definizione}{Derivata lungo una curva}
  La \textbf{derivata lungo la curva} $\gamma$ in $t$ è definita come il vettore $\gamma'(t)=(x_1'(t), \dots, x_n'(t))$.\\
  Se esiste la derivata seconda di $\gamma$ in $t$ allora si definisce $\gamma''(t)=(x_1''(t), \dots, x_n''(t))$ e così via.
\end{definizione}
Ora data una funzione $f:\Omega \inc \R^n \rightarrow \R$ e una curva $\gamma: [a,b] \rightarrow \R^n$ possiamo definire la derivata di $f$ lungo $\gamma$, grazie al seguente teorema:

\begin{teorema}{}
  Sia $f:\Omega \inc \R^n \rightarrow \R$ e $\gamma: [a,b] \rightarrow \R^n$ una curva. Se $f$ è differenziabile in $\gamma(t)$ per ogni $t \in [a,b]$ e se $\gamma$ è derivabile in $t$, allora la funzione composta $f \circ \gamma$ è derivabile in $t$ e si ha:
  $\frac{d}{dt}(f \circ \gamma)(t) = \nabla f(\gamma(t)) \cdot \gamma'(t)$.
\end{teorema}

\begin{proof}
Sia $f:\Omega \subset \R^n \to \R$ una funzione differenziabile in $\gamma(t)$ e sia $\gamma:[a,b] \to \R^n$ una curva derivabile in $t$. Consideriamo la funzione composta $h(t) = f(\gamma(t))$. Vogliamo dimostrare che $h(t)$ è derivabile in $t$ e che $\frac{d}{dt} h(t) = \nabla f(\gamma(t)) \cdot \gamma'(t)$.

Per definizione di derivata, abbiamo:
\[
\frac{d}{dt} h(t) = \lim_{h \to 0} \frac{f(\gamma(t+h)) - f(\gamma(t))}{h}.
\]

Poiché $f$ è differenziabile in $\gamma(t)$, possiamo scrivere lo sviluppo di Taylor di $f$ attorno a $\gamma(t)$:
\[
f(\gamma(t+h)) = f(\gamma(t)) + \nabla f(\gamma(t)) \cdot (\gamma(t+h) - \gamma(t)) + o(\|\gamma(t+h) - \gamma(t)\|).
\]

Sostituendo nello sviluppo di $h(t)$, otteniamo:
\[
\frac{f(\gamma(t+h)) - f(\gamma(t))}{h} = \frac{\nabla f(\gamma(t)) \cdot (\gamma(t+h) - \gamma(t))}{h} + \frac{o(\|\gamma(t+h) - \gamma(t)\|)}{h}.
\]

Poiché $\gamma$ è derivabile in $t$, possiamo scrivere:
\[
\gamma'(t) = \lim_{h \to 0} \frac{\gamma(t+h) - \gamma(t)}{h}.
\]

Quindi, $\gamma(t+h) - \gamma(t) = h \gamma'(t) + o(h)$, e sostituendo:
\[
\frac{\nabla f(\gamma(t)) \cdot (\gamma(t+h) - \gamma(t))}{h} = \nabla f(\gamma(t)) \cdot \gamma'(t).
\]

Infine, il termine $o(\|\gamma(t+h) - \gamma(t)\|)/h$ tende a $0$ per $h \to 0$, poiché $\|\gamma(t+h) - \gamma(t)\| = O(h)$. Pertanto:
\[
\frac{d}{dt} h(t) = \nabla f(\gamma(t)) \cdot \gamma'(t).
\]
\end{proof}

Possiamo ora dimostrare il punto $(5)$ del teorema precedente.\\
$(5)$\\
Sia $f$ differenziabile in $P_0 \in \R^2$ e supponiamo che $\nabla f(P_0) \neq (0, 0)$. Consideriamo la curva di livello di $f$ passante per $P_0$, definita come $L = \{P \in \R^2 : f(P) = f(P_0)\}$. Vogliamo dimostrare che $\nabla f(P_0)$ è ortogonale a $L$ in $P_0$.

Poiché $f$ è differenziabile in $P_0$, possiamo parametrizzare localmente $L$ attorno a $P_0$ tramite una curva $\gamma(t)$ tale che $\gamma(0) = P_0$ e $\gamma(t) \in L$ per ogni $t$.

Poiché $\gamma(t) \in L$, abbiamo che $f(\gamma(t)) = f(P_0)$ per ogni $t$. Derivando rispetto a $t$, otteniamo:
\[
\frac{d}{dt} f(\gamma(t)) = \nabla f(\gamma(t)) \cdot \gamma'(t).
\]

In particolare, valutando in $t = 0$, si ha:
\[
\frac{d}{dt} f(\gamma(t)) \big|_{t=0} = \nabla f(P_0) \cdot \gamma'(0).
\]

Poiché $f(\gamma(t)) = f(P_0)$ è costante, la derivata rispetto a $t$ è zero:
\[
\nabla f(P_0) \cdot \gamma'(0) = 0.
\]

Quindi, $\nabla f(P_0)$ è ortogonale al vettore tangente $\gamma'(0)$ alla curva $L$ in $P_0$. Poiché $\gamma'(0)$ rappresenta un vettore tangente a $L$, concludiamo che $\nabla f(P_0)$ è ortogonale a $L$ in $P_0$.



\section{Derivate successive}
\begin{definizione}{Derivate successive}
  Se $f$ è differenziabile in un intorno di $P_0$ allora si definiscono le \textbf{derivate seconde} di $f$ (nei punti in cui esistono) come le derivate parziali delle derivate parziali: $\frac{\partial^2 f}{\partial x_i \partial x_j}(P_0) = \frac{\partial}{\partial x_i} \left(\frac{\partial f}{\partial x_j}\right)(P_0)$.  \\
  Quando le derivate seconde esistono si può provare ad andare a avanti a derivare fino a ottenere le derivate terze e così via.
\end{definizione}
\begin{osservazione}{}
  In linea di princio ci sono al più $n$ derivate prime, $n^2$ derivate seconde, $n^k$ derivate k-esime.
\end{osservazione}

Consideriamo ad esempio la funzione $f(x,y) = (x^2 + y^2)\sin(2y)$. Calcoliamo le derivate parziali prime e seconde di $f$.


\[
\frac{\partial f}{\partial x} = \frac{\partial}{\partial x} \left( (x^2 + y^2)\sin(2y) \right) = 2x\sin(2y)
\]
\[
\frac{\partial f}{\partial y} = \frac{\partial}{\partial y} \left( (x^2 + y^2)\sin(2y) \right) = 2y\sin(2y) + (x^2 + y^2)2\cos(2y)
\]

\[
\frac{\partial^2 f}{\partial x^2} = \frac{\partial}{\partial x} \left( 2x\sin(2y) \right) = 2\sin(2y)
\]
\[
\frac{\partial^2 f}{\partial y^2} = \frac{\partial}{\partial y} \left( 2y\sin(2y) + (x^2 + y^2)2\cos(2y) \right)
\]
\[
= 2\sin(2y) + 2y \cdot 2\cos(2y) + 2\cos(2y) \cdot 2y + (x^2 + y^2) \cdot 2(-2\sin(2y))
\]
\[
= 2\sin(2y) + 4y\cos(2y) + 4y\cos(2y) - 4(x^2 + y^2)\sin(2y)
\]
\[
= 2\sin(2y) + 8y\cos(2y) - 4(x^2 + y^2)\sin(2y)
\]
\[
= 2\sin(2y)(1 - 2(x^2 + y^2)) + 8y\cos(2y)
\]

\[
\frac{\partial^2 f}{\partial x \partial y} = \frac{\partial}{\partial y} \left( 2x\sin(2y) \right) = 2x \cdot 2\cos(2y) = 4x\cos(2y)
\]
\[
\frac{\partial^2 f}{\partial y \partial x} = \frac{\partial}{\partial x} \left( 2y\sin(2y) + (x^2 + y^2)2\cos(2y) \right) = 4x\cos(2y)
\]

Osserviamo che in questo esempio le derivate miste ($\frac{\partial^2 f}{\partial y \partial x}$,$\frac{\partial^2 f}{\partial x \partial y}$) sono uguali. In generale questo non è un caso.

\begin{teorema}{Teorema di Schwarz}
  Se le derivate miste $\frac{\partial^2 f}{\partial x \partial y}$ e $\frac{\partial^2 f}{\partial y \partial x}$ esistono e sono continue in un intorno di $P_0$ allora sono uguali.
\end{teorema}

\begin{corollario}{}
  Se $f$ è di classe $C^k$ allora le derivate miste di ordine $k$ sono indipendenti dall'ordine di derivazione.
\end{corollario}

Vediamo ora come si calcolano le derivate direzionali di ordine 2.
\begin{definizione}{Matrice Hessiana}
  Se $f$ è $C^2$ allora si definisce \textbf{matrice Hessiana} di $f$ in $P_0$ la matrice:
  $Hf(P_0)=\begin{Bmatrix}
    \frac{\partial^2 f}{\partial x_1^2}(P_0) & \frac{\partial^2 f}{\partial x_1 \partial x_2}(P_0) & \cdots & \frac{\partial^2 f}{\partial x_1 \partial x_n}(P_0)\\
    \frac{\partial^2 f}{\partial x_2 \partial x_1}(P_0) & \frac{\partial^2 f}{\partial x_2^2}(P_0) & \cdots & \frac{\partial^2 f}{\partial x_2 \partial x_n}(P_0)\\
    \vdots & \vdots & \ddots & \vdots\\
    \frac{\partial^2 f}{\partial x_n \partial x_1}(P_0) & \frac{\partial^2 f}{\partial x_n \partial x_2}(P_0) & \cdots & \frac{\partial^2 f}{\partial x_n^2}(P_0)
  \end{Bmatrix}$
\end{definizione}
\begin{osservazione}{}
  La matrice Hessiana è simmetrica.
\end{osservazione}

\begin{definizione}{Derivata direzionale di ordine 2}
  Se $f$ è $C^2$ allora la \textbf{derivata direzionale di ordine 2} di $f$ in $P_0$ lungo il vettore $\vec v$ è data da:
  $D^2_{\vec v}f(P_0)=\vec v^T Hf(P_0) \vec v$
\end{definizione}

\begin{teorema}{Sviluppo di Taylor di ordine 2}
  Se $f$ è $C^2$ allora si ha lo sviluppo di Taylor di ordine 2 di $f$ in $P_0$:
  $f(P) = f(P_0) + \nabla f(P_0) \cdot (P - P_0) + \frac{1}{2} (P - P_0)^T Hf(P_0) (P - P_0) + o(|P - P_0|^2)$
\end{teorema}

\subsection{Esempio di calcolo dello sviluppo di Taylor di ordine 2}
Consideriamo la funzione $f(x,y) = e^{x+y}$. Vogliamo trovare lo sviluppo di Taylor di ordine 2 di $f$ attorno al punto $(0,0)$.

Calcoliamo le derivate parziali prime di $f$:
\[
\frac{\partial f}{\partial x} = e^{x+y}, \quad \frac{\partial f}{\partial y} = e^{x+y}
\]

Calcoliamo le derivate parziali seconde di $f$:
\[
\frac{\partial^2 f}{\partial x^2} = e^{x+y}, \quad \frac{\partial^2 f}{\partial y^2} = e^{x+y}, \quad \frac{\partial^2 f}{\partial x \partial y} = e^{x+y}
\]

Valutiamo le derivate parziali nel punto $(0,0)$:
\[
\frac{\partial f}{\partial x}(0,0) = 1, \quad \frac{\partial f}{\partial y}(0,0) = 1
\]
\[
\frac{\partial^2 f}{\partial x^2}(0,0) = 1, \quad \frac{\partial^2 f}{\partial y^2}(0,0) = 1, \quad \frac{\partial^2 f}{\partial x \partial y}(0,0) = 1
\]

Lo sviluppo di Taylor di ordine 2 di $f$ attorno al punto $(0,0)$ è dato da:
\[
f(x,y) \approx f(0,0) + \frac{\partial f}{\partial x}(0,0)x + \frac{\partial f}{\partial y}(0,0)y + \frac{1}{2}\left( \frac{\partial^2 f}{\partial x^2}(0,0)x^2 + 2\frac{\partial^2 f}{\partial x \partial y}(0,0)xy + \frac{\partial^2 f}{\partial y^2}(0,0)y^2 \right)
\]
\[
= 1 + x + y + \frac{1}{2}(x^2 + 2xy + y^2)
\]
\[
= 1 + x + y + \frac{1}{2}(x^2 + 2xy + y^2)
\]
\[
= 1 + x + y + \frac{1}{2}x^2 + xy + \frac{1}{2}y^2
\]

Quindi, lo sviluppo di Taylor di ordine 2 di $f(x,y) = e^{x+y}$ attorno al punto $(0,0)$ è:
\[
f(x,y) \approx 1 + x + y + \frac{1}{2}x^2 + xy + \frac{1}{2}y^2
\]

\section{Massimi e minimi di una funzione}

Questa sezione è dedicata allo studio dei punti di massimo e minimo delle funzioni di più variabili e alla loro classificazione.

\begin{definizione}{Punto di minimo}
  Sia $f:\Omega \inc \R^n \rightarrow \R$ e sia $P_0 \in \Omega$. Si dice che $f$ ha un \textbf{punto di minimo} locale in $P_0$ se esiste un intorno di $P_0$ tale che $f(P) \geq f(P_0)$ per ogni $P$ nell'intorno.
\end{definizione}
Analogamente si definisce il punto di massimo locale.
\begin{teorema}{}
  Se $f$ ha un punto di minimo o massimo locale in $P_0$ e $f$ è differenziabile in $P_0$ allora $\nabla f(P_0) = 0$.
\end{teorema}


\begin{osservazione}{}
  Questa condizione corrisponde a un sistema di equazioni:
  \[
  \frac{\partial f}{\partial x_1}(P_0) = 0, \quad \frac{\partial f}{\partial x_2}(P_0) = 0, \quad \dots, \quad \frac{\partial f}{\partial x_n}(P_0) = 0
  \]
\end{osservazione}

\begin{definizione}{Punti stazionari}
  I punti $P_0$ tali che $\nabla f(P_0) = 0$ sono detti \textbf{punti stazionari}.
\end{definizione}

Facciamo un esempio.
Consideriamo la funzione $f(x,y) = x^3 - xy^2 + 2xy$. Vogliamo trovare i punti critici di $f$.

Calcoliamo le derivate parziali prime di $f$:
\[
\frac{\partial f}{\partial x} = 3x^2 - y^2 + 2y
\]
\[
\frac{\partial f}{\partial y} = -2xy + 2x
\]

Poniamo le derivate parziali uguali a zero per trovare i punti critici:
\[
3x^2 - y^2 + 2y = 0
\]
\[
-2xy + 2x = 0
\]

Dalla seconda equazione, possiamo fattorizzare:
\[
2x(-y + 1) = 0
\]

Quindi, abbiamo due casi:
1. $2x = 0 \implies x = 0$\\
2. $-y + 1 = 0 \implies y = 1$

Caso 1: $x = 0$
\[
3(0)^2 - y^2 + 2y = 0 \implies -y^2 + 2y = 0 \implies y(y - 2) = 0
\]
Quindi, $y = 0$ o $y = 2$. I punti critici sono $(0,0)$ e $(0,2)$.

Caso 2: $y = 1$
\[
3x^2 - (1)^2 + 2(1) = 0 \implies 3x^2 - 1 + 2 = 0 \implies 3x^2 + 1 = 0
\]
Questa equazione non ha soluzioni reali.

Quindi, i punti critici della funzione $f(x,y) = x^3 - xy^2 + 2xy$ sono $(0,0)$ e $(0,2)$.\\

Come possiamo capire la natura di un punto critico?
Se $P_0$ è un punto critico di $f$ e $f$ è $C^2$ in un intorno di $P_0$ allora si può studiare lo sviluppo al secondo ordine di $f$ in $P_0$ per capire la natura del punto critico. Lo sviluppo si riduce a: $f(P) - f(P_0) = \frac{1}{2} (P - P_0)^T Hf(P_0) (P - P_0) + o(|P - P_0|^2)$.\\
Si tratta dunque di capire che segno ha il membro destro in un intorno di $P_0$.
\begin{itemize}
\item $P_0$ è un punto di massimo se e solo se il membro destro è sempre negativo in un intorno di $P_0$.
\item $P_0$ è un punto di minimo se e solo se il membro destro è sempre positivo in un intorno di $P_0$.
\item $P_0$ è un punto di sella se e solo se il membro destro cambia segno in un intorno di $P_0$.
\end{itemize}
\begin{osservazione}{}
  Il segno del membro destro dipende solo da segno di $\frac{1}{2} (P - P_0)^T Hf(P_0) (P - P_0)$, che è uguale al segno di $\frac{1}{2} \frac{(P - P_0)^T}{\|P - P_0\|} Hf(P_0) \frac{P-P_0}{\|P - P_0\|}$, e avendo $ \frac{P - P_0}{\|P - P_0\|}$ norma 1, il segno non dipende dal raggio dell'intorno di $P_0$ considerato.
\end{osservazione}

Ricordiamo che una matrice simmetrica $A$ è definita positiva se per ogni vettore $v \neq 0$ si ha $v^T A v > 0$. Analogamente si definiscono le matrici definite negative quelle per cui $v^T A v < 0$ per ogni $v \neq 0$ e indefinite quelle per cui $v^T A v$ può assumere segni diversi a seconda di $v$.

\begin{teorema}{Classificazione dei punti critici}
  Sia $f$ una funzione $C^2$ e sia $P_0$ un punto critico di $f$. Allora:
  \begin{itemize}
    \item Se $Hf(P_0)$ è definita positiva allora $P_0$ è un punto di minimo locale.
    \item Se $Hf(P_0)$ è definita negativa allora $P_0$ è un punto di massimo locale.
    \item Se $Hf(P_0)$ è indefinita allora $P_0$ è un punto di sella.
  \end{itemize}
\end{teorema}

\subsection{Esempio di classificazione dei punti critici}
Consideriamo la funzione $f(x,y) = x^3 - xy^2 + 2xy$. Abbiamo già trovato i punti critici $(0,0)$ e $(0,2)$.

Calcoliamo le derivate parziali seconde di $f$:
\[
\frac{\partial^2 f}{\partial x^2} = 6x, \quad \frac{\partial^2 f}{\partial y^2} = -2x, \quad \frac{\partial^2 f}{\partial x \partial y} = -2y + 2, \quad \frac{\partial^2 f}{\partial y \partial x} = -2y + 2
\]

Valutiamo le derivate parziali seconde nei punti critici.

Per il punto $(0,0)$:
\[
\frac{\partial^2 f}{\partial x^2}(0,0) = 0, \quad \frac{\partial^2 f}{\partial y^2}(0,0) = 0, \quad \frac{\partial^2 f}{\partial x \partial y}(0,0) = 2, \quad \frac{\partial^2 f}{\partial y \partial x}(0,0) = 2
\]

La matrice Hessiana in $(0,0)$ è:
\[
Hf(0,0) = \begin{pmatrix}
0 & 2 \\
2 & 0
\end{pmatrix}
\]

Il determinante della matrice Hessiana è:
\[
\det(Hf(0,0)) = (0)(0) - (2)(2) = -4
\]

Poiché il determinante è negativo, la matrice Hessiana è indefinita. Quindi, il punto $(0,0)$ è un punto di sella.

Per il punto $(0,2)$:
\[
\frac{\partial^2 f}{\partial x^2}(0,2) = 0, \quad \frac{\partial^2 f}{\partial y^2}(0,2) = 0, \quad \frac{\partial^2 f}{\partial x \partial y}(0,2) = -2, \quad \frac{\partial^2 f}{\partial y \partial x}(0,2) = -2
\]

La matrice Hessiana in $(0,2)$ è:
\[
Hf(0,2) = \begin{pmatrix}
0 & -2 \\
-2 & 0
\end{pmatrix}
\]

Il determinante della matrice Hessiana è:
\[
\det(Hf(0,2)) = (0)(0) - (-2)(-2) = -4
\]

Poiché il determinante è negativo, la matrice Hessiana è indefinita. Quindi, il punto $(0,2)$ è un punto di sella.

In conclusione, i punti critici della funzione $f(x,y) = x^3 - xy^2 + 2xy$ sono entrambi punti di sella.

\section{Massimi e minimi vincolati}
In questa sezione ci occuperemo di trovare i massimi e i minimi di una funzione $f$ di più variabili vincolata da una o più equazioni.\\
Esempio: vogliamo trovare i punti critici per una certa $f$ vincolati a stare su una circonferenza di equazione $x^2 + y^2 = 16$.

\begin{osservazione}{}
  Trovare i punti critici per $f$ qui non è significativo, perché potrebbero non essere sulla circonferenza.
\end{osservazione}

\begin{teorema}{Moltiplicatori di Lagrange}
  Sia $f$ una funzione $C^1$ su $V\in \R^n$, $V={(x_1, \dots, x_n)\in \R^n | g(x_1, \dots, x_n) = 0}$, con $g$ una funzione $C^1$ su $V$ e tale che $\nabla g(p) \neq (0, 0, \dots, 0)$ per ogni $p \in V$. Sia $P_0$ un punto critico di $f$ ristretta a $V$. Allora esiste un $\lambda \in \R$ tale che $\nabla f(P_0) = \lambda \nabla g(P_0)$.
\end{teorema}


