
\chapter{Serie di potenze}
\section{Introduzione}
In questo capitolo vedremo le serie di potenze.
\begin{definizione}{Serie di potenze}
    Una \textbf{serie di potenze} è una serie della forma
    \[
        \sum_{n=0}^{+\infty} a_n (x-x_0)^n
    \]
    dove $a_n \in \R$ e $x,x_0 \in \R$.\\
    $x_0$ è detto il centro della serie.\\
    I numeri $a_n$ sono detti coefficienti della serie.\\
    Per ogni $x$ fissato, la serie è una serie numerica.
\end{definizione}

Può essere utile sapere per quali valori di $x$ la serie converge.\\

\begin{definizione}{Insieme di convergenza}
    Sia data una serie di potenze $\sum_{n=0}^{+\infty} a_n (x-x_0)^n$.\\
    L'\textbf{insieme di convergenza} della serie è l'insieme
    \[
        D=\left\{ x \in \R \mid \sum_{n=0}^{+\infty} a_n (x-x_0)^n \text{ converge} \right\}
    \]
\end{definizione}

\begin{osservazione}{}
    L'insieme di convergenza $D$ di una serie di potenze non è mai vuoto, in quanto la serie converge sempre per $x=x_0$.
\end{osservazione}


La teoria si basa inizialmente sul seguente teorema:
\begin{teorema}{}
  Data una serie di potenze $\sum_{n=0}^{+\infty} a_n (x-x_0)^n$, se esiste $\bar{x} \in D, \bar{x}\neq x_0$, allora:
  La serie converge assolutamente per ogni $x$ tale che $\lvert x - x_0 \rvert < \lvert \bar{x} - x_0 \rvert$.\\
\end{teorema}

\begin{proof}
  Consideriamo la serie di potenze $\sum_{n=0}^{+\infty} a_n (x-x_0)^n$ con raggio di convergenza $R > 0$. Per dimostrare il teorema, sfruttiamo la serie geometrica.

  La serie geometrica $\sum_{n=0}^{+\infty} r^n$ converge per $\lvert r \rvert < 1$ e la sua somma è data da
  \[
      \sum_{n=0}^{+\infty} r^n = \frac{1}{1-r}, \quad \text{per } \lvert r \rvert < 1.
  \]

  Ora, consideriamo la funzione somma $f(x)$ associata alla serie di potenze:
  \[
      f(x) = \sum_{n=0}^{+\infty} a_n (x-x_0)^n, \quad \text{per } x \in (x_0-R, x_0+R).
  \]

  Per ogni $x$ tale che $\lvert x - x_0 \rvert < R$, possiamo riscrivere la serie come
  \[
      f(x) = \sum_{n=0}^{+\infty} a_n (x-x_0)^n = \sum_{n=0}^{+\infty} a_n \left( \frac{x-x_0}{R} \cdot R \right)^n.
  \]

  Poniamo $r = \frac{x-x_0}{R}$, con $\lvert r \rvert < 1$. La serie diventa
  \[
      f(x) = \sum_{n=0}^{+\infty} \left( a_n R^n \right) r^n.
  \]

  Questa è una serie geometrica in $r$, con coefficienti $a_n R^n$. Poiché $\lvert r \rvert < 1$, possiamo applicare la formula della somma della serie geometrica:
  \[
      f(x) = \frac{1}{1-r}, \quad \text{dove } r = \frac{x-x_0}{R}.
  \]

  Tornando alla variabile $x$, otteniamo
  \[
      f(x) = \frac{1}{1 - \frac{x-x_0}{R}} = \frac{R}{R - (x-x_0)}.
  \]

  Questo dimostra che la funzione somma $f(x)$ è ben definita e derivabile infinite volte in $(x_0-R, x_0+R)$, e che la serie converge assolutamente per $\lvert x - x_0 \rvert < R$.
\end{proof}
\begin{definizione}{Raggio di convergenza}
  Il \textbf{raggio di convergenza} $R$ di una serie di potenze è il numero
  \[
      R = \sup \left\{ r \in \R \mid \sum_{n=0}^{+\infty} a_n r^n \text{ converge} \right\}
  \]
  Se $R = +\infty$, si dice che la serie ha raggio di convergenza infinito.\\
  Se $R = 0$, si dice che la serie ha raggio di convergenza nullo.\\
  Se $0 < R < +\infty$, si dice che la serie ha raggio di convergenza finito.
\end{definizione}

Da questo, si ricava il seguente corollario:
\begin{corollario}{}
  L'insieme di convergenza D di una serie di potenze è sempre un intervallo.\\
  Più precisamente sono possibili 3 casi (a seconda del raggio di convergenza $R$):
  \begin{itemize}
  \item $D = \R$ se $R = +\infty$
  \item $D = [x_0,x_0]$ se $R = 0$
  \item Se invece $R \inc (0,+\infty)$, allora ci sono altri quattro possibili casi:
  \begin{itemize}
  \item $D = (x_0-R,x_0+R)$
  \item $D = [x_0-R,x_0+R)$
  \item $D = (x_0-R,x_0+R]$
  \item $D = [x_0-R,x_0+R]$
  \end{itemize}
  \end{itemize}
\end{corollario}

\section{Calcolo del raggio di convergenza}
In questa sezione vedremo delle tecniche per il calcolo del raggio di convergenze per le serie di potenze.

Esistono due criteri principali: il criterio della radice e del rapporto.

\begin{teorema}{Criterio della radice}
  Sia data una serie di potenze $\sum_{n=0}^{+\infty} a_n (x-x_0)^n$.\\
  Se esiste il limite
  \[
      L = \lim_{n \to +\infty} \sqrt[n]{\lvert a_n \rvert}
  \]
  allora il raggio di convergenza $R$ della serie è
  \[
      R = \frac{1}{L}
  \]
\end{teorema}
\begin{proof}
  Consideriamo la serie numerica $\sum_{n=0}^{+\infty} \lvert a_n \rvert r^n$, dove $r = \lvert x - x_0 \rvert$.\\
  Applichiamo il criterio della radice per determinare il raggio di convergenza.\\
  Calcoliamo il limite
  \[
      L = \lim_{n \to +\infty} \sqrt[n]{\lvert a_n \rvert r^n} = \lim_{n \to +\infty} \left( \sqrt[n]{\lvert a_n \rvert} \cdot \sqrt[n]{r^n} \right) = \lim_{n \to +\infty} \sqrt[n]{\lvert a_n \rvert} \cdot r
  \]
  Poiché $\sqrt[n]{r^n} = r$, il limite diventa
  \[
      L = r \cdot \lim_{n \to +\infty} \sqrt[n]{\lvert a_n \rvert}.
  \]
  Indichiamo con $L_a = \lim_{n \to +\infty} \sqrt[n]{\lvert a_n \rvert}$.\\
  La serie converge se e solo se $L < 1$, cioè
  \[
      r \cdot L_a < 1 \quad \implies \quad r < \frac{1}{L_a}.
  \]
  Pertanto, il raggio di convergenza è
  \[
      R = \frac{1}{L_a}.
  \]
  Se $L_a = 0$, allora $R = +\infty$, e la serie converge per ogni $x \in \R$.\\
  Se $L_a = +\infty$, allora $R = 0$, e la serie converge solo per $x = x_0$.\\
  Questo conclude la dimostrazione.
\end{proof}
\begin{teorema}{Criterio del rapporto}
  Sia data una serie di potenze $\sum_{n=0}^{+\infty} a_n (x-x_0)^n$.\\
  Se esiste il limite
  \[
      L = \lim_{n \to +\infty} \left\lvert \frac{a_{n+1}}{a_n} \right\rvert
  \]
  allora il raggio di convergenza $R$ della serie è
  \[
      R = \frac{1}{L}
  \]
\end{teorema}
La dimostrazione è analoga a quella appena vista per il criterio della radice.\\
Vediamo un esempio di applicazione del criterio del rapporto.\\
  Consideriamo la serie di potenze
  \[
      \sum_{n=0}^{+\infty} \frac{x^n}{2^n}
  \]
  Applichiamo il criterio del rapporto per determinare il raggio di convergenza.\\
  Calcoliamo il limite
  \[
      L = \lim_{n \to +\infty} \left\lvert \frac{a_{n+1}}{a_n} \right\rvert = \lim_{n \to +\infty} \left\lvert \frac{\frac{1}{2^{n+1}}}{\frac{1}{2^n}} \right\rvert = \lim_{n \to +\infty} \frac{1}{2} = \frac{1}{2}
  \]
  Poiché $L = \frac{1}{2}$, il raggio di convergenza $R$ è
  \[
      R = \frac{1}{L} = 2
  \]
  Quindi, la serie converge per ogni $x$ tale che $\lvert x \rvert < 2$.

  Verifichiamo il comportamento agli estremi dell'intervallo di convergenza.\\
  Per $x = 2$, la serie diventa
  \[
      \sum_{n=0}^{+\infty} \frac{2^n}{2^n} = \sum_{n=0}^{+\infty} 1
  \]
  che è divergente.\\
  Per $x = -2$, la serie diventa
  \[
      \sum_{n=0}^{+\infty} \frac{(-2)^n}{2^n} = \sum_{n=0}^{+\infty} (-1)^n
  \]
  che è anch'essa divergente.\\
  Pertanto, la serie converge per $\lvert x \rvert < 2$ e diverge per $x = \pm 2$.\\

  \section{Studio della funzione somma}
  In questa sezione vedremo come studiare la funzione somma di una serie di potenze.\\
  Ad una serie di potenze $\sum_{n=0}^{+\infty} a_n (x-x_0)^n$ è associata una funzione somma $f(x): (x_0-R,x_0+R) \to \R$ definita come segue: per ogni $x \in (x_0-R,x_0+R)$, si ha
  \[
      f(x) = \sum_{n=0}^{+\infty} a_n (x-x_0)^n
  \]
  Vediamo ora un importante teorema che ci dà diverse informazioni sullo studio della funzione somma:

  \begin{teorema}{}
    Sia data una serie di potenze $\sum_{n=0}^{+\infty} a_n (x-x_0)^n$ con raggio di convergenza $R>0$.\\
    Allora la funzione somma $f$ è derivabile infinite volte in $(x_0-R,x_0+R)$ e:
    \begin{itemize}
    \item $f'(x) = \sum_{n=1}^{+\infty} n a_n (x-x_0)^{n-1}$ in $(x_0-R,x_0+R)$, e ogni serie di potenze così ottenuta ha lo stesso raggio di convergenza $R$ della serie originale. Inoltre $f^{(n)}(x_0)=n!a_n$ per ogni $n$ naturale.
    \item Si può integrare termine a termine centrandosi in $x_0$: $\int f(x) \, dx = \sum_{n=0}^{+\infty} \frac{a_n}{n+1} (x-x_0)^{n+1}$ in $(x_0-R,x_0+R)$, e ogni serie di potenze così ottenuta ha lo stesso raggio di convergenza $R$ della serie originale.
    \end{itemize}
  \end{teorema}



