\chapter{Integrali tripli}

\section{Introduzione}
La costruzione è la stessa degli integrali doppi, ma stavolta siamo in $\R^3$ e non in $\R^2$.\\
Ingredienti: un dominio $\Omega \subseteq \R^3$ e una funzione $f:\Omega \rightarrow \R$ continua e limitata.\\
L'obiettivo è definire l'integrale triplo di $f$ su $\Omega$:
\[
\iiint_{\Omega} f(x,y,z) \, dx \, dy \, dz
\]
e capirne il significato.\\
Per arrivare all'integrale triplo effettuiamo le seguente costruzione in più passaggi:
\begin{enumerate}
  \item Scelgo $\epsilon>0$ e divido $\R^3$ in cubetti di lato $\epsilon$.
  \item Chiamo $C_{\epsilon}$ la famiglia di cubetti interamente contenuti in $\Omega$. (Saranno in numero finito poichè $\Omega$ è limitato).
  \item Campiono $f(x,y,z)$ su ogni cubetto della famiglia, cioè prendo un punto $P_i$ in ogni cubetto $c_i\in C_{\epsilon}$ e calcolo la somma: $S(\epsilon)= \sum_{c_i \in C_{\epsilon}} f(P_i) \cdot \text{volume}(c_i)$.
  \item Si dimostra che esiste finito il limite: $\lim_{\epsilon \to 0} S(\epsilon)$.
\end{enumerate}

\begin{definizione}{Integrale triplo}
  Seguendo quanto detto detto prima, si chiama \textbf{integrale triplo} di $f$ su $\Omega$ il valore:
  \[
  \lim_{\epsilon \to 0} \sum_{c_i \in C_{\epsilon}} f(P_i) \cdot \text{volume}(c_i) = \iiint_{\Omega} f(x,y,z) \, dx \, dy \, dz
  \]
\end{definizione}


\section{Significati degli integrali tripli}
Gli integrali tripli hanno diversi significati e interpretazioni, vediamone alcuni:
\begin{itemize}
  \item Se $f(x,y,z)=1$ allora $\iiint_{\Omega} f(x,y,z) \, dx \, dy \, dz$ è il volume di $\Omega$.
  \item La quantità $\frac{\iiint_{\Omega} f(x,y,z) \, dx \, dy \, dz}{\text{volume}(\Omega)}$ è il valore medio pesato di $f$ su $\Omega$.
  \item Se $f(x,y,z)=x$ allora $\iiint_{\Omega} f(x,y,z) \, dx \, dy \, dz$ è la coordinata $x$ del baricentro di $\Omega$.
  \item Se $f(x,y,z)=y$ allora $\iiint_{\Omega} f(x,y,z) \, dx \, dy \, dz$ è la coordinata $y$ del baricentro di $\Omega$.
  \item Se $f(x,y,z)=z$ allora $\iiint_{\Omega} f(x,y,z) \, dx \, dy \, dz$ è è la coordinata $z$ del baricentro di $\Omega$.
  \item Se $f(x,y,z)$ è una densità di massa allora $\iiint_{\Omega} f(x,y,z) \, dx \, dy \, dz$ è la massa di $\Omega$.
  \item Se $f(x,y,z)$ è una densità di massa allora $\iiint_{\Omega} (x^2 + y^2) f(x,y,z) \, dx \, dy \, dz$ è il momento d'inerzia di $\Omega$ rispetto all'asse $z$.
\end{itemize}


\section{Cambiamento di variabili negli integrali tripli}
Vediamo ora come si comportano gli integrali tripli sotto cambiamento di variabili.\\
In 3 variabili consideriamo il seguente cambiamento di coordinate:\\
$
  x = x(u,v,w) \quad y = y(u,v,w)  \quad z = z(u,v,w)
$\\
con $(u,v,w) \in \Omega \subseteq \R^3$ e $(x,y,z) \in \Omega' \subseteq \R^3$.\\
L'integrale triplo diventa:
\[
\iiint_{\Omega'} f(x,y,z) \, dx \, dy \, dz = \iiint_{\Omega} f(x(u,v,w),y(u,v,w),z(u,v,w)) \cdot |J(u,v,w)| \, du \, dv \, dw
\]
dove $J(u,v,w)$ è il determinante della matrice jacobiana del cambio di variabili:
\[
J(u,v,w) = \begin{vmatrix}
  \frac{\partial x}{\partial u} & \frac{\partial x}{\partial v} & \frac{\partial x}{\partial w}\\
  \frac{\partial y}{\partial u} & \frac{\partial y}{\partial v} & \frac{\partial y}{\partial w}\\
  \frac{\partial z}{\partial u} & \frac{\partial z}{\partial v} & \frac{\partial z}{\partial w}
\end{vmatrix}
\]

\subsection{Coordinate cilindriche}
Consideriamo il cambiamento di coordinate cilindriche:
\[
\begin{cases}
  x = \rho \cos \theta \\
  y = \rho \sin \theta \\
  z = z
\end{cases}
\]
con $\rho \geq 0$, $0 \leq \theta < 2\pi$ e $-\infty < z < \infty$.\\
Il determinante della matrice jacobiana è:
\[
J(\rho, \theta, z) = \begin{vmatrix}
  \frac{\partial x}{\partial \rho} & \frac{\partial x}{\partial \theta} & \frac{\partial x}{\partial z} \\
  \frac{\partial y}{\partial \rho} & \frac{\partial y}{\partial \theta} & \frac{\partial y}{\partial z} \\
  \frac{\partial z}{\partial \rho} & \frac{\partial z}{\partial \theta} & \frac{\partial z}{\partial z}
\end{vmatrix} = \begin{vmatrix}
  \cos \theta & -\rho \sin \theta & 0 \\
  \sin \theta & \rho \cos \theta & 0 \\
  0 & 0 & 1
\end{vmatrix} = \rho
\]
Quindi l'integrale triplo in coordinate cilindriche diventa:
\[
\iiint_{\Omega'} f(x,y,z) \, dx \, dy \, dz = \iiint_{\Omega} f(\rho \cos \theta, \rho \sin \theta, z) \cdot \rho \, d\rho \, d\theta \, dz
\]

\subsection{Coordinate sferiche}
Consideriamo il cambiamento di coordinate sferiche:
\[
\begin{cases}
  x = \rho \sin \phi \cos \theta \\
  y = \rho \sin \phi \sin \theta \\
  z = \rho \cos \phi
\end{cases}
\]
con $\rho \geq 0$, $0 \leq \theta < 2\pi$ e $0 \leq \phi \leq \pi$.\\
Il determinante della matrice jacobiana è:
\[
J(\rho, \theta, \phi) = \begin{vmatrix}
  \frac{\partial x}{\partial \rho} & \frac{\partial x}{\partial \theta} & \frac{\partial x}{\partial \phi} \\
  \frac{\partial y}{\partial \rho} & \frac{\partial y}{\partial \theta} & \frac{\partial y}{\partial \phi} \\
  \frac{\partial z}{\partial \rho} & \frac{\partial z}{\partial \theta} & \frac{\partial z}{\partial \phi}
\end{vmatrix} = \begin{vmatrix}
  \sin \phi \cos \theta & -\rho \sin \phi \sin \theta & \rho \cos \phi \cos \theta \\
  \sin \phi \sin \theta & \rho \sin \phi \cos \theta & \rho \cos \phi \sin \theta \\
  \cos \phi & 0 & -\rho \sin \phi
\end{vmatrix} = \rho^2 \sin \phi
\]
Quindi l'integrale triplo in coordinate sferiche diventa:
\[
\iiint_{\Omega'} f(x,y,z) \, dx \, dy \, dz = \iiint_{\Omega} f(\rho \sin \phi \cos \theta, \rho \sin \phi \sin \theta, \rho \cos \phi) \cdot \rho^2 \sin \phi \, d\rho \, d\theta \, d\phi
\]

\section{Calcolo degli integrali tripli}
Per calcolare gli integrali tripli possiamo procedere in due modi: per fili e per strati.
Per fili significa pensare $\Omega$ come un insieme di fili paralleli ad un asse.\\
Se esistono due funzioni $g_1(x,y,z)$ e $g_2(x,y,z)$ tali che $\Omega = {(x,y,z)\in \R^3|(x,y)\in D, g_1(x,y,z) \leq z \leq g_2(x,y,z)}$ allora:
\[
\iiint_{\Omega} f(x,y,z) \, dx \, dy \, dz = \iint_{D} \left( \int_{g_1(x,y)}^{g_2(x,y)} f(x,y,z) \, dz \right) \, dx \, dy
\]

Per strati invece significa pensare $\Omega$ come un insieme di strati paralleli ad un piano, cioè di sezioni orizontali.\\
Dato $\Omega$ definiamo le sezioni a quota $z$: $S(z) = \{(x,y)\in \R^2 | (x,y,z)\in \Omega\}$.\\
Possiamo dunque ricavare la seguente formula di integrazione per strati:
\[
\iiint_{\Omega} f(x,y,z) \, dx \, dy \, dz = \int_{-\infty}^{+\infty} \left( \iint_{S(z)} f(x,y,z) \, dx \, dy \right) \, dz
\]
ma di fatto gli estremi di integrazione su $z$ sono quelli che definiscono l'altezza di $\Omega$ e si riducono perciò a quelli per cui $S(z)\neq 0$ (esistono di fatto uno $Z_{\text{min}}$ e uno $Z_{\text{max}}$).
