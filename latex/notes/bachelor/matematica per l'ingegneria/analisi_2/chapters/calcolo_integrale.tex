\chapter{Calcolo integrale}
	\section{Integrali doppi}
	\deffname{integrale doppio}
	Sia $f:E \inc \R^2 \rightarrow R$ con $E$ misurabile, poniamo:
	$$\int_{\genfrac{}{}{0pt}{}{E}{*}} f(x,y)  dxdy=sup\{\in ts(x,y)dxdy: s\le f, s \text{ a scalino}\} \quad \text{Integrale inferiore di }f$$
	$$\int_E^* f(x,y)  dxdy=sup\{\int s(x,y)dxdy: f\le s, s \text{ a scalino}\} \quad \text{Integrale superiore di }f$$
	$f$ si dice \textbf{integrabile} secondo Riemann se:
	$$\int_{\genfrac{}{}{0pt}{}{E}{*}} f(x,y)  dxdy = \int_E^* f(x,y)  dxdy$$
	e $\int_E f(x,y)  dxdy$ è detto \textbf{integrale} di Riemann di $f$ su $E$.
	\thhname{Riduzione di Fubini}
	Se $f:[a,b]\times[c,d]\rightarrow \R$ è integrabile, allora:
	$$\int_E f(x,y)  dxdy = \int_{c}^{d} \left(\int_{a}^{b} f(x,y)dx\right) dy$$
	se $\int_a^b f(x,y)dx$ esiste finito, oppure:
	$$\int_E f(x,y)  dxdy = \int_{a}^{b} \left(\int_{c}^{d} f(x,y)dx\right) dy$$
	se $\int_c^d f(x,y)dx$ esiste finito.
	\deffname{Verticalmente convesso}
	$E\inc \R^2$ si dice \textbf{normale}, o semplice, rispetto all'asse $y$, oppure \textbf{verticalmente convesso} se è nella forma:
	$$E=\{(x,y)\in\R^2:a\le x \le b, h(x) \le y \le g(x)\}$$
	\prop
	Sia $E\{(x,y)\in\R^2:a\le x \le b, h(x) \le y \le g(x)\}$, allora:
	$$\int_E f(x,y) dxdy=\int_a^b\left(\int_{h(x)}^{g(x)}f(x,y)dy\right)dx$$
	segue da Fubini.
	\deffname{Orizzontalmente convesso}
	$E\inc \R^2$ si dice \textbf{normale}, o semplice, rispetto all'asse $x$, oppure \textbf{orizzontalmente convesso} se è nella forma:
	$$E=\{(x,y)\in\R^2:a\le y \le b, h(x) \le x \le g(x)\}$$
	\prop
	Sia $E\{(x,y)\in\R^2:c\le y \le d, \tilde h(x) \le x \le \tilde g(x)\}$, allora:
	$$\int_E f(x,y) dxdy=\int_c^d\left(\int_{\tilde h(x)}^{\tilde g(x)}f(x,y)dx\right)dy$$
	segue da Fubini.
	\section{Integrali tripli}
	\deffname{Integrale triplo}
	Analogamente a $\R^2$, in $\R^3$, con $E \inc \R^3$ misurabile, si definisce:
	$$\int_E f(x,y,z)  dxdydz$$
	\propname{Integrazione per fili}
	Se $E=\{(x,y,z)\in\R^3:(x,y)\in D \inc \R^2, h(x,y) \le z \le g(x,y)\}$, allora:
	$$\int_E f(x,y,z) dxdydz=\int_D\left(\int_{h(x)}^{g(x)}f(x,y,z)dz\right)dxdy$$
	\propname{Integrazione per strati}
	Se $E=\{(x,y,z)\in\R^3:a\le z\le b(x,y)\in A_z\}$, allora:
	$$\int_E f(x,y,z) dxdydz=\int_a^b\left(\int_{A_z}f(x,y,z)dxdy\right)dz$$
	\thhname{Th. di Pappo-Guldino}
	Sia $S$ contenuto nel piano $xz$ e sia $E$ il ruotato di $S$ di un angolo $\theta_0$ attorno all'asse $z$, cioè:
	$$E=\{(x,y,z)\in\R^3:x=\rho \cos \theta, y=\rho \sin \theta, (\rho,z)\in S \text{ e } 0 \le \theta \le \theta_0\}$$
	allora:
	$$Vol(E)=\int_e dxdydx=\int_0^{\theta_0}\int_s \rho d\rho dz d\theta=Area(s)(X_g \theta_0)$$
	\section{Cambiamento di variabile}
	\thhname{Cambiamento di variabile negli integrali doppi}
	Sia $\Phi:E \inc \R^2 \rightarrow F \inc \R^2$ con $E$, $F$ aperti misurabili, supponendo $\Phi$ invertibile e sia $\Phi$ che $\Phi^{-1}$ di classe $C^1$, se $\Phi(u,v)=\left(\Phi_1(u,v),\Phi_2(u,v)\right)\in F \ \forall (u,v)\in E$, allora:
	$$\int_E f(x,y) dxdy=\int_{F=\Phi^{-1}(E)}f\left(\Phi_1(u,v),\Phi_2(u,v)\right)|detJ\Phi(u,v)|dudv$$
	\propname{Coordinate polari}
	$$\int_D f(x,y)dxdy \Rightarrow \begin{cases} x = \rho \cos \theta \\ y = \rho \sin \theta \\ detJ\Phi(\rho,\theta)=\rho \end{cases} \Rightarrow \int_{\theta_1}^{\theta_2}\int_{r_1}^{r_2}f(\rho \cos\theta, \rho \sin \theta)\ detJ\Phi\  d\rho d\theta$$
	\propname{Coordinate polari traslate}
	$$\int_D f(x,y)dxdy \Rightarrow \begin{cases} x = x_0 + \rho \cos \theta \\ y =y_0 + \rho \sin \theta \\ detJ\Phi(\rho,\theta)=\rho \end{cases} \Rightarrow \int_{\theta_1}^{\theta_2}\int_{r_1}^{r_2}f(\rho \cos\theta, \rho \sin \theta)\ detJ\Phi\  d\rho d\theta$$
	\propname{Coordinate polari asimmetriche}
	$$\int_D f(x,y)dxdy \Rightarrow \begin{cases} x = a \rho \cos \theta \\ y = b \rho \sin \theta \\ detJ\Phi(\rho,\theta)=ab\rho \end{cases} \Rightarrow \int_{\theta_1}^{\theta_2}\int_{r_1}^{r_2}f(\rho \cos\theta, \rho \sin \theta)\ detJ\Phi\  d\rho d\theta$$
	\propname{Coordinate polari asimmetriche variate}
	$$\int_D f(x,y)dxdy \Rightarrow \begin{cases} x = x_0 + a \rho \cos \theta \\ y = y_0 + b \rho \sin \theta \\ detJ\Phi(\rho,\theta)=ab\rho \end{cases} \Rightarrow \int_{\theta_1}^{\theta_2}\int_{r_1}^{r_2}f(\rho \cos\theta, \rho \sin \theta)\ detJ\Phi\  d\rho d\theta$$
	\propname{Coordinate cilindriche}
	$$\int_D f(x,y,z)dxdydz \Rightarrow \begin{cases} x = \rho \cos \theta \\ y = \rho \sin \theta \\z=z\\ detJ\Phi(\rho,\theta)=\rho \end{cases} \Rightarrow \int_{\theta_1}^{\theta_2}\int_{r_1}^{r_2}\int_{z_1}^{z_2}f(\rho \cos\theta, \rho \sin \theta, z)\ detJ\Phi\  dzd\rho d\theta$$
	\propname{Coordinate sferiche}
	$$\int_D f(x,y,z)dxdydz \Rightarrow \begin{cases} x = \rho \cos \theta \sin \varphi \\ y = \rho \sin \theta \sin \varphi \\z=\rho \cos \varphi\\ detJ\Phi(\rho,\theta)=\rho^2 \sin \varphi \end{cases}$$
	$$\Rightarrow \int_{\varphi_1}^{\varphi_2}\int_{\theta_1}^{\theta_2}\int_{r_1}^{r_2}f(\rho \cos \theta \sin \varphi,\rho \sin \theta \sin \varphi, \rho \cos \varphi)\ detJ\Phi\  d\rho d\theta d\varphi$$
