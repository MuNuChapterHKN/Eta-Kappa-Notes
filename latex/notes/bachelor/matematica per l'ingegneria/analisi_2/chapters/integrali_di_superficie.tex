\chapter{Integrali di superficie}

\section{Introduzione}
Intuitivamente una superficie in $\R^3$ è un insieme di punti dove "ci si può muovere" con due gradi di libertà (a differenza di una curva che ne ha solo uno).

\begin{definizione}{Superficie in $\R^3$}
  Si dice \textbf{superficie} in $\R^3$ un'applicazione $S:\Omega\to\R^3$ dove $\Omega\inc\R^2$ è un insieme aperto (limitato e connesso) e $S$ è di classe $C^1$ su $\Omega$.
\end{definizione}

\section{Superfici}
L'idea è quella di descrivere la geometria della superficie lavorando in $\Omega$ con le coordinate $u$ e $v$.\\
Per fare questo definiamo i vettori fondamentali della superficie:
\begin{definizione}{Vettori fondamentali}
  Sia $S:\Omega\to\R^3$ una superficie in $\R^3$. Allora i \textbf{vettori fondamentali} sono:
  \begin{itemize}
    \item $P_u(u,v)=\frac{\partial S}{\partial u}(u,v)$
    \item $P_v(u,v)=\frac{\partial S}{\partial v}(u,v)$
  \end{itemize}
\end{definizione}
Essi hanno il seguente significato geometrico:
\begin{itemize}
  \item $P_u(u,v)$ è il vettore tangente alla curva $S(u,v)$ ottenuta fissando $v=v_0$ e facendo variare $u$.
  \item $P_v(u,v)$ è il vettore tangente alla curva $S(u,v)$ ottenuta fissando $u=u_0$ e facendo variare $v$.
\end{itemize}
Perciò i vettori fondamentali sono entrambi tangenti alla superficie (a una curva della superficie) in $S(u_0,v_0)$ e se sono linearmente indipendenti allora individuano univocamente il piano tangente alla superficie in $S(u_0,v_0)$.

\begin{definizione}{Superficie regolare}
  Una superficie $S:\Omega\to\R^3$ si dice \textbf{regolare} se i vettori fondamentali sono linearmente indipendenti in ogni punto di $\Omega$.
\end{definizione}

\begin{osservazione}{}
  Se $S$ è regolare allora $N(u_0,v_0) = \frac{P_u(u_0,v_0)\times P_v(u_0,v_0)}{\|P_u(u_0,v_0)\times P_v(u_0,v_0)\|}$ è un versone normale alla superficie in $S(u_0,v_0)$.\\
  Inoltre $-N(u_0,v_0)$ è l'altro versore normale.
\end{osservazione}

\begin{definizione}{Superficie orientata}
  Una superficie $S:\Omega\to\R^3$ si dice \textbf{orientata} se è regolare e se è definita una scelta di versore normale in ogni punto di $\Omega$.
\end{definizione}

\begin{osservazione}{}
  Non tutte le superfici possono essere orientate, cioè non è detto che sia possibile definire un versore normale in ogni punto in modo che il campo risultante sulla superficie sia continuo.
\end{osservazione}

\subsection{Esempio: Sfera di raggio $R$}
  Consideriamo la sfera di raggio $R$ parametrizzata dalle coordinate sferiche $(\theta, \phi)$ con $\theta \in [0, 2\pi]$ e $\phi \in [0, \pi]$. La parametrizzazione è data da:
  \[
  S(\theta, \phi) = \begin{pmatrix}
    R \sin(\phi) \cos(\theta) \\
    R \sin(\phi) \sin(\theta) \\
    R \cos(\phi)
  \end{pmatrix}
  \]
  Calcoliamo i vettori fondamentali:
  \[
  P_\theta(\theta, \phi) = \frac{\partial S}{\partial \theta} = \begin{pmatrix}
    -R \sin(\phi) \sin(\theta) \\
    R \sin(\phi) \cos(\theta) \\
    0
  \end{pmatrix}
  \]
  \[
  P_\phi(\theta, \phi) = \frac{\partial S}{\partial \phi} = \begin{pmatrix}
    R \cos(\phi) \cos(\theta) \\
    R \cos(\phi) \sin(\theta) \\
    -R \sin(\phi)
  \end{pmatrix}
  \]
  Il prodotto vettoriale dei vettori fondamentali è:
  \[
  P_\theta \times P_\phi = \begin{vmatrix}
    \mathbf{i} & \mathbf{j} & \mathbf{k} \\
    -R \sin(\phi) \sin(\theta) & R \sin(\phi) \cos(\theta) & 0 \\
    R \cos(\phi) \cos(\theta) & R \cos(\phi) \sin(\theta) & -R \sin(\phi)
  \end{vmatrix}
  = R^2 \sin(\phi) \begin{pmatrix}
    \sin(\phi)\cos(\theta) \\
    \sin(\phi)\sin(\theta) \\
    \cos(\phi)
  \end{pmatrix}
  \]
  La norma del vettore $P_\theta \times P_\phi$ è:
  \[
  \|P_\theta \times P_\phi\| = R^2 \sin(\phi) \sqrt{(\sin^2(\phi) \cos^2(\theta)) + ( \sin^2(\phi) \sin^2(\theta)) + (\cos^2(\phi))}
  \]\\
  \[ = R^2 \sin(\phi) \sqrt{\sin^2(\phi) + \cos^2(\phi)}
  = R^2 \sin(\phi)
  \]
  Il versore normale è quindi:
  \[
  N(\theta, \phi) = \frac{P_\theta \times P_\phi}{\|P_\theta \times P_\phi\|} = \begin{pmatrix}
    \cos(\theta) \sin(\phi) \\
    \sin(\theta) \sin(\phi) \\
    \cos(\phi)
  \end{pmatrix}
  \]
  che è il versore radiale della sfera.

\subsection{Esempio: Cono con angolo $\phi_0$ fissato}
    Consideriamo un cono con angolo $\phi_0$ fissato parametrizzato dalle coordinate $(\theta, r)$ con $\theta \in [0, 2\pi]$ e $r \in [0, \infty)$. La parametrizzazione è data da:
    \[
    S(\theta, r) = \begin{pmatrix}
      r \sin(\phi_0) \cos(\theta) \\
      r \sin(\phi_0) \sin(\theta) \\
      r \cos(\phi_0)
    \end{pmatrix}
    \]
    Calcoliamo i vettori fondamentali:
    \[
    P_\theta(\theta, r) = \frac{\partial S}{\partial \theta} = \begin{pmatrix}
      -r \sin(\phi_0) \sin(\theta) \\
      r \sin(\phi_0) \cos(\theta) \\
      0
    \end{pmatrix}
    \]
    \[
    P_r(\theta, r) = \frac{\partial S}{\partial r} = \begin{pmatrix}
      \sin(\phi_0) \cos(\theta) \\
      \sin(\phi_0) \sin(\theta) \\
      \cos(\phi_0)
    \end{pmatrix}
    \]
    Il prodotto vettoriale dei vettori fondamentali è:
    \[
    P_\theta \times P_r = \begin{vmatrix}
      \mathbf{i} & \mathbf{j} & \mathbf{k} \\
      -r \sin(\phi_0) \sin(\theta) & r \sin(\phi_0) \cos(\theta) & 0 \\
      \sin(\phi_0) \cos(\theta) & \sin(\phi_0) \sin(\theta) & \cos(\phi_0)
    \end{vmatrix}
    = r \sin(\phi_0) \begin{pmatrix}
      -\cos(\theta) \cos(\phi_0) \\
      -\sin(\theta) \cos(\phi_0) \\
      \sin(\phi_0)
    \end{pmatrix}
    \]
    Il versore normale è quindi:
    \[
    N(\theta, r) = \frac{P_\theta \times P_r}{\|P_\theta \times P_r\|} = \begin{pmatrix}
      -\cos(\theta) \cos(\phi_0) \\
      -\sin(\theta) \cos(\phi_0) \\
      \sin(\phi_0)
    \end{pmatrix}
    \]

\subsection{Esempio: Cilindro di raggio $R$ fissato}
  Consideriamo un cilindro di raggio $R$ fissato parametrizzato dalle coordinate $(\theta, z)$ con $\theta \in [0, 2\pi]$ e $z \in \R$. La parametrizzazione è data da:
  \[
  S(\theta, z) = \begin{pmatrix}
    R \cos(\theta) \\
    R \sin(\theta) \\
    z
  \end{pmatrix}
  \]
  Calcoliamo i vettori fondamentali:
  \[
  P_\theta(\theta, z) = \frac{\partial S}{\partial \theta} = \begin{pmatrix}
    -R \sin(\theta) \\
    R \cos(\theta) \\
    0
  \end{pmatrix}
  \]
  \[
  P_z(\theta, z) = \frac{\partial S}{\partial z} = \begin{pmatrix}
    0 \\
    0 \\
    1
  \end{pmatrix}
  \]
  Il prodotto vettoriale dei vettori fondamentali è:
  \[
  P_\theta \times P_z = \begin{vmatrix}
    \mathbf{i} & \mathbf{j} & \mathbf{k} \\
    -R \sin(\theta) & R \cos(\theta) & 0 \\
    0 & 0 & 1
  \end{vmatrix}
  = \begin{pmatrix}
    R \cos(\theta) \\
    R \sin(\theta) \\
    0
  \end{pmatrix}
  \]
  Il versore normale è quindi:
  \[
  N(\theta, z) = \frac{P_\theta \times P_z}{\|P_\theta \times P_z\|} = \begin{pmatrix}
    \cos(\theta) \\
    \sin(\theta) \\
    0
  \end{pmatrix}
  \]

\section{Superfici grafici di funzioni}
Un caso particolare di superfici sono quelle ottenibili come grafico di una funzione $g(x,y)$ di due variabili, $g:\Omega \inc \R^2  \to \R$.\\
In questo caso la superficie è data da $\Sigma = {(x,y,z)|(x,y)\in \Omega, z=g(x,y)}$.\\
La scelta più naturale sarà in questo caso parametrizzare con $u=x$ e $v=y$ e scegliere $S(u,v)=(u,v,g(u,v))$.\\
I vettori fondamentali sono:
\begin{itemize}
  \item $P_x(x,y)=(1,0,\frac{\partial g}{\partial x})$
  \item $P_y(x,y)=(0,1,\frac{\partial g}{\partial y})$
\end{itemize}
Per calcolare il versore normale si calcola il prodotto vettoriale dei vettori fondamentali e si normalizza il risultato:
$$N(x,y)=\frac{P_x(x,y)\times P_y(x,y)}{\|P_x(x,y)\times P_y(x,y)\|}$$, che in questo caso vale:
$$N(x,y)=\frac{(-\frac{\partial g}{\partial x},-\frac{\partial g}{\partial y},1)}{\sqrt{1+(\frac{\partial g}{\partial x})^2+(\frac{\partial g}{\partial y})^2}}=\frac{(-\frac{\partial g}{\partial x},-\frac{\partial g}{\partial y},1)}{\sqrt{1+\|\nabla g\|^2}}$$.

\begin{osservazione}{}
  Un grafico è sempre orientabile.
\end{osservazione}

\section{Integrali su superfici}
\begin{definizione}{Integrale su superficie}
  Sia $S:\Omega\to\R^3$ una superficie regolare e orientata. Sia $f:S(\Omega)\to\R$ una funzione continua. Allora l'\textbf{integrale di superficie} di $f$ su $S$ è:
  $$\int_{S} f dS = \int_{\Omega} f(x(u,v),y(u,v),z(u,v))\|P_u\times P_v\|dudv$$
\end{definizione}

Il significato è quello di:
\begin{enumerate}
  \item Scegliere una partizione di $\Omega$ e calcolare l'integrale di $f$ su ogni pezzo di superficie.
  \item Sommare i risultati.
  \item Passare al limite.
\end{enumerate}
Alcune possibili interpretazioni geometriche sono:
\begin{itemize}
\item In particolare se $f=1$ allora l'integrale di $f$ su $S$ è l'area di $S$.\\
\item Se $f(x,y,z)=x$ allora l'integrale di $f$ su $S$ è la media di $x$ su $S$, ovvero la coordinata x del baricentro.\\
\item In generale $\frac{\iint_S f dS}{\text{Area}(S)}$ è la media di $f$ su $S$.\\
\end{itemize}

\section{Flusso di un campo vettoriale attraverso una superficie}\label{sec:flusso-superficie}
C'è un caso particolare che rientra nella definizione precedente, ovvero il flusso di un campo vettoriale attraverso una superficie.\\
\begin{definizione}{Flusso di un campo}
  Sia $\vec F:\R^3\to\R^3$ un campo vettoriale e sia $S:\Omega\to\R^3$ una superficie regolare e orientata. Allora si può scegliere $f(x,y,z)=N\cdot \vec F$. Allora si definisce il \textbf{flusso} di $\vec F$ attraverso $S$:
  $$\int_{S} \vec F \cdot N d S = \int_{S} \vec F \cdot \vec (dS) = \int_{\Omega} \vec F(x(u,v),y(u,v),z(u,v))\cdot (P_u\times P_v)dudv$$
\end{definizione}
\begin{osservazione}{}
  $\|P_u \times P_v\|$ si semplifica e di fatto non compare.
\end{osservazione}

Notiamo anche che se $S$ è il grafico di una funzione $g(x,y)$ allora la formula per l'integrale di $f(x,y,z)$ su $S$ diventa:
$$\int_{S} f dS = \int_{\Omega} f(x,y,g(x,y))\sqrt{1+\|g(x,y)\|^2}dxdy$$.\\
Inoltre, quando $S$ è il grafico di una funzione $g(x,y)$ allora il flusso di $\vec F$ attraverso $S$ è:
$$\int_{S} \vec F \cdot N d S = \int_{\Omega} \vec F(x,y,g(x,y))\cdot \left(-\frac{\partial g}{\partial x},-\frac{\partial g}{\partial y},1\right)dxdy$$.\\
