\chapter{Integrali curvilinei e di superficie}
	\section{Definizione delle curve}
	\deff
	Una funzione $\gamma : [ a, b ] \subseteq \R \Rightarrow \R^m$ si dice \textbf{curva} in $\R^m$ se è continua.\\
	\\
	L'insieme $\gamma ([a,b])= \{\gamma(t)\in \R^m : t \in [a,b] \}$ si dice sostegno di $\gamma$, cioè il sostegno è l'immagine di $\gamma$
	\deff
	$\gamma:[a,b]\Rightarrow \R^m$, curva, si dice \textbf{semplice} se vale l'implicazione:
	$$t \in [a, b],\ s \in (a, b)\ \text{e} \ s\ne t \Rightarrow \gamma(t)\ne \gamma(s)$$
	$\gamma:[a,b]\Rightarrow \R^m$, curva, si dice \textbf{chiusa} se vale l'implicazione:
	$$\gamma(a)=\gamma(b)$$
	Se $\gamma:[a,b]\Rightarrow \R^m$ è una curva chiusa e semplice si dice \textbf{curva di Jordan}
	\deff
	Sia $\gamma : [a, b] \Rightarrow \R^m$ curva con $\gamma(t) = (\gamma_1(t),\dots, \gamma_m(t))$, $\gamma$ si dice \textbf{regolare} se $\gamma \in C^1([a, b])$ e se $\gamma'(t)\ne \vec{0}, \ \forall t \in [a, b]$.\\
	\\
	$\gamma'(t)$ si dice \textbf{vettore tangente} a $\gamma$ in $\gamma(t)$, oppure vettore derivata o vettore velocità.\\
	\\
	Se $\gamma'(t)\ne \vec 0 \  \forall t$ significa che $\gamma(t)$ non si fermerà mai.
	\deff
	$\gamma:[a,b]\rightarrow \R^m$, curva, si dice \textbf{regolare a tratti} se $\exists a=t_0<t_1<\dots<t_{n-1}<t_{n}=b$ tale che $\gamma \in C^1([t_{k-1}, t_k]) \ \forall k \in {1, \dots, n}$ e $\gamma'(t)\ne \vec 0 \ \forall t \ne t_k$
	\section{Integrali curvilinei di prima specie (integrali di funzione su curve)}
	\deff
	Sia $\gamma:[a, b] \rightarrow \R^m$ curva regolare a tratti, sia $f:D\subseteq \R^m \rightarrow \R$ tale che $\gamma([a,b])\subseteq D$, si pone:
	$$\int_\gamma f ds = \int_\gamma \gamma f = \int_a^b f(\gamma(t))|\gamma'(t)|dt$$
	se l'integrale esiste.\\
	\\
	Se $\gamma$ è di Jordan si scrive $\oint_\gamma f ds = \int_\gamma fds$.\\
	\\
	Si pone $L(\gamma)=\int_\gamma ds=\int_{a}^{b}|\gamma'(t)|dt$ lunghezza di $\gamma$.
	\prop
	Sia $\gamma:[a, b] \rightarrow \R^m$ curva regolare a tratti e sia $\phi : [c, d] \rightarrow [a, b] \ C^1$ strettamente monotone e suriettiva e sia $\tilde{\gamma} : [c, d]\rightarrow \R^m$ definita come $\tilde{\gamma}=\gamma ( \phi (r) )$, ovvero  $\phi$ è un riscalamento del parametro t e $\tilde{\gamma}$ ha la stessa traiettoria di $\gamma$ ma la percorre in modo diverso, allora:
	$$\int_{\tilde{\gamma}} fds=\int_\gamma fds$$
	Quindi gli integrali di prima specie non dipendo né dalla parametrizzazione $\gamma$, né dal verso di percorrenza, ma dipendono solo dal sostegno $\Gamma=\gamma([a,b])$
	\deff
	Sia $\Gamma \subseteq \R^m$ che si sostegno si una curva $\gammadef$ regolare a tratti semplice, allora si pone:
	$$\int_\Gamma fds= \int_\gamma fds$$
	Allo stesso modo $L(\Gamma)=L(\gamma)$
	\deff
	Sia $\gamma :  [a,b]\rightarrow \R ^3$ curva regolare a tratti semplice, alla quale è associata una \textbf{funzione di densità} (di massa) $\mu:  \Gamma =\gamma([a,b])\subseteq \R^3 \rightarrow \R$, si dice \textbf{massa} di $\gamma$ (o di $\Gamma$) il numero:
	$$M(\Gamma)=m(\Gamma)=\int_\Gamma \mu \ d\sigma$$
	Il \textbf{baricentro} di $\Gamma$ è il punto $(x_G, y_G, z_G)$ tale che:
	$$x_G=\frac{1}{M(\Gamma)}\int_\Gamma x \mu (x,y,z) d \sigma$$
	$$y_G=\frac{1}{M(\Gamma)}\int_\Gamma y \mu (x,y,z) d \sigma$$
	$$z_G=\frac{1}{M(\Gamma)}\int_\Gamma z \mu (x,y,z) d \sigma$$
	\section{Integrali curvilinei di seconda specie (lavori)}
	Sia $\gammadef$ curva regolare a tratti e  sia $F:D\inc \R^m \rightarrow \R^m$ con $\gamma([a,b])\inc D$, si dice \textbf{lavoro di} $\vec F$ \textbf{lungo} $\gamma$ (o integrale di seconda specie) il numero:
	$$\int_\gamma F \cdot dl = \int_a^bF(\gamma(t))\cdot \gamma'(t) dt$$
	Se $\gamma$ è di Jordan si scrive $\oint_\gamma F \cdot dl = \int_\gamma F \cdot dl$ detto \textbf{circuitazione di $F$ lungo $\gamma$}.\\
	\\
	Sia $\vec r(t)= \frac{\gamma'(t)}{|\gamma'(t)|}\ \forall t \in [a,b]$ detto \textbf{versore tangente a} $\gamma$, si ha:
	$$\int_\gamma F \cdot dl = \int_\gamma (F\cdot \vec r) d\sigma$$
	\prop
	Sia $\gamma:[a, b] \rightarrow \R^m$ curva regolare a tratti e sia $\phi : [c, d] \rightarrow [a, b] \ C^1$ strettamente crescente e suriettiva e sia $\tilde{\gamma} : [c, d]\rightarrow \R^m$ definita come $\tilde{\gamma}=\gamma ( \phi (r) )$, ovvero $\tilde{\gamma}$ percorre il sostegno di $\gamma$ lo stesso numero di volte nello stesso verso, allora:
	$$\int_{\tilde{\gamma}} F \cdot dl = \int_\gamma F \cdot dl$$
	Nel caso in cui $\hat \gamma$, definita come $\tilde{\gamma}$, sia strettamente decrescente, e quindi percorra il sostegno di $\gamma$ lo stesso numero di volte ma in verso opposto, allora:
	$$\int_{\hat{\gamma}} F \cdot dl = - \int_\gamma F \cdot dl$$
	\section{Integrali di superficie}
	\deffname{Superfici parametrizzate}
	Una funzione $\sigma:\bar{A} \inc \R^2 \rightarrow \R^3$ con $A \inc \R^2$ aperto e con $\sigma(u,v)=(\sigma_1(u,v),\sigma_2(u,v),\sigma_3(u,v))$ si dice \textbf{superficie parametrizzata} se $\sigma \in C^1$, $\sigma$ è iniettiva in $A$ e la matrice Jacobiana ha rango 2, allora:
	$$\Sigma=\sigma(\bar{A})$$
	ed è detto \textbf{sostegno} di $\sigma$.
	\deff
	Le derivate parziali di $\sigma$ in $u$ e $v$ generano in piano in $\R^3$ ,detto \textbf{piano tangente} a $\Sigma$ in $\sigma(u_0,v0)$, che ha equazione:
	$$\Pi(u,v)=\sigma(u_0,v_0)+\frac{\partial \sigma}{\partial u}(u_0,v_0)(u-u_0)+\frac{\partial \sigma}{\partial v}(u_0,v_0)(v-v_0)$$
	Definiamo il \textbf{vettore normale} a $\Sigma$ in $\sigma(u_0,v0)$:
	$$\vec N(u_0, v_0)=\frac{\partial \sigma}{\partial u} \times \frac{\partial \sigma}{\partial v}$$
	da cui possiamo ricavare il versore $\vec n (u_0,v_0) = \frac{\vec N (u_0,v_0)}{|\vec N (u_0,v_0)|}$.
	\deffname{Superfici cartesiane}
	Sia $g:\bar A \inc \R^2 \rightarrow \R$, con $A$ aperto e $g \in C^1(A)$, data una superficie $\sigma(u,v)=(u,v,g(u,v))$ si ha che $\Sigma = \sigma (\bar A)$ è il grafico di $g$. Sappiamo quindi che:
	$$\Sigma  = \sigma(\bar A)=Gr(g)=\{(x,y,z)\in \R^3:(x,y)\in \bar A , z = g(x,y)\}$$
	$$\vec N(x, y)=\frac{\partial \sigma}{\partial u} \times \frac{\partial \sigma}{\partial v}=\left(-\frac{\partial g}{\partial x}, -\frac{\partial g}{\partial y}, 1 \right)$$
	$$|\vec N (x,y)|=\sqrt{1+|\nabla g|^2}$$
	\deffname{Integrali di superficie di prima specie}
	Sia $\sigma:\bar{A} \inc \R^2 \rightarrow \R^3$ superficie con $A$ misurabile, poniamo:
	$$\int_\sigma f d\sigma = \int_\sigma f(x,y,z)d\sigma =\int_{\bar A} f(x,y,g(x,y))|\vec N (x,y)|dxdy$$
	\prop
	L'integrale è indipendente dalla parametrizzazione, infatti se due superfici sono diverse ma hanno stesso sostegno l'integrale non cambia.
	\section{Flusso di un campo vettoriale}
	\deffname{Superfici orientabili}
	Una superficie $\sigma:\bar{A} \inc \R^2 \rightarrow \R^3$ si dice \textbf{orientabile} se la funzione $\Sigma \rightarrow \R^3$ è continua.
	\oss
	Se due superfici orientabili con lo stesso sostegno allora i versori normali possono essere solo uguali od opposti tra loro.
	\prop
	Se $\Sigma = \sigma (\bar A)$ è il sostegno di $\sigma:\bar{A} \inc \R^2 \rightarrow \R^3$ e se anche $\Sigma=\partial \Omega$ con $\Omega$ aperto, connesso e limitato, allora $\Sigma$ è orientabile. \\
	\\
	In particola re $\vec n$ punta verso l'interno di $\Omega$ allora viene detto \textbf{entrante}, altrimenti, se punta verso l'esterno, viene detto \textbf{uscente}. Il verso positivo del vettore è quello uscente dalla superficie.
	\deffname{Flusso di un campo vettoriale}
	Sia $\sigma:\bar{A} \inc \R^2 \rightarrow \R^3$ superficie orientabile e sia $F:D\inc \R^3 \rightarrow \R^3$ campo vettoriale continuo con $\Sigma = \sigma (\bar A)\inc D$, allora il \textbf{flusso} di $\vec F$ attraverso $\sigma$ nella direzione $\vec n$ è:
	$$\int_\sigma (\vec F \cdot \vec n) d\sigma = \int_{\bar A}F(\sigma(u,v))\cdot \frac{\vec N(u,v)}{|\vec N(u,v)|}|\vec N(u,v)|dudv=\int_D F(x,y,g(x,y))\cdot \vec N(x,y)dxdy$$
	\deff
	Sia $\Sigma$ una superficie orientabile, se $\Sigma=\partial \Omega$ con $\Omega$ aperto, connesso, limitato e misurabile allora:
	$$\int_{\partial \Omega} \vec F \cdot \\vec n$$
	per convenzione denota il \textbf{flusso uscente} da $\Omega$ , cioè il flusso di $F$ lungo $\vec n$ uscente.
	\thhname{Th. della divergenza di Gauss}
	Sia $\Omega \inc \R^3$ connesso, limitato e misurabile, tale che sia $\Sigma=\partial \Omega$ una superficie orientata con $\vec n$ uscente, sia $\vec F:D \inc \R^3 \rightarrow \R^3$, allora:
	$$\int_{\partial \Omega}(\vec F \cdot \vec n)d\sigma=\int_\Omega div \vec F dxdydz$$
	\deffname{Aperto con bordo}
	Sia $D \inc \R^2$ aperto, connesso e limitato. $D$ si dice \textbf{aperto con bordo} se $\partial D$ è l'unione di un numero finito di sostegni di curve di Jordan regolari a tratti a due a due disgiunti. Su $\partial D$ si definisce come orientazione positiva quella per cui percorrendo $\partial D$ vedo $D$ a sinistra.
	\thhname{Formula di Green nel piano}
	Sia $\vec F:E\ inc \R^2 \rightarrow \R^2$ di classe $C^1$ con $E$ aperto, sia $D\inc \R^2$ aperto con bordo con $\bar D \inc E$, allora:
	$$\int_{\partial D} \vec F \cdot dl = \int_D \left(\frac{\partial F_2}{\partial x}-\frac{\partial F_1}{\partial y}\right)dxdy$$
	\thhname{Th. del rotore di Stokes}
	Sia $D \inc \R^2$ aperto con bordo e sia $\sigma : \bar D \inc \R^2 \rightarrow \R^3$ superficie orientabile iniettiva si $\bar D$, chiamiamo $\partial \sigma = \sigma (\partial D)$, l'immagine di $\partial D$ tramite $\sigma$, \textbf{frontiera della superficie} $\sigma$. Orientiamo $\partial \sigma$ con l'orientazione indotta dall'orientazione positiva di $\partial D$, ovvero quando il versore normale $\vec n$ percorre $\partial \sigma$ vede $\sigma$ a sinistra. Allora:
	$$\int_{\partial\sigma} \vec F \cdot dl = \int_\sigma \left(rot \vec F \cdot \vec n\right)d\sigma$$
	ovvero il lavoro di $F$ lungo $\partial \sigma$ è uguale al flusso del rotore di $\vec F$ attraverso $\sigma$.
	\section{Campi conservativi}
	\deffname{Campi conservativi}
	Un campo $\vec F: \Omega \inc \R^n \rightarrow \R^n$ continuo con $\Omega$ aperto si dice \textbf{conservativo} se $\exists \ \Phi : \Omega \inc \R^n \rightarrow \R$ tale che $\nabla \Phi = \vec F$, in tal caso $\Phi$ si dice \textbf{potenziale} di $\vec F$, oppure primitiva di $\vec F$.
	\prop
	Sia $\gamma:[a,b]\rightarrow \Omega \inc \R^n$ curva regolare a tratti, se $F$ è \textbf{conservativo} si ha che il lavoro lungo $\gamma$ è:
	$$\int_\gamma \vec F \cdot dl = \int_a^b \vec F(\gamma (t))\cdot \gamma'(t)dt=\int_{a}^{b} \nabla \Phi (\gamma(t))\cdot \gamma'(t)dt=\Phi(\gamma(b))-\Phi(\gamma(a))$$
	ovvero se $F$ è conservativo ($F\nabla \Phi$) il lavoro lungo una curva $\gamma$ dipende sola dal punto iniziale e da quello finale.
	\thh
	Sia $\vec F: \Omega \inc \R^n \rightarrow \R^n$ conservativo ($F\nabla \Phi$), allora per ogni curva $\gamma:[a,b]\rightarrow \Omega \inc \R^3$ si ha:
	$$\int_\gamma \vec F \cdot dl =\Phi(\gamma(b))-\Phi(\gamma(a))$$
	Se $\gamma_1$ e $\gamma_2$ sono due curve con stesso punto iniziale e finale:
	$$\int_{\gamma_2} \vec F \cdot dl = \int_{\gamma_1} \vec F \cdot dl$$
	Se $gamma$ è chiusa allora:
	$$\int_\gamma \vec F \cdot dl =0$$
	\thh
	Sia $\vec F: \Omega \inc \R^n \rightarrow \R^n$ continuo con $\Omega$ connesso, allora le tre affermazioni sono equivalenti:
	$$\begin{Bmatrix}
		F \text{ è conservativo}\\
		  \Updownarrow\\
		 \int_{\gamma_2} \vec F \cdot dl = \int_{\gamma_1} \vec F \cdot dl\\
		 (\text{con } \gamma_1,\gamma_2 \text{ curve con stesso inizio e fine})\\
		 \Updownarrow\\
		 \int_\gamma \vec F \cdot dl =0\\
		 (\text{con } \gamma \text{ chiusa})
	\end{Bmatrix}$$
	\thh
	Sia $F: \Omega \subseteq \mathbb{R}^n \rightarrow \mathbb{R}^n$ di classe $C^1$ ($\Sigma$ aperto), se $F$ è conservativo allora:
	$$\frac{\partial F_i}{\partial x_j} = \frac{\partial F_j}{\partial x_i} \qquad \forall i, j \in \{ 1,\dots, n\} \text{ in } \Omega$$
	\deff
	Se $rotF=0$, $F$ si dice \textbf{irrotazionale}.
	\prop
	Se $F$ è conservativo e di classe $C^1$ $\Rightarrow$ $F$ è irrotazionale.\\
	\\
	In generale la $\Rightarrow$ non si può invertire.
	\deff
	Un aperto connesso $\Omega$ si dice \textbf{semplicemente connesso} se ogni curva $\gamma$ di Jordan può essere deformata con continuità fino a contrarsi ad un punto, rimanendo sempre dentro $\Omega$.\\
	\\
	In $\mathbb{R}^2$ un insieme semplicemente connesso è un insieme "privo di buchi". Sono esempi di insiemi semplicemente connessi: tutti gli aperti convessi, tutti gli aperti limitati con frontiera costituita da un'unica curva e il piano privato di una semiretta. Sono esempi di insiemi non semplicemente connessi: tutti gli aperti privati di un punto, le corone circolari e il piano privato di una retta (perché non è connesso).\\
	\\
	In $\mathbb{R}^3$ sono esempi di insiemi semplicemente connessi: tutti gli aperti convessi, lo spazio privato di un punto e le corone sferiche. Sono esempi di insiemi non semplicemente connessi: lo spazio privato di una retta o di un piano.
	\thh
	Sia $\Omega \subseteq \mathbb{R}^3$ un aperto semplicemente connesso, e sia $F$ un campo
	vettoriale di classe $C^1 $, se $rotF=0$ (irrotazionale) allora $F$ è conservativo.
