\chapter{Stokes}
\section{Teorema di Stokes}
In quest'altro breve capitolo vedremo il teorema di Stokes, che è un'estensione del teorema di Green a campi vettoriali in tre dimensioni.

Il teorema di Stokes permette di legare tra loro il flusso del rotore di un campo vettoriale, attraverso una superficie, la circuitazione del campo stesso lungo il bordo della superficie.

Vediamo innanzitutto una definizione preliminare:\\


\begin{definizione}{Compatibilità delle orientazioni}
  Dati una superficie orientata $ S $ e il suo bordo orientato $\partial S $, diciamo che le due orientazioni sono compatibili se percorrendo $\gamma$ dal lato di $N$, $S$ risulta essere a sinistra.
\end{definizione}

Ora possiamo enunciare il teorema di Stokes:
\begin{teorema}{Teorema di Stokes}
    Sia $ S $ una superficie orientata con bordo $ \partial S $ orientato compatibilmente. Sia $ \vec{F} $ un campo vettoriale di classe $ C^1 $ in un intorno aperto di $ S $. Allora vale la seguente formula:
    \[
        \int_S \nabla \times \vec{F} \cdot \hat{N} \, dS = \int_{\partial S} \vec{F} \cdot d\vec{r}
    \]
\end{teorema}

Un'importante applicazione del teorema di Stokes è la seguente:\\
Se $S_1$ e $S_2$ sono due superfici orientate con bordo $\partial S_1=\partial S_2 =\gamma$ orientati compatibilmente, allora:
\[
    \int_{\gamma} \vec{F} \cdot d\vec{r}=
\int_{S_1} \nabla \times \vec{F} \cdot \hat{N} \, dS + \int_{S_2} \nabla \times \vec{F} \cdot \hat{N} \, dS
\]

\subsection{Esempio di applicazione del teorema di Stokes}

Consideriamo il campo vettoriale $\vec{F} = (y, -x, z)$ e la superficie $S$ definita dal paraboloide $z = 1 - x^2 - y^2$ con $z \geq 0$. Calcoliamo il flusso del rotore di $\vec{F}$ attraverso $S$ e la circuitazione di $\vec{F}$ lungo il bordo di $S$.

Innanzitutto, calcoliamo il rotore di $\vec{F}$:
\[
\nabla \times \vec{F} = \left( \frac{\partial z}{\partial y} - \frac{\partial (-x)}{\partial z}, \frac{\partial x}{\partial z} - \frac{\partial z}{\partial x}, \frac{\partial (-x)}{\partial y} - \frac{\partial y}{\partial x} \right) = (1, 0, -2)
\]

La normale unitaria alla superficie $S$ è data da:
\[
\hat{N} = \frac{\nabla (1 - x^2 - y^2 - z)}{|\nabla (1 - x^2 - y^2 - z)|} = \frac{(-2x, -2y, -1)}{\sqrt{4x^2 + 4y^2 + 1}}
\]

Il flusso del rotore di $\vec{F}$ attraverso $S$ è quindi:
\[
\int_S \nabla \times \vec{F} \cdot \hat{N} \, dS = \int_S (1, 0, -2) \cdot \frac{(-2x, -2y, -1)}{\sqrt{4x^2 + 4y^2 + 1}} \, dS = \int_S \frac{-2x + 2}{\sqrt{4x^2 + 4y^2 + 1}} \, dS
\]

Passiamo alle coordinate polari, dove $x = r \cos \theta$ e $y = r \sin \theta$. La superficie $S$ è definita da $z = 1 - r^2$ con $0 \leq r \leq 1$ e $0 \leq \theta < 2\pi$. Il differenziale di superficie in coordinate polari è $dS = \sqrt{1 + (\frac{\partial z}{\partial r})^2} \, r \, dr \, d\theta = \sqrt{1 + 4r^2} \, r \, dr \, d\theta$.

Quindi l'integrale diventa:
\[
\int_S \frac{-2r \cos \theta + 2}{\sqrt{4r^2 + 4r^2 + 1}} \sqrt{1 + 4r^2} \, r \, dr \, d\theta = \int_0^{2\pi} \int_0^1 \frac{-2r \cos \theta + 2}{\sqrt{4r^2 + 1}} \sqrt{4r^2 + 1} \, r \, dr \, d\theta
\]

Semplificando, otteniamo:
\[
\int_0^{2\pi} \int_0^1 (-2r^2 \cos \theta + 2r) \, dr \, d\theta
\]

Separiamo gli integrali:
\[
\int_0^{2\pi} \int_0^1 -2r^2 \cos \theta \, dr \, d\theta + \int_0^{2\pi} \int_0^1 2r \, dr \, d\theta
\]

Il primo integrale è nullo perché l'integrale di $\cos \theta`$ su $[0, 2\pi]$ è zero:
\[
\int_0^{2\pi} \cos \theta \, d\theta = 0
\]

Quindi rimane:
\[
\int_0^{2\pi} \int_0^1 2r \, dr \, d\theta = 2 \int_0^{2\pi} d\theta \int_0^1 r \, dr = 2 \cdot 2\pi \cdot \left[ \frac{r^2}{2} \right]_0^1 = 2 \cdot 2\pi \cdot \frac{1}{2} = 2\pi
\]

Abbiamo quindi:
\[
\int_S \nabla \times \vec{F} \cdot \hat{N} \, dS = 2\pi
\]

Calcoliamo ora la circuitazione di $\vec{F}$ lungo il bordo di $S$. Il bordo di $S$ è la circonferenza $x^2 + y^2 = 1$ nel piano $z = 0$. Parametrizziamo il bordo come $\vec{r}(t) = (\cos t, \sin t, 0)$ con $t \in [0, 2\pi]$. La circuitazione di $\vec{F}$ lungo il bordo è:
\[
\int_{\partial S} \vec{F} \cdot d\vec{r} = \int_0^{2\pi} \vec{F}(\cos t, \sin t, 0) \cdot \frac{d\vec{r}}{dt} \, dt
\]
\\
\[= \int_0^{2\pi} (\sin t, -\cos t, 0) \cdot (-\sin t, \cos t, 0) \, dt = \int_0^{2\pi} (\sin^2 t + \cos^2 t) \, dt = -2\pi
\]
Tuttavia per avere una parametrizzazione coerente dovremmo considerare il flusso con la normale opposta al calcolo effettuato in precedenza così da avere un riaultato corretto.
Abbiamo quindi verificato il teorema di Stokes per questo esempio.


