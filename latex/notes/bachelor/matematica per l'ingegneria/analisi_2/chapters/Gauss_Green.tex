\chapter{Gauss-Green}
\section{Teorema di Gauss-Green}
Questo breve capitolo è interamente dedicato al teorema di Gauss-Green anche noto come teorema della divergenza.\\
Diamo intanto la definizione di divergenza di un campo vettoriale.
\begin{definizione}{Divergenza di un campo vettoriale}
  Sia $\vec{F} \in C^1(\Omega, \R^3)$ un campo vettoriale definito su un insieme aperto $\Omega \inc \R^3$. La divergenza di $\vec{F}$ è la funzione $\nabla \cdot \vec{F} : \Omega \to \R$ definita come:
  $$
    \nabla \cdot \vec{F} = \frac{\partial F_1}{\partial x} + \frac{\partial F_2}{\partial y} + \frac{\partial F_3}{\partial z}
  $$
\end{definizione}
Ora possiamo enunciare il teorema di Gauss-Green.
\begin{teorema}{Teorema di Gauss-Green}
 Sia $\Omega \inc \R^3$ un solido (aperto e limitato) la cui frontiera $\partial \Omega$ è una superficie regolare orientata con la scelta della normale uscente. Sia $\vec{F} \in C^1(\Omega, \R^3)$ un campo vettoriale. Allora vale la seguente formula:
  $$
    \int_{\Omega} \nabla \cdot \vec{F} \, dV = \int_{\partial \Omega} \vec{F} \cdot \vec{n} \, dS$$
  dove $\nabla \cdot \vec{F}$ è la divergenza di $\vec{F}$ e $\vec{n}$ è il vettore normale alla superficie $\partial \Omega$.
\end{teorema}

Vediamone alcune applicazioni.
\begin{itemize}
\item Se ad esempio $\vec{F} = (0, 0, z)$, allora $\nabla \cdot \vec{F} = 1$ e il teorema di Gauss-Green diventa:
  $$
    \int_{\Omega} 1 \, dV = \int_{\partial \Omega} z \, dS
  $$
  cioè l'integrale del volume di $\Omega$ è uguale all'integrale della funzione $z$ sulla superficie $\partial \Omega$.
\item Possiamo usarlo per calcolare il volume di una sfera considerando un campo vettoriale $F(x,y,z)=(x,y,z)$: la divergenza di $F$ è $3$ e quindi:
  $$
    \int_{\Omega} 3 \, dV = 3 \cdot \text{Volume}(\Omega) = \int_{\partial \Omega} \vec{F} \cdot \vec{n} \, dS = \int_{\partial \Omega} (x,y,z) \cdot \vec{n} \, dS
  $$
  ma $(x,y,z) \cdot \vec{n} = r$ e quindi:
  $$
    3 \cdot \text{Volume}(\Omega) = \int_{\partial \Omega} r \, dS
  $$
  e quindi:
  $$
    \text{Volume}(\Omega) = \frac{1}{3} \int_{\partial \Omega} r \, dS
  $$
  Per una sfera di raggio $R$, la superficie $\partial \Omega$ è una sfera di raggio $R$ e quindi:
  $$
    \text{Volume}(\Omega) = \frac{1}{3} \int_{\partial \Omega} r \, dS = \frac{1}{3} \int_{\partial \Omega} R \, dS = \frac{R}{3} \int_{\partial \Omega} dS
  $$
  L'integrale $\int_{\partial \Omega} dS$ è semplicemente l'area della superficie della sfera, che è $4 \pi R^2$. Quindi:
  $$
    \text{Volume}(\Omega) = \frac{R}{3} \cdot 4 \pi R^2 = \frac{4 \pi R^3}{3}
  $$
  che è la ben nota formula per il volume di una sfera di raggio $R$.
\end{itemize}

In generale nei casi in cui la divergenza del campo ha un'espressione semplice, il teorema di Gauss-Green può essere molto utile per calcolare integrali di volume.


