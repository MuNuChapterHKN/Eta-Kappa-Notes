\chapter{Green}
Questo capitolo sarà interamente dedicato al teorema di Green nel piano. Esso è un caso particolare del teorema del rotore (o teorema di Stokes) che verrà affrontato prossimamente.\\

Il teorema di Green consente di caclolare la circuitazione di un campo riducendola a un integrale doppio.

\section{Teorema di Green}
\begin{teorema}{Teorema di Green}
Sia $\Omega \inc \R^2$ aperto limitato tale che la sua frontiera è l'unione di $N$ curve chiuse disgiunte $\gamma_1, \dots, \gamma_N$  e sia $\vec F:\Omega\rightarrow \R^2$ di classe $C^1$, allora:
$$ \int_\Omega \left(\frac{\partial F_2}{\partial x}-\frac{\partial F_1}{\partial y}\right) dxdy = \sum_{i=1}^{N} \int_{\gamma_i} \vec{F} \cdot d\vec{r} $$,\\dove ogni $\gamma_i$ è orientata in modo da avere $\Omega$ alla sua sinistra.
\end{teorema}

Il teorema si può applicare in vari modi ad esempio:
\begin{itemize}
\item Calcolare la circuitazione di un campo lungo una curva chiusa.
\item Confrontare tra loro due circuitazioni (specialmente quando $\vec F$ è irrotazionale ma non conservativo).
\end{itemize}

Inoltre grazie al teorema si ricava un utile fatto generale:
\begin{corollario}{}
  Se ho due curve chiuse $\gamma_1$ e $\gamma_2$ tali che $\gamma_1$ è interna a $\gamma_2$ allora: $$\int_{\gamma_1} \vec F \cdot d\vec r = \int_{\gamma_2} \vec F \cdot d\vec r$$\\
  per ogni campo $\vec F$ irrotazionale (nella sezione di piano che contiene le curve).
\end{corollario}

Applichiamo ora il teorema al calcolo dell'area racchiusa tra curve chiuse.\\
Se scelgo un campo $\vec F$ tale che $\frac{\partial F_2}{\partial x}-\frac{\partial F_1}{\partial y}=1$ (ad esempio $\vec F = (0,x)$)allora ottengo l'area di $\Omega$ come circuitazione di $\vec F$ lungo il bordo di $\Omega$.\\
Riassumendo:
\begin{teorema}{}
  Se $\gamma(t)= (x(t), y(t)), t \in [a,b]$ è una curva chiusa semplice e regolare, allora l'area racchiusa da $\gamma$ è data da:
  $$\text{Area}(\Omega) = \int_{a}^{b} x(t)\cdot y'(t) dt$$.\\
  Sono possibili anche altre scelte per $\vec F$ che generano formule analoghe per il calcolo dell'area di $\Omega$.
\end{teorema}
