\chapter{Boxes}
\section{Boxing examples}

Additionally, the numbering between different types of boxes (such as Definitions, Theorems, Exercises, etc.) is independent. This means that the numbering for Definitions will not affect the numbering for Theorems, and so on.
The numbering of the various boxes (such as Definitions, Theorems, Exercises, etc.) is automatically reset at the beginning of each chapter. This means that every time a new chapter starts, the box counter resets to 1.

\begin{definition}{An example of a definition}
  This is an example of a colored box with the title "Definition" in English.
\end{definition}

\begin{theorem}{An example of a theorem}
  This is an example of a colored box with the title "Theorem" in English.
\end{theorem}

\begin{corollary}{An example of a corollary}
  This is an example of a colored box with the title "Corollary" in English.
\end{corollary}

\begin{observation}{An example of an observation}
  This is an example of a colored box with the title "Observation" in English.
\end{observation}

\begin{exercise}{An example of an exercise}
  This is an example of a colored box with the title "Exercise" in English.
\end{exercise}
