\chapter{Formattare codice}

In questo capitolo verrà mostrato come utilizzare il pacchetto \texttt{listings} per formattare il codice sorgente in LaTeX. Questo pacchetto permette di evidenziare la sintassi di diversi linguaggi di programmazione e di personalizzare lo stile per adattarlo al proprio documento.

\section{Inserimento del codice}

Dopo aver definito lo stile, è possibile utilizzarlo per formattare il codice all'interno del documento. Per farlo, si usa il comando \texttt{\textbackslash lstset} per impostare lo stile e il comando \texttt{\textbackslash lstinputlisting} o \texttt{\textbackslash begin{lstlisting}...\textbackslash end{lstlisting}} per includere il codice. Ecco un esempio:

\subsection{Esempio con file esterno} Puoi anche includere un file esterno contenente il codice:

\lstinputlisting[language=Python, breaklines, breakautoindent]{res/py/2d_example.py}

\section{Formattare il Codice}

Il pacchetto \texttt{listings} consente di visualizzare il codice sorgente all'interno di un documento LaTeX, mantenendo la formattazione e l'evidenziazione della sintassi. In questa sezione, esploreremo come utilizzare \texttt{listings} per includere e formattare il codice nel nostro documento.

\subsection{Stili Predefiniti}
\texttt{listings} fornisce alcuni stili predefiniti per evidenziare la sintassi di vari linguaggi di programmazione. Questi stili permettono di includere facilmente blocchi di codice con una corretta evidenziazione della sintassi.

Ad esempio, per includere codice Python nel documento, puoi utilizzare il seguente comando:

\begin{lstlisting}
	\begin{lstlisting}[language=Python]
		def hello_world():
		print("Hello, World!")
	/end{lstlisting}
\end{lstlisting}

Risultato:
\begin{lstlisting}[language=Python]
	def hello_world():
	print("Hello, World!")
\end{lstlisting}

In questo esempio, il pacchetto \texttt{listings} riconosce automaticamente la sintassi Python e la evidenzia correttamente.

\subsection{Linguaggi di programmazione supportati}
\texttt{listings} supporta molti linguaggi di programmazione, quindi è possibile utilizzare la sintassi di evidenziazione automatica per una vasta gamma di linguaggi. Basta specificare il linguaggio tramite l'opzione \texttt{language} nell'ambiente \texttt{lstlisting}. Alcuni dei linguaggi supportati includono:

\begin{itemize}
	\item Python: \texttt{language=Python}
	\item C/C++: \texttt{language=C}
	\item Java: \texttt{language=Java}
	\item SQL: \texttt{language=SQL}
	\item HTML: \texttt{language=HTML}
	\item CSS: \texttt{language=CSS}
	\item JavaScript: \texttt{language=JavaScript}
	\item XML: \texttt{language=XML}
\end{itemize}

Esempio di codice SQL:

\begin{lstlisting}[language=SQL]
	SELECT * FROM users WHERE age > 18;
\end{lstlisting}

\subsection{Personalizzare gli Stili}
Anche se il pacchetto offre stili predefiniti, è possibile personalizzare l'aspetto del codice. Se ci fosse la necessità di creare nuovi stili, contattare il responsabile. Alcuni parametri personalizzabili, qui elencati per evidenziare le potenzialità degli stili, sono:

\begin{itemize}
	\item \texttt{language}: Definisce il linguaggio del codice.
	\item \texttt{basicstyle}: Imposta il tipo di carattere di base per il codice.
	\item \texttt{keywordstyle}: Modifica lo stile delle parole chiave.
	\item \texttt{stringstyle}: Modifica lo stile delle stringhe.
	\item \texttt{commentstyle}: Definisce lo stile per i commenti.
	\item \texttt{numbers}: Visualizza i numeri di riga.
	\item \texttt{frame}: Aggiunge un bordo attorno al codice.
	\item \texttt{backgroundcolor}: Imposta il colore di sfondo del codice.
\end{itemize}

Questo è il prima e dopo l'applicazione dello stile sql style definito all'interno di HKNtools.tex :
\begin{lstlisting}[language=SQL]
	-- Query per selezionare i dati degli utenti
	SELECT name, age
	FROM users
	WHERE age > 18
	ORDER BY age DESC;
\end{lstlisting}

\begin{lstlisting}[style=sqlstyle]
	-- Query per selezionare i dati degli utenti
	SELECT name, age
	FROM users
	WHERE age > 18
	ORDER BY age DESC;
\end{lstlisting}

\subsection{Configurazione Globale}
Puoi anche configurare \texttt{listings} globalmente per applicare le stesse impostazioni di stile a tutto il documento utilizzando il comando \texttt{\\lstset}. Ad esempio:

\begin{lstlisting}
\lstset{
	language=Python,
	numbers=left,
	frame=single,
	backgroundcolor=\color{lightgray},
	stepnumber=1
}
\end{lstlisting}

Questo imposterà le opzioni per tutti i blocchi di codice Python nel documento.
Ecco gli effetti prima e dopo l'uso dell'istruzione:

\begin{lstlisting}[language=Python]
	def hello_world():
	print("Hello, World!")
\end{lstlisting}

\lstset{
	language=Python,
	numbers=left,
	frame=single,
	backgroundcolor=\color{lightgray},
	stepnumber=1
}

\begin{lstlisting}[language=Python]
	def hello_world():
	print("Hello, World!")
\end{lstlisting}

\section{Stili personalizzati attualmente esistenti}
Ulteriori stili verranno implementati man mano che ne nascerà l'esigenza da parte dei collaboratori. Per ora, questi sono quelli attualmente definiti:
\begin{itemize}
	\item SQL: sqlstyle
\end{itemize}
